\documentclass{article}

%% Language and font encodings
\usepackage[english]{babel}
\usepackage[utf8x]{inputenc}
\usepackage[T1]{fontenc}
\usepackage{amsfonts}
\usepackage[margin=41mm]{geometry}

%% Sets page size and margins
%\usepackage[a4paper,top=3cm,bottom=2cm,left=3cm,right=3cm,marginparwidth=1.75cm]{geometry}

%% Useful packages
\usepackage{amsmath}
\usepackage{xcolor}
\usepackage{amssymb}
\usepackage{amsthm}
\usepackage{stmaryrd}
\usepackage{graphicx}
\usepackage{enumitem}

%Theorem
\newtheorem*{theorem}{Theorem}
\newtheorem*{proposition}{Proposition}
\newtheorem*{lemma}{Lemma}
\newtheorem*{definition}{Definition}

%Commands definitions
\newcommand{\C}{\mathbb{C}}
\newcommand{\R}{\mathbb{R}}
\newcommand{\Q}{\mathbb{Q}}
\newcommand{\Z}{\mathbb{Z}}
\newcommand{\N}{\mathbb{N}}
\newcommand{\F}{\mathbb{F}}
\newcommand{\E}{\mathbb{E}}
\newcommand{\K}{\mathbb{K}}
\renewcommand{\L}{\mathcal{L}}
\newcommand{\lnorm}[2]{\left\lVert#2 \right\rVert_{#1}}
\newcommand{\norm}[1]{\left\lVert#1 \right\rVert}
\newcommand{\abs}[1]{\left\lvert#1 \right\rvert}
\newcommand{\Iint}[2]{\llbracket #1 , #2 \rrbracket}
\renewcommand{\Im}{\text{Im}}
\newcommand{\card}{\text{card}}
\newcommand{\isomorphic}{\cong}
\newcommand{\Aut}{\text{Aut}}
\newcommand{\td}{\textcolor{red}{\textbf{TODO}}}

%Example environment
\newenvironment{example}{\noindent\textbf{Example:} \vspace{-0.2cm}\begin{itemize}}{\end{itemize}}

%Remark environment
\newenvironment{remark}{\noindent\textbf{Remark:}}{}

%Set QED symbol to blacksquare
\renewcommand\qedsymbol{$\blacksquare$}


\title{MATH 358 Notes : Honours Advanced Calculus}
\author{Samy Lahlou}
\date{}

\begin{document}

\maketitle

These notes are based on lectures given by Professor John Toth at McGill University in Winter 2025. The subject of these lectures is \td . \\
As a disclaimer, it is more than possible that I made some mistakes. Feel free to correct me or ask me anything about the content of this document at the following address : samy.lahloukamal@mcgill.ca

\tableofcontents

\newpage

\section{Integration over Surfaces in $\R^3$}

\begin{definition}[Surface Parametrization]
    Let $D \subset \R^2$ be a piecewise $C^1$ bounded domain and let $S = \phi(D)$ be the corresponding surface where $\phi : D \to S \subset \R^3$ is a $C^1$ mapping. We call $D$ the parametrization domain.
\end{definition}

\begin{definition}[Regular Surfaces]
    Given a parametrized surface $S$ with parameter $\phi$ and parametrization domain $D$, we say that $S$ is regular if $\phi_u(u_0, v_0) \times \phi_v(u_0, v_0) \neq 0$ for all $(u_0, v_0) \in D$.
\end{definition}

\begin{definition}[Tangent Plane]
    Given a regular surface $S$ with parameter $\phi$ and parametrization domain $D$, then for all $d \in D$, we can define the vector $n = \phi_u(d) \times \phi_v(d)$ which is normal to the surface at $\phi(d)$. From this, we define 
    $$T_{\phi(d)}S = \{X \in \R^3 : (X - \phi(d))\cdot n = 0\}$$
    as the tangent plane to $S$ at $\phi(d)$.
\end{definition}

\begin{definition}[Area of a Surface]
    Given a regular surface $S$ with parameter $\phi$ and parametrization domain $D$, we can define the area of $S$ as
    $$\text{area}(S) = \iint_D \|\phi_u \times \phi_v\| dA.$$
\end{definition}



\end{document}