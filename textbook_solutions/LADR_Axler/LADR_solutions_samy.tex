\documentclass[12pt, oneside]{book}

%% Language and font encodings
\usepackage[english]{babel}
\usepackage[utf8x]{inputenc}
\usepackage[T1]{fontenc}
\usepackage{amsfonts}

%% Sets page size and margins
\usepackage[a4paper,top=3cm,bottom=2cm,left=3cm,right=3cm,marginparwidth=1.75cm]{geometry}

%% Useful packages
\usepackage{amsmath}
\usepackage{xcolor}
\usepackage{amssymb}
\usepackage{amsthm}
\usepackage{stmaryrd}
\usepackage{graphicx}
\usepackage[colorinlistoftodos]{todonotes}
\usepackage[colorlinks=true, allcolors=blue]{hyperref}
\usepackage{enumitem}

%Commands definitions
\renewcommand{\thesection}{\thechapter\Alph{section}}
\newcommand{\setbackgroundcolour}{\pagecolor[rgb]{0.1,0.1,0.1}}  
\newcommand{\settextcolour}{\color[rgb]{1,1,1}}    
\newcommand{\invertbackgroundtext}{\setbackgroundcolour\settextcolour}
\newcommand{\F}{\mathbf{F}}
\newcommand{\C}{\mathbf{C}}
\newcommand{\R}{\mathbf{R}}
\newcommand{\Q}{\mathbf{Q}}
\newcommand{\Z}{\mathbf{Z}}
\newcommand{\N}{\mathbf{Z}^+}
\newcommand{\Iint}[2]{\llbracket #1 , #2 \rrbracket}
\renewcommand{\Im}{\text{Im}}
\newcommand{\card}{\text{card}}
\newcommand{\td}{\textcolor{red}{\textbf{TODO}}}
\renewcommand{\L}{\mathcal{L}}
\renewcommand{\P}{\mathcal{P}}


%Command execution. 
%If this line is commented, then the appearance remains as usual.
%\invertbackgroundtext

%Set QED symbol to blacksquare
\renewcommand\qedsymbol{$\blacksquare$}

%% Table packages
\usepackage{array}
\newcolumntype{P}[1]{>{\centering\arraybackslash}p{#1}}
\usepackage{multirow}

% Définir un nouveau compteur d'exercice
\newcounter{exercise}[section] % Compteur qui est remis à zéro à chaque nouvelle section
%\renewcommand{\theexercise}{\arabic{exercise}}

% Définir un environnement pour l'exercice
\newenvironment{exercise}{
    \refstepcounter{exercise} % Incrémenter le compteur
    \noindent\textbf{Exercise \arabic{exercise}} 
    \\ 
}{} % Fin de l'environnement

% Définir un environnement pour les solutions
\newenvironment{solution}{
    \noindent\textbf{Solution} 
}{}


\title{Solutions to Linear Algebra Done Right (4th Ed)
\\ - Sheldon Axler}
\author{Samy Lahlou}

\begin{document}
\maketitle

\chapter*{Preface}

The goal of this document is to share my personal solutions to the exercises in the Fourth Edition of Linear Algebra Done Right by Sheldon Axler during my reading. \\ 
As a disclaimer, the solutions are not unique and there will probably be better or more optimized solutions than mine. Feel free to correct me or ask me anything about the content of this document at the following address : samy.lahloukamal@mcgill.ca

\tableofcontents

% CHAPTER 1
\chapter{Vector Spaces}
\section{$\R^n$ and $\C^n$}

\begin{exercise}
    Show that $\alpha + \beta = \beta + \alpha$ for all $\alpha, \beta \in \C$. \\
\end{exercise}

\begin{solution}
    \\ First, suppose that
    $$\alpha = a+ ib \quad \text{ and } \quad \beta = c + id$$
    where $a,b,c,d \in \R$, then
    \begin{align*}
        \alpha + \beta &= (a + ib) + (c + id) \\
        &= (a + c) + i(b + d) \\
        &= (c + a) + i(d + b) \\
        &= (c + id) + (a + ib) \\
        &= \beta + \alpha
    \end{align*}
    which proves that addition is commutative in $\C$ using the fact that it is commutative in $\R$. \\
\end{solution}

\begin{exercise}
    Show that $(\alpha + \beta) + \lambda = \alpha + (\beta + \lambda)$ for all $\alpha, \beta, \lambda \in \C$. \\
\end{exercise}

\begin{solution}
    \\ First, suppose that
    $$\alpha = a+ ib, \quad \beta = c + id \quad \text{ and } \quad \lambda = e + if$$
    where $a,b,c,d,e,f \in \R$, then
    \begin{align*}
        (\alpha + \beta) + \lambda &= [(a + ib) + (c + id)] + (e + if) \\
        &= [(a + c) + i(b + d)] + (e + if) \\
        &= ([a + c] + e) + i([b + d] + f) \\
        &= (a + [c + e]) + i(b + [d + f]) \\
        &= (a + ib) + [(c + e) + i(d + f)] \\
        &= (a + ib) + [(c + id) + (e + if)] \\
        &= \alpha + (\beta + \lambda)
    \end{align*}
    which proves that addition is associative in $\C$ using the fact that it is associative in $\R$. \\
\end{solution}

\begin{exercise}
    Show that $(\alpha\beta)\lambda = \alpha (\beta \lambda)$ for all $\alpha, \beta, \lambda \in \C$. \\
\end{exercise}

\begin{solution}
    \\ First, suppose that
    $$\alpha = a+ ib, \quad \beta = c + id \quad \text{ and } \quad \lambda = e + if$$
    where $a,b,c,d,e,f \in \R$, then
    \begin{align*}
        (\alpha \beta)\lambda &= [(a + ib)(c + id)](e + if) \\
        &= [(ac - bd) +  i(ad + bc)] (e + if) \\
        &= ([ac - bd]e - [ad + bc]f) + i ([ac - bd]f + [ad + bc]e) \\
        &= (ace - bde - adf - bcf) + i(acf - bdf + ade + bce) \\
        &= (a[ce - fd] - b [cf + de)) + i (a[cf + de] + b[ce - fd]) \\
        &= (a + ib) [(ce - fd) + i(cf + de)] \\
        &= (a + ib) [(c + id) (e + if)] \\
        &= \alpha (\beta \lambda)
    \end{align*}
    which proves that multiplication is associative in $\C$ using the fact that multiplication is associative and addition is commutative in $\R$. \\
\end{solution}

\begin{exercise}
    Show that $\lambda(\alpha + \beta) = \lambda \alpha + \lambda \beta$ for all $\lambda, \alpha, \beta \in \C$. \\
\end{exercise}

\begin{solution}
    \\ First, suppose that
    $$\alpha = a+ ib, \quad \beta = c + id \quad \text{ and } \quad \lambda = e + if$$
    where $a,b,c,d,e,f \in \R$, then
    \begin{align*}
        \lambda(\alpha + \beta) &= (e + if)[(a+ib) + (c + id)] \\
        &= (e + if)[(a+c) + i(b+d)] \\
        &= [e(a+c) - f(b+d)] + i[e[b+d] + f[a+c]] \\
        &= (ea + ec - fb - fd) + i(eb + ed + fa + fc) \\
        &= [(ea - fb) + i (eb + fa)] + [(ec - fd) + i(ed + fc)] \\
        &= [(e + if)(a + ib)] + [(e + if)(c + id)] \\
        &= \lambda\alpha + \lambda\beta
    \end{align*}
    which proves the distributivity in $\C$ using the distributivity in $\R$.\\
\end{solution}

\begin{exercise}
    Show that for every $\alpha \in \C$, there exists a unique $\beta \in \C$ such that $\alpha + \beta = 0$. \\
\end{exercise}

\begin{solution}
    \\ Let $\alpha = a + ib$ and consider $\beta = (-a) + i(-b)$, then we get
    \begin{align*}
        \alpha + \beta &= (a + ib) + ([-a] + i[-b]) \\
        &= (a + [-a]) + i(b + [-b]) \\
        &= 0 + i0 \\
        &= 0
    \end{align*}
    which proves the existence of such a complex number $\beta$. To prove the uniqueness of such a complex number, let $\beta_1$ and $\beta_2$ be two complex numbers satisfying $\alpha + \beta_1 = 0$ and $\alpha + \beta_2 = 0$, this implies that $\alpha + \beta_1 = \alpha + \beta_2$. If we add $\beta_1$ on both sides, we get
    \begin{align*}
        \beta_1 + (\alpha + \beta_1) = \beta_1 + (\alpha + \beta_2) &\implies (\beta_1 + \alpha) + \beta_1 = (\beta_1 + \alpha) + \beta_2 \\
        &\implies (\alpha + \beta_1) + \beta_1 = (\alpha + \beta_1) + \beta_2 \\
        &\implies 0 + \beta_1 = 0 + \beta_2 \\
        &\implies \beta_1 = \beta_2
    \end{align*}
    which proves that such a complex number is unique.\\
\end{solution}

\begin{exercise}
    Show that for every $\alpha \in \C$ with $\alpha \neq 0$, there exists a unique $\beta \in \C$ such that $\alpha \beta = 1$. \\
\end{exercise}

\begin{solution}
    \\ Let $\alpha = a + ib \neq 0$, then notice that we must have $a^2 + b^2 \neq 0$. Hence, consider 
    $$\beta = \left(\frac{a}{a^2 + b^2}\right) + i\left(-\frac{b}{a^2 + b^2}\right)$$
    Thus, we get
    \begin{align*}
        \alpha \beta &= (a + ib) \left[\left(\frac{a}{a^2 + b^2}\right) + i\left(-\frac{b}{a^2 + b^2}\right)\right]\\
        &= \left(a\left(\frac{a}{a^2 + b^2}\right) - b\left(-\frac{b}{a^2 + b^2}\right)\right) + i\left(a\left(-\frac{b}{a^2 + b^2}\right) + b\left(\frac{a}{a^2 + b^2}\right)\right) \\
        &= \frac{a^2 + b^2}{a^2 + b^2} + i\frac{-ab + ba}{a^2 + b^2} \\
        &= 1 + i0 \\
        &= 1
    \end{align*}
    which proves the existence of such a complex number $\beta$. To prove the uniqueness of such a complex number, let $\beta_1$ and $\beta_2$ be two complex numbers satisfying $\alpha\beta_1 = 1$ and $\alpha \beta_2 = 1$, this implies that $\alpha \beta_1 = \alpha \beta_2$. If we multiply by $\beta_1$ on both sides, we get
    \begin{align*}
        \beta_1 (\alpha \beta_1) = \beta_1 (\alpha \beta_2) &\implies (\beta_1 \alpha) \beta_1 = (\beta_1 \alpha) \beta_2 \\
        &\implies (\alpha  \beta_1)  \beta_1 = (\alpha  \beta_1)  \beta_2 \\
        &\implies 1 \cdot \beta_1 = 1 \cdot \beta_2 \\
        &\implies \beta_1 = \beta_2
    \end{align*}
    which proves that such a complex number is unique.\\
\end{solution}

\begin{exercise}
    Show that
    $$\frac{-1 + \sqrt{3}i}{2}$$
    is a cube root of 1 (meaning that its cube equals 1). \\
\end{exercise}

\begin{solution}
    \\This is pretty straightforward:
    \begin{align*}
        \left(\frac{-1 + \sqrt{3}i}{2}\right)^3 &= \frac{(-1 + \sqrt{3}i)^3}{2^3} \\
        &= \frac{(-1)^3 + 3(-1)^2(\sqrt{3}i) + 3(-1)^1(\sqrt{3}i)^2 + (\sqrt{3}i)^3}{8} \\
        &= \frac{-1 + 3\sqrt{3}i + 3\cdot 3 - 3(\sqrt{3}i)}{8} \\
        &= \frac{8}{8} \\
        &= 1\\
    \end{align*}
\end{solution}

\begin{exercise}
    Find two distinct square roots of $i$.\\
\end{exercise}

\begin{solution}
    \\ Consider $\alpha = \frac{\sqrt{2}}{2} + i\frac{\sqrt{2}}{2}$ and $\beta = -\frac{\sqrt{2}}{2} - i\frac{\sqrt{2}}{2}$. Hence,
    \begin{align*}
        \alpha^2 &= \left(\frac{\sqrt{2}}{2} + i\frac{\sqrt{2}}{2}\right)^2 \\
        &= \left(\frac{\sqrt{2}}{2}\right)^2 + 2\cdot \frac{\sqrt{2}}{2} \cdot i\frac{\sqrt{2}}{2} + \left(i\frac{\sqrt{2}}{2}\right)^2 \\
        &= \frac{2}{4} + i - \frac{2}{4} \\
        &= i
    \end{align*}
    and
    \begin{align*}
        \beta^2 &= \left(-\frac{\sqrt{2}}{2} - i\frac{\sqrt{2}}{2}\right)^2 \\
        &= \left(-\frac{\sqrt{2}}{2}\right)^2 + 2\cdot \left( -\frac{\sqrt{2}}{2}\right)\cdot \left( - i\frac{\sqrt{2}}{2}\right) + \left(-i\frac{\sqrt{2}}{2}\right)^2 \\
        &= \frac{2}{4} + i - \frac{2}{4} \\
        &= i
    \end{align*}
    Therefore, $\alpha$ and $\beta$ are two distinct square roots of $i$. \\
\end{solution}

\begin{exercise}
    Find $x \in \R^4$ such that
    $$(4, -3, 1, 7) + 2x = (5, 9, -6, 8).$$
\end{exercise}

\begin{solution}
    \\ First, suppose that such an element $x$ exists, then there exist $a,b,c,d \in \R$ such that $x = (a,b,c,d)$ and 
    $$(4 + 2a, -3 + 2b, 1 + 2c, 7 + 2d) = (5, 9, -6, 8)$$
    But notice that this is equivalent to the following system of equations:
    $$\begin{cases}
        4 + 2a = 5 \\ -3 +2b = 9 \\ 1 + 2c = -6 \\ 7 + 2d = 8
    \end{cases}$$
    which implies that
    $$\begin{cases}
        a = \frac{1}{2} \\ b = 6 \\ c = \frac{7}{2} \\ d = \frac{1}{2}
    \end{cases}$$
    Therefore, we get that $x = (\frac{1}{2}, 6, \frac{7}{2}, \frac{1}{2}) \in \R^4$ is indeed a solution to our original equation. \\
\end{solution}

\begin{exercise}
    Explain why there is does not exist $\lambda \in \C$ such that
    $$\lambda (2 - 3i, 5 + 4i, -6 + 7i) = (12 - 5i, 7 + 22i, -32 -9i).$$
\end{exercise}

\begin{solution}
    \\ By contradiction, suppose there exists a complex number $\lambda = a + ib$ such that
    $$\lambda (2 - 3i, 5 + 4i, -6 + 7i) = (12 - 5i, 7 + 22i, -32 -9i)$$
    Then, we would get the following system of equation:
    $$\begin{cases}
        \lambda(2 - 3i) = 12 - 5i \\ \lambda(5 + 4i) = 7 + 22i \\ \lambda(-6 + 7i) = -32 - 9i
    \end{cases}$$
    which is equivalent to
    $$\begin{cases}
        \lambda = 3 + 2i \\ \lambda = 3 + 2i \\ \lambda = \frac{129}{85} + i\frac{278}{85}
    \end{cases}$$
    We clearly have a contradiction since $3 + 2i \neq \frac{129}{85} + i\frac{278}{85}$. Therefore, there doesn't exist such a complex number $\lambda$.\\
\end{solution}

\begin{exercise}
    Show that $(x + y) + z = x + (y + z)$ for all $x,y,z \in \F^n$. \\
\end{exercise}

\begin{solution}
    \\ First, write
    $$x = (x_1, ..., x_n), \quad y = (y_1, ..., y_n) \quad \text{ and } \quad z = (z_1, ..., z_n)$$
    Since addition is commutative in $\F$, we get
    \begin{align*}
        (x + y) + z &= [(x_1, ..., x_n) + (y_1, ..., y_n)] + (z_1, ..., z_n) \\
        &= (x_1 + y_1, ..., x_n + y_n) + (z_1, ..., z_n) \\
        &= ([x_1 + y_1] + z_1, ..., [x_n + y_n] + z_n) \\
        &= (x_1 + [y_1 + z_1], ..., x_n + [y_n + z_n]) \\
        &= (x_1, ..., x_n) + (y_1 + z_1, ..., y_n + z_n) \\
        &= (x_1, ..., x_n) + [(y_1, ..., y_n) + (z_1, ..., z_n)] \\
        &= x + (y+z)
    \end{align*}
    which proves that addition is associative in $\F^n$.\\
\end{solution}

\begin{exercise}
    Show that $(ab)x = a(bx)$ for all $x \in \F^n$ and all $a,b \in \F$.\\
\end{exercise}

\begin{solution}
    \\ First, write $x = (x_1, ..., x_n)$. Using associativity of multiplication in $\F$, we get
    \begin{align*}
        (ab)x &= (ab)(x_1, ..., x_n) \\
        &= ((ab)x_1, ..., (ab)x_n) \\
        &= (a(bx_1), ..., a(bx_n)) \\
        &= a(bx_1, ..., bx_n) \\
        &= a[b(x_1, ..., x_n)] \\
        &= a(bx)
    \end{align*}
    which proves the desired formula for all $x \in \F^n$ and all $a,b \in \F$.\\
\end{solution}

\begin{exercise}
    Show that $1x = x$ for all $x \in \F^n$.\\
\end{exercise}

\begin{solution}
    \\ Let $x = (x_1, ..., x_n) \in \F^n$. Hence,
    \begin{align*}
        1x &= 1(x_1, ..., x_n) \\
        &= (1\cdot x_1, ..., 1 \cdot x_n) \\
        &= (x_1, ..., x_n) \\
        &= x
    \end{align*}
    which proves the desired formula for all $x \in \F^n$.\\
\end{solution}

\begin{exercise}
    Show that $\lambda(x+y) = \lambda x + \lambda y$ for all $\lambda \in \F$ and $x,y \in \F^n$.\\
\end{exercise}

\begin{solution}
    \\ Let $\lambda \in \F$ and $x,y \in \F^n$ with $x = (x_1, ..., x_n)$ and $y = (y_1, ..., y_n)$. Using distributivity in $\F$, we get
    \begin{align*}
        \lambda(x + y) &= \lambda [(x_1, ..., x_n) + (y_1, ..., y_n)] \\
        &= \lambda(x_1 + y_1, ..., x_n + y_n) \\
        &= (\lambda(x_1 + y_1), ..., \lambda (x_n + y_n))\\
        &= (\lambda x_1 + \lambda y_1, ..., \lambda x_n + \lambda y_n) \\
        &= (\lambda x_1, ..., \lambda x_n) + (\lambda y_1, ..., \lambda y_n) \\
        &= \lambda (x_1, ..., x_n) + \lambda (y_1, ...,y_n) \\
        &= \lambda x + \lambda y
    \end{align*}
    which proves the desired formula.\\
\end{solution}

\begin{exercise}
    Show that $(a + b)x = ax + bx$ for all $a,b \in \F$ and all $x \in \F^n$. \\
\end{exercise}

\begin{solution}
    \\ Let $a,b \in \F$ and $x = (x_1, ..., x_n) \in \F^n$. Using distributivity in $\F$, we get
    \begin{align*}
        (a+b)x &= (a+b)(x_1, ..., x_n) &= ((a+b)x_1, ..., (a+b)x_n) \\
        &= (ax_1 + bx_1, ..., ax_n + bx_n) \\
        &= (ax_1, ..., ax_n) + (bx_1, ..., bx_n) \\
        &= a(x_1, ..., x_n) + b(x_1, ..., x_n) \\
        &= ax + bx
    \end{align*}
    which proves the desired formula.
\end{solution}
\section{Definition of Vector Space}

\begin{exercise}
    Prove that $-(-v) = v$ for every $v \in V$.\\
\end{exercise}

\begin{solution}
    \\ Let $v \in V$, by definition, we know that by definition, $-v$ is defined as the only vector in $V$ satisfying
    $$v + (-v) = 0$$
    which is equivalent to
    $$(-v) + v = 0$$ 
    by commutativity of addition in $V$. However, notice that by definition, $-(-v)$ is the unique vector satisfying
    $$(-v) + [-(-v)] = 0$$
    But since $v$ itself also satisfies this equation, we get $-(-v) = v$ by uniqueness.\\
\end{solution}

\begin{exercise}
    Suppose $a \in \F$, $v \in V$, and $av = 0$. Prove that $a = 0$ or $v = 0$.\\
\end{exercise}

\begin{solution}
    \\ Suppose that $a \neq 0$, then by properties of $\F$, the inverse $a^{-1}$ exists. Hence, if we multiply by $a^{-1}$ on both sides, we get
    \begin{align*}
        av = 0 &\implies a^{-1}(av) = a^{-1}0 \\
        &\implies (a^{-1}a)v = 0 \\
        &\implies 1v = 0 \\
        &\implies v = 0
    \end{align*}
    Therefore, we either have $a = 0$ or $v = 0$.\\
\end{solution}

\begin{exercise}
    Suppose $v, w \in V$. Explain why there exists a unique $x \in V$ such that $v + 3x = w$.\\
\end{exercise}

\begin{solution}
    \\ By properties of vector spaces, since $v \in V$, then $-v \in V$. Similarly, since $w$ and $-v$ are in $V$, then $w + (-v) \in V$. Finally, since $w + (-v) \in V$, then $3^{-1}(w + (-v)) \in V$. Thus, define $x_0$ as the vector $3^{-1}(w + (-v))$ in $V$. Notice that
    \begin{align*}
        v + 3x_0 &= v + 3[3^{-1}(w + (-v))] \\
        &= v + (3 \cdot 3^{-1})(w + (-v)) \\
        &= v + 1(w + (-v)) \\
        &= v + (w + (-v)) \\
        &= v + ((-v) + w) \\
        &= (v + (-v)) + w \\
        &= 0 + w \\
        &= w
    \end{align*}
    which shows that the equation has at least one solution. To prove uniqueness, let $x_1 \in V$ be an arbitrary solution to the equation, then we get
    \begin{align*}
        v + 3x_1 = w &\implies (-v) + (v + 3x_1) = (-v) + w \\
        &\implies ((-v) + v) + 3x_1 = w + (-v) \\
        &\implies 0 + 3x_1 = w + (-v) \\
        &\implies 3x_1 = w + (-v) \\
        &\implies 3^{-1}(3x_1) = 3^{-1}(w + (-v)) \\
        &\implies (3^{-1}3)x_1 = x_0 \\
        &\implies 1x_1 = x_0 \\
        &\implies x_1 = x_0
    \end{align*} 
    which proves that $x_0$ is the unique solution to the equation. \\
\end{solution}

\begin{exercise}
    The empty set is not a vector space. The empty set fails to satisfy only one of the requirements listed in the definition of a vector space. Which one? \\
\end{exercise}

\begin{solution}
    \\ The empty set doesn't satisfy the axiom that states that there must be an additive identity since the empty set is empty by definition.\\
\end{solution}

\begin{exercise}
    Show that in the definition of a vector space, the additive inverse condition can be replaced with the condition that
    $$0v = 0 \text{ for all } v\in V.$$
    Here, the 0 on the left side is the number 0, and the 0 on the right side is the additive identity of $V$.\\
\end{exercise}

\begin{solution}
    \\ We already know that the axioms of a vector space imply that $0v = 0$ for all $v \in V$. Hence, it suffices to prove that if we assume the axioms of a vector space without the additive inverse condition, then we can prove the additive inverse condition if we also assume the property that $0v = 0$ for all $v \in V$. Let $v \in V$, the by the distributive condition, we get
    \begin{align*}
        0v = 0 &\implies (1 + (-1))v = 0 \\
        &\implies 1v + (-1)v = 0 \\
        &\implies v + (-1)v = 0
    \end{align*}
    which proves that $v$ has an additive inverse for all $v \in V$.\\
\end{solution}

\begin{exercise}
    Let $\infty$ and $-\infty$ denote two distinct objects, neither of which is in $\R$. Define an addition and scalar multiplication on $\R \cup \{\infty, -\infty\}$ as you could guess from the notation. Specifically, the sum and product of two reals numbers is as usual, and for $t \in \R$ define 
    $$t\infty = \begin{cases}
        - \infty & \text{if } t < 0,\\
        0 & \text{if } t = 0,\\
        \infty & \text{if } t>0,
    \end{cases} \qquad t(-\infty) = \begin{cases}
        \infty & \text{if } t < 0,\\
        0 & \text{if } t = 0,\\
        -\infty & \text{if } t>0,
    \end{cases}$$
    and
    \begin{align*}
        t + \infty &= \infty + t = \infty + \infty = \infty \\
        t + (-\infty) &= (-\infty) + t = (-\infty) + (-\infty) = -\infty \\
        \infty + (-\infty) &= (-\infty) + \infty = 0
    \end{align*}
    With these operations of addition and scalar multiplication, is $\R \cup \{\infty, -\infty\}$ a vector space over $\R$? Explain. \\
\end{exercise}

\begin{solution}
    \\ With these operations of addition and scalar multiplication, $\R \cup \{\infty, -\infty\}$ cannot be a vector space since
    $$((-\infty) + \infty) + \infty = 0 + \infty = \infty$$
    and
    $$(-\infty) + (\infty + \infty) = (-\infty) + \infty = 0$$
    which proves that addition isn't associative under this operation. \\
\end{solution}

\begin{exercise}
    Suppose $S$ is a nonempty set. Let $V^S$ denote the set of functions from $S$ to $V$. Define a natural addition and scalar multiplication on $V^S$, and show that $V^S$ is a vector space with these definitions. \\
\end{exercise}

\begin{solution}
    \\ For any $f$ and $g$ in $V^S$, define $f + g : S \to V$ by $s \mapsto f(s) + g(s)$ for all $s \in S$. Similarly, for all $\alpha \in \F$ and $f \in V^S$, define $\alpha f : S \to V$ by $s \mapsto \lambda f(s)$ for all $s \in S$. With these definitions, let's prove that $V^S$ is a vector space.
    \begin{itemize}
        \item (\textbf{commutativity}) Let $f, g \in V^S$, let's show that $f + g = g + f$. Let $s \in S$, then by commutativity in $V$, we obviously have
        $$(f+g)(s) = f(s) + g(s) = g(s) + f(s) = (g+f)(s)$$
        Since it holds for all $s$, then $f + g = g + f$.
        \item (\textbf{associativity}) Let $f,g,h \in V^S$ and $s \in S$, then by associativity in $V$, we have
        \begin{align*}
            [(f+g)+h](s) &= (f+g)(s) + h(s) \\
            &= [f(s) + g(s)] + h(s) \\
            &= f(s) + [g(s) + h(s)] \\
            &= f(s) + (g+h)(s) \\
            &= [f + (g+h)](s)
        \end{align*}
        Since it holds for all $s \in S$, then $(f+g) + h= f + (g+h)$. \\
        Let now $f \in V^S$, $a,b \in \F$ and $s \in S$, then by associativity in $V$, we get:
        \begin{align*}
            [(ab)f](s) &= (ab)f(s) \\
            &= a(bf(s)) \\
            &= a(bf)(s) \\
            &= [a(bf)](s)
        \end{align*}
        Since it holds for all $s \in S$, then $(ab)f = a(bf)$.
        \item (\textbf{additive identity}) Let's denote by $0_{V^S}$ the zero function in $V^S$, then for all $f \in V^S$ and $s \in S$, we have
        $$(f+0_{V^S})(s) = f(s) + 0_{V^S}(s) = f(s) + 0 = f(s)$$
        Since it holds for all $s \in S$, then $f + 0_{V^S} = f$ for all $f \in V^S$.
        \item (\textbf{additive inverse}) Again, let's denote by $0_{V^S}$ the zero function in $V^S$, then for all $f \in V^S$, we can define the function $g = (-1)f \in V^S$. Hence, for all $s \in S$, we get
        \begin{align*}
            (f + g)(s) &= f(s) + g(s) \\
            &= f(s) + (-1)f(s) \\
            &= f(s) + (-f(s)) \\
            &= 0 \\
            &= 0_{V^S}(s)
        \end{align*}
        Since it holds for all $s \in S$, then $f + g = 0_{V^S}$.
        \item (\textbf{multiplicative identity}) Let $f \in V^S$, then for all $s \in S$, we have
        $$(1 f)(s) = 1 f(s) = f(s)$$
        Since it holds for all $s \in S$, then $1f = f$.
        \item (\textbf{distributive property}) Let $f,g \in V^S$, $a \in \F$ and $s \in S$, then
        \begin{align*}
            [a(f+g)](s) &= a(f+g)(s) \\
            &= a(f(s) + g(s)) \\
            &= a f(s) + ag(s) \\
            &= (af)(s) + (ag)(s) \\
            &= (af + ag)(s)
        \end{align*}
        Since it holds for all $s \in S$, then $a(f+g) = af + ag$. Similarly, for all $f \in V^S$, $a,b \in \F$ and $s \in S$, we have
        \begin{align*}
            [(a+b)f](s) &= (a+b)f(s) \\
            &= af(s) + bf(s) \\
            &= (af)(s) + (bf)(s) \\
            &= (af + bf)(s)
        \end{align*}
        Since it holds for all $s \in S$, then $(a+b)f = af + bf$.
    \end{itemize}
    Therefore, $V^S$ is a vector space under these definitions.\\
\end{solution}

\begin{exercise}
    Suppose $V$ is a real vector space.
    \begin{itemize}
        \item The \textit{complexification} of $V$, denoted by $V_{\C}$, equals $V \times V$. An element of $V_{\C}$ is an ordered pair $(u,v)$, where $u,v \in V$, but we write this as $u+iv$.
        \item Addition on $V_{\C}$ is defined by
        $$(u_1+iv_1) + (u_2 + iv_2) = (u_1 + u_2) + i(v_1 + v_2)$$
        for all $u_1, v_1, u_2, v_2 \in V$.
        \item Complex scalar multiplication on $V_{\C}$ is defined by 
        $$(a+ib)(u+iv) = (au - bv) + i(av + bu)$$
        for all $a,b \in \R$ and all $u,v\in V$.
    \end{itemize}
    Prove that with these definitions of addition and scalar multiplication as above, $V_{\C}$ is a complex vector space.\\
\end{exercise}

\begin{solution}
    \begin{itemize}
        \item (\textbf{commutativity}) Let $u_1, v_1, u_2, v_2 \in V$, then by commutativity in $V$, we have
        \begin{align*}
            (u_1+iv_1) + (u_2 + iv_2) &= (u_1 + u_2) + i(v_1 + v_2) \\
            &= (u_2 + u_1) + i(v_2 + v_1) \\
            &= (u_2+iv_2) + (u_1 + iv_1)
        \end{align*}
        which proves that addition is commutative.
        \item (\textbf{associativity}) Let $u_1, v_1, u_2, v_2, u_3,v_3 \in V$, then by associativity in $V$, we have
        \begin{align*}
            [(u_1+iv_1) + (u_2 + iv_2)] + (u_3+iv_3) &= [(u_1 + u_2) + i(v_1 + v_2)] + (u_3+iv_3) \\
            &= ([u_1 + u_2] + u_3) + i([v_1 + v_2] + v_3) \\
            &= (u_1 + [u_2 + u_3]) + i(v_1 + [v_2 + v_3]) \\
            &= (u_1 + iv_1) + [(u_2 + u_3) + i(v_2 + v_3)] \\
            &= (u_1 + iv_1) + [(u_2 + iv_2) + (u_3 + iv_3)]
        \end{align*}
        Let now $a,b,c,d \in \R$ and $u,v \in V$, then we get:
        \begin{align*}
            [(a+bi)(c+di)]&(u+iv) \\ 
            &= [(ac - bd) + i(ad + bc)](u+iv) \\
            &= [(ac - bd)u - (ad+bc)v] + i [(ac - bd)v + (ad+bc)u] \\
            &= [acu - bdu - adv -bcv] + i [acv - bdv + adu+bcu] \\
            &= [a(cu - dv) - b(cv + du)] + i[a(cv + du) + b(cu - dv)] \\
            &= (a+ib)[(cu - dv)+i(cv+du)] \\
            &= (a+ib)[(c+id)(u+iv)]
        \end{align*}
        which proves the associativity condition.
        \item (\textbf{additive identity}) For all $u,v \in V$,
        $$(u + iv) + (0+i0) = (u+0) + i(v+0) = u + iv$$
        which proves that $0+i0$ is an additive identity.
        \item (\textbf{additive inverse}) Let $u,v \in V$, then since $(-u), (-v) \in V$, we get
        $$(u+iv) + ([-u] + i[-v]) = (u + [-u]) + i(v + [-v]) = 0 + i0$$
        which proves that every element has an additive inverse.
        \item (\textbf{multiplicative identity}) Let $u,v \in V$, then
        $$(1 + i0)(u +iv) = (1u - 0v) + i(1v + 0u) = u + iv$$
        which proves that $1 = 1 + i0$ is a multiplicative identity.
        \item (\textbf{distributive property}) Let $a,b \in \R$ and $u_1, v_1, u_2, v_2 \in V$, then
        \begin{align*}
            (a+ib)[(u_1 + i&v_1) + (u_2 + iv_2)] \\
            &= (a+ib)([u_1 + u_2] + i[v_1 + v_2]) \\
            &= (a[u_1 + u_2] - b[v_1 + v_2]) + i(a[v_1 + v_2] + b[u_1 + u_2]) \\
            &= (au_1 + au_2 - bv_1 - bv_2) + i(av_1 + av_2 + v_1 + bv_2) \\
            &= ([au_1 - bv_1] + [au_2 - bv_2]) + i([av_1 + bu_1] + [av_2 + bu_1]) \\
            &= [(au_1 - bv_1) + i(av_1 + bu_1)] + [(au_2 - bv_2) + i(av_2 + bu_2)] \\
            &= [(a+ib)(u_1 + iv_1)] + [(a+ib)(u_2 + iv_2)]
        \end{align*}
        Similarly, for all $a,b,c,d \in \R$, and $u,v \in \R$, we have
        \begin{align*}
            [(a+ib) + (c+&id)](u+iv)\\
            &= ([a+c] + i[b+d])(u+iv) \\
            &= ([a+c]u - [b+d]v) + i([a+c]v + [b+d]u) \\
            &= (au + cu -bv -dv) + i(av + cv + bu+du) \\
            &= ([au - bv] + [cu - dv]) + i([av + bu] + [cv + du]) \\
            &= [(au - bv) + i(av + bu)] + [(cu - dv) + i(cv + du)] \\
            &= (a+ib)(u+iv) + (c+id)(u+iv)
        \end{align*}
        which proves the distributive property.
    \end{itemize}
    Therefore, $V_{\C}$ is a vector space under these definitions.
\end{solution}
\section{Subspaces}

\begin{exercise}
    For each of the following subsets of $\F^3$, determine whether it is a subspace of $\F^3$.
    \begin{enumerate}[label=(\alph*)]
        \item $\{(x_1,x_2,x_3)\in \F^3 : x_1 + 2x_2 + 3x_3  = 0\}$
        \item $\{(x_1,x_2,x_3)\in \F^3 : x_1 + 2x_2 + 3x_3  = 4\}$
        \item $\{(x_1,x_2,x_3)\in \F^3 : x_1 x_2 x_3  = 0\}$
        \item $\{(x_1,x_2,x_3)\in \F^3 : x_1 = 5x_3\}$\\
    \end{enumerate}
\end{exercise}

\begin{solution}
    \begin{enumerate}[label=(\alph*)]
        \item First, define
        $$U = \{(x_1,x_2,x_3)\in \F^3 : x_1 + 2x_2 + 3x_3  = 0\}$$
        Let's prove that it is indeed a subspace of $\F^3$. Since $0 + 2\cdot 0 + 3 \cdot 0 = 0$, then $0 = (0,0,0) \in U$. Now, let $x, y \in U$ be two arbitrary elements where $x = (x_1, x_2,x_3)$ and $y = (y_1, y_2, y_3)$, then by definition:
        $$\begin{cases}
            x_1 + 2x_2 + 3x_3  &= 0 \\
            y_1 + 2y_2 + 3y_3  &= 0
        \end{cases}$$
        Adding the two equations gives us
        $$(x_1 + y_1) + 2(x_2 + y_2) + 3(x_3 + y_3) = 0 + 0 = 0$$
        which proves that $x + y = (x_1 + y_1, x_2 + y_2, x_3 + y_3) \in U$. Similarly, let $x = (x_1, x_2, x_3)$ be an arbitrary element in $U$ and $\alpha$ an arbitrary scalar in $\F$, then by definition of $U$:
        $$x_1 + 2x_2 + 3x_3 = 0 $$
        Multiplying by $\alpha$ on both sides gives us
        $$(\alpha x_1) + 2(\alpha x_2) + 3(\alpha x_3) = \alpha \cdot 0 = 0 $$
        which proves that $\alpha x \in U$. Therefore, $U$ is a subspace of $\F^3$.
        \item  Since $0 = (0,0,0)$ doesn't satisfy $x_1 + 2x_2 + 3x_3 = 4$, then the set of such vectors cannot be a subspace since it doesn't contain the zero vector.
        \item Let $U = \{(x_1,x_2,x_3)\in \F^3 : x_1 x_2 x_3  = 0\}$ and notice that that both $x = (1,1,0)$ and $y = (0,0,1)$ are in $U$. However, $x + y$ is obviously not in $U$ since $x + y = (1,1,1)$ and $1 \cdot 1 \cdot 1 = 1$. Therefore, $U$ is not a subspace of $\F^3$.
        \item Define $U = \{(x_1,x_2,x_3)\in \F^3 : x_1 = 5x_3\}$ and let's show that it is a subspace of $\F^3$. First, since $0 = 5 \cdot 0$, then $0 = (0,0,0) \in U$. To prove that $U$ is closed under addition, let $x = (x_1, x_2, x_3)$ and $y=(y_1, y_2, y_3)$ be two arbitrary elements of $U$, then by definition:
        $$\begin{cases}
            x_1 = 5 x_3 \\
            y_1 = 5 y_3
        \end{cases}$$
        By adding the two equations together, we get
        $$x_1 + y_1 = 5(x_3 + y_3)$$
        Thus, $x + y = (x_1 + y_1, x_2 + y_2, x_3 + y_3) \in U$. Finally, to prove that $U$ is closed under scalar multiplication, let $x = (x_1, x_2, x_3)$ be an element of $U$ and $\alpha \in \F$, then
        \begin{align*}
            x_1 = 5x_3 &\implies \alpha x_1 = \alpha(5x_3) \\
            &\implies \alpha x_1 = 5 (\alpha x_3)
        \end{align*}
        Thus, $\alpha x = (\alpha x_1, \alpha x_2, \alpha x_3) \in U$. Therefore, $U$ is a subspace of $\F^3$. \\
    \end{enumerate}
\end{solution}

\begin{exercise}
    Verify all assertions about subspaces in Example 1.35:
    \begin{enumerate}[label=(\alph*)]
        \item If $b \in \F$, then
        $$\{(x_1, x_2, x_3, x_4)\in \F^4 : x_3 = 5x_4 + b\}$$
        is a subspace of $\F^4$ if and only if $b=0$.
        \item The set of continuous real-valued functions on the interval [0,1] is a subspace of $\R^{[0,1]}$.
        \item The set of differentiable real-valued functions on $\R$ is a subspace of $\R^{\R}$.
        \item The set of differentiable real-valued functions $f$ on the interval (0,3) such that $f'(2) = b$ is a subspace of $\R^{(0,3)}$ if and only if $b = 0$.
        \item The set of all sequences of complex numbers with limit 0 is a subspace of $\C^{\infty}$. \\
    \end{enumerate}
\end{exercise}

\begin{solution}
    \begin{enumerate}[label=(\alph*)]
        \item Define $U_b = \{(x_1, x_2, x_3, x_4)\in \F^4 : x_3 = 5x_4 + b\}$ for all $b \in \F$ and suppose first that $U$ is a subspace of $\F^4$, then it must contain the zero vector. Hence, since $(0,0,0,0) \in U$, then by definition: 
        $$0 = 5\cdot 0 + b$$
        which is equivalent to $b=0$. \\
        For the converse, let's show that $U_0$ is a subspace of $\F^4$. Since $0 = 5\cdot 0$, then $0 = (0,0,0,0) \in U_0$. If $x = (x_1, x_2, x_3, x_4)$ and $y = (y_1, y_2, y_3, y_4)$ are arbitrary elements of $U_0$, then $x_3 = 5x_4$ and $y_3 = 5y_4$. By adding these two equations and by distributivity, we get
        $$x_3 + y_3 = 5(x_4 + y_4)$$
        which implies that $x+y \in U_0$. Similarly, if $x = (x_1, x_2, x_3, x_4) \in U_0$ and $\alpha \in \F$, then we get
        $$x_3 = 5x_4 \implies \alpha x_3 = 5(\alpha x_4)$$
        which implies that $\alpha x \in U_0$. Thus, $U_0$ is a subspace of $\F^4$. Therefore, $U_b$ is a subspace of $\F^4$ if and only if $b=0$.
        \item Let $C$ denote the set of real-valued continuous functions on the interval [0,1] and $0_{\R^{[0,1]}}$ the zero function which acts as the additive identity in $\R^{[0,1]}$. Since the constant zero function is continuous, then $0_{\R^{[0,1]}} \in C$. Similarly, since the sum of two continuous functions is continuous and the multiplication of a continuous function with a scalar is still continuous, then $C$ is closed under addition and scalar multiplication. Therefore, $C$ is a subspace of $\R^{[0,1]}$.
        \item The proof is similar to part (b). The constant zero function is differentiable on $\R$. Moreover, differentiable functions are closed under addition and scalar multiplication. Therefore, the set of differentiable real-valued functions on $\R$ is a subspace of $\R^{\R}$.
        \item Define $U_b = \{f:(0,3) \to \R \text{ differentiable } : f'(2) = b\}$ for all $b \in \R$. Suppose that $U_b$ is a subspace of $\R^{(0,3)}$, then we must have $0_{(0,3)} \in U_b$ where $0_{(0,3)}$ denotes the constant zero function on (0,3). By definition of $U_b$, it implies that $0_{(0,3)}'(2) = b$. However, we know that $0_{(0,3)}'(2) = 0$. Thus, $b = 0$. \\
        Conversely, let's show that $U_0$ is a subspace of $\R^{(0,3)}$. First, the constant zero function $0_{\R^{(0,3)}}$ on (0,3) which acts as the additive identity in $\R^{(0,3)}$, is differentiable on (0,3) and its derivative at 2 is 0. Hence, $0_{\R^{(0,3)}} \in U_0$. Now, let $f, g \in U_0$, then $f+g$ is differentiable on (0,3) and
        $$(f+g)'(2) = f'(2) + g'(2) = 0 + 0 = 0$$
        so $f+g \in U_0$. Similarly, for any $f \in U_0$ and $\alpha \in \F$, the function $\alpha f$ is still differentiable on (0,3) and 
        $$(\alpha f)'(2) = \alpha f'(2) = \alpha \cdot 0 = 0$$
        so $\alpha f \in U_0$. Thus, $U_0$ is a subspace of $\R^{(0,3)}$. Therefore, $U_b$ is a subspace if and only if $b=0$.
        \item Let $S$ be the set of sequences of complex numbers with limit 0. Since the additive identity $(0,0,...)$ of $C^{\infty}$ converges to 0, then it is in $S$. Let $(a_n)_n, (b_n)_n \in S$, then
        $$\lim_{n \rightarrow \infty}a_n = 0 \quad \text{ and } \quad \lim_{n \rightarrow \infty}b_n = 0$$
        which implies
        $$\lim_{n \rightarrow \infty}(a_n+b_n) = \lim_{n \rightarrow \infty}a_n + \lim_{n \rightarrow \infty}b_n = 0+0 = 0$$
        Thus, $(a_n)_n + (b_n)_n \in S$. Similarly, for all $(a_n)_n \in S$ and $\alpha \in \C$, we have
        $$\lim_{n \rightarrow \infty}\alpha a_n = \alpha \lim_{n \rightarrow \infty}a_n = \alpha \cdot 0 = 0 $$
        so $\alpha (a_n)_n \in S$. Therefore, $S$ is a subspace of $\C^{\infty}$. \\
    \end{enumerate}
\end{solution}

\begin{exercise}
    Show that the set of differentiable real-valued functions $f$ on the interval (-4, 4) such that $f'(-1) = 3f(2)$ is a subspace of $\R^{(-4,4)}$.\\
\end{exercise}

\begin{solution}
    \\ Define the set $U = \{f : (-4, 4) \to \R \text{ differentiable } : f'(-1) = 3f(2)\}$ and let's show that it is a subspace of $\R^{(-4, 4)}$. First, denote by $f_0$ to constant zero function on $(-4, 4)$ which is also the additive identity in $\R^{(-4, 4)}$. We know that $f_0$ is differentiable on $(-4, 4)$ with $f_0' = f_0$. Hence, $f_0'(-1)= 0 = 3f_0(2)$ which proves that $f_0 \in U$.\\
    To show that it is closed under addition, let $f,g \in U$, then by definition, $f$ and $g$ are differentiable on (-4, 4) and 
    $$\begin{cases}
        f'(-1) = 3f(2) \\ g'(-1) = 3g(2)
    \end{cases}$$
    If we add these two equations, we get
    $$(f+g)'(-1) = 3(f+g)(2)$$
    which proves that $f+g \in U$ since $f+g$ is differentiable on (-4, 4).\\
    To prove that it is closed under scalar multiplication, let $f \in U$ and $\alpha \in \R$, then
    \begin{align*}
        f'(-1) = 3f(2) &\implies \alpha f'(-1) = \alpha \cdot 3f(2) \\
        &\implies (\alpha f)'(-1) = 3(\alpha f)(2)
    \end{align*}
    which proves that $\alpha f \in U$ since $\alpha f$ is differentiable on (-4, 4). Therefore, $U$ is a subspace of $\R^{(-4,4)}$. \\
\end{solution}

\begin{exercise}
    Suppose $b \in \R$. Show that the set of continuous real-valued functions $f$ on the interval $[0,1]$ such that $\int_{0}^{1}f = b$ is a subspace of $\R^{[0,1]}$ if and only if $b=0$. \\
\end{exercise}

\begin{solution}
    \\ Let $b \in \R$ and define $I = \{f : [0,1] \to \R \text{ continuous }: \int_{0}^{1}f = b \}$. Suppose that $I$ is a subspace of $\R^{[0,1]}$, then the additive identity $0 : x \mapsto 0$ must be in $I$ so $\int_{0}^{1}0 = b$. But we know that $\int_{0}^{1}0 = 0$ so it follows that $b=0$. \\
    Conversly, let's show that $I$ is a subspace of $\R^{[0,1]}$ when $b=0$. First, the additive identity $0$ is obviously continuous with $\int_{0}^{1}0 = 0$ so $0 \in I$. Now, let $f, g \in I$, then $f$ and $g$ are continuous and 
    $$\int_{0}^{1}f = \int_{0}^{1}g = 0$$
    It follows that $f+g$ is a continuous function that satisfies
    $$\int_{0}^{1}(f+g) = \int_{0}^{1}f + \int_{0}^{1}g = 0$$
    Hence, $f+g \in I$. Similarly, if $f \in I$ and $\alpha \in \R$, then $f$ is continuous and
    $$\int_{0}^{1}f = 0$$
    which implies that $\alpha f$ is also continuous and
    $$\int_{0}^{1}(\alpha f) = \alpha \int_{0}^{1}f = \alpha \cdot 0 = 0$$
    Hence, $\alpha f \in I$. Therefore, $I$ is a subspace of $\R^{[0,1]}$. \\
\end{solution}

\begin{exercise}
    Is $\R^2$ a subspace of the complex vector space $\C^2$?\\
\end{exercise}

\begin{solution}
    \\ No, it isn't because it is not closed under scalar multiplication since the scalars are complex numbers. For example, $(1,1) \in \R^2$ but $i(1,1) = (i,i) \notin \R^2$. Therefore, $\R^2$ is not a subspace of the complex vector space $\C^2$.\\
\end{solution}

\begin{exercise}
    \begin{enumerate}[label=(\alph*)]
        \item Is $\{(a,b,c) \in \R^3 : a^3 = c^3\}$ a subspace of $\R^3$? \\
        \item Is $\{(a,b,c) \in \C^3 : a^3 = c^3\}$ a subspace of $\C^3$? \\
    \end{enumerate}
\end{exercise}

\begin{solution}
    \begin{enumerate}[label=(\alph*)]
        \item In $\R$, the function $x \mapsto x^3$ is bijective so if we define $I = \{(a,b,c) \in \R^3 : a^3 = c^3\}$, then we actually have $I = \{(a,b,c) \in \R^3 : a = c\}$. Hence, it is easier now to show that $I$ is a subspace of $\R^3$. Obviously, $(0,0,0) \in I$ since $0 = 0$. Moreover, if $(x_1,x_2,x_3)$ and $(y_1, y_2, y_3)$ are in $I$, then $x_1 = x_3$ and $y_1 = y_3$ which implies that $x_1 + y_1 = x_3 + y_3$. Hence, $(x_1 + y_1, x_2 + y_2, x_3 + y_3)$ in in $I$. Similarly, for $(x_1, x_2, x_3) \in I$ and $\alpha \in \R$, we must have $x_1 = x_3$ which implies that $\alpha x_1 = \alpha x_3$. Thus, $(\alpha x_1, \alpha x_2, \alpha x_3) \in I$. Therefore, $I$ is a subspace of $\R^3$.
        \item If we let $I = \{(a,b,c) \in \R^3 : a^3 = c^3\}$, notice that $(\frac{-1+\sqrt{3}i}{2}, 0, 1)$ and $(\frac{-1-\sqrt{3}i}{2}, 0, 1)$ are both elements of $I$. However, their sum is not in $I$ since
        $$\left( \frac{-1+\sqrt{3}i}{2}, 0, 1 \right) + \left( \frac{-1-\sqrt{3}i}{2}, 0, 1 \right) = (-1, 0, 2) \notin I$$
        Therefore, it is not a subspace of $\C^3$ since it is not closed under addition. \\
    \end{enumerate}
\end{solution}

\begin{exercise}
    Prove or give a counterexample: If $U$ is a nonempty subset of $\R^2$ such that $U$ is closed under addition and under taking inverses (meaning $-u \in U$ whenever $u \in U$), then $U$ is a subspace of $\R^2$. \\
\end{exercise}

\begin{solution}
    \\ Consider the set $U = \{(k,k) : k\in\Z\}$ which is obviously closed under addition and taking inverses. Notice that $U$ is not a subspace because it is not closed under scalar multiplication: $(1,1) \in U$ and $\pi \in \R$ but $\pi(1,1) = (\pi, \pi) \notin U$. \\
\end{solution}

\begin{exercise}
    Give an example of a nonempty subset $U$ of $\R^2$ such that $U$ is closed under scalar multiplication, but $U$ is not a subspace of $\R^2$. \\
\end{exercise}

\begin{solution}
    \\ Consider the set $U = \{(x,y) \in \R^2 : xy \geq 0\}$, let's first show that it is closed under scalar multiplication. Given $(x,y) \in U$ and $\alpha \in \R$, we know by definition of $U$ tht $xy \geq 0$. Moreover, since $\alpha$ is a real number, then $\alpha^2 \geq 0$. Hence,
    $$(\alpha x)(\alpha y) = \alpha^2 xy \geq 0$$
    Thus, $(\alpha x, \alpha y) \in U$ so $U$ is indeed closed under scalar multiplication. To show that $U$ is not a subspace, consider the elements $(-1, 0)$ and $(0, 1)$ in $U$ and notice that their addition cannot be in $U$ since $(-1) \cdot 1 \not\geq 0$. Thus, $U$ is not closed under addition which proves that it is not a subspace.\\
\end{solution}

\begin{exercise}
    A function $f : \R \to \R$ is called \textit{periodic} if there exists a positive number $p$ such that $f(x+p) = f(x)$ for all $x \in \R$. Is the set of periodic functions from $\R$ to $\R$ a subspace of $\R^{\R}$? Explain.\\
\end{exercise}

\begin{solution}
    \\ Let's prove that this set is not a subspace of $\R^{\R}$ by showing that it is not closed under addition. To do so, consider the functions $x \mapsto \cos(x)$ and $x \mapsto \cos(\pi x)$ defined on $\R$. Obviously, both are periodic since the first one has period $2\pi$ and the second one has period 2. Consider their sum $f : \cos(x) + \cos(\pi x)$ and suppose by contradiction that there exists a $p > 0$ such that
    \[f(x) = f(x+p) \tag*{(1)}\]
    for all $x \in \R$. Notice that
    \begin{align*}
        f(x) = 2 &\implies \cos(x) + \cos(\pi x) = 2\\
        &\implies \cos(x) = 1 \quad \text{ and } \quad \cos(\pi x) = 1 \\
        &\implies x \in 2\pi \Z \quad \text{ and } \quad x \in 2 \Z \\
        &\implies x = 0
    \end{align*}
    Hence, $f$ is equal to 2 if and only if $x = 0$. Thus, if we plug-in $x = 0$ in equation (1), we get
    $$f(p) = f(0) = 2$$
    which implies that $p = 0$, a contradiction since $p > 0$. Therefore, $f$ is not periodic which proves that periodic functions are not closed under addition. With a similar argument, periodic functions are not closed under multiplication either. \\
\end{solution}

\begin{exercise}
    Suppose $V_1$ and $V_2$ are subspaces of $V$. Prove that $V_1 \cap V_2$ is a subspace of $V$. \\
\end{exercise}

\begin{solution}
    \\ Let's show that $V_1 \cap V_2$ satisfies the three subspace conditions:
    \begin{itemize}
        \item (\textbf{additive identity}) Since $V_1$ and $V_2$ are subspaces, then they both contain the additive identity $0$ of $V$. It follows that $0 \in V_1 \cap V_2$ since it is contained in both sets.
        \item (\textbf{closed under addition}) Let $u$ and $v$ be two vectors in $V_1 \cap V_2$, then $u$ and $v$ must be contained in $V_1$. Since $V_1$ is a subspace, then it is closed under addition so $u+v$ must also be an element of $V_1$. Similarly, $u$ and $v$ are contained in $V_2$ so for the same reasons, $u+v$ must be an element of $V_2$. Thus, $u+v \in V_1 \cap V_2$ since $u+v \in V_1$ and $u+v \in V_2$.
        \item (\textbf{closed under scalar multiplication}) Let $a \in \F$ and $u \in V_1 \cap V_2$, then $u$ must be contained in $V_1$. Since $V_1$ is a subspace, then it is closed under scalar multiplication so $au$ must also be an element of $V_1$. Similarly, $u$ is contained in $V_2$ so for the same reasons, $au$ must be an element of $V_2$. Thus, $au \in V_1 \cap V_2$ since $au \in V_1$ and $au \in V_2$.
    \end{itemize}
    Therefore, $V_1 \cap V_2$ is a subspace of $V$. \\
\end{solution}

\begin{exercise}
    Prove that the intersection of every collection of subspaces of $V$ is a subspace of $V$. \\
\end{exercise}

\begin{solution}
    \\ Let $\{V_i\}_{i \in I}$ be an arbitrary collection of subspaces of $V$, let's show that $\cap_{i \in I}V_i$ is also a subspace of $V$ by proving the three subspace conditions:
    \begin{itemize}
        \item (\textbf{additive identity}) Since $V_i$ is a subspace of $V$, then $0 \in V_i$ for all $i \in I$. It follows that $0 \in \cap_{i \in I} V_i$.
        \item (\textbf{closed under addition}) Let $u$ and $v$ be two vectors in $\cap_{i \in I} V_i$, then $u$ and $v$ must be contained in $V_i$ for all $i \in I$. For any $i \in I$, $V_i$ is a subspace so it is closed under addition, hence $u+v \in V_i$. It follows that $u +v \in \cap_{i \in I}V_i$.
        \item (\textbf{closed under scalar multiplication}) Let $a \in \F$ and $v \in \cap_{i \in I}$. For all $i \in I$, since $u \in V_i$ and $V_i$ is a subspace, then $au \in V_i$. It follows that $au \in \cap_{i \in I}V_i$ since $au \in V_i$ for all $i \in I$.
    \end{itemize}
    Therefore, $\cap_{i \in I}V_i$ is a subspace of $V$.\\
\end{solution}

\begin{exercise}
    Prove that the union of two subspaces of $V$ is a subspace of $V$ if and only if one of the subspaces is contained in the other. \\
\end{exercise}

\begin{solution}
    \\ Let $V_1$ and $V_2$ be subspaces of $V$. If $V_1 \subset V_2$ or $V_2 \subset V_1$, then $V_1 \cup V_2$ must be a subspace of $V$ as well. To show the converse, suppose now that $V_1 \cup V_2$ is a subspace of $V$ and that $V_1 \not\subset V_2$. Then there exists a vector $u_1 \in V_1$ such that $u_1 \notin V_2$. Let's prove that $V_2 \subset V_1$ in that case. Let $v \in V_2$ be arbitrary, since $u_1$ and $v$ are both vectors in $V_1 \cup V_2$, then $u_1 + v \in V_1 \cup V_2$ since it is a subspace. But this implies that $u_1 + v$ is either in $V_1$ or in $V_2$. If $u_1 + v \in V_2$, then we must have
    $$u_1 = (u_1 + v) - v \in V_2$$
    since $v \in V_2$ and $V_2$ is a subspace. A contradiction since $u_1 \notin V_2$. It follows that $u_1 + v \in V_1$. But again, since $V_1$ is a subspace and $u_1 \in V_1$, then
    $$v = (u_1 + v) - u_1 \in V_1$$
    which proves that $V_2 \subset V_1$. Therefore, if $V_1 \cup V_2$ is a subspace, then we either have $V_1 \subset V_2$ or $V_2 \subset V_1$. \\
\end{solution}

\begin{exercise}
    Prove that the union of three subspaces of $V$ is a subspace of $V$ if and only if one of the subspaces contains the other two. \\
\end{exercise}

\begin{solution}
    \\ Let $V_1$, $V_2$ and $V_3$ be three subspaces of $V$. Obviously, if one contains the other two, then $V_1 \cup V_2 \cup V_3$ is also a subspace of $V$. To show the converse, suppose that $V_1 \cup V_2 \cup V_3$ is a subspace of $V$. \td \\
\end{solution}

\begin{exercise}
    Suppose
    $$U = \{(x, -x, 2x) \in \F^3:x \in \F\} \quad \text{ and } \quad W = \{(x, x, 2x) \in \F^3:x \in \F\} $$
    Describe $U+W$ using symbols, and also give a description of $U + W$ that uses no symbols. \\
\end{exercise}

\begin{solution}
    \\ By definition, we have
    \begin{align*}
        U+W &= \{(x, -x, 2x) \in \F^3:x \in \F\} + \{(x, x, 2x) \in \F^3:x \in \F\} \\
        &= \{(x+y, -x+y, 2x+2y) \in \F^3:x,y \in \F\} \\
        &= \{(x+y, -x+y, 2(x+y) \in \F^3:x,y \in \F\}
    \end{align*}
    From this expression, let's prove that
    $$U+W = \{(a, b, 2a) \in \F^3:a,b \in \F\}$$
    Obviously, $U+W \subset \{(a, b, 2a) \in \F^3:a,b \in \F\}$ because for any vector $(x+y, -x+y, 2(x+y) \in \F^3$, if we let $a = x+y$ and $b = -x+y$, we can rewrite this vector as $(a,b,2a)$ which is in $ \{(a, b, 2a) \in \F^3:a,b \in \F\}$. Similarly, given an arbitrary vector $(a, b, 2a) \in \F^3$, if we let
    $$x = \frac{a-b}{2} \qquad \text{ and } \qquad y = \frac{x+y}{2}$$
    then we can rewrite the vector as $(x+y, -x+y, 2(x+y))$ which is obviously in $U+W$. It follows that the sets are equal. Without symbols, this just means that $U+W$ is precisely the set of vectors in $V$ such that the third component is twice the first component. \\
\end{solution}

\begin{exercise}
    Suppose $U$ is a subspace of $V$. What is $U+U$? \\
\end{exercise}

\begin{solution}
    \\ Let's show that $U = U+U$. An arbitrary element in $U+U$ is of the form $x+y$ where $x$ and $y$ are in $U$. Since $U$ is a subspace, then it is closed under addition which implies that $x+y \in U$. It follows that $U+U \subset U$. \\
    For the reverse inclusion, take an arbitrary $u \in U$ and notice that we can write $u = u + 0$. Again, since $U$ is a subspace of $V$, then $0 \in V$. Thus, in the expression $u+0$, both vectors are in $U$. It follows that $u  = u+0 \in U+U$. Therefore, $U = U+U$.\\
\end{solution}

\begin{exercise}
    Is the operation of addition on the subspaces of $V$ commutative? In other words, if $U$ and $W$ are subspaces of $V$, is $U+W = W +U$? \\
\end{exercise}

\begin{solution}
    \\ Let $U$ and $W$ be subspaces of $V$. Then by commutativity of addition in $V$, we get
    \begin{align*}
        U+W &= \{u+w : u \in U \text{ and } w \in W\} \\
        &= \{w+u : w \in W \text{ and } u \in U\} \\
        &= W + U
    \end{align*}
    Therefore, the operation of addition on subspaces of $V$ is commutative. \\
\end{solution}

\begin{exercise}
    Is the operation of addition on the subspaces of $V$ associative? In other words, if $V_1$, $V_2$ and $V_3$ are subspaces of $V$, is
    $$(V_1 + V_2) + V_3 = V_1 + (V_2 + V_3)?$$ \\
\end{exercise}

\begin{solution}
    \\ Let $V_1$, $V_2$ and $V_3$ are subspaces of $V$ and let's show that 
    $$(V_1 + V_2) + V_3 = V_1 + (V_2 + V_3)$$
    First, take an arbitrary $x + y \in (V_1 + V_2) + V_3$ where $x \in V_1 + V_2$ and $y \in V_3$. Since $x \in V_1 + V_2$, then there exist vectors $a \in V_1$ and $b \in V_2$ such that $x = a+b$. It follows from the associativity of addition in $V$ that
    $$x+y = (a+b) + y = a + (b + y)$$
    Since $b \in V_2$ and $y \in V_3$, then $b+y \in V_2 + V_3$. Hence, $a + (b+y) \in V_1 + (V_2 + V_3)$ using the fact that $a \in V_1$. Thus, the arbitrary $x+y \in (V_1 + V_2) + V_3$ is in $V_1 + (V_2 + V_3)$ as well so 
    $$(V_1 + V_2) + V_3 \subset V_1 + (V_2 + V_3)$$
    The reverse inclusion has the same proof. The desired equality follows. \\
\end{solution}

\begin{exercise}
    Does the operation of addition on subspaces of $V$ have an additive identity? Which subspaces have additive inverses? \\
\end{exercise}

\begin{solution}
    \\ First, let's show that indeed, the operation of addition on subspaces of $V$ has an additive identity. Define $I = \{0\}$, the subspace of $V$ containing the zero vector only. Take an arbitrary subspace $U$ of $V$ and notice that
    \begin{align*}
         U + I &= \{u+i: u\in U \text{ and } i \in I\} \\
         &= \{u+0 : u \in U \text{ and } i \in I\} \\
         &= \{u : u \in U\} \\
         &= U
    \end{align*}
    By commutativity of addition of subspaces of $V$, we also have $I + U = U$. Therefore, $I$ is an additive identity for the addition on subspaces of $V$. \\
    Concerning additive inverses, let's determine which subspaces of $V$ have an additive inverse by taking an arbitrary subspace $U$ of $V$ and supposing that there is a subspace $W$ of $V$ such that $V + W = I$. Since $W$ is a subspace of $V$, then $0 \in W$. It follows that for all $u \in U$, 
    $$u  = u + 0 \in U+W = I = \{0\}$$
    In other words, $U = \{0\} = I$. Since $I$ obviously has an additive inverse (itself), then the unique subspace having an additive inverse is $I$. \\
\end{solution}

\begin{exercise}
    Prove or give a counterexample: If $V_1$, $V_2$, $U$ are subspaces of $V$ such that
    $$V_1 + U = V_2 + U,$$
    then $V_1 = V_2$.\\
\end{exercise}

\begin{solution}
    \\ Consider the following counterexample. Let $V_1 = U =V$ and $V_2 = \{0\}$. We know from Exercise 15 of this section that
    $$V_1 + U = V + V = V$$
    Moreover, from Exercise 19, we also have
    $$V_2 + U = \{0\} + V = V$$
    Thus,
    $$V_1 + U = V_2 + U$$
    but $V_1 \neq V_2$. \\
\end{solution}

\begin{exercise}
    Suppose 
    $$U = \{(x,x,y,y) \in \F^4 : x,y \in \F\}.$$
    Find a subspace $W$ of $\F^4$ such that $\F^4 = U \oplus W$. \\
\end{exercise}

\begin{solution}
    \\ Consider the subspace
    $$W = \{(0,a,0,b) \in \F^4 : a,b \in \F\}$$
    and the sum $U + W$. First, let's show that the sum is direct by proving that $(0,0,0,0)$ has a unique representation in this sum. Suppose $(x,x,y,y) \in U$ and $(0,a,0,b) \in W$ satisfy
    $$(0,0,0,0) = (x,x,y,y) + (0,a,0,b)$$
    This is equivalent to the system of equation
    $$\begin{cases}
        x = 0 \\ a+x = 0 \\ y = 0 \\ b + y = 0
    \end{cases}$$
    which clearly has the following unique solution
    $$\begin{cases}
        x = 0 \\ a = 0 \\ y = 0 \\ b = 0
    \end{cases}$$
    Therefore, in $U + W$, the zero vector can only be written as the sum of two zero vectors. It follows that the sum is direct.\\
    Let's now show that $U \oplus W = \F^4$ by taking an arbitrary vector $(x_1, x_2, x_3, x_4)$. Consider the vectors 
    $$u = (x_1, x_1, x_3, x_3) \in U \quad \text{ and } \quad w = (0, x_2 - x_1, 0, x_4 - x_3) \in W$$
    and notice that
    \begin{align*}
        u+w &= (x_1, x_1, x_3, x_3) + (0, x_2 - x_1, 0, x_4 - x_3) \\
        &= (x_1, x_1 + x_2 - x_1, x_3, x_3 + x_4 - x_3) \\
        &= (x_1, x_2, x_3, x_4)
    \end{align*}
    which shows that $(x_1, x_2, x_3, x_4) \in U \oplus W$. Thus, $\F^4 \subset U \oplus W$. Since $U$ and $W$ are subspaces of $\F^4$, then $U \oplus W$ must also be a subspace of $\F^4$: $U \oplus W \subset \F^4$. Therefore, we get $U \oplus W = \F^4$. \\
\end{solution}

\begin{exercise}
    Suppose 
    $$U = \{(x, y, x+y, x-y, 2x) \in \F^5 : x,y \in \F \}.$$
    Find subspace $W$ of $\F^5$ such that $\F^5 = U \oplus W$. \\
\end{exercise}

\begin{solution}
    \\ Consider the subspace
    $$W = \{(0, 0, a, b, c) \in \F^5 : a,b,c \in \F \}$$
    and consider the sum $U + W$. Let's first prove that it is actually a direct sum by focusing on the vector zero. Let $(x, y, x+y, x-y, 2x) \in U$ and $(0, 0, a, b, c) \in W$ be two vectors such that
    $$(0,0,0,0) = (x, y, x+y, x-y, 2x) + (0, 0, a, b, c)$$
    This translates to the following system of equation:
    $$\begin{cases}
        x=0 \\ y = 0\\x + y + a = 0 \\ x - y + b = 0 \\ 2x + c = 0
    \end{cases}$$
    which is equivalent to
    $$\begin{cases}
        x=0 \\ y=0 \\ a=0 \\ b=0 \\ c=0
    \end{cases}$$
    Therefore, since the zero vector can only be written as the sum of two zero vectors, the sum is direct. \\
    Let's now show that $U \oplus W = \F^5$. Obviously, since $U \oplus W$ is a subspace of $\F^5$, we have $U \oplus W \subset \F^5$. Moreover, for any $(x_1, x_2, x_3, x_4, x_5) \in \F^5$, we have
    \begin{align*}
        (&x_1, x_2, x_3, x_4, x_5)\\
        &= (x_1, x_2, [x_1 + x_2] + [x_3 - x_2 - x_1], [x_1 - x_2] + [x_4 + x_2 - x_1], 2x_1 + [x_5 - 2x_1]) \\
        &= (x_1, x_2, x_1+x_2, x_1 - x_2,2x_1) + (0,0,x_3 - x_2 - x_1, x_4 + x_2 - x_1, x_5 - 2x_1) \\
        &= (x,y,x+y,x-y, 2x) + (0,0,a,b,c) \\
        &\in U \oplus W
    \end{align*}
    where $x = x_1$, $y=x_2$, $a = x_3 - x_2 - x_1$, $b = x_4 + x_2 - x_1$ and $c = x_5 - 2x_1$. Thus, $\F^5 \subset U \oplus W$. Therefore, $U \oplus W = \F^5$. \\
\end{solution}

\begin{exercise}
    Suppose 
    $$U = \{(x, y, x+y, x-y, 2x) \in \F^5 : x,y \in \F \}.$$
    Find three subspaces $W_1$, $W_2$, $W_3$ of $\F^5$, none of which equals $\{0\}$, such that $\F^5 = U \oplus W_1 \oplus W_2 \oplus W_3$. \\
\end{exercise}

\begin{solution}
    \\ Consider the subspaces
    \begin{align*}
        W_1 &= \{(0, 0, a, 0, 0) \in \F^5 : a \in \F \} \\
        W_2 &= \{(0, 0, 0, b, 0) \in \F^5 : b \in \F \} \\
        W_3 &= \{(0, 0, 0, 0, c) \in \F^5 : c \in \F \}
    \end{align*}
    and their sum $U + W_1 + W_2 + W_3$. Let's first prove that it is actually a direct sum by focusing on the zero vector. Let 
    \begin{align*}
        u &= (x, y, x+y, x-y, 2x) \in U \\
        w_1 &= (0, 0, a, 0, 0) \in W_1 \\
        w_2 &= (0,0,0,b,0) \in W_2 \\
        w_3 &= (0,0,0,0,c) \in W_3
    \end{align*}
    be arbitrary vectors in their respective sets such that
    $$(0,0,0,0,0) = u + w_1 + w_2 + w_3$$
    This can be rewritten into the following system of equation:
    $$\begin{cases}
        x=0 \\ y = 0\\x + y + a = 0 \\ x - y + b = 0 \\ 2x + c = 0
    \end{cases}$$
    which is equivalent to
    $$\begin{cases}
        x=0 \\ y=0 \\ a=0 \\ b=0 \\ c=0
    \end{cases}$$
    Hence, $u = w_1 = w_2 = w_3 = (0,0,0,0,0)$. Therefore, since the zero vector can only be written as the sum of zero vectors, the sum is direct. \\
    Let's now show that $U \oplus W_1 \oplus W_2 \oplus W_3 = \F^5$. Obviously, since $U \oplus W_1 \oplus W_2 \oplus W_3$ is a subspace of $\F^5$, we have $U \oplus W_1 \oplus W_2 \oplus W_3 \subset \F^5$. Moreover, for any $(x_1, x_2, x_3, x_4, x_5) \in \F^5$, we have
    \begin{align*}
        (&x_1, x_2, x_3, x_4, x_5)\\
        &= (x_1, x_2, [x_1 + x_2] + [x_3 - x_2 - x_1], [x_1 - x_2] + [x_4 + x_2 - x_1], 2x_1 + [x_5 - 2x_1]) \\
        &= (x_1, x_2, x_1+x_2, x_1 - x_2,2x_1) + (0,0,x_3 - x_2 - x_1, x_4 + x_2 - x_1, x_5 - 2x_1) \\
        &= (x,y,x+y,x-y, 2x) + (0,0,a,b,c) \\
        &= (x,y,x+y,x-y, 2x) + (0,0,a,0,0) + (0,0,0,b,0) + (0,0,0,0,c) \\
        &\in U \oplus W
    \end{align*}
    where $x = x_1$, $y=x_2$, $a = x_3 - x_2 - x_1$, $b = x_4 + x_2 - x_1$ and $c = x_5 - 2x_1$. Thus, $\F^5 \subset U \oplus W_1 \oplus W_2 \oplus W_3$. Therefore, $U \oplus W_1 \oplus W_2 \oplus W_3 = \F^5$. \\
\end{solution}

\begin{exercise}
    Prove or give a counterexample: If $V_1$, $V_2$, $U$ are subspaces of $V$ such that
    $$V = V_1 \oplus U \quad \text{ and } \quad V = V_2 \oplus U,$$
    then $V_1 = V_2$. \\
\end{exercise}

\begin{solution}
    \\ Consider the following counterexample:
    \begin{align*}
        V &= \R^2 \\
        V_1 &= \{(0, x) : x\in \R\} \\
        V_2 &= \{(x,x) : x \in \R\} \\
        U &= \{(x,0) : x \in \R\}
    \end{align*}
    I will not prove that $V_1$, $V_2$ and $U$ are subspaces of $V$ because it is not goal of this exercice. I assume that this exer
\end{solution}

% CHAPTER 2
\chapter{Finite-Dimensional Vector Spaces}

\section{Span and Linear Independence}

[Coming soon...]
\section{Bases}

\begin{exercise}
    Find all vector spaces that have exactly one basis. \\
\end{exercise}

\begin{solution}
    \\ \td \\
\end{solution}

\begin{exercise}
    Verifiy all assertions in Example 2.27. \\
\end{exercise}

\begin{solution}
    \\ \td \\
\end{solution}
\section{Measures and Their Properties}

\begin{exercise}
    Explain why there does not exist a measure space $(X, \mathcal{S}, \mu)$ with the property that $\{ \mu(E) : E \in \mathcal{S}\} = [0, 1)$. \\
\end{exercise}

\begin{solution}
    \\ By contradiction, let $(X, \mathcal{S}, \mu)$ be such a measure space. Hence, by our assumptions, $\mu(X) < 1$. However, if we let $\alpha$ be any real number in $(\mu(X), 1)$, then $\alpha \in \{ \mu(E) : E \in \mathcal{S}\}$ which implies that there is a set $F \in \mathcal{S}$ such that $\mu(F) = \alpha$. But $F \subset X$, so by monotonicity:
    $$\mu(X) < \alpha = \mu(F) \leq \mu(X)$$
    A contradiction. Therefore, such a measure space cannot exist.\\
\end{solution}

\begin{exercise}
    Suppose $\mu$ is a measure on $(\N, 2^{\N})$. Prove that there is a sequence $w_1, w_2, ...$ in $[0, \infty]$ such that
    $$\mu(E) = \sum_{k \in E}w_k$$
    for every set $E \subset \N$. \\
\end{exercise}

\begin{solution}
    \\ First, define the sequence $\{w_k\}_k$ as follows
    $$w_k = \mu( \{k\} )$$
    for all $k \in \N$. Hence, for all $E \in 2^{\N}$, we can write $E$ as the disjoint union of the singletons of its elements. This disjoint union is either finite or countable since $E \subset \N$. Therefore, by finite or countable additivity:
    \begin{align*}
        \mu(E) &= \mu \left(\bigcup_{k \in E} \{k\}\right) \\
        &= \sum_{k \in E} \mu(\{k\}) \\
        &= \sum_{k \in E}w_k
    \end{align*}
    which proves our claim. \\
\end{solution}

\begin{exercise}
    Give an example of a measure $\mu$ on $(\N, 2^{\N})$ such that
    $$\{ \mu(E) : E \subset \N\} = [0, 1].$$\\
\end{exercise}

\begin{solution}
    \\ Define a measure $\mu$ on $(\N, 2^{\N})$ by:
    $$\mu(E) = \sum_{k \in E}\frac{1}{2^k}$$
    To prove that $\{ \mu(E) : E \subset \N\} = [0, 1]$, let $c \in [0,1]$ and let's show that there is a $E \subset \N$ such that $\mu(E) = c$. Notice that $c$ has a binary representation of the form:
    $$c = \sum_{k=1}^{\infty}a_k\frac{1}{2^k}$$
    where the $a_k$'s are either 0 or 1. Hence, if we define $E = \{k : a_k = 1\}$, we get
    $$\mu(E) = \sum_{k \in E}\frac{1}{2^k} = \sum_{k=1}^{\infty}a_k\frac{1}{2^k} = c$$
    which proves our claim.\\
\end{solution}

\begin{exercise}
    Give an example of a measure space $(X, \mathcal{S}, \mu)$ such that 
    $$\{ \mu(E) : E \subset \N\} = \{\infty\} \cup \bigcup_{k=0}^{\infty}[3k, 3k+1].$$\\
\end{exercise}

\begin{solution}
    \\ Let $X = \Z$, $\mathcal{S} = 2^{\Z}$ and define $\mu$ by
    $$\mu(\{k\}) = \begin{cases}
        \frac{1}{2^k} & \text{if } x \geq 1, \\
        3 & \text{if } x\leq 0
    \end{cases}$$
    Simply use the countable additivity of $\mu$ to extend $\mu$ on all of $2^{\Z}$ and not just the singletons. Let's show that for all $c \in \{\infty\} \cup \bigcup_{k=0}^{\infty}[3k, 3k+1]$, there is a $E \subset \Z$ such that $\mu(E) = c$. First, consider the case $c = \infty$, then defining $E = \{-k : k \geq 0\}$ gives us
    $$\mu(E) = \sum_{k \in E}\mu(\{k\}) = \sum_{k=0}^{\infty}\mu(\{-k\}) = \sum_{k=0}^{\infty}3 = \infty$$
    Now, consider the case $c \in [3k, 3k+1]$, then there exists an integer $k \geq 0$ and a real number $\alpha \in [0, 1]$ such that $c = 3k + \alpha$. Since we can write $\alpha$ in binary form as follows:
    $$c = \sum_{n=1}^{\infty}a_n\frac{1}{2^n}$$
    where the $a_n$'s are either 0 or 1, then we can define the set $E = \{-n : n \in \Iint{1}{k}\} \cup \{n \in \N : a_n = 1\}$:
    \begin{align*}
        \mu(E) &= \mu(\{-n : n \in \Iint{1}{k}\} \cup \{n \in \N : a_n = 1\}) \\
        &= \mu(\{-n : n \in \Iint{1}{k}\}) + \mu(\{n \in \N : a_n = 1\}) \\
        &= \sum_{n = 1}^{k}\mu(\{-n\}) + \sum_{n=1}^{\infty}a_n \mu(\{n\})\\
        &= \sum_{n = 1}^{k}3 + \sum_{n=1}^{\infty}a_n \frac{1}{2^n}\\
        &= 3k + \alpha \\
        &= c
    \end{align*}
    Since we covered all cases, we have,
    $$\{\infty\} \cup \bigcup_{k=0}^{\infty}[3k, 3k+1] \subset \{ \mu(E) : E \subset \N\}$$
    For the reverse inclusion, consider $E \subset \Z$ and let's show that $\mu(E) \in \{\infty\} \cup \bigcup_{k=0}^{\infty}[3k, 3k+1]$. To do so, notice that by definition of $\mu$, we get
    \begin{align*}
        \mu(E) &= \mu(\{k \in E : k\leq 0\}) + \mu(\{k \in E : k\geq 1\}) \\
        &= \sum_{\substack{k \in E \\ k \leq 0 }}\mu(\{k\}) + \sum_{k=1}^{\infty}a_k\mu(\{k\}) \\
        &= \sum_{\substack{k \in E \\ k \leq 0 }}3 + \sum_{k=1}^{\infty}a_k \cdot \frac{1}{2^k} \\
        &= 3\card \{k \in E : k\leq 0\} + \sum_{k=1}^{\infty}a_k \cdot \frac{1}{2^k}
    \end{align*}
    where $a_k = 1$ when $k \in E$, otherwise, $a_k = 0$. If the set $\{k \in E : k\leq 0\}$ is infinite, then $\mu(E) = \infty \in \{\infty\} \cup \bigcup_{k=0}^{\infty}[3k, 3k+1]$. If the set is finite, define $n = \card \{k \in E : k\leq 0\}$ and notice that the term $\alpha = \sum_{k=1}^{\infty}a_k \cdot \frac{1}{2^k}$ is simply the binary representation of a number in $[0, 1]$. Hence, $\mu(E) = 3n + \alpha \in [3n, 3n+1] \subset \{\infty\} \cup \bigcup_{k=0}^{\infty}[3k, 3k+1]$. Thus,
    $$\{ \mu(E) : E \subset \N\} \subset \{\infty\} \cup \bigcup_{k=0}^{\infty}[3k, 3k+1]$$
    Therefore, 
    $$\{ \mu(E) : E \subset \N\} = \{\infty\} \cup \bigcup_{k=0}^{\infty}[3k, 3k+1]$$ \\
\end{solution}

\begin{exercise}
    Suppose $(X, \mathcal{S}, \mu)$ is a measure space such that $\mu(X) < \infty$. Prove that if $\mathcal{A}$ is a set of disjoint sets in $\mathcal{S}$ such that $\mu(A) > 0$ for every $A \in \mathcal{A}$, then $\mathcal{A}$ is a countable set. \\
\end{exercise}

\begin{solution}
    \\ Let $n \in \N$ and define the collection
    $$\mathcal{A}_n = \{A \in \mathcal{A} : \mu(A) \geq \tfrac{1}{n}\}$$
    Suppose that $\mathcal{A}_n$ is infinite, then we can extract a countable sequence $\{A_n\}_n$ of sets in $\mathcal{A}_n$. Since the $A_n$'s are disjoint and in $\mathcal{S}$, then $\cup_{n=1}^{\infty}A_n \in \mathcal{S}$. Moreover, by monotonicity, we get
    \begin{align*}
        \mu(X) &\geq \mu \left(\bigcup_{n=1}^{\infty}A_n\right) \\
        &= \sum_{n=1}^{\infty}\mu(A_n) \\
        &\geq \sum_{n=1}^{\infty}\frac{1}{n} \\
        &= \infty
    \end{align*}
    A contradiction. Thus, for all $n \in \N$, the collection $\mathcal{A}_n$ is finite. But since
    $$\mathcal{A} = \bigcup_{n=1}^{\infty}\mathcal{A}_n$$
    is a countable union of finite sets, then $\mathcal{A}$ is countable. \\
\end{solution}

\begin{exercise}
    Find all $c \in [3, \infty)$ such that there exists a measure space $(X, \mathcal{S}, \mu)$ with
    $$\{\mu(E) : E\in\mathcal{S}\} = [0,1] \cup [3, c].$$ \\
\end{exercise}

\begin{solution}
    \\ Let $c \in [3, \infty)$ and suppose that there exists a measure space $(X, \mathcal{S}, \mu)$ such that 
    $$\{\mu(E) : E\in\mathcal{S}\} = [0,1] \cup [3, c]$$
    Since $\mu(X)$ is both in the set $\{\mu(E) : E\in\mathcal{S}\}$ and an upperbound for the set $\{\mu(E) : E\in\mathcal{S}\}$, then 
    $$\mu(X) = \sup \{\mu(E) : E\in\mathcal{S}\} = \sup ([0,1] \cup [3, c]) = c$$
    Since there is a $E \in \mathcal{S}$ such that $\mu(E) = 1$, then
    $$\mu(X \setminus E) = c - 1$$
    It follows that $c$ cannot be in the interval $[3, 4)$ (otherwise, we would get a set of measure in the interval [2, 3) which would contradict our assumption on the measure space). Hence, we must have $c \geq 4$. Suppose that $c > 4$, then $3 < c-1$ which implies that there must be a $\epsilon \in (0, 1)$ such that $3 \leq c - 1 - \epsilon$. Hence, there must be a set $E \in \mathcal{S}$ such that $\mu(E) = c - 1 - \epsilon$. Thus:
    \begin{align*}
        \mu(X \setminus E) &= \mu(X) - \mu(E) \\
        &= c - (c - 1 - \epsilon) \\
        &= 1 + \epsilon \\
        &\in (1, 2)
    \end{align*}
    which is a contradiction. Therefore, the only possible value for $c$ is 4. Let's now prove that there actually is a measure space $(X, \mathcal{S}, \mu)$ such that 
    $$\{\mu(E) : E\in\mathcal{S}\} = [0,1] \cup [3, 4]$$
    Consider the set $X = \N \cup \{0\}$, $\mathcal{S} = 2^X$ and define the measure $\mu$ by
    $$\mu(\{n\}) = \begin{cases}
        3 & \text{if } n = 0, \\
        \frac{1}{2^n} & \text{if }n \geq 1
    \end{cases}$$
    We can easily extend the definition of $\mu$ to any subset of $X$ by countable additivity. Let's now show that this measure space has the right property. Let $c \in [0,1] \cup [3, 4]$ and let's show that there is a set $E \subset X$ such that $\mu(E) = c$. If $c \in [0,1]$, then write $c$ in its binary form:
    $$c = \sum_{n=1}^{\infty}a_n\frac{1}{2^n}$$
    where the $a_n$'s are in the set $\{0, 1\}$. Define the set $E = \{n \in \N : a_n =1\} \subset X$. Hence, by construction:
    $$\mu(E) = \sum_{n \in E}\mu(\{n\}) = \sum_{n=1}^{\infty}a_n\frac{1}{2^n} = c$$
    Similarly, if $c \in [3,4]$, consider the binary expansion of $c - 3$, construct the set $E$ as previously and add the element 0 to the set, we would get:
    $\mu(E) = 3 + \sum_{n=1}^{\infty}b_n \frac{1}{2^n} = 3 + c - 3 = c$
    Thus, 
    $$[0,1] \cup [3, 4] \subset \{\mu(E) : E\in\mathcal{S}\}$$
    To prove the reverse inclusion, let $E \subset X$. If $0 \in E$, then
    $$\mu(E) = 3 + \sum_{n=1}^{\infty}a_n\frac{1}{2^n} \in [3, 4] \subset [0,1] \cup [3, 4]$$
    Similarly, for the same reasons, if $0 \notin E$, then $\mu(E) \in [0,1] \subset [0,1] \cup [3, 4]$. Therefore,
    $$\{\mu(E) : E\in\mathcal{S}\} = [0,1] \cup [3, 4]$$
    which proves that $c = 4$ is the only possible value.\\
\end{solution} 

\begin{exercise}
    Give an example of a measure space $(X, \mathcal{S}, \mu)$ such that
    $$\{\mu(E) : E\in\mathcal{S}\} = [0,1] \cup [3, \infty)$$ \\
\end{exercise}

\begin{solution}
    \\ Consider the measure space defined by $X = \Z$, $\mathcal{S} = 2^{\Z}$ and $\mu$ defined by 
    $$\mu(\{k\}) = \begin{cases}
        \frac{1}{2^k} & \text{if } k \geq 1,\\
        3 - k & \text{if } k\leq 0
    \end{cases}$$
    From this, it is easy to extend the definition of $\mu$ to any subset of $\Z$. Let's show that
    $$\{\mu(E) : E\in\mathcal{S}\} = [0,1] \cup [3, \infty)$$
    First, let $E \subset \Z$ and split it into $E_1 = \{k \in E : k \geq 1\}$ and $E_2 = \{k \in E : k \leq 0\}$ which are disjoint. Notice that by countable additivity, $\mu(E_2)$ can be written as $\sum_{k=1}^{\infty}a_n\frac{1}{2^n}$ where $a_n$ is 0 when $k \notin E$ and 1 when $k \in E$. But notice that this is simply a base 2 representation of a number in $[0, 1]$. Hence, $\mu(E_2) \in [0,1]$. Now, for $E_1$, notice that $\mu(E_1)$ is a sum of integers greater than or equal to 3 by countable additivity. It follows that $\mu(E_1)$ is either 0 (if it is empty) or an integer greater than or equal to 4. Therefore:
    $$\mu(E) = \mu(E_1) + \mu(E_2) \in [0,1] \cup [3, \infty)$$
    which shows that 
    $$\{\mu(E) : E\in\mathcal{S}\} \subset [0,1] \cup [3, \infty)$$
    For the reverse inclusion, Let $c \in [0,1] \cup [3, \infty)$ and let's show that there is a subset $E$ of $\Z$ such that $\mu(E) = c$. If $c \in [0,1]$, write it in binary form as $\sum_{n=1}^{\infty}a_n\frac{1}{2^n}$ and define the set $E = \{n : a_n = 1\}$, it follows that
    $$\mu(E) = \sum_{k \in E} \mu(\{k\}) = \sum_{n=1}^{\infty}a_n\frac{1}{2^n} = c$$
    Similarly, if $c \in [3, \infty)$, then there is an integer $c_0 \geq 3$ and a real $\alpha \in [0,1]$ such that $c = c_0 + \alpha$. As previously, write $\alpha$ in base 2 and define the set $E$ in the same way. Moreover, add to the set $E$ the integer $3 - c_0$, it will follow that $\mu(E) = c_0 + \alpha = c$ for the same reasons as above. Therefore, 
    $$\{\mu(E) : E\in\mathcal{S}\} = [0,1] \cup [3, \infty)$$
    which proves our claim. \\
\end{solution}

\begin{exercise}
    Give an example of a set $X$, a $\sigma$-algebra $\mathcal{S}$ of subsets of $X$, a set $\mathcal{A}$ of subsets of $X$ such that the smallest $\sigma$-algebra on $X$ containing $\mathcal{A}$ is $\mathcal{S}$, and two measures $\mu$ and $\nu$ on $(X, \mathcal{S})$ such that $\mu(A) = \nu(A)$ for all $A \in \mathcal{A}$ and $\mu(X) = \nu(X) < \infty$, but $\mu \neq \nu$. \\
\end{exercise}

\begin{solution}
    \\ Let $X = [0, 8]$, $\mathcal{A} = \{[0, 4], [2, 6]\}$, $\mathcal{S}$ the $\sigma$-algebra generated by $\mathcal{A}$ and the two follwing measures on $(X, \mathcal{S})$:
    $$\mu = \delta_1 + \delta_5$$
    $$\nu = \delta_3 + \delta_7$$
    where $\delta_i$ is the Dirac delta meaasure at $i$. We can easily prove that both $\mu$ and $\nu$ are indeed measures on $(X, \mathcal{S})$ (it will be proved in the following exercise). First, notice that
    $$\mu([0, 4]) = \delta_1([0,4]) + \delta_5([0,4]) = 1 + 0 = 1$$
    $$\nu([0, 4]) = \delta_3([0,4]) + \delta_7([0,4]) = 1 + 0 = 1$$
    and 
    $$\mu([2, 6]) = \delta_1([2, 6]) + \delta_5([2,6]) = 0 + 1 = 1$$
    $$\nu([2, 6]) = \delta_3([2,6]) + \delta_7([2,6]) = 1 + 0 = 1$$
    which implies that $\mu(A) = \nu(A)$ for all $A \in \mathcal{A}$. Moreover, 
    $$\mu(X) = \delta_1(X) + \delta_5(X) = 1 + 1 = 2$$
    $$\nu(X) = \delta_3(X) + \delta_7(X) = 1 + 1 = 2$$
    so $\mu(X) = \nu(X) < \infty$. Consider now the set $[2, 4] = [0, 4] \cap [2 ,6] \in \mathcal{S}$:
    $$\mu([2, 4]) = \delta_1([2, 4]) + \delta_5([2,4]) = 0 + 0 = 0$$
    $$\nu([2, 4]) = \delta_3([2,4]) + \delta_7([2,4]) = 1 + 0 = 1$$
    so $\mu \neq \nu$. Therefore, $X$, $\mathcal{S}$, $\mathcal{A}$, $\mu$ and $\nu$ satisfy all the desired properties. \\
\end{solution}

\begin{exercise}
    Suppose $\mu$ and $\nu$ are measures on a measurable space $(X, \mathcal{S})$. Prove that $\mu + \nu$ is a measure on $(X, \mathcal{S})$. $[$Here, $\mu + \nu$ is the usual sum of two functions: if $E \in \mathcal{S}$, then $(\mu + \nu)(E) = \mu(E) + \nu(E)$.$]$ \\
\end{exercise}

\begin{solution}
    \\ We only have two properties to prove, that the empty set is mapped to zero and the countable additivity. First,
    $$(\mu + \nu)(\varnothing) = \mu(\varnothing) + \nu(\varnothing) = 0 + 0 = 0$$
    Moreover, if $\{E_i\}_i$ is a countable collection of pairwise disjoint sets in $\mathcal{S}$, then
    \begin{align*}
        (\mu + \nu)\left(\bigcup_{i=1}^{\infty}E_i\right) &= \mu\left(\bigcup_{i=1}^{\infty}E_i\right) + \nu\left(\bigcup_{i=1}^{\infty}E_i\right) \\
        &= \sum_{i=1}^{\infty}\mu(E_i) + \sum_{i=1}^{\infty}\nu(E_i) \\
        &= \sum_{i=1}^{\infty}[\mu(E_i) + \nu(E_i)] \\
        &= \sum_{i=1}^{\infty}(\mu + \nu)(E_i) \\
    \end{align*}
    Therefore, $\mu + \nu$ is a measure on $(X, \mathcal{S})$.\\ 
\end{solution}

\begin{exercise}
    Give an example of a measure space $(X, \mathcal{S}, \mu)$ and a decreasing sequence $E_1 \supset E_2 \supset ... $ of sets in $\mathcal{S}$ such that
    $$\mu\left(\bigcap_{k=1}^{\infty}E_k\right) \neq \lim_{k \rightarrow \infty} \mu(E_k).$$\\
\end{exercise}

\begin{solution}
    \\ Let $X = \R$, $\mathcal{S} = 2^{\R}$ and $\mu$ be the counting measure on $(\R, 2^{\R})$. Consider the sequence of sets defined by $E_k = [k, \infty)$ for all $k \in \N$. Notice that for all $k \in \N$, the set $E_k$ is infinite so it has infinite measure. Since it holds for all $k \in \N$, then
    $$\lim_{k \rightarrow \infty} \mu(E_k) = \infty$$
    Moreover, since $\cap_{k=1}^{\infty}E_k = \varnothing$, then
    $$\mu\left(\bigcap_{k=1}^{\infty}E_k\right) = 0$$
    Therefore,
    $$\mu\left(\bigcap_{k=1}^{\infty}E_k\right) \neq \lim_{k \rightarrow \infty} \mu(E_k)$$\\
\end{solution}

\begin{exercise}
    Suppose $(X, \mathcal{S}, \mu)$ is a measure space and $C, D, E \in \mathcal{S}$ are such that
    $$\mu(C \cap D) < \infty, \mu(C \cap E) < \infty, \text{ and }\mu(D \cap E) < \infty.$$
    Find an prove a formula for $\mu(C \cup D \cup E)$ in terms of $\mu(C)$, $\mu(D)$, $\mu(E)$, $\mu(C \cap D)$, $\mu(C\cap E)$, $\mu(D \cap E)$, and $\mu(C \cap D \cap E)$. \\
\end{exercise}

\begin{solution}
    \\ First, by 2.61, since $\mu(D \cap E) < \infty$, then 
    \[\mu(D \cup E) = \mu(D) + \mu(E) - \mu(D \cap E)\tag*{(1)}\]
    Moreover, by 2.61, since $\mu(C\cap D \cap E) \leq \mu(D \cap E) < \infty$ (by monotonicity), then 
    \[\mu((C \cap D) \cup (C \cap E)) = \mu(C \cap D) + \mu(C \cap E) - \mu(C \cap D \cap E)\tag*{(2)}\]
    Lastly, combining (1) and (2) and applying 2.61 with the fact that $\mu((C \cap D) \cup (C \cap E)) \leq \mu(C \cap D) + \mu(C \cap E) < \infty$ gives us
    \begin{align*}
        \mu(C \cup D \cup E) &= \mu( C \cup (D \cup E)) \\
        &= \mu(C) + \mu(D\cup E) - \mu(C\cap(D \cup E)) \\
        &= \mu(C) + \mu(D\cup E) - \mu((C \cap D) \cup (C \cap E)) \\
        &= \mu(C) + \mu(D) + \mu(E) - \mu(D \cap E)\\
        & \quad - [\mu(C \cap D) + \mu(C \cap E) - \mu(C \cap D \cap E)] \\
        &= \mu(C) + \mu(D) + \mu(E) - \mu(D \cap E)\\
        & \quad - \mu(C \cap D) - \mu(C \cap E) + \mu(C \cap D \cap E) 
    \end{align*}
    which is a satsfying formula for what was asked.\\
\end{solution}

\begin{exercise}
    Suppose $X$ is a set and $\mathcal{S}$ is the $\sigma$-algebra of all subsets $E$ of $X$ such that $E$ is countable or $X \setminus E$ is countable. Give a complete description of the set of all measures on $(X, \mathcal{S})$. \\
\end{exercise}

\begin{solution}
    \\ In this proof, the goal will be to show that the measures on $(X, \mathcal{S})$ are precisely the functions of the form
    $$\mu(E) = \begin{cases}
        \sum_{x \in E}w(x) & \text{if } E \text{ is countable} \\
        \alpha + \sum_{x \in E}w(x) & \text{if } E \text{ is uncountable}
    \end{cases}$$
    where $\alpha \in [0, \infty]$ and $w: X \to [0, \infty]$ is a function. More precisely, the goal is to show that any measure on $(X, \mathcal{S})$ is of this form and that any function of this form is a measure on $(X, \mathcal{S})$. 
    \begin{itemize}
        \item Let $\mu : \mathcal{S} \to [0, \infty]$ be a measure on $(X, \mathcal{S})$. Define $w : X \to [0, \infty]$ by $w : x \mapsto \mu(\{x\})$. This function is well-defined since all singletons are in $\mathcal{S}$ since they are countable. It follows that for all countable sets $E = \{e_1, e_2, ... \}$,
        $$\mu(E) = \sum_{i=1}^{\infty}\mu(\{e_i\}) = \sum_{x \in E}w(x)$$
        Now, suppose that $\sum_{x \in E}w(x) = \infty$, then define $\alpha = \infty$ which makes the following equation true
        $$\mu(E) = \alpha + \sum_{i=1}^{\infty}\mu(\{e_i\}) = \sum_{x \in E}w(x)$$
        for all uncountable set $E \in \mathcal{S}$. In that case, we have shown that $\mu$ is of the desired form. \\ Suppose now that there is an uncountable set $E_0 \in \mathcal{S}$ such that $\sum_{x \in E_0}w(x) < \infty$, define 
        $$\alpha = \mu(E_0) - \sum_{x \in E_0}w(x) \geq 0$$
        $[$Notice that $\alpha$ is positive since $\mu(E_0)$ is an upper bound for the set $\{\sum_{i=1}^{n}w(x_i) : x_1, ..., x_n \in E_0\}$ and that $\sum_{x \in E_0}w(x) := \sup \{\sum_{i=1}^{n}w(x_i) : x_1, ..., x_n \in E_0\}$.$]$ \\
        Let's show that $\mu(E) = \alpha + \sum_{x \in E}w(x)$ for all uncountable sets $E \in \mathcal{S}$. If $\sum_{x \in E}w(x) = \infty$, then
        $$\mu(E) = \infty = \alpha + \sum_{x \in E}w(x)$$
        If $\sum_{x \in E}w(x) < \infty$, then
        \begin{align*}
            \mu(E) &= \mu(E_0) - \sum_{x \in E_0 \setminus E}w(x) + \sum_{x \in E \setminus E_0}w(x) \\
            &= \mu(E_0) - \sum_{x \in E_0 \setminus E}w(x) - \sum_{x \in E \cap E_0}w(x) + \sum_{x \in E \cap E_0}w(x) + \sum_{x \in E \setminus E_0}w(x) \\
            &= \mu(E_0) - \sum_{x \in E_0}w(x) + \sum_{x \in E}w(x) \\
            &= \alpha + \sum_{x \in E}w(x)
        \end{align*}
        which shows that in any case, $\mu$ is of the desired form.

        \item Consider now a function $\mu : \mathcal{S} \to [0, \infty]$ such that 
        $$\mu(E) = \begin{cases}
            \sum_{x \in E}w(x) & \text{if } E \text{ is countable} \\
            \alpha + \sum_{x \in E}w(x) & \text{if } E \text{ is uncountable}
        \end{cases}$$
        where $\alpha \in [0, \infty]$ and $w: X \to [0, \infty]$ is a function. Let's prove that $\mu$ is a measure on $(X, \mathcal{S})$. First,
        $$\mu(\varnothing) = \sum_{x \in \varnothing}w(x) = 0$$
        Now, let $\{E_i\}_i$ be a countable pairwise disjoint collection of sets in $\mathcal{S}$. Since they are disjoint, then there is at most one uncountable $E_i$. Otherwise, if $E_{i_1}$ and $E_{i_2}$ are both uncountable, then $E_{i_2}$ is contained in the complement of $E_{i_1}$ which implies that $E_{i_1}^c$ is also uncountable. A contradiction with the definition of $\mathcal{S}$. Thus, there is at most one uncountable $E_i$. If none of the $E_i$'s are uncountable, then
        \begin{align*}
            \mu\left(\bigcup_{i=1}^{\infty}E_i\right) &= \sum_{x \in \cup_{i=1}^{\infty}E_i}w(x) \\
            &= \sum_{i=1}^{\infty}\sum_{j=1}^{\infty}w(e_{i,j}) \\
            &= \sum_{i=1}^{\infty}\mu(E_i)
        \end{align*}
        If (wlog) $E_1$ is uncountable, then
        \begin{align*}
            \mu\left(\bigcup_{i=1}^{\infty}E_i\right) &= \alpha + \sum_{x \in \cup_{i=1}^{\infty}E_i}w(x) \\
            &= \alpha + \sum_{x \in E_1}w(x) + \sum_{x \in \cup_{i=2}^{\infty}E_i}w(x) \\
            &= \mu(E_1) + \sum_{i=2}^{\infty}\sum_{j=1}^{\infty}w(e_{i,j}) \\
            &= \mu(E_1) + \sum_{i=2}^{\infty}\mu(E_i) \\
            &= \sum_{i=1}^{\infty}\mu(E_i) 
        \end{align*}
        Therefore, $\mu$ is a measure on $(X, \mathcal{S})$.
    \end{itemize}
    Therefore, we have a complete description of the measures on the measurable space $(X, \mathcal{S})$.
\end{solution}

% CHAPTER 3
\chapter{Linear Maps}

\section{Vector Space of Linear Maps}

\begin{exercise}
    Suppose $b,c \in \R$. Define $T: \R^3 \to \R^2$ by
    $$T(x,y,z) = (2x-4y + 3z + b, 6x + cxyz).$$
    Show that $T$ is linear if and only if $b = c = 0$. \\
\end{exercise}

\begin{solution}
    \\ ($\implies$) Suppose that $T$ is linear, then we know from Proposition 3.10 that $T0 = 0$. Thus, it follows that
    $$T(0,0,0) = (b, 0) = (0,0)$$
    which implies that $b = 0$. To prove that $c = 0$, notice that by linearity of $T$, we have
    $$T(2,2,2) = 2T(1,1,1).$$
    If we plug-in the values into the definition of $T$, we get
    $$(4 - 8 + 6 + 0, 12 + 8c) = 2(2 - 4 + 3 + 0, 6 + c)$$
    which is equivalent to
    $$(2, 12 + 8c) = (2, 12 + 2c).$$
    It follows that $12 + 8c = 12 + 2c$ which can only be true when $c = 0$. Thus, $b = c = 0$. \\
    $( \ \Longleftarrow \ )$ Suppose now that $b = c = 0$, then $T(x,y,z)$ becomes
    $$T(x,y,z) = (2x-4y + 3z, 6x)$$
    for all $x,y,z \in \R$. Let's show that $T$ is linear. First, take $(x,y,z), (x',y',z') \in \R^3$ and notice that
    \begin{align*}
        T((x,y,z) + (x',y',z')) &= T(x+x', y+y', z+z') \\
        &= (2(x + x')-4(y+y') + 3(z+z'), 6(x+x')) \\
        &= (2x - 4y + 3z + 2x' - 4y' + 3z', 6x + 6x') \\
        &= (2x-4y + 3z, 6x) + (2x'-4y' + 3z', 6x') \\
        &= T(x,y,z) + T(x',y',z')
    \end{align*}
    Moreover, given any $\lambda \in \R$ and $(x,y,z) \in \R^3$, we have
    \begin{align*}
        T(\lambda(x,y,z)) &= T(\lambda x, \lambda y, \lambda z) \\
        &= (2(\lambda x)-4(\lambda y) + 3(\lambda z), 6(\lambda x)) \\
        &= (\lambda(2x - 4y + 3z), \lambda(6x)) \\
        &= \lambda(2x - 4y + 3z, 6x) \\
        &= \lambda T(x,y,z)
    \end{align*}
    Therefore, $T$ is linear.\\
\end{solution}

\begin{exercise}
    Suppose $b,c \in \R$. Define $T : \mathcal{P}(\R) \to \R^2$ by
    $$Tp = \left(3p(4) + 5p'(6) + bp(1)p(2), \int_{-1}^{2}x^3p(x)dx + c \sin p(0) \right).$$
    Show that $T$ is linear if and only if $b = c = 0$. \\
\end{exercise}

\begin{solution}
    \\ ($\implies$) Suppose that $T$ is linear, then if we let $p$ be the constant polynomial equal to $\pi/2$, we get that $T$ must satisfy
    $$T(2p) = 2Tp.$$
    If we rewrite this using the definition of $T$ and $p$, we obtain
    $$\left(3\pi + b\pi^2, \pi\int_{-1}^{2}x^3dx + c\sin(\pi)\right) = 2\left(3\frac{\pi}{2} + b\frac{\pi^2}{4}, \frac{\pi}{2}\int_{-1}^{2}x^3dx + c\sin\left(\frac{\pi}{2}\right)\right)$$
    which can be simplified to
    $$\left(3\pi + b\pi^2, \pi\int_{-1}^{2}x^3dx\right) = \left(3\pi + b\frac{\pi^2}{2}, \pi\int_{-1}^{2}x^3dx + c\right).$$
    This gives us the following system of equations:
    $$\begin{cases}
        3\pi + b\pi^2 = 3\pi + b\frac{\pi^2}{2} \\
        \pi\int_{-1}^{2}x^3dx = \pi\int_{-1}^{2}x^3dx + c
    \end{cases} 
    \implies
    \begin{cases}
        b = \frac{1}{2}b \\
        c = 0
    \end{cases}
    \implies b = c = 0.$$
    $( \ \Longleftarrow \ )$ Suppose that $b = c = 0$, then for all $p \in \mathcal{P}(\R)$, we have
    $$Tp = \left(3p(4) + 5p'(6), \int_{-1}^{2}x^3p(x)dx \right).$$
    Thus, for any $p_1, p_2 \in \mathcal{P}(\R)$, we get 
    \begin{align*}
        T(p_1 + p_2) &= \left(3(p_1 + p_2)(4) + 5(p_1 + p_2)'(6), \int_{-1}^{2}x^3(p_1 + p_2)(x)dx \right) \\
        &= \left(3(p_1(4) + p_2(4)) + 5(p_1'(6) + p_2'(6)), \int_{-1}^{2}x^3(p_1(x) + p_2(x))dx \right) \\
        &= \left(3p_1(4) + 3p_2(4) + 5p_1'(6) + 5p_2'(6), \int_{-1}^{2}x^3p_1(x)dx + \int_{-1}^{2}x^3p_2(x)dx \right) \\
        &= \left(3p_1(4) + 5p_1'(6), \int_{-1}^{2}x^3p_1(x)dx \right) + \left(3p_2(4) + 5p_2'(6), \int_{-1}^{2}x^3p_2(x)dx \right) \\
        &= Tp_1 + Tp_2.
    \end{align*}
    Similarly, for all $\lambda \in \R$ and $p \in \mathcal{P}(\R)$, we have
    \begin{align*}
        T(\lambda p) &= \left(3(\lambda p)(4) + 5(\lambda p)'(6), \int_{-1}^{2}x^3(\lambda p)(x)dx \right) \\
        &= \left(3\lambda p(4) + 5\lambda p'(6), \int_{-1}^{2}x^3 \lambda p(x)dx \right) \\
        &= \left(\lambda (3p(4) + 5p'(6)), \lambda\int_{-1}^{2}x^3 p(x)dx \right) \\
        &= \lambda \left(3p(4) + 5p'(6), \int_{-1}^{2}x^3 p(x)dx \right) \\
        &= \lambda Tp.
    \end{align*}
    Therefore, $T$ is linear. \\
\end{solution}

\begin{exercise}
    Suppose that $T \in \L(\F^n, \F^m)$. Show that there exist scalars $A_{j,k} \in \F$ for $j = 1, ..., m$ and $k= 1, ..., n$ such that 
    $$T(x_1, ..., x_n) = (A_{1,1}x_1 + \dots + A_{1,n}x_n, ..., A_{m,1}x_1 + \dots + A_{m,n}x_n)$$
    for every $(x_1, ..., x_n) \in \F^n$. \\
\end{exercise}

\begin{solution}
    \\ Denote by $e_1, ..., e_n$ the standard basis of $\F^n$ and by $f_1, ..., f_m$ the standard basis for $\F^m$, then for all $k \in \{1, ..., n\}$, there exist scalars $A_{1,k}, ..., A_{m,k} \in \F$ such that
    $$Te_k = A_{1,k}f_1 + \dots + A_{m,k}f_m.$$
    Therefore, by linearity, for all $(x_1, ..., x_n) \in \F^n$:
    \begin{align*}
        T(x_1, ..., x_n) &= x_1Te_1 + \dots + x_n Te_n \\
        &= x_1(A_{1,1}f_1 + \dots + A_{m,1}f_m) + \dots + x_n(A_{1,n}f_1 + \dots + A_{m,n}f_m) \\
        &= (A_{1,1}x_1 + \dots + A_{1,n}x_n)f_1 + \dots + (A_{m,1}x_1 + \dots + A_{m,n}x_n)f_m \\
        &= (A_{1,1}x_1 + \dots + A_{1,n}x_n, ..., A_{m,1}x_1 + \dots + A_{m,n}x_n).
    \end{align*}
    Therefore, any linear transformation has this form. \\
\end{solution}

\begin{exercise}
    Suppose $T \in \L(V, W)$ and $v_1, ..., v_m$ is a list of vectors in $V$ such that $Tv_1, ..., Tv_m$ is a linearly independent list in $W$. Prove that $v_1, ..., v_m$ is linearly independent. \\
\end{exercise}

\begin{solution}
    \\ To prove that $v_1, ..., v_m$ is linearly independent, take arbitrary scalars $\alpha_1, ..., \alpha_m \in \F$ such that
    $$\alpha_1 v_1 + ... + \alpha_m v_m = 0.$$
    By evaluating on both sides by $T$, we get by linearity of $T$ the following equation:
    $$\alpha_1 Tv_1 + ... + \alpha_m Tv_m = 0.$$
    But since the list $Tv_1, ..., Tv_m$ is a linearly independent in $W$, then 
    $$\alpha_1 = ... = \alpha_m = 0$$
    which proves that $v_1, ..., v_m$ is linearly independent. \\
\end{solution}

\begin{exercise}
    Prove that $\L(V, W)$ is a vector space, as was asserted in 3.6. \\
\end{exercise}

\begin{solution}
    \\ We already proved in Section 1B Exercise 7 that for any nonempty set $S$ and vector space $U$, the set $U^S$ equipped with the usual addition and scalar multiplication is a vector space. Hence, if we let $S = V$ and $U = W$, we already know that the set of functions from $V$ tp $W$ is a vector space. Since $\L(V, W) \subset W^V$, then it suffices to show that $\L(V, W)$ is a subspace. \\
    First, notice that $\L(V, W)$ is non-empty since it contains the additive identity map: the constant zero map is linear. Given two linear maps $T_1, T_2 \in \L(V, W)$, we can show that $T_1 + T_2 \in \L(V, W)$ by proving that it is a linear map from $V$ to $W$. Hence, take arbitrary $x,y \in V$ and $\lambda \in \F$ to get:
    \begin{align*}
        (T_1 + T_2)(x + y) &= T_1(x + y) + T_2(x + y) \\
        &= T_1(x) + T_1(y) + T_2(x) + T_2(y) \\
        &= (T_1 + T_2)(x) + (T_1 + T_2)(y),
    \end{align*}
    and
    \begin{align*}
        (T_1 + T_2)(\lambda x) &= T_1(\lambda x) + T_2(\lambda x) \\
        &= \lambda T_1(x) + \lambda T_2(x) \\
        &= \lambda (T_1(x) + T_2(x)) \\
        &= \lambda (T_1 + T_2)(x).
    \end{align*}
    Thus, $\L(V, W)$ is closed under addition. Similarly, given a linear map $T \in \L(V, W)$ and $\alpha \in \F$, we get that $\alpha T \in \L(V, W)$ because for all $x,y \in V$ and $\lambda \in \F$, we have the following:
    \begin{align*}
        (\alpha T)(x + y) &= \alpha T(x + y) \\
        &= \alpha (T(x) + T(y)) \\
        &= \alpha T(x) + \alpha T(y) \\
        &= (\alpha T)(x) + (\alpha T)(y)
    \end{align*}
    and
    \begin{align*}
        (\alpha T)(\lambda x) &= \alpha T(\lambda x) \\
        &= \alpha \lambda T(x) \\
        &= \lambda \alpha T(x) \\
        &= \lambda (\alpha T)(x).
    \end{align*}
    Therefore, $\L(V, W)$ is a vector space since it is a subspace of $W^V$. \\
\end{solution}

\begin{exercise}
    Prove that the multiplication of linear maps has the associative, identity and distributive properties asserted in 3.8. \\
\end{exercise}

\begin{solution}
    \vspace{-0.25cm}
    \begin{itemize}
        \item (Associativity) Let $V_1, V_2, V_3, V_4$ be vector spaces and $T_1 : V_1 \to V_2$, $T_2 : V_2 \to V_3$ and $T_3 : V_3 \to V_4$ be linear maps. Associativity follows from the fact that for all $x \in V_1$:
        \begin{align*}
            ((T_1 T_2)T_3)(x) &= (T_1 T_2)(T_3(x)) \\
            &= T_1(T_2(T_3(x))) \\
            &= T_1(T_2T_3(x)) \\
            &= (T_1(T_2T_3))(x).
        \end{align*}
        Since it holds for all $x \in X$, then $(T_1 T_2)T_3 = T_1(T_2T_3)$.
        \item (Identity) Let $V$ and $W$ be vector space. Consider the identity map $I_V : V \to W$ and let's show that it is indeed linear. For all $x,y \in V$:
        $$I_V(x + y) = x + y = I_V(x) + I_V(y)$$
        and for any $\lambda \in \F$ and $x \in V$:
        $$I_V(\lambda x) = \lambda x = \lambda I_V(x).$$
        Therefore, $I_V$ is linear. To prove that it is the multiplicative identity in $\L(V,W)$, let $T: V \to W$ be a linear map and $x \in V$, then
        $$(I_V T)(x) = I_V(Tx) = Tx$$
        and
        $$(TI_V)(x) = T(I_Vx) = Tx$$
        so $I_V T = T I_V = T$ for all linear maps $T \in \L(V,W)$.
        \item (Distributivity 1) Let $U,V,W$ be vector spaces, $S_1, S_2 \in \L(V,W)$ and $T \in \L(U,V)$, then for all $x \in V$, we have
        \begin{align*}
            [(S_1 + S_2)T](x) &= (S_1 + S_2)(Tx) \\
            &= S_1(Tx) + S_2(Tx) \\
            &= (S_1 T)(x) + (S_2 T)(x) \\
            &= [S_1 T + S_2 T](x).
        \end{align*}
        Since it holds for all $x \in U$, then $(S_1 + S_2) T = S_1 T  + S_2 T$.
        \item (Distributivity 2) Let $U,V,W$ be vector spaces, $S \in \L(V,W)$ and $T_1, T_2 \in \L(U,V)$, then for all $x \in V$ and by linearity of $S$, we have
        \begin{align*}
            [S(T_1 + T_2)](x) &= S((T_1 + T_2)(x)) \\
            &= S(T_1(x) + T_2(x)) \\
            &= S(T_1(x)) + S(T_2(x)) \\
            &= (S T_1)(x) + (S T_2)(x) \\
            &= [ST_1 + ST_2](x).
        \end{align*}
        Since it holds for all $x \in U$, then $S(T_1 + T_2) = ST_1 + ST_2$.\\
    \end{itemize}
\end{solution}

\begin{exercise}
    Show that every linear map from a one-dimensional vector space to itself is multiplicative by some scalar. More precisely, prove that if $\dim V = 1$ and $T \in \L(V)$, then there exists $\lambda \in \F$ such that $Tv = \lambda v$ for all $v \in V$. \\
\end{exercise}

\begin{solution}
    \\ Since $\dim V = 1$, then there is a $v_0 \in V$ such that $V = \text{span}(v_0)$. We have $Tv_0 \in \text{span}(v_0)$ so there is a $\lambda \in \F$ satisfying $Tv_0 = \lambda v_0$. Take $v \in V$, since $v \in \text{span}(v_0)$, then there is an $\alpha \in \F$ such that $v = \alpha v_0$. Thus:
    $$Tv = T\alpha v_0 = \alpha Tv_0 = \alpha \lambda v_0 = \lambda v.$$
\end{solution}

\begin{exercise}
    Give an example of a function $\varphi : \R^2 \to \R$ such that
    $$\varphi(av) = a\varphi(v)$$
    for all $a \in \R$ and all $v \in \R^2$ but $\varphi$ is not linear.\\
\end{exercise}

\begin{solution}
    \\ Consider the function $\varphi : \R^2 \to \R$ defined by $\varphi(x,y) = \sqrt[3]{(x + y)^3}$, then for all $x,y \in \R$ and $a \in \R$, we have
    \begin{align*}
        \varphi(ax, ay) &= \sqrt[3]{(ax + ay)^3} \\
        &= \sqrt[3]{a^3(x + y)^3} \\
        &= a\sqrt[3]{(x + y)^3} \\
        &= a \varphi(x,y).
    \end{align*}
    However, notice that $\varphi(1, 0) = \varphi(0,1) = 1$ but $\varphi(1, 1) = \sqrt[3]{2}$ so $\varphi(1, 1) \neq \varphi(1, 0) + \varphi(0,1)$ so $\varphi$ is not linear. \\
\end{solution}

\begin{exercise}
    Give an example of a function $\varphi : \C \to \C$ such that
    $$\varphi(w + z) = \varphi(w) + \varphi(z)$$
    for all $w,z \in \C$ but $\varphi$ is not linear. (Here, $\C$ is thought of as a complex vector space.)\\
\end{exercise}

\begin{solution}
    \\ Consider the function $\varphi : \C \to \C$ defined by $\varphi(z) = \Re(z)$, then for all $w,z \in \C$, we know that
    $$\Re(w + z) = \Re(w) + \Re(z).$$
    However, $\Re(i) = 0$ and $i \Re(1) = i$ so $\Re(i \cdot 1) \neq i\Re(1)$. Therefore, $\varphi$ is not linear. \\
\end{solution}

\begin{exercise}
    Prove or give a counterexample: If $q \in \P(\R)$ and $T: \P(\R) \to \P(\R)$ is defined by $Tp = q \circ p$, then $T$ is a linear map. \\
\end{exercise}

\begin{solution}
    \\ Consider the following counterexample: Take $q = x^2$ and define the map $T : \P(\R) \to \P(\R)$ by 
    $$Tp = q \circ p = p^2$$
    for all $p \in \P(\R)$. Notice that $T(x + 1) = x^2 + 2x + 1$ but $T(x) + T(1) = x^2 + 1$. Thus, $T(x + 1) \neq T(x) + T(1)$ so $T$ is not a linear map. \\
\end{solution}

\begin{exercise}
    Suppose $V$ is a finite-dimensional vector space and $T \in \L(V)$. Prove that $T$ is a scalar multiple of the identity if and only if $ST = TS$ for all $S \in \L(V)$. \\
\end{exercise}

\begin{solution}
    \\ First, let $T$ be a scalar multiple of the identity, then there is a $\lambda \in \F$ such that $Tv = \lambda v$ for all $v \in V$. Let $S$ be an arbitrary linear map from $V$ to $V$, then for all $v \in V$:
    $$(ST)v = S(Tv) = S(\lambda v) = \lambda Sv = T(Sv) = (TS)v.$$
    Since it holds for all $v \in V$, then $ST = TS$. \\
    To prove that the converse holds, fix a basis $v_1, ..., v_n$ of $V$ and for all $i$ between 1 and $n$, define $S_i$ as the linear map satisfying $S_iv_1 = v_i$ and $S_i v_k = 0$ for all $k \neq 1$ (such a linear map is well-defined and unique by Lemma 3.4). Let $T \in \L(V)$, then for all $i$ between 1 and $n$, there exist scalars $A_{i,1}, ..., A_{i,n} \in \F$ such that
    $$Tv_i = A_{i,1}v_1 + ... + A_{i,n}v_n. $$
    Suppose that $T$ satisfies $ST = TS$ for all $S \in \L(V)$, then in particular, for all fixed $i$ between 1 and $n$, we have $(S_i T)v_1 = (T S_i)v_1$. Using the definitions and properties of $S_i$ and $T$, we get that
    \begin{align*}
        (S_i T)v_1 = (T S_i)v_1 &\implies S_i(A_{1,1}v_1 + ... + A_{1,n}v_n) = T v_i \\
        &\implies A_{1,1}v_i = A_{i,1}v_1 + ... + A_{i,n}v_n
    \end{align*}
    and by uniqueness of representations of vectors in $V$ as linear combinations of the basis, we get that $A_{i,j} = 0$ for all $i \neq j$ and $A_{1,1} = A_{i,i}$. Thus, if we let $\lambda = A_{1,1}$, we obtain that for all $i$,
    $$Tv_i = A_{i,1}v_1 + ... + A_{i,n}v_n = \lambda v_i.$$
    Therefore, it follows that $T$ is equal to $\lambda$ times the identity map, i.e., a scalar multiple of the identity. \\
\end{solution}

\begin{exercise}
    Suppose $U$ is a subspace of $V$ with $U \neq V$. Suppose $S \in \L(U,W)$ and $S \neq 0$ (which means that $Su \neq 0$ for some $u \in U$). Define $T : V \to W$ by
    $$Tv = \begin{cases} Sv & \text{if } v \in U, \\ 0 & \text{if } v \in V \text{ and } v \notin U. \end{cases}$$
    Prove that $T$ is not a linear map on $V$. \\
\end{exercise}

\begin{solution}
    \\ Since $U \neq V$, then there is a $v_0 \in V \setminus U$. The fact that $S$ is not the zero transformation implies that there is a vector $u \in U$ such that $Su \neq 0$. Moreover, since $U$ is a subspace and $u \in U$, then $v_0 + u \in U$ implies that $v_0 \in U$. A contradiction that shows that $u + v_0 \in V \setminus U$. Thus, by definition of $T$, we have $Tv_0 = 0$ and $T(v_0 + u) = 0$. If $T$ is linear, then we would get
    $$0 = T(v_0 + u) = Tv_0 + Tu = Su.$$
    But this is a contradiction since we defined $u$ such that $Su \neq 0$. Thus, no such linear transformation $T$ exists. \\
\end{solution}

\begin{exercise}
    Suppose $V$ is finite-dimensional. Prove that every linear map on a subspace of $V$ can be extended to a linear map on $V$. In other words, show that if $U$ is a subspace of $V$ and $S \in \L(U,W)$, then there exists $T \in \L(V,W)$ such that $Tu = Su$ for all $u \in U$. \\
\end{exercise}

\begin{solution}
    \\ Let $u_1, ..., u_n$ be a basis of $U$, then it can be extended to a basis $u_1, ..., u_m$ of $V$ where $m \geq n$. Define $T$ on this basis as follows: $Tu_i = Su_i$ if $i \leq n$ and $Tu_i = 0$ otherwise. By Lemma 3.4, $T$ is a well-defined linear map from $V$ to $W$. Let's now prove that $T$ extends $S$. Let $u \in U$, then there exist scalars $\alpha_1, ..., \alpha_n \in \F$ such that 
    $$u = \alpha_1 u_1 + ... + \alpha_n u_n.$$
    Applying $T$ on both sides an using the linearity of $T$, we get
    $$Tu = \alpha_1 Tu_1 + ... + \alpha_n Tu_n.$$
    By construction of $T$, we know that $Tu_i = Su_i$ for all $i$ between 1 and $n$:
    $$Tu = \alpha_1 Su_1 + ... + \alpha_n Su_n.$$
    Finally, by linearity of $S$:
    $$Tu = S(\alpha_1 u_1 + ... + \alpha_n u_n) = Su.$$
    It follows that $T$ is linear map that extends $S$ on $V$. \\
\end{solution}

\begin{exercise}
    Suppose $V$ is finite-dimensional with $\dim V > 0$, and suppose $W$ is infinite-dimensional. Prove that $\L(V,W)$ is infinite-dimensional. \\
\end{exercise}

\begin{solution}
    \\ Let $v_1, ..., v_n$ be a basis of $V$. From Section 2A Exercise 17, we know that there exists a sequence $w_1, w_2, ...$ in $W$ such that the list $w_1, ..., w_m$ is linearly independent for all $m$. For all $k$, define the map $T_k : V \to W$ to be the unique linear map such that $T_k v_1 = w_k$ and $T_k v_i = 0$ for all $i$ between 2 and $n$. Let's show that for all $m$, the list $T_1, ..., T_m$ is linearly independent in $\L(V,W)$. Let $\alpha_1, ..., \alpha_n \in \F$ be scalars such that $$\alpha_1 T_1 + ... + \alpha_m T_m = 0,$$
    then in particular, if we plug-in $v_1$, we get
    $$\alpha_1 w_1 + ... + \alpha_m w_m = 0.$$
    By our assumption on the sequence $w_1, w_2, ...$, we know that it implies that $\alpha_1 = ... = \alpha_m = 0$. Thus, the list $T_1, ..., T_m$ is linearly independent. Since it holds for all $m$, then by Section 2A Exercise 17, $\L(V,W)$ is infinite-dimensional. \\
\end{solution}

\begin{exercise}
    Suppose $v_1, ..., v_m$ is a linearly dependent list of vectors in $V$. Suppose also that $W \neq \{0\}$. Prove that there exist $w_1, ..., w_m \in W$ such that no $T \in \L(V,W)$ satisfies $Tv_k = w_k$ for each $k = 1, ..., m$. \\
\end{exercise}

\begin{solution}
    \\ If the list has length 1, then $v_1$ must be zero vector so it suffices to take $w_1 \in W\setminus \{0\}$. Hence, every linear map $T$ would map $v_1$ to zero which is different than $w_1$. \\
    Assume that $m > 1$, since the list $v_1, ..., v_m$ is linearly independent, then without loss of generality, we can assume that $v_m$ can be written as a linear combination of the other vectors. Thus, let $w_1 = ... = w_{m-1} = 0$ and $w_m \in W \setminus \{0\}$ (which must exists since $W \neq \{0\}$). Let $T$ be a linear map and suppose that $T v_k = w_k$ for each $k = 1, ..., m$. However, since there exist scalars $\alpha_1, ..., \alpha_{m-1}$ such that
    $$v_m = \alpha_1 v_1 + ... + \alpha_{m-1}v_{m-1},$$
    then by applying $T$ on both sides, we get
    $$Tv_m = \alpha_1 Tv_1 + ... + \alpha_{m-1}Tv_{m-1} = 0 \neq w_m.$$
    Therefore, no linear map $T$ satisfies $T v_k = w_k$ for each $k = 1, ..., m$.\\
\end{solution}

\begin{exercise}
    Suppose $V$ is finite-dimensional with $\dim V > 1$. Prove that there exist $S,T \in \L(V)$ such that $ST \neq TS$. \\
\end{exercise}

\begin{solution}
    \\ We know from Exercise 11 that the linear maps that commute with every other linear map are precisely the scalar multiples of the identity map. Hence, it suffices to show that there exists a linear map that is not a scalar multiple of the identity. Let $v_1, ..., v_n$ be a basis of $V$ (so $n \geq 2$) and define $T : V \to V$ to be the unique linear map such that $Tu_1 = u_1$ and $Tu_i = 0$ for $i$ between 2 and $n$. Such a transformation exists by Lemma 3.4. If $T$ was a scalar multiple of the identity, then $Tu_1 = u_1$ would imply that $T$ is the identity since $u_1$. However, $Tu_2 = 0$ even if $u_2 \neq 0$. Thus, by contradiction, $T$ is not a scalar multiple of the identity. Therefore, there must be a linear map $S$ such that $ST \neq TS$.\\
\end{solution}

\begin{exercise}
    Suppose $V$ is finite-dimensional. Show that the only two-sided ideals of $\L(V)$ are $\{0\}$ and $\L(V)$. \\
\end{exercise}

\begin{solution}
    \\ Let $\mathcal{E}$ be a two-sided ideal of $\L(V)$, if $\mathcal{E} = \{0\}$, then we are done. Assume that $\mathcal{E} \neq \{0\}$, then there must be a non-zero linear map $T$ in $\mathcal{E}$ and scalars $\{A_{i,j}\} \subset \F$ such that 
    $$T v_j = A_{1,j}v_1 + ... + A_{n,j}v_n$$
    for all $j$ between 1 and $n$. Since $T$ is non-zero, then it follows that there exist $i_0$ and $j_0$ between 1 and $n$ such that $A_{i_0, j_0} \neq 0$. Moreover, for all $i$ and $j$ between 1 and $n$, define the linear map $S_{i,j} \in \L(V)$ by
    $$S_{i,j} v_k = \begin{cases} v_i & k = j, \\ 0 & k \neq j. \end{cases}$$
    Consider the map $\frac{1}{A_{i_0,j_0}}S_{i_0,i_0}T S_{j_0, j_0}$, since $\mathcal{E}$ is a two-sided ideal, then this map belongs to $\mathcal{E}$. Let $k$ be an integer between 1 and $n$, if $k \neq j_0$, then
    $$\frac{1}{A_{i_0,j_0}}S_{i_0,i_0}T S_{j_0, j_0} v_k = \frac{1}{A_{i_0,j_0}}S_{i_0,i_0}T(0) = 0,$$ 
    and if $k = j_0$, then 
    \begin{align*}
        \frac{1}{A_{i_0,j_0}}S_{i_0,i_0}T S_{j_0, j_0} v_k &= \frac{1}{A_{i_0,j_0}}S_{i_0,i_0}Tv_{j_0} \\
        &= \frac{1}{A_{i_0,j_0}}S_{i_0,i_0}(A_{1,j_0}v_1 + ... + A_{n,j_0}v_n)\\
        &= \frac{1}{A_{i_0,j_0}}A_{i_0, j_0}v_{i_0}\\
        &= v_{i_0}.
    \end{align*}
    Thus, by definition of the maps $S_{i,j}$'s and by uniqueness part of Lemma 3.4, we get that
    $$\frac{1}{A_{i_0,j_0}}S_{i_0,i_0}T S_{j_0, j_0} = S_{i_0, j_0}.$$
    Hence, the map $S_{i_0, j_0}$ is in $\mathcal{E}$. From this, we get that for all $i$ and $j$ between 1 and $n$, the map $S_{i,i_0}S_{i_0, j_0} S_{j_0, j}$ is in $\mathcal{E}$ as well. But notice that for all $k$ between 1 and $n$, if $k \neq j$, then
    $$S_{i,i_0}S_{i_0, j_0} S_{j_0, j} v_k = 0,$$
    and $k = j$, then 
    $$S_{i,i_0}S_{i_0, j_0} S_{j_0, j} v_k = S_{i,i_0}S_{i_0, j_0} v_{j_0} = S_{i,i_0}v_{i_0} = v_i.$$
    Thus, again, by the uniqueness part of Lemma 3.4 and since it holds for all $i,j$, then $S_{i,j} \in \mathcal{E}$ for all $i,j$. We are now ready to show that $\mathcal{E} = \L(V)$. Since $\mathcal{E}$ is a subspace of $\L(V)$, then it suffices to prove that $\L(V) \subset \mathcal{E}$. Let $S \in \L(V)$, then there exist scalars $\{B_{i,j}\}_{i,j} \subset \F$ such that
    $$S v_j = B_{1,j}v_1 + ... + B_{n,j}v_n,$$
    for all $j$. Consider the map $\tilde{S}$ defined by
    $$\tilde{S} = \sum_{i=1}^{n}\sum_{k=1}^{n}B_{i,k}S_{i,k}.$$
    Since $\mathcal{E}$ is a subspace that contains all the $S_{i,j}$'s, then $\tilde{S} \in \mathcal{E}$. Moreover, notice that for all $j$,
    $$\tilde{S}v_j = \sum_{i=1}^{n}\sum_{k=1}^{n}B_{i,k}S_{i,k} v_j = \sum_{i=1}^{n}B_{i,j}v_i = Sv_j.$$
    Since it holds for all $j$, then by Lemma 3.4, we have that $S = \tilde{S} \in \mathcal{E}$. Since it holds for all $S \in \L(V)$, then $\L(V) = \mathcal{E}$. Therefore, the only two-sided ideals of $\L(V)$ are $\{0\}$ and $\L(V)$.  \\
\end{solution}

\section{Null Spaces and Ranges}

\begin{exercise}
    \td \\
\end{exercise}

\begin{solution}
    \\ \td \\
\end{solution}
\section{Matrices}

\begin{exercise}
    Suppose $T \in \L(V,W)$. Show that with respect to each choice of bases of $V$ and $W$, the matrix of $T$ has at least $\dim \range T$ nonzero entries. \\
\end{exercise}

\begin{solution}
    \\ \td \\
\end{solution}

\begin{exercise}
    Suppose $V$ and $W$ are finite-dimensional and $T \in \L(V,W)$. Prove that $\dim \range T = 1$ if and only if there exist a basis of $V$ and a basis of $W$ such that with respect to these bases, all entries of $\M(T)$ equal 1. \\
\end{exercise}

\begin{solution}
    \\ \td \\
\end{solution}

\begin{exercise}
    Suppose $v_1, ..., v_n$ is a basis of $V$ and $w_1, ..., w_m$ is a basis of $W$.
    \begin{enumerate}[label=(\alph*)]
        \item Show that if $S,T \in \L(V,W)$, then $\M(S+T) = \M(S) + \M(T)$.
        \item Show that if $\lambda \in \F$ and $T \in \L(V,W)$, then $\M(\lambda T) = \lambda \M(T)$. \\
    \end{enumerate}
\end{exercise}

\begin{solution}
    \\ \td \\
\end{solution}

\begin{exercise}
    Suppose that $D \in \L(\P_3(\R), \P_2(\R))$ is the differentiation map defined by $Dp = p'$. Find a basis of $\P_3(\R)$ and a basis of $\P_2(\R)$ such that the matrix of $D$ with respect to these bases is
    $$\begin{pmatrix}
        1 & 0 & 0 & 0 \\ 0 & 1 & 0 & 0 \\ 0 & 0 & 1 & 0 
    \end{pmatrix}$$
\end{exercise}

\begin{solution}
    \\ Consider the the basis $p_0, p_1, p_2, p_3$ of $\P_3(\R)$ and $q_0, q_1, q_2$ of $\P_2(\R)$ defined by 
    $$p_0(x) = x, \qquad p_1(x) = \frac{1}{2}x^2, \qquad p_2(x) = \frac{1}{6}x^3, \qquad p_3(x) = 1$$
    and
    $$q_0(x) = 1, \qquad q_1(x) = x, \qquad q_2(x) = \frac{1}{2}x^2.$$
    Since each of these lists have the right amount of elements and the polynomials have distinct degrees, then the lists are indeed bases. Now, let's compute the matrix of $D$ with respect to these bases. For the first column, since $Dp_0 = 1 = q_0 = 1\cdot q_0 + 0 \cdot q_1 + 0\cdot q_2$, then the first column is
    $$\begin{pmatrix}
        1 & * & * & * \\ 0 & * & * & * \\ 0 & * & * & * 
    \end{pmatrix}.$$
    For the second column, we have $Dp_1 = x = q_1 = 0\cdot q_0 + 1 \cdot q_1 + 0\cdot q_2$. This gives us
    $$\begin{pmatrix}
        1 & 0 & * & * \\ 0 & 1 & * & * \\ 0 & 0 & * & * 
    \end{pmatrix}.$$
    For the third column, we have $Dp_2 = \frac{1}{2}x^2 = q_2 = 0\cdot q_0 + 0 \cdot q_1 + 1\cdot q_2$. This gives us
    $$\begin{pmatrix}
        1 & 0 & 0 & * \\ 0 & 1 & 0 & * \\ 0 & 0 & 1 & * 
    \end{pmatrix}.$$
    Finally, since $Dp_3 = 0 =  0\cdot q_0 + 0 \cdot q_1 + 0\cdot q_2$, then our matrix is 
    $$\begin{pmatrix}
        1 & 0 & 0 & 0 \\ 0 & 1 & 0 & 0 \\ 0 & 0 & 1 & 0 
    \end{pmatrix}.$$\\
\end{solution}

\begin{exercise}
    Suppose $V$ and $W$ are finite-dimensional and $T \in \L(V,W)$. Prove that there exist a basis of $V$ and a basis of $W$ such that with respect to these bases, all entries of $\M(T)$ are 0 except that the entries in row $k$, column $k$, equal 1 if $1 \leq k \leq \dim \range T$. \\
\end{exercise}

\begin{solution}
    \\ \td \\
\end{solution}

\begin{exercise}
    Suppose $v_1, ..., v_m$ is basis of $V$ and $W$ is finite-dimensional. Suppose $T \in \L(V,W)$. Prove that there exist a basis $w_1, ..., w_n$ of $W$ such that all entries in the first column of $\M(T)$ [with respect to the bases $v_1, ..., v_m$ and $w_1, ..., w_n$] are 0 except for possibly a 1 in the first row, first column. \\
\end{exercise}

\begin{solution}
    \\ \td \\
\end{solution}
\section{Invertibility and Isomorphisms}

\begin{exercise}
    \td \\
\end{exercise}

\begin{solution}
    \\ \td \\
\end{solution}
\section{Products and Quotients of Vector Spaces}

\begin{exercise}
    \td \\
\end{exercise}

\begin{solution}
    \\ \td \\
\end{solution}
\section{Duality}

\begin{exercise}
    \td \\
\end{exercise}

\begin{solution}
    \\ \td \\
\end{solution}

\end{document}