\documentclass{article}

%% Language and font encodings
\usepackage[english]{babel}
\usepackage[utf8x]{inputenc}
\usepackage[T1]{fontenc}
\usepackage{amsfonts}
\usepackage[margin=41mm]{geometry}
\usepackage{mathtools}

%% Sets page size and margins
%\usepackage[a4paper,top=3cm,bottom=2cm,left=3cm,right=3cm,marginparwidth=1.75cm]{geometry}

%% Useful packages
\usepackage[colorlinks=true, allcolors=black]{hyperref}
\usepackage{amsmath}
\usepackage{xcolor}
\usepackage{amssymb}
\usepackage{amsthm}
\usepackage{stmaryrd}
\usepackage{graphicx}
\usepackage{enumitem}
\usepackage{tikz, pgfplots}
\usetikzlibrary{positioning}
\usepackage{tikz-cd}

%Usual Sets
\newcommand{\C}{\mathbb{C}}
\newcommand{\R}{\mathbb{R}}
\newcommand{\Q}{\mathbb{Q}}
\newcommand{\Z}{\mathbb{Z}}
\newcommand{\N}{\mathbb{N}}
\newcommand{\F}{\mathbb{F}}
\newcommand{\E}{\mathbb{E}}
\newcommand{\K}{\mathbb{K}}
\newcommand{\Zn}[1]{\mathbb{Z}/ #1 \mathbb{Z}}

%Categories
\newcommand{\CatC}{\textbf{C}}
\newcommand{\CatD}{\textbf{D}}
\newcommand{\CatTop}{\textbf{Top}}
\newcommand{\CatSets}{\textbf{Sets}}
\newcommand{\CatGps}{\textbf{Gps}}
\newcommand{\CatAbGps}{\textbf{AbGps}}
\newcommand{\CatMod}[1]{_\textbf{#1}\textbf{Mod}}
\newcommand{\CatVect}[1]{_\textbf{#1}\textbf{VSp}}

%Special Sets
\newcommand{\Iint}[2]{\llbracket #1 , #2 \rrbracket}

%Math Operators
\let\Re\relax
\let\Im\relax
\DeclareMathOperator{\Im}{Im}
\DeclareMathOperator{\Re}{Re}
\DeclareMathOperator{\Null}{null}
\DeclareMathOperator{\range}{range}
\DeclareMathOperator{\card}{card}
\DeclareMathOperator{\Aut}{Aut}
\DeclareMathOperator{\Hom}{Hom}
\DeclareMathOperator{\Mor}{Mor}
\DeclareMathOperator{\Gal}{Gal}
\DeclareMathOperator{\Char}{char}
\DeclareMathOperator{\rad}{rad}

%Others
\newcommand{\td}{\textcolor{red}{\textbf{TODO}}}
\newcommand{\isomorphic}{\cong}

\title{Algebraic Geometry : Homework 5}
\author{Samy Lahlou}
\date{}

\begin{document}

\maketitle

\noindent \textbf{Exercise 3:} For a homogeneous polynomial $f \in k[x_1, ..., x_{n+1}]$ of degree $d \geq 1$, let $U_f \subset \mathbb{P}^n(k)$ be the open subset
$$U_f := \{p \in \mathbb{P}^n(k) \ | \ f(p) \neq 0\} = \mathbb{P}^n(k) \setminus V_p(f).$$
In the last lecture, we have seen that $U_f$ is an affine variety when $f = a_1x_1 + \dots + a_{n+1}x_{n+1}$ is a linear polynomial. Use it to show that $U_f$ is affine for any $f$.\\
\noindent \textit{Hint: use the Veronese embedding.} \\

\noindent \textbf{Solution :} If we let $m_1$, $m_2$, ..., $m_N$ denote the set of monomials of degree $d$ (here, $N = \binom{n + d}{n}$), then we can write $f = \sum_i a_im_i$ where $a_i \in k$. From that, we can define $g = \sum_{i}a_ix_i \in k[x_1, ..., x_N]$ a linear map. If we consider the Veronomese map $\nu_d : \mathbb{P}^n(k) \to \mathbb{P}^{N - 1}(k)$, we have that $\nu_d^*(g) = f$ by construction of $g$. Hence, we can write $U_f = \nu_d^{-1}(U_g) = V \cap U_g$ where $V$ is the image of $\nu_d$. Since $U_g$ is affine by the content of the last lecture, we have that $V \cap U_g$ must be affine as well ($V$ is a projective variety). Therefore, $U_f$ is affine.

\end{document}