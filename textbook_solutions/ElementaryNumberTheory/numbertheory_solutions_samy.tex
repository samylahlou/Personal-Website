\documentclass[12pt, oneside]{book}

%% Language and font encodings
\usepackage[english]{babel}
\usepackage[utf8x]{inputenc}
\usepackage[T1]{fontenc}
\usepackage{amsfonts}

%% Sets page size and margins
\usepackage[a4paper,top=3cm,bottom=2cm,left=3cm,right=3cm,marginparwidth=1.75cm]{geometry}

%% Useful packages
\usepackage{amsmath}
\usepackage{amssymb}
\usepackage{amsthm}
\usepackage{stmaryrd}
\usepackage{graphicx}
\usepackage[colorinlistoftodos]{todonotes}
\usepackage[colorlinks=true, allcolors=black]{hyperref}
\usepackage{enumitem}
\setlist[enumerate,1]{label={(\alph*)}}

%Usual Sets
\newcommand{\Iint}[2]{\llbracket #1 , #2 \rrbracket}

%Useful Commands
\DeclareMathOperator{\lcm}{lcm}
\newcommand{\td}{\textcolor{red}{\textbf{TODO}}}

% Définir un nouveau compteur d'exercice
\newcounter{exercise}[section] % Compteur qui est remis à zéro à chaque nouvelle section
%\renewcommand{\theexercise}{\arabic{exercise}}

% Définir un environnement pour l'exercice
\newenvironment{exercise}{
    \refstepcounter{exercise} % Incrémenter le compteur
    \noindent\textbf{\arabic{exercise}.}
}{} % Fin de l'environnement

% Définir un environnement pour les solutions
\newenvironment{solution}{
    \noindent\textbf{Solution} 
}{}


\title{Solutions to Elementary Number Theory \\ (Second Edition) by David M. Burton}
\author{Samy Lahlou}

\begin{document}
\maketitle

\chapter*{Preface}

The goal of this document is to share my personal solutions to the exercises in the Second Edition of Elementary Number Theory by David M. Burton during my reading. To make my solutions clear, for each exercise, I will assume nothing more than the content of the book and the results proved in the preceding exercises. Moreover, it should be noted that a lot of the exercises can be done very easily using a calculator or using a computer program. It is for this reason that I chose to do every exercise with \textbf{no calculator and without writing any computer program}. I took this decision because I believe that I will learn more in this way.

As a disclaimer, the solutions are not unique and there will probably be better or more optimized solutions than mine. Feel free to correct me or ask me anything about the content of this document at the following address: 

samy.lahloukamal@mail.mcgill.ca

\tableofcontents

\iffalse
% CHAPTER 1
\chapter{Some Preliminary Considerations}

\section{Mathematical Induction}

\begin{exercise}
    Establish the formulas below by mathematical induction:
    \begin{enumerate}
        \item $\displaystyle 1 + 2 + 3 + \dots + n = \frac{n(n+1)}{2}$ for all $n \geq 1$;
        \item $1  + 3 + 5 + \dots + (2n-1) = n^2$ for all $n \geq 1$;
        \item $\displaystyle 1\cdot 2 + 2\cdot 3 + 3\cdot 4 + \dots + n(n+1) = \frac{n(n+1)(n+2)}{3}$ for all $n \geq 1$;
        \item $\displaystyle 1^2 + 3^2 + 5^2 + \dots + (2n-1)^2 = \frac{n(4n^2 - 1)}{3}$ for all $n \geq 1$;
        \item $\displaystyle 1^3 + 2^3 + 3^3 + \dots + n^3 = \left[\frac{n(n+1)}{2}\right]^2$ for all $n \geq 1$; \\
    \end{enumerate}
\end{exercise}

\begin{solution}
    \begin{enumerate}
        \item First, when $n=1$, we have that both sides of the equation are equal to 1, so the basis for the induction is verified. Suppose now that the equation holds for a natural number $k$, then adding $k+1$ on both sides gives us
        $$1 + 2 + 3 + \dots + k + (k+1) = \frac{k(k+1)}{2} + (k+1).$$
        But since
        $$\frac{k(k+1)}{2} + (k+1) = (k+1)\left(\frac{k}{2} + 1\right) = \frac{(k+1)(k+2)}{2},$$
        then 
        $$1 + 2 + 3 + \dots + k + (k+1) = \frac{(k+1)(k+2)}{2}$$
        which implies that the equation holds for $k+1$. Therefore, by induction, it holds for all $n \geq 1$. 
        \item First, when $n=1$, we have that both sides of the equation are equal to 1, so the basis for the induction is verified. Suppose now that the equation holds for a natural number $k$, then adding $2k+1$ on both sides gives us
        $$1 + 3 + 5 + \dots + (2k-1) + (2k+1) = k^2 + (2k+1) = (k+1)^2$$
        which implies that the equation holds for $k+1$. Therefore, by induction, it holds for all $n \geq 1$. 
        \item First, when $n=1$, we have that both sides of the equation are equal to 2, so the basis for the induction is verified. Suppose now that the equation holds for a natural number $k$, then adding $(k+1)(k+2)$ on both sides gives us
        $$1\cdot 2 + 2\cdot 3 + \dots + k(k+1) + (k+1)(k+2) = \frac{k(k+1)(k+2)}{3} + (k+1)(k+2).$$
        But since
        $$\frac{k(k+1)(k+2)}{3} + (k+1)(k+2) = (k+1)(k+2)\left(\frac{k}{3} + 1\right) = \frac{(k+1)(k+2)(k+3)}{3},$$
        then 
        $$1\cdot 2 + 2\cdot 3 + \dots + k(k+1) + (k+1)(k+2) = \frac{(k+1)(k+2)(k+3)}{3}$$
        which implies that the equation holds for $k+1$. Therefore, by induction, it holds for all $n \geq 1$. 
        \item First, when $n=1$, we have that both sides of the equation are equal to 1, so the basis for the induction is verified. Suppose now that the equation holds for a natural number $k$, then adding $(2k+1)^2$ on both sides gives us
        $$1^2 + 3^2 + 5^2 + \dots + (2k-1)^2 + (2k+1)^2 = \frac{k(4k^2 - 1)}{3} + (2k+1)^2.$$
        But since
        \begin{align*}
        \frac{k(4k^2 - 1)}{3} + (2k+1)^2 &= \frac{4k^3 - k + 3(2k+1)^2}{3} \\
        &= \frac{4k^3 - k + 12k^2 + 12k + 3}{3} \\
        &= \frac{4k^3 + 12k^2 + 11k + 3}{3} \\
        &= \frac{(k+1)(4k^2 + 8k + 3)}{3} \\
        &= \frac{(k+1)(4(k+1)^2 - 1)}{3},
        \end{align*}
        then
        $$1^2 + 3^2 + 5^2 + \dots + (2k-1)^2 + (2k+1)^2 = \frac{(k+1)(4(k+1)^2 - 1)}{3}$$
        which implies that the equation holds for $k+1$. Therefore, by induction, it holds for all $n \geq 1$. 
        \item First, when $n=1$, we have that both sides of the equation are equal to 1, so the basis for the induction is verified. Suppose now that the equation holds for a natural number $k$, then adding $(k+1)^3$ on both sides gives us
        $$1^3 + 2^3 + 3^3 + \dots + k^3 + (k+1)^3 = \left(\frac{k(k+1)}{2}\right)^2 + (k+1)^3.$$
        But since
        $$\left(\frac{k(k+1)}{2}\right)^2 + (k+1)^3 = (k+1)^2\left(\frac{k^2}{2^2} + (k+1)\right) = \left(\frac{(k+1)(k+2)}{2}\right)^2,$$
        then 
        $$1^3 + 2^3 + 3^3 + \dots + k^3 + (k+1)^3 = \left(\frac{(k+1)(k+2)}{2}\right)^2$$
        which implies that the equation holds for $k+1$. Therefore, by induction, it holds for all $n \geq 1$. 
    \end{enumerate}
\end{solution}

\begin{exercise}
    If $r \neq 1$, show that
    $$a + ar + ar^2 + \dots ar^n = \frac{a(r^{n+1} - 1)}{r-1}$$
    for any positive integer $n$. \\
\end{exercise}

\begin{solution}
    \\ When $n = 1$, both sides of the equation are equal to $a(r+1)$ so the basis for induction is verified. Suppose now that the equation holds for a positive integer $k$, then adding $ar^{k+1}$ on both sides of the equation gives us
    $$a + ar + ar^2 + \dots ar^k + ar^{k+1} = \frac{a(r^{k+1} - 1)}{r-1} + ar^{k+1}.$$
    But since
    $$\frac{a(r^{k+1} - 1)}{r-1} + ar^{k+1} = \frac{ar^{k+1} - a + ar^{k+2} - ar^{k+1}}{r-1} = \frac{a(r^{k+2} - 1)}{r-1},$$
    then 
    $$a + ar + ar^2 + \dots ar^k + ar^{k+1} = \frac{a(r^{k+2} - 1)}{r-1}$$
    and so the equation holds for all $k+1$. Therefore, it holds for all $n \geq 1$. \\ 
\end{solution}

\begin{exercise}
    Use the Second Principle of Finite Induction to establish that
    $$a^n - 1 = (a-1)(a^{n-1} + a^{n-2} + a^{n-3} + \dots + a + 1)$$
    for all $n \geq 1$. \\
\end{exercise}

\begin{solution}
    \\ When $n = 1$, both sides of the equation are equal to $a - 1$, so the basis for the induction is verified. Suppose now that there exists a positive integer $k$ such that the equation holds for all $n = 1, ..., k$. From the identity
    $$a^{n+1} - 1 = (a+1)(a^n - 1) - a(a^{n-1} - 1),$$ 
    and by the inductive hypothesis for $n = k$ and $n = k-1$, we obtain:
    \begin{align*}
        a^{n+1} - 1 &= (a+1)(a-1)(a^{n-1} + \dots + 1) -a(a-1)(a^{n-2} + \dots + 1) \\
        &= (a-1)[(a+1)(a^{n-1} + \dots + 1) -a(a^{n-2} + \dots + 1)] \\
        &= (a-1)[(a+1)(a^{n-1} + \dots + 1) -(a^{n-1} + \dots + 1 - 1)] \\
        &= (a-1)[(a+1)(a^{n-1} + \dots + 1) -(a^{n-1} + \dots + 1) + 1] \\
        &= (a-1)[a(a^{n-1} + \dots + 1) + 1] \\
        &= (a-1)(a^n + a^{n-1} + \dots + a + 1)
    \end{align*}
    which proves that the equation holds for $n = k+1$. Therefore, by induction, it holds for all $n \geq 1$. \\
\end{solution}

\begin{exercise}
    Prove that the cube of any integer can be written as the difference of two squares. \\
\end{exercise}

\begin{solution}
    \\ Using part (e) of exercice 1, we get
    \begin{align*}
        n^3 &= (1^3 + 2^3 + \dots + n^3) - (1^3 + 2^3 + \dots (n-1)^3) \\
        &= \left[\frac{n(n+1)}{2}\right]^2 - \left[\frac{n(n-1)}{2}\right]^2
    \end{align*}
    which proves that any cube can be written as the difference of two squares. \\
\end{solution}

\begin{exercise}
    \begin{enumerate}
        \item Find the values of $n \leq 7$ for which $n! + 1$ is a perfect square (it is unknown whether $n! + 1$ is a square for any $n > 7$).
        \item True or false? For positive integers $m$ and $n$, $(mn)! = m!n!$ and $(m+n)! = m! + n!$. \\
    \end{enumerate}
\end{exercise}

\begin{solution}
    \begin{enumerate}
        \item For $n = 0, 1$, we have $n! + 1 = 2$ which is not a square. For $n = 2$, we have $2! + 1 = 3$ which is not a square. For $n = 3$, we have $3! + 1 = 7$ which is not a square. When $n = 4$ and $n = 5$, we obtain $4! + 1 = 5^2$ and $5! + 1 = 11^2$. For $n = 6$, we get $6! + 1 = 721$ which is strictly between $26^2 = 676$ and $27^2 = 729$ so it cannot be a square. Finally, for $n = 7$, we obtain $7! + 1 = 71^2$. 
        \item In both cases, $m=n=2$ is a counterexample since $(m+n)! = (mn)! = 24$ and $m!n! = m! + n! = 4$.  \\
    \end{enumerate}
\end{solution}

\begin{exercise}
    Prove that $n! > n^2$ for every integer $n \geq 4$, while $n! > n^3$ for every integer $n \geq 6$. \\
\end{exercise}

\begin{solution}
    \\ When $n = 4$, then $n! = 24$ and $n^2 = 16$ so the strict inequality is satisfied. Now that the basis for the induction is verified, suppose that the inequality is satisfied for a positive integer $k$, then multiplying on both sides by $k+1$ gives the inequality
    $$(k+1)! > k^2(k+1) \geq (k+1)(k+1) = (k+1)^2$$
    using the fact that $k^2 \geq k+1$ for all $k \geq 2$. Thus, since the inequality is also satisfied by $k+1$, then it is for all $n \geq 4$ by induction.

    When $n = 6$, then $n! = 720$ and $n^3 = 216$ so the strict inequality is satisfied. Now that the basis for the induction is verified, suppose that the inequality is satisfied for a positive integer $k$, then multiplying on both sides by $k+1$ gives the inequality
    $$(k+1)! > k^3(k+1) \geq (k+1)^2(k+1) = (k+1)^3$$
    using the fact that $k^3 \geq (k+1)^2$ for all $k \geq 4$. Thus, since the inequality is also satisfied by $k+1$, then it is for all $n \geq 6$ by induction. \\
\end{solution}

\begin{exercise}
    Use mathematical induction to derive the formula
    $$1\cdot (1!) + 2\cdot (2!) + 3\cdot (3!) + \dots + n\cdot (n!) = (n+1)! - 1$$
    for all $n \geq 1$. \\
\end{exercise}

\begin{solution}
    \\ If $n = 1$, then both expressions on the two side of the desired equation are equal to 1; so the basis for the induction is verified. Next, if we suppose that the equation holds for a positive integer $k$, then adding $(k+1)\cdot (k+1)!$ on both sides gives us
    \begin{align*}
        1\cdot (1!) + 2\cdot (2!) + 3\cdot (3!) + \dots + (k+1)\cdot (k+1)! &= (k+1)! - 1 + (k+1)\cdot(k+1)! \\
        &= (k+2)\cdot (k+1)! - 1 \\
        &= (k+2)! - 1
    \end{align*}
    which shows that the equation also holds for $n = k+2$. Therefore, by induction, it holds for all $n \geq 1$. \\
\end{solution}

\begin{exercise}
    \begin{enumerate}
        \item Verify that
        $$2\cdot 6 \cdot 10 \cdot 14 \cdot ... \cdot (4n-2) = \frac{(2n)!}{n!}$$
        for all $n \geq 1$.
        \item Use part (a) to to obtain the inequality $2^n (n!)^2 \leq (2n)!$ for all $n \geq 1$. \\
    \end{enumerate}
\end{exercise}

\begin{solution}
    \begin{enumerate}
        \item Let's prove it by induction on $n$. When $n = 1$, then both expressions on the two sides of the equation are equal to 2, so the basis for the induction is verified. Now, if we suppose that the equation holds for a positive integer $k$, then by multiplying both sides by $(4k+2)$ gives us
        $$2\cdot 6 \cdot 10 \cdot ... \cdot (4n+2) = \frac{(2n)!}{n!}(4n+2) = \frac{(2n)!}{n!}\cdot \frac{(2n+1)(2n+2)}{n+1} = \frac{(2(n+1)!)}{(n+1)!}.$$
        Thus, the equation also holds for $n =k+1$. Therefore, by induction, it holds for all $n \geq 1$.
        \item First fix a $n \geq 1$ and notice that
        $$n! = 1\cdot 2 \cdot 3 \cdot ... \cdot n \leq 1 \cdot 3 \cdot 5 \cdot ... \cdot (2n-1).$$ 
        Next, multiplying both sides by $2^n$ and using part (a) gives us
        $$2^n \cdot n! \leq 2 \cdot 6 \cdot 10 \cdot ... \cdot (4n-2) = \frac{(2n)!}{n!}.$$
        Finally, multiplying both sides by $n!$ gives us the desired inequality. 
    \end{enumerate}
\end{solution}

\begin{exercise}
    Establish the Bernoulli inequality: if $1 + a > 0$, then 
    $$(1 + a)^n \leq 1 + na$$
    for all $n \geq 1$. \\
\end{exercise}

\begin{solution}
    \\ Let's prove it by induction on $n$. When $n = 1$, then $(1 + a)^n = 1 + a \geq 1 + a = 1 + na$, and so the basis for induction is verified. Next, suppose that the inequality holds for a positive integer $k$, then multiplying both sides by $(1+a)$ preserves the inequality since it is positive. Hence, we obtain:
    \begin{align*}
        (1+a)^{k+1} &= (1+a)(1+a)^k \\
        &\geq (1+a)(1+ka) \\
        &= 1 + (k+1)a + ka^2 \\
        &\geq 1 + (k+1)a
    \end{align*}
    which shows that the inequality must also hold for $n = k+1$. Therefore, by induction, it holds for all $n \geq 1$.  \\
\end{solution}

\begin{exercise}
    Prove by mathematical induction that
    $$\frac{1}{1^2} + \frac{1}{2^2} + \frac{1}{3^2} + \dots + \frac{1}{n^2} \leq 2 - \frac{1}{n}$$
    for all $n \geq 1$. \\
\end{exercise}

\begin{solution}
    \\ When $n = 1$, both sides of the inequality are equal to 1, so the inequality holds and so the basis for the induction is verified. Next, if we suppose that the inequality holds for a positive integer $k$, then adding $\frac{1}{(k+1)^2}$ on both sides gives us
    $$\frac{1}{1^2} + \frac{1}{2^2} + \frac{1}{3^2} + \dots + \frac{1}{(k+1)^2} \leq 2 - \frac{1}{k} + \frac{1}{(k+1)^2}.$$
    But notice that
    \begin{align*}
        0 \leq 1 &\implies 2k + k^2 \leq 1 + 2k + k^2 \\
        &\implies 2k + k^2 \leq (k+1)^2 \\
        &\implies 1 - \frac{(k+1)^2}{k} \leq -(k+1) \\
        &\implies \frac{1}{(k+1)^2} - \frac{1}{k} \leq - \frac{1}{(k+1)}
    \end{align*}
    and so 
    $$\frac{1}{1^2} + \frac{1}{2^2} + \frac{1}{3^2} + \dots + \frac{1}{(k+1)^2} \leq 2 + \frac{1}{(k+1)^2} - \frac{1}{k} \leq 2 - \frac{1}{k+1}.$$
    Thus, the inequality holds for $n = k+1$. Therefore, by induction, the inequality holds for all $n \geq 1$. 
\end{solution}
\section{The Binomial Theorem}

\begin{exercise}
    Prove that for $n \geq 1$:
    \begin{enumerate}
        \item $\displaystyle \binom{2n}{n} = \frac{1\cdot 3 \cdot 5 \cdot \cdots \cdot (2n-1)}{n!}2^n .$
        \item $\displaystyle \binom{4n}{2n} = \frac{1\cdot 3 \cdot 5 \cdot \cdots \cdot (4n-1)}{[1\cdot 3 \cdot 5 \cdot \cdots \cdot (2n-1)]^2}\binom{2n}{n}.$
    \end{enumerate}
\end{exercise}

\begin{solution}
    \begin{enumerate}
        \item Let's prove it by induction. When $n = 1$, we have 
        $$\binom{2n}{n} = 2$$
        and 
        $$\frac{1\cdot 3 \cdot 5 \cdot \cdots \cdot (2n-1)}{n!}2^n = 2$$
        and so it holds in that case. If we now suppose that it holds when $n = k$ for some integer $k \geq 1$, then it follows that
        $$\frac{(2k)!}{(k!)^2} = \frac{1\cdot 3 \cdot 5 \cdot \cdots \cdot (2k-1)}{k!}2^k.$$
        Multiplying both sides by $\frac{(2k+1)(2k+2)}{(k+1)^2}$ gives us
        \begin{align*}
            \binom{2(k+1)}{k+1} &= \frac{(2k+1)(2k+2)}{(k+1)^2} \cdot \frac{1\cdot 3 \cdot 5 \cdot \cdots \cdot (2k-1)}{k!}2^k \\
            &= 2\frac{(2k+1)}{k+1} \cdot \frac{1\cdot 3 \cdot 5 \cdot \cdots \cdot (2k-1)}{k!}2^k \\
            &= \frac{1\cdot 3 \cdot 5 \cdot \cdots \cdot (2k+1)}{(k+1)!}2^{k+1} 
        \end{align*}
        which shows that the equation also holds for $n = k+1$. Therefore, by induction, it holds for all integers $n \geq 1$.
        \item First, notice that by part (a), it suffices to prove that
        $$\binom{4n}{2n} = \frac{1\cdot 3 \cdot 5 \cdot \cdots \cdot (4n-1)}{n! \cdot 1\cdot 3 \cdot 5 \cdot \cdots \cdot (2n-1)}2^n $$
        holds for all $n \geq 1$. Let's prove it by induction on $n$. When $n = 1$, then both sides are equal to 6 and so the statement holds in that case. Suppose now that it holds for some integer $n = k \geq 1$, then 
        $$\frac{(4k)!}{(2k!)^2} = \frac{1\cdot 3 \cdot 5 \cdot \cdots \cdot (4k-1)}{k! \cdot 1\cdot 3 \cdot 5 \cdot \cdots \cdot (2k-1)}2^k.$$
        Multiplying both sides by $\frac{(4k+1)(4k+2)(4k+3)(4k+4)}{(2k+1)^2(2k+2)^2}$ gives us
        \begin{align*}
            \binom{4(k+1)}{2(k+1)} &= \frac{(4k+1)(4k+2)(4k+3)(4k+4)}{(2k+1)^2(2k+2)^2} \cdot \frac{1\cdot 3 \cdot 5 \cdot \cdots \cdot (4k-1)}{k! \cdot 1\cdot 3 \cdot 5 \cdot \cdots \cdot (2k-1)}2^k \\
            &= \frac{(4k+2)(4k+4)}{(2k+1)(2k+2)^2} \cdot \frac{1\cdot 3 \cdot 5 \cdot \cdots \cdot (4k+1)}{k! \cdot 1\cdot 3 \cdot 5 \cdot \cdots \cdot (2k+1)}2^k \\
            &= \frac{4k+4}{(2k+2)(2k+2)} \cdot \frac{1\cdot 3 \cdot 5 \cdot \cdots \cdot (4k+1)}{k! \cdot 1\cdot 3 \cdot 5 \cdot \cdots \cdot (2k+1)}2^{k+1} \\
            &= \frac{1}{k+1} \cdot \frac{1\cdot 3 \cdot 5 \cdot \cdots \cdot (4k+1)}{k! \cdot 1\cdot 3 \cdot 5 \cdot \cdots \cdot (2k+1)}2^{k+1} \\
            &= \frac{1\cdot 3 \cdot 5 \cdot \cdots \cdot (4k+1)}{(k+1)! \cdot 1\cdot 3 \cdot 5 \cdot \cdots \cdot (2k+1)}2^{k+1}
        \end{align*}
        which shows that the equation also holds for $n = k+1$. Therefore, by induction, it holds for all integers $n \geq 1$.
    \end{enumerate}
\end{solution}

\begin{exercise}
    If $2\leq k \leq n - 2$, show that
    $$\binom{n}{k} = \binom{n-2}{k-2} + 2\binom{n - 2}{k-1} + \binom{n-2}{k}, \qquad n \geq 4.$$
\end{exercise}

\begin{solution}
    \\ This simply follows from Pascal's Rule:
    \begin{align*}
        \binom{n-2}{k-2} + 2\binom{n - 2}{k-1} + \binom{n-2}{k} &= \left[\binom{n-2}{k-2} + \binom{n - 2}{k-1}\right] + \left[\binom{n - 2}{k-1} + \binom{n-2}{k}\right] \\
        &= \binom{n-1}{k-1} + \binom{n-1}{k} \\
        &= \binom{n}{k}.
    \end{align*}
\end{solution}

\begin{exercise}
    For $n \geq 1$, derive each of the identities below:
    \begin{enumerate}
        \item $\displaystyle \binom{n}{0} + \binom{n }{1} + \binom{n }{2} + \dots + \binom{n }{n } = 2^n$;
        [\textit{Hint:} Let $a = b = 1$ in the binomial theorem.]
        \item $\displaystyle \binom{n}{0} - \binom{n }{1} + \binom{n }{2} - \dots + (-1)^n\binom{n }{n } = 0$;
        \item $\displaystyle \binom{n}{1} + 2\binom{n }{2} + 3\binom{n }{3} + \dots + n\binom{n }{n } = n2^{n-1}$;
        [\textit{Hint:} After expanding $n(1 + b)^{n-1}$ by the binomial theorem, let $b = 1$: note also that
        $$n \binom{n-1}{k } = (k+1)\binom{n }{k+1}.]$$
        \item $\displaystyle \binom{n}{0} + 2\binom{n }{1} + 2^2\binom{n }{2} + \dots + 2^n\binom{n }{n } = 3^n$;
        \item $\displaystyle \binom{n}{0} + \binom{n }{2} + \binom{n }{4} + \binom{n }{6 } + \dots$
        
        $\displaystyle \qquad \quad \ \binom{n}{1} + \binom{n }{3} + \binom{n }{5} + \dots = 2^{n-1};$
        [\textit{Hint:} Use parts (a) and (b).]
        \item $\displaystyle \binom{n}{0} - \frac{1}{2}\binom{n }{1} + \frac{1}{3}\binom{n }{2} - \dots + \frac{(-1)^n}{n+1}\binom{n }{n } = \frac{1}{n+1}$;
        [\textit{Hint:} the left-hand side equals
        
        $ \displaystyle \frac{1}{n+1} \left[\binom{n+1}{1} - \binom{n+1}{2} + \binom{n+1}{3} - \dots + (-1)^n\binom{n+1}{n+1}\right]$.]
    \end{enumerate}
\end{exercise}

\begin{solution}
    \begin{enumerate}
        \item Taking $a = b = 1$ in the Binomial Theorem gives us
        $$2^n = (1 + 1)^n = \sum_{k=0}^{n}\binom{n}{k}a^{n-k}b^k = \binom{n}{0} + \binom{n }{1} + \binom{n }{2} + \dots + \binom{n }{n }.$$
        \item Taking $a = 1$ and $b = -1$ in the Binomial Theorem gives us
        $$0 = (1 - 1)^n = \sum_{k=0}^{n}\binom{n}{k}a^{n-k}b^k = \binom{n}{0} - \binom{n }{1} + \binom{n }{2} - \dots (-1)^n \binom{n }{n }.$$
        \item From the hint, it follows that
        $$ \binom{n}{1} + 2\binom{n }{2} + 3\binom{n }{3} + \dots + n\binom{n }{n } = n\binom{n-1}{0} + n\binom{n-1}{1} + \dots + n\binom{n-1}{n-1} = n2^{n-1}$$
        where the last equality follows from part (a).
        \item Taking $a = 1$ and $b = 2$ in the Binomial Theorem gives us
        $$3^n = (1 + 2)^n = \sum_{k=0}^{n}\binom{n}{k}a^{n-k}b^k = \binom{n}{0} + 2\binom{n }{1} + 2^2\binom{n }{2} + \dots 2^n \binom{n }{n }.$$
        \item From part (b), we have that 
        $$\binom{n }{0} + \binom{n }{2} + \binom{n }{4} + \dots = \binom{n }{1} + \binom{n }{3} + \binom{n }{5} + \dots$$
        Thus, using part (a), we get
        \begin{align*}
            \binom{n }{0} + \binom{n }{2} + \binom{n }{4} + \dots &= \binom{n }{1} + \binom{n }{3} + \binom{n }{5} + \dots \\
            &= \frac{1}{2}\left[\binom{n }{0} + \binom{n }{1} + \binom{n }{2} + \dots + \binom{n }{n }\right] \\
            &= 2^{n-1}
        \end{align*}
        \item Using the hint, we easily get 
        \begin{align*}
            & \binom{n}{0} - \frac{1}{2}\binom{n }{1} + \frac{1}{3}\binom{n }{2} - \dots + \frac{(-1)^n}{n+1}\binom{n }{n } \\
            =& \frac{1}{n+1} \left[\binom{n+1}{1} - \binom{n+1}{2} + \binom{n+1}{3} - \dots + (-1)^n\binom{n+1}{n+1}\right] \\
            =& \frac{1}{n+1} \left(1 - \left[\binom{n }{0} - \binom{n+1}{1} + \binom{n+1}{2} - \binom{n+1}{3} + \dots + (-1)^{n+1}\binom{n+1}{n+1}\right]\right) \\
            =& \frac{1}{n+1}(1 - 0) \\
            =& \frac{1}{n+1}
        \end{align*}
    \end{enumerate}
\end{solution}

\begin{exercise}
    Prove that for $n \geq 1$:
    \begin{enumerate}
        \item $\displaystyle \binom{n }{r } < \binom{n }{r+1}$ if and only if $0 \leq r < \displaystyle \frac{1}{2}(n-1)$.
        \item $\displaystyle \binom{n }{ r} > \binom{n }{r+1}$ if and only if $n-1 \geq r > \displaystyle \frac{1}{2}(n-1)$.
        \item $\displaystyle \binom{n }{r } = \binom{n }{r+1 }$ if and only if $n$ is an odd integer, and $\displaystyle r = \frac{1}{2}(n-1)$.
    \end{enumerate}
\end{exercise}

\begin{solution}
    \begin{enumerate}
        \item Let $0 \leq r \leq n-1$ be an integer, then
        \begin{align*}
            \binom{n }{r } < \binom{n }{r+1} &\iff \frac{n!}{(n-r)! r!} < \frac{n!}{(n-r - 1)! (r+1)!} \\
            &\iff (n-r - 1)! (r+1)! < (n-r)! r! \\
            &\iff r+1 < n-r \\
            &\iff r < \frac{1}{2}(n-1).
        \end{align*}
        \item Let $0 \leq r \leq n-1$ be an integer, then
        \begin{align*}
            \binom{n }{r } > \binom{n }{r+1} &\iff \frac{n!}{(n-r)! r!} > \frac{n!}{(n-r - 1)! (r+1)!} \\
            &\iff (n-r - 1)! (r+1)! > (n-r)! r! \\
            &\iff r+1 > n-r \\
            &\iff r > \frac{1}{2}(n-1).
        \end{align*}
        \item Let $0 \leq r \leq n-1$ be an integer, then
        \begin{align*}
            \binom{n }{r } = \binom{n }{r+1} &\iff \frac{n!}{(n-r)! r!} = \frac{n!}{(n-r - 1)! (r+1)!} \\
            &\iff (n-r - 1)! (r+1)! = (n-r)! r! \\
            &\iff r+1 = n-r \\
            &\iff r = \frac{1}{2}(n-1) \\
            &\iff n = 2r+1.
        \end{align*}
    \end{enumerate}
\end{solution}

\begin{exercise}
    For $n \geq 1$, show that the expressions $\displaystyle \frac{(2n)!}{n!(n+1)!}$ and $\displaystyle \frac{(3n)!}{6^n n!}$ are both integers. \\
\end{exercise}

\begin{solution}
    \\ For the first expression, it suffices to notice that
    \begin{align*}
        \binom{2n }{n } - \binom{2n }{n+1} &= \frac{(2n)!}{n!n!} - \frac{(2n)!}{(n-1)! (n+1)!} \\
        &= \frac{(2n)!(n+1) - (2n)!n}{n!(n+1)!} \\
        &= \frac{(2n)!}{n!(n+1)!}.
    \end{align*}
    Since the binomial coefficients are integers, then it follows that the expression $\frac{(2n)!}{n!(n+1)!}$ is also an integer. For the second expression, let's prove it by induction. When $n = 1$, we have
    $$\frac{(3n)!}{6^n n!} = \frac{3!}{6 \cdot 1} = 1$$
    which proves that it holds for $n = 1$. Suppose now that the expression is an integer for some $n = k \geq 1$, then 
    \begin{align*}
        \frac{(3(k+1))!}{6^{k+1}(k+1)!} &= \frac{(3k+1)(3k+2)(3k+3)}{6(k+1)} \cdot \frac{(3k)!}{6^k k!} \\
        &= \frac{(3k+1)(3k+2)}{2} \cdot \frac{(3k)!}{6^k k!} 
    \end{align*}
    where $\frac{(3k)!}{6^k k!} $ is an integer by the inductive hypothesis. Moreover, notice that $3k+1$ and $3k+2$ are two consecutive numbers and so one of them must be divisible by two. Thus, $\frac{(3k+1)(3k+2)}{2}$ is also an integer. Therefore, the case $n = k+1$ also holds since $\frac{(3(k+1))!}{6^{k+1}(k+1)!}$ can be written as the product of two integers. \\
\end{solution}

\begin{exercise}
    \begin{enumerate}
        \item For $n \geq 2$, prove that
        $$\binom{2 }{2 } + \binom{3}{2} + \binom{4}{2} + \dots + \binom{n }{2} = \binom{n+1}{3}.$$
        [\textit{Hint} Use induction and Pascal's rule.]
        \item From part (a) and the fact that $\displaystyle \binom{ m }{2} + \binom{m+1 }{2} = m^2$ for $m \geq 2$, deduce the formula
        $$1^2 + 2^2 + 3^2 + \dots + n^2 = \frac{n(n+1)(2n+1)}{6}.$$
    \end{enumerate}
\end{exercise}

\begin{solution}
    \begin{enumerate}
        \item Let's prove it by induction on $n$. When $n =2$, we have
        $$\binom{2}{2} + \dots + \binom{n }{2} = \binom{2}{2} = 1 = \binom{3 }{3} = \binom{n + 1}{3}$$
        and so the proposition holds in that case. Suppose now that the proposition holds for $n = k \geq 2$, then 
        $$\binom{2 }{2 } + \binom{3}{2} + \binom{4}{2} + \dots + \binom{k }{2} = \binom{k+1}{3}.$$
        Adding $\displaystyle \binom{k+1}{2}$ on both sides gives
        $$\binom{2 }{2 } + \binom{3}{2} + \binom{4}{2} + \dots + \binom{k +1 }{2} = \binom{k+1 }{2} + \binom{k+1}{3} = \binom{k+2}{3}$$
        and so the proposition holds for $n = k+1$. Therefore, by induction, it holds for all $n \geq 2$.
        \item Using the fact that $\displaystyle \binom{ m }{2} + \binom{m+1 }{2} = m^2$, we can write
        \begin{align*}
            1^2 + 2^2 + 3^2 + \dots + n^2 &= 1 + \left[\binom{2}{2} + \binom{3}{2}\right] + \left[\binom{3}{2} + \binom{4}{2}\right] + \dots + \left[\binom{n }{2} + \binom{n+1}{2}\right] \\
            &= 2\left[\binom{2}{2} + \binom{3}{2} + \binom{4}{2} + \dots + \binom{n }{2}\right] + \binom{n+1}{2} \\
            &= 2\binom{n+1}{3} + \binom{n+1}{2} \\
            &= \frac{2(n+1)n(n-1)}{6} + \frac{(n+1)n}{2} \\
            &= \frac{2(n+1)n(n-1) + 3(n+1)n}{6} \\
            &= \frac{n(n+1)(2n+1)}{6}.
        \end{align*}
        which proves the desired formula.
    \end{enumerate}
\end{solution}

\begin{exercise}
    For $n \geq 1$, verify that 
    $$1^2 + 3^2 + 5^2 + \dots + (2n-1)^2 = \binom{2n+1}{3}.$$
\end{exercise}

\begin{solution}
    \\ Let's prove it by induction on $n$. When $n = 1$, we have 
    $$1^2 + \dots + (2n-1)^2 = 1 = \binom{3}{3} = \binom{2n+1}{3}.$$
    Thus, the proposition holds for $n = 1$. Suppose now that it holds for $n = k \geq 1$, then 
    \begin{align*}
        1^2 + 3^2 + 5^2 + \dots + (2k+1)^2 &= (1^2 + 3^2 + 5^2 + \dots + (2k-1)^2) + (2k+1)^2 \\
        &= \binom{2k+1}{3} + (2k+1)^2 \\
        &= \frac{(2k+1)(2k)(2k-1)}{6} + (2k+1)(2k+1) \\
        &= \frac{(2k+1)[2k(2k-1) + 6(2k+1)]}{6} \\
        &= \frac{(2k+1)(4k^2 + 10k + 6)}{6} \\
        &= \frac{(2k+3)(2k+2)(2k+1)}{6} \\
        &= \binom{2(k+1) + 1}{3}
    \end{align*}
    which shows that it holds for $n = k+1$. Therefore, by induction, the proposition holds for all $n \geq 1$.\\
\end{solution}

\begin{exercise}
    Establish the inequality $2^n < \displaystyle \binom{2n}{n } < 2^{2n}$ for $n > 1$. 
\end{exercise}

\begin{solution}
    \\ Let's prove it by induction on $n$. When $n = 2$, we have 
    $$2^n = 4 < 6 = \binom{2n}{2} < 16 = 2^{2n}$$
    and so it holds for this case. Suppose now that holds for an integer $n = k \geq 2$, then
    $$2^k < \binom{2k}{k } < 2^{2k}.$$
    Multiplying both sides by $\frac{(2k+2)(2k+1)}{(n+1)^2} = 2\frac{2k+1}{k+1}$ gives us
    $$2^{k+1} \leq 2^k \cdot 2\frac{2k+1}{k+1} < \binom{2(k+1)}{k+1} < 2^{2k} \cdot 2\frac{2k+1}{k+1} \leq 2^{2(k+1)} $$
    which shows that it holds for $n = k+1$. Therefore, by induction, the proposition holds for all integers $n > 1$.
\end{solution}
\section{Early Number Theory}

\td 

% CHAPTER 2
\chapter{Divisibility Theory in the Integers}

\section{The Division Algorithm}

\begin{exercise}
    Prove that if $a$ and $b$ are integers, with $b > 0$, then there exist unique integers $q$ and $r$ satisfying $a = qb + r$, where $2b \leq r < 3b$. \\
\end{exercise}

\begin{solution}
    By the division algorithm, we know that there exist unique $q_0$ and $r_0$ such that $a=q_0 b + r_0$ and $0 \leq r_0 < b$. This implies that if we let $q = q_0 - 2$ and $r = r_0 + 2b$, then $a = qb + r$ with $2b \leq r < 3b$. To prove that $q$ and $r$ are unique, let $q'$ and $r'$ be integers such $a = q'b + r'$ and $2b \leq r' < 3b$, then equivalently: $a = (q' + 2)b + (r' - 2b)$ where $0 \leq r' - 2b < 3b$. But by uniqueness of $q_0$ and $r_0$, we have $r' - 2b = r_0$ and so $r' = r_0 + 2b = r$ by definition of $r$. This concludes the proof. \\
\end{solution}

\begin{exercise}
    Show that any integer of the form $6k + 5$ is also of the form $3k + 2$, but not conversely. \\
\end{exercise}

\begin{solution}
    Let $n$ be an integer of the form $6k + 5$, then
    $$n = 3 \cdot 2k + 3 + 2 = 3(2k+1) + 2$$
    which proves that $n$ is of the form $3k+2$. However, the converse does not hold since the integer $8 = 3 \cdot 2 + 2$ can also be written as $6 \cdot 1 + 2$. This shows that the converse cannot hold since otherwise, we would have a number that has both forms $6k + 2$ and $6k + 5$ which would contradict the uniqueness part of the Division Algorithm. \\
\end{solution}

\begin{exercise}
    Use the Division Algorithm to establish that
    \begin{enumerate}
        \item the square of any integer is either of the form $3k$ or $3k + 1$;
        \item the cube of any integer has one of the forms $9k$, $9k + 1$ or $9k+8$;
        \item the fourth power of any integer is either of the form $5k$ or $5k+1$.
    \end{enumerate}
\end{exercise}

\begin{solution}
    \begin{enumerate}
        \item Let $n$ be an integer, then by the Division Algorithm, we have that $n$ has one of the following forms: $3k$, $3k+1$ or $3k+2$. Let's split the proof in these three cases. If $n = 3k$, then $n^2 = 3(3k^2)$. If $n = 3k+1$, then $n^2 = 3(3k^2 + 2k) + 1$. If $n = 3k+2$, then $n^2 = 3(3k^2 + 4k + 1) + 1$. Therefore, for any integer $n$, $n^2$ has either the form $3k$ or $3k+1$.
        \item Let $n$ be an integer, then by the Division Algorithm, we have that $n$ has one of the following forms: $3k$, $3k+1$ or $3k+2$. Let's split the proof in these three cases. If $n = 3k$, then $n^3 = 9(3k^3)$. If $n = 3k+1$, then $n^3 = 9(3k^3 + 3k^2 + k) + 1.$ If $n = 3k+2$, then $n^3 = 9(3k^3 + 6k^2 + 4k) + 8.$. Therefore, for any integer $n$, $n^3$ has either the form $9k$, $9k+1$ or $9k + 8$.
        \item Let $n$ be an integer, then by the Division Algorithm, we have that $n$ has one of the following forms: $5k$, $5k+1$, $5k+2$, $5k+3$ or $5k+4$. Let's split the proof in these five cases. If $n = 5k$, then $n^4 = 5(5^3k^4)$. If $n = 5k+1$, then
        $$n^4 = 5(5^3k^4 + 4\cdot 5^2 k^3 + 6\cdot 5 k^2 + 4k) + 1.$$
        If $n = 5k+2$, then
        $$n^4 = 5(5^3k^4 + 2\cdot 4\cdot 5^2 k^3 + 4 \cdot 6\cdot 5 k^2 + 8\cdot 4k + 3) + 1.$$
        If $n = 5k+3$, then
        $$n^4 = 5(5^3k^4 + 3\cdot 4\cdot 5^2 k^3 + 9 \cdot 6\cdot 5 k^2 + 27\cdot 4k + 16) + 1.$$
        If $n = 5k+4$, then
        $$n^4 = 5(5^3k^4 + 4\cdot 4\cdot 5^2 k^3 + 16 \cdot 6\cdot 5 k^2 + 64\cdot 4k + 51) + 1.$$
        Therefore, for any integer $n$, $n^4$ has either the form $5k$ or $5k + 1$.
    \end{enumerate}
\end{solution}

\begin{exercise}
    Prove that $3a^2 - 1$ is never a perfect square. [\textit{Hint:} Problem 3(a).] \\
\end{exercise}

\begin{solution}
    It suffices to notice that $3a^2 - 1 + 3(a^2 - 1) + 2$ and to use the fact that no square can be of the form $3k+2$ from Exercise 3(a). \\
\end{solution}

\begin{exercise}
    For $n \geq 1$, prove that $n(n+1)(2n+1)/6$ is an integer. [Hint: By the Division Algorithm, $n$ has one of the forms $6k$, $6k + 1$, ..., $6k+5$; establish the result in each of these six cases.] \\
\end{exercise}

\begin{solution}
    Let $n$ be an integer, then by the Division Algorithm, we have that $n$ has one of the following forms: $6k$, $6k+1$, $6k+2$, $6k+3$, $6k+4$ or $6k+5$. Let's split the proof in these six cases. If $n = 6k$, then
    $$\frac{n(n+1)(2n+1)}{6} = k(n+1)(2n+1).$$
    If $n = 6k+1$, then
    $$\frac{n(n+1)(2n+1)}{6} = \frac{n(6k+2)(12k+3)}{6} = n(3k+1)(4k+1).$$
    If $n = 6k+2$, then
    $$\frac{n(n+1)(2n+1)}{6} = \frac{(6k+2)(6k+3)(2n+1)}{6} = (3k+1)(2k+1)(2n+1).$$
    If $n = 6k+3$, then
    $$\frac{n(n+1)(2n+1)}{6} = \frac{(6k+3)(6k+4)(2n+1)}{6} = (2k+1)(3k+2)(2n+1).$$
    If $n = 6k+4$, then
    $$\frac{n(n+1)(2n+1)}{6} = \frac{(6k+4)(n+1)(12k+9)}{6} = (3k+2)(n+1)(4k+3).$$
    If $n = 6k+5$, then
    $$\frac{n(n+1)(2n+1)}{6} = \frac{n(6k+6)(2n+1)}{6} = n(k+1)(2n+1).$$
    Therefore, $n(n+1)(2n+1)/6$ is an integer. \\
\end{solution}

\begin{exercise}
    Verify that if an integer is simultaneously a square and a cube (as is the case with $64 = 8^2 = 4^2$), then it must be either of the form $7k$ of $7k+1$.\\
\end{exercise}

\begin{solution}
    Let's look at possible remainders for squares and cubes of integers when divided by 7. Since every integer can be written as $7k$, $7k + 1$, ..., $7k+6$, then we get
    \begin{align*}
        (7k)^2 &= 7(7k^2) + 0 \\
        (7k + 1)^2 &= 7(7k^2 + 2k) + 1 \\
        (7k+2)^2 &= 7(7k^2 + 2\cdot 2k) + 4 \\
        (7k+3)^2 &= 7(7k^2 + 3\cdot 2k + 1) + 2 \\
        (7k+4)^2 &= 7(7k^2 + 4\cdot 2k + 2) + 2 \\
        (7k+5)^2 &= 7(7k^2 + 5\cdot 2k + 3) + 4 \\
        (7k+6)^2 &= 7(7k^2 + 6\cdot 2k + 5) + 1 
    \end{align*}
    and 
    \begin{align*}
        (7k)^3 &= 7(7^2k^3) + 0\\
        (7k+1)^3 &= 7(7^2k^3 + 3 \cdot 7k^2 + 3k) + 1\\
        (7k+2)^3 &= 7(7^2k^3 + 3 \cdot 2 \cdot 7k^2 + 4 \cdot 3k + 1) + 1\\
        (7k+3)^3 &= 7(7^2k^3 + 3 \cdot 3 \cdot 7k^2 + 9 \cdot 3k + 3) + 6\\
        (7k+4)^3 &= 7(7^2k^3 + 3 \cdot 4 \cdot 7k^2 + 16 \cdot 3k + 9) + 1\\
        (7k+5)^3 &= 7(7^2k^3 + 3 \cdot 5 \cdot 7k^2 + 25 \cdot 3k + 17) + 6\\
        (7k+6)^3 &= 7(7^2k^3 + 3 \cdot 6 \cdot 7k^2 + 36 \cdot 3k + 30) + 6.
    \end{align*}
    Therefore, the only possible remainders after dividing a square by 7 are 0, 1, 2 and 4; and the only possible remainders after dividing a cube by 7 are 0, 1 and 6. Thus, if a number is a square and a cube at the same time, then it can only be of the form $7k$ or $7k+1$ since 0 and 1 are the only common remainders after dividing 7 for squares and cubes. \\
\end{solution}

\begin{exercise}
    Obtain the following version of the Division Algorithm: For integers $a$ and $b$, with $b \neq 0$, there exist unique integers $q$ and $r$ satisfying $a = qb + r$, where $-\frac{1}{2}|b| < r \leq \frac{1}{2}|b|$. [\textit{Hint:} First write $a = q'b + r'$, where $0 \leq r' < |b|$. When $0 \leq r' \leq \frac{1}{2}|b|$, let $r = r'$ and $q = q'$; when $\frac{1}{2}|b| < r' \leq |b|$, let $r = r' - |b|$ and $q = q' + 1$ if $b > 0$ or $q = q' - 1$ if $b < 0$.]\\
\end{exercise}

\begin{solution}
    The hint already gives a major part of the exercice but let's still do it. First, by the Division Algorithm, there exist unique integers $q'$ and $r'$ such that $a = q'b + r'$ and $0 \leq r' < |b|$. Consider the two following cases: either $0 \leq r' \leq \frac{1}{2}|b|$ or $\frac{1}{2}|b| < r' < |b|$. In the first case, let $q = q'$ and $r = r'$ to obtain $a = qb + r$ with $-\frac{1}{2}|b| < r \leq \frac{1}{2}|b|$. Similarly, if $\frac{1}{2}|b| < r' < |b|$, let $r = r' - |b|$ and $q = q' + 1$ if $b > 0$ or $q = q' - 1$ if $b. < 0$. From this, we get that $a = qb + r$ with $-\frac{1}{2}|b| < r \leq \frac{1}{2}|b|$. 

    To prove the uniqueness of $q$ and $r$, suppose that there exist integers $q_0$ and $r_0$ such that $a = q_0 b + r_0$ and $-\frac{1}{2}|b| < r_0 \leq \frac{1}{2}|b|$. If $0 \leq r_0 \leq \frac{1}{2}|b| < |b|$, then by uniqueness of $q'$ and $r'$, we get that $q_0 = q'$ and $r_0 = r'$. Moreover, since $0 \leq r' \leq \frac{1}{2}|b|$, then by definition of $q$ and $r$ in that case, we get that $q_0 = q$ and $r_0 = r$. Otherwise, $ - \frac{1}{2}|b| < r_0 < 0$. If $b > 0$, then we can write
    $$a = (q_0 - 1)b + (r_0 + b)$$
    with $\frac{1}{2}|b| < r_0 + b < |b|$. By uniqueness of $q'$ and $r'$, we get $q' = q_0 - 1$ and $r_0 + b = r'$. But since in that case $r = r' - b$ and $q = q' + 1$, then $r_0 = r$ and $q_0 = q$. Otherwise, if $b < 0$, then we can write 
    $$a = (q_0 + 1)b + (r_0 + |b|)$$
    with $\frac{1}{2}|b| < r_0 + |b| < |b|$. By uniqueness of $q'$ and $r'$, we get $q' = q_0 + 1$ and $r_0 + |b| = r'$. But since in that case $r = r' - |b|$ and $q = q' - 1$, then $r_0 = r$ and $q_0 = q$. Therefore, in all possible cases, $r_0 = r$ and $q_0 = q$. It follows that $q$ and $r$ are unique.\\
\end{solution}

\begin{exercise}
    Prove that no integer in the sequence
    $$11, \ 111, \ 1111, \ 11111, \ \dots$$
    is a perfect square. [\textit{Hint:} A typical term 111 \dots 111 can be written as $111\dots111 = 111\dots 108 + 3 = 4k+3$.]\\
\end{exercise}

\begin{solution}
    Since every element in the sequence can be written in the form $4k+3$, then using one of the example in the section stating that squares must have the form $4k$ or $4k+1$, it follows that no element in the sequence can be a square. \\
\end{solution}

\begin{exercise}
    Show that the cube of any integer is of the form $7k$ or $7k \pm 1$. \\
\end{exercise}

\begin{solution}
    Let's look at possible remainders of cubes of integers when divided by 7. Since every integer can be written as $7k$, $7k + 1$, ..., $7k+6$, then we get
    \begin{align*}
        (7k)^3 &= 7(7^2k^3) + 0\\
        (7k+1)^3 &= 7(7^2k^3 + 3 \cdot 7k^2 + 3k) + 1\\
        (7k+2)^3 &= 7(7^2k^3 + 3 \cdot 2 \cdot 7k^2 + 4 \cdot 3k + 1) + 1\\
        (7k+3)^3 &= 7(7^2k^3 + 3 \cdot 3 \cdot 7k^2 + 9 \cdot 3k + 4) -1 \\
        (7k+4)^3 &= 7(7^2k^3 + 3 \cdot 4 \cdot 7k^2 + 16 \cdot 3k + 9) + 1\\
        (7k+5)^3 &= 7(7^2k^3 + 3 \cdot 5 \cdot 7k^2 + 25 \cdot 3k + 18) - 1\\
        (7k+6)^3 &= 7(7^2k^3 + 3 \cdot 6 \cdot 7k^2 + 36 \cdot 3k + 31) - 1.
    \end{align*}
    It follows that every cube must be of the form $7k$ or $7k \pm 1$. \\
\end{solution}

\begin{exercise}
    For $n \geq 1$, establish that the integer $n(7n^2 + 5)$ is of the form $6k$.\\
\end{exercise}

\begin{solution}
    Notice that $n$ must have one of the following form: $6k$, $6k + 1$, ..., $6k + 5$. If $n = 6k$, then 
    $$n(7n^2 + 5) = 6[k(7n^2 + 5)].$$
    If $n = 6k + 1$, then
    $$n(7n^2 + 5) = n(7\cdot 6^2 k^2 + 2 \cdot 7\cdot 6k + 7 + 5) = 6[n(7\cdot 6k^2 + 2k + 2)].$$
    If $n = 6k + 2$, then 
    $$n(7n^2 + 5) = (6k+2)(7\cdot 6^2 k^2 + 7\cdot 4 \cdot 6k + 33) = 6[(3k+1)(84k^2+56k+11)].$$
    If $n = 6k + 3$, then
    $$n(7n^2 + 5) = (6k+3)(7\cdot 6^2 k^2 + 7\cdot 6 \cdot 6k + 68) = 6[(2k+1)(126k^2 + 126k + 34)].$$
    If $n = 6k + 4$, then
    $$n(7n^2 + 5) = (6k+4)(7\cdot 6^2k^2 + 7\cdot 8 \cdot 6k + 117) = 6[(3k+2)(84k^2 + 112k + 39)].$$
    If $n = 6k+5$, then 
    $$n(7n^2 + 5) = n(7\cdot 6^2 k^2 + 7\cdot 10 \cdot 6k + 180) = 6[n(42k + 70k + 30)]$$
    Therefore, since it holds for all possible cases, it is clear that $n(7n^2 + 5)$ is of the form $6k$. \\
\end{solution}

\begin{exercise}
    If $n$ is an odd integer, show that $n^4 + 4n^2 + 11$ is of the form $16k$. \\
\end{exercise}

\begin{solution}
    If $n$ is an odd integer, then it can be written as $n = 4k + 1$ or $4k -1$ (Exercise 7). If $n = 4k+1$, then:
    \begin{align*}
        n^4 + 4n^2 + 11 &= (4k + 1)^4 + 4(4k + 1)^2 + 11 \\
        &= 4^4k^4 + 4 \cdot 4^3k^3 + 6\cdot 4^2k^2 + 4\cdot 4k + 1 + 4\cdot 4^2 k^2 + 4\cdot 4 \cdot 2k + 4 + 11 \\
        &= 16(16k^4 + 16k^3 + 10k^2 + 3k + 1).
    \end{align*}
    If $n = 4k-1$, then:
    \begin{align*}
        n^4 + 4n^2 + 11 &= (4k - 1)^4 + 4(4k - 1)^2 + 11 \\
        &= 4^4k^4 - 4 \cdot 4^3k^3 + 6\cdot 4^2k^2 - 4\cdot 4k + 1 + 4\cdot 4^2 k^2 - 4\cdot 4 \cdot 2k + 4 + 11 \\
        &= 16(16k^4 - 16k^3 + 10k^2 - 3k + 1).
    \end{align*}
    Therefore, since it holds for all possible cases, it is clear that $n^4 + 4n^2 + 11$ is of the form $16k$.
\end{solution}
\section{The Greatest Common Divisor}

\begin{exercise}
    If $a \mid b$, show that $(-a) \mid b$, $a \mid (-b)$ and $(-a) \mid (-b)$. \\
\end{exercise}

\begin{solution}
    Since $a \mid b$, then there exists and integer $k$ such that $b = ka$. Rewriting this equation as $b = (-k)(-a)$ lets us conclude that $(-a) \mid b$. Similarly, we can multiply both sides by $-1$ to obtain the new equation $-b = -ka$. Interpreting this equation as $(-b) = (-k)a$ implies that $a \mid (-b)$, and interpreting it as $(-b) = k(-a)$ implies that $(-a) \mid (-b)$. \\
\end{solution}

\begin{exercise}
    Given integers $a$, $b$, $c$, $d$, verifiy that
    \begin{enumerate}
        \item if $a \mid b$, then $a \mid bc$;
        \item if $a \mid b$ and $a \mid c$, then $a^2 \mid bc$;
        \item $a \mid b$ if and only if $ac \mid bc$, where $c \neq 0$;
        \item if $a \mid b$ and $c \mid d$, then $ac \mid bd$.
    \end{enumerate}
\end{exercise}

\begin{solution}
    \begin{enumerate}
        \item Since $a \mid b$, then there exists an integer $k$ such that $b = ka$. Multiplying by $c$ on both sides of the previous equation implies $bc = (kc)a$ and so $a \mid bc$.
        \item Since $a \mid b$ and $a \mid c$, then there exist integers $k_1$ and $k_2$ such that $b = k_1a$ and $c = k_2 a$. Multiplying these two equations together gives us $bc = (k_1 k_2)a^2$ which implies that $a^2 \mid bc$.
        \item Suppose that $c$ is non-zero. By definition, $a \mid b$ if and only if $b = ka$ for some integer $k$. Since $c$ is non-zero, then this equation holds if and only if $bc = k(ac)$. Again, by definition, this equation holds if and only if $ac \mid bc$. Therefore, $a \mid b$ if and only if $ac \mid bc$.
        \item Since $a \mid b$ and $c \mid d$, then there exist integers $k_1$ and $k_2$ such that $b = k_1a$ and $d = k_2 c$. Multiplying these two equations together gives us $bd = (k_1 k_2)(ac)$ which implies that $ac \mid bd$.
    \end{enumerate}
\end{solution}

\begin{exercise}
    Prove or disprove: if $a \mid (b+c)$, then either $a \mid b$ or $a \mid c$. \\
\end{exercise}

\begin{solution}
    This is false because $2 \mid 1 + 1$ but $2$ does not divide 1.\\
\end{solution}

\begin{exercise}
    For $n \geq 1$, use mathematical induction to establish each of the following divisibility statements:
    \begin{enumerate}
        \item $8 \mid 5^{2n} + 7$;
        
        [\textit{Hint:} $5^{2k+1} + 7 = 5^2(5^{2k} + 7) + (7 - 5^2 \cdot 7).$]
        \item $15 \mid 2^{4n} - 1$;
        \item $5 \mid 3^{3n+1} + 2^{n+1}$;
        \item $21 \mid 4^{n+1} + 5^{2n-1}$;
        \item $24 \mid 2\cdot 7^n + 3\cdot 5^n - 5$.
    \end{enumerate}
\end{exercise}

\begin{solution}
    \begin{enumerate}
        \item When $n = 1$, we have that $5^{2n} + 7 = 25 + 7 = 32 = 8\cdot 4$ and so $8 \mid 5^{2n} + 7$. Suppose now that $8 \mid 5^{2k} + 7$ for some $k \geq 1$, then $8 \mid 5^2(5^{2k} + 7)$; and notice that $8 \mid 7(1 - 5^2)$ since $7(1 - 5^2) = 7 \cdot (-3) \cdot 8$, then 
        $$8 \mid 5^2(5^{2k} + 7) + 7(1 - 5^2) = 5^{2(k+1)} + 7.$$
        Since it also holds for $n = k+1$, then it holds for all $n \geq 1$ by induction.
        \item When $n = 1$, we have that $2^{4n} -1 = 16 -1 = 15\cdot 1$ and so $15 \mid 2^{4n} -1$. Suppose now that $15 \mid 2^{4k} - 1$ for some $k \geq 1$, then $15 \mid 2^4(2^{4k} - 1)$; and notice that $15 \mid 2^4 - 1$, then 
        $$15 \mid 2^4(2^{4k} - 1) + 2^4 - 1 = 2^{4(k+1)} - 1.$$
        Since it also holds for $n = k+1$, then it holds for all $n \geq 1$ by induction.
        \item When $n = 1$, we have that $3^{3n+1} + 2^{n+1} = 85 = 5\cdot 9$ and so $5 \mid 3^{3n+1} + 2^{n+1}$. Suppose now that $5 \mid 3^{3k+1} + 2^{k+1}$ for some $k \geq 1$, then $5 \mid 3^3(3^{3k+1} + 2^{k+1})$; and since $5 \mid 2 - 3^3 = -25$, then 
        $$5 \mid 3^3(3^{3k+1} + 2^{k+1}) + (2 - 3^3)2^{k+1} = 3^{3(k+1)+1} + 2^{(k+1)+1}.$$
        Since it also holds for $n = k+1$, then it holds for all $n \geq 1$ by induction.
        \item When $n = 1$, we have that $4^{n+1} + 5^{2n-1} = 16 + 5 = 21\cdot 1$ and so $21 \mid 4^{n+1} + 5^{2n-1}$. Suppose now that $21 \mid 4^{k+1} + 5^{2k-1}$ for some $k \geq 1$, then $21 \mid 4(4^{k+1} + 5^{2k-1})$; and since $21 \mid (5^2 - 1)5^{2k-1}$, then 
        $$21 \mid 4(4^{k+1} + 5^{2k-1}) + (5^2 - 4)5^{2k-1} = 4^{(k+1)+1} + 5^{2(k+1)-1}.$$
        Since it also holds for $n = k+1$, then it holds for all $n \geq 1$ by induction.
        \item Let's use the Second Principle of Mathematical Induction. First notice that when $n = 1$, we have
        $$2\cdot 7^n + 3\cdot 5^n - 5 = 14 + 15 - 5 = 24\cdot 1$$
        so it holds in that case. Moreover, when $n = 2$, then
        $$2\cdot 7^n + 3\cdot 5^n - 5 = 98 + 75 - 5 = 168 = 24 \cdot 7$$
        so it holds in that case as well. Suppose now that $24 \mid 2\cdot 7^q + 3\cdot 5^q - 5$ for all integers $q$ smaller than or equal to some integer $k \geq 2$. Notice that 
        \begin{align*}
            & 2\cdot 7^{k+1} + 3\cdot 5^{k+1} - 5 \\
            =& 7(2\cdot 7^k + 3\cdot 5^k - 5) - 7\cdot 3 \cdot 5^k + 5(2\cdot 7^k + 3\cdot 5^k - 5) - 2\cdot 5 \cdot 7^k + 55\\
            =& 12(2\cdot 7^k + 3\cdot 5^k - 5) - 35(2\cdot 7^{k-1} + 3\cdot 5^{k-1} - 5) - 24\cdot 5.
        \end{align*}
        But since by our assumption 24 divides $2\cdot 7^k + 3\cdot 5^k - 5$, $2\cdot 7^{k-1} + 3\cdot 5^{k-1} - 5$ and $24\cdot 5$, then 24 divides any linear combinations of these three terms. In particular, 24 divides the one above which is equal to $2\cdot 7^{k+1} + 3\cdot 5^{k+1} - 5$. Thus, the statement holds for the case $n = k+1$. Therefore, by induction, it holds for all $n \geq 1$.
    \end{enumerate}
\end{solution}

\begin{exercise}
    Prove that for any integer $a$ one of the integers $a$, $a+1$, $a+4$ is divisible by 3. [\textit{Hint:} By the Division Algorithm, the integer $a$ must be of the forms $3k$, $3k+1$, or $3k+2$.] \\
\end{exercise}

\begin{solution}
    Consider the three following cases: When $a = 3k$, then $a$ is obviously divisible by 3. When $a = 3k+1$, then $a + 2 = 3k+1 + 2 = 3(k+1)$ which makes it divisible by 3. Finally, when $a = 3k+2$, then $a + 4 = 3(k+2)$ and so it is divisible by 3. Therefore, in all possible cases for $a$, one of $a$, $a+2$, $a+4$ must be divisible by 3. \\
\end{solution}

\begin{exercise}
    For an arbitrary integer $a$, verify that
    \begin{enumerate}
        \item $2 \mid a(a+1)$, and $3 \mid a(a+1)(a+2)$;
        \item $3 \mid a(2a^2 + 7)$;
        \item if $a$ is odd, then $32 \mid (a^2 + 3)(a^2 + 7)$.
    \end{enumerate}
\end{exercise}

\begin{solution}
    \begin{enumerate}
        \item Consider the two following cases for $a$: when $a = 2k$, then $2 \mid a$ which implies that $2 \mid a(a+1)$. When $a = 2k+1$, then $2 \mid 2(k+1) = a+1$ and so $2 \mid a(a+1)$. Thus, $2 \mid a(a+1)$ for all integers $a$. Let's use the same technique for the second statement by considering the three following cases: when $a = 3k$, then $3 \mid a$ and so $3 \mid a(a+1)(a+2)$. When $a = 3k+1$, then $3 \mid 3(k+1) = a+2$ and so $3 \mid a(a+1)(a+2)$. When $a = 3k+1$, then $3 \mid 3(k+1) = a+1$ and so $3 \mid a(a+1)(a+2)$. Therefore, $3 \mid a(a+1)(a+2)$ for all integers $a$.
        \item Consider the three following cases: When $a = 3k$, then $3 \mid a$ and so $3 \mid a(2a^2 + 7)$. When $a = 3k+1$, then
        $$2a^2 + 7 = 2 \cdot 3^2k^2 + 4\cdot 3k + 2 + 7 = 3(6k^2 + 4k + 3)$$
        and so $3 \mid 2a^2 + 7$ which impliues that $3\mid a(2a^2 + 7)$. When $a = 3k+2$, then
        $$2a^2 + 7 = 2 \cdot 3^2k^2 + 4 \cdot 2\cdot 3k + 8 + 7 = 3(6k^2 + 8k + 5)$$
        and so $3 \mid 2a^2 + 7$ which impliues that $3\mid a(2a^2 + 7)$. Therefore, $3 \mid a(2a^2 + 7)$ for all integers $a$.
        \item Let $a$ be an odd integer, then $a$ must be of the form $2k+1$. It follows that
        \begin{align*}
            (a^2 + 3)(a^2 + 7) &= ((2k+1)^2 + 3)((2k+1)^2 + 7) \\
            &= (4k^2 + 4k + 4)(4k^2 + 4k + 8) \\
            &= 16(k^2 + k + 1)(k^2 + k + 2).
        \end{align*}
        Now, if we let $m = k^2 + k + 1$, then we already proved that 2 must divide $m(m+1)$ and so $(k^2 + k + 1)(k^2 + k + 2) = 2q$. Thus, 
        $$(a^2 + 3)(a^2 + 7) = 32q$$
        which implies that $32 \mid (a^2 + 3)(a^2 + 7)$.
    \end{enumerate}
\end{solution}

\begin{exercise}
    Prove that if $a$ and $b$ are both odd integers, then $16 \mid a^4 + b^4 - 2$. \\
\end{exercise}

\begin{solution}
    If $a$ and $b$ are both odd integers, then there exist integers $k$ and $q$ such that $a = 2k+1$ and $b = 2q+1$. It follows that
    \begin{align*}
        a^4 + b^4 - 2 &= (2k+1)^4 + (2q+1)^4 - 2 \\
        &= 2^4k^4 + 4\cdot 2^3k^3 + 6\cdot 2^2 k^2 + 4\cdot 2k \\
        & \quad + 2^4q^4 + 4\cdot 2^3q^3 + 6\cdot 2^2 q^2 + 4\cdot 2q \\
        &= 16(k^4 + q^4 + 2k^3 + 2q^3) + 8(3k^2 + k) + 8(3q^2 + q).
    \end{align*}
    Let's focus on the term $3k^2 + k$. If $k$ is even, then it follows that $3k^2 + k$ is even as well. If $k$ is odd, then $3k^2$ must also be odd. But then, $3k^2 + k$ is even since it is the sum of two odd numbers. Therefore, $3k^2 + k$ is even for all $k$. It follows that $3k^2 + k = 2k_0$ for some integer $k_0$. The same argument shows that $3q^2 + q = 2q_0$ for some integer $q_0$. Hence, we can rewrite the above equation as follows:
    $$a^4 + b^4 - 2 = 16(k^4 + q^4 + 2k^3 + 2q^3 + k_0 + q_0)$$
    from which we directly see that $16 \mid a^4 + b^4 - 2$.\\
\end{solution}

\begin{exercise}
    Prove that
    \begin{enumerate}
        \item the sum of the squares of two odd integers cannot be a perfect square;
        \item the product of four consecutive integers is 1 less than a perfect square.
    \end{enumerate}
\end{exercise}

\begin{solution}
    \begin{enumerate}
        \item Let $2k+1$ and $2q+1$ be two odd integers, then the sum of their squares
        $$(2k+1)^2 + (2q+1)^2 = 4k^2 + 4k + 1 + 4q^2 + 4q + 1 = 4(k^2 + q^2 + k + q) + 2$$
        is of the form $4m + 2$. However, we already proved that perfect squares must have the forms $4n$ or $4n+1$. Therefore, the sum of two odd integers cannot be a square.
        \item Let $a$ be an integer, then
        \begin{align*}
            a(a+1)(a+2)(a+3) &= a(a^2 + 3a + 2)(a+3) \\
            &= a(a^3 + 6a^2 + 11a + 6) \\
            &= [a^4 + 6a^3 + 11a^2 + 6a + 1] - 1 \\
            &= (a^2 + 3a + 1)^2 - 1.
        \end{align*}
        Since it holds for all integers $a$, then the product of any four consecutive integers is 1 less than a square.
    \end{enumerate}
\end{solution}

\begin{exercise}
    Establish that the difference of two consecutive cubes is never divisible by 2. \\
\end{exercise}

\begin{solution}
    Let $a$ be an integer, then
    $$(a+1)^3 - a^3 = 3(a^2 + a) + 1.$$
    Since taking the square of a number preserves its parity, then $a^2$ and $a$ must have the same parity. It follows that their sum must be even and so $a^2 + a = 2k$. Thus:
    $$(a+1)^3 - a^3 = 2(3k) + 1$$
    which implies that the difference of two cubes is always odd. \\
\end{solution}

\begin{exercise}
    For a nonzero integer $a$, show that $\gcd(a,0) = |a|$, $\gcd(a,a) = |a|$, and $\gcd(a,1) = 1$. \\
\end{exercise}

\begin{solution}
    We know that the greatest common divisor of $a$ and $b$ can be intepreted as the smallest positive linear combination of $a$ and $b$. But since the positive linear combinations of $a$ and $0$ are precisely the positive multiples of $a$, then it follows that $\gcd(a, 0) = |a|$ since $|a|$ is the least positive multiple of $a$. Similarly, the positive linear combinations of $a$ and $a$ are precisely the positive multiples of $a$ and so $\gcd(a,a) = |a|$ for the same reasons. Finally, since 
    $$a\cdot 0 + 1\cdot 1 = 1,$$
    then $\gcd(a,1) = 1$ by Theorem 2-4. \\
\end{solution}

\begin{exercise}
    If $a$ and $b$ are integers, not both of which are zero, verify that
    $$\gcd(a,b) = \gcd(-a, b) = \gcd(a, -b) = \gcd(-a, -b).$$
\end{exercise}

\begin{solution}
    Notice that
    $$(-a)(-x) + by = ax + (-b)(-y) = (-a)(-x) + (-b)(-y)$$
    implies that the set of positive linear combinations of $a$ and $b$ is precisely equal to the set of linear combinations of $-a$ and $b$, $a$ and $-b$, and $-a$ and $-b$. Therefore, it follows that
    $$\gcd(a,b) = \gcd(-a, b) = \gcd(a, -b) = \gcd(-a, -b).$$
\end{solution}

\begin{exercise}
    Prove that, for a positive integer $n$ and any integer $a$, $\gcd(a, a+n)$ divides $n$; hence, $\gcd(a, a+1) = 1$. \\
\end{exercise}

\begin{solution}
    First, we know that $\gcd(a, a+n)$ must divide any linear combination of $a$ and $a+n$. In particular, it must divide
    $$a\cdot(-1) + (a+n)\cdot 1 = n.$$ 
\end{solution}

\begin{exercise}
    Given integers $a$ and $b$, prove that
    \begin{enumerate}
        \item there exist integers $x$ and $y$ for which $c = ax+by$ if and only if $\gcd(a,b) \mid c$.
        \item if there exist integers $x$ and $y$ for which $ax + by = \gcd(a,b)$, then $\gcd(x,y) = 1$.
    \end{enumerate}
\end{exercise}

\begin{solution}
    \begin{enumerate}
        \item If $c$ is a linear combination of $a$ and $b$, then $\gcd(a,b)$ divides it since it divides both $a$ and $b$. If $\gcd(a,b) \mid c$, then $c = k \cdot \gcd(a,b)$. But since there exist integers $x_0$ and $y_0$ such that $\gcd(a,b) = ax_0 + by_0$, then $c = a(kx_0) + b(ky_0)$. Therefore, $c$ is a linear combination of $a$ and $b$ if and only if it is a multiple of $\gcd(a,b)$.
        \item We know that if $\gcd(a,b) = ax+by$, then $ax + by$ is the least positive linear combination of $a$ and $b$. By contradiction, if $\gcd(x,y) \neq 1$, then $\gcd(x,y) > 1$ and so
        $$0< a\left(\frac{x}{\gcd(x,y)}\right) + b\left(\frac{y}{\gcd(x,y)}\right) < ax+by$$
        which contradicts the fact that $ax+ by$ is the smallest positive linear combination. Therefore, $\gcd(x,y) = 1$.
    \end{enumerate}
\end{solution}

\begin{exercise}
    For any integer $a$, show that
    \begin{enumerate}
        \item $\gcd(2a+1, 9a+4) = 1$;
        \item $\gcd(5a+2, 7a+3) = 1$;
        \item if $a$ is odd, then $\gcd(3a, 3a+2) = 1$.
    \end{enumerate}
\end{exercise}

\begin{solution}
    \begin{enumerate}
        \item It suffices to notice that
        $$(2a+1)\cdot 5 + (9a + 4)\cdot (-1) = 1.$$
        \item It suffices to notice that
        $$(5a+2)\cdot (-4) + (7a + 3)\cdot 3 = 1.$$
        \item We know that $\gcd(3a, 3a+2)$ divides any linear combination of $3a$ and $3a+2$. In particular, it must divide
        $$3a \cdot (-1) + (3a+2) \cdot 1 = 2.$$
        Hence, $\gcd(3a, 3a+2)$ is either 1 or 2. By contradiction, if $\gcd(3a, 3a+2) = 2$, then $2 \mid 3a$. Moreover, since $\gcd(2,3) = 1$, then $2 \mid 3a$ implies that $2 \mid a$ which is impossible since $a$ is odd. Therefore, $\gcd(3a, 3a+2) = 1$.
    \end{enumerate}
\end{solution}

\begin{exercise}
    If $a$ and $b$ are integers, not both of which are zero, prove that $\gcd(2a - 3b, 4a - 5b)$ divides $b$; hence, $\gcd(2a+3, 4a+5) = 1$. \\
\end{exercise}

\begin{solution}
    Since $\gcd(2a - 3b, 4a - 5b)$ divides all linear combinations of $2a - 3b$ and $4a - 5b$, then it divides 
    $$(2a - 3b)\cdot (-2) + (4a - 5b)\cdot 1 = b.$$
\end{solution}

\begin{exercise}
    Given an odd integer $a$, establish that
    $$a^2 + (a+2)^2 + (a+4)^2 + 1$$
    is divisible by 12. \\
\end{exercise}

\begin{solution}
    Since $a$ is odd, then $a = 2k+1$ for some integer $k$. Thus:
    \begin{align*}
        a^2 + (a+ 2)^2 + (a+4)^2 + 1 &= (2k+1)^2 + (2k+3)^2 + (2k+5)^2 + 1 \\
        &= 12k^2 + 36k + 36 \\
        &= 12(k^2 + 3k + 3).
    \end{align*}
    Therefore, $12 \mid a^2 + (a+2)^2 + (a+4)^2 + 1$.\\
\end{solution}

\begin{exercise}
    Prove that $(2n)!/n!(n+1)!$ is an integer for all $n \geq 0$.

    [\textit{Hint:} Note that $\displaystyle \binom{2n }{n }(2n+1) = \binom{2n+1}{n+1}(n+1).$] \\
\end{exercise}

\begin{solution}
    Using the definition of the binomial coefficients, we have that 
    $$\frac{(2n)!}{n!(n+1)!} = \frac{1}{n+1}\binom{2n}{n }.$$
    Hence, it suffices to show that $n+1$ divides $\displaystyle \binom{2n}{n }$. But since
    $$\binom{2n}{n }(2n+1) = \binom{2n+1}{n+1}(n+1),$$
    then by definition, $n+1$ divides $\displaystyle \binom{2n}{n }(2n+1)$. However, from the fact that
    $$2(n+1) - (2n+1) = 1,$$
    we have that $\gcd(2n+1, n+1) = 1$ which lets us conclude, by Euclid's Lemma, that $n+1$ divides $\displaystyle \binom{2n }{n}$. Therefore, $(2n)!/n!(n+1)!$ is an integer. \\
\end{solution}

\begin{exercise}
    Prove: the product of any three consecutive integers is divisible by 6; the product of any four consecutive integers is divisible by 24; the product of any five consecutive integers is divisible by 120. [\textit{Hint:} See Corollary 2 to Theorem 2-4.]\\
\end{exercise}

\begin{solution}
    First, we know that for any integer $a$, one of $a$ and $a+1$ is divisible by 2 by considering the cases where $a$ is even and odd. Similarly, we know that one of $a$, $a+1$, $a+2$ must be divisible by 3 by considering the cases $a = 3k$, $a = 3k+1$, $a = 3k+2$. In the same way, one of $a$, $a+1$, $a+2$, $a+3$ is divisible by 4 and another of them is divisible by 2 which makes the product of the four factors divisible by 8. Finally, As we did above, we can easily prove that one of $a$, $a+1$, $a+2$, $a+3$, $a+4$ is divisible by 5. Hence, it follows from Corollary 2 that the product of three consecutive factors is divisble by 6 since $\gcd(2, 3) = 1$; the product of four factors is divisible by 24 since $\gcd(3, 8) = 1$; and the product of five factors is divisible by 120 since $\gcd(24, 5) = 1$. \\
\end{solution}

\begin{exercise}
    Establish each of the assertions below:
    \begin{enumerate}
        \item If $a$ is an arbitrary integer, then $6 \mid a(a^2 + 11)$.
        \item If $a$ is an odd integer, then $24 \mid a(a^2 - 1)$. [\textit{Hint:} The square of an odd integer is of the form $8k+1$.]
        \item If $a$ and $b$ are odd integers, then $8 \mid (a^2 - b^2)$.
        \item If $a$ is an integer not divisible by $2$ or $3$, then $24 \mid (a^2 + 23)$. [\textit{Hint:} Any integer $a$ must assume one of the forms $6k$, $6k+1$, ..., $6k+5$.]
    \end{enumerate}
\end{exercise}

\begin{solution}
    \begin{enumerate}
        \item Let's split the proof in six cases. If $a = 6k$, then $6 \mid a$ and so $6 \mid a(a^2 + 11)$. If $a = 6k+1$, then
        $$a^2 + 11 = 6^2k^2 + 2 \cdot 6k + 12 = 6(6k^2 + 2k + 2)$$
        and so $6\mid a(a^2 + 11)$. If $a = 6k+2$, then 
        $$a(a^2 + 11) = (6k+2)(6^2k^2 + 4\cdot 6k + 15) = 6(3k+1)(12k^2 + 8k + 5)$$
        and so $6\mid a(a^2 + 11)$. If $a = 6k+3$, then 
        $$a(a^2 + 11) = (6k+3)(6^2k^2 + 6^2k + 20) = 6(2k+1)(18k^2 + 18k + 10)$$
        and so $6\mid a(a^2 + 11)$. If $a = 6k+4$, then 
        $$a(a^2 + 11) = (6k+4)(6^2k^2 + 8\cdot 6k + 27) = 6(3k+2)(12k^2 + 16k + 9)$$
        and so $6\mid a(a^2 + 11)$. If $a = 6k+5$, then 
        $$a(a^2 + 11) = a(6^2k^2 + 10\cdot 6k + 36) = 6a(6k^2 + 10k + 6)$$
        and so $6\mid a(a^2 + 11)$. Therefore, $6 \mid a(a^2 + 11)$ for all integers $a$.
        \item First, rewrite $a(a^2 - 1)$ as $(a-1)a(a+1)$ which shows that it is the product of three succesive integers. Hence, it must be divisible by 3. Since $a$ is odd, then $a-1$ is even and so $(a-1)(a+1)$ is of the form $m(m+2)$ where $m$ is even. By considering the cases $m = 4k$ and $m = 4k+2$, we get that $(a-1)(a+1)$ must be divisible by 8. Thus, $a(a^2 - 1)$ is divisible by both 8 and 3. Since $\gcd(8, 3) = 1$, then $24 \mid a(a^2 - 1)$.
        \item If $a = 2k+1$ and $b = 2q+1$, then
        $$a^2 - b^2 = (2k - 2q)(2k + 2q + 2) = 4(k - q)(k + q + 1).$$
        If $k$ and $q$ have the same parity, then $k - q$ is even and so $a^2 - b^2 = 8k_0$. If $k$ and $q$ have distinct parities, then $k + q + 1$ is even and so $a^2 - b^2 = 8k_0$. Thus, in all possible cases, $8 \mid (a^2 - b^2)$.
        \item If $a$ is not divisble by 2 or by 3, then it must have the form $6k+1$ or $6k + 5$. In the case $a = 6k + 1$, we have 
        $$a^2 + 23 = 36k^2 + 12k + 24 = 12(3k^2 + k) + 24.$$
        Since $3k^2$ has the same parity as $k$, then $3k^2 + k$ must be even, and so it can be written as $2k_0$. Hence, $a^2 + 23 = 24(k_0 + 1)$ which implies that $24 \mid (a^2 + 23)$. Next, if $a = 6k + 5$, then equivalently, it has the form $a = 6q - 1$. In that case,
        $$a^2 + 23 = 36k^2 - 12k + 24 = 12(3k^2 - k) + 24.$$
        Using the same argument as above, $3k^2 - k = 2k_0$ and so $a^2 + 23 = 24(k_0 + 1)$ which proves that $24 \mid (a^2 + 23)$.
    \end{enumerate}
\end{solution}

\begin{exercise}
    Confirm the following properties of the greatest common divisor:
    \begin{enumerate}
        \item If $\gcd(a,b) = 1$, and $\gcd(a,c) = 1$, then $\gcd(a, bc) = 1$. [\textit{Hint:} Since $1 = ax + by = au + cv$ for some $x,y,u,v$, $1 = (ax + by)(au + cv) = a(aux + byu) + bc(yv)$.]
        \item If $\gcd(a,b) = 1$, and $c\mid a$, then $\gcd(b,c) = 1$.
        \item If $\gcd(a,b) = 1$, then $\gcd(ac,b) = \gcd(c,b)$.
        \item If $\gcd(a,b) = 1$, and $c \mid a+b$, then $\gcd(a,c) = \gcd(b,c) = 1$. [\textit{Hint:} Let $d = \gcd(a,c)$. Then $d \mid a$, $d \mid c$ implies that $d \mid (a+b) - a$, or $d\mid b$.]
        \item If $\gcd(a,b) = 1$, $d \mid ac$, and $d \mid bc$, then $d \mid c$.
        \item If $\gcd(a,b) = 1$, then $\gcd(a^2, b^2) = 1$. [\textit{Hint:} First show that $\gcd(a^2, b) = \gcd(a, b^2) = 1$.] 
    \end{enumerate}
\end{exercise}

\begin{solution}
    \begin{enumerate}
        \item We know that there exist integers $x,y,u,v$ such that $ax + by = 1$ and $au + cv = 1$. Multiplying these two equations gives us $a(aux + byu) + bc(yv) = 1$ and so $\gcd(a, bc) = 1$.
        \item Since $\gcd(a,b) = 1$, then there exist integers $x$ and $y$ such that $ax + by = 1$. Since $c \mid a$, then there exists an integer $k$ such that $a = kc$. Replacing the value of $a$ with this new expression in the linear combination gives us $c(kx) + by = 1$. Therefore, $\gcd(c, b) = 1$.
        \item Since $\gcd(c,b)$ divides both $ac$ and $b$, then it divides $\gcd(ac, b)$. Conversely, $\gcd(ac, b)$ divides $b$. Moreover, since $\gcd(ac, b)$ divides $b$ and $\gcd(a, b) = 1$, then $\gcd(a, \gcd(ac, b)) = 1$. It follows that from the fact that $\gcd(ac, b) \mid ac$, we get that $\gcd(ac, b) \mid c$. Thus, $\gcd(ac, b) \mid \gcd(c,b)$ since it divides both $c$ and $b$. Therefore, $\gcd(ac, b) = \gcd(c,b)$ since both divide the other and both are positive.
        \item Let $d_a = \gcd(a,c)$, then by definition, $d_a \mid a$ and $d_a \mid c$. From the fact that $c \mid a + b$, we get that $d_a$ divides both $a$ and $a+b$. It follows that $d_a \mid (a+b) - a = b$. Since it divides both $a$ and $b$, then it divides $\gcd(a,b) = 1$. Therefore, $\gcd(a,c) = d_a = 1$. The proof is strictly the same for $d_b = \gcd(b,c)$.
        \item If $\gcd(a,b) = 1$, then there exist integers $x$ and $y$ such that $ax+by = 1$. Since $d$ divides both $ac$ and $bc$, then it divides any of their linear combinations. In particular, $d$ divides
        $$acx + bcy = c(ax + by) = c.$$
        \item Using part (c) of this exercise with $c = a$, we have that $\gcd(a^2, b) = \gcd(a,b) = 1$. Similarly, if we now apply part (c) with $a = b$, $b = a^2$ and $c = b$, we obtain $\gcd(a^2, b^2) = \gcd(a^2, b) = 1$.
    \end{enumerate}
\end{solution}

\begin{exercise}
    Prove that if $d \mid n$, then $2^d - 1 \mid 2^n - 1$. [\textit{Hint:} Employ the identity $x^k - 1 = (x-1)(x^{k-1} + x^{k-2} + \dots + 1).$] \\
\end{exercise}

\begin{solution}
    Since $d \mid n$, then $n = dk$ for some integer $k$. It follows that
    $$2^n - 1 = \frac{(2^d)^k - 1}{2^d - 1}(2^d - 1) = ((2^d)^{k-1} + \dots + 1)(2^d - 1).$$
    Therefore, $2^d - 1 \mid 2^n - 1$.
\end{solution}
\section{The Euclidean Algorithm}

\begin{exercise}
    Find $\gcd(143, 227)$, $\gcd(306, 657)$ and $\gcd(272, 1479)$. \\
\end{exercise}

\begin{solution}
    Let's apply the Euclidean Algorithm:
    \begin{align*}
        227 &= 1\cdot 143 + 84 \\
        143 &= 1\cdot 84 + 59 \\
        84 &= 1\cdot 59 + 25 \\
        59 &= 2 \cdot 25 + 9 \\
        25 &= 2 \cdot 9 + 7 \\
        9 &= 1 \cdot 7 + 2 \\
        7 &= 3 \cdot 2 + 1 \\
        2 &= 2\cdot1 + 0
    \end{align*}
    and so $\gcd(143, 227) = 1$. 
    \begin{align*}
        657 &= 2\cdot 306 + 45 \\
        306 &= 6\cdot 45 + 36 \\
        45 &= 1\cdot 36 + 9 \\
        36 &= 4 \cdot 9 + 0
    \end{align*}
    and so $\gcd(306, 657) = 9$. 
    \begin{align*}
        1479 &= 5\cdot 272 + 119 \\
        272 &= 2\cdot 119 + 34 \\
        119 &= 3\cdot 34 + 17 \\
        34 &= 2 \cdot 17 + 0
    \end{align*}
    and so $\gcd(272, 1479) = 17$. \\ 
\end{solution}

\begin{exercise}
    Use the Euclidean Algorithm to obtain integers $x$ and $y$ satisfying 
    \begin{enumerate}
        \item $\gcd(56, 72) = 56 x + 72 y$;
        \item $\gcd(24, 138) = 24 x + 138 y$;
        \item $\gcd(119, 272) = 119 x + 272 y$;
        \item $\gcd(1769, 2378) = 1769 x + 2378 y$;
    \end{enumerate}
\end{exercise}

\begin{solution}
    \begin{enumerate}
        \item First, let's apply the Euclidean Algorithm:
        \begin{align*}
            72 &= 1\cdot 56 + 16 \\
            56 &= 3\cdot 16 + 8 \\
            16 &= 2 \cdot 8 + 0.
        \end{align*}
        Now, running these equations backward gives us 
        \begin{align*}
            8 &= 56 - 3 \cdot 16 \\
            &= 56 - 3(72 - 56) \\
            &= 4 \cdot 56 - 3 \cdot 72.
        \end{align*}
        Thus, $x = 4$ and $y = -3$.
        \item First, let's apply the Euclidean Algorithm:
        \begin{align*}
            138 &= 5\cdot 24 + 18 \\
            24 &= 1\cdot 18 + 6 \\
            18 &= 3 \cdot 6 + 0.
        \end{align*}
        Now, running these equations backward gives us 
        \begin{align*}
            6 &= 24 - 18 \\
            &= 24 - (138 - 5 \cdot 24) \\
            &= 6\cdot 24 - 138
        \end{align*}
        Thus, $x = 6$ and $y = -1$.
        \item First, let's apply the Euclidean Algorithm:
        \begin{align*}
            272 &= 2\cdot 119 + 34 \\
            119 &= 3\cdot 34 + 17 \\
            34 &= 2 \cdot 17 + 0.
        \end{align*}
        Now, running these equations backward gives us 
        \begin{align*}
            17 &= 119 - 3\cdot 34 \\
            &= 119 - 3(272 - 2\cdot 119) \\
            &= 7\cdot 119 - 3\cdot 272
        \end{align*}
        Thus, $x = 7$ and $y = -3$.
        \item First, let's apply the Euclidean Algorithm:
        \begin{align*}
            2378 &= 1\cdot 1769 + 610 \\
            1769 &= 2\cdot 610 + 549 \\
            610 &= 1 \cdot 549 + 61 \\
            549 &= 9 \cdot 61 + 0.
        \end{align*}
        Now, running these equations backward gives us 
        \begin{align*}
            61 &= 610 - 549 \\
            &= 610 - (1769 - 2 \cdot 610) \\
            &= 3\cdot 610 - 1769 \\
            &= 3(2378 - 1769) - 1769 \\
            &= 3 \cdot 2378 - 4 \cdot 1769.
        \end{align*}
        Thus, $x = -4$ and $y = 3$.
    \end{enumerate}
\end{solution}

\begin{exercise}
    Prove that if $d$ is a common divisor of $a$ and $b$, then $d = \gcd(a,b)$ if and only if $\gcd(a/d, b/d) = 1$. [\textit{Hint:} Use Theorem 2-7.] \\
\end{exercise}

\begin{solution}
    Suppose that $d = \gcd(a,b)$ and write $d = ax + by$, then dividing both sides by $d$ gives us $1 = (a/d)x + (b/d)y$. Thus, $\gcd(a/d, b/d) = 1$. Suppose now that $\gcd(a/d, b/d) = 1$, then by multiplying both sides by $d$, we get
    $$\gcd(a, b) = d \gcd(a/d, b/d) = d.$$
\end{solution}

\begin{exercise}
    Assuming that $\gcd(a,b) = 1$, prove the following:
    \begin{enumerate}
        \item $\gcd(a+b, a-b) = 1$ or 2.
        
        [\textit{Hint:} Let $d = \gcd(a+b, a-b)$ and show that $d \mid 2a$, $d\mid 2b$; thus, that $d \leq \gcd(2a, 2b) = 2\gcd(a,b)$.]
        \item $\gcd(2a + b, a + 2b) = 1$ or 3.
        \item $\gcd(a + b, a^2 + b^2) = 1$ or 2.
        
        [\textit{Hint:} $a^2 + b^2 = (a + b)(a - b) + 2b^2$.]
        \item $\gcd(a + b, a^2 - ab + b^2) = 1$ or 3.
        
        [\textit{Hint:} $a^2 - ab + b^2 = (a + b)^2 - 3ab$.]
    \end{enumerate}
\end{exercise}

\begin{solution}
    \begin{enumerate}
        \item Let $d = \gcd(a + b, a - b)$, then $d \mid a + b$ and $d \mid a - b$. It follows that $d \mid (a + b) + (a - b) = 2a$ and $d \mid (a + b) - (a - b) = 2b$. Hence, $d \leq \gcd(2a, 2b) = 2\gcd(a,b) = 2$. It follows that $\gcd(a + b, a - b) = d$ is either 1 or 2.
        \item Let $d = \gcd(2a + b, a + 2b)$, then $d \mid 2a + b$ and $d \mid a + 2b$. It follows that $d \mid 2(a + 2b) - (2a+b) = 3b$ and $d \mid 2(2a + b) - (a + 2b) = 3a$. Hence, $d \mid \gcd(3a, 3b) = 3\gcd(a,b) = 3$. Therefore, $d$ is either 1 or 3.
        \item Let $d = \gcd(a + b, a^2 + b^2)$, then $d \mid a + b$ and $d \mid a^2 + b^2$. It follows that $d \mid (a^2 + b^2) - (a+b)(a-b) = 2b^2$ and $d \mid 2(a^2 + b^2) - 2b^2 = 2a^2$. Hence, $d \mid \gcd(2a^2, 2b^2) = 2\gcd(a^2,b^2) = 2$ (Exercise 2.2.20(f)). Therefore, $d$ is either 1 or 2.
        \item Let $d = \gcd(a+b, a^2 - ab + b^2)$ and recall that $\gcd(a,b) = 1 \implies \gcd(a^2, b^2)$. Since $d \mid a+ b$ and $d \mid a^2 - ab + b^2$, then $d \mid (a+b)^2 - (a^2 - ab + b^2) = 3ab$. But since $d \mid 3a(a+b)$ and $d \mid 3ab$, we get that $d \mid 3a^2 + 3ab - 3ab = 3a^2$. Similarly, since $d \mid 3b(a+b)$ and $d \mid 3ab$, we get that $d \mid 3ab + 3b^2 - 3ab = 3b^2$. Thus, $d$ divides both $3a^2$ and $3b^2$ and so $d \mid \gcd(3a^2, 3b^2) = 3\gcd(a,b) = 3$. Therefore, $d = 1$ or $d = 3$.
    \end{enumerate}
\end{solution}

\begin{exercise}
    For positive integers $a$, $b$ and $n \geq 1$, show that
    \begin{enumerate}
        \item If $\gcd(a,b) = 1$, then $\gcd(a^n, b^n) = 1$. [\textit{Hint:} See Problem A°(a), Section 2.2.]
        \item The relation $a^n \mid b^n$ implies that $a \mid b$. [\textit{Hint:} Put $d = \gcd(a,b)$ and write $a = rd$, $b = sd$, where $\gcd(r,s) = 1$. By part (a), $\gcd(r^n, s^n) = 1$. Show that $r = 1$, whence $a = d$.]
    \end{enumerate}
\end{exercise}

\begin{solution}
    \begin{enumerate}
        \item First, let's prove by induction that if $\gcd(c_1, c_2) = 1$, then $\gcd(c_1,c_2^n) = 1$ for all $n \geq 1$. When $n = 1$, it holds from our assumption. Suppose now that $\gcd(c_1, c_2^k) = 1$ for some integer $k \geq 1$, then using the fact that $\gcd(c_1,c_2) = 1$ and Exercise 2.2.20(a), we get that $\gcd(c_1,c_2^{k+1}) = 1$. Thus, by induction, $\gcd(c_1,c_2^n) = 1$ for all $n \geq 1$. Taking $c_1 = a$ and $c_2 = b$, we get that $\gcd(a, b^n) = 1$ for all $n \geq 1$. Fixing $n \geq 1$ and taking now $c_1 = b^n$ and $c_2 = a$, we get that $\gcd(a^m, b^n) = 1$ for all $m \geq 1$. In particular, if we take $m = n$, we get that $\gcd(a^n, b^n) = 1$.
        \item Suppose that $a^n \mid b^n$, then $\gcd(a^n, b^n) = a^n$. Let $d = \gcd(a,b)$, then there exist relatively prime integers $r$ and $s$ such that $a = rd$ and $b = sd$. Since $\gcd(r, s) = 1$, then $\gcd(r^n, s^n) = 1$ by part (a). It follows that from the equations $a^n = r^n d^n$ and $b^n = s^n d^n$, since $r^n$ and $s^n$ are relatively prime, then $d^n = \gcd(a^n, b^n) = a^n = r^n d^n$. By cancelling out the $d^n$'s on both sides we get $r^n = 1$. Since both $a$ and $d$ are positive, then $r$ must be positive as well from the equation $a = rd$. Hence, from $r^n = 1$ we conclude that $r = 1$. Thus, $a = d = \gcd(a,b) \mid b$.
    \end{enumerate}
\end{solution}

\begin{exercise}
    Prove that if $\gcd(a,b) = 1$, then $\gcd(a + b, ab) = 1$. \\
\end{exercise}

\begin{solution}
    Let $d = \gcd(a + b, ab)$, then $d \mid a+b$ and $d \mid ab$. It follows that $d \mid a(a+b) - ab = a^2$. Similarly, $d \mid b(a+b) - ab = b^2$. Thus, $d \mid \gcd(a^2, b^2) = 1$ since it divides both $a^2$ and $b^2$.\\
\end{solution}

\begin{exercise}
    For nonzero integers $a$ and $b$, verify that the following conditions are equivalent:
    $$(\text{a}) \ a \mid b \qquad (\text{b}) \ \gcd(a,b) = |a| \qquad (\text{c}) \ \text{lcm}(a,b) = |b|$$
\end{exercise}

\begin{solution}
    Suppose that $a \mid b$, then $|a|$ divides both $a$ and $b$. Since any divisor of $a$ must divide $|a|$, then it follows that $\gcd(a,b) = |a|$.

    Suppose that $\gcd(a,b) = |a|$, then the equation $\gcd(a,b) \lcm(a,b) = |a| \cdot |b|$ becomes $\lcm(a,b) = |b|$.

    Suppose that $\lcm(a,b) = |b|$, then $|b|$ is a multiple of $a$. Equivalently, $b$ is a multiple of $a$ which is another way of saying that $a \mid b$. \\
\end{solution}

\begin{exercise}
    Find $\lcm(143, 227)$, $\lcm(306, 657)$ and $\lcm(272, 1479)$. \\
\end{exercise}

\begin{solution}
    For each of these, let's find their greatest common divisor first using the Euclidean Algorithm.
    \begin{align*}
        227 &= 1 \cdot 143 + 84 \\
        143 &= 1 \cdot 84 + 59 \\
        84 &= 1 \cdot 59 + 25 \\
        59 &= 2 \cdot 25 + 9 \\
        25 &= 2 \cdot 9 + 7 \\
        9 &= 1\cdot 7 + 2 \\
        7 &= 3 \cdot 2 + 1 \\
        2 &= 2 \cdot 1 + 0
    \end{align*}
    which shows that $\gcd(143, 227) = 1$. It follows that
    $$\lcm(143, 227) = 143 \cdot 227 = 32461.$$
    Let's apply the same procedure to find $\lcm(306, 657)$:
    \begin{align*}
        657 &= 2 \cdot 306 + 45 \\
        306 &= 6 \cdot 45 + 36 \\
        45 &= 1 \cdot 36 + 9 \\
        36 &= 4 \cdot 9 + 0.
    \end{align*}
    Hence, $\gcd(306, 657) = 9$. It follows that
    $$\lcm(306, 657) = \frac{306 \cdot 657}{9} = 306 \cdot 73 = 22338.$$
    Let's apply the same procedure to find $\lcm(306, 657)$:
    \begin{align*}
        657 &= 2 \cdot 306 + 45 \\
        306 &= 6 \cdot 45 + 36 \\
        45 &= 1 \cdot 36 + 9 \\
        36 &= 4 \cdot 9 + 0.
    \end{align*}
    Hence, $\gcd(306, 657) = 9$. It follows that
    $$\lcm(306, 657) = \frac{306 \cdot 657}{9} = 306 \cdot 73 = 22338.$$
    Let's apply the same procedure to find $\lcm(272, 1479)$:
    \begin{align*}
        1479 &= 5\cdot 272 + 119 \\
        272 &= 2\cdot 119 + 34 \\
        119 &= 3\cdot 34 + 17 \\
        34 &= 2 \cdot 17 + 0
    \end{align*}
    Hence, $\gcd(272, 1479) = 17$. It follows that
    $$\lcm(272, 1479) = \frac{272 \cdot 1479}{17} = 16 \cdot 1479 = 23664.$$
\end{solution}

\begin{exercise}
    Prove that the greatest common divisor of two positive integers always divides their least common multiple. \\
\end{exercise}

\begin{solution}
    Let $a$ and $b$ be two positive integers, then $\gcd(a,b)$ divides $a$ which in turns divides $\lcm(a,b)$. Hence, by transitivity, $\gcd(a,b) \mid \lcm(a,b)$.\\
\end{solution}

\begin{exercise}
    Given nonzero integers $a$ and $b$, establish the following facts concerning $\lcm(a,b)$:
    \begin{enumerate}
        \item $\gcd(a,b) = \lcm(a,b)$ if and only if $a = b$.
        \item If $k > 0$, then $\lcm(ka, kb) = k\lcm(a,b)$.
        \item If $m$ is any common multiple of $a$ and $b$, then $\lcm(a,b) \mid m$.
        
        [\textit{Hint:} Put $t = \lcm(a,b)$ and use the Division Algorithm to write $m = qt + r$, where $0 \leq r < t$. Show that $r$ is a common multiple of $a$ and $b$.]
    \end{enumerate}
\end{exercise}

\begin{solution}
    \begin{enumerate}
        \item Suppose that $\gcd(a,b) = \lcm(a,b)$. Since $\gcd(a,b) \mid a$ and $a \mid \lcm(a,b) = \gcd(a,b)$, then $a = \gcd(a,b)$. Similarly, since $\gcd(a,b) \mid b$ and $b \mid \lcm(a,b) = \gcd(a,b)$, then $\gcd(a,b) = b$. Therefore, $a = \gcd(a,b) = b$. Conversely, if $a = b$, then $\gcd(a,b) = a = b$ and $\lcm(a,b) = a = b$ which shows that $\lcm(a,b) = \gcd(a,b)$.
        \item Let $k > 0$ be an integer, then from the formula of the least common multiple in terms of the greatest common divisor, we obtain:
        $$\lcm(ka, kb) = \frac{k^2ab}{\gcd(ka, kb)} = k\frac{ab}{\gcd(a,b)} = k \lcm(a,b).$$
        \item Suppose that $m$ is a common multiple of $a$ and $b$, then by the Division Algroithm, there exist integers $q$ and $r$ such that $m = q\lcm(a,b) + r$ and $0 \leq r < \lcm(a,b)$. Suppose that $r\neq 0$ and notice that $r = m - q\lcm(a,b)$ must be divisible by both $a$ and $b$ since $a$ and $b$ divide both $m$ and $\lcm(a,b)$, it follows that $\lcm(a,b) \leq r$ contradicting the fact that $r < \lcm(a,b)$. Thus, $r = 0$ and so $\lcm(a,b) \mid m$.
    \end{enumerate}
\end{solution}

\begin{exercise}
    Let $a,b,c$ be integers, no two of which are zero, and $d = \gcd(a,b,c)$. Show that 
    $$d = \gcd(\gcd(a,b), c) = \gcd(a, \gcd(b,c)) = \gcd(\gcd(a,c), b).$$
\end{exercise}

\begin{solution}
    Let $d_0 = \gcd(\gcd(a,b), c)$, then $d_0 \mid \gcd(a,b)$ and $d_0 \mid c$. Since $\gcd(a,b)$ divides $a$ and $b$, then $d_0$ also divides $a$ and $b$. It follows that $d_0$ is a common divisor of $a$, $b$ and $c$. To show that it is the greatest, let $e$ be a common divisor of $a$, $b$ and $c$, since $e$ divides both $a$ and $b$, then it must divide $\gcd(a,b)$. Hence, $e$ divides both $\gcd(a,b)$ and $c$ which implies that $e \leq \gcd(\gcd(a,b), c)$. Therefore, $d = d_0$. The proofs of the other equalities are strictly the same. \\
\end{solution}

\begin{exercise}
    Find integers $x$, $y$, $z$ satisfying
    $$\gcd(198, 288, 512) = 198x + 288y + 512z.$$
    [\textit{Hint:} Put $d = \gcd(198, 288)$. Since $\gcd(198, 288, 512) = \gcd(d, 512)$, first find integers $u$ and $v$ for which $\gcd(d, 512) = du + 512v$.] \\
\end{exercise} 

\begin{solution}
    First, let's find $\gcd(198, 288)$ using the Euclidean Algorithm:
    \begin{align*}
        288 &= 1 \cdot 198 + 90 \\
        198 &= 2 \cdot 90 + 18 \\
        90 &= 5 \cdot 18 + 0.
    \end{align*}
    Hence, $\gcd(198, 288) = 18$. Let's use these equations to find the linear combination:
    \begin{align*}
        18 &= 198 - 2\cdot 90 \\
        &= 198 - 2(288 - 198) \\
        &= 3 \cdot 198 - 2 \cdot 288.
    \end{align*}
    Now, let's find $\gcd(198, 228, 512) = \gcd(18, 512)$ using the Euclidean Algorithm:
    \begin{align*}
        512 &= 28 \cdot 18 + 8 \\
        18 &= 2 \cdot 8 + 2 \\
        8 &= 4 \cdot 2 + 0.
    \end{align*}
    Hence, $\gcd(198, 288, 512) = 1$. Let's use these equations to find the linear combination:
    \begin{align*}
        2 &= 18 - 2\cdot 8 \\
        &= 18 - 2(512 - 28 \cdot 18) \\
        &= 57 \cdot 18 - 2 \cdot 512.
    \end{align*}
    Replacing 18 with the linear combination of 198 and 288 gives us
    $$\gcd(198, 288, 512) = 171 \cdot 198 - 114 \cdot 288 - 2 \cdot 512$$
    giving us $x = 171$, $y = 114$ and $z = -2$.
\end{solution}
\section{The Diophantine Equation $ax + by = c$}

\begin{exercise}
    Which of the following Diophantine equations cannot be solved ?
    \begin{enumerate}
        \item $6x + 51y = 22$;
        \item $33x + 14y = 115$;
        \item $14x + 35y = 93$.
    \end{enumerate}
\end{exercise}

\begin{solution}
    \begin{enumerate}
        \item Let's use the Euclidean Algorithm to find $\gcd(6, 51)$:
        \begin{align*}
            51 &= 4\cdot 6 + 3 \\
            6 &= 2\cdot 3 + 0. 
        \end{align*}
        Hence, $\gcd(6, 51) = 3$. But 22 is not divisible by 3 so this equation cannot be solved.
        \item Let's use the Euclidean Algorithm to find $\gcd(33, 14)$:
        \begin{align*}
            33 &= 2\cdot 14 + 5 \\
            14 &= 2\cdot 5 + 4 \\
            5 &= 1\cdot 4 + 1 \\
            4 &= 4\cdot 1 + 0. 
        \end{align*}
        Hence, $\gcd(33, 14) = 1$. Since 115 is divisible by 1, then this equation can be solved.
        \item Let's use the Euclidean Algorithm to find $\gcd(14, 35)$:
        \begin{align*}
            35 &= 2\cdot 14 + 7 \\
            14 &= 2\cdot 7 + 0. 
        \end{align*}
        Hence, $\gcd(14, 35) = 7$. But $93 = 13\cdot 7 + 2$ is not divisible by 7 so this equation cannot be solved.
    \end{enumerate}
\end{solution}

\begin{exercise}
    Determine all solutions in the integers of the following Diophantine equations:
    \begin{enumerate}
        \item $56x + 72y = 40$;
        \item $24x + 138y = 18$;
        \item $221x + 35y = 11$.
    \end{enumerate}
\end{exercise}

\begin{solution}
    \begin{enumerate}
        \item First, let's apply the Euclidean Algorithm to find $\gcd(56, 72)$:
        \begin{align*}
            72 &= 1\cdot 56 + 16 \\
            56 &= 3 \cdot 16 + 8 \\
            16 &= 2\cdot 8 + 0.
        \end{align*}
        Hence, $\gcd(56, 72) = 8$. Since $40 = 5 \cdot 8$ is divisible by 8, then this equation has integer solutions. First, let's find one solution by reversing the Euclidean Algorithm:
        \begin{align*}
            8 &= 56 - 3\cdot 16 \\
            &= 56 - 3(72 - 56) \\
            &= 4\cdot 56 - 3\cdot 72.
        \end{align*}
        Multiplying both sides by 5:
        $$20 \cdot 56 - 15 \cdot 72 = 40$$
        gives us the solution $x_0 = 20$, $y_0 = -15$. By Theorem 2-9, we have that the general solution is given by $x = x_0 + \frac{72}{8}t = 20 + 9t$ and $y = y_0 - \frac{56}{8}t = -15 - 7t$ where $t$ is an integer.
        \item First, let's apply the Euclidean Algorithm to find $\gcd(24, 138)$:
        \begin{align*}
            138 &= 5\cdot 24 + 18 \\
            24 &= 1 \cdot 18 + 6 \\
            18 &= 3\cdot 6 + 0.
        \end{align*}
        Hence, $\gcd(24, 138) = 6$. Since $18$ is divisible by 6, then this equation has integer solutions. First, let's find one solution by reversing the Euclidean Algorithm:
        \begin{align*}
            6 &= 24 - 18 \\
            &= 24 - (138 - 5\cdot 24) \\
            &= 6\cdot 24 - 138.
        \end{align*}
        Multiplying both sides by 3:
        $$18 \cdot 24 - 3 \cdot 138 = 18$$
        gives us the solution $x_0 = 18$, $y_0 = -3$. By Theorem 2-9, we have that the general solution is given by $x = x_0 + \frac{138}{6}t = 18 + 23t$ and $y = y_0 - \frac{24}{6}t = -3 - 4t$ where $t$ is an integer.
        \item First, let's apply the Euclidean Algorithm to find $\gcd(221, 35)$:
        \begin{align*}
            221 &= 6\cdot 35 + 11 \\
            35 &= 3 \cdot 11 + 2 \\
            11 &= 5\cdot 2 + 1 \\
            2 &= 2\cdot 1 + 0.
        \end{align*}
        Hence, $\gcd(221, 35) = 1$. Since $11$ is divisible by 1, then this equation has integer solutions. First, let's find one solution by reversing the Euclidean Algorithm:
        \begin{align*}
            1 &= 11 - 5\cdot 2 \\
            &= 11 - 5(35 - 3\cdot 11) \\
            &= 16 \cdot 11 - 5 \cdot 35 \\
            &= 16 (221 - 6\cdot 35) - 5\cdot 35 \\
            &= 16 \cdot 221 - 101 \cdot 35.
        \end{align*}
        Multiplying both sides by 11:
        $$176 \cdot 221 - 1111 \cdot 35 = 11$$
        gives us the solution $x_0 = 176$, $y_0 = -1111$. By Theorem 2-9, we have that the general solution is given by $x = x_0 + \frac{35}{1}t = 176 + 35t$ and $y = y_0 - \frac{221}{1}t = -1111 - 221t$ where $t$ is an integer.
    \end{enumerate}
\end{solution}

\begin{exercise}
    Determine all solutions in the positive integers of the following Diophantine equations:
    \begin{enumerate}
        \item $18x + 5y = 48$;
        \item $54x + 21y = 906$;
        \item $123x + 360y = 99$;
        \item $158x - 57y = 7$.
    \end{enumerate}
\end{exercise}

\begin{solution}
    \begin{enumerate}
        \item First, let's apply the Euclidean Algorithm to find $\gcd(18, 5)$:
        \begin{align*}
            18 &= 3\cdot 5 + 3 \\
            5 &= 1 \cdot 3 + 2 \\
            3 &= 1\cdot 2 + 1 \\
            2 &= 2 \cdot 1 + 0.
        \end{align*}
        Hence, $\gcd(18, 5) = 1$. Since $48$ is divisible by 1, then this equation has integer solutions. First, let's find one solution by reversing the Euclidean Algorithm:
        \begin{align*}
            1 &= 3 - 2 \\
            &= 3 - (5 - 3) \\
            &= 2 \cdot 3 - 5 \\
            &= 2(18 - 3\cdot 5) - 5 \\
            &= 2\cdot 18 - 7\cdot 5.
        \end{align*}
        Multiplying both sides by 48:
        $$96 \cdot 18 - 336 \cdot 5 = 48$$
        gives us the solution $x_0 = 96$, $y_0 = -336$. By Theorem 2-9, we have that the general solution is given by $x = x_0 + \frac{5}{1}t = 96 + 5t$ and $y = y_0 - \frac{18}{1}t = -336 - 18t$ where $t$ is an integer. To find the positive solutions, it suffices to solve the following inequalities: $x = 96 + 5t > 0$ and $y = -336 - 18t > 0$. This is equivalent to the inequality:
        $$-\frac{96}{5} < t < -\frac{336}{18}$$
        Since $t$ is an integer, then the only possible value tp have $x,y > 0$ is at $t = -19$.
        \item First, let's apply the Euclidean Algorithm to find $\gcd(54, 21)$:
        \begin{align*}
            54 &= 2\cdot 21 + 12 \\
            21 &= 1 \cdot 12 + 9 \\
            12 &= 1\cdot 9 + 3 \\
            9 &= 3 \cdot 3 + 0.
        \end{align*}
        Hence, $\gcd(54, 21) = 3$. Since $906$ is divisible by 3, then this equation has integer solutions. First, let's find one solution by reversing the Euclidean Algorithm:
        \begin{align*}
            3 &= 12 - 9 \\
            &= 12 - (21 - 12) \\
            &= 2 \cdot 12 - 21 \\
            &= 2(54 - 2\cdot 21) - 21 \\
            &= 2\cdot 54 - 5 \cdot 21.
        \end{align*}
        Multiplying both sides by 302:
        $$604 \cdot 54 - 1510 \cdot 21 = 906$$
        gives us the solution $x_0 = 604$, $y_0 = -1510$. By Theorem 2-9, we have that the general solution is given by $x = x_0 + \frac{21}{3}t = 604 + 7t$ and $y = y_0 - \frac{54}{3}t = -1510 - 18t$ where $t$ is an integer. To find the positive solutions, it suffices to solve the following inequalities: $x = 604 + 7t > 0$ and $y = -1510 - 18t > 0$. This is equivalent to the inequality:
        $$-\frac{604}{7} < t < -\frac{1510}{18}$$
        Since $t$ is an integer, then $t$ must range from $-86$ to $-84$ to have $x,y > 0$.
        \item First, let's apply the Euclidean Algorithm to find $\gcd(123, 360)$:
        \begin{align*}
            360 &= 2\cdot 123 + 114 \\
            123 &= 1 \cdot 114 + 9 \\
            114 &= 12\cdot 9 + 6 \\
            9 &= 1 \cdot 6 + 3 \\
            6 &= 2 \cdot 3 + 0.
        \end{align*}
        Hence, $\gcd(123, 360) = 3$. Since $99$ is divisible by 3, then this equation has integer solutions. First, let's find one solution by reversing the Euclidean Algorithm:
        \begin{align*}
            3 &= 9 - 6 \\
            &= 9 - (114 - 12 \cdot 9) \\
            &= 13 \cdot 9 - 114 \\
            &= 13(123 - 114) - 114 \\
            &= 13 \cdot 123 - 14 \cdot 114 \\
            &= 13 \cdot 123 - 14 (360 - 2\cdot 123) \\
            &= 41 \cdot 123 - 14 \cdot 360.
        \end{align*}
        Multiplying both sides by 33:
        $$1353 \cdot 123 - 462 \cdot 360 = 99$$
        gives us the solution $x_0 = 1353$, $y_0 = -462$. By Theorem 2-9, we have that the general solution is given by $x = x_0 + \frac{360}{3}t = 1353 + 120t$ and $y = y_0 - \frac{123}{3}t = -462 - 41t$ where $t$ is an integer. To find the positive solutions, it suffices to solve the following inequalities: $x = 1353 + 120t > 0$ and $y = -462 - 41t > 0$. This is equivalent to the inequality:
        $$-\frac{1353}{120} < t < -\frac{462}{41}$$
        Since $t$ is an integer, then $t$ must be both greater than or equal to -11 and less than or equal to -12. Therefore, this equation has no solutions in the positive integers.
        \item First, let's apply the Euclidean Algorithm to find $\gcd(158, -57) = \gcd(158, 57)$:
        \begin{align*}
            158 &= 2\cdot 57 + 44 \\
            57 &= 1 \cdot 44 + 13 \\
            44 &= 3\cdot 13 + 5 \\
            13 &= 2 \cdot 5 + 3 \\
            5 &= 1 \cdot 3 + 2 \\
            3 &= 1 \cdot 2 + 1 \\
            2 &= 2 \cdot 2 + 0.
        \end{align*}
        Hence, $\gcd(158, -57) = 1$. Since $7$ is divisible by 1, then this equation has integer solutions. First, let's find one solution by reversing the Euclidean Algorithm:
        \begin{align*}
            1 &= 3 - 2 \\
            &= 3 - (5 - 3) \\
            &= 2\cdot 3 - 5 \\
            &= 2(13 - 2\cdot 5) - 5 \\
            &= 2 \cdot 13 - 5 \cdot 5 \\
            &= 2 \cdot 13 - 5(44 - 3\cdot 13) \\
            &= 17 \cdot 13 - 5 \cdot 44 \\
            &= 17(57 - 44) - 5 \cdot 44 \\
            &= 17 \cdot 57 - 22 \cdot 44 \\
            &= 17 \cdot 57 - 22 (158 - 2 \cdot 57) \\
            &= 61 \cdot 57 - 22 \cdot 158 \\
            &= (-22) \cdot 158 + (-61) \cdot (-57).
        \end{align*}
        Multiplying both sides by 7:
        $$158 \cdot (-154) + (-57)\cdot(-427) = 7$$
        gives us the solution $x_0 = -154$, $y_0 = -427$. By Theorem 2-9, we have that the general solution is given by $x = x_0 - \frac{57}{1}t = -154 - 57t$ and $y = y_0 - \frac{158}{1}t = 427 - 158t$ where $t$ is an integer. To find the positive solutions, it suffices to solve the following inequalities: $x = -154 - 57t > 0$ and $y = 427 - 158t > 0$. This is equivalent to the inequality:
        $$t < \min\left(-\frac{154}{57}, \frac{427}{158}\right) = -\frac{154}{57} = -\left(2 + \frac{40}{57}\right).$$
        Since $t$ is an integer, then $t$ must smaller than or equal to $3$ to have $x,y > 0$.
    \end{enumerate}
\end{solution}

\begin{exercise}
    If $a$ and $b$ are relatively prime positive integers, prove that the Diophantine equation $ax - by = c$ has infinitely many solutions in the positive integers.

    \noindent [\textit{Hint:} There exist integers $x_0$ and $y_0$ such that $ax_0 + by_0 = 1$. For any integer $t$, which is larger than both $|x_0|/b$ and $|y_0|/a$, $x = x_0 + bt$ and $y = -(y_0 - at)$ are a positive solution of the given equation.] \\
\end{exercise}

\begin{solution}
    First, let $b' = -b$, then $d = \gcd(a, b') = \gcd(a,b) = 1$. It follows that the equation $ax + b'y = c$ has a solution since $c$ is divisible by 1. Let $x_0$ and $y_0$ be integers such that $ax_0 + b'y_0 = c$, then we know that for all integers $t$, $x = x_0 + (b'/d)t = x_0 - bt$ and $y = y_0 - (a/d)t = y_0 - at$ are also solutions. If we want $x$ and $y$ to be positive, we need $t$ to satisfy the inequalities $x_0 > bt$ and $y_0 > at$. Equivalently, we need $t$ to be less than $\min(x_0/b, y_0/a)$. Since there are infinitely many such values of $t$, then there are infinitely many positive solutions to the equation $ax - by = c$.\\
\end{solution}

\begin{exercise}
    \begin{enumerate}
        \item Prove that the Diophantine equation $ax + by + cz = d$ is solvable in the integers if and only if $\gcd(a,b,c)$ divides $d$.
        \item Find all solutions in the integers of $15x + 12y + 30z = 24$. [\textit{Hint:} Put $y = 3s - 5t$ and $z = -s + 2t$.]
    \end{enumerate}
\end{exercise}

\begin{solution}
    \begin{enumerate}
        \item First, suppose that there are integers $x_0$, $y_0$ and $z_0$ such that $ax_0 + by_0 + cz_0 = d$, then $d$ must be divisible by $\gcd(a,b,c)$ since $\gcd(a,b,c)$ divides $a$, $b$, $c$, and hence, any of their linear combination, such as $d$. Conversely, suppose that $d$ is divisible by $\gcd(a,b,c)$ such that $d = s\cdot \gcd(a,b,c)$. Recall from Exercise 2.3.11 that $\gcd(a,b,c) = \gcd(\gcd(a,b), c)$, hence, there exist integers $x'$ and $z_0$ such that $\gcd(a,b,c) = \gcd(a,b)x' + cz_0$. Similarly, there exist integers $x_0$ and $y_0$ such that $\gcd(a,b) = ax_0 + by_0$ and so it follows that $\gcd(a,b,c) = a(x'x_0) + b(x'y_0) + cz_0$. Thus, we have $d = a(sx'x_0) + b(sx'y_0) + c(sz_0)$. Therefore, the equation $ax + by + cz = d$ is solvable in the integers.
        \item (This solution does not follow the hint.) First, fix $z = t$ and consider the equation $15x + 12y = 24 - 30t$. Since $\gcd(15, 12) = 3\gcd(5, 4) = 3$ divides $24 - 30t = 3(8 - 10t)$, then the equation is solvable in the integers. To find a solution, notice that from $15 - 12 = 3$, we have $(8-10t)\cdot 15 - (8-10t)\cdot 12 = 24 - 30t$ which gives us the particular solution $x_0 = 8 - 10t$ and $y_0 = -8 + 10t$. It follows that the general solution is given by $x = 8 - 10t + 4s$ and $y = -8 + 10t - 5s$. Therefore, $x = 8 - 10t + 4s$, $y = -8 + 10t - 5s$ and $z = t$ are solutions to the original equation for all integers $s$ and $t$. We can prove that every solution can be written in this form as follows: suppose that the integers $x$, $y$ and $z$ satisfy the equation $15x + 12y + 30z = 24$, then equivalently, $x$ and $y$ satisfy the equation $15x + 12y = 24 - 30z$. Since $x_0 = 8 - 10z$, $y_0 = -8 + 10z$ is a particular solution to that equation, then there must be an integer $s_0$ such that $x = 8 - 10z + 4s_0$ and $y = -8 + 10z - 5s_0$. Thus, if we let $t = z$ and $s = s_0$, then $x$, $y$ and $z$ are indeed of the form $x = 8 - 10t + 4s$, $y = -8 + 10t - 5s$ and $z = t$. Therefore, these are all the integer solutions to the equation.
    \end{enumerate}
\end{solution}

\begin{exercise}
    \begin{enumerate}
        \item A man has \$$4.55$ in change composed entirely of dimes and quarters. What are the maximum and minimum number of coins that he can have? Is it possible for the numbers of dimes to equal the number of quarters?
        \item The neighborhood theater charges \$$1.80$ for adult admissions and 75 cents for children. On a particular evening, the total receipts were \$90. Assuming that more adults than children were present, how many people attended?
        \item A certain number of sixes and nines are added to give a sum of 126; if the number of sixes and nines are interchanged, the new sum is 114. How many of each were there originally?
    \end{enumerate}
\end{exercise}

\begin{solution}
    \begin{enumerate}
        \item First, notice that we can think of this problem as being the same as solving the equation $10x + 25y = 455$ where $x$ corresponds to the number of dimes, and $y$ corresponds to the number of quarters. Since $\gcd(10, 25) = 5\gcd(2, 5) = 5$ divides $455 = 5\cdot 91$, then the equation is solvable in the integers. By multiplying the equation $10\cdot(-2) + 25 = 5$ by $91$, we get the particular solution $x_0 = -182$, $y_0 = 91$. It follows that the general solution is given by $x = -182 + 5t$, $y = 91 - 2t$ where $t$ is an integer. Since we want both $x$ and $y$ to be positive, then we want the following inequalities to be satisfied simultaneously: $x = -182 + 5t > 0$, $y = 91 - 2t > 0$. Equivalently, $t$ must satisfy $36.4 = \frac{182}{5} < t < \frac{91}{2} = 45.5$. Since $t$ is an integer, then $t$ must range from 37 to 45. The total number of coins can be expressed by $x + y = -182 + 5t + 91 - 2t = 3t - 91$. Since $x + y$ is an increasing function of $t$, then the maximum number of coins is 44 (at $t = 45$) and the minimum number of coins is 20 (at $t = 37$). For the number of dimes and quarters to be the same, we must have 
        $x = y$ which is equivalent to $-182 + 5t = 91 - 2t$. Solving for $t$, we get $t = 39$. Thus, a possible solution is $x = y = 13$.
        \item First, if we denote the number of adults by $x$ and the number of children by $y$, then it suffices to solve the equation $180x + 75y = 9000$. To do so, notice that $\gcd(180, 75) = 15\gcd(12, 5) = 15$. From the equation $(-2)\cdot 180 + 5\cdot 75 = 15$, we get the equation $(-1200)\cdot 180 + 3000\cdot 75 = 9000$ which gives us the particular solution $x_0 = -1200$ and $y_0 = 3000$. It follows that the general solution is given by $x = -1200 + 5t$ and $y = 3000 - 12t$ where $t$ is an integer. Since there are more adults than children, then this translates into $x > y > 0$. In terms of $t$, then inequality $x > y$ becomes $t > \frac{4200}{17}$, and the inequality $y > 0$ becomes $\frac{3000}{12}>t$. Hence, $t$ must satisfy $\frac{4200}{17} < t < \frac{3000}{12}$. Since $t$ is an integer, then it ranges from 248 to 250. Since the total number of people is $x + y = -1200 + 5t + 3000 - 12t = 1800 - 7t$, then in total, either 64 (at $t = 248$), 57 (at $t = 249$) or 50 (at $t = 250$) people came.
        \item We need to solve the equation $6x + 9y = 126$ such that $6y + 9x = 114$. Since $\gcd(6, 9) = 3$ and $(-1)\cdot 6 + 1\cdot 9 = 3$, then $(-42)\cdot 6 + 42\cdot 9 = 126$ which gives us the particular solution $x_0 = -42$ and $y_0 = 42$. It follows that the general solution is given by $x = -42 + 3t$ and $y = 42 - 2t$ where $t$ is an integer. Now, plugging these values in the second equation gives us $6(42 - 2t) + 9(-42 + 3t) = 114$. This equation can be simplified into $15t = 240$, and so $t = 16$. Therefore, we get $x = 6$ and $y = 10$ which corresponds to six 6s and ten 9s.
    \end{enumerate}
\end{solution}

\begin{exercise}
    A farmer purchased one hundred head of livestock for a total cost of \$4000. Prices were as follow: calves, \$120 each; lambs, \$ 50 each; piglets, \$25 each. If the farmer obtained at least one animal of each type how many did he buy ?\\
\end{exercise}

\begin{solution}
    If we denote by $x$ the number of calves, by $y$ the number of lambs, and by $z$ the number of piglets, then we need to solve the equation $120x + 50y + 25z = 4000$ where $x,y,z > 0$ and $x + y + z = 100$. Since we can rewrite the previous equation as $z = 100 - x - y$, then it suffices to solve the equation $120x + 50y + 25(100 - x - y) = 4000$. This equation can be simplified into $19x + 5y = 300$. From $19 \cdot (-1) + 5 \cdot 4 = 1$, we get $19 \cdot (-300) + 5 \cdot 1200 = 300$ which gives us the particular solution $x_0 = -300$, $y_0 = 1200$. It follows that the general solution is given by $x = -300 + 5t$, $y = 1200 - 19t$ where $t$ is an integer. Since we want $x,y,z > 0$, then we need to solve the following inequalities in terms of $t$: $-300 + 5t > 0$, $1200 - 19t > 0$ and $100 > 900 - 14t$. From these inequalities, we get that $t$ must range from 61 to 63. It follows that we must have one of the three following cases: ($t = 61$) 5 calves, 41 lambs, 54 piglets; ($t = 62$) 10 calves, 22 lambs, 68 piglets; ($t = 63$) 15 calves, 3 lambs, 82 piglets. \\
\end{solution}

\begin{exercise}
    When Mr. Smith cashed a check at his bank, the teller mistook the number of cents for the number of dollars and vice versa. Unaware of this, Mr. Smith spent 68 cents and then noticed to his surprise that he had twice the amount of the original check. Determine the smallest value for which the check could have been written. [\textit{Hint:} If $x$ is the number of dollars and $y$ is the number of cents in the check, then $100y + x - 68 = 2(100x + y)$.]\\
\end{exercise}

\begin{solution}
    Let $x$ be the number of dollars and $y$ be the number of cents, then the value of the check is given by $x + \frac{1}{100}y$. Thus, we can translate the situation into the equation $y + \frac{1}{100}x - \frac{68}{100} = 2(x + \frac{1}{100}y)$. Multiplying both sides by 100 gives us $100y + x - 68 = 2(100x + y)$. Putting all the terms together gives us the equation $- 199x + 98y = 68$. Let's apply the Euclidean Algorithm to find $\gcd(-199, 98) = \gcd(199, 98)$:
    \begin{align*}
        199 &= 2\cdot 98 + 3 \\
        98 &= 32 \cdot 3 + 2 \\
        3 &= 1 \cdot 2 + 1 \\
        2 &= 2 \cdot 1 + 0.
    \end{align*}
    From that, we get
    \begin{align*}
        1 &= 3 - 2 \\
        &= 3 - (98 - 32\cdot 3) \\
        &= 33 \cdot 3 - 98 \\
        &= 33(199 - 2\cdot 98) - 98 \\
        &= 33 \cdot 199 - 67 \cdot 98 \\
        &= (-33) \cdot (-199) + (-67) \cdot 98.
    \end{align*}
    By multiplying both sides by 68, we get
    $$(-199)\cdot (-2244) + 98 \cdot (-4556) = 68$$
    which gives us the particular solution $x_0 = -2244$, $y_0 = -4556$. It follows that the general solution is given by $x = -2244 + 98t$, $y = -4556 + 199t$ where $t$ is an integer. From the fact that $x,y > 0$, we get the following inequalities for $t$: $t > \frac{2244}{98}$ and $t > \frac{4556}{199}$. Since $t$ is an integer, then it follows that $t \geq 23$. But recall that $y$ represents the number of cents so we also have the inequality $y < 100$. In terms of $t$, this inequality becomes $t < \frac{4656}{199}$. Since $t$ is an integer, then it means that $t \leq 23$. Combining the two inequalities, we get that $t = 23$. Therefore, the value of the check is \$$10.21$. \\
\end{solution}

\begin{exercise}
    Solve each of the puzzle-problems below:
    \begin{enumerate}
        \item Alcuin of York, 775. A hundred bushels of grain are distributed among 100 persons in such a way that each man receives 3 bushels, each woman 2 buchels, and each child 1/2 bushel. How many men, women, and children are there ?
        \item Mahaviracarya, 850. There were 63 equal piles of plantain fruit put together and 7 single fruits. They were divided evenly among 23 travelers. What is the number of fruits in each pile? [\textit{Hint:} Consider the Diophantine equation $63x + 7 = 23y$].
        \item Yen Kung, 1372. We have an unknown number of coins. If you make 77 strings of them, you are 50 coins short; but if you make 78 strings, it is exact. How many coins are there? [\textit{Hint:} If $N$ is the number of coins, then $N = 77x + 27 = 78y$ for integers $x$ and $y$.]
        \item Christoff Rudolf, 1526. Find the number of men, women and children in a company of 20 persons if together they pay 20 coins, each man paying 3, each woman 2, and each child 1/2.
        \item Euler, 1770. Divide 100 into two summands such that one is divisible by 7 and the other by 11.
    \end{enumerate}
\end{exercise}

\begin{solution}
    \begin{enumerate}
        \item Let $x$ be the number of men, $y$ be the number of women, and $z$ be the number of child, then we want to solve the equation $3x + 2y + \frac{1}{2}z = 100$. Since we know that $x + y + z = 100$, then we can replace $z$ by $100 - x - y$ in the equation to obtain $5x + 3y = 100$. From the equation $5 \cdot (-1) + 3 \cdot 2 = 1$, we get $5 \cdot (-100) + 3 \cdot 200 = 100$ by multiplying both sides by 100. This gives us the particular solution $x_0 = -100$, $y_0 = 200$. It follows that the general solution is given by $x = -100 + 3t$, $y = 200 - 5t$ where $t$ is an integer. Since we want $x,y,z \geq 0$, then we get the following inequalities in terms of $t$: $t \geq \frac{100}{3}$, $t \leq \frac{200}{5}$, $t \geq 0$. Hence, $t$ must range from 34 to 40. Thus, the possible triplets $(x,y,z)$ are the following: $(2,30,68)$, $(5, 25, 70)$, $(8, 20, 72)$, $(11, 15, 74)$, $(14, 10, 76)$, $(17, 5, 78)$, $(20, 0, 80)$.
        \item Let $x$ be the number of fruits in each pile, then we should have $23 \mid 63x + 7$, or $63x + 7 = 23y$ where $x,y \geq 1$. This can be rewritten as $63x - 23y = -7$. Let's apply the Euclidean Algorithm to find $\gcd(63, -23) = \gcd(63, 23)$:
        \begin{align*}
            63 &= 2 \cdot 23 + 17 \\
            23 &= 1 \cdot 17 + 6 \\
            17 &= 2 \cdot 6 + 5 \\
            6 &= 1 \cdot 5 + 1 \\
            5 &= 5 \cdot 1 + 0.
        \end{align*}
        Reversing the algorithm gives us $63 \cdot (-4) + (-23)\cdot (-11) = 1$, from which we get $63 \cdot 28 + (-23) \cdot 77 = -7$ by multiplying both sides by $-7$. Thus, we have the particular solution $x_0 = 28$, $y_0 = 77$. It follows that the general solution is given by $x = 28 - 23t$, $y = 77 - 63 t$ where $t$ is an integer. Since we want $x, y \geq 1$, then we must have $t \leq 1$. Therefore, all the possible values of fruits in each pile are $28 - 23t$ where $t \leq 1$.
        \item Let $N$ be the number of coins, then from the statement of part (c), we have that $N = 77x + 27$ and $N = 78y$ for some integers $x$ and $y$. Moreover, we must have $N > 0$. Thus, we get the equation $77x + 27 = 78y$ which can be rewritten as $77x - 78y = -27$. Since $77 \cdot (-1) + (-78) \cdot (-1) = 1$, then multiplying both sides by $-27$ gives us $77 \cdot 27 + (-78) \cdot 27 = -27$. Hence, we have the particular solution $x_0 = y_0 = 27$. It follows that the general solution is given by $x = 27 - 78t$, $y = 27 - 77t$ where $t$ is an integer. Since $N > 0$, then $t \leq 0$. It follows that a possible value for $N$ is $N = 2106$ which happens when $t = 0$. More generally, the possible values of $N$ are precisely $78(27 - 77t)$ where $t \leq 0$.
        \item Let $x$ be the number of men, $y$ be the number of women, and $z$ be the number of children, then we have the two equations $3x + 2y + \frac{1}{2}z = 20$ and $x + y + z = 20$. By multiplying the first equation by 2 on both sides and plugging $z = 20 - x - y$, we obtain $6x + 4y + (20 - x - y) = 40$. After some simplifications, this equation becomes $5x + 3y = 20$. From the equation $5 \cdot (-1) + 3 \cdot 2 = 1$, we get $5 \cdot (-20) + 3 \cdot 40 = 20$ by multiplying both sides by 20. This gives us the particular solution $x_0 = -20$, $y_0 = 40$. It follows that the general solution is given by $x = -20 + 3t$, $y = 40 - 5t$ where $t$ is an integer. Since we want $x,y,z \geq 0$, then we get the following inequalities in terms of $t$: $t \geq \frac{20}{3}$, $t \leq \frac{40}{5}$, $t \geq 0$. Hence, $t$ must be either 7 or 8. Thus, the possible triplets $(x,y,z)$ are the following: $(1,5, 14)$, $(4, 0, 16)$.
        \item This problem can be simply solved by considering the equation $100 = 7x + 11y$. Since $7 \cdot (-3) + 11 \cdot 2 = 1$, then $7 \cdot (-300) + 11 \cdot 200 = 100$. Hence, we get the particular solution $x_0 = -300$, $y_0 = 200$. It follows that the general solution is given by $x = -300 + 11t$, $y = 200 - 7t$ where $t$ is an integer. If we want $x, y \geq 0$, then this is only satisfied when $t = 28$. In that case, we have $x = 8$ and $y = 4$. This gives us the solution $100 = 56 + 44$.
    \end{enumerate}
\end{solution}

% CHAPTER 3
\chapter{Primes and Their Distribution}

\section{The Fundamental Theorem of Arithmetic}

\begin{exercise}
    It has been conjectured that there are infinitely many primes of the form $n^2 - 1$. Exhibit five such primes. \\
\end{exercise}

\begin{solution}
    When $n = 2$, we have $n^2 - 2 = 2$ which is prime. When $n = 3$, we have $n^2 - 2 = 7$ which is prime. When $n = 5$, we have $n^2 - 2 = 23$ which is prime. When $n = 7$, we have $n^2 - 2 = 47$ which is prime. When $n = 9$, we have $n^2 - 2 = 79$ which is prime.\\
\end{solution}

\begin{exercise}
    Give an example to show that the following conjecture is not true: Every positive integer can be written in the form $p + a^2$, where $p$ is either a prime or 1, and $a \geq 0$. \\
\end{exercise}

\begin{solution}
    Suppose that $25 = p + a^2$, then $25 - a^2$ is a prime number for some $a \geq 0$. Moreover, $0 \leq a \leq 5$ since otherwise, $25 - a^2$ is negative. However, when $a = 0$, $25 - a^2 = 25$ is not a prime; when $a = 1$, $25 - a^2 = 24$ is not a prime; when $a = 2$, $25 - a^2 = 21$ is not a prime; when $a = 3$, $25 - a^2 = 16$ is not a prime; when $a = 4$, $25 - a^2 = 9$ is not a prime; when $a = 5$, $25 - a^2 = 0$ is not a prime. Therefore, 25 cannot is not of the form $p + a^2$ contradicting the conjecture. \\
\end{solution}

\begin{exercise}
    Prove each of the assertions below:
    \begin{enumerate}
        \item Any prime of the form $3n+1$ is also of the form $6m+1$.
        \item Each integer of the form $3n+2$ has a prime factor of this form.
        \item The only prime of the form $n^3 - 1$ is 7. [\textit{Hint:} Write $n^3 - 1$ as $(n-1)(n^2 + n + 1)$.]
        \item The only prime $p$ for which $3p + 1$ is a perfect square is $p = 5$.
        \item The only prime of the form $n^2 - 4$ is 5.
    \end{enumerate}
\end{exercise}

\begin{solution}
    \begin{enumerate}
        \item Suppose that $p = 3n+1$, then $n$ is either of the form $2m$ or $2m+1$. If $n = 2m+1$, then $p = 3(2m+1) + 1 = 2(3m+2)$ which is impossible. Thus, $p = 6m+1$.
        \item Consider the integer $k = 3n+2$. If one of the prime factor is of the form $3m$, then will be automatically of the form $3m$. Suppose that none of the prime factors of $k$ are of the form $3m+2$, then from the previous observation, it follows that all of its prime factors must be of the form $3m+1$. However, from the fact that
        $$(3k_1 + 1)(3k_2 + 1) = 3(3k_1k_2 + k_1 + k_2) + 1,$$
        then by induction, we have that $k$ must also be of the form $3m+1$, a contradiction. Therefore, $k$ must have a prime of the form $3m+2$.
        \item Suppose that $n^3 - 1 = (n-1)(n^2 + n + 1)$ is prime, then either $n - 1$ or $n^2 + n + 1$ is equal to 1. In the first case, we get that $n = 2$ and so that $n^3 - 1 = 7$ which is indeed a prime. In the second case, we get that $n = 0$ or $n = 1$. If $n = 0$, then $n^3 - 1 = -1$ which is not a prime. Therefore, the only prime of the form $n^3 - 1$ is 7.
        \item Suppose that $3p + 1$ is a perfect square, then $3p + 1 = a^2$. Equivalently, this implies that $3p = (a-1)(a+1)$. Since 3 is a prime number, then either $3 \mid a - 1$ or $3 \mid a + 1$. If $3 \mid a - 1$, then $a - 1 = 3k$ and so $3p = 3k(3k+2)$. In that case, $p = k(k3k+2)$ which implies that either $k = 1$ or $3k+2 = 1$. Since $3k+2 \neq 1$ for any $k$, then we must have $k = 1$ and hence $p = 5$. Suppose now that $3 \mid a+1$, then $a+1 = 3k$ and so $3p = 3k(3k-2)$. Again, with the same argument as before, it follows that $p = 5$. Therefore, $p = 5$ is the only prime number for which $3p + 1$ is a perfect square.
        \item Suppose that $p = n^2 - 4 = (n-2)(n+2)$ is prime, then either $n - 2 = 3$ or $n + 2 = 1$. In the first case, we get that $n = 3$ and hence, $p = 5$ which is indeed prime. In the second case, we get $n = -1$ and hence $p = -3$ which is not a prime. Therefore, the only such prime is $p = 5$.
    \end{enumerate}
\end{solution}

\begin{exercise}
    If $p \geq 5$ is a prime number, show that $p^2 + 2$ is composite. [\textit{Hint:} $p$ takes one of the forms $6k+1$ or $6k+5$.] \\
\end{exercise}

\begin{solution}
    Notice that $p$ cannot be of the form $6k$, $6k+2 = 2(3k+1)$, $6k+3 = 3(2k+1)$ or $6k+4 = 2(3k+2)$ because in all of these cases, since $p \geq 5$, $k \neq 0$ and so it is composite. If $p = 6k+1$, then
    $$p^2 +2 = (6k+1)^2 + 2 = 6^2k^2 + 2\cdot 6k + 1 + 2 = 3(18k^2 + 4k + 1)$$
    which is composite. Similarly, if $p = 6k+5$, then 
    $$p^2 +2 = (6k+5)^2 + 2 = 6^2k^2 + 2\cdot 6k + 25 + 2 = 3(18k^2 + 4k + 9)$$
    which is composite. Therefore, $p^2 + 2$ is composite for all prime numbers $p \geq 5$.\\
\end{solution}

\begin{exercise}
    \begin{enumerate}
        \item Given that $p$ is a prime and $p \mid a^n$, prove that $p^n \mid a^n$.
        \item If $\gcd(a,b) = p$, a prime, what are the possible values of $\gcd(a^2, b^2)$, $\gcd(a^2, b)$ and $\gcd(a^3, b^2)$ ?
    \end{enumerate}
\end{exercise}

\begin{solution}
    \begin{enumerate}
        \item Since $p$ is a prime and $p \mid a^n$, then $p$ must divide $a$. It follows that $p^n \mid a^n$.
        \item We know that $\gcd(a,b) = 1$ implies that $\gcd(a^2, b^2) = 1$ and that $\gcd(ka,kb) = k\gcd(a,b)$. It follows that in this case, $\gcd(\frac{a}{p}, \frac{b}{p}) = 1$ and so $\gcd(\frac{a^2}{p^2}, \frac{b^2}{p^2}) = 1$ which implies that $\gcd(a^2, b^2) = p^2$. 
        
        Similarly, since $\gcd(a,b) = p$, then $a = pk_1$ and $b = pk_2$ where $\gcd(k_1, k_2)$. It follows that $\gcd(k_1^2, k_2) = 1$ (Exercise 2.2.20(f)) and so there exist integers $x$ and $y$ such that $k_1^2x + k_2y = 1$. Multiplying both sides by $p$ gives us that $(pk_1^2)x + k_2(py) = p$ and so $\gcd(pk_1^2, k_2) \mid p$. It follows that $\gcd(pk_1^2, k_2)$ is either 1 or $p$ and so that $\gcd(a^2, b)$ is either $p$ or $p^2$. It is impossible to lower the number of possibilities because when $a = p$ and $b = p$, we have $\gcd(a^2, b) = p$ and when $a = p$ and $b = p^2$, we have $\gcd(a^2, b) = p^2$. Therefore, $p$ and $p^2$ are precisely the possible values of $\gcd(a^2, b)$.

        For the third value, since $\gcd(a,b) = p$, then $a = pk_1$ and $b = pk_2$ where $\gcd(k_1, k_2)$. It follows that $\gcd(k_1^3, k_2^2) = 1$ (Exercise 2.2.20(f)) and so there exist integers $x$ and $y$ such that $k_1^3x + k_2^2y = 1$. Multiplying both sides by $p^2$ gives us that $(pk_1^3)x + k_2^2(py) = p$ and so $\gcd(pk_1^3, k_2^2) \mid p$. It follows that $\gcd(pk_1^3, k_2^2)$ is either 1 or $p$ and so that $\gcd(a^3, b^2)$ is either $p^2$ or $p^3$. It is impossible to lower the number of possibilities because when $a = p$ and $b = p$, we have $\gcd(a^3, b^2) = p^2$ and when $a = p$ and $b = p^2$, we have $\gcd(a^3, b^2) = p^3$. Therefore, $p^2$ and $p^3$ are precisely the possible values of $\gcd(a^3, b^2)$.
    \end{enumerate}
\end{solution}

\begin{exercise}
    Establish each of the following statements:
    \begin{enumerate}
        \item Every integer of the form $n^4 + 4$, with $n > 1$, is composite.
        
        [\textit{Hint:} Write $n^4 + 4$ as a product of two quadratic factors.]
        \item If $n > 4$ is composite, then $n$ divides $(n-1)!$.
        \item Any integer of the form $8^n + 1$, where $n \geq 1$, is composite.
        
        [\textit{Hint:} $2^n + 1 \mid 2^{3n} + 1$].
        \item Each integer $n > 11$ can be written as the sum of two composite numbers. [\textit{Hint:} If $n$ is even, say $n = 2k$, then $n - 6 = 2(k-3)$; for $n$ odd, consider the integer $n - 9$.]
    \end{enumerate}
\end{exercise}

\begin{solution}
    \begin{enumerate}
        \item First, notice that $n^4 + 4 = (n^2 - 2n + 2)(n^2 + 2n + 2)$ for all $n$. Since both factors are strictly bigger than 1 when $n > 1$, then $n^4 + 4$ must be a composite number.
        \item By the Fundamental Theorem of Arithmetic, we know that $n = p_1^{k_1}\cdot ... \cdot p_m^{k_m}$. If $m > 1$, then we can define the two distinct (by the Fundamental Theorem of Arithmetic) integers $a = p_1^{k_1}$ and $b = p_2^{k_2}\cdot ... \cdot p_m^{k_m}$ such that $a,b > 1$ and $ab = n$. Since these factors are non-trivial, then they must satisfy $a,b \leq n-1$. Since they are distinct and less than $n - 1$, then we can write
        $$(n-1)! = 1\cdot 2 \cdot ... \cdot a \cdot ... \cdot b \cdot ... \cdot (n-1)$$
        which shows that $n = ab \mid (n-1)!$. If $m = 1$, then $n = p^k$. If $k > 2$, then we can let $a = p$, $b = p^{k-1}$ such that $a \neq b$, $a,b < n-1$ and $ab = n$. Hence, with the same argument as above, we can conclude that $n = ab \mid (n-1)!$. If $k \not > 2$, then $k = 2$ since $k = 1$ would imply that $n$ is not composite. Hence, the last case is $n = p^2$. Notice that $p \neq 2$ since otherwise, $n = 4$. Here, let $a = p$ and $b = 2p$ and notice that both numbers are distinct and less than $n-1$. Hence, we must have that $2n = ab \mid (n-1)!$ and so that $n \mid (n-1)!$. Therefore, in all possible cases, $n \mid (n-1)!$.
        \item If we replace $x$ with $2^n$ in the relation $x^3 + 1 = (x+1)(x^2 - x + 1)$, we get that $2^n + 1 \mid 2^{3n} + 1 = 8^n + 1$. It follows that $8^n + 1$ is always composite when $n \geq 1$.
        \item Let $n > 11$ be an integer. Suppose first that $n = 2k$, then
        $$n = (n-6) + 6 = 2(k-3) + 2\cdot 3.$$
        Since both $2(k-3)$ and $2\cdot 3$ are composite (if $2(k-3) = 0$, then $11 < n = 6$), then $n$ is indeed the sum of two composite numbers. Similarly, if $n = 2k+1$, then 
        $$n = (n - 9) + 9 = 2(k-4) + 3\cdot3.$$
        Since both $2(k-4)$ and $3\cdot 3$ are composite (if $2(k-4) = 0$, then $11 < n = 9$), then $n$ is again the sum of two composite numbers. Therefore, it holds for all integers $n > 11$.
    \end{enumerate}
\end{solution}

\begin{exercise}
    Find all prime numbers that divide $50!$. \\
\end{exercise}

\begin{solution}
    First, notice that every prime number less than 50 must divide $50!$ by the definition of $n!$. Moreover, let $p$ be a prime number dividing $50!$, then $p$ must divide a number less than 50, and hence, $p$ must be less than 50. Therefore, the prime numbers dividing $50!$ are precisely the prime numbers that are less than 50:
    $$2,3,5,7,11,13,17,19,23,29,31,37,41,43,47.$$
\end{solution}

\begin{exercise}
    If $p \geq q \geq 5$ and $p$ and $q$ are both primes, prove that $24 \mid p^2 - q^2$. \\
\end{exercise}

\begin{solution}
    Since $p$ and $q$ are prime, then both are either of the form $4k+1$ or $4k+3$. If $p = 4r+1$ and $q = 4t + 1$, then $p + q = 4(r+t) + 2 = 2(2(r+t) + 1)$ and $p - q = 4(r - t)$, and so $8 \mid (p+q)(p-q) = p^2 - q^2$. If $p = 4r+3$ and $q = 4t + 1$, then $p + q = 4(r+t + 1)$ and $p - q = 4(r - t) + 2 = 2(2(r-t) + 1)$, and so $8 \mid (p+q)(p-q) = p^2 - q^2$. The same calculation proves that $8 \mid p^2 - q^2$ when $p = 4r+1$ and $q = 4t + 3$. Finally, when $p = 4r+3$ and $q = 4t + 3$, then $p + q = 4(r+t + 1) + 2 = 2(2(r+t+1) + 1)$ and $p - q = 4(r-t)$, and so $8 \mid p^2 - q^2$. Therefore, in all possible cases, $8 \mid p^2 - q^2$.

    Moreover, since $p$ and $q$ are primes, then both are of the form $3k+1$ or $3k+2$. If both are of the form $3k+1$, then $p-q$ is of the form $3k$ and so $3 \mid p^2 - q^2$. If one is of the form $3k+1$ and the other is of the form $3k+2$, then $p + q$ is of the form $3k$ and so $3 \mid p^2 - q^2$. If both are of the form $3k+2$, then $p-q$ is of the form $3k$ and so $3 \mid p^2 - q^2$. Therefore, in all cases, $3 \mid p^2 - q^2$. 

    From this, we get that both 3 and 8 divide $p^2 - q^2$. Since $\gcd(3,8) = 1$, then $24 = 3 \cdot 8 \mid p^2 - q^2$. \\
\end{solution}

\begin{exercise}
    \begin{enumerate}
        \item An unanswered question is whether there are infinitely many primes which are 1 more than a power of 2, such as $5 = 2^2 + 1$. Find two or more of these primes.
        \item A more general conjecture is that there exist infinitely many primes of the form $n^2 + 1$; for example, $257 = 16^2 + 1$. Exhibit five more primes of this type.
    \end{enumerate}
\end{exercise}

\begin{solution}
    \begin{enumerate}
        \item We have $2^0 + 1 = 2$ and $2^1 + 1 = 3$ which are both primes.
        \item We have $1^1 + 1 = 2$, $2^2 + 1 = 5$, $4^2 + 1 = 17$, $6^2 + 1 = 37$ and $10^2 + 1 = 101$ which are all primes.
    \end{enumerate}
\end{solution}

\begin{exercise}
    If $p \neq 5$ is an odd prime, prove that either $p^2-1$ or $p^2 + 1$ is divisible by 10. [\textit{Hint:} $p$ takes one of the forms $10k + 1$, $10k+3$, $10k + 7$ or $10k + 9$.] \\
\end{exercise}

\begin{solution}
    Since $p$ is a prime, then it must be of the form $10k + 1$,  $10k+3$, $10k + 7$ or $10k + 9$ since otherwise, it would be composite. If $p = 10k + 1$, then $p^2 - 1 = 10(10k^2 + 2k)$ and so it is divisible by 10. If $p = 10k + 3$, then $p^2 + 1 = 10(10k^2 + 6k + 1)$ and so it is divisible by 10. If $p = 10k + 7$, then $p^2 + 1 = 10(10k^2 + 17k + 5)$ and so it is divisible by 10. If $p = 10k + 9$, then $p^2 - 1 = 10(10k^2 + 18k + 8)$ and so it is divisible by 10. Therefore, it holds for all primes $p \neq 5$. \\
\end{solution}

\begin{exercise}
    Another unproven conjecture is that there are an infinitude of primes which are 1 less than a power of 2, such as $3 = 2^2 - 1$.
    \begin{enumerate}
        \item Find four more of these primes.
        \item If $p = 2^k - 1$ is prime, show that $k$ is an odd integer, except when $k = 2$. [\textit{Hint:} $3 \mid 4^n - 1$ for all $n \geq 1$.]
    \end{enumerate}
\end{exercise}

\begin{solution}
    \begin{enumerate}
        \item We have that $2^3 - 1 = 7$, $2^5 - 1 = 31$, $2^7 - 1 = 127$ and $2^{13} - 1 = 8191$ are all prime numbers.
        \item Let's prove the contrapositive. Suppose that $k = 2m$ is an even integer not equal to 2, then 
        $$2^k - 1 = (3 + 1)^m - 1 = 3\sum_{n=1}^{m}\binom{m}{n}3^{n-1}$$
        is composite since it is divisible by 3 and it is not equal to 3 (the sum is not equal to 1 since $m > 1$).
    \end{enumerate}
\end{solution}

\begin{exercise}
    Find the prime factorization of the integers 1234, 10140, and 36000. \\
\end{exercise}

\begin{solution}
    Since 1234 is even, then we can write it as $2 \cdot 617$. Since 617 is prime, then the prime factorization is $1234 = 2 \cdot 617$.

    Since 10140 is divisible by 10 and $10 = 2 \cdot 5$ where both 2 and 5 are prime, then we can write 10140 as $2 \cdot 5 \cdot 1014$. Since 1014 is even, then we can write it as $2 \cdot 507$. Since 507 is divisible by 3, then we can write it as $3 \cdot 169$. Since 169 is $13^2$ where 13 is prime, then we can write $10140 = 2 \cdot 5 \cdot 2 \cdot 3 \cdot 13^2 = 2^2 \cdot 3 \cdot 5 \cdot 13^2$ where all the factors are prime numbers.

    For 36000, simply notice that
    $$36000 = 36 \cdot 1000 = 6^2 \cdot 10^3 = 2^2 \cdot 3^2 \cdot 2^3 \cdot 5^3 = 2^5 \cdot 3^2 \cdot 5^3.$$
\end{solution}

\begin{exercise}
    If $n > 1$ is an integer not of the form $6k+3$, prove that $n^2 + 2^n$ is composite. [\textit{Hint:} Show that either 2 or 3 divides $n^2 + 2^n$.] \\
\end{exercise}

\begin{solution}
    First, notice that if $n$ is even, then both $n^2$ and $2^n$ are even and so is their sum. Thus, $n^2 + 2^n$ in that case. The only cases left are $n = 6k+1$ and $6k+5$. But first, let's prove that $3 \mid 2^n + 1$ when $n$ is odd. It follows from the fact that
    \begin{align*}
        2^n + 1 &= (3-1)^n + 1 \\
        &= \sum_{l=0}^{n}\binom{n}{l}(-1)^{n-l}3^l + 1 \\
        &= 3\sum_{l=1}^{n}\binom{n}{l}(-1)^l3^{l-1} - 1 + 1 \\
        &= 3\sum_{l=1}^{n}\binom{n}{l}(-1)^l3^{l-1}.
    \end{align*}
    Hence, if $n = 6k+1$, then both terms in the sum $n^2 + 2^n = 3(12k^2 + 4k) + (2^n + 1)$ are divisible by 3 since $n$ is odd. Hence, $n^2 + 2^n$ is composite. Similarly, both terms in the sum $n^2 + 2^n = 3(12k^2 + 4k + 4) + (2^n + 1)$ are divisible by 3 and so it is composite. \\
\end{solution}

\begin{exercise}
    It has been conjectured that every even integer can be written as the difference of two consecutive primes in infinitely many ways. For example, 
    $$6 = 29 - 23 = 137 - 131 = 599 - 593 = 1019 - 1013 = \dots \ .$$
    Express the integer $10$ as the difference of two consecutive primes in fifteen ways. \\
\end{exercise}

\begin{solution}
    By looking at the tables, we get
    \begin{align*}
        10 &= 149 - 139 \\
        &= 191 - 181 \\
        &= 251 - 241 \\
        &= 293 - 283 \\
        &= 347 - 337 \\
        &= 419 - 409 \\
        &= 431 - 421 \\
        &= 557 - 547 \\
        &= 587 - 577 \\
        &= 641 - 631 \\
        &= 701 - 691 \\
        &= 719 - 709 \\
        &= 797 - 787 \\
        &= 821 - 811 \\
        &= 839 - 829.
    \end{align*}
\end{solution}

\begin{exercise}
    Prove that a positive integer $a > 1$ is a square if and only if in the canonical form of $a$ all the exponents of the primes are even integers. \\
\end{exercise}

\begin{solution}
    First, suppose that $a$ is a square, then $a = n^2$ for some integer. Write $n = p_1^{k_1}\cdot ... \cdot p_m^{k_m}$, then by the rule of exponents, we get $a = p_1^{2k_1}\cdot ... \cdot p_m^{2k_m}$ and so all the exponents are even in the canonical form of $a$. 

    Conversely, suppose that $a$ has the following canonical form: $a = p_1^{2k_1}\cdot ... \cdot p_m^{2k_m}$, then by the rules of exponents, if we let $n = p_1^{k_1}\cdot ... \cdot p_m^{k_m}$, we get that $a = n^2$ and so $a$ is a square.\\
\end{solution}

\begin{exercise}
    An integer is said to be \textit{square-free} if it is not divisible by the square of any integer greater than $1$. Prove that
    \begin{enumerate}
        \item an integer $n> 1$ is square-free if and only if $n$ can be factored into a product of disctint primes.
        \item every integer $n > 1$ is the product of a square-free integer and a perfect square. [\textit{Hint:} If $n = p_1^{k_1}p_2^{k_2}\cdot ... \cdot p_s^{k_s}$ is the canonical factorization of $n$, write $k_i = 2q_i + r_i$ where $r_i = 0$ or $1$ according as $k_i$ is even or odd.]
    \end{enumerate}
\end{exercise}

\begin{solution}
    \begin{enumerate}
        \item First, suppose that $n$ is square-free and write it as $n = p_1^{k_1}\cdot ... \cdot p_m^{k_m}$. Suppose that there is a $i$ between $1$ and $m$ such that $k_i > 1$, then we would get that $p_i^2 \mid n$ which contradicts the fact that $n$ is square-free. Thus, $k_i = 1$ for all $1 \leq i \leq m$ and so $n = p_1 \cdot \dots \cdot p_m$. Hence, $n$ is a product of distinct primes.
        
        Conversely, suppose that $n$ is a product of distinct primes, then $n = p_1 \cdot \dots \cdot p_m$. By contradiction, if $n$ is not square-free, then there is an integer $d \neq 1$ such that $d^2 \mid n$. Since $d \neq 1$, then there must be a prime $p$ such that $p \mid d$ and so $p^2 \mid n = p_1 \cdot \dots \cdot p_m$. By the property of prime numbers, there is a $i$ such that $p \mid p_i$ but since they are primes, we must have $p = p_i$. Canceling these two one both sides, we get that $p \mid p_1 \dots p_{i-1} \cdot p_{i+1} \dots p_m$. Again, by the property of prime numbers, we get that $p = p_k$ for some $k \neq i$. But this implies that $p_i = p_k$ which is impossible since we assumed that the $p_l$'s are distinct. Thus, by contradiction, $n$ is square-free.
        \item Let $n = p_1^{k_1}\cdot ... \cdot p_m^{k_m}$ be an integer. For all $k_i$, define $q_i$ and $r_i$ as the unique integers such that $k_i = 2q_i + r_i$ where $r_i = 0,1$ (by the Division Algroithm). Denote by $i_1, ..., i_s$ the integers $i$ such that $r_i = 1$, and denote by $j_1, ..., j_t$ the integers $j$ such that $r_j = 0$, then we can rewrite $n$ as
        $$(p_{i_1} \dots p_{i_s}) \cdot (p_{i_1}^{q_{i_1}} \dots p_{i_s}^{q_{i_s}} \cdot p_{j_1}^{q_{j_1}} \dots p_{j_t}^{q_{j_t}})^2.$$
        Hence, if we let $a = p_{i_1} \dots p_{i_s}$ and $b = p_{i_1}^{q_{i_1}} \dots p_{i_s}^{q_{i_s}} \cdot p_{j_1}^{q_{j_1}} \dots p_{j_t}^{q_{j_t}}$, we get that $a$ is square-free using part (a). Therefore, $n = a\cdot b^2$ where $a$ is a square-free integer.
    \end{enumerate}
\end{solution}

\begin{exercise}
    Verify that any integer $n$ can be expressed as $n = 2^k m$, where $k \geq 0$ and $m$ is an odd integer. \\
\end{exercise}

\begin{solution}
    First, write $n$ in its canonical form: $n = p_1^{k_1}\cdot ... \cdot p_s^{k_s}$ where the $p_i$'s are disctint and such that $p_i < p_{i+1}$. If $p_1 \neq 2$, then $n$ cannot be even because otherwise, $2 \mid p_1^{k_1}\cdot ... \cdot p_m^{k_m}$ implies that $2 = p_i$ for some $i$ since $2$ and the $p_i$'s are prime, but $i \neq 1$ since $p_1 \neq 2$ and $i > 1$ because we would get $2 < p_1 < p_i = 2$. Thus, $n$ is odd and so we can write $n = 2^0 n$ where $n$ is odd. Suppose now that $p_1 = 2$, then we can let $m = p_2^{k_2}\cdot ... \cdot p_s^{k_s}$ where $p_2 \neq 2$. As we showed above, $m$ mut be odd and so $n = 2^{k_1}m$ where $m$ is odd and $k_1 \geq 0$. \\
\end{solution}

\begin{exercise}
    Numerical evidences makes it plausible that there are infinitely many primes $p$ such that $p + 50$ is also prime. List fifteen of these primes. \\
\end{exercise}

\begin{solution}
    By looking at the tables, we get that the following primes $p$ are such that $p + 50$ is also a prime:
    $$3, \ 11, \ 17, \ 23, \ 29, \ 47, \ 53, \ 59, \ 89, \ 101, \ 107, \ 113, \ 131, \ 149, \ 173.$$
\end{solution}
\section{The Sieve of Eratosthenes}

\begin{exercise}
    Determine whether the integer $701$ is prime by testing all primes $p \leq \sqrt{701}$ as possible divisors. Do the same for the integer $1009$. \\
\end{exercise}

\begin{solution}
    We know that $701$ is between $26^2 = 676$ and $27^2 = 729$, hence, the primes $p$ less than $\sqrt{701}$ are precisely $2$, $3$, $5$, $7$, $11$, $13$, $17$, $19$ and $23$. We can easily see that $701$ is not divisible by $2$, $3$, $5$ or $7$. When $p = 11$, we have $701 = 11 \cdot 63 + 8$ so $11$ doesn't divide $701$. When $p = 13$, we have $701 = 13 \cdot 53 + 12$ so $13$ doesn't divide $701$. When $p = 17$, we have $701 = 17 \cdot 41 + 4$ so $17$ doesn't divide $701$. When $p = 19$, we have $701 = 19 \cdot 36 + 17$ so $19$ doesn't divide $701$. Finally, when $p = 23$, we have $701 = 23 \cdot 30 + 11$ so $23$ doesn't divide $701$. Therefore, $701$ is a prime number.

    Let's apply the same method to determine if $1009$ is a prime number. We know that $1009$ is between $31^2 = 961$ and $32^2 = 1024$, hence, the primes $p$ less than $\sqrt{1009}$ are precisely $2$, $3$, $5$, $7$, $11$, $13$, $17$, $19$, $23$ and $29$. We can easily see that $1009$ is not divisible by $2$, $3$ and $5$. When $p = 7$, we have $1009 = 7 \cdot 144 + 1$ so $7$ doesn't divide $1009$. When $p = 11$, we have $1009 = 11 \cdot 91 + 8$ so $11$ doesn't divide $1009$. When $p = 13$, we have $1009 = 13 \cdot 77 + 8$ so $13$ doesn't divide $1009$. When $p = 17$, we have $1009 = 17 \cdot 59 + 6$ so $17$ doesn't divide $1009$. When $p = 19$, we have $1009 = 19 \cdot 53 + 2$ so $19$ doesn't divide $1009$. When $p = 23$, we have $1009 = 23 \cdot 43 + 20$ so $23$ doesn't divide $1009$. Finally, when $p = 29$, we have $1009 = 29 \cdot 34 + 23$ so $29$ doesn't divide $1009$. Therefore, $1009$ is a prime number. \\
\end{solution}

\begin{exercise}
    Employing the Sieve of Eratosthenes, obtain all the primes between 100 and 200. \\
\end{exercise}

\begin{solution}
    Let's put all the numbers between 1 and 200 in a table and put in red the numbers that are removed as described by the algorithm:
    \begin{center}
        \begin{tabular}{ |c|c|c|c|c|c|c|c|c|c| } 
        \hline
         & 2 & 3 & \textcolor{red}{4} & 5 & \textcolor{red}{6} & 7 & \textcolor{red}{8} & \textcolor{red}{9} & \textcolor{red}{10} \\ 
        \hline
        11 & \textcolor{red}{12} & 13 & \textcolor{red}{14} & \textcolor{red}{15} & \textcolor{red}{16} & 17 & \textcolor{red}{18} & 19 & \textcolor{red}{20} \\ 
        \hline
        \textcolor{red}{21} & \textcolor{red}{22} & 23 & \textcolor{red}{24} & \textcolor{red}{25} & \textcolor{red}{26} & \textcolor{red}{27} & \textcolor{red}{28} & 29 & \textcolor{red}{30} \\ 
        \hline
        31 & \textcolor{red}{32} & \textcolor{red}{33} & \textcolor{red}{34} & \textcolor{red}{35} & \textcolor{red}{36} & 37 & \textcolor{red}{38} & \textcolor{red}{39} & \textcolor{red}{40} \\ 
        \hline
        41 & \textcolor{red}{42} & 43 & \textcolor{red}{44} & \textcolor{red}{45} & \textcolor{red}{46} & 47 & \textcolor{red}{48} & \textcolor{red}{49} & \textcolor{red}{50} \\ 
        \hline
        \textcolor{red}{51} & \textcolor{red}{52} & 53 & \textcolor{red}{54} & \textcolor{red}{55} & \textcolor{red}{56} & \textcolor{red}{57} & \textcolor{red}{58} & 59 & \textcolor{red}{60} \\ 
        \hline
        61 & \textcolor{red}{62} & \textcolor{red}{63} & \textcolor{red}{64} & \textcolor{red}{65} & \textcolor{red}{66} & 67 & \textcolor{red}{68} & \textcolor{red}{69} & \textcolor{red}{70} \\ 
        \hline
        71 & \textcolor{red}{72} & 73 & \textcolor{red}{74} & \textcolor{red}{75} & \textcolor{red}{76} & \textcolor{red}{77} & \textcolor{red}{78} & 79 & \textcolor{red}{80} \\
        \hline
        \textcolor{red}{81} & \textcolor{red}{82} & 83 & \textcolor{red}{84} & \textcolor{red}{85} & \textcolor{red}{86} & \textcolor{red}{87} & \textcolor{red}{88} & 89 & \textcolor{red}{90} \\ 
        \hline
        \textcolor{red}{91} & \textcolor{red}{92} & \textcolor{red}{93} & \textcolor{red}{94} & \textcolor{red}{95} & \textcolor{red}{96} & 97 & \textcolor{red}{98} & \textcolor{red}{99} & \textcolor{red}{100} \\ 
        \hline
        101 & \textcolor{red}{102} & 103 & \textcolor{red}{104} & \textcolor{red}{105} & \textcolor{red}{106} & 107 & \textcolor{red}{108} & 109 & \textcolor{red}{110} \\ 
        \hline
        \textcolor{red}{111} & \textcolor{red}{112} & 113 & \textcolor{red}{114} & \textcolor{red}{115} & \textcolor{red}{116} & \textcolor{red}{117} & \textcolor{red}{118} & \textcolor{red}{119} & \textcolor{red}{120} \\ 
        \hline
        \textcolor{red}{121} & \textcolor{red}{122} & \textcolor{red}{123} & \textcolor{red}{124} & \textcolor{red}{125} & \textcolor{red}{126} & 127 & \textcolor{red}{128} & \textcolor{red}{129} & \textcolor{red}{130} \\ 
        \hline
        131 & \textcolor{red}{132} & \textcolor{red}{133} & \textcolor{red}{134} & \textcolor{red}{135} & \textcolor{red}{136} & 137 & \textcolor{red}{138} & 139 & \textcolor{red}{140} \\ 
        \hline
        \textcolor{red}{141} & \textcolor{red}{142} & \textcolor{red}{143} & \textcolor{red}{144} & \textcolor{red}{145} & \textcolor{red}{146} & \textcolor{red}{147} & \textcolor{red}{148} & 149 & \textcolor{red}{150} \\ 
        \hline
        151 & \textcolor{red}{152} & \textcolor{red}{153} & \textcolor{red}{154} & \textcolor{red}{155} & \textcolor{red}{156} & 157 & \textcolor{red}{158} & \textcolor{red}{159} & \textcolor{red}{160} \\ 
        \hline
        \textcolor{red}{161} & \textcolor{red}{162} & 163 & \textcolor{red}{164} & \textcolor{red}{165} & \textcolor{red}{166} & 167 & \textcolor{red}{168} & \textcolor{red}{169} & \textcolor{red}{170} \\ 
        \hline
        \textcolor{red}{171} & \textcolor{red}{172} & 173 & \textcolor{red}{174} & \textcolor{red}{175} & \textcolor{red}{176} & \textcolor{red}{177} & \textcolor{red}{178} & 179 & \textcolor{red}{180} \\
        \hline
        181 & \textcolor{red}{182} & \textcolor{red}{183} & \textcolor{red}{184} & \textcolor{red}{185} & \textcolor{red}{186} & \textcolor{red}{187} & \textcolor{red}{188} & \textcolor{red}{189} & \textcolor{red}{190} \\ 
        \hline
        191 & \textcolor{red}{192} & 193 & \textcolor{red}{194} & \textcolor{red}{195} & \textcolor{red}{196} & 197 & \textcolor{red}{198} & 199 & \textcolor{red}{200} \\ 
        \hline
    \end{tabular}
    \end{center}

    Therefore, the prime numbers that are between 100 and 200 are:
    $$101, \ 103, \ 107, \ 109, \ 113, \ 127, \ 131, \ 137, \ 139, \ 149, \ 151,$$
    $$157, \ 163, \ 167, \ 173, \ 179, \ 181, \ 191, \ 193, \ 197, \ 199.$$
\end{solution}

\begin{exercise}
    Given that $p \mid n$ for all primes $p \leq \sqrt[3]{n}$, show that $n$ is either a prime or the product of two primes. [\textit{Hint:} Assume to the contrary that $n$ contains at least three prime factors.] \\
\end{exercise}

\begin{solution}
    Suppose that there are three prime numbers $p_1$, $p_2$ and $p_3$ such that $p_1p_2p_3 \mid n$. From the assumption in the statement of this exercice, we must have $p_1, p_2, p_3 > \sqrt[3]{n}$. Multiplying these three inequalities together gives us $p_1p_2p_3 > n$ which is impossible since $p_1p_2p_3$ divides $n$. Therefore, $n$ must be factored into at most two primes. \\
\end{solution}

\begin{exercise}
    Establish the following facts:
    \begin{enumerate}
        \item $\sqrt{p}$ is irrational for any prime $p$.
        \item If $a > 0$ and $\sqrt[n]{a}$ is rational, then $\sqrt[n]{a}$ must be an integer.
        \item For $n \geq 2$, $\sqrt[n]{n}$ is irrational. [\textit{Hint:} Use the fact that $2^n > n$.]
    \end{enumerate}
\end{exercise}

\begin{solution}
    \begin{enumerate}
        \item By contradiction, suppose that there exist integers $a$ and $b$ such that $\sqrt{p} = a/b$. By the Well Ordering Principle, we can assume that $a$ and $b$ are relatively prime. As a consequence, we get that in the canonical factorizations 
        $$a = p_{i_1}^{k_{i_1}} \dots p_{i_s}^{k_{i_s}} \quad  \text{and} \quad b = p_{j_1}^{k_{j_1}} \dots p_{j_t}^{k_{j_t}},$$
        none of the $p_{i_r}$'s are equal to the $p_{j_r}$'s. Using these canonical factorizations, we can rewrite our previous equation as follows:
        $$p \cdot p_{j_1}^{2k_{j_1}} \dots p_{j_t}^{2k_{j_t}} = p_{i_1}^{2k_{i_1}} \dots p_{i_s}^{2k_{i_s}}.$$
        From this, we get that $p = p_{i_r}$ for some $1 \leq r \leq s$. By the uniqueness of the canonical factorization, since there is an even number of $p$'s on the right hand side of the equation, then there must be an even number of $p$'s on the left hand side of the equation. However, none of the $p_{j_n}$'s are equal to $p_{i_r}$ and hence, there is only one $p$ on the left hand side of the equation. Therefore, by contradiction, $\sqrt{p}$ must be irrational.
        \item Since $\sqrt[n]{a}$ is rational, then there exist positive integers $c$ and $d$ such that $\sqrt[n]{a} = c/d$. By the Well Ordering Principle, we can assume that $c$ and $d$ are relatively prime, and so they have no common divisor. Suppose by contradiction that $d \neq 1$, then there exists a prime number $p$ such that $p \mid d$. If we rewrite the equation $\sqrt[n]{a} = c/d$ as $a \cdot d^n = a^n$, then $p \mid d$ implies that $p \mid a \cdot d^n = c^n$. By properties of prime numbers, it follows that $p \mid c$. But this is impossible because we get that $p \neq 1$ is a common divisor of $c$ and $d$. Therefore, by contradiction, we must have that $d = 1$ which means that $\sqrt[n]{a}$ is an integer.
        \item By contradiction, suppose that $\sqrt[n]{n}$ is rational, then by part (b), there is an integer $k$ such that $k^n = n$. We have that $k \neq 1$ because otherwise, $n = 1$ which is false. Hence, $k \geq 2$ which implies that $n < 2^n \leq k^n = n$, a contradiction. Therefore, $\sqrt[n]{n}$ is irrational.
    \end{enumerate}
\end{solution}

\begin{exercise}
    Show that any composite three-digit number must have a prime factor less than or equal to $31$. \\
\end{exercise}

\begin{solution}
    Let $n$ be a three-digit composite number, then it must satisfy $n \leq 1000$ and so $\sqrt{n} \leq \sqrt{1000}$. Moreover, $n$ must have a prime factor $p \leq \sqrt{n} \leq \sqrt{1000}$. Since $1000$ is between $31^2 = 961$ and $32^2 = 1024$, and $p$ is an integer, then $p$ must be less than or equal to $31$. \\ 
\end{solution}

\begin{exercise}
    Fill in any missing details in this sketch of a proof of the infinitude of primes: Assume that there are only finitely many primes, say $p_1$, $p_2$, ..., $p_n$. Let $A$ be the product of any $r$ of these primes and put $B = p_1p_2\dots p_n/A$. Then each $p_k$ divides either $A$ or $B$, but not both. Since $A + B > 1$, $A+B$ has a prime divisor different from any of the $p_k$, a contradiction. \\
\end{exercise}

\begin{solution}
    First, let's prove that each $p_k$ divides either $A$ or $B$ but not both. First, since the list $p_1$, ..., $p_n$ is a list of distinct primes, then $p_i \neq p_j$ whenever $i \neq j$. By construction of $A$, there are two possibilities, either $p_k$ is in the product of $r$ primes that constitutes $A$ and so $p_k \mid A$, either it is not. In that case, $p_k$ doesn't divide $A$ since otherwise, it would be equal to one of the primes constituing $A$ which is impossible. Hence, since $p_k \mid p_1 \dots p_n = AB$, then $p_k \mid B$ since it doesn't divide $A$. Therefore, as we saw from these two cases, $p_k$ must divide either $A$ or $B$.Suppose now that it divides both $A$ and $B$, then we must have a contradiction because by construction, $A$ is composed of primes distinct than $B$.
    
    Let's show that each $p_k$ cannot divide $A+B$. By contradiction suppose that $p_k \mid A+B$ and assume without loss of generality that $p_k \mid A$, then $p_k \mid (A+B) - A = B$ which is impossible since $p_k$ cannot divide both.\\
\end{solution}

\begin{exercise}
    Modify Euclid's proof that there are infinitely many primes by assuming the existence of a largest prime $p$ and using the integer $N = p! + 1$ to arrive at a contradiction. \\
\end{exercise}

\begin{solution}
    Suppose that there is a largest prime number $p$ and define the integer $N = p! + 1$. Since $N > 1$, then there must be a prime number $q$ that divides $N$. But since $q \leq p$, then $q \mid p!$ and so $q \mid N - p! = 1$ which is impossible. Therefore, by contradiction, there is no largest prime number. \\
\end{solution}

\begin{exercise}
    Give another proof of the infinitude of primes by assuming that there are only finitely many primes, say $p_1$, $p_2$, ..., $p_n$, and using the integer
    $$N = p_2p_3\dots p_n + p_1p_3\dots p_n + \dots + p_1p_2\dots p_{n-1}$$
    to arrive at a contradiction. \\
\end{exercise}

\begin{solution}
    Suppose that there are finitely many primes $p_1$, $p_2$, ..., $p_n$ and define the integer  
    $$N = p_2p_3\dots p_n + p_1p_3\dots p_n + \dots + p_1p_2\dots p_{n-1},$$
    then from the fact that $N > 1$, there must be a $p_k$ such that $p_k \mid N$. by construction of $N$, $p_k$ divides every term of the form $p_1 \dots p_{i-1} p_{i+1} \dots p_n$ except when $i = k$. Hence,
    $$p_k \mid N - \sum_{i\neq k}p_1 \dots p_{i-1} p_{i+1} \dots p_n = p_1 \dots p_{k-1} p_{k+1} \dots p_n.$$
    It follows that $p_k = p_t$ for some $t \neq k$ which is impossible since the $p_j$'s are distinct. Therefore, by contradiction, there are infinitely many primes. \\
\end{solution}

\begin{exercise}
    \begin{enumerate}
        \item Prove that if $n > 2$, then there exists a prime $p$ satisfying $n < p < n!$. [\textit{Hint:} If $n! - 1$ is not a prime, then it has a prime divisor $p$; and $p \neq n$ implies $p \mid n!$, leading to a contradiction.] 
        \item For $n > 1$, show that every prime divisor of $n! + 1$ is an odd integer greater than $n$.
    \end{enumerate}
\end{exercise}

\begin{solution}
    \begin{enumerate}
        \item Let $n > 2$ and consider the number $n! - 1$. If it is a prime, then we are done. If it is isn't, then $n!-1 > 1$ implies that there exist a prime $p$ that divides it. If $p \leq n$, then $p \mid n!$ and so $p \mid n! - (n!-1) = 1$, a contradiction. Therefore, in all cases, there is a prime $p$ satisfying $n < p < n!$.
        \item Let $p$ be a prime number dividing $n! + 1$ where $n > 1$. If $p = 2$, then $p \leq n$ and so $p \mid n!$. It follows that $p \mid (n! + 1) - n! = 1$, a contradiction. Therefore, $p$ must be odd. More generally, if $p \leq n$, then $p \mid (n!+1) - n! = 1$ which is again a contradiction. Therefore, $p$ must be odd and greater than $n$.
    \end{enumerate}
\end{solution}

\begin{exercise}
    Let $q_n$ be the smallest prime which is strictly greater than $P_n = p_1p_2 \dots p_n + 1$. It has been conjectured that the difference $(p_1p_2 \dots p_n) - q_n$ is always a prime. Confirm this for the first five values of $n$. \\
\end{exercise}

\begin{solution}
    When $n = 1$, we have $P_1 = p_1 + 1 = 3$ and $q_1 = 5$. Hence, $q_1 - p_1 = 3$ which is a prime number. When $n = 2$, we have $P_2 = p_1p_2 + 1 = 7$ and $q_2 = 11$. Hence, $q_2 - p_1p_2 = 5$ which is a prime number. When $n = 3$, we have $P_3 = p_1p_2p_3 + 1 = 31$ and $q_3 = 37$. Hence, $q_3 - p_1p_2p_3 = 7$ which is a prime number. When $n = 4$, we have $P_4 = p_1p_2p_3p_4 + 1 = 211$ and $q_4 = 223$. Hence, $q_4 - p_1p_2p_3p_4 = 23$ which is a prime number. Finally, when $n = 5$, we have $P_5 = p_1p_2p_3p_4p_5 + 1 = 2311$ and $q_5 = 2333$. Hence, $q_5 - p_1p_2p_3p_4p_5 = 23$ which is a prime number. Therefore, the conjecture holds for $n = 1,2,3,4,5$. \\
\end{solution}

\begin{exercise}
    If $p_n$ denotes the $n$th prime number, put $d_n = p_{n+1} - p_n$. An open question is whether the equation $d_n = d_{n+1}$ has infinitely many solutions; give five solutions. \\
\end{exercise}

\begin{solution}
    When $n = 2$, we have $d_2 = p_3 - p_2 = 5 - 3 = 2$, and $d_3 = p_4 - p_3 = 7 - 5 = 2$. Hence, we get $d_2 = d_3$. When $n = 15$, we have $d_{15} = p_{16} - p_{15} = 53 - 47 = 6$, and $d_{16} = p_{17} - p_{16} = 59 - 53 = 6$. Hence, we get $d_{15} = d_{16}$. When $n = 36$, we have $d_{36} = p_{37} - p_{36} = 157 - 151 = 6$, and $d_{37} = p_{38} - p_{37} = 163 - 157 = 6$. Hence, we get $d_{36} = d_{37}$. When $n = 39$, we have $d_{39} = p_{40} - p_{39} = 173 - 167 = 6$, and $d_{41} = p_{42} - p_{41} = 179 - 173 = 6$. Hence, we get $d_{39} = d_{40}$. When $n = 46$, we have $d_{46} = p_{47} - p_{46} = 211 - 199 = 12$, and $d_{47} = p_{48} - p_{47} = 223 - 211 = 12$. Hence, we get $d_{46} = d_{47}$. Therefore, $n = 2, 15, 36, 39, 46$ are all solutions of the equation $d_n = d_{n+1}$. \\
\end{solution}

\begin{exercise}
    Assuming that $p_n$ is the $n$th prime number, establish each of the following statements:
    \begin{enumerate}
        \item $p_n > 2n - 1$ for $n \geq 5$.
        \item None of the integers $P_n = p_1p_2\dots p_n + 1$ is a perfect square. [\textit{Hint:} Each $P_n$ is of the form $4k+3$.]
        \item The sum
        $$\frac{1}{p_1} + \frac{1}{p_2} + \dots + \frac{1}{p_n}$$
        is never an integer.
    \end{enumerate}
\end{exercise}

\begin{solution}
    \begin{enumerate}
        \item Let $n \geq 5$ be an integer and notice that there are $n - 1$ odd integers less than $2n-1$. Since prime numbers are all odd except two, then we get that there are at most $n$ prime numbers less than $2n-1$. However, if we add the fact that $2n-1 \geq 9$ and that $9$ is not a prime number, we can lower the upper bound and get that there are at most $n-1$ prime numbers less than or equal to $2n-1$. It follows that $p_n > 2n-1$ since otherwise, we would get that there are at least $n$ primes less than $2n-1$.
        \item If we consider the two cases $m = 2k$ and $m = 2k+1$, we get that $m^2 = 4k_0$ or $m^2 = 4k_1 + 1$. Hence, in general, squares are either of the form $4k$ or $4k+1$. If we let $n$ be an integer, then the integer $P_n = p_1p_2 \dots p_n + 1$ has the form $4k+3$ because $p_1 = 2$, all the $p_i$'s are odd for $i > 1$ and so $p_2p_3 \dots p_n = 2k+1$. It follows that $P_n = 2(2k+1) + 1 = 4k+3$. Therefore, $P_n$ cannot be a square.
        \item By contradiction, suppose that the sum of fractions is an integer, then equiva-lently, if we add the fractions together, we get that
        $$\frac{p_2p_3\dots p_n + p_1p_3\dots p_n + \dots + p_1p_2\dots p_{n-1}}{p_1p_2\dots p_n}$$
        is an integer, and hence that $p_1p_2\dots p_n \mid N$ where $N = p_2p_3\dots p_n + p_1p_3\dots p_n + \dots + p_1p_2\dots p_{n-1}$. It follows that $p_1 \mid N$. But notice that $p_1$ divides all the terms in the definition of $N$ except the first one, it follows that
        $$p_1 \mid N - \sum_{i=2}^{n}p_1 \dots p_{i-1} p_{i+1} \dots p_n = p_2p_3\dots p_n.$$
        Since $p_1$ is a prime number, then $p_1 \mid p_i$ for some $i \neq 1$. Since $p_i$ is a prime, then $p_1 = p_i$. But this is a contradiction since the $p_j$'s are distinct. Therefore, the original sum of fractions cannot be an integer.
    \end{enumerate}
\end{solution}

\begin{exercise}
    \begin{enumerate}
        \item For the repunits $R_n$, prove that if $k \mid n$, then $R_k \mid R_n$. [\textit{Hint:} If $n = kr$, consider the identity
        $$x^n - 1 = (x^k - 1)(x^{(r-1)k} + x^{(r-2)k} + \dots + x^k + 1) .]$$
        \item Use part (a) to obtain the prime factors of the repunit $R_{10}$.
    \end{enumerate}
\end{exercise}

\begin{solution}
    \begin{enumerate}
        \item In the identity
        $$x^n - 1 = (x^k - 1)(x^{(r-1)k} + x^{(r-2)k} + \dots + x^k + 1),$$
        replace $x$ by $10$ and divide both sides by $9$ to obtain
        $$R_n = R_k (1 + 10^k + 10^{2k} + \dots + 10^{(r-1)k}).$$
        It directly follows that $R_k \mid R_n$.
        \item From part (a), we have that $R_{10}$ is divisible by both $R_2 = 11$ and $R_5 = 41 \cdot 271$. It follows that $R_{10} = 11 \cdot 41 \cdot 271 \cdot 9091$. Since $9091$ is a prime number, then we are done.
    \end{enumerate}
\end{solution}
\section{The Goldbach Conjecture}

\begin{exercise}
    Verify that the integers $1949$ and $1951$ are twin primes. \\
\end{exercise}

\begin{solution}
    Let's show that both integers are prime numbers by proving that none of the prime less than their square roots are divisors. Since $44^2 < 1949, 1951 < 45^2$, then it suffices to consider the primes that are less than $44$: $2$, $3$, $5$, $7$, $11$, $13$, $17$, $19$, $23$, $29$, $31$, $37$, $41$, $43$. Obviously, both integers are not divisible by $2$, $3$ and $5$.
    \begin{align*}
        1949&=7\cdot 278 + 1 & 1951&=7\cdot 278 + 3 \\
        1949&=11\cdot 177 + 2 & 1951&=11\cdot 177 + 4\\
        1949&=13\cdot 149 + 12 & 1951&=13\cdot 150 + 1\\
        1949&=17\cdot 114 + 11 & 1951&=17\cdot 114 + 13\\
        1949&=19\cdot 102 + 11 & 1951&=19\cdot 102 + 13\\
        1949&=23\cdot 88 + 17 & 1951&=23\cdot 88 + 19\\
        1949&=29\cdot 67 + 3 & 1951&=29\cdot 67 + 5\\
        1949&=31\cdot 62 + 27 & 1951&=31\cdot 62 + 29\\
        1949&=37\cdot 52 + 25 & 1951&=37\cdot 52 + 27\\
        1949&=41\cdot 47 + 22 & 1951&=41\cdot 47 + 24\\
        1949&=43\cdot 45 + 14 & 1951&=43\cdot 45 + 16.
    \end{align*}
    As it can be seen from the previous equations, none of these primes divide the two integers. Therefore, they form a pair of twin primes. \\
\end{solution}

\begin{exercise}
    \begin{enumerate}
        \item If $1$ is added to a product of twin primesn prove that a perfect square is always obtained.
        \item Show that the sum of twin primes $p$ and $p+2$ is divisible by $12$, provided that $p > 3$.
    \end{enumerate}
\end{exercise}

\begin{solution}
    \begin{enumerate}
        \item Let $p$ and $q$ be twin primes, then there exists an integer $n$ such that $p = n-1$ and $q = n+1$. It follows that $pq + 1 = (n-1)(n+1) + 1 = (n^2 - 1) + 1 = n^2$ which is a perfect square. 
        \item First, since $p > 3$, then $p$ mustbe odd since the only even prime is $2$. Moreover, $p$ cannot be of the form $3k$ since $p \neq 3$. Thus, either $p$ is of the form $3k + 1$ or of the form $3k+2$. However, $p+2$ is also a prime by our assumption and so if $p = 3k+1$, then $p+2 = 3(k+1)$ which is divisible than 3 and distinct than 3, a contradiction. It follows that $p = 3k+2$. Therefore, $p + 1$ is divisible by both 2 and 3 and since $\gcd(2,3) = 1$, then $6 \mid p+1$. It follows that $p + (p+2) = 2(p+1)$ is divisible by 12.
    \end{enumerate}
\end{solution}

\begin{exercise}
    Find all pairs of primes $p$ and $q$ satisfying $p - q = 3$. \\
\end{exercise}

\begin{solution}
    Notice that if $q$ is even, then $p = q+3$ is odd, and if $q$ is odd, then $p = q+3$ is even. It follows that $p$ and $q$ don't have the same parity. But the only even prime is 2 so the only possible pair is $p = 5$ and $q = 2$. \\
\end{solution}

\begin{exercise}
    Sylvester (1896) rephrased Goldbach's Conjecture so as to read: Every even integer $2n$ greater than $4$ is the sum of twin primes, one larger than $n/2$ and the other less than $3n/2$. Verify this version of the conjecture for all even integers between $6$ and $76$. \\
\end{exercise}

\begin{solution}
    \begin{center}
        \begin{tabular}{ |c|c|c|c|c|c|c| } 
        \hline 
        $n$ & $2n$ & $n/2$ & $3n/2$ & twin pair $p \geq n/2$ & twin pair $q\leq 3n/2$ & $p+q$\\
        \hline
        3 & 6 & 1.5 & 4.5 & 3 & 3 & 6 \\
        \hline
        4 & 8 & 2 & 6 & 3 & 5 & 8 \\
        \hline
        5 & 10 & 2.5 & 7.5 & 5 & 5 & 10 \\
        \hline
        6 & 12 & 3 & 9 & 5 & 7 & 12 \\
        \hline
        7 & 14 & 3.5 & 10.5 & 7 & 7 & 14 \\
        \hline
        8 & 16 & 4 & 12 & 5 & 11 & 16 \\
        \hline
        9 & 18 & 4.5 & 13.5 & 5 & 13 & 18 \\
        \hline
        10 & 20 & 5 & 15 & 7 & 13 & 20 \\
        \hline
        11 & 22 & 5.5 & 16.5 & 11 & 11 & 22 \\
        \hline
        12 & 24 & 6 & 18 & 11 & 13 & 24 \\
        \hline
        13 & 26 & 6.5 & 19.5 & 13 & 13 & 26 \\
        \hline
        14 & 28 & 7 & 21 & 11 & 17 & 28 \\
        \hline
        15 & 30 & 7.5 & 22.5 & 11 & 19 & 30 \\
        \hline
        16 & 32 & 8 & 24 & 13 & 19 & 32 \\
        \hline
        17 & 34 & 8.5 & 25.5 & 31 & 3 & 34 \\
        \hline
        18 & 36 & 9 & 27 & 31 & 5 & 36 \\
        \hline
        19 & 38 & 9.5 & 28.5 & 31 & 7 & 38 \\
        \hline
        20 & 40 & 10 & 30 & 29 & 11 & 40 \\
        \hline
        21 & 42 & 10.5 & 31.5 & 31 & 11 & 42 \\
        \hline
        22 & 44 & 11 & 33 & 31 & 13 & 44 \\
        \hline
        23 & 46 & 11.5 & 34.5 & 41 & 5 & 46 \\
        \hline
        24 & 48 & 12 & 36 & 41 & 7 & 48 \\
        \hline
        25 & 50 & 12.5 & 37.5 & 43 & 7 & 50 \\
        \hline
        26 & 52 & 13 & 39 & 41 & 11 & 52 \\
        \hline
        27 & 54 & 13.5 & 40.5 & 41 & 13 & 54 \\
        \hline
        28 & 56 & 14 & 42 & 43 & 13 & 56\\
        \hline
        29 & 58 & 14.5 & 43.5 & 41 & 17 & 58 \\
        \hline
        30 & 60 & 15 & 45 & 43 & 17 & 60 \\
        \hline
        31 & 62 & 15.5 & 46.5 & 43 & 19 & 62 \\
        \hline
        32 & 64 & 16 & 48 & 59 & 5 & 64 \\
        \hline
        33 & 66 & 16.5 & 49.5 & 59 & 7 & 66 \\
        \hline
        34 & 68 & 17 & 51 & 61 & 7 & 68 \\
        \hline
        35 & 70 & 17.5 & 52.5 & 41 & 29 & 70 \\
        \hline
        36 & 72 & 18 & 54 & 41 & 31 & 72 \\
        \hline
        37 & 74 & 18.5 & 55.5 & 43 & 31 & 74 \\
        \hline
        38 & 76 & 19 & 57 & 71 & 5 & 76 \\
        \hline
    \end{tabular}
    \end{center}
\end{solution}

\begin{exercise}
    In 1752, Goldbach submitted the following conjecture to Euler: Every odd integer can be written in the form $p + 2a^2$, where $p$ is either a prime or 1s and $a \geq 0$. Show that the integer $5777$ refutes this conjecture. \\
\end{exercise}

\begin{solution}
    To show that 5777 refutes the conjecture, let's show that $5777 - 2a^2$ is never a prime number. First, notice that if the conjecture is true, then $a$ should be contained in the interval $a = 0$, $a = 53$ since $2 \cdot 53^2 = 5618$ and $2 \cdot 54^2 = 5832$. Hence, we need to show that $5777 - 2a^2$ is not a prime for all $0 \leq a \leq 53$:
    \begin{center}
    \begin{tabular}{ |c|c|c|c|c|c|c|c| }
        \hline $a$ & $2a^2$ & $5777 - 2a^2$ & Prime ? & $a$ & $2a^2$ & $5777 - 2a^2$ & Prime ? \\
        \hline 0 & 0 & 5777 & No & 27 & 1458 & 4319 & No \\
        \hline 1 & 2 & 5775 & No & 28 & 1568 & 4209 & No \\
        \hline 2 & 8 & 5769 & No & 29 & 1682 & 4095 & No \\
        \hline 3 & 18 & 5759 & No & 30 & 1800 & 3977 & No \\
        \hline 4 & 32 & 5745 & No & 31 & 1922 & 3855 & No \\
        \hline 5 & 50 & 5727 & No & 32 & 2048 & 3729 & No \\
        \hline 6 & 72 & 5705 & No & 33 & 2178 & 3599 & No \\
        \hline 7 & 98 & 5679 & No & 34 & 2312 & 3465 & No \\
        \hline 8 & 128 & 5649 & No & 35 & 2450 & 3327 & No \\
        \hline 9 & 162 & 5615 & No & 36 & 2592 & 3185 & No \\
        \hline 10 & 200 & 5577 & No & 37 & 2738 & 3039 & No \\
        \hline 11 & 242 & 5535 & No & 38 & 2888 & 2889 & No \\
        \hline 12 & 288 & 5489 & No & 39 & 3042 & 2735 & No \\
        \hline 13 & 338 & 5439 & No & 40 & 3200 & 2577 & No \\
        \hline 14 & 392 & 5385 & No & 41 & 3362 & 2415 & No \\
        \hline 15 & 450 & 5327 & No & 42 & 3528 & 2249 & No\\
        \hline 16 & 512 & 5265 & No & 43 & 3698 & 2079 & No \\
        \hline 17 & 578 & 5199 & No & 44 & 3872 & 1905 & No \\
        \hline 18 & 648 & 5129 & No & 45 & 4050 & 1727 & No \\
        \hline 19 & 722 & 5055 & No & 46 & 4232 & 1545 & No \\
        \hline 20 & 800 & 4977 & No & 47 & 4418 & 1359 & No \\
        \hline 21 & 882 & 4895 & No & 48 & 4608 & 1169 & No \\
        \hline 22 & 968 & 4809 & No & 49 & 4802 & 975 & No \\
        \hline 23 & 1058 & 4719 & No & 50 & 5000 & 777 & No \\
        \hline 24 & 1152 & 4625 & No & 51 & 5202 & 575 & No \\
        \hline 25 & 1250 & 4527 & No & 52 & 5408 & 369 & No \\
        \hline 26 & 1352 & 4425 & No & 53 & 5618 & 159 & No \\
        \hline
    \end{tabular}
    \end{center}
    Therefore, 5777 refutes the conjecture. \\
\end{solution}

\begin{exercise}
    Prove that Goldbach's Conjecture that every even integer greater than 2 is the sum of two primes is equivalent to the statement that every integer greater than 5 is the sum of three primes. [\textit{Hint:} If $2n-2 = p_1 + p_2$, then $2n = p_1 + p_2 + 2$ and $2n + 1 = p_1 + p_2 + 3$.] \\
\end{exercise}

\begin{solution}
    Suppose that Goldbach's Conjecture is true and let $n \geq 3$, then there exist two prime numbers $p_1$ and $p_2$ such that $2n - 2 = p_1 + p_2$. It follows that $2n = p_1 + p_2 + 2$ and $2n + 1 = p_1 + p_2 + 3$. Therefore, it holds for all integers greater than 5. Conversely, suppose that every integer greater than 5 is the sum of three primes and let $n \geq 3$ be an integer, then $2n = p_1 + p_2 + p_3$ for some prime numbers $p_1$, $p_2$ and $p_3$. Since $2n$ is even, then $p_1$, $p_2$, $p_3$ cannot all be odd and so one of them must be even. Without loss of generality, we can assume that $p_3$ is even and hence, $p_3 = 2$. Thus, $2n - 2 = p_1 + p_2$. This proves Goldbach's Conjecture.\\
\end{solution}

\begin{exercise}
    A conjecture of Lagrange (1775) asserts that every odd integer greater than 5 can be written as a sum $p_1 + 2p_2$, where $p_1,p_2$ are both primes. Confirm this for all odd integers through 75. \\
\end{exercise}

\begin{solution}
    \begin{center}
    \begin{tabular}{cccc}
        $7 = 3 + 2\cdot 2$ & $25 = 19 + 2\cdot 3$ & $43 = 37 + 2\cdot 3$ & $61 = 47 + 2\cdot 7$ \\
        $9 = 5 + 2\cdot 2$ & $27 = 23 + 2\cdot 2$ & $45 = 41 + 2\cdot 2$ & $63 = 59 + 2\cdot 2$ \\
        $11 = 7 + 2\cdot 2$ & $29 = 23 + 2\cdot 3$ & $47 = 41 + 2\cdot 3$ & $65 = 59 + 2\cdot 3$ \\
        $13 = 7 + 2\cdot 3$ & $31 = 17 + 2\cdot 7$ & $49 = 43 + 2\cdot 2$ & $67 = 61 + 2\cdot 3$ \\
        $15 = 11 + 2\cdot 2$ & $33 = 29 + 2\cdot 2$ & $51 = 47 + 2\cdot 2$ & $69 = 59 + 2\cdot 5$ \\
        $17 = 13 + 2\cdot 2$ & $35 = 31 + 2\cdot 2$ & $53 = 47 + 2 \cdot 3$ &$71 = 67 + 2\cdot 2$ \\
        $19 = 13 + 2\cdot 3$ & $37 = 31 + 2\cdot 3$ & $55 = 41 + 2\cdot 7$ & $73 = 67 + 2\cdot 3$ \\
        $21 = 17 + 2\cdot 2$ & $39 = 29 + 2\cdot 5$ & $57 = 53 + 2\cdot 2$ & $75 = 71 + 2\cdot 2$ \\
        $23 = 19 + 2\cdot 2$ & $41 = 37 + 2\cdot 2$ & $59 = 53 + 2\cdot 3$ & \\
    \end{tabular}
    \end{center}
    Therefore, the conjecture is true for all odd numbers smaller than 75.\\
\end{solution}

\begin{exercise}
    Given a positive integer $n$, it can be shown that there exists an even integer $a$ which is representable as the sum of two odd primes in $n$ different ways. Confirm that the integers $60$, $78$, and $84$ can be written as the sum of two primes in six, seven and eight ways, respectively. \\
\end{exercise}

\begin{solution}
    Simply notice that
    \begin{align*}
        60 &= 7 + 53 = 13 + 47 = 17 + 43 = 19 + 41 = 23 + 37 = 29 + 31\\
        78 &= 5 + 73 = 7 + 71 = 11 + 67 = 17 + 61 = 19 + 59 = 31 + 47 = 37 + 41\\
        84 &= 5 + 79 = 11 + 73 = 13 + 71 = 17 + 67 = 23 + 61 = 31 + 53 = 37 + 47 = 41 + 43 \\
    \end{align*}
\end{solution}

\begin{exercise}
    \begin{enumerate}
        \item For $n > 3$, show that the integers $n$, $n+2$, $n+4$ cannot all be prime.
        \item Three integers $p$, $p+2$, $p+6$ which are prime are called a \textit{prime-triplet}. Find five sets of prime-triplets.
    \end{enumerate}
\end{exercise}

\begin{solution}
    \begin{enumerate}
        \item Suppose that there is a prime number $n > 3$ such that $n + 2$ and $n+4$ are also prime. Since $n$ is prime and $n > 3$, then either $n = 3k+1$ or $n = 3k+2$. In the first case, we have $n + 2 = 3(k+1)$ which is a contradiction. Hence, $n = 3k+2$ but in that case, $n+4 = 3(k+2)$ which is again a contradiction. It follows that no such integer $n$ exists.
        \item By looking at the tables, we can find the following prime-triplets: $(5, 7, 11)$, $(11, 13, 17)$, $(17, 19, 23)$, $(41, 43, 47)$ and $(101, 103, 107)$.
    \end{enumerate}
\end{solution}

\begin{exercise}
    Establish that the sequence
    $$(n+1)!-2, \ (n+1)!-2, \ \dots, \ (n+1)!-(n+1)$$
    produces $n$ consecutive composite integers for $n > 1$. \\
\end{exercise}

\begin{solution}
    By construction, the sequence is composed of $n$ consecutive integers. Take now the term $(n+1)! - k$ in the sequence where $k$ is an integer satisfying $2 \leq k \leq n+1$. Since $k \leq n+1$, then $k \mid (n+1)!$ and hence, $k\mid (n+1)! - k$. Since $k \geq 2$, then $k$ is a non-trivial factor of $(n+1)! - k$ showing that all the terms of the sequence are composite integers.\\ 
\end{solution}

\begin{exercise}
    Find the smallest positive integer $n$ for which the function $f(n) = n^2 + n + 17$ is composite. Do the same for the functions $g(n) = n^2 + 21n + 1$ and $h(n) = 3n^2 + 3n + 23$. \\
\end{exercise}

\begin{solution}
    For the function $f$, the smallest $n$ is $n = 16$ because $f(16) = 16\cdot 17 + 17 = 17^2$ which is composite, and because $f(1) = 19$, $f(2) = 23$, $f(3) = 29$, $f(4) = 37$, $f(5) = 47$, $f(6) = 59$, $f(7) = 73$, $f(8) = 89$, $f(9) = 107$, $f(10) = 127$, $f(11) = 149$, $f(12) = 173$, $f(13) = 199$, $f(14) = 227$ and $f(15) = 257$ are all prime numbers.
    
    For the function $g$, the smallest $n$ is $n = 18$ because $f(18) = 703 = 19\cdot 37$ which is composite, and because $g(1) = 23$, $g(2) = 47$, $g(3) = 73$, $g(4) = 101$, $g(5) = 131$, $g(6) = 163$, $g(7) = 197$, $g(8) = 233$, $g(9) = 271$, $g(10) = 311$, $g(11) = 353$, $g(12) = 397$, $g(13) = 443$, $g(14) = 491$, $g(15) = 541$, $g(16) = 593$ and $g(17) = 647$ are all prime numbers.
    
    For the function $h$, the smallest $n$ is $n = 2$ because $h(22) = 1541 = 23\cdot 67$ which is composite, and because $h(1) = 29$, $h(2) = 41$, $h(3) = 59$, $h(4) = 83$, $h(5) = 113$, $h(6) = 149$, $h(7) = 191$, $h(8) = 239$, $h(9) = 293$, $h(10) = 353$, $h(11) = 419$, $h(12) = 491$, $h(13) = 569$, $h(14) = 653$, $h(15) = 743$, $h(16) = 839$, $h(17) = 941$, $h(18) = 1049$, $h(19) = 1163$, $h(20) = 1283$ and $h(21) = 1409$ are all prime numbers.\\
\end{solution}

\begin{exercise}
    The following result was conjectured by Bertrand, but first proved by Tchebychef in 1850: For every positive integer $n > 1$, there exists at least one prime $p$ satisfying $n < p < 2n$. Use Bertrand's Conjecture to show that $p_n < 2^n$, where $p_n$ is the $n$th prime number. \\
\end{exercise}

\begin{solution}
    First, define the sequence $P_n$ as $P_1 = 2$, $P_2 = 3$ and $P_n$ as the prime number satisfying $2^{n-1} < P_n < 2^n$ where $n \geq 3$. By construction, we have that $P_n$ is a strictly increasing sequence of prime numbers. Hence, since there are at least $n-1$ prime numbers less than $P_n$, we must have the inequality $p_n \leq P_n$. Therefore, by construction, we have $p_n \leq P_n < 2^n$ and so $p_n < 2^n$. \\
\end{solution}

\begin{exercise}
    Apply the same method of proof as in Theorem 3-6 to show that there are infinitely many primes of the form $6n+5$.\\
\end{exercise}

\begin{solution}
    Suppose that there are finitely many primes of the form $6n+5$, and denote them by $q_1$, $q_2$, ..., $q_n$. Consider the integer
    $$N = 6(q_1q_2\dots q_n) - 1 = 6(q_1q_2\dots q_n - 1) + 5$$
    Since $N$ is neither even, nor a multiple of $3$, then its prime factors are of the form $6n+1$ or $6n+5$. If all prime factors of $N$ are of the form $6n+1$, then the equation
    $$(6k+1)(6k'+1) = 6(6kk' + k + k') + 1$$
    tells us that $N$ must have the form $6n+1$, which is false. Therefore, there must be a prime $p$ of the form $6n+5$. By our assumption, $p = q_i$ for some $i$ and so $p\mid 6(q_1 q_2 \dots q_n)$. It follows that $p \mid 6(q_1 q_2 \dots q_n) - N = 1$ which is a contradiction since $p \neq 1$. Therefore, there are infinitely many primes of the form $6n+5$. \\
\end{solution}

\begin{exercise}
    Find a prime divisor of the integer $N = 4(3\cdot 7 \cdot 11) - 1$ of the form $4n+3$. Do the same for $N = 4(3\cdot 7 \cdot 11 \cdot 15) - 1$.\\
\end{exercise}

\begin{solution}
    We have that $4(3\cdot 7 \cdot 11) - 1 = 923$ is divisble by the prime $71$ which is of the form $4n+3$. Since $4(3\cdot 7 \cdot 11 \cdot 15) - 1$ is a prime number (it took me a lot of time to arrive at this conclusion), then it is itself a prime factor of the form $4n+3$. \\
\end{solution}

\begin{exercise}
    Another unanswered question is whether the exist an infinite number of sets of five consecutive integers of which four are primes. Find five such sets of integers. \\
\end{exercise}

\begin{solution}
    The following sets satisfy the property above: $\{3,5,7,9,11\}$, $\{11,13,15,17,19\}$, $\{101,103,105,107,109\}$, $\{191,193,195,197,199\}$ and $\{461,463,465,467,469\}$. \\
\end{solution}

\begin{exercise}
    Let the sequence of primes, with $1$ adjoined, be denoted by $p_0 = 1$, $p_1 = 2$, $p_2 = 3$, $p_3 = 5$, ... For each $n \geq 1$, it is known that there exists a suitable choice of coefficients $\epsilon_k = \pm 1$ such that
    $$p_{2n} = p_{2n-1} + \sum_{k=0}^{2n-2}\epsilon_kp_k, \quad p_{2n+1} = 2p_{2n} + \sum_{k=0}^{2n}\epsilon_k p_k .$$
    To illustrate:\\
    $13 = 1+2 - 3 - 5 + 7 - 11$ and \\
    $17 = 1 + 2 - 3 - 5 + 7  - 11 + 2\cdot 13$.\\
    Determine similar expressions for the primes $23$, $29$, $31$, and $37$.\\
\end{exercise}

\begin{solution}
    We have
    \begin{align*}
        23 &= -1 + 2 - 3 -5 + 7 - 11 + 13 - 17 + 2\cdot 19, \\
        29 &= 1 + 2 - 3 - 5 + 7 - 11 + 13 - 17 + 19 + 23,\\
        31 &= 1 - 2 + 3 + 5 - 7 + 11 - 13 + 17 - 19 - 23 + 2\cdot 29,\\
        37 &= 1 - 2 + 3 + 5 - 7 + 11 - 13 - 17 + 19 - 23 + 29 + 31.\\
    \end{align*}
\end{solution}

\begin{exercise}
    In 1848 de Polignac claimed that every odd integer is the sum of a prime and a power of $2$. For example, $55 = 47 + 2^3 = 23 + 2^5$. Show that the integers $509$ and $877$ discredit this claim. \\
\end{exercise}

\begin{solution}
    To show that $509$ is a counterexample, let's show that $509 - 2^n$ is not a prime for all $n \geq 0$. First, notice that if $n \geq 9$, $509 - 2^n < 0$ and so it cannot be a prime number. Thus, we only need to consider the values $n = 0,1, ..., 8$. When $n = 0$, $509 - 2^n = 508 = 2\cdot 254$ which is composite. When $n = 1$, $509 - 2^n = 507 = 3\cdot 169$ which is composite. When $n = 2$, $509 - 2^n = 505 = 5\cdot 101$ which is composite. When $n = 3$, $509 - 2^n = 501 = 3\cdot 167$ which is composite. When $n = 4$, $509 - 2^n = 493 = 17 \cdot 29$ which is composite. When $n = 5$, $509 - 2^n = 477 = 3\cdot 159$ which is composite. When $n = 6$, $509 - 2^n = 445 = 5\cdot 89$ which is composite. When $n = 7$, $509 - 2^n = 381 = 3\cdot 127$ which is composite. Finally, when $n = 8$, $509 - 2^n = 253 = 11\cdot 23$ which is composite. Therefore, $509$ cannot be written in the form $p + 2^n$ where $p$ is prime and $n$ is a positive integer.
    
    Similarly, to show that $877$ is a counterexample, let's show that $877 - 2^n$ is not a prime for all $n \geq 0$. First, notice that if $n \geq 10$, $877 - 2^n < 0$ and so it cannot be a prime number. Thus, we only need to consider the values $n = 0,1, ..., 9$. When $n = 0$, $877 - 2^n = 876 = 2\cdot 438$ which is composite. When $n = 1$, $877 - 2^n = 875 = 5\cdot 175$ which is composite. When $n = 2$, $877 - 2^n = 873 = 3\cdot 291$ which is composite. When $n = 3$, $877 - 2^n = 869 = 11\cdot 79$ which is composite. When $n = 4$, $877 - 2^n = 861 = 3\cdot 287$ which is composite. When $n = 5$, $877 - 2^n = 845 = 5\cdot 169$ which is composite. When $n = 6$, $877 - 2^n = 813 = 3\cdot 271$ which is composite. When $n = 7$, $877 - 2^n = 749 = 7\cdot 107$ which is composite. When $n = 8$, $877 - 2^n = 621 = 3\cdot 207$ which is composite. Finally, when $n = 9$, $877 - 2^n = 365 = 5\cdot 73$ which is composite. Therefore, $877$ cannot be written in the form $p + 2^n$ where $p$ is prime and $n$ is a positive integer. \\
\end{solution}

\begin{exercise}
    \begin{enumerate}
        \item If $p$ is a prime and $p \not \mid \ b$, prove that in the arithmetic progression
        $$a, \ a + b, \ a + 2b, \ a+3b, \ . \ . \ .$$
        every $p$th term is divisible by $p$. [\textit{Hint:} Since $\gcd(p,b) = 1$, there exists integers $r$ and $s$ satisfying $pr + bs = 1$. Put $n_k = kp - as$ for $k=1,2,...$ and show that $p \mid a + n_k b$.]
        \item From part (a), conclude that if $b$ is an odd integer, then every other term in the indicated progression is even.
    \end{enumerate}
\end{exercise}

\begin{solution}
    \begin{enumerate}
        \item Since $p$ doesn't divide $b$, then $\gcd(p,b) = 1$ and so there exist integers $r$ and $s$ such that $pr + bs = 1$. Hence, if we define the sequence $n_k = kp - as$, then $a + n_kb = a + (kp - as)b = a + bkp - abs = a + bkp - a(1 - pr) = p(bk r)$. It follows that the term $a + n_kb$ is always divisible by $p$.
        \item If we put $p = 2$ and $b$ be an odd integer, then part (a) tells us that at least one of the even-indexed terms or odd-indexed terms are even.
    \end{enumerate}
\end{solution}

\begin{exercise}
    In 1950, it was proven that any integer $n > 9$ can be written as a sum of distinct odd primes. Express the integers $25$, $69$, $81$, and $125$ in this fashion. \\
\end{exercise}

\begin{solution}
    We have
    \begin{align*}
        25 &= 5+7+13\\
        69 &= 3 + 5 + 61, \\
        81 &= 3 + 5 + 73, \\
        125 &= 5 + 7 + 113.\\
    \end{align*}
\end{solution}

\begin{exercise}
    If $p$ and $p^2 + 8$ are both prime numbers, prove that $p^3 + 4$ is also a prime. \\
\end{exercise}

\begin{solution}
    Let $p$ be a prime number such that $p^2 + 8$ is also a prime number. Consider the case in which $p = 3k+1$, then $p^2 + 8 = 3k' + 9 = 3(k' + 3)$ which is not a prime, hence, $p$ is not of the form $3k+1$. Similarly, if $p = 3k+2$, then $p^2 + 8 = 3k' + 12 = 3(k' + 4)$ which is not a prime. Thus, $p$ must be of the form $3k$. But since $p$ is prime, then $p = 3$. Therefore, $p^3 + 4 = 31$ is indeed a prime number. \\
\end{solution}

\begin{exercise}
    \begin{enumerate}
        \item For any integer $k > 0$, establish that the arithmetic progression
        $$a+b, \ a+2b, \ a+3b, \ . \ . \ . \ ,$$
        where $\gcd(a,b) = 1$, contains $k$ consecutive terms which are composite. 
        
        [\textit{Hint:} Put $n = (a+b)(a+2b)\dots (a+kb)$ and consider the $k$ terms
        $$a+(n+1)b, \ a+(n+2)b, \ . \ . \ ., \ a+(n+k)b . ] $$
        \item Find five consecutive composite terms in the arithmetic progression
        $$6, \ 11, \ 16, \ 21, \ 26, \ 31, \ 36, \ . \ . \ . $$
    \end{enumerate}
\end{exercise}

\begin{solution}
    \begin{enumerate}
        \item Let $n = (a+b)(a+2b)\dots (a+kb)$ and notice that for all $1 \leq i \leq k$, $n$ is divisible by $a+ib$. It follows that
        $$a+(n+i)b = (a+ib) + nb = (a+ib)\left(1 + b\frac{n}{a+ib}\right).$$
        This proves that the term $a+(n+i)b$ is composite.
        \item Using part (a), we get that $14894891$, $14894896$, $14894901$, $14894906$ and $14894911$ are five consecutive composite terms of the sequence.
    \end{enumerate}
\end{solution}

\begin{exercise}
    Show that $13$ is the largest prime that can divide two successive integers of the form $n^2 + 3$. \\
\end{exercise}

\begin{solution}
    Let $p$ be a prime number such that $p \mid n^2 + 3$ and $p \mid (n+1)^2 + 3$, then we can easily derive that $p \mid (n+1)^2 + 3 - n^2 - 3 = 2n+1$. It follows that $pk = 2n+1$ for some integer $k$, and so that $n = \frac{pk - 1}{2}$. Hence, we rewrite the fact that $p \mid n^2 + 3$ into the equation $pt = (\frac{pk - 1}{2})^2 + 3$ where $t$ is an integer. From this equation, we can derive
    \begin{align*}
        pt = \left(\frac{pk - 1}{2}\right)^2 + 3 &\implies 4pt = (pk - 1)^2 + 12 \\
        &\implies 4pt = p^2k^2 - 2pk + 1 + 12 \\
        &\implies p(4t - pk^2 + 2k) = 13 \\
        &\implies p \mid 13.
    \end{align*}
    Therefore, $p = 13$ since $p$ and $13$ are prime numbers.\\
\end{solution}

\begin{exercise}
    \begin{enumerate}
        \item The arithmetic mean of the twin primes $5$ and $7$ is the triangular number $6$. Are there any other twin primes with triangular mean?
        \item The arithmetic of the twin primes $3$ and $5$ is the perfect square $4$. Are there any other twin primes with a square mean?
    \end{enumerate}
\end{exercise}

\begin{solution}
    \begin{enumerate}
        \item Suppose that we have a pair of twin primes $p$ and $q$ such that $p = t_n - 1$ and $q = t_n + 1$ for some $n$, then using the formula for $t_n$, we get
        \begin{align*}
            p &= \frac{n(n+1)}{2} - 1\\
            &= \frac{n^2 + n - 2}{2} \\
            &= \frac{(n-1)(n+2)}{2}.
        \end{align*}
        Since either $n-1$ or $n+2$ is even, then we have $p = (n+2)\frac{n-1}{2}$ or $p = (n-1)\frac{n+2}{2}$. But since $p$ is a prime number, then in the first case, $p$ must be equal to 5 (by solving $\frac{n-1}{2} = 1$) and in the second case, $p$ must be equal to 2. In the second case, $p = 2$ is impossible since it is not a twin prime. Therefore, the only pair of twin primes with an arithmetic mean equal to a triangular number is the pair $p = 5$ and $q = 7$.
        \item Suppose that we have a pair of twin primes $p$ and $q$ such that $p = n^2- 1$ and $q = n^2 + 1$ for some $n$, then $p = (n-1)(n+1)$. Since $p$ is a prime, then either $n-1 = 1$ or $n + 1 = 1$. In the first case, we get $n = 2$ which implies that $p = 3$. In the second case, we get that $n = 0$ and so that $p = 0$, which is impossible. Therefore, the only twin pair with a square mean is the pair $p = 3$ and $q = 5$.
    \end{enumerate}
\end{solution}

\begin{exercise}
    Determine all twin primes $p$ and $q = p+2$ for which $pq - 2$ is also prime. \\
\end{exercise}

\begin{solution}
    First, notice that $p$ cannot be of the form $3k+1$ because this would imply that $q$ is of the form $3k'$ with $k'$, a contradiction since $q$ is prime. Thus, either $p$ is of the form $3k$ (which only happens in the case $p = 3$), or $p$ is of the form $3k+2$. When $p = 3$, we have $pq - 2 = 3\cdot 5 - 2 = 13$ which is a prime number. When $p = 3k+2$, we have
    $$pq - 2 = (3k+2)(3k+4) - 2 = 9k^2 + 18k + 6 = 3(3k^2 + 6k + 2)$$
    which is composite. Therefore, the only pair that satisfies the condition is the pair $p = 3$ and $q = 5$. \\
\end{solution}

\begin{exercise}
    Let $p_n$ denote the $n$th prime. For $n > 3$, show that
    $$p_n < p_1 + p_2 + \dots + p_{n-1}.$$
    [\textit{Hint:} Use induction and Bertrand's Conjecture.] \\
\end{exercise}

\begin{solution}
    Let's prove it by induction on $n$. When $n = 4$, we have
    $$p_1 + p_2 + p_3 = 2 + 3 + 5 = 10 > 7 = p_4.$$
    Hence, the statement holds for $n = 4$. Suppose now that there is an integer $k > 3$ such that
    $$p_k < p_1 + p_2 + \dots + p_{k-1},$$
    then equivalently, we have 
    $$0 < p_1 + p_2 + \dots + p_{k-1} - p_k.$$
    Moreover, by Bertrand's Conjecture, we have that there is a prime $p$ such that $\frac{p_{k+1} - 1}{2} < p < p_{k+1} - 1$. Since $p < p_{k+1}$, then we must have $p \leq p_k$, then we get that $\frac{p_{k+1} - 1}{2} < p_k$ and so that $p_{k+1} - 1 < 2p_k$. Since both $p_{k+1} - 1$ and $2p_k$ are even, then we get that $p_{k+1} < 2p_k$. Finally, using the inequality above, we obtain
    $$p_{k+1} < 2p_k < p_1 + p_2 + \dots + p_{k-1} - p_k + 2p_k < p_1 + p_2 + \dots + p_k.$$
    Therefore, by induction, the statement holds for all $n > 3$.\\
\end{solution}

\begin{exercise}
    Verify the following:
    \begin{enumerate}
        \item There exist infinitely many primes ending in $33$, such as $233$, $433$, $733$, $1033$, .... [\textit{Hint:} Apply Dirichlet's Theorem.]
        \item There exist infinitely many primes which do not belong to any pair of twin primes. [\textit{Hint:} Consider the arithmetic progression $21k + 5$ for $k = 1,2,$...]
        \item There exists a prime ending in as many consecutive 1's as desired. [\textit{Hint:} To obtain a prime ending in $n$ consecutive 1's, consider the arithmetic progression $10^nk + R_n$ for $k=1,2,....$]
    \end{enumerate}
\end{exercise}

\begin{solution}
    \begin{enumerate}
        \item Since $100\cdot 1 + 33\cdot (-3) = 1$, then $\gcd(100, 33) = 1$, and so by Dirichlet's Theorem there exist infinitely many primes of the form $100n + 33$. These primes are precisely the ones ending with $33$.
        \item Since $21\cdot 1 + 5\cdot (-4) = 1$, then $\gcd(21, 5) = 1$, and so by Dirichlet's Theorem there exist infinitely many primes of the form $21n + 5$. Now, let $p$ be a prime of the form $21n + 5$, then $p + 2 = 21n + 7 = 7(3n + 1)$ which is composite, and similarly, $p - 2 = 21n + 3 = 3(7n + 1)$ which is also composite. Thus, $p$ cannot be a twin primes. Therefore, there are infinitely many primes which do not belong to a pair of twin primes.
        \item Let $n$ be a positive integer, since $10^n\cdot 1 + R_n\cdot (-9) = 1$, then $\gcd(10^n, R_n) = 1$, and so by Dirichlet's Theorem there exist infinitely many primes of the form $10^nk + R_n$. In other words, there are infinitely many primes that end with $n$ 1's. Since this holds for all $n$, then it follows that there exists a prime ending in as many consecutive 1's as desired. \\
    \end{enumerate}
\end{solution}

\begin{exercise}
    Prove that for every $n \geq 2$ there exists a prime $p$ with $p < n < 2p$. [\textit{Hint:} If $n = 2k+1$, then by Bertrand's Conjecture there exists a prime $p$ such that $k < p < 2k$.] \\
\end{exercise}

\begin{solution}
    Let $n \geq 2$ be an integer. If $n$ is even, then $n = 2k$ for some non-zero integer $k$. By Bertrand's Conjecture, we get that $k < p < 2k$ for some prime number $p$. From the inequality $k < p$, we get $n < 2p$ by multiplying both sides by 2, and from $p < 2k$, we get $p < n$ by definition of $k$. Thus, we get that $p < n < 2p$. Similarly, if $n$ is odd, then $n = 2k + 1$ for some non-zero integer $k$. By Bertrand's Conjecture, we get that $k < p < 2k$ for some prime number $p$. From the inequality $k < p$, we get $n < 2p$ by multiplying both sides by 2, and from $p < 2k < 2k+ 1$, we get $p < n$ by definition of $k$. Thus, we get that $p < n < 2p$. Therefore, it holds for all integers $n \geq 2$. \\
\end{solution}

\begin{exercise}
    \begin{enumerate}
        \item If $n > 1$, show that $n!$ is never a perfect square.
        \item Find the values of $n \geq 1$ for which
        $$n! + (n+1)! + (n+2)!$$
        is a perfect square. [\textit{Hint:} Note that $n! + (n+1)! + (n+2)! = n!(n+2)^2$.]
    \end{enumerate}
\end{exercise}

\begin{solution}
    \begin{enumerate}
        \item Let $n > 1$ be an integer and consider the integer $n! > 1$. Let $p$ be the largest prime smaller than $n$, then $2p$ must be greater than $n$ since otherwise, by Bertrand's Conjecture, there would be a prime $q$ satisfying $p < q < 2p \leq n$ contradicting the fact that $p$ is the greatest. It follows that $p$ is the only integer divisible by $p$ that is less than $n$. It follows that $n! = p^1 \cdot p_1^{r_1}\cdot \dots \cdot p_k^{r_k}$ where $p_i \neq p$ for all $i$. This shows that $n!$ is not a square because an integer is a square if and only if every exponent in its canonical form is even (which is not the case for the exponent of $p$).
        \item When $n = 1$, we have that
        $$n! + (n+1)! + (n+2)! = 1 + 2 + 6 = 3^2.$$
        Suppose now that $n > 1$, then
        $$n! + (n+1)! + (n+2)! = n!(n+2)^2.$$
        Since $n!$ is not a square (part a), then it must contain a prime $p$ with an odd exponent in its canonical form. It follows that the exponent of $p$ in the canonical form of $n!(n+2)^2$ is also odd. Therefore, $n! + (n+1)! + (n+2)!$ cannot be a square in that case, and hence, it is only a square when $n = 1$.
    \end{enumerate}
\end{solution}

% CHAPTER 4
\chapter{The Theory of Congruences}

\section{Karl Friedrich Gauss}

There are no exercises in this section.

\section{Basic Properties of Congruence}

\begin{exercise}
    Prove each of the following assertions:
    \begin{enumerate}
        \item If $a \equiv b \pmod n$ and $m \mid n$, then $a \equiv b \pmod m$.
        \item If $a \equiv b \pmod n$ and $c > 0$, then $ca \equiv cb \pmod{cn}$.
        \item If $a \equiv b \pmod n$ and the integers $a$, $b$, $n$ are all divisible by $d > 0$, then $a/d \equiv b/d \pmod{n/d}$.
    \end{enumerate}
\end{exercise}

\begin{solution}
    \begin{enumerate}
        \item If $a \equiv b \pmod n$, then $n \mid a - b$. Moreover, $m \mid n$ so $m \mid a - b$, and hence, by definition, $a \equiv b \pmod m$.
        \item If $a \equiv b \pmod n$, then $n \mid a - b$, and hence, $cn \mid c(a-b) = ca - cb$. It follows that $ca \equiv cb \pmod{cn}$.
        \item If $a \equiv b \pmod n$, then $a - b = nk$ for some integer $k$. This equation can be rewritten as $d(a/d - b/d) = d(kn/d)$. Since $d \neq 0$, then we get that $a/d - b/d = kn/d$ and so that by definition, we obtain $a/d \equiv b/d \pmod{n/d}$.
    \end{enumerate}
\end{solution}

\begin{exercise}
    Give an example to show that $a^2 \equiv b^2 \pmod n$ need not imply that $a \equiv b \pmod n$. \\
\end{exercise}

\begin{solution}
    We have that $1 \not\equiv -1 \pmod 3$ but $1^2 \equiv (-1)^2 \pmod 3$.\\
\end{solution}

\begin{exercise}
    If $a \equiv b \pmod n$, prove that $\gcd(a,n) = \gcd(b,n)$.\\
\end{exercise}

\begin{solution}
    First, recall that if $a \equiv b \pmod n$, then there is an integer $k$ such that $a = b + kn$. It follows that any integer linear combination $ax + ny$ of $a$ and $n$ can be written as a linear combination $bx + n(kx + y)$ of $b$ and $n$. Similarly, any linear combination of $b$ and $n$ can also be written as a linear combination of $a$ and $n$. It follows that the smallest positive integer that can be written as a linear combination of $a$ and $n$ must be equal to the smallest positive integer that can be written as a linear combination of $b$ and $n$. In other words, $\gcd(a,n) = \gcd(b,n)$. \\
\end{solution}
    
\begin{exercise}
    \begin{enumerate}
        \item Find the remainders when $2^{50}$ and $41^{65}$ are divided by $7$.
        \item What is the remainder when the sum
        $$1^5 + 2^5 + 3^5 + \dots + 99^5 + 100^5$$
        is divided by $4$ ?
    \end{enumerate}
\end{exercise}

\begin{solution}
    \begin{enumerate}
        \item Since $8 \equiv 1 \pmod 7$, then
        $$2^{50} \equiv 2^{48}\cdot 4 \equiv 8^{16}\cdot 4 \equiv 1^{16}\cdot 4 \equiv 4 \pmod 7.$$
        Hence, $2^{50}$ has a remainder of $4$ after a division by $7$. Similarly, since $41 \equiv -1 \pmod 7$, then 
        $$41^{65} \equiv (-1)^{65} \equiv -1 \equiv 6 \pmod 7.$$
        Therefore, $41^{65}$ has a remainder of $6$ after a division by 7.
        \item First, notice that every even number taken to the power of $5$ must be divisible by $4$. It follows that every even number in the sum is congruent to $0$ modulo 4, and hence, can be discarded. The remaining integers $1$, $3$, $5$, ... are alternatively congruent to $1$ and $-1$ modulo 4, and hence, when taken to the power of 5, are still respectively congruent to $1$ and $-1$. It follows that
        $$1^5 + 2^5 + \dots + 99^5 + 100^5 \equiv 1^5 + 3^5 + \dots + 99^5 \equiv 1 - 1 + \dots - 1 \equiv 0 \pmod 4.$$
        Therefore, the sum has a remainder of 0 after a division by 4.
    \end{enumerate}
\end{solution}

\begin{exercise}
    Prove that $53^{103} + 103^{53}$ is divisible by $39$, and also that $111^{333} + 333^{111}$ is divisible by $7$. \\
\end{exercise}

\begin{solution}
    First, notice that $53 \equiv 14 \pmod{39}$ and $103 \equiv -14 \pmod{39}$. Hence, we get that $53^{103} + 103^{53} \equiv 14^{103} + (-14)^{53} \equiv 14^{103} - 14^{53} \pmod{39}$. Now, notice that $14^2 \equiv 196 \equiv 1 \pmod{39}$, and so it follows that 
    $$53^{103} + 103^{53} \equiv 14^{103} - 14^{53} \equiv 14\cdot 14^{2\cdot 51} - 14\cdot 14^{2\cdot 26} \equiv 14 - 14 \equiv 0 \pmod{39}.$$
    Therefore, $53^{103} + 103^{53}$ is divisible by $39$.

    Similarly, notice first that $111 \equiv -1 \pmod 7$ and $333 \equiv 4 \pmod 7$. Hence, $111^{333} + 333^{111} \equiv 1 + 4^{111}$. Now, since $4^3 \equiv -1 \pmod 7$, then
    $$4^{111} \equiv 4^{3\cdot 37} \equiv (-1)^{37} \equiv -1 \pmod 7$$
    and so it follows that $111^{333} + 333^{111} \equiv 1 + 4^{111} \equiv 1 - 1 \equiv 0 \pmod 7$. Therefore, $111^{333} + 333^{111}$ is divisible by 7. \\
\end{solution}

\begin{exercise}
    For $n \geq 1$, use congruence theory to establish each of the following divisibility statements:
    \begin{enumerate}
        \item $7 \mid 5^{2n} + 3\cdot 2^{5n - 2}$;
        \item $13 \mid 3^{n+2} + 4^{2n+1}$;
        \item $27 \mid 2^{5n+1} + 5^{n+2}$;
        \item $43 \mid 6^{n+2} + 7^{2n+1}$;
    \end{enumerate}
\end{exercise}

\begin{solution}
    \begin{enumerate}
        \item First, notice that $5^{2n} \equiv 25^n \equiv 4^n \pmod 7$ and
        $$2^{5n-2} \equiv 2^{5(n-1) + 3} \equiv 8\cdot (2^5)^{n-1} \equiv 1\cdot 4^{n-1}.$$
        Hence, $5^{2n} + 3\cdot 2^{5n - 2} \equiv 4^n + (-4) \cdot 4^{n-1} \equiv 0 \pmod 7$.
        \item Simply notice that
        $$3^{n+2} + 4^{2n+1} \equiv 9\cdot 3^n + 4 \cdot 3^n \equiv 13\cdot 3^n \equiv 0 \pmod{13}.$$
        \item Simply notice that
        $$2^{5n+1} + 5^{n+2} \equiv 2\cdot 32^n + 25 \cdot 5^n \equiv 2\cdot 5^n + 25 \cdot 5^n \equiv 27\cdot 5^n \equiv 0 \pmod{27}.$$
        \item Simply notice that
        $$6^{n+2} + 7^{2n+1} \equiv 36\cdot 6^n + 7 \cdot 49^n \equiv 36\cdot 6^n + 7 \cdot 6^n \equiv 43\cdot 6^n \equiv 0 \pmod{43}.$$
    \end{enumerate}
\end{solution}

\begin{exercise}
    For $n\geq 1$, show that
    $$(-13)^{n+1} \equiv (-13)^n + (-13)^{n-1} \pmod{181}.$$
    [\textit{Hint:} Notice that $(-13)^2 \equiv -13 + 1 \pmod{181}$; use induction on $n$.]\\
\end{exercise}

\begin{solution}
    Let's prove it by induction. When $n = 1$, we have
    $$(-13)^2 \equiv 169 \equiv - 12 \equiv -13 + 1 \equiv (-13)^1 + (-13)^0 \pmod{181}.$$
    Hence, it holds for $n = 1$. Suppose now that
    $$(-13)^{k+1} \equiv (-13)^k + (-13)^{k-1} \pmod{181}$$
    for some integer $k \geq 1$, then multiplying both sides by $(-13)$ gives us
    $$(-13)^{(k+1)+1} \equiv (-13)^{k+1} + (-13)^{(k+1)-1} \pmod{181}$$ 
    which shows that the statement holds for $n = k+1$. Therefore, the proposition holds for all integers $n \geq 1$. \\
\end{solution}

\begin{exercise}
    Prove the assertions below:
    \begin{enumerate}
        \item If $a$ is an odd integer, then $a^2 \equiv 1 \pmod{8}$.
        \item For any integer $a$, $a^3 \equiv 0$, $1$, or $6 \pmod 7$.
        \item For any integer $a$, $a^4 \equiv 0$, or $1 \pmod 5$.
        \item If the integer $a$ is not divisible by $2$ or $3$, then $a^2 \equiv 1 \pmod{24}$.
    \end{enumerate}
\end{exercise}

\begin{solution}
    \begin{enumerate}
        \item If $a$ is odd, then either $a = 4k+1$ or $a = 4k+3$. In the first case:
        $$a^2 \equiv 16k^2 + 8k + 1 \equiv 1 \pmod 8.$$
        In the second case:
        $$a^2 \equiv 16k^2 + 8\cdot 3k + 9 \equiv 1 \pmod 8.$$
        Therefore, $a^2 \equiv 1 \pmod 8$ for all odd integers $a$.
        \item Let's work in modulo 7. If $a \equiv 0$, then $a^3 \equiv 0$. If $a \equiv 1$, then $a^3 \equiv 1$. If $a \equiv 2$, then $a^3 \equiv 8 \equiv 1$. If $a \equiv 3$, then $a^3 \equiv 27 \equiv 6$. If $a \equiv 4$, then $a^3 \equiv (-3)^3 \equiv -27 \equiv 1$. If $a \equiv 5$, then $a^3 \equiv (-2)^3 \equiv -8 \equiv 6$. If $a \equiv 6$, then $a^3 \equiv (-1)^3 \equiv -1 \equiv 6$. Therefore, in general, $0$, $1$ and $6$ are the three only possible residues of cubes modulo 7.
        \item Let's work in modulo 5. If $a \equiv 0$, then $a^4 \equiv 0$. If $a \equiv 1$, then $a^4 \equiv 1$. If $a \equiv 2$, then $a^4 \equiv 16 \equiv 1$. If $a \equiv 3$, then $a^4 \equiv (-2)^4 \equiv 1$. If $a \equiv 4$, then $a^4 \equiv (-1)^4 \equiv 1$. Therefore, 0 and 1 are the two only possible residues of a power of 4 modulo 5.
        \item First, notice that an integer $a$ is divisible by 2 or by 3 if and only if its residue modulo 24 is divisible by 2 or by 3. This comes from the fact that $24$ is divisible by 2 and 3. Therefore, if an integer is not divisible by 2 and 3, then its possible residues modulo 24 are 0, 1, 5, 7, 11, 13, 17, 19 and 23. Let's now work in modulo 24. If $a \equiv 0$, then $a^2 \equiv 0$. If $a \equiv 1$, then $a^2 \equiv 1$. If $a \equiv 5$, then $a^2 \equiv 25 \equiv 1$. If $a \equiv 7$, then $a^2 \equiv 49 \equiv 1$. If $a \equiv 11$, then $a^2 \equiv 121 \equiv 1$. If $a \equiv 13$, then $a^2 \equiv (-11)^2 \equiv 1$. If $a \equiv 17$, then $a^2 \equiv (-7)^2 \equiv 1$. If $a \equiv 19$, then $a^2 \equiv (-5)^2 \equiv 1$. If $a \equiv 23$, then $a^2 \equiv (-1)^2 \equiv 1$. Therefore, if $a$ is not divisible by 2 and 3, then $a^2$ must be 0 or 1 modulo 7.
    \end{enumerate}
\end{solution}

\begin{exercise}
    If $p$ is a prime satisfying $n < p < 2n$, show that
    $$\binom{2n}{n} \equiv 0 \pmod p.$$
\end{exercise}

\begin{solution}
    First, write
    $$\binom{2n}{n} = \frac{2n(2n-1) \cdot ... \cdot (n+1)}{n!},$$
    then this can be rewritten as
    $$2n(2n-1) \cdot ... \cdot (n+1) = \binom{2n}{n} \cdot n!.$$
    Since $n < p < 2n$, then $p$ must be one of the factors on the left hand side. It follows that $p$ divides the right hand side. However, since $p > n$, then $p$ cannot divide $n!$, and hence, $p$ must divide $\binom{2n}{n}$. Therefore, $\binom{2n}{n} \equiv 0 \pmod p$. \\
\end{solution}

\begin{exercise}
    If $a_1$, $a_2$, . . . , $a_n$ is a complete set of residues modulo $n$ and $\gcd(a,n) = 1$, prove that $aa_1$, $aa_2$, . . . , $aa_n$ is also a complete set of residues modulo $n$. [\textit{Hint:} It suffices to show that the numbers in question are incongruent modulo $n$.]\\
\end{exercise}

\begin{solution}
    It suffices to show that $aa_i$ is incongruent to $aa_j$ modulo $n$ when $i \neq j$. Let $i \neq j$ be two integers between 1 and $n$, suppose by contradiction that $aa_i \equiv aa_j \pmod n$, then by the fact that $\gcd(a,n) = 1$, we get that $a_i \equiv a_j \pmod n$. But since $a_1$, ..., $a_n$ forms a complete set of residues modulo $n$, then $a_i \equiv a_j \pmod n$ implies that $i = j$, a contradiction. Therefore, $aa_i \not\equiv aa_j \pmod n$ whenever $i \neq j$. It follows that $aa_1$, ..., $aa_n$ forms a complete set of residues modulo $n$. \\
\end{solution}

\begin{exercise}
    Verify that $0$, $1$, $2$, $2^2$, $2^3$, . . . , $2^9$ form a complete set of residues modulo 11, but $0$, $1^2$, $2^2$, $3^2$ . . . , $10^2$ do not. \\
\end{exercise}

\begin{solution}
    First, we have
    \begin{align*}
        0 &\equiv 0 \pmod{11} \\
        1 &\equiv 1 \pmod{11} \\
        2 &\equiv 2 \pmod{11} \\
        2^2 &\equiv 4 \pmod{11} \\
        2^3 &\equiv 8 \pmod{11} \\
        2^4 &\equiv 16 \equiv 5 \pmod{11} \\
        2^5 &\equiv 2\cdot 5 \equiv 10 \pmod{11} \\
        2^6 &\equiv 2\cdot 10 \equiv 20 \equiv 9 \pmod{11}  \\
        2^7 &\equiv 2\cdot 9 \equiv 18 \equiv 7 \pmod{11} \\
        2^8 &\equiv 2\cdot 7 \equiv 14 \equiv 3 \pmod{11} \\
        2^9 &\equiv 2\cdot 3 \equiv 6 \pmod{11} 
    \end{align*}
    Since they are all incongruent to each other, then we have that $0$, $1$, $2$, ..., $2^9$ forms a complete set of residues modulo 11. However, the first 11 squares do not because $1^1 \equiv 1 \equiv 10^2 \pmod{11}$.\\
\end{solution}

\begin{exercise}
    Prove the following statements:
    \begin{enumerate}
        \item If $gcd(a,n) = 1$, then the integers
        $$c, \ c+a, \ c+2a, \ c+3a, \ . \ . \ . \ , \ c+(n-1)a$$
        form a complete set of residues modulo $n$ for any $c$.
        \item Any $n$ consecutive integers form a complete set of residues modulo $n$. [\textit{Hint:} Use part (a).]
        \item The product of any set of $n$ consecutive integers is divisible by $n$.
    \end{enumerate}
\end{exercise}

\begin{solution}
    \begin{enumerate}
        \item First, recall that $0$, $1$, $2$, ..., $n-1$ forms a complete set of residues modulo $n$. This implies that for integers $i$ and $j$ between 0 and $n-1$, $i \equiv j \pmod n$ only if $i = j$. Now, let $i$ and $j$ be two integers between 0 and $n-1$ and suppose that $c + ai \equiv c+aj \pmod n$, then by subtracting by $c$ on both sides, we get that $ai \equiv aj \pmod n$. Now, since $\gcd(a,n) = 1$, then we can "divide" both sides by $a$ to obtain $i \equiv j \pmod n$, which in turns implies that $i = j$. Therefore, the integers $c$, $c+a$, ..., $c+(n-1)a$ are all incongruent and so they form a complete set of residues modulo $n$.
        \item Notice that any $n$ consecutive integers can be written as $c$, $c + 1$, ..., $c+(n-1)$ where $c$ is the first of the $n$ integers. By taking $a = 1$ in part (a), we immediately get that these $n$ consecutive integers must form a complete set of residues modulo $n$.
        \item Let $a_1$, $a_2$, ..., $a_n$ be a list of $n$ consecutive integers, then by part (b), it is a complete set of residues modulo $n$. It follows that there is a $i$ between 1 and $n$ such that $a_i \equiv 0 \pmod n$. Thus, 
        $$a_1 a_2 \cdot \cdot \cdot a_i \cdot \cdot \cdot a_n \equiv a_1 a_2 \cdot \cdot \cdot 0 \cdot \cdot \cdot a_n \equiv 0 \pmod n$$
        which implies that $n \mid a_1 a_2 \cdot \cdot \cdot a_n$.
    \end{enumerate}
\end{solution}

\begin{exercise}
    Verify that if $a\equiv b \pmod{n_1}$ and $a \equiv b \pmod{n_2}$, then $a \equiv b \pmod{n}$, where the integer $n = \lcm(n_1, n_2)$. Hence, whenever $n_1$ and $n_2$ are relatively prime, $a \equiv b \pmod{n_1n_2}$. \\
\end{exercise}

\begin{solution}
    If $a\equiv b \pmod{n_1}$ and $a \equiv b \pmod{n_2}$, then by definition, $a - b$ is a common multiple of $n_1$ and $n_2$. Thus, by properties of the least common multiple, we must have that $\lcm(n_1, n_2) \mid a - b$. It follows that $a \equiv b \pmod{n}$ where $n = \lcm(n_1, n_2)$. When $n_1$ and $n_2$ are relatively prime, we have $\lcm(n_1, n_2) = n_1n_2$. \\
\end{solution}

\begin{exercise}
    Give an example to show that $a^k \equiv b^k \pmod n$ and $k \equiv j \pmod n$ need not imply $a^j \equiv b^j \pmod n$. \\
\end{exercise}

\begin{solution}
    We know that $1^2 \equiv (-1)^2 \pmod 3$ but $1^5 \not\equiv (-1)^5 \pmod 3$ even though $2 \equiv 5 \pmod 3$. \\
\end{solution}

\begin{exercise}
    Establish that if $a$ is an odd integer, then
    $$a^{2^n} \equiv 1 \pmod{2^{n+2}}$$
    for any $n \geq 1$. [\textit{Hint:} Proceed by induction on $n$.] \\
\end{exercise}

\begin{solution}
    Let's prove it by induction on $n$. The case $n = 1$ is exactly part (a) of Problem 8 of this section. Suppose now that $a^{2^k} \equiv 1 \pmod{2^{k+2}}$ for some integer $k \geq 1$, then $2^{k+2} \mid a^{2^k} - 1$. Now, notice that $a^{2^{k+1}} - 1 = (a^{2^k} - 1)(a^{2^k} + 1)$ is both divisible by $2^{k+2}$ (from the first factor) and divisible by $2$ (from the second factor since $a^{2^k} + 1$ is even). It follows that $2^{k+3} \mid a^{2^{k+1}} - 1$ and so $a^{2^{k+1}} \equiv 1 \pmod{2^{(k+1)+2}}$. Therefore, it holds for all $n \geq 1$ by induction. \\
\end{solution}

\begin{exercise}
    Use the theory of congruences to verify that
    $$89 \mid 2^{44} - 1 \qquad \text{ and } \qquad 97 \mid 2^{48} - 1.$$
\end{exercise}

\begin{solution}
    First, notice that $2^{44} = (2^{11})^4 = 2048^4$. Since $2048 \equiv 1 \pmod{89}$, then $2^{44} \equiv 1^4 \equiv 1 \pmod{89}$. It follows that $89 \mid 2^{44} - 1$.

    Similarly, notice that $2^{48} = (2^{12})^4 = 4096^4$. Now, we can check that $4096 \equiv 22 \pmod{97}$ and so $2^{48} \equiv 22^4 \pmod{97}$. Moreover, $22^2 = 484$ which can be checked to be congruent to $-1$ modulo $97$. It follows that $2^{48} \equiv 22^4 \equiv (-1)^2 \equiv 1 \pmod{97}$. Therefore, $97 \mid 2^{48} - 1$. \\
\end{solution}

\begin{exercise}
    Prove that if $ab \equiv cd \pmod n$ and $b \equiv d \pmod n$, with $\gcd(b,n) = 1$, then $a \equiv c \pmod n$. \\
\end{exercise}

\begin{solution}
    From the equation $b \equiv d \pmod n$, we get $cb \equiv cd \pmod n$ by multiplying both sides by $c$. From $ab \equiv cd \pmod n$ and $cb \equiv cd \pmod n$, we get $ab \equiv cb \pmod n$. Finally, using the fact that $\gcd(b,n) = 1$ lets us conclude that $a \equiv c \pmod n$. \\
\end{solution}

\begin{exercise}
    If $a \equiv b \pmod{n_1}$ and $a \equiv c \pmod{n_2}$, prove that $b \equiv c \pmod{n}$, where the integer $n = \gcd(n_1, n_2)$. \\
\end{exercise}

\begin{solution}
    If we let $n = \gcd(n_1, n_2)$, then from the fact that $n \mid n_1, n_2$ and Problem 1.(a) of this section, we get that $a \equiv b \pmod{n}$ and $a \equiv c \pmod{n}$. By transitivity, it follows that $b \equiv c \pmod n$.
\end{solution}
\section{Special Divisibility Tests}

\begin{exercise}
    Prove the following statements:
    \begin{enumerate}
        \item For any integer $a$, the units digit of $a^2$ is $0$, $1$, $4$, $5$, $6$, or $9$.
        \item Any one of the integers 0, 1, 2, 3, 4, 5, 6, 7, 8, 9 can occur as the units digit of $a^3$.
        \item For any integer $a$, the units digit of $a^4$ is $0$, $1$, $5$, or $6$.
        \item The units digit of a triangular number is 0, 1, 3, 5, 6 or 8.
    \end{enumerate}
\end{exercise}

\begin{solution}
    \begin{enumerate}
        \item It suffices to prove that the only possible residues modulo 10 of a square are $0$, $1$, $4$, $5$, $6$, or $9$. Let's prove it by cases. If $a \equiv 0$, then $a^2 \equiv 0$. If $a \equiv 1$, then $a^2 \equiv 1$. If $a \equiv 2$, then $a^2 \equiv 4$. If $a \equiv 3$, then $a^2 \equiv 9$. If $a \equiv 4$, then $a^2 \equiv 6$. If $a \equiv 5$, then $a^2 \equiv 5$. If $a$ is congruent to 6, 7, 8 or 9, then $-a$ will be congruent to one of the previous cases and so the square are goind to be the same since $(-a)^2 = a^2$. Therefore, from this cases, we get that the possible residues are 0, 1, 4, 5, 6, 9. 
        \item It suffices to prove that for all $0 \leq i \leq 9$, there is an integer $a$ such that $a^3 \equiv i \pmod{10}$. For $i = 0,1$, simply take $a = i$. For $i = 2$, we have $8^3 \equiv 512 \equiv i$. For $i = 3$, we have $7^3 \equiv 343 \equiv i$. For $i = 4$, we have $4^3 \equiv 64 \equiv i$. For $i = 5$, we have $5^3 \equiv 125 \equiv i$. For $i = 6$, we have $6^3 \equiv 216 \equiv i$. For $i = 7$, we have $3^3 \equiv 27 \equiv i$. For $i = 8$, we have $2^3 \equiv i$. For $i = 9$, we have $9^3 \equiv (-1)^3 \equiv -1 \equiv i$.
        \item Using part (a), it suffices to look at the residues modulo 10 of the squares of 0, 1, 4, 5, 6, 9. 
        $$0^2 \equiv 0, \qquad 1^2 \equiv 1, \qquad 4^2 \equiv 6, \qquad 5^2 \equiv 5, \qquad 6^2 \equiv 6, \qquad 9^2 \equiv 1.$$
        Therefore, the only possible units of an integer taken to the fourth power are 0, 1, 5 or 6.
        \item Here it is temmpting to simply consider the ten cases of residues modulo 10. However, we never proved that $i \equiv j$ implies $t_i \equiv t_j$. This is nontrivial since this not true in modulo 10. To do this, let's first find the residues of $n(n+1)$ and then deduce the possible residues of triangle numbers. Using the same technique as in part (a) or part (b), we get that the possible residues of $n(n+1)$ are 0, 2 or 6. Now, to find the possible residues of $t_n$, it suffices to answer the question, what are the numbers such when multiplied by 2, have residue 0, 2, or 6. Again, by cases, we get that these numbers are 0, 1, 3, 5, 6 or 8. Since for each of these numbers it is possible to find a triangular number congruent to it, then this list represents precisely the possible units of a triangular number.
    \end{enumerate}
\end{solution}

\begin{exercise}
    Find the last two digits of the number $9^{9^9}$. [\textit{Hint:} $9^9 \equiv 9 \pmod{10}$, hence $9^{9^9} = 9^{9 + 10k}$; now use the fact that $9^{10} \equiv 1 \pmod{100}$.]\\
\end{exercise}

\begin{solution}
    First, notice that it suffices to find the residue of $9^{9^9}$ modulo 100. To do this, let's first find the residue of $9^9$ modulo 100:
    \begin{align*}
        9^9 &\equiv 9^8\cdot 9 \\
        &\equiv (9^2)^4\cdot 9 \\
        &\equiv ((-19)^2)^2 \cdot 9 \\
        &\equiv (-39)^2\cdot 9 \\
        &\equiv 21 \cdot 9 \\ 
        &\equiv 89.
    \end{align*} 
    Therefore, we have that $9^9 = 89 + 100n$. This implies that $9^9 = 9 + 10k$ and so we have
    $$9^{9^9} = 9^{9 + 10k} = 9^9 \cdot (9^{10})^k.$$
    Since $9^9 \equiv 89 \equiv -11 \pmod{100}$, then $9^{10} \equiv -99 \equiv 1 \pmod{100}$. It follows that 
    $$9^{9^9} \equiv 89 \cdot 1^k \equiv 89 \pmod{100}.$$
    Therefore, the last two digits are 89.\\
\end{solution}

\begin{exercise}
    Without performing the divisions, determine whether the integers $176,521,221$ and $149,235,678$ are divisible by 9 or 11. \\
\end{exercise}

\begin{solution}
    We simply need to take the (alternating) sum of the digits and examine the resulting numbers. Let's verify if the first number is divisible by 9:
    $$1 + 7 + 6 + 5 + 2 + 1 + 2 + 2 + 1 = 27$$
    which is divisible by 9. Thus, the first number is divisible by 9. Let's do the same with the second number:
    $$1 + 4 + 9 + 2 + 3 + 5 + 6 + 7 + 8 = 45$$
    which is again divisible by 9. Thus, even the second number is divisible by 9. Now, let's check the divisibility by 11.
    $$1 -2 + 2 - 1 + 2 - 5 + 6 - 7 + 1 = -3$$
    and
    $$8 - 7 + 6 - 5 + 3 - 2 + 9 - 4 + 1 = 9.$$
    Since none of the results are divisible by 11, then none of the numbers are either. \\
\end{solution}

\begin{exercise}
    \begin{enumerate}
        \item Obtain the following generalization of Theorem 4-5: If the integer $N$ is represented in the base $b$ by
        $$N = a_mb^m + \dots + a_2b^2 + a_1b + a_0, \quad 0 \leq a_k \leq b-1$$
        then $b-1 \mid N$ if and only if $b-1 \mid (a_m + \dots + a_2 + a_1 + a_0)$.
        \item Give criteria for the divisibility the divisibility of $N$ by 3 and 8 which depend on the digits of $N$ when written in base 9.
        \item Is the integer $(447836)_9$ divisible by 3 and 8?
    \end{enumerate}
\end{exercise}

\begin{solution}
    \begin{enumerate}
        \item Consider the polynomial $P(x) = \sum_{k=0}^{m}a_kx^k$. Since $b \equiv 1 \pmod{b-1}$, then
        $$\sum_{k=0}^{m}a_k \equiv \sum_{k=0}^{m}a_kb^k \equiv N \pmod{b-1}.$$
        Therefore,
        \begin{align*}
            b-1\mid N &\iff N \equiv 0 \pmod{b-1} \\
            &\iff \sum_{k=0}^{m}a_k \equiv 0 \pmod{b-1} \\
            &\iff b-1 \mid a_m + \dots + a_2 + a_1 + a_0
        \end{align*}
        which is exactly what we wanted to prove.
        \item Using part (a), we get that $N$ is divisible by 8 if and only if the sum of the digits of its base-9 representation is divisible by 8. To find a criteria for the divisibility modulo 3, write $N = \sum_{k=0}^{m}a_k 9_k$, then $N \equiv a_0 \pmod{3}$ which shows that $N$ is divisible by 3 if and only if its first digit is divisible by 3 in its base-9 representation.
        \item Since 6 is divisible by 3, then it follows from part (b) that $(447836)_9$ is divisible by 3. Next,
        $$4 + 4 + 7 + 8 + 3 + 6 = 32$$
        is divisible by 8 and so by part (b), $(447836)_9$ is divisible by 8.
    \end{enumerate}
\end{solution}

\begin{exercise}
    Working modulo 9 or 11, find the missing digits in the calculations below:
    \begin{enumerate}
        \item $51840 \cdot 273581 = 1418243x040$;
        \item $2x99561 = [3(523 + x)]^2$;
        \item $2784x = x \cdot 5569$;
        \item $512 \cdot 1x53125 = 1000000000$.
    \end{enumerate}
\end{exercise}

\begin{solution}
    \begin{enumerate}
        \item If we take this equation modulo 11, we get
        $$(0 - 4 + 8 - 1 + 5)(1 - 8 + 5 - 3 + 7 - 2) \equiv 0 - 4 + 0 - x + 3 - 4 + 2 - 8 + 1 - 4 + 1$$
        which is the same as $8 \cdot 0 \equiv -13 - x \pmod{11}$. Hence, $x \equiv -13 \equiv 9 \pmod{11}$. But since $x$ is a digit in base 10, then $0 \leq x \leq 9$ and so $x \equiv 9 \pmod{11}$ implies $x = 9$.
        \item If we take this equation modulo 9, we get
        $$2 + x + 9 + 9 + 5 + 6 + 1 \equiv 0 \pmod 9$$
        which is equivalent to $x \equiv - 5 \equiv 6 \pmod 9$. Since $0 \leq x \leq 9$, then $x = 6$.
        \item If we take the equation modulo 11, we get
        $$x - 4 + 8 - 7 + 2 \equiv x(9 - 6 + 5 - 5) \pmod{11}$$
        which is equivalent to $x - 1 \equiv 3x \pmod{11}$. Rearranging the equation gives $2x \equiv 2 \cdot 5 \pmod{11}$. Since $\gcd(2, 11) = 1$, then $x = 5 \pmod{11}$ and so $x = 5$.
        \item If we take the equation modulo 11, we get
        $$(2 - 1 + 5)(5 - 2 + 1 - 3 + 5 - x + 1) \equiv -1 \pmod{11}$$
        which is equivalent to $6(7 - x) \equiv -1 \pmod{11}$. Simplifying the equation gives us $43 \equiv 6x \pmod{11}$ and so $6x \equiv 1 \pmod{11}$. Multiplying by 9 on both sides gives us $x \equiv 54x \equiv 9 \pmod{11}$. Therefore, $x = 9$.
    \end{enumerate}
\end{solution}

\begin{exercise}
    Establish the following divisibility criteria:
    \begin{enumerate}
        \item An integer is divisible by 2 if and only if the sum of its digit is 0, 2, 4, 6, or 8.
        \item An integer is divisible by 3 if and only if the sum of its digits is divisible by 3.
        \item An integer is divisible by 4 if and only if the number formed by by its tens and units digits is divisible by 4. [\textit{Hint:} $10^k \equiv 0 \pmod 4$ for $k \geq 2$.]
        \item An integer is divisible by 5 if and only if its unit digit is 0 or 5.
    \end{enumerate}
\end{exercise}

\begin{solution}
    \begin{enumerate}
        \item Let $N$ be an arbitrary integer and write it as
        $$N = a_m \cdot 10^m + \dots + a_1 \cdot 10 + a_0$$
        where the $a_i$'s satisfy $0 \leq a_i \leq 9$. Notice that if we take the above equation modulo 2, we get that all the terms become zero except maybe $a_0$: $N \equiv a_0 \pmod{2}$. Hence, we get that the divisibility by 2 of $N$ is equivalent to the divisibility by 2 of $a_0$. Since $0 \leq a_0 \leq 9$, then $a_0$ is only divisible by 2 when $a_0 = 0,2,4,6,8$. Therefore, $N$ is divisible by 2 if and only if its unit digit is 0, 2, 4, 6 or 8.
        \item Let $N$ be an arbitrary integer and write it as
        $$N = a_m \cdot 10^m + \dots + a_1 \cdot 10 + a_0$$
        where the $a_i$'s satisfy $0 \leq a_i \leq 9$. Notice that if we take the above equation modulo 3, we get that
        $$N \equiv a_m + \dots + a_1 + a_0 \pmod 3$$
        since $10 \equiv 1 \pmod 3$. Hence, $N$ is divisible by 3 if and only the sum of its digit is divisible by 3.
        \item Let $N$ be an arbitrary integer and write it as
        $$N = a_m \cdot 10^m + \dots + a_1 \cdot 10 + a_0$$
        where the $a_i$'s satisfy $0 \leq a_i \leq 9$. Notice that if we take the above equation modulo 4, we get that $N \equiv 10a_1 + a_0 \pmod 4$. Since $10a_1 + a_0 = (a_1a_0)_{10}$, then it follows that $N$ is divisible by 4 if and only if $N \equiv 0 \pmod 4$, if and only if $(a_1a_0)_{10} \equiv 0 \pmod 4$, if and only if $(a_1a_0)_{10}$ is divisible by 4.
        \item Let $N$ be an arbitrary integer and write it as
        $$N = a_m \cdot 10^m + \dots + a_1 \cdot 10 + a_0$$
        where the $a_i$'s satisfy $0 \leq a_i \leq 9$. Notice that if we take the above equation modulo 5, we get that all the terms become zero except maybe $a_0$: $N \equiv a_0 \pmod{5}$. Hence, we get that the divisibility by 5 of $N$ is equivalent to the divisibility by 5 of $a_0$. Since $0 \leq a_0 \leq 9$, then $a_0$ is only divisible by 5 when $a_0 = 0,5$. Therefore, $N$ is divisible by 5 if and only if its unit digit is 0 or 5.
    \end{enumerate}
\end{solution}

\begin{exercise}
    For any integer $a$, show that $a^2 -a + 7$ ends in one of the digits 3, 7, or 9. \\
\end{exercise}

\begin{solution}
    It suffices to show that 3, 7 and 9 are the only possible residues of $a^2 -a + 7$ modulo 10. Let's work by cases in modulo 10.
    \begin{itemize}
        \item $a\equiv 0 \implies a^2 - a + 7 \equiv 0 - 0 + 7 \equiv 7$;
        \item $a\equiv 1 \implies a^2 - a + 7 \equiv 1 - 1 + 7 \equiv 7$;
        \item $a\equiv 2 \implies a^2 - a + 7 \equiv 4 - 2 + 7 \equiv 9$;
        \item $a\equiv 3 \implies a^2 - a + 7 \equiv 9 - 3 + 7 \equiv 3$;
        \item $a\equiv 4 \implies a^2 - a + 7 \equiv 16 - 4 + 7 \equiv 9$;
        \item $a\equiv 5 \implies a^2 - a + 7 \equiv 25 - 5 + 7 \equiv 7$;
        \item $a\equiv 6 \implies a^2 - a + 7 \equiv 36 - 6 + 7 \equiv 7$;
        \item $a\equiv 7 \implies a^2 - a + 7 \equiv 49 - 7 + 7 \equiv 9$;
        \item $a\equiv 8 \implies a^2 - a + 7 \equiv 64 - 8 + 7 \equiv 3$;
        \item $a\equiv 9 \implies a^2 - a + 7 \equiv 81 - 9 + 7 \equiv 9$;
    \end{itemize}
    The proposition now clearly follows.\\
\end{solution}

\begin{exercise}
    Find the remainder when $4444^{4444}$ is divided by 9. [\textit{Hint:} Observe that $2^3 \equiv -1 \pmod{9}$.] \\
\end{exercise}

\begin{solution}
    First, we have that $4444 \equiv 4 + 4 + 4 + 4 \equiv 2^4 \pmod{9}$ and so 
    $$4444^{4444} \equiv (2^{4444})^4 \pmod{9}.$$
    Now, consider the fact that $4444 = 3\cdot 1481 + 1$ and use it to get 
    \begin{align*}
        4444^{4444} &\equiv [2^{3\cdot 1481 + 1}]^4 \\
        &\equiv [2\cdot (2^3)^{1481}]^4 \\
        &\equiv [2\cdot (-1)^{1481}]^4 \\
        &\equiv (-2)^4 \\
        &\equiv 16 \\
        &\equiv 7
    \end{align*}
    which implies that 7 is the remainder of $4444^{4444}$ after a division by 9. \\
\end{solution}

\begin{exercise}
    Prove that no integer whose digits add up to 15 can be a square or a cube. [\textit{Hint:} For any $a$, $a^3 \equiv 0, 1,$ or $8 \pmod 9$.] \\
\end{exercise}

\begin{solution}
    Let's find the possible residues of the squares and the cubes modulo 9.
    \begin{itemize}
        \item $a \equiv 0 \implies a^2 \equiv 0 \text{ and } a^3 \equiv 0$;
        \item $a \equiv 1 \implies a^2 \equiv 1 \text{ and } a^3 \equiv 1$;
        \item $a \equiv 2 \implies a^2 \equiv 4 \text{ and } a^3 \equiv 8$;
        \item $a \equiv 3 \implies a^2 \equiv 0 \text{ and } a^3 \equiv 0$;
        \item $a \equiv 4 \implies a^2 \equiv 7 \text{ and } a^3 \equiv 1$;
        \item $a \equiv 5 \implies a^2 \equiv 7 \text{ and } a^3 \equiv 8$;
        \item $a \equiv 6 \implies a^2 \equiv 0 \text{ and } a^3 \equiv 0$;
        \item $a \equiv 7 \implies a^2 \equiv 4 \text{ and } a^3 \equiv 1$;
        \item $a \equiv 8 \implies a^2 \equiv 1 \text{ and } a^3 \equiv 8$;
    \end{itemize}
    Let $N$ be an integer such that the sum of its digits is 15, then in other words, $N \equiv 15 \equiv 6 \pmod 9$. Since 6 appears nowhere in the cases above, then $N$ is neither a square or a cube. \\
\end{solution}

\begin{exercise}
    Assuming that $495$ divides $273x49y5$, obtain the digits $x$ and $y$. \\
\end{exercise}

\begin{solution}
    Since $495$ divides $273x49y5$, then there is an integer $k$ such that $495k = 273x49y5$. Taking this equation modulo 9 gives us
    $$2 + 7 + 3 + x + 4 + 9 + y + 5 \equiv (4 + 9 + 5)k \equiv 0 \pmod 9$$
    which can be simplified into $3 + x + y \equiv 0 \pmod 9$. Similarly, if we take the equation modulo 11, we get 
    $$5 - y + 9 - 4 + x - 3 + 7 - 2 \equiv (5 - 9 + 4)k \equiv 0 \pmod{11}$$
    which can be simplified into $1 + x \equiv y \pmod{11}$. Since both $1 + x$ and $y$ are between 0 and $10$, then this equation becomes $1 + x = y$. Making this substitution into the first equation we found gives us $2x \equiv - 4 \pmod 9$. Since $\gcd(2, 9) = 1$, then $x \equiv -2 \equiv 7 \pmod 9$. Since $0 \leq x \leq 9$, then $x = 7$ and so $y = 8$. \\
\end{solution}

\begin{exercise}
    Determine the last three digits of the number $7^{999}$. [\textit{Hint:} $7^{4n} \equiv (1 + 400)^n \equiv 1 + 400n \pmod{1000}$.] \\
\end{exercise}

\begin{solution}
    It suffices to find the residue of $7^{999}$ modulo 1000. Notice that
    $$7^4 \equiv (50-1)^2 \equiv 2500 - 100 + 1 \equiv 1 + 400 \pmod{1000}$$
    and so 
    $$7^{4n} \equiv (1 + 400)^n \equiv 1 + 400n + 0 + 0 + ... + 0 \equiv 1 + 400n \pmod{1000}.$$
    Since $999 = 4\cdot 249 + 3$, then 
    \begin{align*}
        7^{999} &\equiv 7^3 (1 + 400 \cdot 249) \\
        &\equiv 343 \cdot (1 + 400 \cdot 9) \\
        &\equiv 343 \cdot (1 + 600) \\
        &\equiv 343 + 205800 \\
        &\equiv 343 + 800 \\
        &\equiv 143.
    \end{align*}
    Therefore, the last three digits of $7^{999}$ are 143. \\
\end{solution}

\begin{exercise}
    If $t_n$ denotes the $n$th triangular number, show that $t_{n + 2k} \equiv t_n \pmod{k}$; hence, $t_n$ and $t_{n + 20}$ must have the same last digit. \\
\end{exercise}

\begin{solution}
    Let $n$ and $k$ be positive integers, then
    \begin{align*}
        t_{n + 2k} &\equiv \frac{(n + 2k)(n + 2k + 1)}{2} \\
        &\equiv \frac{n(n + 2k + 1)}{2} + k(n + 2k + 1) \\
        &\equiv \frac{n(n + 2k + 1)}{2} \\
        &\equiv \frac{n(n+1)}{2} + kn \\
        &\equiv t_n \pmod k
    \end{align*}
    which proves the statement. \\
\end{solution}

\begin{exercise}
    For any $n > 1$, prove that there exists a prime with at least $n$ of its digits equal to 0. [\textit{Hint:} Consider the arithmetic progression $10^{n+1}k + 1$ for $k = 1,2,\dots$.] \\
\end{exercise}

\begin{solution}
    Consider the arithmetic progression $10^{n+1}k + 1$ where $k = 1,2,\dots$, since $\gcd(10^{n+1}, 1) = 1$, then by Dirichlet's Theorem there are infinitely many primes numbers of the form $10^{n+1}k + 1$, and so there must be at least ine such prime. Notice now that any number of that form has a unit digit of 1 and the following $n$ digits equal to 0. Therefore, there exists a prime with at least $n$ of its digits equal to 0. \\
\end{solution}

\begin{exercise}
    Find the values of $n \leq 1$ for which $1! + 2! + 3! + \dots + n!$ is a perfect square. [\textit{Hint:} Problem 1(a).] \\
\end{exercise}

\begin{solution}
    Let's consider the residues of the sequence $1! + 2! + ... + n!$ modulo $n$. We have that $1! \equiv 1$, $1! + 2! \equiv 3$, $1! + 2! + 3! \equiv 9$, $1! + 2! + 3! + 4! \equiv 3$, and since $n! \equiv 0$ for all $n \geq 5$, then $1! + 2! + \dots + n! \equiv 3$ for all $n \geq 5$. Since the possible residues of a square in modulo 10 are 0, 1, 4, 5, 6 or 9, then we know for sure that $1! + 2! + \dots + n!$ is not a square for $n = 2$ and for all $n \geq 4$. Since $1!$ and $1! + 2! + 3!$ are squares, then we have that the sum is a square precisely when $n = 1$ or $n = 3$.\\
\end{solution}

\begin{exercise}
    Show that $2^n$ divides an integer $N$ if and only if $2^n$ divides the number made up of the last $n$ digits of $N$. [\textit{Hint:} $10^k \equiv 2^k 5^k \equiv 0 \pmod{2^n}$ for $k \geq n$.] \\
\end{exercise}

\begin{solution}
    First, suppose that $N$ is made up of more than $n$ digits, otherwise, the statement is trivial. Write $N$ as $a_m10^m + \dots + 10a_1 + a_0$ such that $N = (a_m\dots a_1a_0)_{10}$, then
    $$N \equiv a_m10^m + \dots + 10a_1 + a_0 \equiv 0 + \dots + 0 + a_{n-1}10^{n-1} + \dots + a_0 \pmod{2^n}$$
    and so $N \equiv (a_{n-1}\dots a_1a_0)_{10} \pmod{2^n}$. Therefore, $2^n$ divides $N$ if and only if $2^n$ divides $(a_{n-1}\dots a_1a_0)_{10}$. \\
\end{solution}

\begin{exercise}
    Let $N = a_m10^m + \dots + a_2 10^2 + a_110 + a_0$, where $0 \leq a_k \leq 9$, be the decimal expansion of a positive integer $N$.
    \begin{enumerate}
        \item Prove that 7, 11, and 13 all divide $N$ if and only if 7, 11, and 13 divide the integer
        $$M = (100a_2 + 10a_1 + a_0) - (100a_5 + 10a_4 + a_3) + (100a_8 + 10a_7 + a_6) - \dots$$
        [\textit{Hint:} If $n$ is even, then $10^{3n} \equiv 1$, $10^{3n+1} \equiv 10$, $10^{3n+2} \equiv 100 \pmod{1001}$; $10^{3n} \equiv -1$, $10^{3n+1} \equiv -10$, $10^{3n+2} \equiv -100 \pmod{1001}$.]
        \item Prove that 6 divides $N$ if and only if 6 divides the integer $M = a_0 + 4a_1 + 4a_2 + \dots + 4a_m$.
    \end{enumerate}
\end{exercise}

\begin{solution}
    \begin{enumerate}
        \item First, notice that
        $$10^{3n} \equiv (1000)^n \equiv (-1)^n \pmod{1001}.$$
        It follows that the residues of the sequence 1, 10, $10^2$, $10^3$, $10^4$, ... are respectively 1, 10, 100, $-1$, $-10$, $-100$, 1, 10, $\dots$. Hence, we get that
        $$N \equiv (100a_2 + 10a_1 + a_0) - (100a_5 + 10a_4 + a_3) + \dots \equiv M \pmod{1001}.$$
        Next, notice that 7, 11, and 13 are all relatively prime so they all divide a number $L$ if and only if their product 1001 divides that same number $L$. From this, we get that 7, 11, and 13 all divide $N$, if and only if 1001 divides $N$, if and only if 1001 divides $M$, if and only if 7, 11, and 13 all divide $M$.
        \item First, let's prove by induction that $10^n \equiv 4 \pmod{6}$ for all $n \geq 1$. For $n = 1$, this is trivial. Suppose now that $10^n \equiv 4 \pmod{6}$ for some $n \geq 1$, then
        $$10^{n+1} \equiv 10^n \cdot 10 \equiv 4 \cdot 4 \equiv 16 \equiv 4 \pmod 6.$$
        Therefore, by induction, the proposition holds for all $n \geq 1$. From this, we get that $N \equiv M \pmod{6}$. Therefore, it follows that $N$ is divisible by 6 if and only if $M$ is divisible by 6.
    \end{enumerate}
\end{solution}

\begin{exercise}
    Without performing the divisions, determine whether the integer $1,010,908,899$ is divisible by 7, 11, and 13. \\
\end{exercise}

\begin{solution}
    Let $M = 899 - 908 + 10 - 1 = 0$, notice that $M$ is trivially divisible by 7, 11, and 13. Therefore, from the previous exercise, $N$ is divisible by 7, 11, and 13. \\
\end{solution}

\begin{exercise}
    \begin{enumerate}
        \item Given an integer $N$, let $M$ be the integer formed by reversing the order of the digits of $N$ (for example, if $N = 6923$, then $M = 3296$). Verify that $N - M$ is divisible by 9.
        \item A \textit{palindrome} is a number that reads the same backwards as forwards (for instance, 373 and 521125 are palindromes). Prove that any palindrome with an even number of digits is divisible by 11.
    \end{enumerate}
\end{exercise}

\begin{solution}
    \begin{enumerate}
        \item Let $N = a_m10^m + \dots + a_1 10 + a_0$ and $M = a_010^m + \dots + a_{m-1} 10 + a_m$, then in modulo 9, we have 
        $$N - M \equiv (a_m + \dots + a_1 + a_0) - (a_0 + \dots + a_{m-1} + a_m) \equiv 0 \pmod 9.$$
        Therefore, 9 divides $N - M$.
        \item Let $N = a_{2n+1}10^{2n+1} + \dots + a_1 10 + a_0$ be a palindrome with an even number of digits, then $a_0 = a_{2n+1}$, $a_1 = a_{2n}$, ..., and so we get 
        $$N \equiv a_0 - a_1 + \dots + a_{2n} - a_{2n+1} \equiv 0 \pmod{11}.$$
        Therefore, $N$ is divisible by 11. \\
    \end{enumerate}
\end{solution}

\begin{exercise}
    Given a repunit $R_n$, show that
    \begin{enumerate}
        \item $9 \mid R_n$ if and only if $9 \mid n$.
        \item $11 \mid R_n$ if and only if $n$ is even.
    \end{enumerate}
\end{exercise}

\begin{solution}
    \begin{enumerate}
        \item Since $R_n \equiv 1 + 1 + ... + 1 \equiv n \pmod 9$, then it directly follows that 9 divides $R_n$ if and only if 9 divides $n$.
        \item Since
        $$R_n \equiv 1 - 1 + 1 - ... \pm 1 \equiv \begin{cases}
            0, & \text{if } n \text{ is even} \\
            1, & \text{if } n \text{ is odd}
        \end{cases} \pmod{11},$$ 
        then 11 divides $R_n$ if and only if $n$ is even. \\
    \end{enumerate}
\end{solution}

\begin{exercise}
    Factor the repunit $R_6 = 111111$ into a product of primes. [\textit{Hint:} Problem 16.] \\
\end{exercise}

\begin{solution}
    Using Problem 16 part (a), we get that 111111 is divisible by 7, 11, and 13 because $111 - 111 = 0$. Hence, we get that $R_6 = 7 \cdot 11 \cdot 13 \cdot 111$. Since $111$ is divisible by 3, then we can write it as $3 \cdot 37$. Since 37 is prime, then we get that $R_6 = 3\cdot 7 \cdot 11 \cdot 13 \cdot 37$. \\
\end{solution}

\begin{exercise}
    Explain why the following curious calculations hold:
    \begin{align*}
        1 \cdot 9 + 2 &= 11 \\
        12 \cdot 9 + 3 &= 111 \\
        123 \cdot 9 + 4 &= 1111 \\
        1234 \cdot 9 + 5 &= 11111 \\
        12345 \cdot 9 + 6 &= 111111 \\
        123456 \cdot 9 + 7 &= 1111111 \\
        1234567 \cdot 9 + 8 &= 11111111 \\
        12345678 \cdot 9 + 9 &= 111111111 \\
        123456789 \cdot 9 + 10 &= 1111111111 \\
    \end{align*}
    [\textit{Hint:} Show that
    \begin{align*}
        (10^{n-1} + 2\cdot 10^{n-2} + 3\cdot 10^{n-3} &+ \dots + n)(10 - 1) \\
        & + (n+1) = (10^{n+1} - 1)/9.]\\
    \end{align*}
\end{exercise}

\begin{solution}
    In this exercice, consider that the notation $(a_m\dots a_1a_0)_{10}$ is just another way of writing 
    $$a_m10^m + ... + a_1 10 + a_0$$
    with no restrictions on the $a_i$'s. Let's prove it by induction on $n$. What we want to prove is that for all $n \geq 1$, we have the equality $(123\dots n)_{10}\cdot 9 + (n+1) = R_{n+1}$. When $n = 1$, we have that $1 \cdot 9 + 2 = R_2$ holds by direct calculation. Now, suppose that the equation $(123\dots n)_{10}\cdot 9 + (n+1) = R_{n+1}$ holds for some $n \geq 1$, then multiplying on both sides by $10$ and then adding 1 gives us 
    \begin{align*}
        R_{n+2} &= 10[(123\dots n)_{10}\cdot 9 + (n+1)] + 1 \\
        &= 10 \cdot (123\dots n)_{10}\cdot 9 + 10(n+1) + 1 \\
        &= (123\dots n 0)_{10}\cdot 9 + 9(n+1) + (n+1) + 1 \\
        &= ((123\dots n 0)_{10} + (n+1))\cdot 9 + (n+2) \\
        &= (123\dots n (n+1))_{10}\cdot 9 + (n+2)
    \end{align*}
    which is exactly the proposition for the case $n + 2$. Therefore, by induction, it holds for all $n \geq 1$. \\
\end{solution}

\begin{exercise}
    An old and somewhat illegible invoice shows that 72 canned hams were purchased for $\$ x67.9y$. Find the missing digits. \\
\end{exercise}

\begin{solution}
    First, notice that the price of 1 canned ham must be an integer $k$ divided by 100 (because a price cannot be more detailed than cents) and so we have the equation in the integers $72k = x679y$. Taking this equation modulo 9 gives us $x + 6 + 7 + 9 + y \equiv 0 \pmod 9$ and so $x + y \equiv 5 \pmod 9$. Moreover, since $8 \mid 72$, then $8 \mid x679y$ and so $8 \mid 79y$ by Problem 15. But notice that the only value of $y$ 0 and 9 that makes $79y$ divisible by 8 is 2 so $y = 2$. Hence, we have $x \equiv 5 - 2 \equiv 3 \pmod{9}$ so $x = 3$. \\ 
\end{solution}

\begin{exercise}
    If $792$ divides the integer $13xy45z$, find the digits $x$, $y$, and $z$. [\textit{Hint:} By Problem 15, $8 \mid 45z$.] \\
\end{exercise}

\begin{solution}
    First, take the equation $792k = 13xy45z$ modulo 9 to get $1 + 3 + x + y + 4 + 5 + z \equiv 0 \pmod 9$, which can be simplified into $x + y + z \equiv 5 \pmod 9$. Now, since $8 \mid 792$, then $8 \mid 13xy45z$ which implies that $8 \mid 45z$ by Problem 15. But notice that the only integer $z$ between 0 and 9 that makes $45z$ divisible by 8 is $z = 6$. Hence, the previous congruence becomes $x + y \equiv 8 \pmod 9$. Now, take the equation $792k = 13xy456$ modulo 11 to get $6 - 5 + 4 - y + x - 3 + 1 \equiv 0 \pmod{11}$, which can be simplified into $y \equiv 3 + x \pmod{11}$. Let's now compare the two congruences
    $$x + y \equiv 8 \pmod 9 \qquad \text{ and } \qquad y \equiv 3 + x \pmod{11}$$
    and proceed by cases. If $x = 0$, then $y = 3$ by the second congruence but it would not satisfy the first one. If $x = 1$, then $y = 4$ by the second congruence but it would not satisfy the first one. If $x = 2$, then $y = 5$ by the second congruence but it would not satisfy the first one. If $x = 3$, then $y = 6$ by the second congruence but it would not satisfy the first one. If $x = 4$, then $y = 7$ by the second congruence but it would not satisfy the first one. If $x = 5$, then $y = 8$ by the second congruence but it would not satisfy the first one. If $x = 6$, then $y = 9$ by the second congruence but it would not satisfy the first one. If $x = 7$, then $y \equiv 10 \pmod{11}$ which is impossible since $0 \leq y \leq 9$. If $x = 8$, then $y = 0$ by the second congruence and the first congruence is also satisfied. If $x = 9$, then $y = 1$ by the second congruence but it would not satisfy the first one. The only case that doesn't lead to a contradiction is when $x = 8$ and $y = 0$. Therefore, the digits are $x = 8$, $y = 0$ and $z = 6$. 
\end{solution}
\section{Linear Congruences}

\begin{exercise}
    Solve the following linear congruences:
    \begin{enumerate}
        \item $25x \equiv 15 \pmod{29}$.
        \item $5x \equiv 2 \pmod{26}$.
        \item $6x \equiv 15 \pmod{21}$.
        \item $36x \equiv 8 \pmod{102}$.
        \item $34x \equiv 60 \pmod{98}$.
        \item $140x \equiv 133 \pmod{301}$. [\textit{Hint:} $\gcd(140, 301) = 7$.]
    \end{enumerate}
\end{exercise}

\begin{solution}
    \begin{enumerate}
        \item Let $d = \gcd(25, 29) = 1$, then by Theorem 4-7, there is a unique solution. Since $7\cdot 25 - 6 \cdot 29 = 1$, then $7 \cdot 25 \equiv 1 \pmod{29}$. Hence, multiplying the original equation by 7 on both sides gives us $x \equiv 7 \cdot 15 \equiv 105 \equiv 18 \pmod{29}$. Therefore, $x = 18$ is the unique solution.
        \item Let $d = \gcd(5, 26) = 1$, then by Theorem 4-7, there is a unique solution. Since $1\cdot 26 - 5 \cdot 5 = 1$, then $(-5) \cdot 5 \equiv 1 \pmod{26}$. Hence, multiplying the original equation by $-5$ on both sides gives us $x \equiv (-5) \cdot 2 \equiv -10 \equiv 16 \pmod{26}$. Therefore, $x = 16$ is the unique solution.
        \item Let $d = \gcd(6, 21) = 3$, then by Theorem 4-7, there are three incongruent solutions since $d \mid 15$. To find a special solution, divide the original congruence by 3 to get the equation $2x \equiv 5 \pmod{7}$ and multiply both sides of the congruence by $4$ to get $x \equiv 20 \equiv 6 \pmod{7}$. It follows that the solutions of the original equation are $x = 6 + 7t$ for $t = 0,1,2$. Therefore, the solutions are precisely $x = 6, 13, 20$.
        \item Let $d = \gcd(36, 102) = 6$. Since 8 is not divisible by $d$, then the equation has no solutions.
        \item Let $d = \gcd(34, 98) = 2$, then by Theorem 4-7, there are two incongruent solutions since $d \mid 60$. To find a special solution, divide the original congruence by 2 to get the equation $17x \equiv 30 \pmod{49}$ and multiply both sides of the congruence by $23$ to get
        $$x \equiv 23\cdot 30 \equiv 690 \equiv 4 \pmod{49}.$$ It follows that the solutions of the original equation are $x = 4 + 49t$ for $t = 0,1$. Therefore, the solutions are precisely $x = 4$ and $x = 53$.
        \item Let $d = \gcd(140, 301) = 7$, then by Theorem 4-7, there are seven incongruent solutions since $d \mid 133$. To find a special solution, divide the original congruence by 7 to get the equation $20x \equiv 19 \pmod{43}$ and multiply both sides of the congruence by $15$ to get $x \equiv 285 \equiv 27 \pmod{43}$. It follows that the solutions of the original equation are $x = 27 + 43t$ for $t = 0,1,2,3,4,5,6$. Therefore, the solutions are precisely $x = 27, 70, 113, 156, 199, 242, 285$. \\
    \end{enumerate}
\end{solution}

\begin{exercise}
    Using congruences, solve the Diophantine equations below:
    \begin{enumerate}
        \item $4x + 51y = 9$. [\textit{Hint:} $4x \equiv 9 \pmod{51}$ gives $x = 15 + 51t$, while $51y \equiv 9 \pmod{4}$ gives $y = 3 + 4s$. Find the relation between $s$ and $t$.]
        \item $12x + 25y = 331$.
        \item $5x - 53y = 17$.
    \end{enumerate}
\end{exercise}

\begin{solution}
    \begin{enumerate}
        \item First, take the equation modulo 51 to get the congruence $4x \equiv 9 \pmod{51}$. Since $4 \cdot 13 - 51 = 1$, then mumtiplying the congruence by $13$ on both sides gives us $x \equiv 9 \cdot 13 \equiv 117 \equiv 15 \pmod{51}$. Hence, $x = 15 + 51t$. Similarly, we can take the original equation modulo 4 to get the congruence $3y \equiv 1 \pmod{4}$. Multiplying by 3 on both sides gives $x \equiv 3 \pmod{4}$. Thus, $y = 3 + 4s$. Plugging $x = 15 + 51t$ and $y = 3 + 4s$ in the original equation gives us 
        $$4(15 + 51t) + 51(3 + 4s) = 9$$
        which can be simplified into $t + s = -1$ and so $s = -1 - t$. Hence, we have $x = 15 + 51t$ and $s = -1 - 4t$ for an integer $t$. Let's now prove that any pair of integers of this form is a solution of the equation:
        $$4(15 + 51t) + 51(-1-4t) = 4 \cdot 15 -51 = 9$$
        for every integer $t$. Therefore, the solutions are precisely $x = 15 + 51t$ and $y = -1 - 4t$ for all integers $t$.
        \item First, take the equation modulo 25 to get the congruence $12x \equiv 6 \pmod{25}$. Since $25 - 2 \cdot 12 = 1$, then mumtiplying the congruence by $-2$ on both sides gives us $x \equiv (-2) \cdot 6 \equiv 13 \pmod{25}$. Hence, $x = 13 + 25t$. Similarly, we can take the original equation modulo 12 to get the congruence $y \equiv 7 \pmod{12}$. Thus, $y = 7 + 12s$. Plugging $x = 13 + 25t$ and $y = 7 + 12s$ in the original equation gives us 
        $$12(13 + 25t) + 25(7 + 12s) = 331$$
        which can be simplified into $t + s = 0$ and so $s =- t$. Hence, we have $x = 13 + 25t$ and $s = 7 - 12t$ for an integer $t$. Let's now prove that any pair of integers of this form is a solution of the equation:
        $$12(13 + 25t) + 25(7-12t) = 12 \cdot 13 + 25 \cdot 7 = 331$$
        for every integer $t$. Therefore, the solutions are precisely $x = 13 + 25t$ and $y = 7 - 12t$ for all integers $t$.
        \item First, take the equation modulo 53 to get the congruence $5x \equiv 17 \pmod{53}$. Since $2 \cdot 53 - 21\cdot 5 = 1$, then multiplying the congruence by $-21$ on both sides gives us $x \equiv (-21) \cdot 17 \equiv -39 \equiv 14 \pmod{53}$. Hence, $x = 14 + 53t$. Similarly, we can take the original equation modulo 5 to get the congruence $2y \equiv 2 \pmod{5}$. Multiplying by 3 on both sides gives $x \equiv 1 \pmod{5}$. Thus, $y = 1 + 5s$. Plugging $x = 14 + 53t$ and $y = 1 + 5s$ in the original equation gives us 
        $$5(14 + 53t) - 53(1 + 5s) = 17$$
        which can be simplified into $t -s = 0$ and so $s = t$. Hence, we have $x = 14 + 53t$ and $s = 1 + 5t$ for an integer $t$. Let's now prove that any pair of integers of this form is a solution of the equation:
        $$5(14 + 53t) - 53(1 + 5t) = 5 \cdot 14 - 53 = 17$$
        for every integer $t$. Therefore, the solutions are precisely $x = 14 + 53t$ and $y = 1 + 5t$ for all integers $t$. \\
    \end{enumerate}
\end{solution}

\begin{exercise}
    Find all solutions of the linear congruence $3x - 7y \equiv 11 \pmod{13}$. \\
\end{exercise}

\begin{solution}
    First, fix $y = t$ an rearrange the equation into $3x \equiv 11 + 7t \pmod{13}$. Since $\gcd(3, 13) = 1$, then this equation has precisely one solution. To find it, multiply both sides of the congruence by 9 to get $x \equiv 9(11 + 7t) \pmod{13}$. Hence, we have that the solutions are $x = 9(11 + 7t)$, $y = t$ for all $t$ between 0 and 12. \\
\end{solution}

\begin{exercise}
    Solve each of the following sets of simultaneous congruences:
    \begin{enumerate}
        \item $x \equiv 1 \pmod 3$, $x \equiv 2 \pmod 5$, $x \equiv 3 \pmod 7$
        \item $x \equiv 5 \pmod{11}$, $x \equiv 14 \pmod{29}$, $x \equiv 15 \pmod{31}$
        \item $x \equiv 5 \pmod 6$, $x \equiv 4 \pmod{11}$, $x \equiv 3 \pmod{17}$
        \item $2x \equiv 1 \pmod 5$, $3x \equiv 9 \pmod 6$, $4x \equiv 1 \pmod 7$, $5x \equiv 9 \pmod{11}$
    \end{enumerate}
\end{exercise}

\begin{solution}
    \begin{enumerate}
        \item Let $n = 3 \cdot 5 \cdot 7 = 105$, $N_1 = n / 3 = 35$, $N_2 = n / 5 = 21$ and $N_3 = n / 7 = 15$. Consider the linear congruences 
        $$35x \equiv 1 \pmod{3}, \qquad 21x \equiv 1 \pmod{5}, \qquad 15x \equiv 1 \pmod{7}$$
        which have the respective solutions $x_1 = 2$, $x_2 = 1$ and $x_3 = 1$. Define the integer $\overline{x} = 52 \equiv 1\cdot 35 \cdot 2 + 2 \cdot 21 \cdot 1 + 3 \cdot 15 \cdot 1 \pmod{n}$. By Theorem 4-8, we have that general solution is given by $x = 52 + 105t$ where $t$ is any integer.
        \item Let $n = 11 \cdot 29 \cdot 31 = 9889$, $N_1 = n / 11 = 899$, $N_2 = n / 29 = 341$ and $N_3 = n / 31 = 319$. Consider the linear congruences 
        $$899x \equiv 1 \pmod{11}, \qquad 341x \equiv 1 \pmod{29}, \qquad 319x \equiv 1 \pmod{31}$$
        which have the respective solutions $x_1 = 7$, $x_2 = 4$ and $x_3 = 7$. Define the integer $\overline{x} = 4944 \equiv 5\cdot 899 \cdot 7 + 14 \cdot 341 \cdot 4 + 15 \cdot 319 \cdot 7 \pmod{n}$. By Theorem 4-8, we have that general solution is given by $x = 4944 + 9889t$ where $t$ is any integer.
        \item Let $n = 6 \cdot 11 \cdot 17 = 1122$, $N_1 = n / 6 = 187$, $N_2 = n / 11 = 102$ and $N_3 = n / 17 = 66$. Consider the linear congruences 
        $$187x \equiv 1 \pmod{6}, \qquad 102x \equiv 1 \pmod{11}, \qquad 66x \equiv 1 \pmod{17}$$
        which have the respective solutions $x_1 = 1$, $x_2 = 4$ and $x_3 = 8$. Define the integer $\overline{x} = 785 \equiv 5\cdot 187 \cdot 1 + 4 \cdot 102 \cdot 4 + 3 \cdot 66 \cdot 8 \pmod{n}$. By Theorem 4-8, we have that general solution is given by $x = 785 + 1122t$ where $t$ is any integer.
        \item First, let's rewrite and simplify each individual congruences. The congruence $2x \equiv 1 \pmod 5$ is equivalent to $x \equiv 3 \pmod 5$ by multiplying by 3 on both sides. Let's solve the congruence $3x \equiv 9 \pmod{6}$ using the methods that we developed in this section. If we let $d = \gcd(3, 6) = 3$, then we have that $d \mid 9$ which implies that the congruence has 3 solutions modulo 6. To find these solutions, divide the congruence by $d$ to get the congruence $x \equiv 1 \pmod 2$. It follows that the solutions are $1, 3, 5$ which correspond precisely to the odd integers. Therefore, the congruence $3x \equiv 9 \pmod 6$ is precisely equivalent to the congruence $x \equiv 1 \pmod{2}$. Next, multiplying on both sides of the congruence $4x \equiv 1 \pmod 7$ by 2 shows that it is equivalent to the congruence $x \equiv 2 \pmod 7$. Finally, multiplying on both sides of the congruence $5x \equiv 9 \pmod{11}$ by 9 shows that it is equivalent to the congruence $x \equiv 4 \pmod{11}$. Therefore, the system becomes 
        $$x \equiv 3 \pmod 5, \quad x \equiv 1 \pmod 2, \quad x \equiv 2 \pmod 7, \quad x \equiv 4 \pmod{11}.$$
        Let $n = 5 \cdot 2 \cdot 7 \cdot 11 = 770$, $N_1 = n / 5 = 154$, $N_2 = n / 2 = 385$, $N_3 = n / 7 = 110$ and $N_4 = n / 11 = 70$. Consider the linear congruences
        \begin{align*}
            154x &\equiv 1 \pmod{5}, &385x &\equiv 1 \pmod{2}, \\
            110x &\equiv 1 \pmod{7}, &70x &\equiv 1 \pmod{11},
        \end{align*} 
        which have the respective solutions $x_1 = 4$, $x_2 = 1$, $x_3 = 3$ and $x_4 = 3$. Define the integer $\overline{x} = 653 \equiv 3\cdot 154 \cdot 4 + 1\cdot 385 \cdot 1 + 2\cdot 110 \cdot 3 + 4 \cdot 70 \cdot 3 \pmod{n}$. By Theorem 4-8, we have that general solution is given by $x = 653 + 770t$ where $t$ is any integer. \\
    \end{enumerate}
\end{solution}

\begin{exercise}
    Solve the linear congruence $17x \equiv 3 \pmod{2\cdot 3\cdot 5 \cdot 7}$ by solving the system
    $$17x \equiv 3 \pmod 2, \ 17x \equiv 3 \pmod 3, \ 17x \equiv 3 \pmod 5, \ 17x \equiv 3 \pmod 7.$$
\end{exercise}

\begin{solution}
    First, let's solve the system of equation. Let's rewrite it as follows:
    $$x \equiv 1 \pmod 2, \quad 2x \equiv 0 \pmod 3, \quad 2x \equiv 3 \pmod 5, \quad 3x \equiv 3 \pmod 7.$$
    Next, multiply both sides of the second congruence by 2, multiply both sides of the third congruence by 3 and cancel out the 3's in the last congruence to obtain the system
    $$x \equiv 1 \pmod 2, \quad x \equiv 0 \pmod 3, \quad x \equiv 4 \pmod 5, \quad x \equiv 1 \pmod 7.$$
    Let's solve this system using the method developed before. Let $n = 2 \cdot 3 \cdot 5 \cdot 7 = 210$, $N_1 = n / 2 = 105$, $N_2 = n / 3 = 70$, $N_3 = n / 5 = 42$ and $N_4 = n / 7 = 30$. Let's now solve the congruences $N_i x_i = 1 \pmod{n_i}$ where $n_1 = 2$, $n_2 = 3$, $n_3 = 5$ and $n_4 = 7$. We get that $x_1 = 1$, $x_2 = 1$, $x_3 = 3$ and $x_4 = 4$. Now, we can define
    $$\overline{x} = 99 \equiv 1 \cdot 105 \cdot 1 + 0 \cdot 70 \cdot 1 + 4 \cdot 42 \cdot 3 + 1 \cdot 30 \cdot 4.$$
    By Theorem 4-8, we have that 99 solves the original system of equation. In other words, we have that 2, 3, 5 and 7 divide the quantity $17\cdot 99 - 3$. Since they are relatively prime, we get that their product also divides this quantity. Therefore, we have that $17 \cdot 99 \equiv 3 \pmod{2 \cdot 3 \cdot 5 \cdot 7}$ and so $x = 99$ is the unique solution modulo $2 \cdot 3 \cdot 5 \cdot 7$. \\
\end{solution}

\begin{exercise}
    Find the smallest integer $a > 2$ such that 
    $$2 \mid a, \ 3 \mid a + 1, \ 4 \mid a + 2, \ 5 \mid a+ 3, \ 6 \mid a + 4.$$
\end{exercise}

\begin{solution}
    First, notice that we can discard the condition that $2 \mid a$ since $4 \mid a + 2$ already implies it. Moreover, we rewrite each of these conditions as the following congruences:
    $$a \equiv 2 \pmod{3}, \quad a \equiv 2 \pmod{4}, \quad a \equiv 2 \pmod{5}, \quad a \equiv 2 \pmod{6}.$$
    If we focus on the first three congruences, then by Theorem 4-8, we have that there is a unique solution modulo $3 \cdot 4 \cdot 5 = 60$. Since 2 is clearly a solution, then now have to solve the congruence $a \equiv 2 \pmod 6$ such that $a = 2 + 60t > 2$. When $t = 1$, we have that $a = 62$ satisfies $a \equiv 2 \pmod 6$. Thus, $a = 62$ is indeed the smallest integer $a > 2$ such that the divisibility conditions are satisfied. \\
\end{solution}

\begin{exercise}
    \begin{enumerate}
        \item Obtain three consecutive integers each having a square factor. [\textit{Hint:}  Find an integer $a$ such that $2^2 \mid a$, $3^2 \mid a + 1$, $5^2 \mid a + 2$.]
        \item Obtain three consecutive integers, the first of which is divisible by a square, the second by a cube, and the third by a fourth power.
    \end{enumerate}
\end{exercise}

\begin{solution}
    \begin{enumerate}
        \item Let's find an integer $a$ such that $2^2 \mid a$, $3^2 \mid a+ 1$ and $5^2 \mid a + 2$. Equivalently, let's solve the system
        $$a \equiv 0 \pmod{2^2}, \qquad a \equiv - 1 \pmod{3^2}, \qquad a \equiv - 2 \pmod{5^2}$$
        by applying the method developed in this section. Let $n = 2^2 \cdot 3^2 \cdot 5^2 = 900$, $N_1 = n / 2^2 = 225$, $N_2 = n / 3^2 = 100$ and $N_3 = n / 5^2 = 36$. Next, notice that $x_2 = 1$ and $x_3 = 16$ satisfy the equations $N_2 x_2 \equiv 1 \pmod{3^2}$ and $N_3 x_3 \equiv 1 \pmod{5^2}$ so we can define the integer 
        $$\overline{x} = 548 \equiv 0 + (-1) \cdot 100 \cdot 1 + (-2) \cdot 36 \cdot 16 \pmod{900}.$$
        It follows that the three consecutive integers 548, 549, 550 are respectively divisible by $2^2$, $3^2$ and $5^2$.
        \item Let's find an integer $a$ such that $5^2 \mid a$, $3^3 \mid a+ 1$ and $2^4 \mid a + 2$. Equivalently, let's solve the system
        $$a \equiv 0 \pmod{5^2}, \qquad a \equiv - 1 \pmod{3^3}, \qquad a \equiv - 2 \pmod{2^4}$$
        by applying the method developed in this section. Let $n = 5^2 \cdot 3^3 \cdot 2^4 = 10800$, $N_1 = n / 5^2 = 432$, $N_2 = n / 3^3 = 400$ and $N_3 = n / 2^4 = 675$. Next, notice that $x_2 = 16$ and $x_3 = 11$ satisfy the equations $N_2 x_2 \equiv 1 \pmod{3^3}$ and $N_3 x_3 \equiv 1 \pmod{2^4}$ so we can define the integer 
        $$\overline{x} = 350 \equiv 0 + (-1) \cdot 400 \cdot 16 + (-2) \cdot 675 \cdot 11 \pmod{10800}.$$
        It follows that the three consecutive integers 350, 351, 352 are respectively divisible by $5^2$, $3^3$ and $2^4$. \\
    \end{enumerate}
\end{solution}

\begin{exercise}
    (Brahmagupta, 7th century A.D.) When eggs in a basket are removed 2,3,4,5,6 at a time there remain respectively, 1, 2, 3, 4, 5 eggs. When they are taken out 7 at a time, none are left over. Find the smallest number of eggs that could have been contained in the basket. \\
\end{exercise}

\begin{solution}
    First, let's rewrite this problem as a system of congruence. Let $x$ be the number of eggs in the basket, then the first part of the problem states that $x \equiv -1 \pmod{i}$ for all $i = 2,3,4,5,6$. The second part states that $x \equiv 0 \pmod 7$. Hence, we have the system
    $$x \equiv -1 \pmod 2, \qquad x \equiv -1 \pmod 3, \qquad x \equiv -1 \pmod 4,$$
    $$x \equiv -1 \pmod 5, \qquad x \equiv -1 \pmod 6, \qquad x \equiv 0 \pmod 7.$$
    Notice that the congruences $x \equiv -1 \pmod 2$ and $x \equiv -1 \pmod 3$ are equivalent to the fact that 2 and 3 divide $x + 1$. Since they are relatively prime, then $6 \mid x + 1$ and so we get that the congruence $x \equiv -1 \pmod 6$ is already implied by the others. Hence, we can remove it from the system since it is redundant. Next, from the congruence $x \equiv -1 \pmod{4}$, we directly get that $x \equiv -1 \pmod 2$ and so this latter congruence can also be removed from the system. Hence, our system is now
    \begin{align*}
        x &\equiv -1 \pmod 3, & x &\equiv -1 \pmod 4, \\
        x &\equiv -1 \pmod 5, & x &\equiv 0 \pmod 7.
    \end{align*}
    Now, notice that 3, 4, and 5 are relatively prime and so by the method we developed in this section, there is a unique solution modulo $3 \cdot 4 \cdot 5 = 60$. Since $-1$ is clearly a solution, then the previous system is equivalent to
    $$x \equiv -1 \pmod{60}, \qquad x \equiv 0 \pmod 7.$$
    Using the first congruence, we get that $x$ must be one of the numbers 59, 119, 179, \dots which are of the form $60t - 1$. By looking at these numbers, we see that 59 is not divisible by 7 but 119 is. Therefore, the smallest number of eggs that could have been contained in the basket is 119. \\
\end{solution}

\begin{exercise}
    The basket-of-eggs problem problem is often phrased in the following form: One egg remains when the eggs are removed from the basket 2, 3, 4, 5, or 6 at a time; but, no eggs remain if they are removed 7 at a time. Find the smallest number of eggs that could have been in the basket. \\
\end{exercise}

\begin{solution}
    The solution to this problem will be very similar to the solution to the previous one. First, let's rewrite this problem as a system of congruence. Let $x$ be the number of eggs in the basket, then the first part of the problem states that $x \equiv 1 \pmod{i}$ for all $i = 2,3,4,5,6$. The second part states that $x \equiv 0 \pmod 7$. Hence, we have the system
    $$x \equiv 1 \pmod 2, \qquad x \equiv 1 \pmod 3, \qquad x \equiv 1 \pmod 4,$$
    $$x \equiv 1 \pmod 5, \qquad x \equiv 1 \pmod 6, \qquad x \equiv 0 \pmod 7.$$
    Notice that the congruences $x \equiv 1 \pmod 2$ and $x \equiv 1 \pmod 3$ are equivalent to the fact that 2 and 3 divide $x - 1$. Since they are relatively prime, then $6 \mid x - 1$ and so we get that the congruence $x \equiv 1 \pmod 6$ is already implied by the others. Hence, we can remove it from the system since it is redundant. Next, from the congruence $x \equiv 1 \pmod{4}$, we directly get that $x \equiv 1 \pmod 2$ and so this latter congruence can also be removed from the system. Hence, our system is now
    \begin{align*}
        x &\equiv 1 \pmod 3, & x &\equiv 1 \pmod 4, \\
        x &\equiv 1 \pmod 5, & x &\equiv 0 \pmod 7.
    \end{align*}
    Now, notice that 3, 4, and 5 are relatively prime and so by the method we developed in this section, there is a unique solution modulo $3 \cdot 4 \cdot 5 = 60$. Since 1 is clearly a solution, then the previous system is equivalent to
    $$x \equiv 1 \pmod{60}, \qquad x \equiv 0 \pmod 7.$$
    Using the first congruence, we get that $x$ must be one of the numbers 61, 121, 181, 241 \dots which are of the form $60t + 1$. By looking at these numbers, we see that 61, 121 and 181 are not divisible by 7 but $241 = 7 \cdot 33$ is. Therefore, the smallest number of eggs that could have been contained in the basket is 241. \\
\end{solution} 

\begin{exercise}
    (Ancient Chinese Problem.) A band of 17 pirates stole a sack of gold coins. When they tried to divide the fortune into equal portions, 3 coins remained. In the ensuing brawl over who should get the extra coins, one pirate was killed. The wealth was redistributed, but this time an equal division left 10 coins. Again an argument developed in which another pirate was killed. But now the total fortune was evenly distributed among the survivors. What was the least number of coins that could have been stolen. \\
\end{exercise}

\begin{solution}
    First, let's translate this problem into a system of congruences. If we denote by $x$ the number of stolen coins, we get the following system:
    $$x \equiv 3 \pmod{17}, \qquad x \equiv 10 \pmod{16}, \qquad x \equiv 0 \pmod{15}.$$
    Since 17, 16 and 15 are relatively prime, then we can directly apply the method developed in this section. Let $a_1 = 3$, $a_2 = 10$, $a_3 = 0$, $n_1 = 17$, $n_2 = 16$, $n_3 = 15$, $n = n_1n_2n_3 = 4080$, $N_1 = n/n_1 = 240$, $N_2 = n / n_2 = 255$ and $N_3 = n / n_3 = 272$. Let's now solve the congruences $N_i x_i \equiv 1 \pmod{n_i}$ to obtain $x_1 = 9$ and $x_2 = -1$. Now, define 
    \begin{align*}
        \overline{x} &= a_1N_1x_1 + a_2 N_2 x_2 + a_3 N_3 x_3 \\
        &= 3 \cdot 240 \cdot 9 + 10 \cdot 255 \cdot (-1) + 0 \\
        &= 6480 - 2550 \\
        &= 3930,
    \end{align*}
    then by Theorem 4-8, $\overline{x}$ is the unique solution to this system modulo $4080$. Since 3930 is the least positive integer of the form $\overline{x} + 4080t$, then 3930 is the least number of coins that could have been stolen. \\
\end{solution}

\begin{exercise}
    Prove that the congruences
    $$x \equiv a \pmod n  \ \text{ and } \ x \equiv b \pmod m$$
    admit a simultaneous solution if and only if $\gcd(n,m) \mid a - b$; if a solution exists, confirm that it is unique modulo $\lcm(n,m)$.\\
\end{exercise}

\begin{solution}
    First, suppose that a solution $x$ exists, then $x = a + kn$ and $x = b + k'm$ for some integers $k$ and $k'$. It follows that $a + kn = b + k'm$ and so $a - b = -kn + k'm$. Since $\gcd(n,m) \mid -kn + k'm$, then $\gcd(n,m) \mid a - b$. Conversely, if $xn + ym = \gcd(n,m) \mid a-b$, then $a - b = kxn + kym$ for some integer $k$. If we let $x = a - kxn = b + kym$, then it is clear that $x \equiv a \pmod n$ and $x \equiv b \pmod m$. This proves the equivalence.

    Suppose now that we have two solutions $x$ and $x'$, then $x \equiv x' \pmod a$ and $x \equiv x' \pmod m$. Equivalently, we have that $n$ and $m$ divide $x - x'$ and so $\lcm(n,m) \mid x - x'$. Therefore, $x \equiv x' \pmod{\lcm(n,m)}$ which implies that it is unique modulo $\lcm(n,m)$. \\
\end{solution}

\begin{exercise}
    Use Problem 11 to show that the system
    $$x \equiv 5 \pmod 6  \ \text{ and } \ x \equiv 7 \pmod{15}$$
    does not possess a solution. \\
\end{exercise}

\begin{solution}
    Since $\gcd(6, 15) = 3$ does not divide $5 - 7 = -2$, then by Problem 11, there is no simultaneous solution to this system of congruences. \\
\end{solution}

\begin{exercise}
    If $x \equiv a \pmod n$, prove that either $x \equiv a \pmod{2n}$ or $x \equiv a + n \pmod{2n}$. \\
\end{exercise}

\begin{solution}
    We have that $x = a + kn$ for some integer $k$ and by the Division Algorithm, we also have that $x = b + 2k'n$ for some integer $k'$. It follows that $a + kn = b + 2k'n$ and so $b = a + n(k - 2k') = a + nk_0$. If we take this equation modulo $2n$, we get that $nk_0$ is either congruent to 0 or $n$ modulo $2n$ and so $b \equiv a \pmod{2n}$ or $b \equiv a + n \pmod{2n}$. It follows that $x \equiv a \pmod{2n}$ or $x \equiv a + n \pmod{2n}$. \\
\end{solution}

\begin{exercise}
    A certain integer between 1 and 1200 leaves the remainders 1, 2, 6 when divided by 9, 11, 13 respectively. What is the integer? \\
\end{exercise}

\begin{solution}
    If we translate this problem into a system of congruences, we get that the number, which we will call $x$, must satisfy
    $$x \equiv 1 \pmod 9, \qquad x \equiv 2 \pmod{11}, \qquad x \equiv 6 \pmod{13}.$$
    Since 9, 11, and 13 are all relatively prime, then we can apply Theorem 4-8. Let $n = 9 \cdot 11 \cdot 13 = 1287$, $N_1 = 11 \cdot 13 = 143$, $N_2 = 9 \cdot 13 = 117$ and $N_3 = 9 \cdot 11 = 99$. Let $x_i$ denote the inverse of $N_i$ modulo 9, 11, and 13 successively, then $x_1 = 8$, $x_2 = -3$ and $x_3 = 5$. Hence, we can define $\overline{x} = 838 \equiv 1 \cdot N_1x_1 + 2 \cdot N_2x_2 + 6 \cdot N_3x_3 \pmod n$. By Theorem 4-8, the solutions to the system are precisely $838 + 1287t$ where $t$ is an integer. Since we want $1 \leq x \leq 1200$, then 838 is the integer we want. \\
\end{solution}

\begin{exercise}
    \begin{enumerate}
        \item Find an integer having the remainders 1, 2, 5, 5 when divided by 2, 3, 6, 12, respectively. (Yih-hing, died 717.)
        \item Find an integer having the remainders 2, 3, 4, 5 when divided by 3, 4, 5, 6, respectively. (Bhaskara, born 1114.)
        \item Find an integer having the remainders 3, 11, 15 when divided by 10, 13, 17, respectively. (Regiomontanus, 1436-1473)
    \end{enumerate}
\end{exercise}

\begin{solution}
    \begin{enumerate}
        \item It suffices to find a simultaneous solution to the congruences
        \begin{align*}
            x &\equiv 1 \pmod{2} & x &\equiv 2 \pmod{3} \\
            x &\equiv 5 \pmod{6} & x &\equiv 5 \pmod{12}
        \end{align*}
        First, notice that the first two congruences can be rewritten as
        $$x \equiv -1 \pmod 2 \qquad x \equiv - 1 \pmod{3}.$$
        Since 2 and 3 are relatively prime, then applying Theorem 4-8 tells us that there is a unique solution modulo 6. Since -1 is clearly a solution, then these two equations are equivalent to the equation $x \equiv -1 \equiv 5 \pmod{6}$ which was already in our system. Thus, we only need to find a simultaneous solution to the equations
        $$x \equiv 5 \pmod{6} \qquad x \equiv 5 \pmod{12}.$$
        But now, notice that the second equation implies the first one so the only thing we need to do is to find an integer $x$ such that its residue modulo 12 is 5. If we take $x = 5$, then it clearly satisfies all the conditions.
        \item It suffices to find a simultaneous solution to the congruences
        \begin{align*}
            x &\equiv -1 \pmod{3} & x &\equiv -1 \pmod{4} \\
            x &\equiv -1 \pmod{5} & x &\equiv -1 \pmod{6}.
        \end{align*}
        From the first and second congruences, we have that both 3 and 4 divide $x + 1$ and so 12 divide $x + 1$ as well ($\gcd(3,4) = 1$). This clearly implies that $6 \mid x + 1$ and hence, $x \equiv -1 \pmod 6$. It follows that we don't need to consider the last congruence, and so we need to solve the following system:
        $$x \equiv -1 \pmod{3}, \qquad x \equiv -1 \pmod{4}, \qquad x\equiv -1 \pmod 5.$$
        Since 3, 4 and 5 are relatively prime, then Theorem 4-8 tells us that there is a unique solution modulo $3 \cdot 4 \cdot 5 = 60$. Since $-1$ is clearly a solution, then we have that $60t - 1$ represents all possible solutions. Thus, we can take $x = 59$.
        \item We need to solve the following system of congruences:
        $$x \equiv 3 \pmod{10}, \qquad x \equiv 11 \pmod{13}, \qquad x\equiv 15 \pmod{17}.$$
        Since 10, 13 and 17 are relatively prime, then we can directly apply the method of Theorem 4-8. Let $n = 10 \cdot 13 \cdot 17 = 2210$, $N_1 = n / 10 = 221$, $N_2 = n / 13 = 170$ and $N_3 = n / 17 = 130$. After solving the equations $N_i x_i \equiv 1 \pmod{n_i}$, we get that $x_1 = 1$, $x_2 = 1$ and $x_3 = -3$. From this, we can define the integer $\overline{x} = 1103 \equiv 3 \cdot 221 \cdot 1 + 11 \cdot 170 \cdot 1 + 15 \cdot 130 \cdot (-3) \pmod n$. Indeed, we can check that 1103 checks all the conditions. \\
    \end{enumerate}
\end{solution} 

\begin{exercise}
    Let $t_n$ denote the $n$th triangular number. For which values of $n$ does $t_n$ divide $t_1^2 + t_2^2 + \dots + t_n^2$? [\textit{Hint:} Since $t_1^2 + t_2^2 + \dots + t_n^2 = t_n(3n^3 + 12n^2 + 13n + 2)/30$, it suffices to determine those $n$ satisfying $3n^3 + 12n^2 + 13n + 2 \equiv 0 \pmod{2 \cdot 3 \cdot 5}$.]\\
\end{exercise}

\begin{solution}
    As the hint tells us, lets find the $n$'s satisfying $3n^3 + 12n^2 + 13n + 2 \equiv 0 \pmod{2 \cdot 3 \cdot 5}$. By the Chinese Remainder Theorem, it suffices to solve the following system of congruences:
    \begin{align*}
        n^3 + n&\equiv 0 \pmod{2} \\
        n + 2 &\equiv 0 \pmod{3} \\
        (3n+2)(n^2+1) &\equiv 0 \pmod{5} 
    \end{align*}
    which we obtain after simplifications. First, notice that $n^3 + n$ is always even so the first equation can be discarded. The second equation can be replaced by $n \equiv 1 \pmod 3$ using the rules of congruences. The last equation tells us that 5 divides one of $3n + 2$ or $n^2 + 1$. For the first term, this happens when $n \equiv 1 \pmod 5$. For the second term, this happens when $n \equiv 2,3 \pmod 5$. Thus, the last congruence is equivalent to $n \equiv 1,2,3 \pmod 5$. Hence, our system now becomes
    \begin{align*}
        n &\equiv 1 \pmod{3} \\
        n &\equiv 1,2,3 \pmod{5} 
    \end{align*}
    To solve it, let's solve separatly the three systems where the last equation is $n \equiv 1 \pmod 5$, $n \equiv 2 \pmod 5$ and $n \equiv 3 \pmod 5$ successively. This is not harder than solving just one system because most of the method remains unchanged. Let's apply the method of Theorem 4-8. Let $n = 3 \cdot 5 = 15$, $N_1 = n / 3 = 5$, $N_2 = n / 5 = 3$. After solving the congruences $N_i x_i \equiv 1 \pmod{n_i}$, we get $x_1 = x_2 = 2$. Hence, the solutions of the system are
    $$1 \cdot 5 \cdot 2 + i \cdot 3 \cdot 2 = 10 + 6i$$
    up to a multiple of 15 and for $i = 1,2,3$. It follows that the solutions are precisely $n \equiv 1,7, 13 \pmod{15}$. \\
\end{solution}

\begin{exercise}
    Find the solutions of the system of congruences
    \begin{align*}
        3x + 4y &\equiv 5 \pmod{13} \\
        2x + 5y &\equiv 7 \pmod{13}.
    \end{align*}
\end{exercise}

\begin{solution}
    First, notice that $9 \cdot 3 \equiv 1 \pmod{13}$ and $7 \cdot 2 \equiv 1 \pmod{13}$ so if we multiply both sides of equations 1 and 2 by 9 and 7 respectively, we get the two equations
    \begin{align*}
        x + 10y &\equiv 6 \pmod{13} \\
        x + 9y &\equiv 10 \pmod{13}.
    \end{align*}
    If we subtract the first equation by the second one, we get that $y \equiv 9 \pmod{13}$. If we plug this last congruence into the second congruence above, we get that
    $$x \equiv 10 - 9 \cdot 9 \equiv 10 - 3 \equiv 7 \pmod{13}.$$ 
    Therefore, the solutions are given by $x \equiv 7 \pmod{13}$, $y \equiv 9 \pmod{13}$.\\
\end{solution}

\begin{exercise}
    Obtain the two incongruent solutions modulo 210 of the system
    \begin{align*}
        2x &\equiv 3 \pmod{5} \\
        4x &\equiv 2 \pmod{6} \\
        3x &\equiv 2 \pmod 7. 
    \end{align*}
\end{exercise}

\begin{solution}
    By multipliying the first equation by 3, we obtain the equivalent equation $x \equiv 4 \pmod{5}$. Let's solve the second equation. If we divide the congruence by 2, we get the equation $2x \equiv 1 \pmod 3$ which is equivalent to $x \equiv 2 \pmod{3}$. If we lift this solution modulo 6, we get that the two solutions of the second equation are $x \equiv 2, 5 \pmod 6$. In the third equation, multiply both sides by 5 to get $x \equiv 3 \pmod 7$. Therefore, our system is now
    \begin{align*}
        x &\equiv 4 \pmod{5} \\
        x &\equiv 2,5 \pmod{6} \\
        x &\equiv 3 \pmod 7. 
    \end{align*}
    To solve this system, we need to solve the two independent systems where the second equation changes and merge the solutions. To do this, let's apply Theorem 4-8. Let $n = 5 \cdot 6 \cdot 7 = 210$, $N_1 = n / 5 = 42$, $N_2 = n / 6 = 35$ and $N_3 = n / 7 = 30$. By solving the equations $N_i x_i \equiv 1 \pmod{n_i}$, we get $x_1 = 3$, $x_2 = -1$, $x_3 = 4$. Hence, if we compute
    $$4 \cdot 42 \cdot 3 + i \cdot 35 (-1) + 3 \cdot 30 \cdot 4 = 864 - 35i \equiv 234 - 35i \pmod{210}$$
    and replace $i$ by 2 and 5, we get that the two incongruent solutions modulo 210 are 59 and 164. \\
\end{solution} 

% CHAPTER 5
\chapter{Fermat's Theorem}

\section{Pierre De Fermat}

There are no exercises in this section.

\section{Fermat's Factorization Method}

\begin{exercise}
    Use Fermat's method to factor
    \begin{enumerate}
        \item 2279;
        \item 10541;
        \item 340663. [\textit{Hint:} The smallest square just exceeding 340663 is $584^2$.]
    \end{enumerate}
\end{exercise}

\begin{solution}
    \begin{enumerate}
        \item The smallest square greater than 2279 is $48^2 = 2304$. Thus, we get 
        $$48^2 - 2279 = 2304 - 2279 = 25 = 5^2$$
        which gives us the factorization
        $$2279 = 48^2 - 5^2 = (48 - 5)(48 + 5) = 43 \cdot 53.$$
        \item The smallest square greater than 10541 is $103^2 = 10609$. Thus, we get
        \begin{align*}
            103^2 - 10541 &= 10609 - 10541 = 68, \\
            104^2 - 10541 &= 10816 - 10541 = 275, \\
            105^2 - 10541 &= 11025 - 10541 = 484 = 22^2
        \end{align*}
        which gives us the factorization
        $$10541 = 105^2 - 22^2 = (105 - 22)(105 + 22) = 83 \cdot 127.$$
        \item The smallest square greater than 340663 is $584^2 = 341056$. Thus, we get
        \begin{align*}
            584^2 - 340663 &= 341056 - 340663 = 393, \\
            585^2 - 340663 &= 342225 - 340663 = 1562, \\
            586^2 - 340663 &= 343396 - 340663 = 2733, \\
            587^2 - 340663 &= 344569 - 340663 = 3906, \\
            588^2 - 340663 &= 345744 - 340663 = 5081, \\
            589^2 - 340663 &= 346921 - 340663 = 6258, \\
            590^2 - 340663 &= 348100 - 340663 = 7437, \\
            591^2 - 340663 &= 349281 - 340663 = 8618, \\
            592^2 - 340663 &= 350464 - 340663 = 9801 = 99^2
        \end{align*}
        which gives us the factorization
        $$340663 = 592^2 - 99^2 = (592 - 99)(592 + 99) = 493 \cdot 691.$$ \\
    \end{enumerate}
\end{solution}

\begin{exercise}
    Prove that a perfect square must end in one of the following pairs of digits: 00, 01, 04, 09, 16, 21, 24, 25, 29, 36, 41, 44, 49, 56, 61, 64, 69, 76, 81, 84, 89, 96. [\textit{Hint:} Since $x^2 \equiv (50 + x)^2 \pmod{100}$ and $x^2 \equiv (50 - x)^2 \pmod{100}$, it suffices to examine the final digits of $x^2$ for the 26 values $x = 0,1,2,...,25$.] \\
\end{exercise}

\begin{solution}
    As the hint tells us, since $x^2 \equiv (50 + x)^2 \pmod{100}$ and $x^2 \equiv (50 - x)^2 \pmod{100}$, it suffices to examine the final digits of $x^2$ for the 26 values $x = 0,1,2,...,25$:
    \begin{align*}
        0^2 &= 0 \equiv 00 \pmod{100} & 13^2 &= 169 \equiv 69 \pmod{100} \\
        1^2 &= 1 \equiv 01 \pmod{100} & 14^2 &= 196 \equiv 96 \pmod{100} \\
        2^2 &= 4 \equiv 04 \pmod{100} & 15^2 &= 225 \equiv 25 \pmod{100} \\
        3^2 &= 9 \equiv 09 \pmod{100} & 16^2 &= 256 \equiv 56 \pmod{100} \\
        4^2 &= 16 \equiv 16 \pmod{100} & 17^2 &= 289 \equiv 89 \pmod{100} \\
        5^2 &= 25 \equiv 25 \pmod{100} & 18^2 &= 324 \equiv 24 \pmod{100} \\
        6^2 &= 36 \equiv 36 \pmod{100} & 19^2 &= 361 \equiv 61 \pmod{100} \\
        7^2 &= 49 \equiv 49 \pmod{100} & 20^2 &= 400 \equiv 00 \pmod{100} \\
        8^2 &= 64 \equiv 64 \pmod{100} & 21^2 &= 441 \equiv 41 \pmod{100} \\
        9^2 &= 81 \equiv 81 \pmod{100} & 22^2 &= 484 \equiv 84 \pmod{100} \\
        10^2 &= 100 \equiv 00 \pmod{100} & 23^2 &= 529 \equiv 29 \pmod{100} \\
        11^2 &= 121 \equiv 21 \pmod{100} & 24^2 &= 576 \equiv 76 \pmod{100} \\
        12^2 &= 144 \equiv 44 \pmod{100} & 25^2 &= 625 \equiv 25 \pmod{100}.
    \end{align*}
    Therefore, a  perfect square must end in one of the following pairs of digits: 00, 01, 04, 09, 16, 21, 24, 25, 29, 36, 41, 44, 49, 56, 61, 64, 69, 76, 81, 84, 89, 96. \\
\end{solution}

\begin{exercise}
    Factor the number $2^{11} - 1$ by Fermat's factorization method. \\
\end{exercise}

\begin{solution}
    We have $2^{11} - 1 = 2048 - 1 = 2047$. The smallest square greater than 2047 is $46^2 = 2116$, thus
    \begin{align*}
        46^2 - 2047 &= 2116 - 2047 = 69, \\
        47^2 - 2047 &= 2209 - 2047 = 162, \\
        48^2 - 2047 &= 2304 - 2047 = 257, \\
        49^2 - 2047 &= 2401 - 2047 = 354, \\
        50^2 - 2047 &= 2500 - 2047 = 453, \\
        51^2 - 2047 &= 2601 - 2047 = 554, \\
        52^2 - 2047 &= 2704 - 2047 = 657, \\
        53^2 - 2047 &= 2809 - 2047 = 762, \\
        54^2 - 2047 &= 2916 - 2047 = 869, \\
        55^2 - 2047 &= 3025 - 2047 = 978, \\
        56^2 - 2047 &= 3136 - 2047 = 1089 = 33^2.
    \end{align*}
    Hence, we get the following factorization:
    $$2^{11} - 1 = 56^2 - 33^2 = (56 - 33)(56 + 33) = 23 \cdot 89.$$\\
\end{solution}

\begin{exercise}
    In 1647, Mersenne noted that when a number can be written as a sum of two relatively prime squares in two distinct ways, it is composite and can be factored as follows: if $n = a^2 + b^2 = c^2 + d^2$, then
    $$n = (ac + bd)(ac - bd)/(a+d)(a - d).$$
    Use this result to factor the numbers
    \begin{align*}
        493 &= 18^2 + 13^2 = 22^2 + 3^2, \\
        38025 &= 168^2 + 99^2 = 156^2 + 117^2.
    \end{align*}
\end{exercise}

\begin{solution}
    For the first equation, we can take $a = 18$, $b = 13$, $c = 22$ and $d = 3$:
    $$493 = \frac{(18\cdot 22 + 13 \cdot 3)(18\cdot 22 - 13 \cdot 3)}{(18 + 3)(18 - 3)} = \frac{357 \cdot 435}{21 \cdot 15} = \frac{357}{21} \cdot \frac{435}{15} = 17 \cdot 29.$$
    For the first equation, we can take $a = 168$, $b = 99$, $c = 117$ and $d = 156$:
    \begin{align*}
        38025 &= \frac{(168 \cdot 117 + 99 \cdot 156)(168 \cdot 117 - 99 \cdot 156)}{(168 + 156)(168 - 156)} \\
        &= \frac{35100 \cdot 4212}{324 \cdot 12} \\
        &= \frac{35100}{12} \cdot \frac{4212}{324} \\
        &= 2925 \cdot 13.
    \end{align*}
\end{solution}
\section{The Little Theorem}

\begin{exercise}
    Verify that $18^6 \equiv 1 \pmod{7^k}$ for $k = 1, 2, 3$. \\
\end{exercise}

\begin{solution}
    When $k = 1$, the congruence $18^6 \equiv 1 \pmod{7^k}$ follows from the little theorem. When $k = 2$, we have
    $$18^6 = (18^2)^3 \equiv 30^3 = 900\cdot 30 \equiv 18 \cdot 30 = 540 \equiv 1 \pmod{7^k}.$$
    Finally, when $k = 3$, we have 
    $$18^6 = (18^2)^3 = 324^3 \equiv (-19)^3 = - 361 \cdot 19 \equiv - 18 \cdot 19 = -342 \equiv 1 \pmod{7^k}.$$\\
\end{solution}

\begin{exercise}
    \begin{enumerate}
        \item If $\gcd(a, 35) = 1$, show that $a^{12} \equiv 1 \pmod{35}$. [\textit{Hint:} From Fermat's Theorem $a^6 \equiv 1 \pmod{7}$ and $a^4 \equiv 1 \pmod 5$.]
        \item If $\gcd(a, 42) = 1$, show that $168 = 3 \cdot 7 \cdot 8$ divides $a^6 - 1$.
        \item If $\gcd(a, 133) = \gcd(b, 133) = 1$, show that $133 \mid a^{18} - b^{18}$. 
    \end{enumerate}
\end{exercise}

\begin{solution}
    \begin{enumerate}
        \item First, since $\gcd(a, 35) = 1$, then $\gcd(a, 5) = \gcd(a, 7) = 1$. Hence, from the little theorem, we have the congruences $a^6 \equiv 1 \pmod 7$ and $a^4 \equiv 1 \pmod 5$. Squaring both sides of the first congruence and cubing both sides of the second congruence gives us the two congruences $a^{12} \equiv 1 \pmod 7$ and $a^{12} \equiv 1 \pmod 5$. Since 5 and 7 are relatively prime, then we get that $a^{12} \equiv 1 \pmod{35}$.
        \item First, since $\gcd(a, 42) = 1$, then $\gcd(a, 3) = \gcd(a, 7) = \gcd(a, 2) = 1$. From the fact that $\gcd(a, 3) = \gcd(a, 7) = 1$, and by the little theorem, we get the congruences $a^2 \equiv 1 \pmod 3$ and $a^6 \equiv 1 \pmod 7$. If we cube the first of these congruences, we get $a^6 \equiv 1 \pmod 3$. Hence, we have that both 3 and 7 divide $a^6 - 1$. Now, the equation $\gcd(a, 2) = 1$ asserts that $a$ is odd, it follows that $a \equiv \pm 1, \pm 3 \pmod 8$. When $a \equiv \pm 1 \pmod 8$, then $a^6 \equiv ((\pm 1)^2)^3 \equiv 1 \pmod 8$. When $a \equiv \pm 3 \pmod 8$, then $a^6 \equiv ((\pm 3)^2)^3 \equiv 1 \pmod 8$. Therefore, for all possible $a$, we have $a^6 \equiv 1 \pmod 8$, and hence, $8 \mid a^6 - 1$. Since 3, 7, and 8 are all relatively prime, then $168 = 3\cdot 7 \cdot 8$ divides $a^6 - 1$.
        \item First, since $\gcd(a, 133) = \gcd(b, 133) = 1$, then $\gcd(a, 19) = \gcd(a, 7) = \gcd(b, 19) = \gcd(b, 7) = 1$ using the fact that $133 = 7 \cdot 19$. Hence, from the little theorem, we have the congruences $a^{18} \equiv 1 \pmod{19}$ and $a^6 \equiv 1 \pmod 7$. Cubing both sides of the second congruence gives us the two congruences $a^{18} \equiv 1 \pmod{19}$ and $a^{18} \equiv 1 \pmod 7$. Since 7 and 19 are relatively prime, then we get that $a^{18} \equiv 1 \pmod{133}$. Since this is also true for $b$, then both $a$ and $b$ are congruent to 1 modulo 133, when taken to the power of 18. Hence, $a^{18} \equiv b^{18} \pmod{133}$, which is equivalent to $133 \mid a^{18} - b^{18}$. \\
    \end{enumerate}
\end{solution}

\begin{exercise}
    Prove that there exist infinitely many composite numbers $n$ for which $a^{n-1} \equiv a \pmod n$. [\textit{Hint:} Take $n = 2p$ where $p$ is an odd prime.]\\
\end{exercise}

\begin{solution}
    Let $n = 2p$ where $p$ is an odd prime, and let $a$ be an arbitrary integer. By the little theorem, we have that $a^{p-1} \equiv 1 \pmod{p}$ when $a$ is not divisible by $p$. In that case, squaring both sides gives us the congruence $a^{2p-2} \equiv 1 \pmod p$. Multiplying by $a$ on both sides gives us $a^{n-1} \equiv a \pmod p$, which is now true with no restrictions on $a$. The congruence $a^{n-1} \equiv a \pmod 2$ is trivial is we check the only cases $a \equiv 0 \pmod 2$ and $a \equiv 1 \pmod 2$. Since $2$ and $p$ are relatively prime, then $a^{n-1} \equiv a \pmod n$. Since there are infinitely many odd primes, then there are infinitely many composite integers $n$ such that $a^{n-1} \equiv a \pmod n$ holds for all integers $a$. \\
\end{solution}

\begin{exercise}
    Derive each of the following congruences:
    \begin{enumerate}
        \item $a^{21} \equiv a \pmod{15}$ for all $a$. [\textit{Hint:} By Fermat's Theorem, $a^5 \equiv a \pmod 5$.]
        \item $a^7 \equiv a \pmod {42}$ for all $a$.
        \item $a^{13} \equiv a \pmod{3 \cdot 7 \cdot 13}$ for all $a$.
        \item $a^{9} \equiv a \pmod{30}$ for all $a$.
    \end{enumerate}
\end{exercise}

\begin{solution}
    \begin{enumerate}
        \item By the little theorem, we have that $a^4 \equiv 1 \pmod 5$ for integers $a$ non-divisible by 5. Taking both sides to the fifth power gives us $a^{20} \equiv 1 \pmod 5$. Multiplying both sides by $a$ implies that $a^{21} \equiv a \pmod 5$ for any integers $a$ now. Similarly, by the little theorem, we have that $a^2 \equiv 1 \pmod 3$ for integers $a$ non-divisible by 3. Taking both sides to the tenth power gives us $a^{20} \equiv 1 \pmod 3$. Multiplying both sides by $a$ implies that $a^{21} \equiv a \pmod 3$ for any integers $a$ now. Since 3 and 5 are relatively prime, then it follows that $a^{21} \equiv a \pmod{15}$ for all integers $a$.
        \item Let $a$ be an integer, then by the little theorem, we have the congruences:
        \begin{align*}
            a^7 &\equiv a\cdot (a^{3})^2 \equiv a\cdot a^3 \equiv (a^2)^2 \equiv a \pmod 2, \\
            a^7 &\equiv a \cdot (a^2)^3 \equiv a\cdot a^2 \equiv a^3 \equiv a \pmod 3,\\
            a^7 &\equiv a \pmod 7.
        \end{align*}
        Since 2, 3, and 7 are relatively prime, then $a^7 \equiv a \pmod{42}$.
        \item Let $a$ be an integer, then by the little theorem, we have the congruences:
        \begin{align*}
            a^{13} &\equiv a\cdot (a^{4})^3 \equiv a\cdot a^4 \equiv a^2 \cdot a^3 \equiv a^2 \cdot a \equiv a^3 \equiv a \pmod 3, \\
            a^{13} &\equiv a^6 \cdot a^7 \equiv a^6\cdot a \equiv a^7 \equiv a \pmod 7,\\
            a^{13} &\equiv a \pmod{13}.
        \end{align*}
        Since 3, 7, and 13 are relatively prime, then $a^{13} \equiv a \pmod{3\cdot 7 \cdot 13}$.
        \item Let $a$ be an integer, then by the little theorem, we have the congruences:
        \begin{align*}
            a^9 &\equiv a\cdot (a^{4})^2 \equiv a\cdot a^4 \equiv a \cdot (a^2)^2 \equiv a \cdot a^2 \equiv a \cdot a \equiv a \pmod 2, \\
            a^9 &\equiv (a^3)^3 \equiv a^3 \equiv a \pmod 3,\\
            a^9 &\equiv a^4 \cdot a^5 \equiv a^4 \cdot a \equiv a^5 \equiv a \pmod 5.
        \end{align*}
        Since 2, 3, and 5 are relatively prime, then $a^9 \equiv a \pmod{30}$. \\
    \end{enumerate}
\end{solution}

\begin{exercise}
    If $\gcd(a, 30) = 1$, show that 60 divides $a^4 + 59$. \\
\end{exercise}

\begin{solution}
    If $\gcd(a, 30) = 1$, then $a$ is not divisible by 2, 3, and 5. Hence, by the little theorem, $a^2 \equiv 1 \pmod 3$ and $a^4 \equiv 1 \pmod 5$. Squaring the first congruence gives us $a^4 \equiv 1 \pmod 3$. Since $a$ is not divisible by 2, then $a \equiv \pm 1 \pmod 4$ and so $a^4 \equiv 1 \pmod 4$. Since 3, 4, and 5 are relatively prime, then $a^4 \equiv 1 \pmod{60}$. \\
\end{solution}

\begin{exercise}
    \begin{enumerate}
        \item Find the units digit of $3^{100}$ by the use of Fermat's Theorem.
        \item For any integer $a$, verify that $a^5$ and $a$ have the same units digit.
    \end{enumerate} 
\end{exercise}

\begin{solution}
    \begin{enumerate}
        \item By the little theorem, we have that $3^{100} \equiv 1^{100} \equiv 1 \pmod 2$ and
        $$3^{100} \equiv ((3^4)^5)^5 \equiv (3^4)^5 \equiv 3^4 \equiv 1 \pmod 5.$$
        Since 2 and 5 are relatively prime, then $3^{100} \equiv 1 \pmod{10}$ which implies that the units digit of $3^{100}$ is 1.
        \item By the little theorem, we have that
        \begin{align*}
            a^5 &\equiv a \cdot (a^2)^2 \equiv a\cdot a^2 \equiv a\cdot a \equiv 1 \pmod 2, \\
            a^5 &\equiv a \pmod 5.
        \end{align*}
        Since 2 and 5 are relatively prime, then $a^5 \equiv a \pmod{10}$ which implies that $a^5$ and $a$ have the same units digit. \\
    \end{enumerate}
\end{solution}

\begin{exercise}
    If $7 \nmid a$, prove that either $a^3 + 1$ or $a^3 - 1$ is divisible by 7. [\textit{Hint:} Apply Fermat's Theorem.] \\
\end{exercise}

\begin{solution}
    Since $7 \nmid a$, then by the little theorem, $a^6 \equiv 1 \pmod 7$, and so $7 \mid a^6 - 1$. But since $a^6 - 1 = (a^3 - 1)(a^3 + 1)$ and 7 is prime, then 7 must divide either $a^3 + 1$ or $a^3 - 1$. \\
\end{solution}

\begin{exercise}
    The three most recent appearances of Halley's comet were in the years 1835, 1910, and 1986; the next occurence will be in 2061. Prove that
    $$1835^{1910} + 1986^{2061} \equiv 0 \pmod 7.$$ \\
\end{exercise}

\begin{solution}
    First, notice that $1835 \equiv 1 \pmod 7$, $1986 \equiv -2 \pmod 7$, and $2061 = 6 \cdot 343 + 3$. By the little theorem, we then have
    \begin{align*}
        1835^{1910} + 1986^{2061} &\equiv 1^{1910} + (-2)^{2061} \\
        &\equiv 1 + (-2)^3((-2)^6)^{343} \\
        &\equiv 1 + (-8)\cdot 1^{343} \\
        &\equiv 1 - 8 \\
        &\equiv 0 \pmod 7
    \end{align*}
    which concludes our proof. \\
\end{solution}

\begin{exercise}
    \begin{enumerate}
        \item Let $p$ be a prime and $\gcd(a,p) = 1$. Use Fermat's Theorem to verify that $x \equiv a^{p-2}b \pmod p$ is a solution of the linear congruence $ax \equiv b \pmod p$.
        \item By applying part (a), solve the linear congruences $2x \equiv 1 \pmod{31}$, $6x \equiv 5 \pmod{11}$, and $3x \equiv 17 \pmod{29}$.
    \end{enumerate}
\end{exercise}

\begin{solution}
    \begin{enumerate}
        \item Suppose that $x \equiv a^{p-2}b$ where $\gcd(a, p) = 1$, then by the little theorem:
        $$ax \equiv a \cdot a^{p-2}b \equiv a^{p-1}b \equiv 1 \cdot b \equiv b \pmod p.$$
        \item For the equation $2x \equiv 1 \pmod{31}$, part (a) gives us the solution
        $$x \equiv 2^{29} = 2^4 \cdot (2^{5})^5 = 16 \cdot 32^5 \equiv 16 \cdot 1^5 = 16 \pmod{31}.$$
        For the equation $6x \equiv 5 \equiv -6 \pmod{11}$, part (a) gives us the solution
        $$x \equiv 6^9\cdot (-6) = -6^{10} \equiv - 3^5 = -3^2 \cdot 3^3 \equiv 2 \cdot 5 = 10 \equiv -1\pmod{11}.$$
        For the equation $3x \equiv 17\pmod{29}$, part (a) gives us the solution
        $$x \equiv 3^{27}\cdot 17 \equiv (-2)^9 \cdot 17 \equiv 2^9 \cdot 12 = 2^{10} \cdot 6 \equiv 3^2 \cdot 2 \cdot 3 \equiv -4 \pmod{29}.$$
    \end{enumerate}
\end{solution}

\begin{exercise}
    Assuming that $a$ and $b$ are integers not divisible by the prime $p$, establish the following:
    \begin{enumerate}
        \item If $a^p \equiv b^p \pmod p$, then $a \equiv b \pmod p$.
        \item If $a^p \equiv b^p \pmod p$, then $a^p \equiv b^p \pmod{p^2}$. [\textit{Hint:} By (a), $a = b = pk$ for some $k$, so that $a^p - b^p = (b+pk)^p - b^p$; now show that $p^2$ divides the latter expression.]
    \end{enumerate}
\end{exercise}

\begin{solution}
    \begin{enumerate}
        \item By the little theorem, we have that $a \equiv a^p \equiv b^p \equiv b \pmod p$.
        \item If $a^p \equiv b^p \pmod p$, then part (a) tells us that $a \equiv b \pmod p$ and so $a = b + pk$ for some integer $k$. Thus, by the Binomial Theorem, we have that
        \begin{align*}
            a^p - b^p &= (b + kp)^p - b^p \\
            &= b^p + \binom{p}{1}b^{p-1}pk + \binom{p}{2}b^{p-2}p^2k^2 + \dots + (pk)^p - b^p \\
            &= \binom{p}{1}b^{p-1}pk + \binom{p}{2}b^{p-2}(pk)^2 + \dots + (pk)^p.
        \end{align*}
        In the latter expression, every term that contains a fact of $(pk)^n$ is congruent to zero modulo $p^2$ for $n \geq 2$. Hence, we get that 
        $$a^p - b^p \equiv \binom{p}{1}b^{p-1}pk = p^2 b^{p-1}k \equiv 0 \pmod{p^2}.$$
        Therefore, $a^p \equiv b^p \pmod{p^2}$. \\
    \end{enumerate}
\end{solution}

\begin{exercise}
    Employ Fermat's Theorem to prove that, if $p$ is an odd prime, then 
    \begin{enumerate}
        \item $1^{p-1} + 2^{p-1} + 3^{p-1} + \dots + (p-1)^{p-1} \equiv -1 \pmod p$.
        \item $1^p + 2^p + 3^p + \dots + (p-1)^p \equiv 0 \pmod p$. [\textit{Hint:} Recall the identity $1 + 2 + 3 + \dots + (p-1) = p(p-1)/2$.]
    \end{enumerate} 
\end{exercise}

\begin{solution}
    \begin{enumerate}
        \item Since every integer between 1 and $p-1$ is not divisible by $p$, then by the little theorem, we get that
        $$1^{p-1} + 2^{p-1} + 3^{p-1} + \dots + (p-1)^{p-1} \equiv 1 + 1 + 1 + \dots + 1 = p-1 \equiv -1 \pmod p.$$
        \item By the little theorem, we get that  $$1^p + 2^p + 3^p + \dots + (p-1)^p \equiv 1 + 2 + 3 +\dots + (p-1) = p\frac{p-1}{2} \pmod p.$$
        Since $p$ is odd, then $(p-1)/2$ is an integer and so $p(p-1)/2 \equiv 0 \pmod p$. Therefore,
        $$1^p + 2^p + 3^p + \dots + (p-1)^p \equiv 0 \pmod p.$$
    \end{enumerate}
\end{solution}

\begin{exercise}
    Prove that if $p$ is an odd prime and $k$ is an integer satisfying $1 \leq k \leq p-1$, then the binomial coefficient
    $$\binom{p-1}{k} \equiv (-1)^k \pmod p.$$
\end{exercise}

\begin{solution}
    We know that the binomial coefficient $\binom{p-1}{k}$ can be written as 
    $$\binom{p-1}{k} = \frac{(p-1)!}{k!(p-k-1)!} = \frac{(p-1)\dots (p-k)}{k!}.$$
    Hence:
    $$k! \binom{p-1}{k} =(p-1)\dots (p-k) \equiv (-1)(-2)\dots(-k) = (-1)^k \cdot k! \pmod p $$
    Since $p$ doesn't divide $k!$, then we can cancel out the $k!$ on both sides of the congruence to get
    $$\binom{p-1}{k} \equiv (-1)^k \pmod p.$$ \\
\end{solution}

\begin{exercise}
    Assume that $p$ and $q$ are distinct odd primes such that $p-1 \mid q-1$. If $\gcd(a, pq) = 1$, show that $a^{q-1} \equiv 1 \pmod{pq}$. \\
\end{exercise}

\begin{solution}
    First, since $\gcd(a, pq) = 1$, then $\gcd(a, p) = \gcd(a, q) = 1$. Hence, by the little theorem, we have that $a^{q-1} \equiv 1 \pmod q$. Similarly, by the little theorem, we have that $a^{p-1} \equiv 1 \pmod p$. Since $p-1 \mid q-1$, then $q-1 = k(p-1)$. Taking both sides of the latter congruence to the $k$th power gives us $a^{q-1} \equiv 1 \pmod p$. Since $p$ and $q$ are relatively prime, then $a^{q-1} \equiv 1 \pmod{pq}$. \\
\end{solution}

\begin{exercise}
    If $p$ and $q$ are distinct primes, prove that
    $$p^{q-1} + q^{p-1} \equiv 1 \pmod{pq}.$$
\end{exercise}

\begin{solution}
    Since $p$ and $q$ are distinct primes, then we can apply the little theorem to get the congruences $p^{q-1} \equiv 1 \pmod{q}$ and $q^{p-1} \equiv 1 \pmod{p}$. If we add $q^{p-1}$ on both sides of the first congruence and if we add $p^{q-1}$ on both sides of the second congruence, we get the new congruences $p^{q-1} + q^{p-1} \equiv 1 + q^{p-1} \equiv 1 \pmod{q}$ and $p^{q-1} + q^{p-1} \equiv p^{q-1} + 1 \equiv 1 \pmod{p}$. Since $p$ and $q$ are relatively prime, then it follows that $p^{q-1} + q^{p-1} \equiv 1 \pmod{pq}$. \\
\end{solution}

\begin{exercise}
    Establish the statements below:
    \begin{enumerate}
        \item If the number $M_p = 2^p - 1$ is composite, where $p$ is a prime, then $M_p$ is pseudoprime.
        \item Every composite number $F_n = 2^{2^n} + 1$ is a pseudoprime ($n = 0,1,2\dots$). [\textit{Hint:} By Problem 21, Section 2.2, $2^{n+1} \mid 2^{2^n}$ implies that $2^{2^{n+1}} - 1 \mid 2^{2^{2^n}} - 1$; but $F_n \mid 2^{2^{n+1}} - 1$.]
    \end{enumerate}
\end{exercise}

\begin{solution}
    \begin{enumerate}
        \item Since $p$ is prime, $p$ divides $2^p - 2$ by the little theorem, and hence, by Problem 21, Section 2.2, we have that $M_p = 2^{p} - 1 \mid 2^{2^p - 2} - 1$. In other words, $M_p \mid 2^{M_p - 1} - 1$. Thus, $M_p \mid 2^{M_p} - 2$ which implies that $M_p$ is a pseudo prime (by assumption, we know that $M_p$ is composite).
        \item Since $n + 1 \leq 2^n$, then $2^{n+1} \mid 2^{2^n}$. By Problem 21, Section 2.2, we get that $2^{2^{n+1}} - 1 \mid 2^{2^{2^n}} - 1$. Since $2^{2^{n+1}} - 1 = (2^{2^n} - 1)(2^{2^n} + 1) = F_n(2^{2^n} + 1)$, then $F_n \mid 2^{2^{n+1}} - 1$ and so $F_n \mid 2^{2^{2^n}} - 1$. Multiplying the latter by 2 gives us that $F_n \mid 2^{2^{2^n} + 1} - 2 = 2^{F_n} - 2$. Since $F_n$ is composite, then it follows that $F_n$ is a pseudoprime.\\
    \end{enumerate}
\end{solution}

\begin{exercise}
    Confirm that the following integers are absolute pseudoprimes:
    \begin{enumerate}
        \item $1105 = 5 \cdot 13 \cdot 17$,
        \item $2821 = 7 \cdot 13 \cdot 31$,
        \item $2465 = 5 \cdot 17 \cdot 29$.
    \end{enumerate}
\end{exercise}

\begin{solution}
    \begin{enumerate}
        \item By the little theorem, we have that if $\gcd(a, 1105) = 1$, then $a^{4} \equiv 1 \pmod 5$, $a^{12} \equiv 1 \pmod{13}$, and $a^{16} \equiv 1 \pmod{17}$. Taking the previous congruences to the $276$th, $92$th, and $69$th power respectively gives us the new congruences $a^{1104} \equiv 1 \pmod 5$, $a^{1104} \equiv 1 \pmod{13}$, and $a^{1104} \equiv 1 \pmod{17}$. Since 5, 13, and 17 are relatively prime, then $a^{1104} \equiv 1 \pmod{1105}$, and so $a^{1105} \equiv a \pmod{1105}$ for every integer $a$. Therefore, 1105 is an absolute pseudoprime.
        \item By the little theorem, we have that if $\gcd(a, 2821) = 1$, then $a^{6} \equiv 1 \pmod 7$, $a^{12} \equiv 1 \pmod{13}$, and $a^{30} \equiv 1 \pmod{31}$. Taking the previous congruences to the $470$th, $235$th, and $94$th power respectively gives us the new congruences $a^{2820} \equiv 1 \pmod 7$, $a^{2820} \equiv 1 \pmod{13}$, and $a^{2820} \equiv 1 \pmod{31}$. Since 7, 13, and 31 are relatively prime, then $a^{2820} \equiv 1 \pmod{2821}$, and so $a^{2821} \equiv a \pmod{2821}$ for every integer $a$. Therefore, 2821 is an absolute pseudoprime.
        \item By the little theorem, we have that if $\gcd(a, 2465) = 1$, then $a^{4} \equiv 1 \pmod 5$, $a^{16} \equiv 1 \pmod{17}$, and $a^{28} \equiv 1 \pmod{29}$. Taking the previous congruences to the $616$th, $154$th, and $88$th power respectively gives us the new congruences $a^{2464} \equiv 1 \pmod 5$, $a^{2464} \equiv 1 \pmod{17}$, and $a^{2464} \equiv 1 \pmod{29}$. Since 5, 17, and 29 are relatively prime, then $a^{2464} \equiv 1 \pmod{2465}$, and so $a^{2465} \equiv a \pmod{2464}$ for every integer $a$. Therefore, 2465 is an absolute pseudoprime. \\
    \end{enumerate}
\end{solution}

\begin{exercise}
    Show that the pseudoprime 341 is not an absolute pseudoprime by showing that $11^{341} \not\equiv 11 \pmod{341}$. [\textit{Hint:} $31 \nmid 11^{341} - 11$.] \\
\end{exercise}

\begin{solution}
    Suppose that $11^{341} \equiv 11 \pmod{341}$, then $11^{341} \equiv 11 \pmod{31}$ since $31 \mid 341$. By the little theorem, we have that 
    $$11^{341} = (11^{11})^{31} \equiv 11^{11} \pmod{31}.$$
    Now, since $11^{11} \equiv 11 \cdot 11^{10} \equiv 11 \cdot (-3)^5 \pmod{31}$, then
    $$11^{11} \equiv (-33) \cdot 3^4 \equiv (-2)\cdot 81 \equiv (-2) \cdot (-12) = 24 \pmod{31}$$
    which implies that $11 \equiv 11^{341} \equiv 24 \pmod{31}$, a contradiction. Therefore, $11^{341} \not\equiv 11 \pmod{341}$ and so 342 is not an absolute pseudoprime. \\
\end{solution}

\begin{exercise}
    \begin{enumerate}
        \item When $n = 2p$, where $p$ is an odd prime, prove that $a^{n-1} \equiv a \pmod{n}$ for any integer $a$.
        \item For $n = 195 = 3 \cdot 5 \cdot 13$, verify that $a^{n-2} \equiv a \pmod{n}$ for any integer $a$.
    \end{enumerate}
\end{exercise}

\begin{solution}
    \begin{enumerate}
        \item By the little theorem, if $a$ is not divisible by $p$, then $a^{p-1} \equiv 1 \pmod p$. Squaring both sides gives us that $a^{n-2} \equiv 1 \pmod p$. Multiplying both sides by $a$ gives us that $a^{n-1} \equiv a \pmod p$ for any integer $a$. Similarly, for any integer $a$, we have that $a^{n-1} \equiv a \pmod 2$ by checking the only two cases $a \equiv 0,1 \pmod{2}$. Since $p$ is odd, then 2 and $p$ are relatively prime, and hence, $a^{n-1} \equiv a \pmod{n}$.
        \item By the little theorem, we have that
        \begin{align*}
            a^{n-2} &= a\cdot (a^{64})^3 \equiv a \cdot a^{64} \equiv a^2 \cdot (a^{21})^3 \equiv a^2 \cdot (a^7)^3 \equiv (a^3)^3 \equiv a \pmod 3, \\
            a^{n-2} &= a^3 (a^{38})^5 \equiv a \cdot a^{40} \equiv a (a^8)^5 \equiv a^4a^5 \equiv a^5 \equiv a \pmod 5,\\
            a^{n-2} &= a^{11}(a^{14})^{13} \equiv a^{11} \cdot a^{14} = a^{12}\cdot a^{13} \equiv a^{13} \equiv a \pmod{13}.
        \end{align*}
        Since 3, 5, and 13 are relatively prime, then $a^{n-2} \equiv a \pmod{n}$ for all $a$.\\
    \end{enumerate}
\end{solution}

\begin{exercise}
    Prove that any integer of the form
    $$n = (6k+1)(12k+1)(18k+1)$$
    is an absolute pseudoprime if all the three factors are prime; hence, $1729 = 7 \cdot 13 \cdot 19$ is an absolute pseudoprime. \\
\end{exercise}

\begin{solution}
    Let $p_1 = 6k+1$, $p_2 = 12k+1$, and $p_3 = 18k + 1$. Let $a$ be an integer not divisible by $p_1$, then by the little theorem, we have that $a^{6k} \equiv 1 \pmod{p_1}$, which implies that $a^{6kp_2p_3} \equiv 1 \pmod{p_1}$. Since $p_1 \mid p_2p_3 - 1$, then $a^{p_2p_3 - 1} \equiv 1 \pmod{p_1}$, and hence, $a^{6kp_2p_3 + p_2p_3 - 1} = a^{n - 1} \equiv 1 \pmod{p_1}$. Multiplying by $a$ on both sides gives us $a^n \equiv a \pmod{p_1}$. Using the same exact argument with the fact that $p_2 \mid p_1p_3 - 1$ and $p_3 \mid p_1p_2 - 1$ lets us conclude that $a^n \equiv a \pmod{p_2}$ and $a^n \equiv a \pmod{p_3}$. Since the $p_i$'s are distinct, then $a^n \equiv a \pmod n$. Therefore, $n$ is an absolute pseudoprime. \\
\end{solution}

\begin{exercise}
    Show that $561 \mid 2^{561} - 2$ and $561 \mid 3^{561} - 3$. It is an unanswered question whether there exist infinitely many composite numbers $n$ with the property that $n \mid 2^n - 2$ and $n \mid 3^n - 3$. \\
\end{exercise}

\begin{solution}
    First, notice that $561 = 3 \cdot 11 \cdot 17$. Using propoerties of congruences, we have that $2^{561} \equiv (-1)^{561} = -1 \equiv 2 \pmod 3$. By the little theorem, we have
    \begin{align*}
        2^{561} &\equiv 2 \cdot (2^{10})^{56} \equiv 2 \cdot 1^{56} = 2 \pmod{11} \\
        2^{561} &\equiv 2\cdot (2^{16})^{35} \equiv 2\cdot 1^{35} = 2 \pmod{17}.
    \end{align*}
    Since 3, 11, and 17 are relatively prime, then $2^{561} \equiv 2 \pmod{561}$. Similarly, we have that $3^{561} \equiv 0 \equiv 3 \pmod 3$, and by the little theorem:
    \begin{align*}
        3^{561} &\equiv 3 \cdot (3^{10})^{56} \equiv 3 \cdot 1^{56} = 3 \pmod{11} \\
        3^{561} &\equiv 3\cdot (3^{16})^{35} \equiv 3\cdot 1^{35} = 3 \pmod{17}.
    \end{align*}
    Since 3, 11, and 17 are relatively prime, then $3^{561} \equiv 3 \pmod{561}$.
\end{solution}
\section{Wilson's Theorem}

\begin{exercise}
    \begin{enumerate}
        \item Find the remainder when $15!$ is divided by 17.
        \item Find the remainder when $2(26!)$ is divided by 29. [\textit{Hint:} By Wilson's Theorem, $2(p-3)! \equiv -1 \pmod{p}$ for any odd prime $p > 3$.]
    \end{enumerate}
\end{exercise}

\begin{solution}
    \begin{enumerate}
        \item As we saw in the proof of Wilson's Theorem, $15! = (17-2)! \equiv 1 \pmod{17}$. Therefore, the remainder when $15!$ is divided by 17 is 1.
        \item As in the previous part, we have that $27! = (29-2)! \equiv 1 \pmod{29}$. Since $27! = 26!\cdot 27 \equiv -2\cdot 26! \pmod{29}$, then $2(26!) \equiv -1 \pmod{29}$. \\
    \end{enumerate}
\end{solution}

\begin{exercise}
    Determine whether 17 is a prime by deciding whether or not $16! \equiv -1 \pmod{17}$.\\
\end{exercise}

\begin{solution}
    Simply notice that
    \begin{align*}
        16! &= 1\cdot 2 \cdot 3 \cdot 4 \cdot 5 \cdot 6 \cdot 7 \cdot 8 \cdot 9 \cdot 10 \cdot 11 \cdot 12 \cdot 13 \cdot 14 \cdot 15 \cdot 16 \\
        &= (1\cdot 16)(2\cdot 9)(3\cdot 6)(4\cdot 13)(5\cdot 7)(8\cdot 15)(10\cdot 12)(11 \cdot 14) \\
        &\equiv (-1) \cdot 1 \cdot 1\cdot 1\cdot 1\cdot 1\cdot 1\cdot 1 \\
        &= -1 \pmod{17}.
    \end{align*}
    Therefore, 17 is a prime number. \\
\end{solution}

\begin{exercise}
    Arrange the integers 2, 3, 4, . . ., 21 in pairs $a$ and $b$ with the property that $ab \equiv 1 \pmod{23}$.\\
\end{exercise}

\begin{solution}
    Since $23 + 1 = 24$ can be written as $2\cdot 12$, $4\cdot 6$ and $8\cdot 3$, then this gives the three pairs $(2, 12)$, $(3, 8)$, and $(4, 6)$. Similarly, since $3\cdot 23 + 1 = 70$ and $70$ can be written as $5 \cdot 14$ and $7 \cdot 10$, then we get the pairs $(5, 14)$ and $(7,10)$. Since $7\cdot 23 + 1 = 162 = 9\cdot 18$, then we get the pair $(9, 18)$. Since $9\cdot 23 + 1 = 208 = 13\cdot 16$, then we get the pair $(13, 16)$. Since $10\cdot 23 + 1 = 231 = 11\cdot 21$, then we get the pair $(11, 21)$. Since $13\cdot 23 + 1 = 300 = 15\cdot 20$, then we get the pair $(15, 20)$ and so the last pair is $(17, 19)$. Therefore, the pairs are:
    $$(2, 12), \ (3,8), \ (4, 6), \ (5, 14), \ (7, 10), \ (9,18), \ (11, 21), \ (13, 16), \ (15, 20), \ (17, 19).$$
\end{solution}

\begin{exercise}
    Show that $18! \equiv -1 \pmod{437}$. \\
\end{exercise}

\begin{solution}
    First, notice that $437 = 19 \cdot 57$. By Wilson's Theorem, we have that $18! \equiv - 1 \pmod{19}$ and $22! \equiv - 1 \pmod{23}$. Since
    $$22! = 18!\cdot 19 \cdot 20 \cdot 21 \cdot 22 \equiv 18!(-4)(-3)(-2)(-1) = 18! \cdot 24 \equiv 18! \pmod{23},$$
    then we get that $18! \equiv - 1 \pmod{23}$ as well. Since 19 and 23 are relatively prime, we have that $18! \equiv - 1 \pmod{437}$. \\
\end{solution}

\begin{exercise}
    \begin{enumerate}
        \item Prove that any integer $n > 1$ is prime if and only if $(n-2)! \equiv 1 \pmod n$.
        \item If $n$ is a composite integer, show that $(n-1)! \equiv 0 \pmod n$, except when $n = 4$.
    \end{enumerate}
\end{exercise}

\begin{solution}
    \begin{enumerate}
        \item Let $n > 1$, then by Wilson's Theorem:
        \begin{align*}
            n \text{ prime} &\iff (n-1)! \equiv -1 \pmod n \\
            &\iff (n-2)!(n-1) \equiv -1 \pmod n \\
            &\iff (n-2)!(-1) \equiv -1 \pmod n \\
            &\iff (n-2)! \equiv 1 \pmod n.
        \end{align*}
        \item Suppose that $n$ is composite, then $n = ab$ with $a,b < n$. If $a \neq b$, then $(n-1)!$ is divisible by $ab$ and hence, $(n-1)! \equiv 0 \pmod n$. Suppose now that $n = a^2$, then using the fact that $4 < n$, we get that $2 < a$. Multiplying by $a$ on both sides gives $2a < n$. Since both $a$ and $2a$ are less than $n$ and distinct, then $a\cdot 2a \mid (n-1)!$. It follows that $(n-1)! \equiv 0 \pmod n$. \\
    \end{enumerate}
\end{solution}

\begin{exercise}
    Given a prime number $p$, establish the congruence
    $$(p-1)! \equiv p-1 \pmod{1 + 2 + 3 + \dots + (p-1)}.$$ 
\end{exercise}

\begin{solution}
    By Wilson's Theorem, we have that $p \mid (p-2)! - 1$, and so $p$ divides $2[(p-2)! - 1]$. Multiplying by $p-1$ on both sides of this division gives us that $p(p-1) \mid 2[(p-1)! - (p-1)]$. Since either $p$ or $p-1$ is even, then we can divide both sides of the division by 2 to get that $p(p-1)/2 \mid (p-1)! - (p-1)$. Finally, since $p(p-1)/2 = 1 + 2 + 3 + \dots + (p-1)$, then
    $$(p-1)! \equiv p-1 \pmod{1 + 2 + 3 + \dots + (p-1)}.$$\\
\end{solution}

\begin{exercise}
    If $p$ is prime, prove that
    $$p \mid a^p + (p-1)!a \ \text{ and } \ p \mid (p-1)!a^p + a$$
    for any integer $a$. [\textit{Hint:} By Wilson's Theorem, $a^p + (p-1)!a \equiv a^p -a \pmod p$.] \\
\end{exercise}

\begin{solution}
    Let $a$ be any integer, then by Fermat's Little Theorem, we have that $a^p \equiv a \pmod p$. By Wilson's Theorem, we have that $(p-1)! \equiv -1 \pmod p$. Thus, combining these two results, we get that $a^p + (p-1)! a \equiv a^p - a \equiv 0 \pmod p$ and $(p-1)!a^p + a \equiv a-a^p \equiv 0 \pmod p$. Therefore, $p$ divides $a^p + (p-1)!a$ and $(p-1)!a^p + a$. \\
\end{solution}

\begin{exercise}
    Find two odd primes $p \leq 13$ for which the congruence $(p-1)! \equiv -1 \pmod{p^2}$ holds. \\
\end{exercise}

\begin{solution}
    When $p = 5$, we have $p^2 = 25$ and $(p-1)! = 24$. Hence, $(p-1)! \equiv -1 \pmod{p^2}$ holds. Similarly, when $p = 13$, we have $p^2 = 169$. Since $6! = 720 \equiv 44 \pmod{169}$, then $7! \equiv 7\cdot 44 = 308 \equiv -30 \pmod{169}$. From this, we have that $8! \equiv 8(-30) = -240 \equiv -71 \pmod{169}$. If we continue this process, we get
    \begin{align*}
        9! &\equiv 9\cdot (-71) = -639 \equiv 37 \pmod{169}; \\
        10! &\equiv 10 \cdot 37 = 370 \equiv 32 \pmod{169}; \\
        11! &\equiv 11\cdot 32 = 352 \equiv 14 \pmod{169}; \\
        12! &\equiv 12 \cdot 14 = 168 \equiv - 1 \pmod{169}.
    \end{align*}
    Therefore, the desired property holds for $p = 5$ and $p = 13$. \\
\end{solution}

\begin{exercise}
    Using Wilson's Theorem, prove that
    $$1^2\cdot 3^2 \cdot 5^2 \dots (p-2)^2 \equiv (-1)^{(p+1)/2} \pmod{p}$$
    for any odd prime $p$. [\textit{Hint:} Since $k \equiv -(-k)\pmod{p}$, it follows that $2\cdot 4 \cdot 6 \dots (p-1) \equiv (-1)^{(p-1)/2}1\cdot 3 \cdot 5 \dots (p-2) \pmod{p}$.] \\
\end{exercise}

\begin{solution}
    Since $1 \equiv -(p-1) \pmod{p}$, $3 \equiv -(p-3)$, ..., $p-2 \equiv -2 \pmod{p}$, then
    \begin{align*}
        1^2\cdot 3^2 \cdot 5^2 \dots (p-2)^2 &\equiv 1(-1)(p-2)\cdot 3(-1)(p-3) \cdot \dots (p-2)(-1)\cdot 2 \\
        &= (-1)^{(p-1)/2}1\cdot 2\cdot 3 \dots (p-2)(p-1) \\
        &= (-1)^{(p-1)/2} (p-1)! \\
        &\equiv (-1)^{(p+1)/2} \pmod{p}.\\
    \end{align*}
\end{solution}

\begin{exercise}
    \begin{enumerate}
        \item For a prime $p$ of the form $4k+3$, prove that either
        $$\left(\frac{p-1}{2}\right)! \equiv 1 \pmod p \ \text{ or } \left(\frac{p-1}{2}\right)! \equiv -1 \pmod p;$$
        hence, $[(p-1)/2]!$ satisfies the quadratic congruence $x^2 \equiv 1 \pmod p$.
        \item Use part (a) to show that if $p = 4k+3$ is prime, then the product of all the even integers less than $p$ is congruent modulo $p$ to either 1 or $-1$. [\textit{Hint:} Fermat's Theorem implies that $2^{(p-1)/2} \equiv \pm 1 \pmod p$.]
    \end{enumerate}
\end{exercise}

\begin{solution}
    \begin{enumerate}
        \item In the second part of the proof of Theorem 5-5, it was shown that 
        $$-1 \equiv (-1)^{(p-1)/2}\left[\left(\frac{p-1}{2}\right)!\right]^2 \pmod p$$
        using Wilson's Theorem. When $p = 4k+3$, we have $(-1)^{(p-1)/2} = (-1)^{2k+1} = -1$ so the congruence becomes $-[(p-1)/2]!^2 \equiv -1 \pmod{p}$ which is equivalent to $[(p-1)/2]!^2 \equiv 1 \pmod{p}$. Rearranging, we get that 
        $$\left[\left(\frac{p-1}{2}\right)! - 1\right]\left[\left(\frac{p-1}{2}\right)! + 1\right] \equiv 0 \pmod{p}.$$
        Since $p$ is prime, then we can conclude that either
        $$\left(\frac{p-1}{2}\right)! \equiv 1 \pmod p \ \text{ or } \left(\frac{p-1}{2}\right)! \equiv -1 \pmod p.$$
        \item First, let $P$ be equal to the product of all even integers less than $p$ and notice that
        \begin{align*}
            P &= 2\cdot 4 \dots (p-3)(p-1) \\
            &= (2\cdot 1)(2\cdot 2)\dots \left(2\cdot \frac{p-3}{2}\right)\left(2\cdot \frac{p-1}{2}\right) \\
            &= 2^{(p-1)/2}\left(\frac{p-1}{2}\right)!
        \end{align*}
        Using part (a), we get that $P \equiv \pm 2^{(p-1)/2} \pmod p$. Next, using Fermat's Little Theorem, we have that $2^{p-1} \equiv 1 \pmod p$ which is equivalent to
        $$(2^{(p-1)/2} - 1)(2^{(p-1)/2} + 1) \equiv 0 \pmod p.$$
        Hence, $2^{(p-1)/2} \equiv \pm 1 \pmod p$. Therefore, $P \equiv \pm 1 \pmod{p}$. \\
    \end{enumerate}
\end{solution}

\begin{exercise}
    Apply Theorem 5-5 to find two solutions to the quadratic congruences $x^2 \equiv -1 \pmod{29}$ and $x^2 \equiv - 1 \pmod{37}$. \\
\end{exercise}

\begin{solution}
    To solve the first congruence, it suffices to find the least positive residue of $[(29-1)/2]! = 14!$ modulo 29. First, since $2\cdot 3 \cdot 5 \equiv 8\cdot 11 \equiv 9\cdot 13 \equiv 1 \pmod{29}$, then
    $$14! \equiv 4\cdot 6 \cdot 7 \cdot 10 \cdot 12 \cdot 14\pmod{29}.$$
    Now, since $6\cdot 10 \equiv 2 \pmod{29}$, and $4\cdot 7 \equiv -1 \pmod{29}$, then
    $$14! \equiv -2\cdot 12 \cdot 14 \pmod{29}.$$
    Finally, since $-2 \cdot 14 \equiv 1 \pmod{29}$, then $14! \equiv 12 \pmod{29}$. Therefore, a first solution to the quadratic congruence is given by $x \equiv 12 \pmod{29}$. Since $-x = -12 \equiv 17 \pmod{29}$ is also a solution, then the two solutions are $x \equiv 12,17 \pmod{29}$.
    
    Let's solve the second congruence. It suffices to find the least positive residue of $[(37 - 1)/2]! = 18!$ modulo 37. Since $2\cdot 18 \equiv -1 \pmod{37}$, then $3\cdot 4 \cdot 6 \cdot 18 \equiv (2\cdot 18)^2 \equiv 1 \pmod{37}$. It follows that
    $$18! \equiv 2\cdot 5 \cdot 7 \cdot 8 \cdot 9 \cdot 10 \cdot 11 \cdot 12 \cdot 13 \cdot 14 \cdot 15 \cdot 16 \cdot 17 \pmod{37}.$$
    Next, since $5 \cdot 15 \equiv 16 \cdot 7 \equiv 8 \cdot 14 \equiv 1  \pmod{37}$, then 
    $$18! \equiv 2 \cdot 9 \cdot 10 \cdot 11 \cdot 12 \cdot 13 \cdot 17 \pmod{37}.$$
    Since $10\cdot 11 \equiv 3\cdot 12 \equiv-1 \pmod{37}$, then $9 \cdot 10 \cdot 11 \cdot 12 \equiv 3 \pmod{37}$ and so
    $$18! \equiv 2 \cdot 3 \cdot 13 \cdot 17 \pmod{37}.$$
    Using the fact that $13\equiv - 24 \pmod{37}$ and $17 \equiv - 20 \pmod{37}$, we get that
    \begin{align*}
        18! &\equiv 2 \cdot 3 \cdot 24 \cdot 20 \\
        &\equiv 2\cdot 2 \cdot (3 \cdot 12) \cdot 20 \\
        &\equiv - 80 \\
        &\equiv 31\pmod{37}.
    \end{align*}
    Therefore, a first solution to the quadratic congruence is given by $x \equiv 31 \pmod{37}$. Since $-x = -31 \equiv 6 \pmod{37}$ is also a solution, then the two solutions are $x \equiv 6,31 \pmod{37}$. \\
\end{solution}

\begin{exercise}
    Show that if $p = 4k + 3$ is prime and $a^2 + b^2 \equiv 0 \pmod p$, then $a\equiv b \equiv 0 \pmod p$. [\textit{Hint:} If $a \not\equiv 0 \pmod p$, then there exists an integer $c$ such that $ac \equiv 1 \pmod p$; use this fact to contradict Theorem 5-5.] \\
\end{exercise}

\begin{solution}
    By contradiction, suppose that $a\neq\equiv 0 \pmod{p}$, then there is an integer $c$ such that $ac \equiv 1 \pmod{p}$. Multiplying both sides of the equation by $c^2$ gives us the new congruence $1 + (cb)^2 \equiv 0 \pmod{p}$ which implies that $cb$ is a solution to the equation $x^2 \equiv -1 \pmod{p}$. By Theorem 5-5, this is impossible so $a \equiv 0 \pmod p$. It follows that $b^2 \equiv 0 \pmod p$. By primality of $p$, it follows that $b \equiv 0 \pmod p$. Therefore, $a\equiv b \equiv 0 \pmod p$. \\
\end{solution}

\begin{exercise}
    Prove that the odd prime divisors of the integer $n^2 + 1$ are of the form $4k+1$. [\textit{Hint:} Theorem 5-5] \\
\end{exercise}

\begin{solution}
    Let $p$ be a prime divisor of $n^2 + 1$, then $n^2 + 1 \equiv 0 \pmod p$ and hence, $n$ is a solution to the equation $x^2 + 1 \equiv 0 \pmod p$. By Theorem 5-5, it directly follows that $p$ is of the form $4k+1$. \\
\end{solution}

\begin{exercise}
    Verify that $4(29!) + 5!$ is divisible by 31. \\
\end{exercise}

\begin{solution}
    Using Wilson's Theorem, we have that $29! \equiv 1 \pmod{31}$ which implies that $4(29!) + 5! \equiv 4 + 5! = 124 \equiv 0 \pmod{31}$. Therefore, 31 divides $4(29!) + 5!$. \\
\end{solution}

\begin{exercise}
    For a prime $p$ and $0 \leq k \leq p-1$, show that $k!(p-1-k)! \equiv (-1)^{k+1} \pmod{p}$. \\
\end{exercise}

\begin{solution}
    First, notice that
    $$k! \binom{p-1}{k} = (p-1)\dots(p-k) \equiv (-1)^k k! \pmod p$$
    and hence, $\binom{p-1}{k} \equiv (-1)^k \pmod p$ (since $k! \not\equiv 0 \pmod p$).
    Using Wilson's Theorem, we get that
    $$k!(p-k-1)!(-1)^k \equiv k!(p-k-1)!\binom{p-1}{k} \equiv (p-1)! \equiv -1 \pmod p.$$
    Therefore, $k!(p-1-k)! \equiv (-1)^{k+1} \pmod{p}$. \\
\end{solution}

\begin{exercise}
    If $p$ and $q$ are distinct primes, prove that
    $$pq \mid a^{pq} - a^p - a^q + a$$
    for any integer $a$. \\
\end{exercise}

\begin{solution}
    Let $a$ be an integer, then by Fermat's Little Theorem, $p \mid a - a^p$. Similarly, if we apply it again to $a^q$, then $p \mid a^{pq} - a^q$ as well. Thus, $p \mid a^{pq} - a^p - a^q + a$. For the same exact reason, $q \mid a^{pq} - a^p - a^q + a$. Since $p$ and $q$ are relatively prime, then $pq \mid a^{pq} - a^p - a^q + a$. \\
\end{solution}

\begin{exercise}
    Prove that if $p$ and $p+2$ are a pair of twin prime, then
    $$4((p-1)! + 1) + p \equiv 0 \pmod{p(p+2)}.$$
\end{exercise}

\begin{solution}
    First, if we work in modulo $p$, then using Wilson's Theorem:
    $$4((p-1)! + 1) + p \equiv 4((p-1)! + 1) \equiv 4(-1 + 1) \equiv 0 \pmod p.$$
    Next, let's work in modulo $p + 2$. Recall that $p \equiv -2 \pmod{p+2}$ and $p! \equiv 1 \pmod{p+2}$ by Wilson's Theorem. It follows that if we let $x = 4((p-1)! + 1) + p$, then
    $$-2x \equiv px \equiv 4(p! + p) + p^2 \equiv 4(1 - 2) + (-2)^2 \equiv 0 \pmod{p+2}.$$
    It follows that $4((p-1)! + 1) + p \equiv 0 \pmod{p+2}$. Since $p$ and $p+2$ are relatively prime, then $4((p-1)! + 1) + p \equiv 0 \pmod{p(p+2)}$.
\end{solution}

% CHAPTER 6
\chapter{Number Theoretic Functions}

\section{The Functions $\tau$ and $\sigma$}

\begin{exercise}
    Let $m$ and $n$ be positive integers and $p_1$, $p_2$, ..., $p_r$ be the distinct primes which divide at least one of $m$ or $n$. Then $m$ and $n$ may be written in the form 
    \begin{align*}
        m &= p_1^{k_1}p_2^{k_2}\dots p_r^{k_r}, \quad \text{ with } k_i \geq 0 \text{ for } i = 1,2,\dots, r \\
        n &= p_1^{j_1}p_2^{j_2}\dots p_r^{j_r}, \quad \text{ with } j_i \geq 0 \text{ for } i = 1,2,\dots, r
    \end{align*}
    Prove that 
    $$\gcd(m,n) = p_1^{u_1}p_2^{u_2}\dots p_r^{u_r}, \quad \lcm(m,n) = p_1^{v_1}p_2^{v_2}\dots p_r^{v_r},$$
    where $u_i = \min\{k_i, j_i\}$, the smaller of $k_i$ and $j_i$; and $v_i = \max\{k_i, j_i\}$, the larger of $k_i$ and $j_i$. \\
\end{exercise}

\begin{solution}
    First, if we let $g = p_1^{u_1}p_2^{u_2}\dots p_r^{u_r}$ with $u_i = \min\{k_i, j_i\}$, then it is clear that $g$ divides both $m$ and $n$. Let $d$ be a common divisor of $m$ and $n$, then any prime dividing $d$ msut divide $m$ and $n$, and hence, it must be equal to one of the $p_i$'s. Thus, $d = p_1^{e_1}...p_r^{e_r}$ for some non-negative integers $e_i$. For all $i$, $p_i^{e_i}$ divides both $p_i^{k_i}$ and $p_i^{j_i}$ so $e_i \leq k_i, j_i$. It follows that $e_i \leq u_i$, and hence, $p_i^{e_i}$ divides $p_i^{u_i}$. Hence, $d$ divides $g$. Therefore, $g$ is the greatest common divisor of $m$ and $n$.
    
    Similarly, let $l = p_1^{v_1}p_2^{v_2}\dots p_r^{v_r}$ with $v_i = \max\{k_i, j_i\}$, then it is clear that $l$ and $n$ both divide $l$. Let $c$ be a common multiple of $m$ and $n$, then any prime dividing $m$ and $n$ must divide $c$, hence, $c = p_1^{a_1}...p_r^{a_r}b$ for some non-negative integers $a_i$ and for an integer $b$ relatively prime to the $p_i$'s. For all $i$, $p_i^{k_i}$ and $p_i^{j_i}$ both divide $p_i^{a_i}$ so $k_i, j_i \leq a_i$. It follows that $v_i \leq a_i$, and hence, $p_i^{v_i}$ divides $p_i^{a_i}$. Hence, $l$ divides $c$. Therefore, $l$ is the least common multiple of $m$ and $n$. \\
\end{solution}

\begin{exercise}
    Use the result of Problem 1 to calculate $\gcd(12378, 3054)$ and $\lcm(12378, 3054)$.\\
\end{exercise}

\begin{solution}
    Since $12378 = 2^1\cdot 3^1 \cdot 2063^1$ and $3054 = 2^1 \cdot 3^1 \cdot 509^1$, then $\gcd(12378, 3054) = 2^1 \cdot 3^1 \cdot 509^0 \cdot 2063^0 = 6$ and $\lcm(12378, 3054) = 2^1 \cdot 3^1 \cdot 509^1 \cdot 2063^1 = 6300402$. \\
\end{solution}

\begin{exercise}
    Deduce from Problem 1 that $\gcd(m,n)\lcm(m,n) = mn$ for positive integers $m$ and $n$. \\
\end{exercise}

\begin{solution}
    Using the notation of Problem 1, we have that for all $i$, $u_i + v_i = \max\{k_i, j_i\} + \min\{k_i, j_i\} = k_i + j_i$. Hence, 
    \begin{align*}
        \gcd(m,n)\lcm(m,n) &= (p_1^{u_1}p_2^{u_2}\dots p_r^{u_r})(p_1^{v_1}p_2^{v_2}\dots p_r^{v_r}) \\
        &= p_1^{u_1 + v_1}p_2^{u_2 + v_2}\dots p_r^{u_r + v_r} \\
        &= p_1^{k_1 + j_1}p_2^{k_2 + j_2}\dots p_r^{k_r + j_r} \\
        &= (p_1^{k_1}p_2^{k_2}\dots p_r^{k_r})(p_1^{j_1}p_2^{j_2}\dots p_r^{j_r}) \\
        &= mn
    \end{align*}
    which proves the statement. \\
\end{solution}

\begin{exercise}
    In the notation of Problem 1, show that $\gcd(m,n) = 1$ if and only if $k_ij_i = 0$ for $i=1,2,...,r$. \\
\end{exercise}

\begin{solution}
    Using the notation of Problem 1, we have that $\gcd(m,n) = 1$ if and only if $\min\{k_i, j_i\} = 0$ for all $i$. But for all $i$, $\min\{k_i, j_i\} = 0$ if and only if $k_i$ or $j_i$ is zero. Equivalently, for all $i$, $\min\{k_i, j_i\} = 0$ if and only if $k_ij_i = 0$. Therefore, $\gcd(m,n) = 1$ if and only if $k_ij_i = 0$ for all $i$. \\
\end{solution}

\begin{exercise}
    \begin{enumerate}
        \item Verify that $\tau(n) = \tau(n+1) = \tau(n+2) = \tau(n+3)$ holds for $n = 3655$ and $n = 4503$.
        \item When $n = 14$, 206, and 957, show that $\sigma(n) = \sigma(n+1)$.
    \end{enumerate}
\end{exercise}

\begin{solution}
    \begin{enumerate}
        \item Since $3655 = 5 \cdot 17 \cdot 43$, $3656 = 2^3 \cdot 457$, $3657 = 3 \cdot 23 \cdot 53$, and $3658 = 2 \cdot 31 \cdot 59$, then
        \begin{align*}
            \tau(3655) &= (1+1)(1+1)(1+1) = 8, \\
            \tau(3656) &= (3+1)(1+1) = 8, \\
            \tau(3657) &= (1+1)(1+1)(1+1) = 8, \\
            \tau(3658) &= (1+1)(1+1)(1+1) = 8.
        \end{align*}
        Similarly, since $4503 = 3\cdot 19 \cdot 79$, $4504 = 2^3 \cdot 563$, $4505 = 5 \cdot 17 \cdot 53$, and $4506 = 2 \cdot 3 \cdot 751$, then
        \begin{align*}
            \tau(4503) &= (1+1)(1+1)(1+1) = 8, \\
            \tau(4504) &= (3+1)(1+1) = 8, \\
            \tau(4505) &= (1+1)(1+1)(1+1) = 8, \\
            \tau(4503) &= (1+1)(1+1)(1+1) = 8.
        \end{align*}
        \item Since $14 = 2 \cdot 7$, $15 = 3 \cdot 5$, $206 = 2 \cdot 103$, $207 = 3^2 \cdot 23$, $957 = 3 \cdot 11 \cdot 29$, and $958 = 2 \cdot 479$, then
        \begin{align*}
            \sigma(14) &= (1 + 2)(1 + 7) = 24 = (1 + 3)(1 + 5) = \sigma(15), \\
            \sigma(206) &= (1 + 2)(1 + 103) = 312 = (1 + 3 + 9)(1 + 23) = \sigma(207), \\
            \sigma(957) &= (1 + 3)(1 + 11)(1 + 29) = 1440 = (1 + 2)(1 + 479) = \sigma(958). \\
        \end{align*}
    \end{enumerate}
\end{solution}

\begin{exercise}
    For any integer $n \geq 1$, establish the inequality $\tau(n) \leq 2\sqrt{n}$. [\textit{Hint:} If $d \mid n$, then one of $d$ or $n / d$ is less than or equal to $\sqrt{n}$.] \\
\end{exercise}

\begin{solution}
    Let $D$ be the set of divisors of $n$, then we can pair each element $d$ of $D$ with another element $d'$ in $D$ such that $dd' = n$. In each pair, one of $d$ or $d'$ must be less than or equal to $\sqrt{n}$. Since there are at most $\sqrt{n}$ divisors of $n$ lesser or equal to $n$, then there are at most $\sqrt{n}$ pairs $(d,d')$ composing $D$. Since the number of elements in $D$, which is equal to $\tau(n)$, is twice (or twice minus 1 when $n$ is a square) the number of pairs $(d,d')$, then there are at most $2\sqrt{n}$ elements in $D$. Therefore, $\tau(n) \leq 2\sqrt{n}$. \\
\end{solution}

\begin{exercise}
    Prove that:
    \begin{enumerate}
        \item $\tau(n)$ is an odd integer if and only if $n$ is a perfect square;
        \item $\sigma(n)$ is an odd integer if and only if $n$ is perfect square or twice a perfect square. [\textit{Hint:} If $p$ is an odd prime, then $1 + p + p^2 + \dots + p^k$ is odd only when $k$ is even.]
    \end{enumerate}
\end{exercise}

\begin{solution}
    \begin{enumerate}
        \item Since $\tau(n) = (k_1 + 1)(k_2 + 1)\dots(k_r + 1)$, then $\tau(n)$ is odd if and only if $k_i = 2e_i$ for some integer $i$ for all $i$. Next, $k_i = 2e_i$ for all $i$ if and only if $n$ is a perfect square. Therefore, $\tau(n)$ is odd if and only if $n$ is a perfect square.
        \item If $\sigma(n)$ is odd, then $1 + p_i + \dots + p_i^{k_i}$ is odd for all $i$. If $p_i$ is not 2, then $k_i$ must be even. It follows that $k_i = 2e_i$ for some $e_i$ for all $i$. Thus, $n = 2^{k_0}a^2$ where $a$ is the product of the $p_i^{e_i}$ for all $i$ where $p_i$ is not 2, and where $k_0 \geq 0$. If $k_0$ is even, then $n$ is a square; if $k_0$ is odd, then $n = 2(2^{(k_0-1)/2}a)^2$ and so $n$ is twice a perfect square.
        
        Conversely, if $n$ is a perfect square or twice a perfect square, then $n = 2^{k_0}a^2$ where $a$ is odd. If we write $a = p_1^{e_1}\dots p_r^{e_r}$, then $n = 2^{k_0}p_1^{2e_1}\dots p_r^{2e_r}$ and so $\sigma(n) = (1 + 2 + \dots + 2^{k_0})\dots (1 + p_r + \dots + p_r^{2e_r})$. Clearly, the factor $(1 + 2 + \dots + 2^{k_0})$ is odd. Moreover, for all $i$, since $p_i$ is odd, then $(1 + p_i + \dots + p_i^{2e_i})$ is odd. Therefore, $\tau(n)$ is odd. \\
    \end{enumerate}
\end{solution}

\begin{exercise}
    Show that $\sum_{d \mid n}1/d = \sigma(n)/n$ for every positive integer $n$. \\
\end{exercise}

\begin{solution}
    First, notice that
    $$n\sum_{d \mid n}\frac{1}{d} = \sum_{d \mid n}\frac{n}{d} = \sum_{d \mid n}d = \sigma(n)$$
    because the terms $n / d$ are divisors of $n$, and $n/d$ ranges over all the divisors of $n$ once when $d$ ranges over all the divisors of $n$. It follows that $\sum_{d \mid n}1/d = \sigma(n)/n$. \\
\end{solution}

\begin{exercise}
    If $n$ is a square-free integer, prove that $\tau(n) = 2^r$ where $r$ is the number of prime divisors of $n$. \\
\end{exercise}

\begin{solution}
    If $n$ is a square-free integer, then $k_i$ must be equal to 1 for all $i$ because otherwise, $n$ would be divisible by $p_i^2$, a contradiction. Hence,
    $$\tau(n) = (k_1 + 1)(k_2 + 1) \cdot \cdot \cdot (k_r + 1) = 2 \cdot 2 \cdot \cdot \cdot 2 = 2^r.$$ \\
\end{solution}

\begin{exercise}
    Establish the assertions below:
    \begin{enumerate}
        \item If $n = p_1^{k_1}p_2^{k_2}\cdot \cdot \cdot p_r^{k_r}$ is the prime factorization of $n > 1$, then
        $$1 > \frac{n}{\sigma(n)} > \left(1 - \frac{1}{p_1}\right)\left(1 - \frac{1}{p_2}\right) \cdot \cdot \cdot \left(1 - \frac{1}{p_r}\right).$$
        \item For any positive integer $n$,
        $$\sigma(n!)/n! \geq 1 + 1/2 + 1/3 + \dots + 1/n.$$
        [\textit{Hint:} See Problem 8.]
        \item If $n > 1$ is a composite number, then $\sigma(n) > n + \sqrt{n}$. [\textit{Hint:} Let $d \mid n$, where $1 < d < n$, so $1 < n/d < n$. If $d \leq \sqrt{n}$, then $n/d \geq \sqrt{n}$.]
    \end{enumerate}
\end{exercise}

\begin{solution}
    \begin{enumerate}
        \item First, since 1 and $n$ are both distinct divisors of $n$, then they are part of the sum defining $\sigma(n)$, and hence, $n < n+1 \leq \sigma(n)$ which implies that $1 > n/\sigma(n)$. Moreover, using the product formula for $\sigma(n)$, we get that
        \begin{align*}
            \frac{n}{\sigma(n)} &= \frac{\prod_{i = 1}^r p_i^{k_i}}{\prod_{i = 1}^r \frac{p_i^{k_i + 1} - 1}{p_i - 1}} \\
            &= \prod_{i=1}^{r} \frac{p_i^{k_i}(p_i - 1)}{p_i^{k_i + 1} - 1} \\
            &> \prod_{i=1}^{r} \frac{p_i^{k_i}(p_i - 1)}{p_i^{k_i + 1}} \\
            &= \prod_{i=1}^{r} \frac{p_i - 1}{p_i} \\
            &= \left(1 - \frac{1}{p_1}\right)\left(1 - \frac{1}{p_2}\right) \cdot \cdot \cdot \left(1 - \frac{1}{p_r}\right)
        \end{align*}
        which completes the proof.
        \item Since $\sigma(n!)/n! = \sum_{d \mid n!}1/d$ and all integers between 1 and $n$ divide $n!$, then 
        $$\frac{\sigma(n!)}{n!} = \sum_{d \mid n!}\frac{1}{d} \geq \sum_{i=1}^{n}\frac{1}{i} = 1 + 1/2 + 1/3 + \dots + 1/n.$$
        \item If $n$ is composite, then there is an integer $d$ such that 1, $d$, and $n$ are divisors of $n$. We can assume that $d \geq \sqrt{n}$ because otherwise, we can replace $d$ by $n/d$, $n/d$ will be distinct from 1 and $n$, and will be greater than $\sqrt{n}$. Hence, $\sigma(n) \geq 1 + d + n > n + \sqrt{n}$.\\
    \end{enumerate}
\end{solution}

\begin{exercise}
    Given a positive integer $k > 1$, show that there are infinitely many integers $n$ for which $\tau(n) = k$, but at most finitely many $n$ with $\sigma(n) = k.$ [\textit{Hint:} Utilize Problem 10(a).]\\
\end{exercise}

\begin{solution}
    First, notice that for any prime $p$, the number $n = p^{k-1}$ will satisfy $\tau(n) = k$. Since there are infinitely many primes, then there are infinitely such numbers $n$. Next, suppose that $\sigma(n) = k$, then by Problem 10(a), we know that $1 > n\sigma(n)$ which implies that $n < k$. It follows that there must be finitely many $n$ such that $\sigma(n) = k$. \\
\end{solution}

\begin{exercise}
    \begin{enumerate}
        \item Find the form of all positive integers $n$ satisfying $\tau(n) = 10$. What is the smallest positive integer for which this is true?
        \item Show that there are no positive integers $n$ satisfying $\sigma(n) = 10$. [\textit{Hint:} Note that for $n > 1$, $\sigma(n) > n$.]
    \end{enumerate}
\end{exercise}

\begin{solution}
    \begin{enumerate}
        \item Let $n = p_1^{k_1}p_2^{k_2}\cdot \cdot \cdot p_r^{k_r}$, then $\tau(n) = 10$ implies that $(k_1 + 1) \cdot \cdot \cdot (k_r + 1) = 10$. Since $10$ can be written as a multiplication in only two ways ($1\cdot 10$ and $2\cdot 5$), then there are at most two $k_i$'s which are nonzero. Hence, either $k_1 + 1 = 1$ and $k_2 + 1 = 10$, or $k_2 + 1 = 2$ and $k_2 + 1 = 5$. Equivalently, either $n = p^9$, or $n = pq^4$ where $p$ and $q$ are distinct primes. Conversely, every number $n$ of the form $p^9$ or $pq^4$ with $p$ and $q$ distinct primes must satisfy $\tau(n) = 10$.
        
        To find the smallest integer $n$ such that $\tau(n) = 10$, consider the two following cases: if $n$ is of the form $p^9$, then its minimum value is $2^9 = 512$; if $n$ is of the form $pq^3$, then its minimum value is $3\cdot 2^3 = 24$. Combining these two observations gives us that $n = 24$.
        \item Let's prove that $\sigma(n) \neq 10$ for all $n \geq 1$. First, since $\sigma(n) > n$ for all $n > 1$, then $\sigma(n) > 10$ for all $n \geq 10$ and so it suffices to show that $\sigma(n) \neq 10$ for integers between 1 and 9. Since $\sigma(1) = 1$, $\sigma(2) = 3$, $\sigma(3) = 4$, $\sigma(4) = 7$, $\sigma(5) = 6$, $\sigma(6) = 12$, $\sigma(7) = 8$, $\sigma(8) = 15$, and $\sigma(9) = 13$, then it follows that $\sigma(n) \neq 10$ for all $n \geq 1$. \\
    \end{enumerate}
\end{solution}

\begin{exercise}
    Prove that there are infinitely many pairs of integers $m$ and $n$ with $\sigma(m^2) = \sigma(n^2)$. [\textit{Hint:} Choose $k$ such that $\gcd(k,10) = 1$ and consider the integers $m = 5k$, $n = 4k$.] \\
\end{exercise}

\begin{solution}
    First, notice that $\sigma(5^2) = \sigma(4^2)$ because $\sigma(5^2) = (5^3 - 1)/(5 -1) = 124/4 = 31$ and $\sigma(4^2) = \sigma(2^4) = 2^5 - 1 = 32 - 1 = 31$. Hence, if we take any integer $k$ relatively prime with $10$, then this integer squared will be relatively primes with $5^2$ and $4^2$. Thus, letting $m = 5k$ and $n = 4k$, we get 
    $$\sigma(m^2) = \sigma(5^2k^2) = \sigma(5^2)\sigma(k^2) = \sigma(4^2)\sigma(k^2) = \sigma(4^2k^2) = \sigma(n).$$
    Finally, since there are infinitely many $k$'s that are relatively prime with 10 (such as all numbers of the form $10t + 1$), then there are infinitely many such pairs of $m$ and $n$.\\
\end{solution}

\begin{exercise}
    For $k \geq 2$, show each of the following:
    \begin{enumerate}
        \item $n = 2^{k-1}$ satisfies the equation $\sigma(n) = 2n-1$;
        \item if $2^k - 1$ is prime, then $n = 2^{k-1}(2^k-1)$ satisfies the equation $\sigma(n) = 2n$;
        \item if $2^k-3$ is prime, then $n = 2^{k-1}(2^k - 3)$ satisfies the equation $\sigma(n) = 2n+2$.
    \end{enumerate}
    It is not known if there are any positive integers $n$ for which $\sigma(n) = 2n+1$.\\
\end{exercise}

\begin{solution}
    \begin{enumerate}
        \item Since 2 is a prime number, then using the product formula for $\sigma$, we get
        $$\sigma(n) = \frac{2^k - 1}{2-1} = 2\cdot 2^{k-1} - 1 = 2n-1.$$
        \item First, notice that $2^k - 1$ is prime relative to $2^{k-1}$ since it is odd. Hence, if we assume that $2^k - 1$ is prime, we get that
        $$\sigma(n) = \sigma(2^{k-1})\sigma(2^k - 1) = (2^k-1)([2^k - 1] + 1) = 2n.$$ 
        \item Again, since $2^k-3$ is odd, then it is prime relative to $2^{k-1}$ giving us
        \begin{align*}
            \sigma(n) &= \sigma(2^{k-1})\sigma(2^k - 3) \\
            &= (2^k-1)([2^k - 3] + 1) \\
            &= 2^k(2^k - 3) + (2^k - 1) - (2^k - 3) - 1 \\
            &= 2n + 1.\\
        \end{align*}
    \end{enumerate}
\end{solution}

\begin{exercise}
    If $n$ and $n+2$ are a pair of twin primes, establish that $\sigma(n+2) = \sigma(n)+ 2$; this also holds for $n = 434$ and 8575. \\
\end{exercise}

\begin{solution}
    First, recall that if $p$ is prime, then $\sigma(p) = p+1$. It follows that when $n$ and $n+2$ are a pair of twin primes, then
    $$\sigma(n+2) = n + 2 + 1 = \sigma(n) + 2.$$
    The converse doesn't hold because when $n = 434 = 2 \cdot 7 \cdot 31$, we have $n + 2 = 2^2\cdot 109$, and hence:
    $$\sigma(n+2) = 770 = 768 + 2 = \sigma(n) + 2.$$
    Similarly, when $n = 8575 = 5^2\cdot 7^3$, then $n + 2 = 8577 = 3^2 \cdot 953$ and hence:
    $$\sigma(n+2) = 12402 = 12400 + 2 = \sigma(n) + 2.$$\\
\end{solution}

\begin{exercise}
    \begin{enumerate}
        \item For any integer $n > 1$, prove that there exist integers $n_1$ and $n_2$ with $\tau(n_1) + \tau(n_2) = n$.
        \item Prove that Goldbach's Conjecture implies that for each even integer $2n$ there exist integers $n_1$ and $n_2$ with $\sigma(n_1) + \sigma(n_2) = 2n$.
    \end{enumerate}
\end{exercise}

\begin{solution}
    \begin{enumerate}
        \item Simply notice that for all $n > 1$:
        $$\tau(2^{n-2}) + \tau(1) = (n-1) + 1 = n.$$
        \item First, if $2n = 2$, then we can take $n_1 = n_2 = 1$. If $n > 2$, then by Goldbach's Conjecture there exist prime numbers $p$ and $q$ such that $p + q = 2n - 2$. It follows that
        $$\tau(p) + \tau(q) = (p+1) + (q+1) = (2n-2) + 2 = 2n.$$ \\
    \end{enumerate}
\end{solution}

\begin{exercise}
    For a fixed integer $k$, show that the function $f$ defined by $f(n) = n^k$ is multiplicative. \\
\end{exercise}

\begin{solution}
    Let $m$ and $n$ be two integers relatively prime to each other, then
    $$f(mn) = (mn)^k = m^kn^k = f(m)f(n).$$
    Therefore, $f$ is multiplicative. \\
\end{solution}

\begin{exercise}
    Let $f$ and $g$ be multiplicative functions such that $f(p^k) = g(p^k)$ for each prime $p$ and $k \geq 1$. Prove that $f = g$. \\
\end{exercise}

\begin{solution}
    Let $n$ be an arbitrary integer, then $n = p_1^{k_1}\cdot \cdot \cdot p_r^{k_r}$ for some primes and integers by the Fundamental Theorem of Arithmetic. Since the integers $p_1^{k_1}$, $p_2^{k_2}$, ..., $p_r^{k_r}$ are pairwise relatively prime, then by multiplicativity of $f$ and $g$:
    $$f(n) = f(p_1^{k_1})\cdot \cdot \cdot f(p_r^{k_r}) = g(p_1^{k_1})\cdot \cdot \cdot g(p_r^{k_r}) = g(n).$$
    Therefore, $f = g$. \\
\end{solution}

\begin{exercise}
    Prove that if $f$ and $g$ are multiplicative functions, then so is their product $fg$ and quotient $f/g$ (whenever the latter function is defined). \\
\end{exercise}

\begin{solution}
    Let $m$ and $n$ be two integers prime relative to each other, then
    $$f(mn)g(mn) = f(m)f(n)g(m)g(n) = [f(m)g(m)][f(n)g(n)]$$
    and
    $$\frac{f(mn)}{g(mn)} = \frac{f(m)f(n)}{g(m)g(n)} = \frac{f(m)}{g(m)}\cdot \frac{f(n)}{g(n)}.$$
    Therefore, the functions $fg$ and $f/g$ are multiplicative. \\
\end{solution}

\begin{exercise}
    Define the function $\rho$ by taking $\rho(1) = 1$ and $\rho(n) = 2^r$, if the prime factorization of $n > 1$ is $n = p_1^{k_1}p_2^{k_2} \cdot \cdot \cdot p_r^{k_r}$. For instance, $\rho(8) = 2$ and $\rho(10) = \rho(2^2) = 2^2$.
    \begin{enumerate}
        \item Deduce that $\rho$ is a multiplicative function.
        \item Find a formula for $F(n) = \sum_{d \mid n}\rho(d)$ in terms of the prime factorization of $n$. \\
    \end{enumerate}
\end{exercise}

\begin{solution}
    There is an error in this exercise. The function $\rho$ is not multiplicative because $\rho(30 \cdot 77) = 5$ while $\rho(30) = 3$ and $\rho(77) = 2$. Hence, $\rho(30 \cdot 77) \neq \rho(30)\rho(77)$ even though $\gcd(30, 77) = 1$. This exercise cannot be done. \\
\end{solution}

\begin{exercise}
    For any positive integer $n$, prove that $\sum_{d \mid n}\tau(n)^3 = (\sum_{d \mid n}\tau(n))^2$.

    \noindent [\textit{Hint:} Both sides of the equation in question are multiplicative functions of $n$, so that it suffices to consider the case $n = p^k$, where $p$ is prime.] \\
\end{exercise}

\begin{solution}
    Suppose without loss of generality that $n = p^k$ where $p$ is a prime and $k$ is an integer, then 
    \begin{align*}
        \sum_{d \ \mid \ p^k}\tau(d)^3 &= \sum_{i=0}^k\tau(p^i)^3 \\
        &= \sum_{i=0}^k(i+1)^3 \\
        &= \left(\sum_{i=0}^{k}(i+1)\right)^2 \\
        &= \left(\sum_{i=0}^{k}\tau(p^i)\right)^2 \\
        &= \left(\sum_{d \ \mid \ p^k}\tau(d)\right)^2.
    \end{align*}
    Therefore, for all integers $n$, we have $\sum_{d \mid n}\tau(n)^3 = (\sum_{d \mid n}\tau(n))^2$. \\
\end{solution}

\begin{exercise}
    Given $n \geq 0$, let $\sigma_s(n)$ denote the sum of the $s$th powers of the positive divisors of $n$, that is,
    $$\sigma_s(n) = \sum_{d \ \mid \ n}d^s.$$
    Verify the following:
    \begin{enumerate}
        \item $\sigma_0 = \tau$ and $\sigma_1 = \sigma$.
        \item $\sigma_s$ is a multiplicative function. [\textit{Hint:} The function $f$ defined by $f(n) = n^s$, is multiplicative.]
        \item If $n = p_1^{k_1}p_2^{k_2}\cdot \cdot \cdot p_r^{k_r}$ is the prime factorization of $n$, then
        $$\sigma_s(n) = \left(\frac{p_1^{s(k_1 + 1)} - 1}{p_1^s - 1}\right)\left(\frac{p_2^{s(k_2 + 1)} - 1}{p_2^s - 1}\right) \cdot \cdot \cdot \left(\frac{p_r^{s(k_r + 1)} - 1}{p_r^s - 1}\right).$$\\
    \end{enumerate}
\end{exercise}

\begin{solution}
    \begin{enumerate}
        \item It is clear that when $s = 0$, then the sum of the $s$th powers of the divisors of $n$ is the same as $\sum_{d \mid n}1$, the number of divisors of $n$. Thus, $\sigma_0(n) = \tau$. Similarly, when $s = 1$, then the sum of the first powers of the divisors of $n$ is simply equal to the sum of the divisors of $n$. Thus, $\sigma_1 = \sigma$.
        \item If we define the function $f(n) = n^s$, then we can write $\sigma_s(n)$ as $\sum_{d \mid n}f(d)$. Since $f$ is mutiplicative (Problem 17), then $\sigma_s$ is multiplicative. 
        \item It suffices that the formula holds for the prime powers since the function is multiplicative. Hence, let $n = p^k$, then
        $$\sigma_s(n) = \sum_{d \ \mid \ p^k}d^s = \sum_{i=0}^{k}p^{is} = \sum_{i=0}^{k}(p^s)^i = \frac{p^{s(i+1)} - 1}{p^s - 1}.$$ \\
    \end{enumerate}
\end{solution}

\begin{exercise}
    For any positive integer $n$, show that
    \begin{enumerate}
        \item $\sum_{d \mid n}\sigma(n) = \sum_{d \mid n}\frac{n}{d}\tau(d)$, and
        \item $\sum_{d \mid n}\frac{n}{d}\sigma(d) = \sum_{d \mid n}d\tau(d)$.
    \end{enumerate}
    [\textit{Hint:} Since the functions
    $$F(n) = \sum_{d \mid n}\sigma(d) \ \ \text{ and } \ \ G(n) = \sum_{d \mid n}(n/d)\tau(d)$$
    are both multiplicative, it suffices to prove that $F(p^k) = G(p^k)$ for any prime $p$.] \\
\end{exercise}

\begin{solution}
    \begin{enumerate}
        \item Without loss of generality, suppose that $n = p^k$, then
        \begin{align*}
            \sum_{d \ \mid \ n}\sigma(d) &= \sum_{i=0}^k\sigma(p^i) \\
            &= 1 + (1 + p) + (1 + p + p^2) + \dots + (1 + \dots + p^k) \\
            &= p^k \cdot 1 + \dots + p^2(k-1) + p^1k + p^0(k+1) \\
            &= \frac{p^k}{p^0}\tau(p^0) + \dots + \frac{p^k}{p^{k-1}}\tau(p^{k-1}) + \frac{p^k}{p^k}\tau(p^k) \\
            &= \sum_{i=0}^{k}\frac{n}{p^i}\tau(p^i) \\
            &= \sum_{d \ \mid \ n}\frac{n}{d}\tau(d).
        \end{align*}
        Hence, the equation holds for all integers $n$.
        \item Similarly, without loss of generality, suppose that $n = p^k$, then
        \begin{align*}
            \sum_{d \ \mid \ n}\frac{n}{d}\sigma(d) &= \sum_{i=0}^k p^{k-i}\sigma(p^i) \\
            &= p^k\cdot 1 + p^{k-1}(1 + p) + \dots + p^1(1  + \dots + p^{k-1}) + p^0(1 + \dots + p^k) \\
            &= p^k + (p^k + p^{k-1}) + \dots + (p^k + \dots + p) + (p^k + \dots + 1) \\
            &= p^0 \cdot 1 + p^1 \cdot 2 + p^2\cdot 3 + \dots + p^{k-1}k + p^k(k+1) \\
            &= \sum_{i=0}^kp^i \tau(p^i) \\
            &= \sum_{d \ \mid \ n}d \tau(d).
        \end{align*}
        Therefore, it holds for all integers $n$.\\
    \end{enumerate}
\end{solution}
\section{The Möbius Inversion Formula}

\begin{exercise}
    \begin{enumerate}
        \item For each positive integer $n$, show that 
        $$\mu(n)\mu(n+1)\mu(n+2)\mu(n+3) = 0.$$
        \item For any integer $n \geq 3$, show that $\sum_{k=1}^{n}\mu(k!) = 1$.
    \end{enumerate}
\end{exercise}

\begin{solution}
    \begin{enumerate}
        \item We already know that in a list of $m$ consecutive integers, one of them must be divisible by $m$, it follows that one of $n$, $n+1$, $n+2$, $n+3$ must be divisible by 4. Since 4 is a square, then it means that one of the four numbers is not square-free. Hence, $\mu(n+i) = 0$ for some $i=0,1,2,3$. Therefore, the multiplication of $\mu(n+i)$ where $i$ ranges from 0 to 3 must be zero.
        \item Notice that $4 \mid k!$ for all $k \geq 4$, so it follows that $\mu(k!) = 0$ for all such $k$. Hence, for all $n \geq 3$:
        $$\sum_{k=1}^{n}\mu(k!) = \mu(1) + \mu(2) + \mu(6) + 0 + \dots + 0 = 1 + (-1) + 1 = 1.$$ \\
    \end{enumerate}
\end{solution}

\begin{exercise}
    The \textit{Mangoldt function} $\Lambda$ is defined by 
    $$\Lambda(n) = \begin{cases}
        \log p, \quad \text{if } n=p^k, \text{ where } p \text{ is a prime and } k \geq 1 \\
        0, \quad \text{otherwise}
    \end{cases}$$
    Prove that $\Lambda(n) = \sum_{d \mid n}\mu(n/d)\log d = -\sum_{d \mid n}\mu(d)\log d$. [\textit{Hint:} First show that $\sum_{d\mid n}\Lambda(d) = \log(n)$ and then apply the Möbius Inversion Formula.] \\
\end{exercise}

\begin{solution}
    First, let's prove that $\sum_{d\mid n}\Lambda(d) = \log(n)$. Given an integer $n = p_1^{k_1} \cdot \cdot \cdot p_r^{k_r}$, we know that divisors of $n$ which are prime powers are precisely $p_i^{k}$ where $i = 1,..., r$ and $k = 1, ..., k_i$. It follows that
    \begin{align*}
        \sum_{d\mid n}\Lambda(d) &= \sum_{i=1}^{r}\sum_{k=1}^{k_i}\Lambda(p_i^k) \\
        &= \sum_{i=1}^{r}\sum_{k=1}^{k_i}\log p_i \\
        &= \sum_{i=1}^{r} \log (p_i^{k_i}) \\
        &= \log(p_1^{k_1} \cdot \cdot \cdot p_r^{k_r}) \\
        &= \log n.
    \end{align*}
    Therefore, if we apply the Möbius Inversion Formula, we get that
    $$\Lambda(n) = \sum_{d \mid n}\mu(n/d)\log d = \sum_{d \mid n}\mu(d)\log(n/d).$$
    Next, notice that
    $$\sum_{d \mid n}\mu(d)\log(n/d) = \log(n)\sum_{d \mid n}\mu(d) - \sum_{d \mid n}\mu(d)\log d.$$
    When $n = 1$, the first term on the right hand side is 0 since $\log 1 = 0$. Moreover, when $n > 1$, the first term on the right hand side is zero since $\sum_{d \mid n}\mu(d) = 0$. Thus, for all $n \geq 1$,
    $$\sum_{d \mid n}\mu(d)\log(n/d) = - \sum_{d \mid n}\mu(d)\log d$$
    and therefore:
    $$\Lambda(n) = \sum_{d \mid n}\mu(n/d)\log d = - \sum_{d \mid n}\mu(d)\log d.$$\\
\end{solution}

\begin{exercise}
    Let $n = p_1^{k_1}p_2^{k_2} \cdot \cdot \cdot p_r^{k_r}$ be the prime factorization of the integer $n > 1$. If $f$ is multiplicative, prove that
    $$\sum_{d \mid n}\mu(d)f(d) = (1 - f(p_1))(1 - f(p_2))\cdot \cdot \cdot (1 - f(p_r)).$$
    [\textit{Hint:} By Theorem 6-4, the function $F$ defined by $F(n) = \sum_{d \mid n}\mu(d)f(d)$ is multiplicative; hence, $F(n)$ is the product of the values $F(p_i^{k_i})$.]\\
\end{exercise}

\begin{solution}
    Since both functions $\mu$ and $f$ are multiplicative, then so is their product. It follows that the function $F(n) = \sum_{d \mid n}\mu(d)f(d)$ is multiplicative. Hence, it suffices to prove the desired formula for the case $n = p^k$. In that case,
    \begin{align*}
        \sum_{d \mid p^k}\mu(d)f(d) &= \sum_{i=0}^{k}\mu(p^i)f(p^i) \\
        &= \mu(1)f(1) + \mu(p)f(p) + \mu(p^2)f(p^2) + \dots + \mu(p^k)f(p^k) \\
        &= 1\cdot 1 + (-1)f(p) + 0 + \dots + 0 \\
        &= 1 - f(p)
    \end{align*}
    where we used the fact that $f(1) = 1$. Therefore, the formula holds for all $n > 1$. \\
\end{solution}

\begin{exercise}
    If the integer $n > 1$ has the prime factorization $n = p_1^{k_1}p_2^{k_2} \cdot \cdot \cdot p_r^{k_r}$, use Problem 3 to establish the following:
    \begin{enumerate}
        \item $\sum_{d \mid n}\mu(d)\tau(d) = (-1)^r$;
        \item $\sum_{d \mid n}\mu(d)\sigma(d) = (-1)^rp_1p_2\cdot \cdot \cdot p_r$;
        \item $\sum_{d \mid n}\mu(d)/d = (1 - 1/p_1)(1 - 1/p_2)\cdot \cdot \cdot (1 - 1/p_r)$;
        \item $\sum_{d \mid n}d\mu(d) = (1 - p_1)(1 - p_2)\cdot \cdot \cdot (1 - p_r)$.
    \end{enumerate}
\end{exercise}

\begin{solution}
    \begin{enumerate}
        \item $\sum_{d \mid n}\mu(d)\tau(d) = (1 - \tau(p_1))\cdot \cdot \cdot (1 - \tau(p_r)) = (1 - 2) \cdot \cdot \cdot (1 - 2) = (-1)^r.$
        \item $\sum_{d \mid n}\mu(d)\sigma(d) = (1 - \sigma(p_1))\cdot \cdot \cdot (1 - \sigma(p_r)) = (-p_1)\cdot \cdot \cdot (-p_r) = (-1)^rp_1p_2\cdot \cdot \cdot p_r$.
        \item If we let $f(n) = 1/n$, then $f$ is multiplicative which lets us write
        $$\sum_{d \mid n}\mu(d)/d = \sum_{d \mid n}\mu(d)f(d) = (1 - f(p_1))\cdot \cdot \cdot (1 - f(p_r)) = (1 - 1/p_1)\cdot \cdot \cdot (1 - 1/p_r).$$
        \item If we let $f(n) = n$, then $f$ is multiplicative which lets us write
        $$\sum_{d \mid n}d\mu(d) = \sum_{d \mid n}\mu(d)f(d) = (1 - f(p_1))\cdot \cdot \cdot (1 - f(p_r)) = (1 - p_1)\cdot \cdot \cdot (1 - p_r).$$\\
    \end{enumerate}
\end{solution}

\begin{exercise}
    Let $S(n)$ denote the number of square-free divisors of $n$. Establish that
    $$S(n) = \sum_{d \mid n}|\mu(d)| = 2^r$$
    where $r$ is the number of distinct prime divisors of $n$. [\textit{Hint:} $S$ is a multiplicative function.] \\
\end{exercise}

\begin{solution}
    First, notice that for $m$ and $n$ relatively prime, we have $|\mu(mn)| = |\mu(m)\mu(n)| = |\mu(m)|\cdot |\mu(n)|$. Hence, the function $n \mapsto |\mu(n)|$ is multiplicative, and therefore, the function $S$ is also multiplicative. Since
    $$S(p^k) = \sum_{d \mid p^k}|\mu(d)| = |\mu(1)| + |\mu(p)| + |\mu(p^2)| + \dots + |\mu(p^k)| = 1 + |-1| + 0 + \dots + 0 = 2,$$
    then by multiplicativity:
    $$S(n) = S(p_1^{k_1})S(p_2^{k_2})\cdot \cdot \cdot S(p_r^{k_r}) = 2\cdot 2 \cdot \cdot \cdot 2 = 2^r$$
    where $n = p_1^{k_1}p_2^{k_2} \cdot \cdot \cdot p_r^{k_r}$. \\
\end{solution}

\begin{exercise}
    Find formulas for $\sum_{d \mid n}\mu^2(d)/\tau(d)$ and $\sum_{d \mid n}\mu^2(d)/\sigma(d)$ in terms of the prime factorization of $n$.\\
\end{exercise}

\begin{solution}
    If we let $n = p_1^{k_1}\cdot \cdot \cdot p_r^{k_r}$, then by Exercise 3, we have
    \begin{align*}
        \sum_{d \ \mid \ n}\frac{\mu^2(d)}{\tau(d)} &= \left(1 - \frac{\mu(p_1)}{\tau(p_1)}\right) \cdot \cdot \cdot \left(1 - \frac{\mu(p_r)}{\tau(p_r)}\right) \\
        &= \left(1 + \frac{1}{2}\right) \cdot \cdot \cdot \left(1 + \frac{1}{2}\right) \\
        &= \left(\frac{3}{2}\right)^r.
    \end{align*} 
    Similarly:
    \begin{align*}
        \sum_{d \ \mid \ n}\frac{\mu^2(d)}{\sigma(d)} &= \left(1 - \frac{\mu(p_1)}{\sigma(p_1)}\right) \cdot \cdot \cdot \left(1 - \frac{\mu(p_r)}{\sigma(p_r)}\right) \\
        &= \left(1 + \frac{1}{p_1 + 1}\right) \cdot \cdot \cdot \left(1 + \frac{1}{p_r + 1}\right) \\
        &= \frac{p_1 + 2}{p_1 + 1} \cdot \cdot \cdot \frac{p_r + 2}{p_r + 1}. \\
    \end{align*} 
\end{solution}

\begin{exercise}
    The \textit{Liouville $\lambda$-function} is defined by $\lambda(1) = 1$ and $\lambda(n) = (-1)^{k_1 + k_2 + \dots + k_r}$, if the prime factorization of $n> 1$ is $n= p_1^{k_1}p_2^{k_2} \cdot \cdot \cdot p_r^{k_r}$. For instance, $\lambda(360) = \lambda(2^3 \cdot 3^2 \cdot 5) = (-1)^{3 + 2 + 1} = (-1)^6 = 1$.
    \begin{enumerate}
        \item Prove that $\lambda$ is a multiplicative function.
        \item Given a positive integer $n$, verify that 
        $$\sum_{d \ \mid \ n}\lambda(n) = \begin{cases}
            1 \ \text{ if } n=m^2 \text{ for some integer } m \\
            0 \ \text{ otherwise}
        \end{cases}$$
    \end{enumerate}
\end{exercise}

\begin{solution}
    \begin{enumerate}
        \item Let $m$ and $n$ be integers relatively prime to each other, then the prime factorizations are $m = p_1^{k_1}\cdot \cdot \cdot p_r^{k_r}$ and $n = q_1^{t_1} \cdot \cdot \cdot q_s^{t_s}$ where the $p_i$'s are distinct from the $q_i$'s. Hence:
        $$\lambda(mn) = (-1)^{k_1 + \dots + k_r + t_1 + \dots + t_s} = (-1)^{k_1 + \dots + k_r}(-1)^{t_1 + \dots + t_s} = \lambda(m)\lambda(n).$$
        Therefore, the function is multiplicative.
        \item Since the function $F(n) = \sum_{d \mid n}\lambda(n)$ is multiplicative, let's first determine its value at powers of primes.
        $$\sum_{d \ \mid \ p^k}\lambda(d) = \sum_{i=0}^{k}\lambda(p^i) = \sum_{i=0}^{k}(-1)^i = \frac{1 + (-1)^k}{2} = \begin{cases}
            1 \ \text{ if } k \text{ is even } \\
            0 \ \text{ if } k \text{ is odd }
        \end{cases} .$$
        Therefore, for any integer $n = p_1^{k_1} \cdot \cdot \cdot p_r^{k_r}$, the sum $\sum_{d \mid n}\lambda(n)$ will be 1 if and only if all the $k_i$'s are even; otherwise it will be zero. But we know that $n$ is a square if and only if all the $k_i$'s are even; thus:
        $$\sum_{d \ \mid \ n}\lambda(n) = \begin{cases}
            1 \ \text{ if } n=m^2 \text{ for some integer } m \\
            0 \ \text{ otherwise}
        \end{cases}.$$ \\
    \end{enumerate}
\end{solution}

\begin{exercise}
    If the integer $n> 1$ has the prime factorization $n = p_1^{k_1}p_2^{k_2}\cdot \cdot \cdot p_r^{k_r}$, establish that $\sum_{d \ \mid \ n}\mu(d)\lambda(d) = 2^r$. \\
\end{exercise}

\begin{solution}
    Since $\lambda$ is a multiplicative function, we can apply Exercise 3 to get
    \begin{align*}
        \sum_{d \ \mid \ n}\mu(d)\lambda(d) &= (1 - \lambda(p_1))(1 - \lambda(p_2)) \cdot \cdot \cdot (1 - \lambda(p_r)) \\
        &= (1 - (-1))(1 - (-1)) \cdot \cdot \cdot (1 - (-1)) \\
        &= 2 \cdot 2 \cdot \cdot \cdot 2 \\
        &= 2^r.
    \end{align*} 
\end{solution}
\section{The Greatest Integer Function}

\begin{exercise}
    Given integers $a$ and $b > 0$, show that there exists a unique integer $r$ with $0 \leq r < b$ satisfying $a = [a/b]b + r$. \\
\end{exercise}

\begin{solution}
    First, we have that $a = qb + r$ for some integers $q$ and $r$. It follows that $a/b = q + r/b$. Since $0 \leq r < b$, then $0 \leq r/b < 1$. It follows that $[a/b] = q$, and hence, that $a = [a/b]b + r$. Note that this $r$ is unique because otherwise, it would contradict the uniqueness of the pair $q,r$.\\
\end{solution}

\begin{exercise}
    Let $x$ and $y$ be real numbers. Prove that the greatest integer function satisfies the following properties:
    \begin{enumerate}
        \item $[x+n] = [x] + n$ for any integer $n$.
        \item $[x] + [-x] = 0$ or $-1$, according as $x$ is an integer or not. [\textit{Hint:} Write $x = [x] + \theta$, with $0 \leq \theta < 1$, so that $-x = -[x] - 1 + (1 - \theta)$.]
        \item $[x] + [y] \leq [x+y]$ and, when $x$ and $y$ are positive, $[x][y] \leq [xy]$.
        \item $[x/n] = [[x]/n]$ for any positive integer $n$. [\textit{Hint:} Let $x/n = [x/n] + \theta$, where $0 \leq \theta < 1$; then $[x] = n[x/n] + [n\theta]$.]
        \item $[nm/k] \geq n[m/k]$ for positive integers $n,m,k$.
        \item $[x] + [y] + [x + y] \leq [2x] + [2y]$. [\textit{Hint:} Let $x = [x] + \theta$, $0 \leq \theta < 1$, and $y = [y] + \theta'$, $0 \leq \theta' < 1$. Consider cases in which neither, one, or both of $\theta$ and $\theta'$ are greater than or equal to $\frac{1}{2}$.]
    \end{enumerate}
\end{exercise}

\begin{solution}
    \begin{enumerate}
        \item Since we can write $x = [x] + \theta$ for some $0 \leq \theta < 1$, then $x+ n = k + \theta$ where $k = [x] + n$ is an integer. It follows that $[x + n] = k = [x] + n$.
        \item If $x$ is an integer, then clearly $[x] + [-x] = x + (-x) = 0$. If $x$ is not an integer, then we can write $x = [x] + \theta$ where $0 < \theta < 1$. Hence, $-x = -[x]  - 1 + (1 - \theta)$ which implies that $[-x] = - [x] - 1$, and thus, that $[x] + [-x] = [x] - [x] - 1 = -1$.
        \item Since $[x] \leq x$ and $[y] \leq y$, then $[x] + [y] \leq x + y$. Since $[x] + [y]$ is an integer, then it must be less than or equal to the greatest integer smaller than $x + y$, in other words, $[x] + [y] \leq [x + y]$.
        
        Similarly, if $x$ and $y$ are positive, then $[x][y] \leq xy$ which implies that $[x][y] \leq [xy]$ using the same argument as for the sum.
        \item Since $x/n = [x/n] + \theta$ for some $0 \leq \theta < 1$, then $x = n[x/n] + n\theta$, and hence, $[x] = [n[x/n] + n\theta] = n[x/n] + [n\theta]$ by part (a). Dividing both sides by $n$ gives us $[x]/n = [x/n] + [n\theta]/n$. Since $0 \leq [n\theta] \leq n \theta < n$, then $0 \leq [n\theta]/n < 1$, and hence, $[[x]/n] = [x/n]$.
        \item Since $[m/k] \leq m/k$, then $n[m/k] \leq nm/k$. Since $n[m/k]$ is an integer less than $nm/k$, then it must be less than or equal to the greatest integer smaller than $nm/k$. Thus, $n[m/k] \leq [nm/k]$.
        \item Let $x = [x] + \theta_1$ and $y = [y] + \theta_2$ where $0 \leq \theta_i < 1$. When $\theta_1 + \theta_2 < 1$, we have that $x + y = [x] + [y] + (\theta_1 + \theta_2)$ where $0 \leq \theta_1 + \theta_2 < 1$, which implies that $[x + y] = [x] + [y]$. When $1 \leq \theta_1 + \theta_2 < 2$, we have that $x + y = [x] + [y] + 1 + \theta$ where $0 \leq \theta_1 + \theta_2 - 1 < 1$, and hence, $[x + y] = [x] + [y] + 1$. In both cases, we have that $[x + y] \leq [x] + [y]$. \\
    \end{enumerate}
\end{solution}

\begin{exercise}
    Find the highest power of $5$ dividing $1000!$ and the highest power of 7 dividing $2000!$. \\
\end{exercise}

\begin{solution}
    Using Theorem 6-9, we have that the highest power of 5 dividing $1000!$ is
    $$\left[\frac{1000}{5}\right] + \left[\frac{1000}{5^2}\right] + \left[\frac{1000}{5^3}\right] + \left[\frac{1000}{5^4}\right]= 200 + 40 + 8 + 1 = 249.$$
    Similarly, the highest power of 7 dividing $2000!$ is
    $$\left[\frac{2000}{7}\right] + \left[\frac{2000}{7^2}\right] + \left[\frac{2000}{7^3}\right] = 285 + 40 + 5 = 330.$$ \\
\end{solution}

\begin{exercise}
    For an integer $n \geq 0$, show that $[n/2] - [-n/2] = n.$ \\
\end{exercise}

\begin{solution}
    If $n$ is even, then $n = 2k$ which implies that
    $$[n/2] - [-n/2] = k - (-k) = 2k = n.$$
    If $n = 2k+1$, then $[n/2] = [k + 1/2] = k$ and $[-n/2] = [-k - 1/2] = -k - 1$. Hence, $[n/2] - [n/2] = k - (-k - 1) = 2k + 1 = n$. Therefore, the formula holds for all $n$.\\
\end{solution}

\begin{exercise}
    \begin{enumerate}
        \item Verify that $1000!$ terminates in $249$ zeros.
        \item For what values of $n$ does $n!$ terminate in 37 zeros? \\
    \end{enumerate}
\end{exercise}

\begin{solution}
    \begin{enumerate}
        \item We already showed that the highest power of 5 dividing $1000!$ is 249. Since $[x] \leq [y]$ when $x \leq y$, then it follows from Theorem 6-9 that the highest power of 2 dividing $1000!$ is higher than 249. It follows that the highest power of 10 dividing $1000!$ is 249, and equivalently, $1000!$ ends with 249 zeros.
        \item Using an argument similar to the one used in the previous part, it suffices to find the values of $n$ such that $\sum_{k=1}^{\infty}[n/5^k] = 37$. From this formula, we get that $[n/5] \leq 37$, and hence, that $n \leq 189$. Since $n = 5^3$ gives $\sum_{k=1}^{\infty}[5^3/5^k] = 31$ which implies that $126 \leq n \geq 189$. From this range, we get that
        $$\sum_{k=1}^{\infty}\left[\frac{n}{5^k}\right] = \left[\frac{n}{5}\right] + \left[\frac{n}{5^2}\right] + \left[\frac{n}{5^3}\right] = \left[\frac{n}{5}\right] + \left[\frac{n}{5^2}\right] + 1.$$
        Moreover, the fact that $126 \leq n \leq 189$ implies that $5 \leq [n/5^2] \leq 7$. It follows that $[n/5] = 29$, $30,$ or 31, which implies that $145 \leq n \leq 160$. Checking the values of $\sum_{k=1}^{\infty}[n/5^k]$ in this range gives us that the possible values of $n$ are $n = 150, 151, 152, 153, 154$. \\
    \end{enumerate}
\end{solution}

\begin{exercise}
    If $n \geq 1$ and $p$ is a prime, prove that
    \begin{enumerate}
        \item $(2n)/(n!)^2$ is an even integer. [\textit{Hint:} Use induction on $n$.]
        \item The exponent of the highest power of $p$ which divides $(2n)!/(n!)^2$ is
        $$\sum_{k=1}^{\infty}([2n/p^k] - 2[n/p^k]).$$
        \item In the prime factorization of $(2n)!/(n!)^2$ the exponent of any prime $p$ such that $n < p < 2n$ is equal to 1.
    \end{enumerate}
\end{exercise}

\begin{solution}
    \begin{enumerate}
        \item Simply notice that
        $$\frac{(2n)!}{(n!)^2} = \frac{2n}{n} \cdot \frac{(2n-1)!}{(n-1)!\cdot n!} = 2\binom{2n-1}{n}.$$
        Since $\binom{2n-1}{n}$ is an integer, then it follows that $(2n)/(n!)^2$ is an even integer.
        \item The highest exponent of $p$ dividing $(2n)/(n!)^2$ is the highest exponent of $p$ dividing $(2n)!$ minus the highest exponent of $p$ dividing $(n!)^2$. Moreover, the highest exponent of $p$ dividing $(n!)^2$ is simply twice the highest exponent of $p$ dividing $n!$. Putting everything together and using Theorem 6-9 gives us that the highest exponent of $p$ dividing $(2n)/(n!)^2$ is 
        $$\sum_{k=1}^{\infty}([2n/p^k] - 2[n/p^k]).$$
        \item When $n < p < 2n$, we get that $[2n/p] = 1$ and $[n/p] = 0$; and that $[2n/p^k] = [n/p^k] = 0$ for all $k \geq 2$. Hence, using part (b), we get that in the prime factorization of $(2n)!/(n!)^2$ the exponent of $p$ is
        $$\sum_{k=1}^{\infty}([2n/p^k] - 2[n/p^k]) = (1 - 2\cdot 0) + 0 + 0 + \dots = 1.$$  \\
    \end{enumerate}
\end{solution}

\begin{exercise}
    Let the positive integer $n$ be written in terms of powers of the prime $p$ so that $n = a_kp^k + \dots + a_2p^2 + a_1p + a_0$ where $0 \leq a_i < p$. Show that the exponent of the highest power of $p$ appearing in the prime factorization of $n!$ is 
    $$\frac{n - (a_k + \dots + a_2 + a_1 + a_0)}{p-1}.$$
\end{exercise}

\begin{solution}
    Using Theorem 6-9, we have that the highest power of $p$ dividing $n!$ is
    \begin{align*}
        \sum_{i=1}^{\infty}\left[\frac{a_kp^k + \dots + a_2p^2 + a_1p + a_0}{p^i}\right] &= \sum_{i=1}^{k}(a_kp^{k-i} + \dots + a_i) \\
        &= \sum_{i=1}^{k}\sum_{j=i}^{k}a_j p^{j-i} \\
        &= \sum_{j=1}^{k}\sum_{i=1}^{j}a_jp^{j-i} \\
        &= \sum_{j=1}^{k}a_jp^j\sum_{i=1}^{j}p^{-i} \\
        &= \sum_{j=1}^{k}a_jp^j \cdot \frac{1}{p}\cdot \frac{1 - p^{-j}}{1 - p^{-1}} \\
        &= \frac{1}{p-1} \sum_{j=1}^{k}a_jp^j (1 - p^{-j}) \\
        &= \frac{\sum_{j=1}^{k}a_jp^j - \sum_{j=1}^{k}a_j}{p-1} \\
        &= \frac{(n- a_0) - (a_k + \dots + a_2 + a_1)}{p-1} \\
        &= \frac{n - (a_k + \dots + a_2 + a_1 + a_0)}{p-1}.\\
    \end{align*}
\end{solution}

\begin{exercise}
    \begin{enumerate}
        \item Using Problem 7, show that the exponent of the highest power of $p$ dividing $(p^k - 1)!$ is $[p^k - (p-1)k - 1]/(p-1)$. [\textit{Hint:} Recall the identity
        $$p^k - 1 = (p-1)(p^{k-1} + \dots + p^2 + p + 1).]$$
        \item Determine the highest power of 3 dividing $80!$ and the highest power of 7 dividing $2400!$. [\textit{Hint:} $2400 = 7^4 - 1$.]
    \end{enumerate}
\end{exercise}

\begin{solution}
    \begin{enumerate}
        \item Since $p^k - 1 = (p-1)(p^{k-1} + \dots + p^2 + p + 1)$, then it means that we can write $p^k - 1$ as $a_{k-1}p^{k-1} + \dots + a_1 p + a_0$ where $a_i = p-1$ for all $i$. Hence, using Problem 7, we get that the highest power of $p$ dividing $(p^k - 1)!$ is
        $$\frac{(p^k - 1) - (a_{k-1} + \dots + a_1 + a_0)}{p-1} = \frac{p^k  - (p-1)k - 1}{p-1}.$$
        \item Since $80 = 3^4 - 1$, then part (a) tells us that the exponent of the highest power of $3$ dividing 80 is $[80 - 2\cdot 4]/2 = 40 - 4 = 36$. Similarly, the exponent of the highest power of $7$ dividing $2400 = 7^4 - 1$ is $[2400 - 6\cdot 4]/6 = 400 - 4 = 396$. \\
    \end{enumerate}
\end{solution}

\begin{exercise}
    Find an integer $n \geq 1$ such that the highest power of 5 contained in $n!$ is 100. [\textit{Hint:} Since the sum of the coefficients of the powers of 5 needed to express $n$ in the base 5 is at least 1, begin by considering the equation $(n-1)/4 = 100.$] \\
\end{exercise}

\begin{solution}
    Let $a_0, a_1, ..., a_k$ be the coefficients of the base 5 expression of $n$, then Problem 7 tells us that $n$ must satisfy the equation $n - (a_k + \dots + a_1 + a_0) = 400$. If the sum the $a_i$'s is 1, then $n = 401$ which is a contradiction since 401 is not a power of 5. If the sum of the $a_i$'s is 2, then $n = 402 = 3\cdot 5^2 + 1\cdot 5^2 + 0\cdot 5 + 2$ which doesn't satisfy the condition that the sum of the coefficients is 2. Similarly, the sum of the coefficients cannot be 3 or 4 because $403 = 3\cdot 5^2 + 1\cdot 5^2 + 0\cdot 5 + 3$ and $404 = 3\cdot 5^2 + 1\cdot 5^2 + 0\cdot 5 + 4$. Next, if the sum of the coefficients is $5$, then $n = 405 = 3\cdot 5^2 + 1\cdot 5^2 + 1\cdot 5 + 0$ which is consistent with the fact that the sum of the coefficients is 5. Indeed, by Problem 7, we have that the highest power of 5 dividing $405! = (3\cdot 5^2 + 1\cdot 5^2 + 1\cdot 5)!$ is
    $$\frac{405 - (3 + 1 + 1)}{4} = \frac{400}{4} = 100.$$
    Therefore, it holds for $n = 405$. \\
\end{solution}

\begin{exercise}
    Given a positive integer $N$, show that
    \begin{enumerate}
        \item $\sum_{n=1}^{N}\mu(n)[N/n] = 1$;
        \item $|\sum_{n=1}^{N}\mu(n) /n| \leq 1$.
    \end{enumerate}
\end{exercise}

\begin{solution}
    \begin{enumerate}
        \item Let $F(n) = \sum_{d \ \mid \ n}\mu(d)$, then we know that $F(1) = 1$ and $F(n) = 0$ for all $n \geq 2$ (Theorem 6-6). Thus, by Theorem 6-11:
        $$\sum_{n=1}^{N}\mu(n)\left[\frac{N}{n}\right] = \sum_{n=1}^{N}F(n) = 1.$$
        \item For all $n \leq N$, define $\theta_n = N/n - [N/n]$, then $0 \leq \theta_n < 1$, and $\theta_1 = 0$. Hence, using part (a):
        \begin{align*}
            \left|\sum_{n=1}^{N}\mu(n)\frac{N}{n}\right| &= \left|\sum_{n=1}^{N}\mu(n)\left(\left[\frac{N}{n}\right] + \theta_n\right)\right|  \\
            &=  \left|\sum_{n=1}^{N}\mu(n)\left[\frac{N}{n}\right] + \sum_{n=1}^{N}\mu(n)\theta_n\right|  \\
            &= \left|1 + \sum_{n=2}^{N}\mu(n)\theta_n\right|  \\
            &\leq 1 + \sum_{n=2}^{N}|\mu(n)\theta_n| \\
            &\leq 1 +  \sum_{n=2}^{N}1 \\
            &= N.
        \end{align*}
        Thus, we have shown that
        $$\left|\sum_{n=1}^{N}\mu(n)\frac{N}{n}\right| \leq N.$$
        Dividing both sides by $N$ gives us the desired inequality. \\
    \end{enumerate}
\end{solution}

\begin{exercise}
    Illustrate Problem 10 in the case $N = 6$.\\
\end{exercise}

\begin{solution}
    \begin{enumerate}
        \item \begin{align*}
            & \ \ \sum_{n=1}^{6}\mu(n)\left[\frac{6}{n}\right] \\
            &= \mu(1)\left[\frac{6}{1}\right] + \mu(2)\left[\frac{6}{2}\right] + \mu(3)\left[\frac{6}{3}\right] + \mu(4)\left[\frac{6}{4}\right] + \mu(5)\left[\frac{6}{5}\right] + \mu(6)\left[\frac{6}{6}\right] \\
            &= 1\cdot 6 + (-1)\cdot 3 + (-1)\cdot 2 + 0 \cdot 1 + (-1) \cdot 1 + 1\cdot 1\\
            &= 6 - 3 - 2 - 1 + 1 \\
            &= 1.
        \end{align*}
        \item \begin{align*}
            \sum_{n=1}^{6}\frac{\mu(n)}{n} &= \frac{\mu(1)}{1} + \frac{\mu(2)}{2} + \frac{\mu(3)}{3} + \frac{\mu(4)}{4} + \frac{\mu(5)}{5} + \frac{\mu(6)}{6} \\
            &= 1 - \frac{1}{2} - \frac{1}{3} - \frac{1}{5} + \frac{1}{6} \\
            &= \frac{2}{15}. \\
        \end{align*}
    \end{enumerate}
\end{solution}

\begin{exercise}
    Verify that the formula
    $$\sum_{n=1}^{N}\lambda(n)[N/n] = [\sqrt{N}]$$
    holds for any positive integer $N$. [\textit{Hint:} Apply Theorem 6-11 to the multiplicative function $F(n) = \sum_{d \ \mid \ n}\lambda(d)$, noting that there are $[\sqrt{N}]$ perfect squares not exceeding $n$.]\\
\end{exercise}

\begin{solution}
    Recall that if we define $F(n) = \sum_{d \ \mid \ n}\lambda(d)$, then $F(n)$ is equal to 1 if and only if $n$ is a square, otherwise $F(n) = 0$. Thus, by Theorem 6-11:
    $$\sum_{n=1}^{N}\lambda(n)[N/n] = \sum_{n=1}^{N}F(n) = [\sqrt{N}]$$
    using the fact that there are $[\sqrt{N}]$ squares between 1 and $N$. \\
\end{solution}

\begin{exercise}
    If $N$ is a positive integer, establish that
    \begin{enumerate}
        \item $N = \sum_{n=1}^{2N}\tau(n) - \sum_{n=1}^{N}[2N/n]$;
        \item $\tau(N) = \sum_{n=1}^{N}([N/n] - [(N-1)/n])$.
    \end{enumerate}
\end{exercise}

\begin{solution}
    \begin{enumerate}
        \item Using Corollary 1, we have that
        $$\sum_{n=1}^{2N}\tau(n) - \sum_{n=1}^{N}\left[\frac{2N}{n}\right] = \sum_{n=1}^{2N}\left[\frac{2N}{n}\right] - \sum_{n=1}^{N}\left[\frac{2N}{n}\right] = \sum_{n=N+1}^{2N}\left[\frac{2N}{n}\right].$$
        Next, we have that for all $N+1 \leq n \leq 2N$, $[2N/n] = 1$, and hence:
        $$\sum_{n=1}^{2N}\tau(n) - \sum_{n=1}^{N}\left[\frac{2N}{n}\right] = \sum_{n=N+1}^{2N}\left[\frac{2N}{n}\right] = \sum_{n=N+1}^{2N}1 = N.$$
        \item Using Corollary 1:
        \begin{align*}
            \sum_{n=1}^{N}\left(\left[\frac{N}{n}\right] - \left[\frac{N_1}{n}\right]\right) &= \sum_{n=1}^{N}\left[\frac{N}{n}\right] - \sum_{n=1}^{N}\left[\frac{N-1}{n}\right] \\
            &= \sum_{n=1}^{N}\tau(n) - \sum_{n=1}^{N-1}\left[\frac{N-1}{n}\right] - \left[\frac{N - 1}{N}\right] \\
            &= \sum_{n=1}^{N}\tau(n) - \sum_{n=1}^{N-1}\tau(n) - 0\\
            &= \tau(N).
        \end{align*}
    \end{enumerate}
\end{solution}
\fi

% CHAPTER 7
\chapter{Euler's Generalization of Fermat's Theorem}

\section{Leonhard Euler}

There are no exercices in this section.

\section{Euler's Phi-Function}

\begin{exercise}
    Calculate $\phi(1001)$, $\phi(5040)$, and $\phi(36,000)$. \\
\end{exercise}

\begin{solution}
    Since $1001 = 7 \cdot 11 \cdot 13$, then 
    $$\phi(1001) = \phi(7)\phi(11)\phi(13) = 6\cdot 10 \cdot 12 = 720.$$
    Since $5040 = 2^4\cdot 3^2 \cdot 5 \cdot 7$, then
    $$\phi(5040) = 2^3(2 - 1) \cdot 3(3-1) \cdot (5-1) \cdot (7-1) = 1152.$$
    Since $36,000 = 2^5 \cdot 3^2 \cdot 5^3$, then
    $$\phi(36,000) = 2^4(2-1)\cdot 3(3-1) \cdot 5^2(5-1) = 9600.$$ \\
\end{solution}

\begin{exercise}
    Verify that the equality $\phi(n) = \phi(n+1) = \phi(n+2)$ holds when $n = 5186$.\\
\end{exercise}

\begin{solution}
    When $n = 5186$, we have $n = 2 \cdot 2593$, $n + 1 = 3 \cdot 7 \cdot 13 \cdot 19$, and $n + 2 = 2^2 \cdot 1297$. Hence:
    \begin{align*}
        \phi(n) &= (2-1) (2593 - 1) = 2592\\
        \phi(n+1) &= (3-1)(7-1)(13-1)(19-1) = 2592 \\
        \phi(n+2) &= 2(2-1)(1297-1) = 2592. \\
    \end{align*}
\end{solution}

\begin{exercise}
    Show that the integers $m = 3^k \cdot 568$ and $n = 3^k \cdot 638$, where $k \geq 0$, satisfy simultaneously
    $$\tau(m) = \tau(n), \ \sigma(m) = \sigma(n), \ \phi(m) = \phi(n).$$\\
\end{exercise}

\begin{solution}
    Since $m = 2^3 \cdot 3^k \cdot 71$ and $n = 2 \cdot 3^k \cdot 11 \cdot 29$, then
    \begin{align*}
        \tau(m) &= (3+1)(k+1)(1+1) = (1 + 1)(k+1)(1+1)(1+1) = \tau(n)\\
        \sigma(m) &= (1 + 2 + 2^2 + 2^3)\sigma(3^k)(1 + 71) = 1080\sigma(3^k) = (1 + 2)\sigma(3^k)(1 + 11)(1 + 29) = \sigma(n)\\
        \phi(m) &= 2^2(2-1)\phi(3^k)(71-1) = 280\phi(3^k) = (2-1)\phi(3^k)(11-1)(29-1) = \sigma(n)
    \end{align*}
    which proves the claim. \\
\end{solution}

\begin{exercise}
    Establish each of the assertions below:
    \begin{enumerate}
        \item If $n$ is an odd integer, then $\phi(2n) = \phi(n)$.
        \item If $n$ is an even integer, then $\phi(2n) = 2\phi(n)$.
        \item $\phi(3n) = 3\phi(n)$ if and only if $3 \mid n$.
        \item $\phi(3n) = 2\phi(n)$ if and only if $3 \not| \ n$.
        \item $\phi(n) = n/2$ if and only if $n = 2^k$ for some $k \geq 1$. [\textit{Hint:} Write $n = 2^kN$, where $N$ is odd, and use the condition $\phi(n) = n/2$ to show that $N=1$.]\\
    \end{enumerate}
\end{exercise}

\begin{solution}
    \begin{enumerate}
        \item If $n$ is odd, then $\gcd(2,n) = 1$, and so it follows that $\phi(2n) = \phi(2)\phi(n) = \phi(n)$.
        \item If $n$ is even, then we can write $n = 2^km$ where $k \geq 1$ and $m$ such that $\gcd(2,m) = 1$. Hence, by multiplicativity:
        $$\phi(2n) = \phi(2^{k+1}m) = \phi(2^{k+1})\phi(m) = 2^k\phi(m) = 2\phi(2^k)\phi(m) = 2\phi(n).$$
        \item Write $n = 3^km$ where $\gcd(3,m) = 1$. If $3\mid n$, then $k \geq 1$ which implies that $3^k(3-1) = 3\cdot 3^{k-1}(3-1)$, and hence, that $\phi(3^{k+1})\phi(m) = 3\phi(3^k)\phi(m)$. Equivalently, this means that $\phi(3n) = 3\phi(n)$. Conversely, if $3 \not | \ n$, then $k = 0$, and hence, $\phi(3n) = \phi(3)\phi(n) = 2\phi(n) \neq 3\phi(n)$.
        \item We showed in the previous part that $3 \not | \ n$ implies that $\phi(3n) = 2\phi(n)$. Conversely, when $3 \mid n$, we know that $\phi(3n) = 3\phi(n)$, and hence, $\phi(3n) \neq 2\phi(n)$.
        \item If $n = 2^k$ for some $k \geq 1$, then
        $$\phi(n) = \phi(2^k) = 2^{k-1} = n/2.$$
        Conversely, suppose that $\phi(n) = n/2$ and write $n = p_1^{k_1}\cdot \cdot \cdot p_r^{k_r}$, then we can rewrite the equation as
        $$n \left(1 - \frac{1}{p_1}\right)\cdot \cdot \cdot \left(1 - \frac{1}{p_r}\right) = \frac{n}{2}.$$
        Rearranging the equation gives us
        $$2(p_1 - 1)\cdot \cdot \cdot (p_r - 1) = p_1\cdot \cdot \cdot p_r.$$
        Since the left hand side is even, then one of the primes in the right hand side must be 2, and hence, $p_1 = 2$. The equation becomes
        $$(p_2 - 1)\cdot \cdot \cdot (p_r - 1) = p_2\cdot \cdot \cdot p_r.$$
        But this equation is impossible (the left hand side is clearly strictly less than the right hand side) when $r \geq 2$, if $r = 1$, this equation is simply $1 = 1$. Hence, $r = 1$, and hence, $n = p_1^{k_1} = 2^{k_1}$ for some $k_1 \geq 1$.  \\
    \end{enumerate}
\end{solution}

\begin{exercise}
    Prove that the equation $\phi(n) = \phi(n+2)$ is satisfied by $n = 2(2p-1)$ whenever $p$ and $2p-1$ are both odd primes. \\
\end{exercise}

\begin{solution}
    Simply notice that
    $$\phi(n) = \phi(2)\phi(2p-1) = 2p-2$$
    and
    $$\phi(n+2) = \phi(4p) = \phi(4)\phi(p) = 2(p-1).$$
    Hence, it follows that $\phi(n) = \phi(n+2)$.\\
\end{solution}

\begin{exercise}
    Show that there are infinitely many integers $n$ for which $\phi(n)$ is a perfect square. [\textit{Hint:} Consider the integers $n = 2^{k+1}$ for $k = 1,2,\dots$.] \\
\end{exercise}

\begin{solution}
    Let $k$ be an arbitrary integer and $n = 2^{2k+1}$, then $\phi(n) = 2^{2k} = (2^k)^2$. Since $(2^k)^2 \neq (2^q)^2$ for $k \neq q$, then there are infinitely many $n$ such that $\phi(n)$ is a square since there are infinitely many $k$'s. \\
\end{solution}

\begin{exercise}
    Verify the following:
    \begin{enumerate}
        \item For any positive integer $n$, $\frac{1}{2}\sqrt{n} \leq \phi(n) \leq n$. [\textit{Hint:} Write $n = 2^{k_0}p_1^{k_1}\cdot \cdot \cdot p_r^{k_r}$, so $\phi(n) = 2^{k_0 - 1}p_1^{k_1-1}\cdot \cdot \cdot p_r^{k_r - 1}(p_1 - 1)\cdot \cdot \cdot (p_r - 1)$. Now, use the inequalities $p-1 > \sqrt{p}$ and $k - \frac{1}{2} \geq k/2$ to obtain $\phi(n) \geq 2^{k_0-1}p_1^{k_1/2}\cdot \cdot \cdot p_r^{k_r/2}$.]
        \item If the integer $n > 1$ has $r$ distinct prime factors, then $\phi(n) \geq n/2^r$.
        \item If $n > 1$ is a composite number, then $\phi(n) \leq n - \sqrt{n}$. [\textit{Hint:} Let $p$ be the smallest prime divisor of $n$, so that $p \leq \sqrt{n}$. Then $\phi(n) \leq n(1 - 1/p)$.] \\
    \end{enumerate}
\end{exercise}

\begin{solution}
    \begin{enumerate}
        \item Let $n = 2^{k_0}p_1^{k_1}\cdot \cdot \cdot p_r^{k_r}$, then $\phi(n) = 2^{k_0 - 1}p_1^{k_1-1}\cdot \cdot \cdot p_r^{k_r - 1}(p_1 - 1)\cdot \cdot \cdot (p_r - 1)$. Since $(p_i - 1) \geq \sqrt{p_i}$, then we get that $\phi(n) \geq \frac{1}{2}\cdot 2^{k_0}p_1^{k_1-\frac{1}{2}}\cdot \cdot \cdot p_r^{k_r -\frac{1}{2}}$. Next, since $k_0 \geq k_0/2$ and $k_i - \frac{1}{2} \geq k_i/2$, then $\phi(n) \geq \frac{1}{2}2^{k_0/2}p_1^{k_1/2}\cdot \cdot \cdot p_r^{k_r} = \frac{1}{2}\sqrt{n}$. Finally, the inequality $\phi(n) \leq n$ follows from the definition of $\phi$.
        \item Let $n = p_1^{k_1}\cdot \cdot \cdot p_r^{k_r}$, then 
        $$\frac{n}{\phi(n)} = \frac{n}{n\left(1 - \frac{1}{p_1}\right)\cdot \cdot \cdot \left(1 - \frac{1}{p_r}\right)} = \frac{p_1}{p_1 - 1}\cdot \cdot \cdot \frac{p_r}{p_r - 1}\leq 2^r$$
        which we can rearrange into $\phi(n) \geq n/2^r$.
        \item Let $p$ be a prime dividing $n$, since $n$ is composite, then $p \leq \sqrt{n}$, hence, $p\sqrt{n} \leq n$. It follows that the numbers $p$, $2p$, ..., $\lfloor \sqrt{n}\rfloor p$ are all less than $n$, and not relatively prime to $n$. There are $\lfloor \sqrt{n}\rfloor$ such numbers. If we let $q$ be a second prime divisor of $n$ (if $n$ is a power of $p$, then the statement is clear from the formula for $\phi(p^k)$ and the inequality $k-1 \geq k/2$ for $k \geq 2$), then we know that $q$ is a number distinct from $p$, $2p$, ..., $\lfloor \sqrt{n} \rfloor$ that is not relatively prime to $n$. Hence, we have found $\lfloor \sqrt{n} \rfloor + 1$ numbers less than $n$ which are not relatively prime to it. Therefore, $\phi(n) \leq n - (\lfloor \sqrt{n} \rfloor + 1) \leq n - \sqrt{n}$. \\
    \end{enumerate}
\end{solution}

\begin{exercise}
    Prove that if the integer $n$ has $r$ distinct odd prime factors, then $2^r \mid \phi(n)$.\\
\end{exercise}

\begin{solution}
    Write $n = 2^{k_0}p_1^{k_1}\cdot \cdot \cdot p_r^{k_r}$ where $k_0 \geq 0$ and $k_i \geq 1$ for $1 \leq i \leq r$, then
    $$\phi(n) = \phi(2^{k_0})p_1^{k_1 -1}\cdot \cdot \cdot p_r^{k_r - 1}(p_1 - 1)\cdot \cdot \cdot (p_r - 1).$$
    Since all the factors $p_i - 1$ are even, and there are $r$ such factors, then $2^r$ divides $\phi(n)$. \\
\end{solution}

\begin{exercise}
    Prove that:
    \begin{enumerate}
        \item If $n$ and $n+2$ are a pair of twin primes, then $\phi(n + 2) = \phi(n) + 2$; this also holds for $n=12$, 14, and 20.
        \item If $p$ and $2p+1$ are both odd primes, then $n = 4p$ satisfies $\phi(n+2) = \phi(n) + 2$.\\
    \end{enumerate}
\end{exercise}

\begin{solution}
    \begin{enumerate}
        \item For any prime $p$, we have that $\phi(p) = p-1$. It follows that
        $$\phi(n+2) = n+1 = (n-1) + 2 = \phi(n) + 2$$
        when both $n$ and $n+2$ are prime. When $n = 12$:
        $$\phi(n+2) = \phi(2)\phi(7) = 6 = \phi(2^2)\phi(3) + 1 = \phi(n) + 2.$$
        When $n = 14$:
        $$\phi(n+2) = \phi(2^4) = 8 = \phi(2)\phi(7) + 2 = \phi(n) + 2.$$
        When $n = 20$:
        $$\phi(n+2) = \phi(2)\phi(11) = 10 = \phi(2^2)\phi(5) + 2 = \phi(n) + 2.$$
        \item Suppose that $p$ and $2p+1$ are odd primes, then for $n = 4p$, we have
        $$\phi(n+2) = \phi(4p+2) = \phi(2)\phi(2p+1) = 2p$$
        and
        $$\phi(n) + 2 = \phi(4p) + 2 = \phi(4)\phi(p) + 2 = 2(p-1) + 2 = 2p$$
        which proves that $\phi(n + 2) = \phi(n) + 1$. \\
    \end{enumerate}
\end{solution}

\begin{exercise}
    If every prime that divides $n$ also divides $m$, establish that $\phi(nm) = n\phi(m)$; in particular, $\phi(n^2) = n\phi(n)$ for every positive integer $n$. \\
\end{exercise}

\begin{solution}
    Write $n = p_1^{k_1}\cdot \cdot \cdot p_r^{k_r}$ and $m = p_1^{t_1}\cdot \cdot \cdot p_r^{t_r}q_1^{s_1}\cdot \cdot \cdot q_u^{s_u}$, then
    \begin{align*}
        \phi(nm) &= \phi(p_1^{k_1 + t_1}\cdot \cdot \cdot p_r^{k_r + t_r}q_1^{s_1}\cdot \cdot \cdot q_u^{s_u}) \\
        &=  p_1^{k_1 + t_1 - 1}(p_1 - 1)\cdot \cdot \cdot p_r^{k_r + t_r - 1}(p_r - 1) \phi(q_1^{s_1}\cdot \cdot \cdot q_u^{s_u}) \\
        &= [p_1^{k_1}\cdot \cdot \cdot p_r^{k_r}]\cdot p_1^{t_1 - 1}(p_1 - 1)\cdot \cdot \cdot p_r^{t_r - 1}(p_r - 1)\phi(q_1^{s_1}\cdot \cdot \cdot q_u^{s_u}) \\
        &= n \phi(p_1^{t_1}\cdot \cdot \cdot p_r^{t_r})\phi(q_1^{s_1}\cdot \cdot \cdot q_u^{s_u}) \\
        &= n \phi(m).
    \end{align*}
     \\
\end{solution}

\begin{exercise}
    \begin{enumerate}
        \item If $\phi(n) \mid n - 1$, prove that $n$ is a square-free integer. [\textit{Hint:} Assume that $n$ has the prime factorization $n = p_1^{k_1}p_2^{k_2}\cdot \cdot \cdot p_r^{k_r}$, where $k_1 \geq 2$. Then $p_1 \mid \phi(n)$, whence $p_1 \mid n-1$, which leads to a contradiction.]
        \item Show that if $n = 2^k$ or $n = 2^k3^j$, with $k$ and $j$ positive integers, then $\phi(n) \mid n$. \\
    \end{enumerate}
\end{exercise}

\begin{solution}
    \begin{enumerate}
        \item Let $p$ be a prime divisor of $n$ and let $k$ be the exponent of $p$ in the prime factorization of $n$, then $\phi(n) = \phi(p^k)\phi(n/p^k) = p^{k-1}(p-1)\phi(n/p^k)$. When $k \geq 2$, we have that $p \mid \phi(n)$ using the expression above, but since $\phi(n) \mid n-1$, then $p \mid n-1$. This is impossible because $p \mid n$. Thus, $k = 1$, and hence, $n$ is square-free.
        \item Let $n = 2^k3^k$ with $k \geq 1$. If $j = 0$, then $\phi(n) = 2^{k-1} \mid 2^k = n$. When $j \geq 1$, we get $\phi(n) = 2^{k-1}3^{j-1}(3-1) = 2^k3^{j-1} \mid 2^k3^j = n$. \\
    \end{enumerate}
\end{solution}

\begin{exercise}
    If $n = p_1^{k_1}p_2^{k_2}\cdot \cdot \cdot p_r^{k_r}$, derive the inequalities
    \begin{enumerate}
        \item $\sigma(n)\phi(n) \geq n^2(1 - 1/p_1^2)(1 - 1/p_2^2)\cdot \cdot \cdot (1 - 1/p_r^2)$ and 
        \item $\tau(n)\phi(n) \geq n$. [\textit{Hint:} Show that $\tau(n)\phi(n) \geq 2^r \cdot n(1/2)^r$.] \\
    \end{enumerate}
\end{exercise}

\begin{solution}
    \begin{enumerate}
        \item First, notice that
        \begin{align*}
            \sigma(n) &= (1 + \dots + p_1^{k_1 - 1} + p_1^{k_1})\cdot \cdot \cdot (1 + \dots + p_r^{k_r - 1} + p_r^{k_r}) \\
            &\geq (p_1^{k_1} + p_1^{k_1 - 1})\cdot \cdot \cdot (p_r^{k_r} + p_r^{k_r - 1}) \\
            &= p_1^{k_1}\cdot \cdot \cdot p_r^{k_r}(1 + 1/p_1)\cdot \cdot \cdot (1 + 1/p_r) \\
            &= n(1 + 1/p_1)\cdot \cdot \cdot (1 + 1/p_r).
        \end{align*}
        Multiplying the resulting inequality by
        $$\phi(n) = n(1 - 1/p_1)\cdot \cdot \cdot (1 - 1/p_r)$$
        gives us
        $$\sigma(n)\tau(n) \geq n^2(1 - 1/p_1^2)\cdot \cdot \cdot (1 - 1/p_r^2).$$
        \item Since
        \begin{align*}
            \tau(n) &= (k_1 + 1)(k_2 + 1)\cdot \cdot \cdot (k_r + 1) \\
            &\geq \left(1 - \frac{1}{p_1}\right)^{-1}\left(1 - \frac{1}{p_2}\right)^{-1} \cdot \cdot \cdot \left(1 - \frac{1}{p_r}\right)^{-1} \\
            &= \frac{n}{\phi(n)},
        \end{align*}
        then $\tau(n)\phi(n) \geq n$. \\
    \end{enumerate}
\end{solution}

\begin{exercise}
    Assuming that $d \mid n$, prove that $\phi(d) \mid \phi(n)$. [\textit{Hint:} Work with the prime factorizations of $d$ and $n$.] \\
\end{exercise}

\begin{solution}
    Let $d = p_1^{k_1}\cdot \cdot \cdot p_r^{k_r}$ and $n = p_1^{t_1}\cdot \cdot \cdot p_r^{t_r}q_1^{s_1}\cdot \cdot \cdot q_u^{s_u}$ with $k_i \leq t_i$, then $p_i^{k_i-1} \mid p_i^{t_i-1}$. It follows that $\prod_{i=1}^{r}p_i^{k_i-1} \mid \prod_{i=1}^{r}p_i^{t_i-1}$. Using the rules of divisibility, we get that $\phi(d)$, which is equal to $\prod_{i=1}^{r}p_i^{k_i-1}(p_i - 1)$ divides $\prod_{i=1}^{r}p_i^{t_i-1}(p_i - 1)$, which in turns divides $\prod_{i=1}^{r}p_i^{t_i-1}(p_i - 1)\prod_{i=1}^{u}q_i^{s_i-1}(q_i - 1) = \phi(n)$. Therefore, $\phi(d) \mid \phi(n)$.\\
\end{solution}

\begin{exercise}
    Obtain the following two generalizations of Theorem 7-2:
    \begin{enumerate}
        \item For positive integers $m$ and $n$,
        $$\phi(m)\phi(n) = \phi(mn)\phi(d)/d,$$
        where $d = \gcd(m,n)$.
        \item For positive integers $m$ and $n$,
        $$\phi(m)\phi(n) = \phi(\gcd(m,n))\phi(\lcm(m,n)).$$
    \end{enumerate}
\end{exercise}

\begin{solution}
    \begin{enumerate}
        \item Let $m = p_1^{k_1}\cdot \cdot \cdot p_r^{k_r}a$ and $n = p_1^{t_1}\cdot \cdot \cdot p_r^{t_r}b$ where the $p_i$'s are the prime common to $m$ and $n$, and where $a$ and $b$ are prime relative to the $p_i$'s, then $d = \gcd(m,n)$ is also formed of the primes $p_1$, ..., $p_r$. It follows that
        \begin{align*}
            \phi(mn)\frac{\phi(d)}{d} &= \phi\left(ab\prod_i p_i^{k_i+t_i}\right) \prod_i\left(1 - \frac{1}{p_i}\right) \\
            &= \phi(a)\phi(b) \prod_i p_i^{k_i + t_i - 1}(p_i - 1) \prod_i \frac{p_i - 1}{p_i}\\
            &= \left[\phi(a) \prod_i p_i^{k_i - 1}(p_i - 1) \right] \left[\phi(b)\prod_i p_i^{t_i - 1}(p_i - 1) \right]\\  
            &= \phi(m)\phi(n).
        \end{align*}
        \item First, recall that $mn = \gcd(m,n)\lcm(m,n)$. Moreover, since $\gcd(m,n) \mid mn$, then
        $$\phi(mn) = \gcd(m,n)\phi\left(\frac{mn}{\gcd(m,n)}\right) = \gcd(m,n)\phi(\lcm(m,n))$$
        by Problem 10. Therefore, using part (a):
        $$\phi(m)\phi(n) = \phi(mn)\frac{\phi(\gcd(m,n))}{\gcd(m,n)} = \phi(\gcd(m,n))\phi(\lcm(m,n)).$$ \\
    \end{enumerate}
\end{solution}

\begin{exercise}
    Prove that:
    \begin{enumerate}
        \item There are infinitely many integers $n$ for which $\phi(n) = n/3$. [\textit{Hint:} Consider $n = 2^k3^j$, where $k$ and $j$ are positive integers.]
        \item There are no integers $n$ for which $\phi(n) = n/4$.\\
    \end{enumerate}
\end{exercise}

\begin{solution}
    \begin{enumerate}
        \item For all positive integers $k$ and $j$, if we let $n = 2^k3^j$, then
        $$\phi(n) = \phi(2^k)\phi(3^j) = 2^{k-1}3^{j-1}(3-1) = \frac{2^k3^j}{3} = \frac{n}{3}.$$
        Since there infinitely many integers of the form $2^k3^j$, then there are infinitely many integers $n$ such that $\phi(n) = n/3$.
        \item Suppose that there exists an integer $n = p_1^{k_1}\cdot \cdot \cdot p_r^{k_r}$ such that $\phi(n) = n/4$, then equivalently:
        $$4p_1^{k_1-1}(p_1 - 1)\cdot \cdot \cdot p_r^{k_r-1}(p_r - 1) = p_1^{k_1}\cdot \cdot \cdot p_r^{k_r}$$
        which we can rewrite as
        $$4(p_1 - 1)\cdot \cdot \cdot (p_r - 1) = p_1\cdot \cdot \cdot p_r.$$
        Since 2 divides the left hand side, then 2 divides $p_1\cdot \cdot \cdot p_r$ which implies that $p_i = 2$ for some $1 \leq i \leq r$. Without loss of generality, we can assume that $p_1 = 2$. If we cancel out $p_1$ and 2 on both sides, we get the equivalent equation
        $$2(p_1 - 1)\cdot \cdot \cdot (p_r - 1) = p_2\cdot \cdot \cdot p_r.$$ 
        Applying the same argument as before ensures that $p_i = 2$ for some $2 \leq i \leq r$, but it would imply that $p_i = p_1$ with $i \neq 1$. This is impossible because the $p_i$'s are distinct from each other. Therefore, by contradiction, there is no integer $n$ such that $\phi(n) = n/4$. \\
    \end{enumerate}
\end{solution}

\begin{exercise}
    \td
\end{exercise}
\section{Euler's Theorem}

\begin{exercise}
    Use Euler's Theorem to establish the following:
    \begin{enumerate}
        \item For any integer $a$, $a^{37} \equiv a\pmod{1729}$. [\textit{Hint:} $1729 = 7 \cdot 13 \cdot 19$.]
        \item For any integer $a$, $a^{13} \equiv a\pmod{2730}$. [\textit{Hint:} $2730 = 2 \cdot 3 \cdot 5 \cdot 7 \cdot 13$.]
        \item For any odd integer $a$, $a^{33} \equiv a\pmod{4080}$. [\textit{Hint:} $4080 = 15 \cdot 16 \cdot 17$.] \\
    \end{enumerate}
\end{exercise}

\begin{solution}
    \begin{enumerate}
        \item Let $a$ be an integer not divisible by 7, then $a^6 \equiv 1 \pmod 7$ (Euler's Theorem). Taking both sides to the power of 6 gives us $a^{36} \equiv 1 \pmod{7}$. Multiplying both sides by $a$ gives us $a^{37} \equiv a \pmod{7}$ which now holds for any integer $a$. Next, let $a$ be an integer not divisible by 13, then $a^{12} \equiv 1 \pmod{13}$ (Euler's Theorem). Taking both sides to the power of 3 gives us $a^{36} \equiv 1 \pmod{13}$. Multiplying both sides by $a$ gives us $a^{37} \equiv a \pmod{13}$ which now holds for any integer $a$. Finally, if we let $a$ be an integer not divisible by 19, then $a^{18} \equiv 1 \pmod{19}$ (Euler's Theorem). Squaring both sides gives us $a^{36} \equiv 1 \pmod{19}$. Multiplying both sides by $a$ gives us $a^{37} \equiv a \pmod{19}$ which now holds for any integer $a$. Therefore, for any integer $a$, we have the three congruences $a^{37} \equiv a \pmod{7}$, $a^{37} \equiv a \pmod{13}$, and $a^{37} \equiv a \pmod{19}$. Since 7, 13, and 19 have no common divisor, then $a^{37} \equiv a \pmod{1729}$.
        \item For any odd integer $a$, we have $a^1 \equiv 1 \pmod 2$. Taking both sides to the power of 12 and multiplying both sides by $a$ gives us $a^{13} \equiv a \pmod{2}$ which holds for all integers $a$. For any integer $a$ not divisible by 3, we have $a^2 \equiv 1 \pmod 3$. Taking both sides to the power of 6 and multiplying both sides by $a$ gives us $a^{13} \equiv a \pmod{3}$ which holds for all integers $a$. For any integer $a$ not divisible by 5, we have $a^4 \equiv 1 \pmod 5$. Taking both sides to the power of 3 and multiplying both sides by $a$ gives us $a^{13} \equiv a \pmod{5}$ which holds for all integers $a$. For any integer $a$ not divisible by 7, we have $a^6 \equiv 1 \pmod 7$. Taking both sides to the power of 2 and multiplying both sides by $a$ gives us $a^{13} \equiv a \pmod{7}$ which holds for all integers $a$. Finally, for any integer $a$, we have $a^{13} \equiv 1 \pmod{13}$. Therefore, $a^{13} \equiv a \pmod{2730}$.
        \item For any integer $a$ not divisible by 3, we have $a^2 \equiv 1 \pmod 3$. Taking both sides to the power of 16 and multiplying both sides by $a$ gives us $a^{33} \equiv a \pmod{3}$ which holds for all integers $a$. For any integer $a$ not divisible by 5, we have $a^4 \equiv 1 \pmod 5$. Taking both sides to the power of 8 and multiplying both sides by $a$ gives us $a^{33} \equiv a \pmod{5}$ which holds for all integers $a$. For any integer $a$ not divisible by 17, we have $a^{16} \equiv 1 \pmod{17}$. Squaring and multiplying both sides by $a$ gives us $a^{33} \equiv a \pmod{17}$ which holds for all integers $a$. Finally, using Euler's Theorem, we have that $a^{8} \equiv 1 \pmod{16}$ for all even integers $a$. Taking both sides to the power of 4 and multiplying both sides by $a$ gives us $a^{33} \equiv a \pmod{16}$. Hence, for all even integers $a$, we have $a^{33} \equiv a \pmod{4080}$. \\
    \end{enumerate}
\end{solution}

\begin{exercise}
    Use Euler's Theorem to confirm that, for any integer $n \geq 0$,
    $$51 \mid 10^{32n+9} - 7.$$ \\
\end{exercise}

\begin{solution}
    Since $51 = 3 \cdot 17$, then $\phi(51) = 2\cdot 16 = 32$. Since $\gcd(10, 51) = 1$, then by Euler's Theorem: $10^{32} \equiv 1 \pmod{51}$. Taking both sides to the power of $n$ gives us $10^{32n} \equiv 1 \pmod{51}$. Since
    $$10^9 = 10 \cdot (100)^4 \equiv 10\cdot (-2)^4 = 160 \equiv 7 \pmod{51},$$
    then $10^{32n+9} = 10^{32n}\cdot 10^9 \equiv 7 \pmod{51}$ which is equivalent to
    $$51 \mid 10^{32n+9} - 7.$$ \\
\end{solution}

\begin{exercise}
    Prove that $2^{15} - 2^3$ divides $a^{15} - a^3$ for any integer $a$.
    
    [\textit{Hint:} $2^{15} - 2^3 = 5 \cdot 7 \cdot 8 \cdot 9 \cdot 13$.]\\
\end{exercise}

\begin{solution}
    For all integers $a$ not divisible by 5, we have $a^4 \equiv a^{\phi(5)} \equiv 1 \pmod 5$. Cubing and multiplying both sides by $a^3$ gives us $a^{15} \equiv a^3 \pmod{5}$ for all integers $a$. For all integers $a$ not divisible by 7, we have $a^6 \equiv a^{\phi(7)} \equiv 1 \pmod 7$. Squaring and multiplying both sides by $a^3$ gives us $a^{15} \equiv a^3 \pmod{7}$ for all integers $a$. For all integers $a$ not divisible by 2, we have $a^4 \equiv a^{\phi(8)} \equiv 1 \pmod 8$. Cubing and multiplying both sides by $a^3$ gives us $a^{15} \equiv a^3 \pmod{8}$ for all integers $a$. For all integers $a$ not divisible by 3, we have $a^6 \equiv a^{\phi(9)} \equiv 1 \pmod 9$. Squaring and multiplying both sides by $a^3$ gives us $a^{15} \equiv a^3 \pmod{9}$ for all integers $a$. Finally, for all integers $a$, we have $a^{13} \equiv a \pmod{13}$ so multiplying both sides by $a^2$ gives us $a^{15} \equiv a^3 \pmod{13}$. Therefore, for all integers $a$, $a^{15} \equiv a^3 \pmod{2^{15} - 2^3}$ which is equivalent to $2^{15} - 2^3$ divides $a^{15} - a^3$.\\
\end{solution}

\begin{exercise}
    Show that if $\gcd(a, n) = \gcd(a-1, n) = 1$, then
    $$1+a+a^2+\dots+a^{\phi(n) - 1} \equiv 0 \pmod{n}.$$
    [\textit{Hint:} Recall that
    $$a^{\phi(n)} - 1 = (a-1)(a^{\phi(n)-1} + \dots + a^2 + a + 1).]$$ \\
\end{exercise}

\begin{solution}
    Since $\gcd(a,n) = 1$, then $a^{\phi(n)} - 1 \equiv 0 \pmod{n}$. Hence, if we take the equation
    $$a^{\phi(n)} - 1 = (a-1)(a^{\phi(n)-1} + \dots + a^2 + a + 1)$$
    modulo $n$, then we get that
    $$(a-1)(a^{\phi(n)-1} + \dots + a^2 + a + 1) \equiv 0 \pmod{n}.$$
    Since $\gcd(a-1, n) = 1$, then
    $$a^{\phi(n)-1} + \dots + a^2 + a + 1 \equiv 0 \pmod{n}$$
    by Corollary 1 of Theorem 4-3. \\
\end{solution}

\begin{exercise}
    If $m$ and $n$ are relatively prime positive integers, prove that
    $$m^{\phi(n)} + n^{\phi(m)} \equiv 1 \pmod{mn}.$$ \\
\end{exercise}

\begin{solution}
    Since $m$ is prime relative to $n$, then
    $m^{\phi(n)} + n^{\phi(m)} \equiv m^{\phi(n)} + 0 \equiv 1 \pmod n$. Similarly, $m^{\phi(n)} + n^{\phi(m)} \equiv 1 \pmod m$. Since $m$ and $n$ are relatively prime, then $m^{\phi(n)} + n^{\phi(m)} \pmod{mn}$. \\
\end{solution}

\begin{exercise}
    Fill any missing details in the following proof of Euler's Theorem: Let $p$ be a prime divisor of $n$ and $\gcd(a,p) = 1$. By Fermat's Theorem, $a^{p-1} \equiv 1 \pmod p$, so that $a^{p-1} = 1 + tp$ for some $t$. Then, $a^{p(p-1)} = (1 + tp)^p = 1 + \binom{p}{1}(tp) + \dots + (tp)^p \equiv 1 \pmod{p^2}$ and, by induction, $a^{p^{k-1}(p-1)} \equiv 1 \pmod{p^k}$ where $k = 1,2,\dots .$ Raise both sides of the congruence to the $\phi(n)/p^{k-1}(p-1)$ power to get $a^{\phi(n)} \equiv 1 \pmod{p^k}$. Thus, $a^{\phi(n)} \equiv 1 \pmod n$. \\
\end{exercise}

\begin{solution}
    Let's prove more rigorously that $a^{p^{k-1}(p-1)} \equiv 1 \pmod{p^k}$ for all $k \geq 1$. The case $k= 1$ follows from Fermat's Theorem. Next, if $a^{p^{k-1}(p-1)} \equiv 1 \pmod{p^k}$ for some $k$, then it means that $a^{p^{k-1}(p-1)} = 1 + tp^k$ for some integer $t$. Taking both sides to the power of $p$ gives us $a^{p^k(p-1)} = (1 + tp^k)^p$. Using the Binomial Formula, we have that
    $$(1 + tp^k)^p = 1 + tp^{k+1} + \sum_{i=2}^{p}\binom{p}{i}t^ip^{ik}.$$
    Since $p^{ik} \equiv 0 \pmod{p^{k+1}}$ for all $i \geq 2$, we get that $a^{p^k(p-1)} = (1 + tp^k)^p \equiv 1 \pmod{p^{k+1}}$. Hence, by indution, $a^{p^{k-1}(p-1)} \equiv 1 \pmod{p^k}$ for all $k \geq 1$. 
    
    Next, if we let $k$ be such that $p^k$ is the highest power of $p$ dividing $n$, then $\phi(n)$ is divisible by $p^{k-1}(p-1)$, so $\phi(n)/p^{k-1}(p-1)$ is an integer. From the part proved by induction, we get that $a^{\phi(n)} \equiv 1 \pmod{p^k}$. Finally, if we write $n = p_1^{k_1} \cdot \cdot \cdot p_r^{k_r}$, then $a^{\phi(n)} \equiv 1 \pmod{p_i^{k_i}}$ for all $i$. Since $p_1^{k_1}$, ..., $p_r^{k_r}$ are relatively prime, then $a^{\phi(n)} \equiv 1 \pmod n$.\\
\end{solution}

\begin{exercise}
    Find the units digit of $3^{100}$ by means of Euler's Theorem. \\
\end{exercise}

\begin{solution}
    Using Euler's Theorem, we have that $3^4 = 3^{\phi(10)} \equiv 1 \pmod{10}$ (since $\gcd(3,10) = 1$). Hence:
    $$3^{100} = (3^4)^{25} \equiv 1^{25} = 1 \pmod{10}.$$ \\
\end{solution}

\begin{exercise}
    \begin{enumerate}
        \item If $\gcd(a,n) = 1$, show that the linear congruence $ax \equiv b \pmod n$ has the solution $x \equiv ba^{\phi(n) - 1} \pmod n$.
        \item Use part (a) to solve the congruences $3x \equiv 5 \pmod{26}$, $13x \equiv 2 \pmod{40}$ and $10x \equiv 21 \pmod{49}$.\\
    \end{enumerate}
\end{exercise}

\begin{solution}
    \begin{enumerate}
        \item If we let $x \equiv ba^{\phi(n) - 1}$, then $ax \equiv ba^{\phi(n)} \equiv b \pmod n$ by Euler's Theorem. Hence, $x \equiv ba^{\phi(n) - 1}$ is the solution of the congruence $ax \equiv b \pmod n$ (the solution is unique modulo $n$).
        \item Using part (a), the solution of the congruence $3x \equiv 5 \pmod{26}$ is
        $$x \equiv 5\cdot 3^{\phi(26) - 1} = 5 \cdot 3^{11} = 5 \cdot 3^2 \cdot (3^3)^3 \equiv 5 \cdot 9 \equiv 19 \pmod{26}.$$
        The solution of the congruence $13x \equiv 2 \pmod{40}$ is
        $$x \equiv 2\cdot 13^{\phi(40) - 1} = 2\cdot 13^{15} = 2 \cdot 13^3 \cdot (13^4)^3 \equiv 2\cdot (-3) \cdot 1 \equiv 34 \pmod{40}.$$
        The solution of the congruence $10x \equiv 21 \pmod{49}$ is
        $$x \equiv 21\cdot 10^{\phi(49) - 1} =  21 \cdot 10 \cdot 100^{20} \equiv 14 \cdot 2^{20} \equiv 14 \cdot (-5)^2 \equiv 7\pmod{49}.$$
    \end{enumerate}
\end{solution}

\begin{exercise}
    Prove that every prime other than 2 or 5 divides infinitely many of the integers 1, 11, 111, 1111, .... \\
\end{exercise}

\begin{solution}
    First, notice that the prime number 3 divides infinitely many of these integers because it divides all such integers with a multiple of 3 number of 1's (by the divisibility test). Let $p$ be a prime distinct from 2, 3, and 5, then $10^{k\phi(p)} \equiv 1 \pmod p$ for all $k$, so $p$ divides $10^{k\phi(p)} - 1$. Since 9 divides $10^{k\phi(p)} - 1$ as well, and $\gcd(p, 9) = 1$, then $9p$ divides $10^{k\phi(p)} - 1$. Equivalently, this means that $p$ divides 11...11 whenever the number of 1's is equal to a multiple of $\phi(p)$. Since there are infinitely many such numbers, then $p$ divides infinitely many of the integers 1, 11, 111, 1111, .... \\
\end{solution}

\begin{exercise}
    For any prime $p$, establish each of the assertions below:
    \begin{enumerate}
        \item $\tau(p!) = 2\tau((p-1)!)$;
        \item $\sigma(p!) = (p+1)\sigma((p-1)!)$;
        \item $\phi(p!) = (p-1)\phi((p-1)!)$.\\
    \end{enumerate}
\end{exercise}

\begin{solution}
    \begin{enumerate}
        \item Since $p$ is prime, then $(p-1)!$ is not divisible by $p$ so it is prime relative to $p$. It follows that
        $$\tau(p!) = \tau(p)\tau((p-1)!) = 2\tau((p-1)!).$$
        \item Since $p$ is prime, then $(p-1)!$ is not divisible by $p$ so it is prime relative to $p$. It follows that
        $$\sigma(p!) = \sigma(p)\sigma((p-1)!) = (p+1)\sigma((p-1)!).$$
        \item Since $p$ is prime, then $(p-1)!$ is not divisible by $p$ so it is prime relative to $p$. It follows that
        $$\phi(p!) = \phi(p)\phi((p-1)!) = (p-1)\phi((p-1)!).$$
    \end{enumerate}
\end{solution}

\begin{exercise}
    Given $n \geq 1$, a set of $\phi(n)$ integers which are relatively prime to $n$ and which are incongruent modulo $n$ is called a \textit{reduced set of residues modulo $n$} (that is, a reduced set of residues are those members of a complete set of residues modulo $n$ which are relatively prime to $n$).
    
    Verify that
    \begin{enumerate}
        \item the integers $-31, -16, -8, 13, 25, 80$ form a reduced set of residues modulo 9;
        \item the integers $3, 3^2, 3^3, 3^4, 3^5, 3^6$ form a reduced set of residues modulo 14;
        \item the integers $2, 2^2, 2^3, ..., 2^{18}$ form a reduced set of residues modulo 27. \\
    \end{enumerate}
\end{exercise}

\begin{solution}
    \begin{enumerate}
        \item If we take the residues of the numbers $-31, -16, -8, 13, 25, 80$ modulo 9, we get the numbers $5, 2, 1, 4, 7, 8$. Since all of these numbers are distint, then the original numbers must be noncongruent. Hence, there are $6 = \phi(9)$ noncongruent elements in the list. Moreover, since all the elements in the new list are prime relative to 9, then the same holds for the original list. Therefore, the integers $-31, -16, -8, 13, 25, 80$ form a reduced set of residues modulo 9.
        \item The integers $3, 3^2, 3^3, 3^4, 3^5, 3^6$ are all prime relative to 14 since they share no prime factors. Moreover, if we take the residues of the numbers in this list, we get the new list $3, 9, 13, 11, 5, 1$ where clearly, no two elements are congruent. Hence, there are $6 = \phi(14)$ noncongruent elements in the list. Therefore, the integers $3, 3^2, 3^3, 3^4, 3^5, 3^6$ form a reduced set of residues modulo 14. 
        \item The integers $2, 2^2, 2^3, ..., 2^{18}$ are all prime relative to 27 since they share no prime factors. Moreover, if we take the residues of the numbers in this list, we get the new list $2, 4, 8, 16, 5, 10, 20, 13, 26, 25, 23, 19, 11, 22, 17, 7, 14, 1$ where clearly, no two elements are congruent. Hence, there are $18 = \phi(27)$ noncongruent elements in the list. Therefore, the integers $2, 2^2, 2^3, ..., 2^{18}$ form a reduced set of residues modulo 14. \\
    \end{enumerate}
\end{solution}

\begin{exercise}
    If $p$ is an odd prime, show that the integers
    $$-\frac{p-1}{2}, ..., -2, -1, 1,2,...,\frac{p-1}{2}$$
    form a reduced set of residues modulo $p$.\\
\end{exercise}

\begin{solution}
    Using Problem 12(b) from section 4.3, we have that the integers
    $$-\frac{p-1}{2}, ..., -2, -1, 0, 1,2,...,\frac{p-1}{2}$$
    form a complete set of residues modulo $n$. But since 0 is the only element which is not prime relative to $p$, then it follows that by removing 0 from these integers, we get the following reduced set of residues modulo $p$:
    $$-\frac{p-1}{2}, ..., -2, -1, 1,2,...,\frac{p-1}{2}$$
\end{solution}


\section{Some Properties of the Phi-Function}

\begin{exercise}
    For a positive integer $n$, prove that
    $$\sum_{d \ \mid \ n}(-1)^{n/d}\phi(d) = \begin{cases}
        0 \text{ if } n \text{ is even} \\
        -n \text{ if } n \text{ is odd} \\
    \end{cases}$$
    [\textit{Hint:} If $n = 2^kN$, where $N$ is odd, then $\sum_{d \mid n}(-1)^{n/d}\phi(d) = \sum_{d \mid 2^{k-1}N}\phi(d) - \sum_{d \mid N}\phi(2^kd)$.]\\
\end{exercise}

\begin{solution}
    First, let $n$ be an even number and write it as $2^k N$ where $k \geq 1$ and $N$ is odd, then the divisors of $n$ are precisely the numbers of the form $2^i d$ where $0 \leq i \leq k$ and $d$ is a divisor of $N$. Hence:
    $$\sum_{d \ \mid \ n}(-1)^{n/d}\phi(d) = \sum_{i=0}^{k}\sum_{d \ \mid \ N}(-1)^{n/2^id}\phi(2^id).$$
    When $i \neq k$, $n/2^id = 2^{k-i}(N/d)$ is an even number, and when $i = k$, it is odd, so we get
    \begin{align*}
        \sum_{d \ \mid \ n}(-1)^{n/d}\phi(d) &= \sum_{i=0}^{k-1}\sum_{d \ \mid \ N}(-1)^{n/2^id}\phi(2^id) + \sum_{d \ \mid \ N}(-1)^{n/2^kd}\phi(2^kd) \\
        &= \sum_{i=0}^{k-1}\sum_{d \ \mid \ N}\phi(2^i)\phi(d) - \sum_{d \ \mid \ N}\phi(2^k)\phi(d) \\
        &= \sum_{i=0}^{k-1}\phi(2^i)\sum_{d \ \mid \ N}\phi(d) - \phi(2^k)\sum_{d \ \mid \ N}\phi(d).
    \end{align*}
    Using Theorem 7-6:
    \begin{align*}
        \sum_{d \ \mid \ n}(-1)^{n/d}\phi(d) &= \sum_{i=0}^{k-1}\phi(2^i)N - \phi(2^k)N \\
        &= N\left[\sum_{i=0}^{k-1}\phi(2^i) - 2^{k-1}\right] \\
        &= N(2^{k-1} - 2^{k-1}) \\
        &= 0.
    \end{align*}
    When $n$ is odd, then for all divisors $d$ of $n$, we have that $n/d$ is odd as well. Thus, using Theorem 7-6:
    $$\sum_{d \ \mid \ n}(-1)^{n/d}\phi(d) = -\sum_{d \ \mid \ n}\phi(d) = -n.$$
    Therefore:
    $$\sum_{d \ \mid \ n}(-1)^{n/d}\phi(d) = \begin{cases}
        0 \text{ if } n \text{ is even} \\
        -n \text{ if } n \text{ is odd} \\
    \end{cases}.$$
\end{solution}

\begin{exercise}
    Confirm that $\sum_{d \mid 36}\phi(d) = 36$ and $\sum_{d \mid 36}(-1)^{36/d}\phi(d) = 0$. \\
\end{exercise}

\begin{solution}
    We have
    \begin{align*}
        \sum_{d \ \mid \ 36}\phi(d) &= \phi(1) + \phi(2) + \phi(3) + \phi(4) + \phi(6) + \phi(9) + \phi(12) + \phi(18) + \phi(36) \\
        &= 1 + 1 + 2 + 2 + 2 + 6 + 4 + 6 + 12 \\
        &= 36
    \end{align*}
    and
    \begin{align*}
        \sum_{d \ \mid \ 36}(-1)^{36/d}\phi(d) &= (-1)^{36}\phi(1) + (-1)^{18}\phi(2) + (-1)^{12}\phi(3) + (-1)^{9}\phi(4) + (-1)^{6}\phi(6) \\
        &\qquad  + (-1)^{4}\phi(9) + (-1)^{3}\phi(12) + (-1)^{2}\phi(18) + (-1)^{1}\phi(36) \\
        &= 1 + 1 + 2 - 2 + 2 + 6 - 4 + 6 - 12 \\
        &= 0.\\
    \end{align*}
\end{solution}

\begin{exercise}
    For a positive integer $n$, prove that $\sum_{d \mid n}\mu^2(d)/\phi(d) = n / \phi(n)$.
    
    [\textit{Hint:} See the hint in Problem 1.]\\
\end{exercise}

\begin{solution}
    Let $F(n) = \sum_{d \mid n}\mu^2(d)/\phi(d)$ and $G(n) = n/\phi(n)$. Since both are multiplicative functions, then to prove that $F = G$, it suffices to show that $F(p^k) = G(p^k)$ for all prime $p$ and $k \geq 1$. Let $p$ be a prime and $k \geq 1$, then
    \begin{align*}
        F(p^k) &= \sum_{d \mid p^k}\frac{\mu^2(d)}{\phi(d)} \\
        &= \frac{\mu^2(1)}{\phi(1)} + \frac{\mu^2(p)}{\phi(p)} + \frac{\mu^2(p^2)}{\phi(p^2)} \dots + \frac{\mu^2(p^k)}{\phi(p^k)} \\
        &= \frac{1}{1} + \frac{1}{p-1} + 0 \dots + 0 \\
        &= \frac{p}{p-1} \\
        &= \frac{p^k}{p^{k-1}(p-1)}\\
        &= \frac{p^k}{\phi(p^k)} \\
        &= G(p^k).
    \end{align*}
    Therefore, $\sum_{d \mid n}\mu^2(d)/\phi(d) = n / \phi(n)$. \\
\end{solution}

\begin{exercise}
    Use Problem 3, Section 6-2, to give a different proof of the fact that
    $$n \sum_{d \ \mid \ n}\mu(d)/d = \phi(n).$$\\
\end{exercise}

\begin{solution}
    Using Problem 3 of Section 6-2, we have that
    $$\sum_{d \ \mid \ n}\mu(d)/d = \left(1 - \frac{1}{p_1}\right) \cdot \cdot \cdot \left(1 - \frac{1}{p_r}\right)$$
    where $n = p_1^{k_1} \cdot \cdot \cdot p_r^{k_r}$. Hence,
    $$n\sum_{d \ \mid \ n}\mu(d)/d = n\left(1 - \frac{1}{p_1}\right) \cdot \cdot \cdot \left(1 - \frac{1}{p_r}\right).$$
    By Theorem 7-3, it follows that
    $$n \sum_{d \ \mid \ n}\mu(d)/d = \phi(n).$$ \\
\end{solution}

\begin{exercise}
    If the integer $n > 1$ has the prime factorization $n = p_1^{k_1} p_2^{k_2} \cdot \cdot \cdot p_r^{k_r}$, establish the following:
    \begin{enumerate}
        \item $\displaystyle \sum_{d \ \mid \ n}\mu(d)\phi(d) = (2 - p_1)(2 - p_2) \cdot \cdot \cdot (2 - p_r)$
        \item $\displaystyle \sum_{d \ \mid \ n}d\phi(d) = \left(\frac{p_1^{2k_1 + 1}+1}{p_1 + 1}\right)\left(\frac{p_2^{2k_2 + 1}+1}{p_2 + 1}\right)$
         
        $\displaystyle \qquad \qquad \qquad \cdot \cdot \cdot \left(\frac{p_r^{2k_r + 1}+1}{p_r + 1}\right)$
        \item $\displaystyle \sum_{d \ \mid \ n}\phi(d)/d = \left(1 + \frac{k_1(p_1 - 1)}{p_1}\right)\left(1 + \frac{k_2(p_2 - 1)}{p_2}\right)$
         
        $\displaystyle \qquad \qquad \qquad \cdot \cdot \cdot \left(1 + \frac{k_r(p_r - 1)}{p_r}\right)$
    \end{enumerate}
    [\textit{Hint:} For part (a), use Problem 3, Section 6-2.] \\
\end{exercise}

\begin{solution}
    \begin{enumerate}
        \item Using Problem 3, Section 6-2:
        \begin{align*}
            \sum_{d \ \mid \ n}\mu(d)\phi(d) &= (1 - \phi(p_1))(1 - \phi(p_2)) \cdot \cdot \cdot (1 - \phi(p_r)) \\
            &= (2 - p_1)(2 - p_2) \cdot \cdot \cdot (2 - p_r).
        \end{align*}
        \item Since the function $n\phi(n)$ is multiplicative, it suffices to prove it for $n = p^k$ where $p$ is prime and $k \geq 1$:
        \begin{align*}
            \sum_{d \ \mid \ n}d\phi(d) &= \sum_{i=0}^{k}p^i\phi(p^i) \\
            &= 1+ \sum_{i=1}^{k}p^i\cdot p^{i-1}(p-1) \\
            &= 1 + \frac{p-1}{p}\sum_{i=1}^{k}(p^2)^i \\
            &= 1 + \frac{p-1}{p} \cdot \frac{p^2(p^{2k} - 1)}{p^2 - 1} \\
            &= 1 + \frac{p(p^{2k} - 1)}{p+1} \\
            &= \frac{p^{2k+1} + 1}{p+1}.
        \end{align*}
        \item Since the function $n\phi(n)$ is multiplicative, it suffices to prove it for $n = p^k$ where $p$ is prime and $k \geq 1$:
        \begin{align*}
            \sum_{d \ \mid \ n}\phi(d)/d &= \sum_{i=0}^{k}\phi(p^i)/p^i \\
            &= 1 + \sum_{i=0}^{k}p^{i-1}(p-1)/p^i \\
            &= 1 + \sum_{i=0}^{k}(p-1)/p \\
            &= 1 + \frac{k(p-1)}{p}.
        \end{align*}
    \end{enumerate}
\end{solution}

\begin{exercise}
    Verify the formula $\sum_{d=1}^{n}\phi(d)[n/d] = n(n+1)/2$ for any positive integer $n$. [\textit{Hint:} This is a direct application of Theorem 6-11 and 7-6.] \\
\end{exercise}

\begin{solution}
    Since $\sum_{d \mid n}\phi(d) = n$, then Theorem 6-11 tells us that
    $$\sum_{d=1}^{n}\phi(d)[n/d] = \sum_{d=1}^{n}d = \frac{n(n+1)}{2}.$$ \\
\end{solution}

\begin{exercise}
    If $n$ is a square-free integer, prove that $\sum_{d \mid n}\sigma(d^{k-1})\phi(d) = n^k$ for all integers $k \geq 2$.\\
\end{exercise}

\begin{solution}
    First, notice that $\sigma(n^{k-1})$ is a multiplicative function since given two relatively prime integers $a$ and $b$, we have that $a^{k-1}$ and $b^{k-1}$ are relatively prime as well so $\sigma((ab)^{k-1}) = \sigma(a^{k-1}\cdot b^{k-1}) = \sigma(a^{k-1})\sigma(b^{k-1})$. Hence, $\sigma(n^{k-1})\phi(n)$ is a multiplicative function, so it follows that $\sum_{d \mid n}\sigma(d^{k-1})\phi(d)$ is multiplicative. Thus, it suffices to prove the desired formula in the special case $n = p$ for some prime $p$:
    \begin{align*}
        \sum_{d \ \mid \ n}\sigma(d^{k-1})\phi(d) &= \sigma(1^{k-1})\phi(1) + \sigma(p^{k-1})\phi(p) \\
        &= 1\cdot 1 + \frac{p^k - 1}{p-1}(p-1) \\
        &= 1 + p^k - 1 \\
        &= p^k.\\
    \end{align*}
\end{solution}

\begin{exercise}
    For a square-free integer $n > 1$, show that $\tau(n^2) = n$ if and only if $n = 3$. \\
\end{exercise}

\begin{solution}
    Let $n = p_1p_2\cdot \cdot \cdot p_r$ be a square-free integer such that $\tau(n^2) = n$, then equivalently, this implies that $\tau(p_1^2)\tau(p_2^2)\cdot \cdot \cdot \tau(p_r^2) = n$. But since $\tau(p^2) = 3$ for all prime $p$, the previous equation becomes $3^r = n$. Since $n$ is square-free, then $r$ must be equal to 1, and hence, $n = 3$. Conversely, when $n = 3$, we have
    $$\tau(3^2) = \tau(9) = 3.$$ \\
\end{solution}

\begin{exercise}
    Prove that $3\mid \sigma(3n+2)$ and $4 \mid \sigma(4n+3)$ for any positive integer $n$. \\
\end{exercise}

\begin{solution}
    Let $3n+2 = p_1^{k_1}\cdot \cdot \cdot p_r^{k_r}$ be an integer. Notice that $p_i$ cannot be of the form $3k$ because otherwise, $3n+2$ would be of the form $3k$ which is impossible. Hence, $p_i$ is either of the form $3k+1$ or $3k+2$. If all primes are of the form $3k+1$, then $3n+2$ will be of the form $3k+1$ as well (a product of numbers of the form $3k+1$ is of the form $3k+1$), a contradiction. Hence, there must be a prime $p_i$ of the form $3k+2$. Now, suppose that $k_i$ is even whenever $p_i$ is of the form $3k+2$, then $p_i^{k_i}$ will be of the form $3k+1$ (it suffices to look at $p_i^{k_i}$ modulo 3), in that case, $p_i^{k_i}$ will be of the form $3k+1$ for all $i$, and hence, $3n+2$ will be of the form $3k+1$, a contradiction. Thus, there is a $i$ such that $p_i$ is of the form $3k+2$ and $k_i$ is odd. It follows that
    \begin{align*}
        \sigma(3n+2) &= \sigma(p_i^{k_i})\sigma(N) \\
        &= (1 + p_i + \dots + p_i^{k_i})\sigma(N) \\
        &\equiv (1 - 1 + \dots + (-1)^{k_i})\sigma(N) \\
        &= 0 \pmod 3
    \end{align*}
    where $N = (3n+2)/p_i^{k_i}$. Therefore, $3\mid \sigma(3n+2)$.
    
    Let $4n+3 = p_1^{k_1}\cdot \cdot \cdot p_r^{k_r}$ be an integer. Notice that $p_i$ cannot be of the form $4k$ or $4k+2$ because otherwise, $4n+3$ would be of the form $4k$ or $4k+2$ which is impossible. Hence, $p_i$ is either of the form $4k+1$ or $4k+3$. If all primes are of the form $4k+1$, then $4n+3$ will be of the form $4k+1$ as well (a product of numbers of the form $4k+1$ is of the form $4k+1$), a contradiction. Hence, there must be a prime $p_i$ of the form $4k+3$. Now, suppose that $k_i$ is even whenever $p_i$ is of the form $4k+3$, then $p_i^{k_i}$ will be of the form $4k+1$ (it suffices to look at $p_i^{k_i}$ modulo 4), in that case, $p_i^{k_i}$ will be of the form $4k+1$ for all $i$, and hence, $4n+3$ will be of the form $4k+1$, a contradiction. Thus, there is a $i$ such that $p_i$ is of the form $4k+3$ and $k_i$ is odd. It follows that
    \begin{align*}
        \sigma(4n+3) &= \sigma(p_i^{k_i})\sigma(N) \\
        &= (1 + p_i + \dots + p_i^{k_i})\sigma(N) \\
        &\equiv (1 - 1 + \dots + (-1)^{k_i})\sigma(N) \\
        &= 0 \pmod 4
    \end{align*}
    where $N = (4n+3)/p_i^{k_i}$. Therefore, $4\mid \sigma(4n+3)$.\\
\end{solution}

\begin{exercise}
    \begin{enumerate}
        \item Given $k > 0$, establish that there exists a sequence of $k$ consecutive integers $n + 1$, $n+2$, ..., $n+k$ satisfying
        $$\mu(n+1) = \mu(n+2) = \dots = \mu(n+k) = 0.$$
        [\textit{Hint:} Consider the system of linear congruences
        \begin{align*}
            x &\equiv -1\pmod4, \ x \equiv -2 \pmod 9, \ \dots,\\
            x&\equiv -k \pmod{p_k^2}
        \end{align*}
        where $p_k$ is the $k$th prime.]
        \item Find four consecutive integers for which $\mu(n) = 0$. \\
    \end{enumerate}
\end{exercise}

\begin{solution}
    \begin{enumerate}
        \item Let $k > 0$ be an integer, and consider the system of congruences
        \begin{align*}
            x&\equiv -1 \pmod{4},\\
            x&\equiv -2 \pmod{9},\\
            & \vdots \\
            x&\equiv -k \pmod{p_k^2}
        \end{align*}
        where $p_k$ is the $k$th prime. Since $2^2$, $9^2$, ..., $p_k^2$ are relatively prime, then the system of congruences has a solution by the Chinese Remainder Theorem. Hence, if we let $n$ be a solution of the system, then $2^2 \mid n+1$, $3^2 \mid n+2$, ..., $k^2 \mid n+k$. It follows that
        $$\mu(n+1) = \mu(n+2) = \dots = \mu(n+k) = 0.$$
        \item Let's solve the following system of congruences:
        \begin{align*}
            x&\equiv -1 \pmod{4},\\
            x&\equiv -2 \pmod{9},\\
            x&\equiv -3 \pmod{25},\\
            x&\equiv -4 \pmod{49}.
        \end{align*}
        To do so, we first need to find integers $a$, $b$, $c$, and $d$ such that 
        \begin{align*}
            9\cdot 25 \cdot 49a &\equiv 1 \pmod{4} \\
            4\cdot 25 \cdot 49b &\equiv 1 \pmod{9} \\
            4\cdot 9 \cdot 49c &\equiv 1 \pmod{25} \\
            4\cdot 9 \cdot 25d &\equiv 1 \pmod{49}.
        \end{align*}
        If we simplify the equations, we get
        \begin{align*}
            a &\equiv 1 \pmod{4} \\
            4b &\equiv 1 \pmod{9} \\
            14 c &\equiv 1 \pmod{25} \\
            18d &\equiv 1 \pmod{49}.
        \end{align*}
        Using the Euclidean Algorithm, we get that $a \equiv 1 \pmod 4$, $b \equiv 7 \pmod 9$, $c \equiv 9 \pmod 25$, and $d \equiv 30 \pmod{49}$. Hence, we can take $a = 1$, $b = -2$, $c = 9$, and $d = -19$. Next, we define the integer $x$ up to a multiple of $4\cdot 9 \cdot 25 \cdot 49 = 44100$ as follows:
    \begin{align*}
        x &= - 9\cdot 25 \cdot 49a -2\cdot 4\cdot 25 \cdot 49b -3\cdot 4\cdot 9 \cdot 49c -4\cdot 4\cdot 9 \cdot 25d \\
        &=  -11025 + 19600 - 47628 + 68400\\
        &= 29347.
    \end{align*}
    By construction, $x$ is a solution of the system of congruences we started with (can be verified directly). Hence, 4 divides $x + 1$, 9 divides $x+2$, 25 divides $x + 3$, and 49 divides $x + 4$. Therefore,
    $$\mu(29348) = \mu(29349) = \mu(29350) = \mu(29351)= 0.$$\\
    \end{enumerate}
\end{solution}

\begin{exercise}
    Prove the statements below:
    \begin{enumerate}
        \item An integer $n$ is prime if and only if $\sigma(n) + \phi(n) = n\tau(n)$.
        
        [\textit{Hint:} First derive the relation $\sum_{d \mid n}\sigma(d)\phi(n/d) = n\tau(n)$.]
        \item An integer $n$ is prime if and only if $\phi(n) \mid n-1$ and also $n + 1 \mid \sigma(n)$. [\textit{Hint:} See Problem 11(a), Section 7-2.] \\
    \end{enumerate}
\end{exercise}

\begin{solution}
    \begin{enumerate}
        \item First, it is clear that the equation doesn't hold for $n = 1$. Now, since $\sigma(n) = 1 + S + n$ where $S$ is the sum of the nontrivial proper divisors of $n$, then we can can write the equation $\sigma(n) + \phi(n) = n\tau(n)$ as
        $$1 + S + n + \phi(n) = n\tau(n).$$
        If we subtract by $n$ on both sides, we get
        $$(1 + \phi(n))+ S = n(\tau(n) - 1) = n + ... + n.$$
        Notice that $S$ represents a sum of $\tau(n) - 2$ terms so if we add the term $[1 + \phi(n)]$, then there are as many terms on the LHS than on the RHS. If $n$ is not prime, then there is a nontrivial proper divisor $d$ of $n$. It follows that the sum $S$ is nonempty, and every term in $S$ is strictly smaller than $n$. Since $(1 + \phi(n)) \leq n$, then comparing term by term gives us
        $$(1 + \phi(n)) + S < n + ... + n = n(\tau(n) - 1),$$
        a contradiction. Therefore, $n$ must be prime. 

        Conversely, if $n$ is prime, then
        $$\sigma(n) + \phi(n) = (n + 1) + (n-1) = 2n = n\tau(n).$$
        \item Despite all my efforts, I was not able to solve this exercise. I used Problem 11(a) to deduce that $n$ is square-free. Then, I used this information to write $\sigma(n)$ as the product of $(p_i + 1)$ where $n$ is the product of the $p_i$'s. Then, I tried to find an upperbound for $\sigma(n)$ in terms of $n$, for example, $\sigma(n) \leq 2n$ or $\sigma(n) \leq 3n$. However, using the fact that $\sum 1/p = \infty$, we get that
        $$\frac{\sigma(n)}{n} = \prod_i\left(1 + \frac{1}{p_i}\right)$$
        can be arbitrarily large. Similarly, every other ideas turned out to be false. This exercice has been removed from the next editions of this book, maybe because the proof that the author had in mind was false, and the actual proof (if the statement is true) is much harder. Hopefully, this is the only time this happens.\\
    \end{enumerate}
\end{solution}

\begin{exercise}
    For $n > 2$, establish the inequality $\phi(n^2) + \phi((n+1)^2) \leq 2n^2$.\\
\end{exercise}

\begin{solution}
    First, using Problem 10, Section 7-2, we can write
    $$\phi(n^2) + \phi((n+1)^2) = n\phi(n) + (n+1)\phi(n+1).$$
    One of $n$, $n+1$ is even. If $n$ is even, then at least half of the integers less than $n$ are no prime relative to $n$ so $\phi(n) \leq n/2$. We cannot say much for $n+1$ but at least we know that there are at most $n$ numbers less than $n+1$ that are prime relative to $n+1$ since we don't count $n+1$. Hence:
    $$\phi(n^2) + \phi((n+1)^2) \leq n\left(\frac{n}{2}\right) + (n+1)n = \frac{3}{2}n^2 + n.$$
    Since $n > 2$, then $n \leq n^2/2$ which implies that $3n^2/2 + n \leq 2n^2$. Therefore,
    $$\phi(n^2) + \phi((n+1)^2) \leq 2n^2.$$ 
    Next, suppose that $n + 1$ is even, then $\phi(n+1) \leq (n+1)/2$ and $\phi(n) \leq n-1$. Hence:
    \begin{align*}
        \phi(n^2) + \phi((n+1)^2) &= n\phi(n) + (n+1)\phi(n+1) \\
        &\leq n(n-1) + n\left(\frac{n+1}{2}\right) \\
        &= \frac{3n^2}{2} - \frac{n}{2} \\
        &\leq 2n^2.
    \end{align*}
    Therefore, the inequality holds for all $n > 2$.\\
\end{solution}

\begin{exercise}
    Given an integer $n$, prove that there exists at least one $k$ for which $n \mid \phi(k)$. \\
\end{exercise}

\begin{solution}
    When $n = 1$, every $k$ works. If $n > 1$, write $n = p_1^{k_1} \cdot \cdot \cdot p_r^{k_r}$ and let $k = p_1^{k_1+1} \cdot \cdot \cdot p_r^{k_r+1}$, then
    $$\phi(k) = p^{k_1}(p_1 - 1)\cdot \cdot \cdot p_r^{k_r}(p_r - 1) = n(p_1 - 1)\cdot \cdot \cdot (p_r - 1).$$
    Hence, $n \mid \phi(k)$. \\
\end{solution}

\begin{exercise}
    Show that if $n$ is a product of two twin primes, say $n = p(p+2)$, then $\phi(n)\sigma(n) = (n+1)(n-3)$. \\
\end{exercise}

\begin{solution}
    If $n = p(p+2)$ where $p$ and $p+2$ are prime, then
    \begin{align*}
        \phi(n)\sigma(n) &= \phi(p)\phi(p+2)\sigma(p)\sigma(p+2) \\
        &= (p-1)(p+1)(p+1)(p+3)\\
        &= (p+1)^2(p-1)(p+3) \\
        &= (p^2 + 2p + 1)(p^2 + 2p - 3) \\
        &= (p(p+2) + 1)(p(p+2) - 3) \\
        &= (n+1)(n-3). \\
    \end{align*}
\end{solution}

\begin{exercise}
    Prove that $\sum_{d \mid n}\sigma(d)\phi(n/d) = n\tau(n)$. \\
\end{exercise}

\begin{solution}
    We can write
    $$\sum_{d \mid n}\sigma(d)\phi\left(\frac{n}{d}\right) = \sum_{d \mid n}\sum_{t \mid d}t\phi\left(\frac{n}{d}\right).$$
    Since the set of pairs $(d,t)$ satisfying $d \mid n$ and $t \mid d$ is the same as the set of pairs $(d,t)$ such that $t \mid n$ and $t \mid d$ where $d \mid n$, then we can invert the two sums to get
    $$\sum_{d \mid n}\sum_{t \mid d}t\phi\left(\frac{n}{d}\right) = \sum_{t \mid n}\sum_{t \mid d}t\phi\left(\frac{n}{d}\right) = \sum_{t \mid n}t\left(\sum_{t \mid d}\phi\left(\frac{n}{d}\right)\right)$$
    where, for a fixed $t$, the second sum runs over all $d$ which are divisors of $n$ and divisible by $t$. Let's look at the sum $\sum_{t \mid d}\phi\left(\frac{n}{d}\right)$ in more details. Notice that whenever $d$ is multiple of $t$ and divisor of $n$, then $n/d$ divides $n/t$. Moreover, every divisor of $n/t$ can be written as $n/d$ where $d$ is a multiple of $t$ that divides $n$. Therefore, we can rewrite the sum $\sum_{t \mid d}\phi\left(\frac{n}{d}\right)$ more simply as $\sum_{d \mid \frac{n}{t}}\phi(d)$. Therefore, putting everything together and using Theorem 7-6:
    $$\sum_{d \mid n}\sigma(d)\phi\left(\frac{n}{d}\right) = \sum_{t \mid n}t\left(\sum_{d \mid \frac{n}{t}}\phi(d)\right) = \sum_{t \mid n}t\left(\frac{n}{t}\right) = \sum_{d \mid n}n = n\tau(n).$$ \\
\end{solution}

\begin{exercise}
    If $a_1, a_2, ..., a_{\phi(n)}$ is a reduced set of residues modulo $n$, show that $a_1 + a_2 + ... + a_{\phi(n)} \equiv 0 \pmod n$. \\
\end{exercise}

\begin{solution}
    If $a_1, a_2, ..., a_{\phi(n)}$ is a reduced set of residues modulo $n$, then $a_i$ is prime relative to $n$ for all $i$. Since $a_i$ up to any multiple multiple of $n$ is prime relative to $n$, then the least positive residue of $n$ is prime relative to $n$ for all $i$. Since the $a_i$'s are noncongruent relative to $n$, then their least positive residues are noncongruent relative to $n$. This means that the least positive resiudes of $a_1, a_2, ..., a_{\phi(n)}$ is a list $b_1, ..., b_{\phi(n)}$ of $\phi(n)$ integers (since they are noncongruent) between 1 and $n$ that are relatively prime to $n$. But since there are precisely $\phi(n)$ integers between 1 and $n$ that are relatively prime to $n$ (by definition of $\phi(n)$), then $b_1, ..., b_{\phi(n)}$ of $\phi(n)$ is precisely the list of integers between 1 and $n$ that are relatively prime to $n$. Therefore:
    $$a_1 + a_2 + ... + a_{\phi(n)} \equiv \sum_{\gcd(k,n) = 1} k = \frac{1}{2}n \phi(n) \pmod n.$$
    Since $n > 2$, then $\phi(n)$ is even, and hence,
    $$\frac{1}{2}n \phi(n) = n \cdot \frac{\phi(n)}{2}$$
    is an integer multiple of $n$. It direcly follows that 
    $$a_1 + a_2 + ... + a_{\phi(n)} \equiv 0 \pmod n.$$ 
\end{solution}


\section{An Application to Cryptography}

\begin{exercise}
    Encrypt the message \textit{RETURN HOME} using the Caesar cipher. \\
\end{exercise}

\begin{solution}
    The Caesar cipher is defined by the rule $C \equiv P + 3 \pmod{26}$. Hence, the encrypted message is \textit{UHWXUQ KRPH}. \\
\end{solution}

\begin{exercise}
    If the Ceasar cipher produced \textit{KDSSB ELUWKGDB}, what is the plaintext message ? \\
\end{exercise}

\begin{solution}
    The equation $C \equiv P + 3 \pmod{26}$ is equivalent to $P \equiv C - 3 \pmod{26}$ so it suffices to replace every letter by the letter which is three positions before in the alphabet. Hence, we get that the plaintext message is \textit{HAPPY BIRTHDAY}.\\
\end{solution}

\begin{exercise}
    \begin{enumerate}
        \item A linear cipher is defined by the congruence $C \equiv aP + b \pmod{26}$, where $a$ and $b$ are integers with $\gcd(a,26) = 1$. Show that the corresponding decrypting congruence is $P \equiv a'(C - b) \pmod{26}$, where the integer $a'$ satisfies $aa' \equiv 1 \pmod{26}$.
        \item Using the linear cipher $C \equiv 5P + 11 \pmod{26}$, encrypt the message \textit{NUMBER THEORY IS EASY}.
        \item Decrypt the message \textit{TZSVIW JQBVMIJ HL MVOOVI}, which was produced using the linear cipher $C \equiv 3 P + 7 \pmod{26}$.\\
    \end{enumerate}
\end{exercise}

\begin{solution}
    \begin{enumerate}
        \item The decypting congruence must be a congruence where $P$ is expressed in terms of $C$. If we let $a'$ be the integer such that $aa' \equiv 1 \pmod{26}$ (which must exist since $\gcd(a, 26) = 1$), then multiplying both sides of the congruence $C \equiv aP + b \pmod{26}$ by $a'$ gives us $a'C \equiv P + a' \pmod{26}$. Subtracting both sides by $a'b$ gives us $P \equiv a'(C - b) \pmod{26}$.
        \item Using this linear cipher, the alphabet becomes
        $$PUZEJOTYDINSXCHMRWBGLQVAFK.$$
        Hence, the message \textit{NUMBER THEORY IS EASY} becomes \textit{CLXUJW GYJHWF DB JPBF}.
        \item From the linear cipher $C \equiv 3P + 7 \pmod{26}$ we get the decrypting congruence $P \equiv 9C - 11 \pmod{26}$. Hence, the decrypted message is \textit{MODERN ALGEBRA IS BETTER}. \\
    \end{enumerate}
\end{solution}

\begin{exercise}
    If $n = pq = 274279$ and $\phi(n) = 272376$, find the primes $p$ and $q$.
    
    [\textit{Hint:} Note that
    \begin{align*}
        p+q &= n - \phi(n) + 1, \\
        p - q &= [(p+q)^2 - 4n]^{1/2}.]\\
    \end{align*}
\end{exercise}

\begin{solution}
    Since $\phi(n) = (p-1)(q-1) = n - (p+q) + 1$, then
    $$p+q = n - \phi(n) + 1 = 1904.$$
    Similarly, since
    $$(p-q)^2 = (p+q)^2 - 4pq = (p+q)^2 - 4n,$$
    then
    $$p-q = \sqrt{1904^2 - 4\cdot 274279} = 2\sqrt{632025} = 1590.$$ 
    Solving the system of equations
    $$\begin{cases}
        p+q&= 1904 \\
        p-q&= 1590
    \end{cases}$$
    gives us $p = 1747$ and $q = 157$. \\
\end{solution}

\begin{exercise}
    When the RSA algorithm is based on the key $(n,k) = (3233, 37)$, what is the recovery exponent for the cryptosystem? \\
\end{exercise}

\begin{solution}
    Since $n = 53 \cdot 61$, then $\phi(n) = 52 \cdot 60 = 3120$. The recovery exponent satisfies $37j \equiv 1 \pmod{3120}$. Using the Euclidean Algorithm, we get that $j = 253$. \\
\end{solution}

\begin{exercise}
    Encrypt the message \textit{GOLD MEDAL} using the RSA algorithm with key $(n,k) = (2419, 3)$. \\
\end{exercise}

\begin{solution}
    First, we convert the message into the following blocks of integers:
    $$071 \quad 512 \quad 040 \quad 013 \quad 050 \quad 401 \quad 12.$$
    Next, we need to raise each of these integers to the power of $k$, and reduce to the least positive residue modulo $n$:
    \begin{align*}
        71^3 &\equiv 71 \cdot 203 \equiv 2318 \pmod{2419} \\
        512^3 &\equiv 512 \cdot 892 \equiv 1932 \pmod{2419} \\
        40^3 &\equiv 40 \cdot 1600 \equiv 1106 \pmod{2419}
    \end{align*}
    \textbf{[Starting from now, I will use a calculator to finish this exercice and the next two exercices. It takes too much time.]}
    \begin{align*}
        13^3 &\equiv 2197 \pmod{2419} \\
        50^3 &\equiv 1631 \pmod{2419}\\
        401^3 &\equiv 337 \pmod{2419} \\
        12^3 &\equiv 1728 \pmod{2419}.
    \end{align*}
    Therefore, the ciphertext is
    $$2318 \quad 1932 \quad 1106 \quad 2197 \quad 1631 \quad 0337 \quad 1728.$$ \\
\end{solution}

\begin{exercise}
    The ciphertext message produced by the RSA algorithm with key $(n,k) = (1643, 223)$ is
    $$1451 \quad 0103 \quad 1263 \quad 0560 \quad 0127 \quad 0897.$$
    Determine the original plaintext message. [\textit{Hint:} The recovery exponent is $j = 7$.] \\
\end{exercise}

\begin{solution}
    Since $1643 = 31 \cdot 53$, then $\phi(n) = 30 \cdot 52 = 1560$. By the Euclidean Algorithm, we get that $j$ must be 7. Thus, we need to raise each block of integers to the 7th power and reduce to the least positive residue modulo 1643:
    \begin{align*}
        1451^7 &\equiv 180 \pmod{1643} \\
        103^7 &\equiv 516 \pmod{1643} \\
        1263^7 &\equiv 122 \pmod{1643} \\
        560^7 &\equiv 500 \pmod{1643} \\
        127^7 &\equiv 141 \pmod{1643} \\
        897^7 &\equiv 523 \pmod{1643}. \\
    \end{align*}
    Hence, the integer associated to the plaintext message is 
    $$180516122500141523$$
    which implies that the plaintext message is 
    $$REPLY \ \ NOW$$ \\
\end{solution}

\begin{exercise}
    Decrypt the ciphertext
    $$1037 \quad 0431 \quad 0629 \quad 0690 \quad 0204 \quad 2267 \quad 0595$$
    that was encrypted using the RSA algorithm with key $(n,k) = (2419, 211)$. [\textit{Hint:} The recovery exponent is 11.] \\
\end{exercise}

\begin{solution}
    Since $2419 = 41 \cdot 59$, then $\phi(n) = 40 \cdot 58 = 2320$. By the Euclidean Algorithm, we get that $j$ must be 11. Thus, we need to raise each block of integers to the 11th power and reduce to the least positive residue modulo 2419:
    \begin{align*}
        1037^{11} &\equiv 190 \pmod{2419} \\
        431^{11} &\equiv 512 \pmod{2419} \\
        629^{11} &\equiv 120 \pmod{2419} \\
        690^{11} &\equiv 19 \pmod{2419} \\
        204^{11} &\equiv 81 \pmod{2419} \\
        2267^{11} &\equiv 518 \pmod{2419} \\
        595^{11} &\equiv 20 \pmod{2419}
    \end{align*}
    Hence, the integer associated to the plaintext message is 
    $$19051212001908151820$$
    which implies that the plaintext message is 
    $$SELL \ \ SHORT$$ \\
\end{solution}



\end{document}