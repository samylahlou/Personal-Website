\chapter{Dirichlet : Primes in Arithmetic Progressions}

\section{Gauss' Disquisitiones Arithmeticae}

\td

\begin{enumerate}
    \item Number Theory after Fermat
    \begin{enumerate}
        \item history of Fermat
        \item Some conjectures: primes as the sum of two squares, integers as the sum of four squares, the little theorem, the last theorem.
        \item Euler's contributions: proof of the little theorem and generalization.
        \item Lagrange's contribution: proof of the four square theorem and others + creation of the notion of quadratic forms (main subject of the next chapter).
        \item Number theory became a popular subject again: Legendre, Gauss
    \end{enumerate}
    \item Gauss' Theory of Congruences
    \begin{enumerate}
        \item Introduce the Disquisitiones Arithmeticae 
        \item Introduce modular arithmetic
        \item Introduce the notion of primitive roots
        \item Introduce the notion of indices.
        \item Prove some basic properties which are the one required for the proof of Dirichlet's Theorem.
    \end{enumerate}
    \item The Law of Quadratic Reciprocity 
    \begin{enumerate}
        \item The proof will be in Section 2.3.
    \end{enumerate}
    \item Exercises 
    \begin{enumerate}
        \item Prove that the set of prime numbers is the only set of integers that satisfy the fundamental theorem of arithmetic.
        \item Prove main properties of Congruences
        \item Solve a system of congruences using Gauss' Method.
        \item Prove main properties of the $\phi$ function
        \item Solve the congruence $ax + b \equiv 0 \pmod m$.
    \end{enumerate}
\end{enumerate}

\noindent\td 

\section{Fourier's Theorem}

\td

\begin{enumerate}
    \item Origins of Fourier's Theorem
    \begin{enumerate}
        \item Petit rappel historique sur les séries trigonométriques (D'Alembert, Euler et Daniel Bernoulli).
        \item Brievement raconter la vie de Fourier avant la publication de son livre.
        \item Travaux de Fourier et publication de son livre
        \item Dire que Fourier à gagner une grande place dans la sphère scientifique de Paris après la publication de ses travaux.
        \item Mentionner qu'un petit Galois lui a envoyé ses travaux mais que malheureusement, il faudra attendre avant d'en entendre parler à nouveau.
    \end{enumerate}
    \item Enter Dirichlet
    \begin{enumerate}
        \item Présenter Dirichlet (Fermat's Last Theorem, ...)
        \item Dire qu'il a rencontré Fourier
        \item Dire que Dirichlet est le premier à démontrer le théorème de Fourier
        \item Présenter le papier de 1829
        \item Commencer par exposer les erreurs dans la tentative de démonstration de Cauchy.
    \end{enumerate}
    \item The Proof of Fourier's Theorem
    \begin{enumerate}
        \item Stating and proving as Dirichlet did his lemma.
        \item Stating Fourier's Theorem and starting with the assumptions made on $\phi(x)$ instead on finishing with this.
        \item Mention the Dirichlet Function and with a discussion on the evolution of the concept of a function compared to Euler (assuming that every function could be expanded as a series, etc...)
    \end{enumerate}
    \item Exercises:
    \begin{enumerate}
        \item Derive the formula for the coefficients as Euler did.
        \item Expand a function as a Fourier Series.
        \item Prove the Integral Mean Value Theorem using the Intermediate Value Theorem.
        \item Prove the property of alternating sums used by Dirichlet in his proof.
        \item Proof of $\zeta(2)$ using Fourier Series (From Fermat to Minkowski).
    \end{enumerate}
\end{enumerate}

\noindent\td 

\section{Evaluating Gauss Sums}

\td

\begin{enumerate}
    \item Gauss Sum
    \begin{enumerate}
        \item Introduce Dirichlet' 1835 paper.
        \item History of Gauss sums
    \end{enumerate}
    \item The Proof 
    \begin{enumerate}
        \item Recall the lemma from the proof of Fourier's Theorem.
    \end{enumerate}
    \item A Proof of the Law of Quadratic Reciprocity
    \begin{enumerate}
        \item Apply the formula for Gauss sums to deduce the law of quadratic reciprocity.
    \end{enumerate}
\end{enumerate}

\noindent\td 

\section{Dirichlet's Theorem}

\td 

\begin{enumerate}
    \item Primes in Arithmetic Progressions
    \begin{enumerate}
        \item Présenter le papier de Dirichlet de 1837 
        \item Mentionner que Legendre a été le premier a vraiment parler de ce théorème lorsqu'il a essayé de démontrer la loi des réciprocité quadratiques, sans succés
        \item donner quelques applications du théorème.
        \item Expliquer que le papier est divisé en deux parties, une première partie qui vise à démontrer un cas simple, et une deuxième partie qui démontre le cas général.
    \end{enumerate}
    \item Set up
    \begin{enumerate}
        \item Introduire les objets et les variables utiles à cette démonstration.
        \item Poser $p$ un nombre premier impair
        \item Poser $c$ une racine primitive
        \item Poser $\gamma_n$ l'indice de $n$.
        \item $\omega$ une racine complexe quelconque de l'équation $x^p - 1 = 0$
        \item $\Omega = e^{\frac{2\pi i}{p}}$ tel que $\omega = \Omega^a$.
        \item Finalement, $L_a$ la fonction de $s$ exprimée par le produit Eulerien et la série.
        \item Démontrer qu'il n'y a pas de soucis de concergence ni quoique ce soit de ce style.
    \end{enumerate}
    \item Properties of the $L$ functions
    \begin{enumerate}
        \item Démontrer que $L_0$ a un pôle d'ordre 1 en $s = 1$.
        \item Dériver la formule d'intégrale pour les fonctions $L$ différentes de $L_0$.
        \item En déduire la formule pour $L$ en $s = 1 + \rho$ avec $\rho$ infiniment petit.
    \end{enumerate}
    \item Proving that $L_{\frac{p-1}{2}} \neq 0$ at $s = 1$.
    \begin{enumerate}
        \item \td
    \end{enumerate}
    \item Proving that $L \neq 0$ at $s = 1$ for $L \neq L_0, L_{\frac{p-1}{2}}$
    \begin{enumerate}
        \item Connecter chaque fonction $L$ complexe avec sa conjuguée et donner un argument sur le type de pôle.
    \end{enumerate}
    \item Proving the Special Case
    \begin{enumerate}
        \item Finir la démonstration.
    \end{enumerate}
    \item Exercises
    \begin{enumerate}
        \item \td 
    \end{enumerate}
\end{enumerate}

\noindent\td 

\section{Proof of the General Case}

\td 

\begin{enumerate}
    \item Set up
    \begin{enumerate}
        \item Recontextualiser la situation
        \item Poser les variables importantes: des racines primitives et un système d'indices, des fonctions $L$, ... 
        \item Donner les trois classes de fonctions $L$ et en profiter pour donner une overview de la démonstration.
    \end{enumerate}
    \item Properties of the $L$ functions
    \begin{enumerate}
        \item Démontrer le pôle de la fonction $L_{0,0,0,...}$
        \item Etablir la formule d'intégrale pour les fonctions de la deuxième et troisième classe.
        \item Déduire la continuité et dérivabilité des fonctions de la troisième classe. 
    \end{enumerate}
    \item Proving $L \neq 0$ for $L$ functions of the third class
    \item Proving the General class
    \begin{enumerate}
        \item En déduire le théorème.
    \end{enumerate}
    \item Conclusion
    \begin{enumerate}
        \item Rappeler qu'il manque une chose à la démonstration.
        \item Faire une ouverture sur les formes quadratiques, sujet du prochain chapitre.
    \end{enumerate}
    \item Exercises
    \begin{enumerate}
        \item \td 
    \end{enumerate}
\end{enumerate}