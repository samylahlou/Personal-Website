\section{A Curious Ratio Between Two Series}

\td

\begin{enumerate}
    \item Faire un rappel de la chronologie des articles que Euler a écrit sur ce sujet en vu de conclure le chapitre avec cette section.
    \item Mentionner quelques autres formules exceptionnelles que Euler a trouvé (notamment celle qui se trouvent dans son Introductio).
    \item Dire que malgré tout ces resultats, il y en a un dernier qui mérite d'être étudié car il refera son apparition bientôt.
    \item Article écrit en 1749 et publié en 1768.
    \item Euler étend la notion de convergence.
    \item Cette équation deviendra très importante plus tard (Riemann) + pour les fonctions L.
\end{enumerate}

$$\frac{1 - 2^{-n} + 3^{-n} - 4^{-n} + \dots }{1 - 2^{n-1} + 3^{n-1} - 4^{n-1} + \dots } = - \frac{1\cdot 2 \cdot 3 \cdot \cdot \cdot (n-1)(2^n-1)}{(2^{n-1} - 1) \pi^n}\cos \left(\frac{n \pi}{2}\right)$$
\td 