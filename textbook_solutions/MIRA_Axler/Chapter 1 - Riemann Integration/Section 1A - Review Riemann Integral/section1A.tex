\chapter{Riemann Integration}

\section{Review : Riemann Integral}

\begin{exercise}
    Suppose $f : [a,b] \to \R$ is a bounded function such that 
    $$L(f, P, [a,b]) = U(f, P, [a,b])$$
    for some partition $P$ of $[a,b]$. Prove that $f$ is a constant function on $[a,b]$. \\
\end{exercise}

\begin{solution}
    \\ Let's prove this on the number of subintervals of $[a,b]$ of the partition $P = \{x_0 < x_1 < ... < x_n\}$. For our base case, let $a < b \in \R$, $f : [a,b] \to \R$ be an arbitrary bounded function and $P = \{a, b\}$ be the trivial partition. Suppose that
    $$L(f, P, [a,b]) = U(f, P, [a,b])$$
    Notice that it is equivalent to
    $$\inf_{[a,b]}f = \sup_{[a,b]}f$$
    If we let $c := \sup_{[a,b]}f$, then for all $x \in [a,b]$, we have
    $$c = \inf_{[a,b]}f  \leq f(x) \leq \sup_{[a,b]}f  = c$$
    Hence, $f \equiv c$ on $[a,b]$ which proves the base case. \\
    For the inductive step, suppose that there is a natural number $k$ such that for all $a < b \in \R$ and for all bounded $f : [a,b] \to \R$, then $f$ is constant on $[a,b]$ whenever $L(f, P, [a,b]) = U(f, P, [a,b])$ where $P$ is a partition splitting $[a,b]$ into $k$ subintervals. Let $a < b \in \R$ be real numbers, $f$ be an arbitrary bounded function on $[a,b]$ and $P = \{a = x_0 < x_1 < ... < x_{k+1} = b\}$ be an arbitrary partition splitting $[a,b]$ into $k+1$ subintervals. Suppose that $L(f, P, [a,b]) = U(f, P, [a,b])$ holds. Let's show that $f$ is constant on $[a,b]$. \\ First, consider the functions $f_1 := f|_{[a, x_k]}$ and $f_2 := f|_{[x_k, b]}$ and the partitions $P_1 := \{a = x_0 < x_1 < ... < x_k\}$ and $P_2 := \{x_k < x_{k+1} = b\}$ partitioning $[a, x_k]$ and $[x_k, b]$ respectiviely. Notice that $L(f, P, [a,b]) = U(f, P, [a,b])$ is actually equivalent to $L(f_1, P_1, [a,x_k]) = U(f_1, P_1, [a,x_k])$ and $L(f_2, P_2, [x_k,b]) = U(f_2, P_2, [x_k,b])$. \\
    It follows by our induction hypothesis that there exist constants $c_1$ and $c_2$ in $\R$ such that $f_1 \equiv c_1$ and $f_2 \equiv c_2$ on there respetive domains. By definition of $f_1$ and $f_2$, we get that $f(x) = c_1$ for all $x \in [a, x_k]$ and $f(x) = c_2$ for all $x \in [x_k, b]$. By plugging-in $x = x_k$, we get that $c_1 = c_2$. It follows that $f$ is constant on $[a,b]$. \\
\end{solution}

\begin{exercise}
    Suppose $a \leq s < t \leq b$. Define $f:[a,b] \to \R$ by
    $$f(x) = \begin{cases}
        1 & \text{if } s < x < t, \\
        0 & \text{otherwise}
    \end{cases}$$
    Prove that $f$ is Riemann integrable on $[a,b]$ and that $\int_{a}^{b}f = t-s$. \\
\end{exercise}

\begin{solution}
    \\ Let $\epsilon > 0$ and consider the partition $P_{\epsilon} = \{a < t - \frac{\epsilon}{2} < t + \frac{\epsilon}{2} < s - \frac{\epsilon}{2} < s + \frac{\epsilon}{2} < b\}$. To make sure that $P_{\epsilon}$ is well defined, take $\epsilon$ small enough so that $a < t - \frac{\epsilon}{2}$, $t + \frac{\epsilon}{2} < s - \frac{\epsilon}{2}$ and $s + \frac{\epsilon}{2} < b$, i.e., consider $\epsilon$ to be stricly smaller than $\min(2(t-a), \ s-t, \ 2(b-s))$. Hence:
    \begin{align*}
        U(f, [a,b]) &\leq U(f, P_{\epsilon}, [a,b]) \\
        &= (t - \frac{\epsilon}{2} - a)\sup_{[a, t - \frac{\epsilon}{2}]}f + (t + \frac{\epsilon}{2} - t + \frac{\epsilon}{2})\sup_{[t - \frac{\epsilon}{2}, t + \frac{\epsilon}{2}]}f \\
        &  + (s - \frac{\epsilon}{2} - t - \frac{\epsilon}{2})\sup_{[t + \frac{\epsilon}{2}, s - \frac{\epsilon}{2}]}f + (s + \frac{\epsilon}{2} - s + \frac{\epsilon}{2})\sup_{[s - \frac{\epsilon}{2}, s + \frac{\epsilon}{2}]}f \\
        & + (b - s - \frac{\epsilon}{2})\sup_{[s + \frac{\epsilon}{2}, b]}f \\
        &= (t - \frac{\epsilon}{2} - a)\cdot 0 + \epsilon \cdot 1 + (s - t - \epsilon) \cdot 1 + \epsilon \cdot 1 + (b - s - \frac{\epsilon}{2})\cdot 0 \\
        &= s - t + \epsilon
    \end{align*}
    But $U(f, [a,b])$ don't depend on $\epsilon$ so it follows that $U(f, [a,b]) \leq s- t$. Similarly, by construction of $P_{\epsilon}$, we can prove that $L(f, [a,b]) \geq s- t$ which gives us
    $$s - t \leq L(f, [a,b]) \leq U(f, [a,b]) \leq s - t$$
    which gives us
    $$U(f, [a,b]) = L(f, [a,b]) = s - t$$
    Therefore, $f$ is Riemann integrable and $\int_{a}^{b}f = s-t$. \\
\end{solution}

\begin{exercise}
    Suppose $f : [a,b] \to \R$ is a bounded function. Prove that $f$ is Riemann integrable if and only if for each $\epsilon > 0$, there exists a partition $P$ of $[a,b]$ such that 
    $$U(f, P, [a,b]) - L(f, P, [a,b]) < \epsilon$$
\end{exercise}

\begin{solution}
    \\ ($\implies$) Suppose that $f$ is Riemann integrable, then by definition, $U(f, [a,b]) = L(f, [a,b])$. Let $\epsilon > 0$, then by properties of the infimimum and the supremum, there exist partitions $P_1$ and $P_2$ of $[a,b]$ such that
    $$U(f, P_1, [a,b]) < U(f, [a,b]) + \frac{\epsilon}{2} $$
    and 
    $$L(f, [a,b]) - \frac{\epsilon}{2} < L(f, P_2, [a,b])$$
    consider $P = P_1 \cup P_2$, then:
    \begin{align*}
        U(f,P, [a,b]) - L(f,P, [a,b]) &\leq U(f,P_1, [a,b]) - L(f,P_2, [a,b]) \\
        &< U(f, [a,b]) + \frac{\epsilon}{2} - L(f, [a,b]) + \frac{\epsilon}{2} \\
        &= \epsilon
    \end{align*}
    which proves the first direction of the equivalence. \\
    ( $\Longleftarrow$ ) Suppose that for all $\epsilon$, there exists a partition $P$ of $[a,b]$ such that 
    $$U(f, P, [a,b]) - L(f, P, [a,b]) < \epsilon$$
    Then, since for all partitions $P$ of $[a,b]$ we have $U(f, [a,b]) \leq U(f, P, [a,b])$ and $L(f, P, [a,b]) \leq L(f, [a,b])$, then it follows that for all $\epsilon$, we have
    $$U(f, [a,b]) - L(f, [a,b]) \leq U(f, P, [a,b]) - L(f, P, [a,b]) < \epsilon$$
    for some partition $P$ by our assumption. Since it holds for all $\epsilon > 0$ and since $U(f, [a,b]) - L(f, [a,b])$ is positive, then it follows that $U(f, [a,b]) = L(f, [a,b])$. By definition, this means that $f$ is Riemann integrable. \\
\end{solution}

\begin{exercise}
    Suppose, $f,g : [a,b] \to \R$ are Riemann integrable. Prove that $f+g$ is Riemann integrable on $[a,b]$ and
    $$\int_{a}^{b}(f+g) = \int_{a}^{b}f + \int_{a}^{b}g$$
\end{exercise}

\begin{solution}
    \\ First, consider the following properties of the upper and lower Riemann sums tat we will prove as follows
    \begin{align*}
        \sup_{[x_i, x_{i+1}]}(f+g) &= \sup \{f(x) + g(x) : x\in [x_i, x_{i+1}]\} \\
        &\leq \sup \{f(x) + g(y) : x, y\in [x_i, x_{i+1}]\} \\
        &= \sup(\{f(x) : x\in [x_i, x_{i+1}]\} + \{g(x) : x\in [x_i, x_{i+1}]\}) \\
        &= \sup\{f(x) : x\in [x_i, x_{i+1}]\} + \sup\{g(x) : x\in [x_i, x_{i+1}]\} \\
        &= \sup_{[x_i, x_{i+1}]}f + \sup_{[x_i, x_{i+1}]}g
    \end{align*}
    where $[x_i, x_{i+1}]$ is an arbitrary closed interval inside $[a,b]$. Similarly, we also have the following property for the infinimum:
    $$\inf_{[x_i, x_{i+1}]}(f+g) \geq \inf_{[x_i, x_{i+1}]}f + \inf_{[x_i, x_{i+1}]}g$$
    Thus, given a partition $P$ of $[a, b]$, we have
    \begin{align*}
        U(f+g, P, [a,b]) &= \sum_{i=1}^n (x_{i+1} - x_i)\sup_{[x_i, x_{i+1}]}(f+g) \\
        &\leq \sum_{i=1}^n (x_{i+1} - x_i)(\sup_{[x_i, x_{i+1}]}f + \sup_{[x_i, x_{i+1}]}g) \\
        &= \sum_{i=1}^n (x_{i+1} - x_i)\sup_{[x_i, x_{i+1}]}f + \sum_{i=1}^n (x_{i+1} - x_i)\sup_{[x_i, x_{i+1}]}g \\
        &= U(f, P, [a,b]) + U(g, P, [a,b])
    \end{align*}
    and similarly:
    $$L(f+g, P, [a,b]) \geq L(f, P, [a,b]) + L(g, P, [a,b])$$
    These are the main inequalities we will use to prove the additivity of the Riemann integral. \\
    Let's now prove that $f+g$ is Riemann integrable on $[a,b]$ using the criterion proved in the previous exercise.
    Let $\epsilon > 0$, then by the criterion, there exist partitions $P_f$ and $P_g$ of $[a,b]$ such that 
    $$U(f, P_f, [a,b]) - L(f, P_f, [a,b]) < \frac{\epsilon}{2}$$
    $$U(g, P_g, [a,b]) - L(g, P_g, [a,b]) < \frac{\epsilon}{2}$$
    Consider now $P$ to be the merging of $P_f$ and $P_g$, i.e., let $P = P_f \cup P_g$, then we get
    $$U(f, P, [a,b]) - L(f, P, [a,b]) < \frac{\epsilon}{2}$$
    $$U(g, P, [a,b]) - L(g, P, [a,b]) < \frac{\epsilon}{2}$$
    Thus, by the previous inequalities:
    \begin{align*}
        U(f+g, P, [a,b]) - L(f+g, P, [a,b]) 
        &\leq U(f, P, [a,b]) + U(g, P, [a,b]) \\ 
        & \qquad - L(f, P, [a,b]) - L(g, P, [a,b]) \\
        &= \left[U(f, P, [a,b]) - L(f, P, [a,b])\right] \\
        & \qquad + \left[U(g, P, [a,b]) - L(g, P, [a,b])\right] \\
        &< \frac{\epsilon}{2} + \frac{\epsilon}{2} \\
        &= \epsilon
    \end{align*}
    which proves the Riemann integrability of $f+g$. \\
    Now, let's proves equality between $\int_{a}^{b}(f+g)$ and $\int_{a}^{b}f + \int_{a}^{b}g$. To do so, let $\epsilon > 0$, then there exist partitions $P_1$ and $P_2$ of $[a,b]$ satisfying
    $$U(f, [a,b]) + \frac{\epsilon}{2} > U(f, P_1, [a,b])$$
    and 
    $$U(g, [a,b]) + \frac{\epsilon}{2} > U(g, P_2, [a,b])$$
    If we consider $P = P_1 \cup P_2$, we get
    \begin{align*}
        \int_{a}^{b}(f+g) &= U(f+g, [a,b]) \\
        &\leq U(f+g, P, [a,b]) \\
        &\leq U(f, P, [a,b]) + U(g, P, [a,b]) \\
        &\leq U(f, P_1, [a,b]) + U(g, P_2, [a,b]) \\
        &< U(f, [a,b]) + \frac{\epsilon}{2} + U(g, [a,b]) + \frac{\epsilon}{2}\\
        &= \int_{a}^{b}f + \int_{a}^{b}g + \epsilon
    \end{align*}
    But $\epsilon$ is arbitrary and nothing depends on it so by letting $\epsilon \rightarrow 0$, we get
    \[\int_{a}^{b}(f+g) \leq \int_{a}^{b}f + \int_{a}^{b}g \tag*{(1)}\]
    For the reverse inequality, again, let $\epsilon > 0$, then there are paritions $P_1$ and $P_2$ of $[a,b]$ satisfying 
    $$L(f, [a,b]) < L(f, P_1, [a,b]) + \frac{\epsilon}{2}$$
    and 
    $$L(g, [a,b]) < L(g, P_2, [a,b]) + \frac{\epsilon}{2}$$
    Thus, by letting $P = P_1 \cup P_2$, we get
    \begin{align*}
        \int_{a}^{b}f + \int_{a}^{b}g &= L(f, [a,b]) + L(g, [a,b]) \\
        &< L(f, P_1, [a,b]) + \frac{\epsilon}{2} + L(g, P_2, [a,b]) + \frac{\epsilon}{2} \\
        &= L(f, P, [a,b]) + L(g, P, [a,b]) + \epsilon \\
        &\leq L(f + g, P, [a,b]) + \epsilon \\
        &\leq L(f + g, [a,b]) + \epsilon \\
        &= \int_{a}^{b}(f+g) + \epsilon
    \end{align*}
    Letting $\epsilon \rightarrow 0$ gives us 
    \[\int_{a}^{b}f + \int_{a}^{b}g \leq \int_{a}^{b}(f+g) \tag*{(2)}\]
    Therefore, combining (1) and (2) gives us
    $$\int_{a}^{b}(f+g) = \int_{a}^{b}f + \int_{a}^{b}g$$\\
\end{solution}

\begin{exercise}
    Suppose $f : [a,b] \to \R$ is Riemann integrable. Prove that the function $-f$ is Riemann integrable on $[a,b]$ and 
    $$\int_{a}^{b}(-f) = -\int_{a}^{b}f$$
\end{exercise}

\begin{solution}
    \\ First, notice that for any partition $P$ of $[a,b]$, we have
    \begin{align*}
        -U(f, P, [a,b]) &= - \sum_{i=1}^n (x_{i+1} - x_i) \sup_{[x_i, x_{i+1}]}f \\
        &= \sum_{i=1}^n (x_{i+1} - x_i)\left(-\sup_{[x_i, x_{i+1}]}f\right)  \\
        &= \sum_{i=1}^n (x_{i+1} - x_i)\inf_{[x_i, x_{i+1}]}(-f)  \\
        &= L(-f, P, [a,b])
    \end{align*}
    Similarly, we also have
    $$-L(f, P, [a,b]) = U(-f, P, [a,b])$$
    Therefore, we get that 
    \begin{align*}
        f \text{ is Riemann integrable} &\implies U(f, [a,b]) = L(f, [a,b]) \\
        &\implies -U(f, [a,b]) = -L(f, [a,b]) \\
        &\implies - \inf_P\{U(f, P, [a,b])\} = - \sup_P\{L(f, P, [a,b])\} \\
        &\implies \sup_P\{ - U(f, P, [a,b])\} = \inf_P\{ - L(f, P, [a,b])\} \\
        &\implies \sup_P\{L(-f, P, [a,b])\} = \inf_P\{ U(-f, P, [a,b])\} \\
        &\implies L(-f, [a,b]) = U(-f, [a,b])\\
        &\implies -f \text{ is Riemann integrable}
    \end{align*}
    Hence, by the previous exercise, we get
    $$\int_{a}^{b}f + \int_{a}^{b}(-f) = \int_{a}^{b}(f + (-f)) = \int_{a}^{b}0 = 0$$
    which directly implies 
    $$\int_{a}^{b}(-f) = - \int_{a}^{b}f$$ \\
\end{solution}

\begin{exercise}
    Suppose $f : [a,b] \to \R$ is Riemann integrable. Suppose $g : [a,b] \to \R$ is a function such that $g(x) = f(x)$ for all except finitely many $x \in [a,b]$. Prove that $g$ is Riemann integrable on $[a,b]$ and 
    $$\int_{a}^{b}g = \int_{a}^{b}f$$
\end{exercise}

\begin{solution}
    \\  Let's prove this by induction on the number of the number of elements in the set $\{x \in [a,b] : g(x) \neq f(x)\}$. For the base case, let $g : [a,b] \to \R$ be a function which differs from $f$ at exactly one point $x_0 \in [a,b]$. Consider the function $h = f-g$ defined on $[a,b]$ and notice that $h$ is zero everywhere except for $x = x_0$. Now, consider the following cases, if $x_0 \in (a,b)$, then to prove that $h$ is Riemann integrable, let $\epsilon > 0$, define $\epsilon_0 = \epsilon/4|h_0|$ and consider the partition $P = \{a, x_0 - \epsilon_0, x_0 + \epsilon_0, b\}$. Then, we get
    \begin{align*}
        U(f, P, [a,b]) - L(f, P, [a,b]) &= \left(\sup_{\left[a, x_0 - \epsilon_0 \right]}f - \inf_{\left[a, x_0 - \epsilon_0 \right]}f \right) (x_0 - \epsilon_0 - a) \\ 
        &+ \left(\sup_{\left[x_0 - \epsilon_0, x_0 + \epsilon_0 \right]}f - \inf_{\left[x_0 - \epsilon_0, x_0 + \epsilon_0 \right]}f \right) (x_0 + \epsilon_0 - x_0 + \epsilon_0 )\\
        &+ \left(\sup_{\left[x_0 + \epsilon_0, b\right]}f - \inf_{\left[x_0 + \epsilon_0,b\right]}f \right) (b - x_0 - \epsilon_0)\\ 
        &= 0\cdot (x_0 - \epsilon_0 - a ) + 2|h(x_0)|\epsilon_0 + 0 \cdot (b - x_0 - \epsilon_0) \\
        &= 2|h(x_0)|\frac{\epsilon}{4|h(x_0)|} \\
        &= \frac{\epsilon}{2} \\
        &< \epsilon
    \end{align*}
    Thus, by the criterion proved in exercise 3, $h$ is Riemann integrable. Since $g = f - h$, then $g$ is Riemann integrable as well by exercises 4 and 5. \\
    Now, suppose without loss of generality that $h(x_0)$ is positive, then $L(f, P, [a,b]) = 0$ for any partition $P$ of $[a,b]$. Hence, if we rewrite the last inequality, we get that
    $$U(f, P, [a,b]) < \epsilon$$
    for some partition $P$ and for all $\epsilon > 0$. Hence, for all $\epsilon > 0$, there is a partition $P$ such that
    $$0 = L(f, P, [a,b]) \leq U(f, [a,b]) \leq U(f, P, [a,b]) < \epsilon $$
    It follows that 
    $$\int_{a}^{b}h = U(f, [a,b]) = 0$$
    by letting $\epsilon \rightarrow 0$.
    Thus, by exercise 4 and 5, we get
    $$\int_{a}^{b}f = \int_{a}^{b}(h + g) = \int_{a}^{b}h + \int_{a}^{b}g = \int_{a}^{b}g$$
    which proves the base case when $x_0 \in (a,b)$. When $x_0 \in \{a,b\}$, the proof is the same up to a small modification of the partition $P$ given $\epsilon > 0$. If $x_0 = a$, define $P = \{a, a + \frac{\epsilon}{2|h(x_0)|}, b\}$ and if $x_0 = b$, define $P = \{a, b - \frac{\epsilon}{2|h(x_0)|}, b\}$. \\
    For the inductive hypothesis, suppose that there is a $k \in \N$ such that any function that differs from a Riemann integrable function $f$ at precisely $k$ points is still Riemann integrable and has its integral to be equal to $\int_{a}^{b}f$. Now, let $g : [a,b] \to \R$ be an arbitrary function that differs from $f$ at precisely $k$ points $x_1, x_2, ..., x_{k+1}$. From this, consider the function $g_0 : [a,b] \to \R$ defined by
    $$g_0(x) = \begin{cases}
        f(x) & x = x_{k+1} \\
        g(x) & \text{otherwise}
    \end{cases}$$
    Notice that $g_0$ differs from $f$ at precisely $k$ points. Hence, by the inductive hypothesis, $g_0$ is integrable and its integral is the same as $f$. Moreover, $g$ differs from $g_0$ at precisely one point, hence, by the base case, since $g_0$ is Riemann integrable, then $g$ is Riemann integrable as well and 
    $$\int_{a}^{b}g = \int_{a}^{b}g_0 = \int_{a}^{b}f$$
    which proves our claim by induction.\\
\end{solution}

\begin{exercise}
    Suppose $f : [a,b] \to \R$ is a bounded function. For $n \in \N$, let $P_n$ denote the partition that divides $[a,b]$ into $2^n$ intervals of equal size. Prove that 
    $$L(f, [a,b]) = \lim_{n \rightarrow \infty}L(f, P_n, [a,b]) \text{ and } U(f, [a,b]) = \lim_{n \rightarrow \infty}U(f, P_n, [a,b])$$
\end{exercise}

\begin{solution}
    \\ Let's prove it for the lower Riemann integral. Since $P_{n+1} \subset P_n$ for all $n \in \N$, then $\{L(f, P_n, [a,b])\}_n$ is an increasing sequence that is bounded by $L(f, [a,b])$, thus, it converges to its supremum. Hence, it suffices to prove that $L(f, [a,b]) = \sup_n L(f, P_n, [a,b])$. \\
    Let $\epsilon > 0$, then by properties of the supremum, there exists a partition $P = \{a=x_0, ..., x_m=b\}$ of $[a,b]$ that satisfies 
    $$L(f, P, [a,b]) > L(f, [a,b]) - \frac{\epsilon}{2}$$
    Let $k \in \Iint{1}{m-1}$, then there are a dyadic numbers $a_k / 2^{n_k}$ and $b_k / 2^{n_k}$ that satisfies the following properties. First, $a_k / 2^{n_k}$ is strictly between $x_k$ and $x_k$ minus half the distance between $x_k$ and $x_{k-1}$. Similarly, $b_k / 2^{n_k}$ is stricly between $x_k$ and $x_k$ plus half the distance between $x_k$ and $x_{k+1}$. This condition is made to ensure that $$\frac{b_{k-1}}{2^{n_{k-1}}} < \frac{a_k}{2^{n_k}} < x_k < \frac{b_k}{2^{n_k}} < \frac{a_{k+1}}{2^{n_{k+1}}}$$
    Moreover, the dyadic numbers also satisfy
    $$x_k - \frac{a_k}{2^{n_k}} < \frac{\epsilon}{4M(m-1)}$$
    and 
    $$\frac{b_k}{2^{n_k}} - x_k < \frac{\epsilon}{4M(m-1)}$$
    It directly follows that
    $$\frac{b_k}{2^{n_k}} - \frac{a_k}{2^{n_k}} < \frac{\epsilon}{2M(m-1)}$$
    From this, define $N$ to be the maximum of the $n_k$'s and notice that we can rewrite 
    $$\frac{\epsilon}{2} = \sum_{k=1}^{m-1}M\frac{\epsilon}{2M(m-1)}$$
    Hence, combining this with the previous inequality gives us
    \begin{align*}
        \frac{\epsilon}{2} &> \sum_{k=1}^{m-1}M\left(\frac{b_k}{2^{n_k}} - \frac{a_k}{2^{n_k}}\right)
    \end{align*}
    But notice that right hand side is an upper bound for the lower Riemann sum with the partition $P_N \cup P$ where the subintervals are precisely the ones between the dyadic approximations of the $x_i$'s. Hence, since we can split $L(f, P_N \cup P, [a,b])$ into two sums, one that iterates over the subintervals of $P_N$ that are not contained between the dyadic approximations of some $x_i$ and another sum that iterates over the subintervals of $P_N \cup P$ that are contained between the dyadic approximations of some $x_i$, then we get the following upper bound:
    $$L(f, P_N, [a,b]) + \frac{\epsilon}{2} > L(f, P_N\cup P, [a,b])$$
    which implies 
    \begin{align*}
        L(f, P_N, [a,b]) + \frac{\epsilon}{2} &> L(f, P_N\cup P, [a,b]) \\
        &\geq L(f, P, [a,b]) \\
        &> L(f, [a,b]) - \frac{\epsilon}{2}
    \end{align*}
    giving us
    $$L(f, P_N, [a,b]) > L(f, [a,b]) - \epsilon$$
    Thus, the sequence $\{L(f, P_n, [a,b])\}_n$ gets arbitrarily close to $L(f, [a,b])$. But $L(f, [a,b])$ is an upper bound for this sequence. It follows that $L(f, [a,b]) = \sup_n L(f, P_n, [a,b])$. Therefore, 
    $$L(f, [a,b]) = \lim_{n \rightarrow \infty}L(f, P_n, [a,b])$$
    The proof for the upper Riemann integral is the same up to some small readjustments. \\
\end{solution}

\begin{exercise}
    Suppose $f:[a,b] \to \R$ is Riemann integrable. Prove that
    $$\int_{a}^{b}f = \lim_{n \rightarrow \infty}\frac{b-a}{n}\sum_{j=1}^{n}f\left(a+\frac{j(b-a)}{n}\right).$$
\end{exercise}

\begin{solution}
    \\ In this solution, for all $n \in \N$, I will denote by $P_n$ the partition of $[a,b]$ that divides the interval into $n$ equaly spaced subintervals. Let's use the definition of the limit for sequences to prove the claim. \\
    Let $\epsilon > 0$, then there exist partitions $P^{(1)}$ and $P^{(2)}$ of $[a,b]$ satisfying 
    $$L(f, P^{(1)}, [a,b]) > L(f, [a,b]) - \frac{\epsilon}{2}$$
    $$U(f, [a,b]) + \frac{\epsilon}{2} > U(f, P^{(2)}, [a,b])$$
    If we consider the merging of the partitions $P = P^{(1)} \cup P^{(2)} = \{a=x_0, x_1, ..., x_m = b\}$, then the previous inequalities still hold even if we replace $P^{(1)}$ and $P^{(2)}$ by $P$:
    $$L(f, P, [a,b]) > L(f, [a,b]) - \frac{\epsilon}{2}$$
    $$U(f, [a,b]) + \frac{\epsilon}{2} > U(f, P, [a,b])$$
    By the Archimedean Property in $\R$, there is a $N \in \N$ such that 
    $$\frac{1}{N} < \frac{1}{b-a}\cdot \frac{\epsilon}{4M(m-1)}$$
    Moreover, to make the rest of the proof simpler, make $N$ large enough so that $(b-a)/N$ is stricly less than the maximum size of the subintervals in $P$. Let $n \geq N$, let's first prove that
    $$L(f, P\cup P_n, [a,b]) \leq L(f, P_n, [a,b]) + 2M(m-1)\frac{b-a}{n }$$
    To do so, since $P_n$ is a partition of $[a,b]$, then any $x_i$ is going to be in a subinterval of $P_n$ of the form $[y_{i_1}, y_{i_2}]$ where $y_{i_1} = a + j\frac{b-a}{n}$ and $y_{i_2} = y_{i_1} + \frac{b-a}{n}$:
    $$y_{i_1} \leq x_i \leq y_{i_2}$$
    By our assumption on $N$, there are no $x_j$ between $y_{i_1}$ and $x_i$ or $x_i$ and $y_{i_2}$. Hence, the lower Riemann sum corresponding to the partition $P\cup P_n$ contains the following terms:
    $$(x_i - y_{i_1})\inf_{[y_{i_1}, x_i]} f + (y_{i_2} - x_i)\inf_{[x_i, y_{i_2}]} f$$
    for all $i \in \Iint{1}{m-2}$. But notice that we can find the following upper bound:
    \begin{align*}
        (x_i - y_{i_1})\inf_{[y_{i_1}, x_i]} f + (y_{i_2} - x_i)\inf_{[x_i, y_{i_2}]} f &\leq (x_i - y_{i_1})M + (y_{i_2} - x_i)M \\
        &= M(y_{i_2} - y_{i_1}) \\
        &= M\frac{b-a}{n}
    \end{align*}
    Summing over all $i$'s gives us
    $$\sum_{i=1}^{m-1}\left[(x_i - y_{i_1})\inf_{[y_{i_1}, x_i]} f + (y_{i_2} - x_i)\inf_{[x_i, y_{i_2}]} f \right] \leq  \sum_{i=1}^{m-1} M\frac{b-a}{n} = M(m-1)\frac{b-a}{n}$$
    Thus, from the $n + (m-1)$ terms of the lower Riemann sum associated with the partition $P \cup P_n$, we can bound above $2(m-1)$ of the terms by $M(m-1)\frac{b-a}{n}$. What it means is that $L(f, P\cup P_n, [a,b])$ can be bounded above by $M(m-1)\frac{b-a}{n}$ plus $L(f, P_n, [a,b])$ without the $m-1$ subintervals containing the $x_i$'s. But each subinterval in $L(f, P_n, [a,b])$ is of the form $\inf_{[y_j, y_{j+1}]}f \frac{b-a}{n}$ so is greater than $-M\frac{b-a}{n}$. Thus, if we denote by $m_k$ the infimum of $f$ on the $k$th subinterval of $P_n$, we get:
    \begin{align*}
        L(f, P\cup P_n, [a,b]) &\leq \sum_{i=1}^{n - (m-1)} \left[ m_{k_i}\frac{b-a}{n} \right] +  M(m-1)\frac{b-a}{n} \\
        &= \sum_{i=1}^{n - (m-1)} \left[ m_{k_i}\frac{b-a}{n} \right] + \sum_{j=1}^{m-1}\left[-M\frac{b-a}{n}\right] + 2M(m-1)\frac{b-a}{n} \\
        &\leq \sum_{i=1}^{n - (m-1)} \left[ m_{k_i}\frac{b-a}{n} \right] + \sum_{j=1}^{m-1}\left[m_{k_j'}\frac{b-a}{n}\right] + 2M(m-1)\frac{b-a}{n} \\
        &= \sum_{i=1}^{n} \left[ m_k\frac{b-a}{n} \right] + 2M(m-1)\frac{b-a}{n} \\
        &= L(f, P_n, [a,b]) + 2M(m-1)\frac{b-a}{n}
    \end{align*}
    which is the desired inequality. Similarly, we can prove an analoguous inequality for the upper Riemann sum:
    $$U(f, P \cup P_n, [a,b]) \geq U(f, P_n, [a,b]) - 2M(m-1)\frac{b-a}{n}$$
    From these inequalities, we get the following:
    \begin{align*}
        \frac{b-a}{n}\sum_{i=1}^{n}f\left(a + i\frac{b-a}{n}\right) + \frac{\epsilon}{2} &\geq \sum_{i=1}^{n}\left[m_i \frac{b-a}{n}\right] + 2M(m-1)\frac{b-a}{n} \\
        &= L(f, P_n, [a,b]) + 2M(m-1)\frac{b-a}{n} \\
        &\geq L(f, P \cup P_n, [a,b]) \\
        &\geq L(f, P, [a,b]) \\
        &> L(f, [a,b]) - \frac{\epsilon}{2}
    \end{align*}
    which implies
    \[\int_{a}^{b}f - \frac{b-a}{n}\sum_{i=1}^{n}f\left(a + i\frac{b-a}{n}\right) < \epsilon \tag*{(1)} \]
    Similarly, with upper Riemann sums, we get
    \[\frac{b-a}{n}\sum_{i=1}^{n}f\left(a + i\frac{b-a}{n}\right) - \int_{a}^{b}f < \epsilon \tag*{(2)} \]
    Combining (1) and (2) gives us
    $$\left| \frac{b-a}{n}\sum_{i=1}^{n}f\left(a + i\frac{b-a}{n}\right) - \int_{a}^{b}f \right| < \epsilon$$
    Therefore, by definition of the limit of a sequence, we have
    $$\lim_{n \rightarrow \infty} \frac{b-a}{n}\sum_{i=1}^{n}f\left(a + i\frac{b-a}{n}\right) = \int_{a}^{b}f$$
    which proves our claim. \\
\end{solution}

\begin{exercise}
    Suppose $f : [a,b] \to \R$ is Riemann integrable. Prove that if $c,d \in \R$ and $a \leq c < d \leq b$, then $f$ is Riemann integrable on $[c,d]$. \\ $[$\textit{To say that $f$ is Riemann integrable on $[c,d]$ means that $f$ with its domain restricted to $[c,d]$ is Riemann integrable.}$]$ \\
\end{exercise}

\begin{solution}
    \\ In this solution, we will denote by $f|_{[c,d]}$ the restriction of $f$ to $[c,d]$. Let's prove this using the criterion proven in exercise 3. Let $\epsilon > 0$, then by Riemann integrability of $f$, there exists a partition $P$ such that
    $$U(f, P, [a,b]) - L(f, P, [a,b]) < \epsilon$$
    Consider now the partition $P' = P \cup \{c, d\}$, then the previous still holds if we replace $P$ by $P'$ since $P'$ is a refinement of $P$:
    $$U(f, P', [a,b]) - L(f, P', [a,b]) < \epsilon$$
    If we write $P'$ as $\{a= x_0, x_1, ..., x_n = b\}$, then there must exist integers $i < j \in \Iint{0}{n}$ such that $x_i = c$ and $x_j = d$. Define now the partition $P_0 = \{c = x_i, x_{i+1}, ..., x_j = d\}$ and notice that
    \begin{align*}
        U(f|_{[c,d]}, P_0, [c,d]) - &L(f|_{[c,d]}, P_0, [c,d]) \\ &= \sum_{k=i}^{j-1}\left(\sup_{[x_i, x_{i+1}]}f|_{[c,d]} - \inf_{[x_i, x_{i+1}]}f|_{[c,d]}\right)(x_{i+1} - x_i) \\
        &= \sum_{k=i}^{j-1}\left(\sup_{[x_i, x_{i+1}]}f - \inf_{[x_i, x_{i+1}]}f\right)(x_{i+1} - x_i) \\
        &\leq \sum_{k=1}^{n-1}\left(\sup_{[x_i, x_{i+1}]}f - \inf_{[x_i, x_{i+1}]}f\right)(x_{i+1} - x_i) \\
        &= U(f, P', [a,b]) - L(f, P', [a,b]) \\
        &< \epsilon
    \end{align*} 
    which proves that $f$ is Riemann integrable on $[c,d]$. \\
\end{solution}

\begin{exercise}
    Suppose $f : [a,b] \to \R$ is a bounded function and $c \in (a,b)$. Prove that $f$ is Riemann integrable on $[a,b]$ if and only if $f$ is Riemann integrable on $[a,c]$ and $f$ is Riemann integrable on $[c,b]$. Furthermore, prove that if these conditions hold, then
    $$\int_{a}^{b}f = \int_{a}^{c}f + \int_{c}^{b}f.$$
\end{exercise}

\begin{solution}
    \\ Before proving this, let's show that
    $$U(f, [a,b]) = U(f, [a,c]) + U(f, [c,d])$$
    and 
    $$L(f, [a,b]) = L(f, [a,c]) + L(f, [c,d])$$
    hold. To do so, we will use properties of the supremum and infimum. Let $\epsilon > 0$, then there exists a partition $P$ of $[a,b]$ such that $U(f, P, [a,b]) < U(f, [a,b]) + \epsilon$. But if we consider $P \cup \{c\} = \{a = x_0, ..., x_j = c, ..., x_n = b,\}$ instead of $P$, we can split it into two partitions $P_1 = \{x_0, ..., x_j\}$ and $P_2 = \{x_j, ..., x_n\}$ of $[a,c]$ and $[c,b]$ respectively. Hence:
    \begin{align*}
        U(f, [a,b]) + \epsilon &> U(f, P, [a,b]) \\
        &\geq U(f, P\cup \{c\}, [a,b]) \\
        &= \sum_{i=1}^{n}(x_i - x_{i-1})\inf_{[x_{i-1}, x_i]}f\\
        &= \sum_{i=1}^{j}(x_i - x_{i-1})\inf_{[x_{i-1}, x_i]}f + \sum_{i=j+1}^{n}(x_i - x_{i-1})\inf_{[x_{i-1}, x_i]}f \\
        &= U(f, P_1, [a,c]) + U(f, P_2, [c,b]) \\
        &\geq U(f, [a,c]) + U(f, [c,b])
    \end{align*}
    In short:
    $$U(f, [a,c]) + U(f, [c,b]) \leq U(f, [a,b]) + \epsilon$$
    But nothing here depends on $\epsilon$ so if just take $\epsilon \rightarrow 0$, we get
    $$U(f, [a,c]) + U(f, [c,b]) \leq U(f, [a,b])$$
    Similarly, for any $\epsilon > 0$, there exist partitions $P_1$ and $P_2$ of $[a,c]$ and $[c,b]$ respectively such that $U(f, P_1, [a,c]) < U(f, [a,c]) + \frac{\epsilon}{2}$ and $U(f, P_2, [c,b]) < U(f, [c,b]) + \frac{\epsilon}{2}$. Hence, if we consider the partition $P = P_1 \cup P_2 =  \{x_0, ..., x_j = c, ..., x_n\}$ of $[a,b]$, we get
    \begin{align*}
        U(f, [a,c]) + U(f, [c,b]) + \epsilon &> U(f, P_1, [a,c]) + U(f, P_2, [c,b]) \\
        &= \sum_{i=1}^{j}(x_i - x_{i-1})\inf_{[x_{i-1}, x_i]}f + \sum_{i=j+1}^{n}(x_i - x_{i-1})\inf_{[x_{i-1}, x_i]}f \\
        &= \sum_{i=1}^{n}(x_i - x_{i-1})\inf_{[x_{i-1}, x_i]}f\\
        &= U(f, P, [a,b]) \\
        &\geq U(f, [a,b])
    \end{align*}
    In short:
    $$U(f, [a,b]) \leq U(f, [a,c]) + U(f, [c,b]) + \epsilon$$
    But nothing here depends on $\epsilon$ so if just take $\epsilon \rightarrow 0$, we get
    $$U(f, [a,b]) \leq U(f, [a,c]) + U(f, [c,b])$$
    It follows that
    $$U(f, [a,b]) = U(f, [a,c]) + U(f, [c,b])$$
    The proof for the lower Riemann integral is the same up to some small modifications. Now that we proved these results, the rest will follow easily.\\
    For the equivalence that we need to prove, notice that the forward implication follows from the previous exercise. For the reverse implication, suppose that $f$ is both Riemann integrable on $[a,c]$ and $[c,b]$, then by definition, we have
    $$U(f, [a,c]) = L(f, [a,c])$$
    and 
    $$U(f, [c,b]) = L(f, [c,b])$$
    Adding the two equations gives us
    $$U(f, [a,c]) + U(f, [c,b]) = L(f, [a,c]) + L(f, [c,b])$$
    which is equivalent to
    $$U(f, [a,b]) = L(f, [a,b])$$
    Thus, $f$ is Riemann integrable on $[a,b]$. \\
    Now, suppose that $f$ is Riemann integrable on $[a,b]$ and consequently, on $[a,c]$ and $[c, b]$ as well, then:
    $$\int_{a}^{b}f = U(f, [a,b]) = U(f, [a,c]) + U(f, [c,b]) = \int_{a}^{c}f + \int_{c}^{b}f$$
    which proves our claim. \\
\end{solution}

\begin{exercise}
    Suppose $f:[a,b] \to \R$ is Riemann integrable. Define $F:[a,b] \to \R$ by
    $$F(t) = \begin{cases}
        0 & \text{if }t=a \\
        \int_{a}^{t}f & \text{if }t \in (a, b]
    \end{cases}$$
    Prove that $F$ is continuous on $[a,b]$. \\
\end{exercise}

\begin{solution}
    \\ First, let $m$ be the infimum of $f$ on $[a,b]$ and $M$ be the supremum of $f$ on $[a,b]$. Define $A$ to be the maximum between $|m|$ and $|M|$. Now, let $x \in [a,b]$ and $(x_n)_n$ a sequence in $[a,b]$ that converges to $x$. For all $n \in \N$, if $x < x_n$ we have
    $$(x_n - x)\inf_{[x, x_n]}f \leq \int_{x}^{x_n}f \leq (x_n - x)\sup_{[x, x_n]}f$$
    But by properties of the infimum and supremum, we have
    $$m(x_n - x) \leq (x_n - x)\inf_{[a,b]}f \leq \int_{x}^{x_n}f \leq (x_n - x)\sup_{[a,b]}f \leq M(x_n - x)$$
    By definition of $A$, we have
    $$-A(x_n - x) \leq m(x_n - x) \leq \int_{x}^{x_n}f \leq M(x_n - x) \leq A(x_n - x)$$
    By the previous exercise and by definition of $F$, we have
    $$F(x_n) - F(x) = \int_{a}^{x_n}f - \int_{a}^{x}f = \int_{x}^{x_n}f$$
    Thus, plugging this in our inequality gives us 
    $$-A(x_n - x) \leq F(x_n) - F(x) \leq A(x_n - x)$$
    which is equivalent to
    $$|F(x_n) - F(x)| \leq A(x_n - x)$$
    We assumed here that $x < x_n$ but we actually get the exact same result if $x = x_n$ or if $x > x_n$. Thus, since our last inequality holds for all $n \in \N$, then by the Squeeze Theorem:
    $$\lim_{n \rightarrow \infty}F(x_n) = F(x)$$
    Since it holds for any sequence $(x_n)_n$ converging to $x$, then by the Sequential Characterization of Continuity, we get that $F$ is continuous at $x$. Since it holds for all $x \in [a,b]$, then $F$ is continuous on $[a,b]$. \\
\end{solution}

 \begin{exercise}
    Suppose $f:[a,b] \to \R$ is Riemann integrable. Prove that $|f|$ is Riemann integrable and that
    $$\left| \int_{a}^{b}f \right| \leq \int_{a}^{b}|f|.$$
 \end{exercise}

 \begin{solution}
    \\ First, let's prove that $|f|$ is Riemann integrable. To do so, let's use the criterion proven in exercise 3. Let $\epsilon > 0$, then there exists a partition $P = \{x_0, ..., x_n\}$ of $[a,b]$ such that
    $$U(f, P, [a,b]) - L(f, P, [a,b]) < \epsilon$$
    Let $k \in \Iint{1}{n}$, define
    $$m_k = \inf_{[x_{k-1}, x_k]}f \qquad \qquad M_k = \sup_{[x_{k-1}, x_k]}f$$
    $$m_k' = \inf_{[x_{k-1}, x_k]}|f| \qquad \qquad M_k' = \sup_{[x_{k-1}, x_k]}|f|$$
    Let's show that $M_k' - m_k' \leq M_k - m_k$. \\
    If $M_k \leq 0$ or $m_k \geq 0$, it is trivial. Suppose that $M_K \geq 0$ and $m_k \leq 0$, then for all $x \in [x_{k-1}, x_k]$:
    \begin{align*}
            m_k \leq f(x) &\implies m_k \leq f(x) + M_k \\
            &\implies m_k - M_k \leq f(x) \\
            &\implies -(M_k - m_k) \leq f(x)
    \end{align*}
    and 
    \begin{align*}
        f(x) \leq M_k &\implies f(x) + m_k \leq M_k \\
        &\implies f(x) \leq M_k - m_k
    \end{align*}
    Putting the last two inequalities together gives us
    $$-(M_k - m_k) \leq f(x) \leq M_k - m_k$$
    which is equivalent to
    $$|f(x)| \leq M_k - m_k$$
    But it holds for all $x \in [x_{k-1}, x_k]$, so we get
    $$M_k' - m_k' \leq M_k' \leq M_k - m_k$$
    which is the desired inequality. \\
    Now, simply notice that
    \begin{align*}
        U(|f|, P, [a,b]) - L(|f|, P, [a,b]) &= \sum_{k=1}^{n}(M_k' - m_k')(x_k - x_{k-1}) \\
        &\leq \sum_{k=1}^{n}(M_k - m_k)(x_k - x_{k-1}) \\
        &= U(f, P, [a,b]) - L(f, P, [a,b]) \\
        &< \epsilon
    \end{align*}
    which proves that $|f|$ is Riemann integrable as well. \\
    To prove the triangle inequality, I find it easier to first prove that the Riemann integral is monotone. To do so, let $g_1,g_2 : [a,b] \to \R$ be two Riemann integrable functions such that $g_1 \leq g_2$, then if we define $h = g_2 - g_1 \geq 0$, by exercises 4 and 5, we know that $h$ is Riemann integrable as well and that
    $$\int_{a}^{b}h = \int_{a}^{b}g_2 - \int_{a}^{b}g_1$$
    Moreover, since $h$ is positive on $[a,b]$, then $\inf_{[a,b]}h$ must be positive as well. It follows that
    $$0 \leq (b-a)\inf_{[a,b]}h \leq \int_{a}^{b}h = \int_{a}^{b}g_2 - \int_{a}^{b}g_1$$
    which directly implies
    $$\int_{a}^{b}g_1 \leq \int_{a}^{b}g_2$$
    Hence, the Riemann integral is monotone. Therefore:
    $$-|f| \leq f \leq |f|$$
    implies by monotonicity and by exercise 5 that
    $$-\int_{a}^{b}|f| \leq \int_{a}^{b}f \leq \int_{a}^{b}|f|$$
    which is equivalent to
    $$\left| \int_{a}^{b}f \right| \leq \int_{a}^{b}|f|$$
    This proves the triangle inequality for the Riemann integral. \\
 \end{solution}

 \begin{exercise}
    Suppose $f:[a,b] \to \R$ is an increasing function, meaning that $c,d \in [a,b]$ with $c < d$ implies $f(c) \leq f(d)$. Prove that $f$ is Riemann integrable on $[a,b]$.\\
 \end{exercise}

 \begin{solution}
    \\ Let's prove that $f$ is Riemann integrable using the criterion proven in exercise 3. Let $\epsilon > 0$, then by the Archimedean property in $\R$, there exists a $n \in \N$ such that 
    $$\frac{(b-a)(f(b) - f(a))}{n} < \epsilon$$
    Now, consider $P = \{x_0, ..., x_n\}$ to be the the partition of $[a,b]$ that divides the interval into $n$ subintervals of equal size. For all $k \in \Iint{1}{n}$, if we define
    $$m_k = \inf_{[x_{k-1}, x_k]}f \qquad \qquad M_k = \sup_{[x_{k-1}, x_k]}f$$
    then we get 
    $$m_k = f\left(a + (k-1)\frac{b-a}{n}\right) \quad \qquad M_k = f\left(a + k\frac{b-a}{n}\right)$$
    since $f$ is increasing. Hence:
    \begin{align*}
        U(f, P, [a,b]) - &L(f, P, [a,b]) \\
        &= \sum_{k=1}^{n}(M_k - m_k)(x_k - x_{k-1}) \\
        &= \frac{b-a}{n}\sum_{k=1}^{n}\left[f\left(a + k\frac{b-a}{n}\right) - f\left(a + (k-1)\frac{b-a}{n}\right)\right] \\
        &= \frac{b-a}{n}(f(b) - f(a)) \\
        &< \epsilon
    \end{align*}
    Therefore, $f$ is Riemann integrable.\\
 \end{solution}

 \begin{exercise}
    Suppose $f_1, f_2, ...$ is a sequence of Riemann integrable functions on $[a,b]$ such that $f_1, f_2, ...$ converges uniformly on $[a,b]$ to a function $f : [a,b] \to \R$. Prove that $f$ is Riemann integrable and 
    $$\int_{a}^{b}f = \lim_{n \rightarrow \infty}\int_{a}^{b}f_n$$
 \end{exercise}

 \begin{solution}
    \\ First, let's show that $f$ is Riemann integrable using the criterion proven in exercise 3. Let $\epsilon > 0$, then by uniform convergence, there is a $N \in \N$ such that
    $$|f(x) - f_N(x)| < \frac{\epsilon}{4(b-a)}$$
    for all $x \in [a,b]$. Since $f_N$ is Riemann integrable, then there is a partition $P = \{x_0, ..., x_n\}$ such that 
    $$U(f_N, P, [a,b]) - L(f_N, P, [a,b]) < \frac{\epsilon}{2}$$
    Let $k \in \Iint{1}{n}$ and define
    $$m_k = \inf_{[x_{k-1}, x_k]}f \qquad \qquad M_k = \sup_{[x_{k-1}, x_k]}f$$
    $$m_k^N = \inf_{[x_{k-1}, x_k]}f_N \qquad \qquad M_k^N = \sup_{[x_{k-1}, x_k]}f_N$$
    Let $x \in [x_{k-1}, x_k]$, then
    \begin{align*}
        |f(x) - f_N(x)| < \frac{\epsilon}{4(b-a)} &\implies f(x) - f_N(x) < \frac{\epsilon}{4(b-a)} \\
        &\implies f(x) < \frac{\epsilon}{4(b-a)} + f_N(x)\\
        &\implies f(x) \leq \frac{\epsilon}{4(b-a)} + M_k^N\\
    \end{align*}
    However, since the last inequality holds for all $x \in [x_{k-1}, x_k]$ and only the left hand side depends on $x$, then it follows that
    \[M_k \leq \frac{\epsilon}{4(b-a)} + M_k^N \tag*{(1)}\]
    Similarly, 
    \begin{align*}
        |f(x) - f_N(x)| < \frac{\epsilon}{4(b-a)} &\implies f_N(x) - f(x) < \frac{\epsilon}{4(b-a)} \\
        &\implies f_N(x) < \frac{\epsilon}{4(b-a)} + f(x)\\
        &\implies m_k^N \leq \frac{\epsilon}{4(b-a)} + f(x)\\
        &\implies m_k^N - \frac{\epsilon}{4(b-a)} \leq f(x)\\
    \end{align*}
    However, since the last inequality holds for all $x \in [x_{k-1}, x_k]$ and only the right hand side depends on $x$, then it follows that
    $$m_k^N - \frac{\epsilon}{4(b-a)} \leq m_k$$
    which implies
    \[-m_k \leq -m_k^N + \frac{\epsilon}{4(b-a)} \tag*{(2)}\]
    Adding (1) and (2) together gives us
    $$M_k - m_k \leq M_k^N - m_k^n + \frac{\epsilon}{2(b-a)}$$
    for all $k \in \Iint{1}{n}$. Thus:
    \begin{align*}
        U(f, P, [a,b]) - &L(f, P, [a,b]) \\
        &= \sum_{k=1}^{n}(M_k - m_k)(x_k - x_{k-1}) \\
        &\leq \sum_{k=1}^{n}\left[(M_k^N - m_k^N) + \frac{\epsilon}{2(b-a)}\right](x_k - x_{k-1}) \\
        &= \sum_{k=1}^{n}(M_k^N - m_k^N)(x_k - x_{k-1}) + \frac{\epsilon}{2(b-a)}\sum_{k=1}^{n}(x_k - x_{k-1}) \\
        &= U(f_N, P, [a,b]) - L(f_N, P, [a,b]) + \frac{\epsilon}{2(b-a)}(b-a) \\
        &< \frac{\epsilon}{2} + \frac{\epsilon}{2} \\
        &= \epsilon
    \end{align*}
    which proves that $f$ is Riemann integrable. \\
    Now, let's prove that $\int_{a}^{b}f_n \rightarrow \int_{a}^{b}f$ as $n \rightarrow \infty$ using the limit definition. Let $\epsilon > 0$, by uniform convergence, there is a $N \in \N$ such that for all $n \geq N$ and $x \in [a,b]$
    $$\left| f(x) - f_n(x) \right| < \frac{\epsilon}{2(b-a)}$$
    Thus, for any $n \geq N$, using the triangle inequality (exercise 12),
    \begin{align*}
        \left| \int_{a}^{b}f - \int_{a}^{b}f_n \right| &= \left| \int_{a}^{b}(f - f_n) \right| \\
        &\leq \int_{a}^{b}|f - f_n| \\
        &\leq \int_{a}^{b} \frac{\epsilon}{2(b-a)} \\
        &= \frac{\epsilon}{2(b-a)}(b-a) \\
        &= \frac{\epsilon}{2} \\
        &< \epsilon
    \end{align*}
    Therefore, by definition,
    $$\int_{a}^{b}f = \lim_{n \rightarrow \infty}\int_{a}^{b}f_n$$
    which proves our claim.
 \end{solution}