\chapter{Dirichlet's Theorem}

\section{Number Theory in the Beginning of the 19th Century}

\td

\begin{enumerate}
    \item Lagrange's Quadratic Forms 
    \item Legendre's Theorem and tentative proof
    \item Gauss' Disquisitiones Arithmeticae
    \begin{enumerate}
        \item Modular Arithmetic
        \item Quadratic Forms 
        \item Quadratic Reciprocity 
    \end{enumerate}
\end{enumerate}

\noindent\td 

\section{Fourier's Theorem}

\td

\begin{enumerate}
    \item Origins of Fourier's Theorem
    \begin{enumerate}
        \item Petit rappel historique sur les séries trigonométriques (D'Alembert, Euler et Daniel Bernoulli).
        \item Brievement raconter la vie de Fourier avant la publication de son livre.
        \item Travaux de Fourier et publication de son livre
        \item Dire que Fourier à gagner une grande place dans la sphère scientifique de Paris après la publication de ses travaux.
        \item Mentionner qu'un petit Galois lui a envoyé ses travaux mais que malheureusement, il faudra attendre avant d'en entendre parler à nouveau.
    \end{enumerate}
    \item Enter Dirichlet
    \begin{enumerate}
        \item Présenter Dirichlet (Fermat's Last Theorem, ...)
        \item Dire qu'il a rencontré Fourier
        \item Dire que Dirichlet est le premier à démontrer le théorème de Fourier
        \item Présenter le papier de 1829
        \item Commencer par exposer les erreurs dans la tentative de démonstration de Cauchy.
    \end{enumerate}
    \item The Proof of Fourier's Theorem
    \begin{enumerate}
        \item Stating and proving as Dirichlet did his lemma.
        \item Stating Fourier's Theorem and starting with the assumptions made on $\phi(x)$ instead on finishing with this.
        \item Mention the Dirichlet Function and with a discussion on the evolution of the concept of a function compared to Euler (assuming that every function could be expanded as a series, etc...)
    \end{enumerate}
    \item Exercises:
    \begin{enumerate}
        \item Derive the formula for the coefficients as Euler did.
        \item Expand a function as a Fourier Series.
        \item Prove the Integral Mean Value Theorem using the Intermediate Value Theorem.
        \item Prove the property of alternating sums used by Dirichlet in his proof.
        \item Proof of $\zeta(2)$ using Fourier Series (From Fermat to Minkowski).
    \end{enumerate}
\end{enumerate}

\noindent\td 

\section{Applying Fourier's Theorem}

\td

\begin{enumerate}
    \item Dirichlet's 1835 paper
\end{enumerate}

\noindent\td 

\section{Dirichlet's Theorem}

\td 

\section{Proof of the General Case}

\td 

\section{The Class Number Formula}

\td