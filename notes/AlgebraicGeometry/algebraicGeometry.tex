\documentclass{article}

%% Language and font encodings
\usepackage[english]{babel}
\usepackage[utf8x]{inputenc}
\usepackage[T1]{fontenc}
\usepackage{amsfonts}
\usepackage[margin=41mm]{geometry}

%% Sets page size and margins
%\usepackage[a4paper,top=3cm,bottom=2cm,left=3cm,right=3cm,marginparwidth=1.75cm]{geometry}

%% Useful packages
\usepackage[colorlinks=true, allcolors=black]{hyperref}
\usepackage{amsmath}
\usepackage{xcolor}
\usepackage{amssymb}
\usepackage{amsthm}
\usepackage{stmaryrd}
\usepackage{graphicx}
\usepackage{enumitem}
\usepackage{tikz, pgfplots}
\usetikzlibrary{positioning}

%Theorem
\theoremstyle{plain}
\newtheorem{theorem}{Theorem}[subsection]
\newtheorem{proposition}[theorem]{Proposition}
\newtheorem{lemma}[theorem]{Lemma}
\newtheorem*{corollary}{Corollary}
\theoremstyle{definition}
\newtheorem*{definition}{Definition}

%Usual Sets
\newcommand{\A}{\mathbb{A}}
\newcommand{\C}{\mathbb{C}}
\newcommand{\R}{\mathbb{R}}
\newcommand{\Q}{\mathbb{Q}}
\newcommand{\Z}{\mathbb{Z}}
\newcommand{\N}{\mathbb{N}}
\newcommand{\F}{\mathbb{F}}
\newcommand{\E}{\mathbb{E}}
\newcommand{\K}{\mathbb{K}}
\newcommand{\Zn}[1]{\mathbb{Z}/ #1 \mathbb{Z}}

%Special Sets
\newcommand{\Iint}[2]{\llbracket #1 , #2 \rrbracket}

%Math Operators
\let\Re\relax
\let\Im\relax
\DeclareMathOperator{\Im}{Im}
\DeclareMathOperator{\Re}{Re}
\DeclareMathOperator{\Null}{null}
\DeclareMathOperator{\range}{range}
\DeclareMathOperator{\card}{card}
\DeclareMathOperator{\Aut}{Aut}
\DeclareMathOperator{\Hom}{Hom}
\DeclareMathOperator{\Gal}{Gal}
\DeclareMathOperator{\Char}{char}

%Others
\newcommand{\td}{\textcolor{red}{\textbf{TODO}}}
\newcommand{\isomorphic}{\cong}

%Example environment
\newenvironment{example}{\noindent\textbf{Example:} \vspace{-0.2cm}\begin{itemize}}{\end{itemize}}

%Remark environment
\newenvironment{remark}{\noindent\textbf{Remark:}}{}

%Notation environment
\newenvironment{notation}{\noindent\textit{Notation:}}{}

%Terminology environment
\newenvironment{terminology}{\noindent\textit{Terminology:}}{}

%Set QED symbol to blacksquare
\renewcommand\qedsymbol{$\blacksquare$}

\title{MATH 518 Notes : Algebraic Geometry}
\author{Samy Lahlou}
\date{}

\begin{document}

\maketitle

These notes are based on lectures given by Professor Eyal Goren at McGill University in Fall 2025. The subject of these lectures is \td. As a disclaimer, it is more than possible that I made some mistakes. Feel free to correct me or ask me anything about the content of this document at the following address : samy.lahloukamal@mcgill.ca

\tableofcontents

\newpage

Let $k$ be an algebraically closed field (ex: $\C$, $\overline{\Q}$, $\overline{\Q_p}$, \dots)

\begin{definition}
    The affine $n$-space $\A_k^n$ (or $\A^n$) is the set of all $n$-tuples in $k$ ($k^n$ without the ector space structure).
\end{definition}

\begin{definition}
    The ring $k[X_1, X_2, ..., X_n]$ is the ring of polynomials. It has a basis of monomials $X_1^{i_1}...X_n^{i_n}$ on which define $\deg(X_1^{i_1}...X_n^{i_n})$ as $i_1 + \dots + i_n$. More generally, we define $\deg(f)$ where $f \in k[X_1, ..., X_n]$ as the maximum degree of the monomials that compose $f$.
\end{definition}

\begin{definition}
    An algebraic subset of $\A_k^n$ is a set of the form
    $$V(S) = \{p \in \A_k^n : f(p) = 0 \ \ \forall f \in S\}$$
    where $S$ is a possibly infinite subset of $k[X_1, ..., X_n]$.
\end{definition}

\begin{example}
    \item When $k = \R$, we get that $V(x^2 + y^2 - 1)$ is the unit circle which is an algebraic subset of $\A_{\R}^2$. Similarly, $V(x^2 + y^2 - z^2)$ is the an algebraic subset of $\A_{\R}^3$ which can be visualized as a double infinite cone.
    \item As a subset of $\A_{\R}^2$, the set $V(xy)$ is simply the the two axis and so it can be written as $V(x)\cup V(y)$. In this case, we say that $V(xy)$ is reducible.
    \item $\A_k^n = V(0)$.
    \item The subset $V(x-a, y-b)$ is simply the point $(a,b)$ in $\A_k^2$.
    \item Elliptic curves 
    \item cubic curve with singularity: $V(y^2 - x^3 - x)$.
\end{example}


\begin{proposition}
    \begin{enumerate}
        \item When $S_1 \subset S_2$, then $V(S_1) \supset V(S_2)$.
        \item Given a set $S$, we can consider the ideal $I(S) = $
    \end{enumerate}
\end{proposition}

\end{document}