\section{Cyclic Groups and Cyclic Subgroups}

\begin{exercise}
    Find all subgroups of $Z_{45}=\langle x\rangle$, giving a generator for
    each. Describe the containments between these subgroups.\\
\end{exercise}

\begin{solution}
    \\ First, recall that the positive divisors of $45$ are the following:
    $$1, 3, 5, 9, 15, 45$$
    By Theorem $7$, the subgroups of $Z_{45}$ are exactly $\langle x^d\rangle$
    where $d$ is a divisor. Hence, the subgroups are :
    \begin{itemize}
        \item $\langle x^1\rangle = Z_{45}$
        \item $\langle x^3\rangle = \{1, x^3, x^6, x^9, x^{12}, x^{15},
        x^{18}, x^{21}, x^{24}, x^{27}, x^{30}, x^{33}, x^{36}, x^{39},
        x^{42}\}$
        \item $\langle x^5\rangle = \{1, x^5, x^{10}, x^{15}, x^{20},
        x^{25}, x^{30}, x^{35}, x^{40}\}$
        \item $\langle x^9\rangle = \{1, x^9, x^{18}, x^{27}, x^{36}\}$
        \item $\langle x^{15}\rangle = \{1, x^{15}, x^{30}\}$
        \item $\langle x^{45}\rangle = \{1\}$
    \end{itemize}
    We can represent the inclusions between the subgroups with the following
    diagram:
    \begin{center}
        \begin{tikzpicture}
            %Nodes
            \node at (4,0) (1) {$Z_{45}$};
            \node at (2,-2) (3) {$\langle x^3\rangle$};
            \node at (6,-2) (5) {$\langle x^5\rangle$};
            \node at (0,-4) (9) {$\langle x^9\rangle$};
            \node at (4,-4) (15) {$\langle x^{15}\rangle$};
            \node at (2,-6) (45) {$\{1\}$};


            %Lines
            \draw (1) -- (3);
            \draw (1) -- (5);
            \draw (3) -- (9);
            \draw (3) -- (15);
            \draw (5) -- (15);
            \draw (9) -- (45);
            \draw (15) -- (45);
        \end{tikzpicture}
    \end{center}
\end{solution}

\begin{exercise}
    If $x$ is an element of the finite group $G$ and $|x|=|G|$, prove that
    $G=\langle x\rangle$. Give an explicit example to show that this result
    need not be true if $G$ is an infinite group.\\
\end{exercise}

\begin{solution}
    \\ As proved in exercise 27 of section $1.1$, $\langle x\rangle$ is a
    subgroup of $G$. Moreover, by Proposition, we must have
    $$|\langle x\rangle|=|x|=G$$
    Hence, $\langle x\rangle$ is a subgroup of $G$ with the same finite
    cardinality, so $G=\langle x\rangle$.\\
    This conclusion may not be true if $G$ is infinite. Take for example
    the groupe $G=\mathbb{Z}$ under addition $2\in G$. The subgroup
    generated by $2$ is the set of even integers which has the same
    cardinality as $G$ but obviously $\langle 2\rangle\neq G$.\\
\end{solution}

\begin{exercise}
    Find all generators of $\mathbb{Z}/48\mathbb{Z}$.\\
\end{exercise}

\begin{solution}
    \\Using Proposition $6$, we know that the generators of
    $\mathbb{Z}/48\mathbb{Z}$ are exactly the elements $\overline{n}$
    such that $(48,n)=1$ where $n$ is between $1$ and $47$. By looking
    at the prime factors of $48$ (which are $2$ and $3$), we find that
    the generators are exactly
    $$\overline{1},\ \overline{5},\ \overline{7},\ \overline{11},\
    \overline{13},\ \overline{17},\ \overline{19},\ \overline{23},\
    \overline{25},\ \overline{29},\ \overline{31},\ \overline{35},\
    \overline{37},\ \overline{41},\ \overline{43},\ \overline{47}$$
\end{solution}

\begin{exercise}
    Find all generators for $\mathbb{Z}/202\mathbb{Z}$.\\
\end{exercise}

\begin{solution}
    \\Using Proposition $6$, we know that the generators of
    $\mathbb{Z}/202\mathbb{Z}$ are exactly the elements $\overline{n}$
    such that $(202,n)=1$ where $n$ is between $1$ and $202$. By looking
    at the prime factors of $202$ (which are $2$ and $101$), we can see
    that the integers $n$ between $2$ and $101$ that are prime with $202$
    are odd integers that don't contain $101$ as a prime factor. Hence, the
    generators are simply $\overline{n}$ where $n$ is odd and different than
    $\overline{101}$.\\
\end{solution}

\begin{exercise}
    Find the number of generators for $\mathbb{Z}/49000\mathbb{Z}$.\\
\end{exercise}

\begin{solution}
    \\By Proposition $6$, we know that $\mathbb{Z}/49000\mathbb{Z}$ has
    $\varphi(49000)$ generators:
    \begin{align*}
        \varphi(49000) &= \varphi(2^3\cdot 5^3\cdot 7^2) \\
        &= 2^2(2-1)5^2(5-1)7^1(7-1)\\
        &= 16800\\
    \end{align*}
\end{solution}

\begin{exercise}
    In $\mathbb{Z}/48\mathbb{Z}$, write out all elements of $\langle
    \overline{a}\rangle$ for every $\overline{a}$. Find all inclusions
    between subgroups in $\mathbb{Z}/48\mathbb{Z}$.\\
\end{exercise}

\begin{solution}
    \\First, let's determine all subgroups of $\mathbb{Z}/48\mathbb{Z}$.
    By Theorem $7$, the subgroups are exactly
    \begin{itemize}
        \item $\langle \overline{1}\rangle = \mathbb{Z}/48\mathbb{Z}$
        \item $\langle \overline{2}\rangle = \{\overline{0}, \overline{2},
        \overline{4}, \overline{6}, \overline{8}, \overline{10},
        \overline{12}, \overline{14}, \overline{16}, \overline{18}, 
        \overline{20}, \overline{22}, \overline{24}, \overline{26},
        \overline{28}, \overline{30}, \overline{32}, \overline{34},
        \overline{36}, \overline{38}, \overline{40}, \overline{42},
        \overline{44}, \overline{46}\}$
        \item $\langle \overline{3}\rangle = \{\overline{0}, \overline{3},
        \overline{6}, \overline{9}, \overline{12}, \overline{15},
        \overline{18}, \overline{21}, \overline{24}, \overline{27},
        \overline{30}, \overline{33}, \overline{36}, \overline{39},
        \overline{42}, \overline{45}\}$

        \item $\langle \overline{4}\rangle = \{\overline{0}, \overline{4},
        \overline{8}, \overline{12}, \overline{16}, \overline{20},
        \overline{24}, \overline{28}, \overline{32}, \overline{36},
        \overline{40}, \overline{44}\}$

        \item $\langle \overline{6}\rangle = \{\overline{0}, \overline{6},
        \overline{12}, \overline{18}, \overline{24}, \overline{30},
        \overline{36}, \overline{42}\}$

        \item $\langle \overline{8}\rangle = \{\overline{0}, \overline{8},
        \overline{16}, \overline{24}, \overline{32}, \overline{40}\}$

        \item $\langle \overline{12}\rangle = \{\overline{0}, \overline{12},
        \overline{24}, \overline{36}\}$

        \item $\langle \overline{16}\rangle = \{\overline{0}, \overline{16},
        \overline{32}\}$

        \item $\langle \overline{24}\rangle = \{\overline{0},
        \overline{24}\}$

        \item $\langle \overline{48}\rangle = \{\overline{0}\}$
    \end{itemize}

    Now, using Theorem $7$, we can classify all $\langle\overline{a}\rangle$
    since $\langle\overline{a}\rangle=\langle\overline{(48,a)}\rangle$:
    \begin{itemize}
        \item $\langle\overline{1}\rangle = \langle\overline{5}\rangle
        = \langle\overline{7}\rangle = \langle\overline{11}\rangle
        = \langle\overline{13}\rangle = \langle\overline{17}\rangle
        = \langle\overline{19}\rangle = \langle\overline{23}\rangle
        = \langle\overline{25}\rangle = \langle\overline{29}\rangle
        = \langle\overline{31}\rangle = \langle\overline{35}\rangle
        = \langle\overline{37}\rangle = \langle\overline{41}\rangle
        = \langle\overline{43}\rangle = \langle\overline{47}\rangle
        =  \mathbb{Z}/48\mathbb{Z}$

        \item $\langle\overline{2}\rangle = \langle\overline{10}\rangle
        = \langle\overline{14}\rangle = \langle\overline{22}\rangle
        = \langle\overline{26}\rangle = \langle\overline{34}\rangle
        = \langle\overline{38}\rangle = \langle\overline{46}\rangle
        = \{\overline{0}, \overline{2},
        \overline{4}, \overline{6}, \overline{8}, \overline{10},
        \overline{12}, \overline{14}, \overline{16},\\ \overline{18}, 
        \overline{20}, \overline{22}, \overline{24}, \overline{26},
        \overline{28}, \overline{30}, \overline{32}, \overline{34},
        \overline{36}, \overline{38}, \overline{40}, \overline{42},
        \overline{44}, \overline{46}\}$

        \item $\langle\overline{3}\rangle = \langle\overline{9}\rangle
        = \langle\overline{15}\rangle = \langle\overline{18}\rangle
        = \langle\overline{21}\rangle = \langle\overline{27}\rangle
        = \langle\overline{33}\rangle = \langle\overline{39}\rangle
        = \langle\overline{45}\rangle
        = \{\overline{0}, \overline{3},
        \overline{6}, \overline{9}, \overline{12}, \overline{15},
        \overline{18}, \\\overline{21}, \overline{24}, \overline{27},
        \overline{30}, \overline{33}, \overline{36}, \overline{39},
        \overline{42}, \overline{45}\}$

        \item $\langle\overline{4}\rangle = \langle\overline{20}\rangle
        = \langle\overline{28}\rangle = \langle\overline{44}\rangle
        = \{\overline{0}, \overline{4},
        \overline{8}, \overline{12}, \overline{16}, \overline{20},
        \overline{24}, \overline{28}, \overline{32}, \overline{36},
        \overline{40}, \overline{44}\}$

        \item $\langle\overline{6}\rangle = \langle\overline{30}\rangle
        = \langle\overline{42}\rangle = 
        \{\overline{0}, \overline{6},
        \overline{12}, \overline{18}, \overline{24}, \overline{30},
        \overline{36}, \overline{42}\}$

        \item $\langle\overline{8}\rangle = \langle\overline{48}\rangle =
        \{\overline{0}, \overline{8},
        \overline{16}, \overline{24}, \overline{32}, \overline{40}\}$

        \item $\langle\overline{12}\rangle = \langle\overline{36}\rangle
        = \{\overline{0}, \overline{12}, \overline{24}, \overline{36}\}$

        \item $\langle\overline{16}\rangle = \langle\overline{32}\rangle
        = \{\overline{0}, \overline{16}, \overline{32}\}$

        \item $\langle\overline{24}\rangle = \{\overline{0},\overline{24}\}$
        
        \item $\langle\overline{0}\rangle = \{\overline{0}\}$
    \end{itemize}

    The inclusions between the subgroups can be represented with the
    following diagram:
    \begin{center}
        \begin{tikzpicture}
            %Nodes
            \node at (1.5,0)   (1)  {$\mathbb{Z}/48\mathbb{Z}$};
            \node at (0,-1.5)  (3)  {$\langle \overline{3} \rangle$};
            \node at (3,-1.5)  (2)  {$\langle \overline{2} \rangle$};
            \node at (1.5,-3)  (6)  {$\langle \overline{6} \rangle$};
            \node at (4.5,-3)  (4)  {$\langle \overline{4} \rangle$};
            \node at (3,-4.5)  (12) {$\langle \overline{12} \rangle$};
            \node at (6,-4.5)  (8)  {$\langle \overline{8} \rangle$};
            \node at (4.5,-6)  (24) {$\langle \overline{24} \rangle$};
            \node at (7.5,-6)  (16) {$\langle \overline{16} \rangle$};
            \node at (6,-7.5)  (0)  {$\langle \overline{0} \rangle$};


            %Lines
            \draw (1) -- (3);
            \draw (1) -- (2);
            \draw (6) -- (2);
            \draw (3) -- (6);
            \draw (6) -- (12);
            \draw (2) -- (4);
            \draw (4) -- (12);
            \draw (12) -- (24);
            \draw (4) -- (8);
            \draw (8) -- (24);
            \draw (0) -- (24);
            \draw (8) -- (16);
            \draw (0) -- (16);
        \end{tikzpicture}
    \end{center}
\end{solution}

\begin{exercise}
    Let $Z_{48}=\langle x\rangle$ and use the isomorphism $\mathbb{Z}/48
    \mathbb{Z} \cong Z_{48}$ given by $\overline{1}\mapsto x$ to list all
    subgroups of $Z_{48}$ as computed in the previous exercise.\\
\end{exercise}

\begin{solution}
    \\By the properties of isomorphisms and cyclic groups, using the
    previous exercise, the subgroups of $Z_{48}$ are exactly:
    $$\langle x\rangle,\ \langle x^2\rangle,\ \langle x^3\rangle,\ 
    \langle x^4\rangle,\ \langle x^6\rangle,\ \langle x^8\rangle,\
    \langle x^{12}\rangle,\ \langle x^{16}\rangle,\ \langle x^{24}\rangle
    ,\ \langle 1\rangle$$
\end{solution}

\begin{exercise}
    Let $Z_{48}=\langle x\rangle$. For which integers $a$ does the map
    $\varphi_a$ defined by $\varphi_a:\overline{1}\mapsto x^a$ extends to
    an \textit{isomorphism} from $\mathbb{Z}/48\mathbb{Z}$ onto $Z_{48}$.\\
\end{exercise}

\begin{solution}
    \\ Let's prove that $\varphi_a$ is an isomorphism iff $(a,48)=1$.
    \begin{proof}
        ($\implies$) Suppose $(a,48)\neq 1$, then $x^a$ is not a generator
        of $Z_{48}$ by Proposition $6$, so $\langle x^a\rangle \subsetneqq
        Z_{48}$. Since $\Ima\varphi_a = \langle x^a\rangle$, then
        $\Ima\varphi_a \neq Z_{48}$. Therefore, $\varphi_a$ is not an
        isomorphism since it is not surjective.\\
        ($\Longleftarrow$) Suppose $(a,48)=1$, then $Z_{48}=\langle x^a
        \rangle$ by Proposition 6$6$. Therefore, by Theorem $4$, the map
        $\varphi_a:\langle \overline{1}\rangle \to \langle x^a\rangle$
        mapping $\overline{1}$ to $x^a$ must extend to an isomorphism.\\
    \end{proof}
\end{solution}

\begin{exercise}
    Let $Z_{36}=\langle x\rangle$. For which integers $a$ does the map 
    $\psi_a$ defined by $\psi_a:\overline{1}\mapsto x^a$ extend to a
    \textit{well defined homomorphism} from $\mathbb{Z}/48\mathbb{Z}$
    into $Z_{36}$. Can $\psi_a$ ever be a surjective homomorphism?\\
\end{exercise}

\begin{solution}
    \\First, let's show that $\psi_a$ is a well defined homomorphism iff
    $3 \dvd a$.
    \begin{proof}
        ($\implies$) Suppose that $\psi_a$ is a well defined homomorphism,
        then
        $$\psi_a(\overline{48})=\psi_a(\overline{0})=1$$
        But we also have
        $$\psi_a(\overline{48})=\psi_a(48\cdot\overline{1})=
        \left(\psi_a(\overline{1})\right)^{48}=x^{48a}$$
        Hence,
        \begin{align*}
            x^{48a} = 1 &= 36 \dvd 48a \\
            &= 3 \dvd 4a \\
            &= 3 \dvd a
        \end{align*}

        ($\Longleftarrow$) Suppose that $3 \dvd a$, let's first prove that
        $\psi_a$ is well defined. Let $\overline{n},
        \overline{m}\in \mathbb{Z}/48\mathbb{Z}$ such that
        $\overline{n}=\overline{m}$, then 
        \begin{align*}
            \overline{n}=\overline{m} &\implies 48 \dvd n-m \\
            &\implies 3\cdot48 \dvd a(n-m) \\
            &\implies 36 \dvd 4\cdot 36 = 3\cdot 48 \dvd a(n-m) \\
            &\implies 36 \dvd a(n-m) \\
            &\implies x^{a(n-m)} = 1 \\
            &\implies x^{an} = x^{am} \\
            &\implies \psi_a(\overline{n}) = \psi_a(\overline{m})
        \end{align*}
        It directly follows that $\psi_a$ is homomorphism since for any
        $\overline{n},\overline{m}\in \mathbb{Z}/48\mathbb{Z}$, we get
        \begin{align*}
            \psi_a(\overline{n} +\overline{m}) &= \psi_a(\overline{n+m})\\
            &= x^{a(n+m)}\\
            &= x^{an}x^{am} \\
            &= \psi_a(\overline{n}) \psi_a(\overline{m})
        \end{align*}
    \end{proof}

    Let's now prove that $\psi_a$ can never be surjective.
    \begin{proof}
        Suppose that there is an integer $a$ such that $\psi_a$ is
        surjective, then there is an integer $y$ such that $\psi_a(
        \overline{y})=x$, however:
        \begin{align*}
            \psi_a(\overline{y})=x &\implies x^{ay}=x \\
            &\implies x^{ay-1}=1 \\
            &\implies 36 \dvd ay-1 \\
            &\implies 3\dvd ay-1
        \end{align*}
        which is impossible because by our previous criterion, $3\dvd a$
        which implies $3\dvd ay$. But if $3$ divides both $ay$ and $ay-1$,
        then $3$ must also divide $1$, a contradiction. Therefore, $\psi_a$
        cannot be surjective.\\
    \end{proof}
\end{solution}

\begin{exercise}
    What is the order of $\overline{30}$ in $\mathbb{Z}/54\mathbb{Z}$?
    Write out all of the elements and their orders in $\langle\overline{30}
    \rangle$.\\
\end{exercise}

\begin{solution}
    \\ By Proposition $5$
    $$|\overline{30}| = \frac{54}{(30,54)} = \frac{54}{6} = 9$$
    The elements in $\langle \overline{30} \rangle$ and their respective
    order are
    \begin{itemize}
        \item $\overline{0}$ has order $1$.
        \item $\overline{30}$ has order $9$.
        \item $\overline{6}$ has order $9$.
        \item $\overline{36}$ has order $3$.
        \item $\overline{12}$ has order $9$.
        \item $\overline{42}$ has order $9$.
        \item $\overline{18}$ has order $3$.
        \item $\overline{48}$ has order $9$.
        \item $\overline{24}$ has order $9$.
    \end{itemize}
    We can find the order of each $\overline{30n}$ by thinking of
    $\langle \overline{30} \rangle$ as $\mathbb{Z}/9\mathbb{Z}$ which
    gives us the formula
    $$|\overline{30n}| = 9/(9,n)$$
\end{solution}

\begin{exercise}
    Find all cyclic subgroups of $D_8$. Find a proper subgroup of $D_8$
    which is not cyclic. \\
\end{exercise}

\begin{solution}
    \\First, since every cyclic subgroup of $D_8$ must be of the form $\langle
    x\rangle$ for a $x\in D_8$ and vice versa, then the cyclic subgroups
    of $D_8$ are the following:
    \begin{itemize}
        \item $\langle 1 \rangle = \{1\}$
        \item $\langle r \rangle = \{1,r,r^2,r^3\}$
        \item $\langle r^2 \rangle = \{1,r^2\}$
        \item $\langle r^3 \rangle = \{1,r^3,r^2,r\}$
        \item $\langle s \rangle = \{1,s\}$
        \item $\langle sr \rangle = \{1,sr\}$
        \item $\langle sr^2 \rangle = \{1,sr^2\}$
        \item $\langle sr^3 \rangle = \{1,sr^3\}$
    \end{itemize}
    However, not every subgroup of $D_8$ is cyclic. As an example, $\{1,
    r^2, sr, sr^3\}$ is a subgroup but every element has order $2$ so it
    cannot be cyclic.\\
\end{solution}

\begin{exercise}
    Prove that the following groups are \textit{not} cyclic:
    \begin{enumerate}[label=(\alph*)]
        \item $Z_2\times Z_2$
        \item $Z_2\times \mathbb{Z}$
        \item $\mathbb{Z}\times \mathbb{Z}$\\
    \end{enumerate}
\end{exercise}

\begin{solution}
    \\ Here, we denote by $x$ the generator of $Z_2$ : $Z_2=
    \langle x\rangle$.
    \begin{enumerate}[label=(\alph*)]
        \item $Z_2\times Z_2$ cannot be cyclic because it has order $4$ and
        all of its elements have order $2$, so $\langle x\rangle\neq
        Z_2\times Z_2$ for all $x\in Z_2\times Z_2$.

        \item Suppose that $Z_2\times \mathbb{Z}$ is cyclic, then there is
        a double $(a,b)$ with $a\in Z_2$ and $b\in\mathbb{Z}$ such that
        $$\langle (a,b) \rangle = Z_2\times \mathbb{Z}$$
        Since every element in $Z_2\times \mathbb{Z}$ is a power of $(a,b)$,
        then there is an integer $n$ such that
        \begin{align*}
            (a,b)^n=(xa^{-1},b) &\implies (a^n, bn) = (xa^{-1},b) \\
            &\implies a^n=xa^{-1} \text{ and } bn=b \\
            &\implies a^n=xa^{-1} \text{ and } n=1 \\
            &\implies a=xa^{-1} \\
            &\implies x=a^2=1
        \end{align*}
        A contradiction since $Z_2=\{1,x\}$ and $x\neq 1$. Therefore,
        $Z_2\times \mathbb{Z}$ cannot be cyclic.

        \item Suppose that $\mathbb{Z}\times \mathbb{Z}$ is cyclic, then 
        there is a double $(a,b)$ with $a, b\in\mathbb{Z}$ such that
        $$\langle (a,b) \rangle = \mathbb{Z}\times \mathbb{Z}$$
        Since every element in $\mathbb{Z}\times \mathbb{Z}$ is a multiple
        of $(a,b)$, then there is an integer $n$ such that
        \begin{align*}
            n\cdot(a,b)=(a+1,b) &\implies (na, nb) = (a+1,b) \\
            &\implies na=a+1 \text{ and } bn=b \\
            &\implies na = a+1 \text{ and } n=1 \\
            &\implies a=a+1 \\
            &\implies 1=0
        \end{align*}
        which is a contradiction. Therefore, $\mathbb{Z}\times \mathbb{Z}$
        cannot be cyclic.
    \end{enumerate}
\end{solution}

\begin{exercise}
    Prove that the following pairs of groups are \textit{not} isomorphic:
    \begin{enumerate}[label=(\alph*)]
        \item $\mathbb{Z}\times Z_2$ and $\mathbb{Z}$
        \item $\mathbb{Q}\times Z_2$ and $\mathbb{Q}$\\
    \end{enumerate}
\end{exercise}

\begin{solution}
    \\First, let's show that isomorphisms preserve cyclicity:
    \begin{proof}
        Let $A$ and $B$ be isomorphic groups such that $A$ is cyclic. Let
        $\varphi:A\to B$ be an isomorphism and $a$ a generator of $A$.
        Define $b:=\varphi(a)\in B$ and let's show that $B$ is generated 
        by $b$.\\
        Let $y\in B$, then there is an $x\in A$ such that $\varphi(x)=y$. 
        But $x\in \langle a \rangle$ so there is an integer $n$ such that
        $x=a^n$. Thus,
        $$y=\varphi(x)=\varphi(a^n)=(\varphi(a))^n=b^n$$
        Hence, $y\in\langle b \rangle$ which proves that $B$ is generated
        by $b$. Therefore, $B$ is cyclic as well.\\
    \end{proof}
    Now that we proved this statement, let's prove part (a) and (b).
    Moreover, as for the previous exercise, denote by $x$ the generator
    of $Z_2$.
    \begin{enumerate}[label=(\alph*)]
        \item Suppose that $\mathbb{Z}\times Z_2$ and $\mathbb{Z}$ are
        isomorphic, since $\mathbb{Z}$ is cyclic, then by our lemma, 
        $\mathbb{Z}\times Z_2$ must be cyclic as well. Moreover, by
        exercise 11 of Section 1.6, $\mathbb{Z}\times Z_2$ and
        $Z_2\times \mathbb{Z}$ are isomorphic so our lemma implies that
        $Z_2\times \mathbb{Z}$ is cyclic. This contradicts part (b) of the
        previous question. Therefore, $\mathbb{Z}\times Z_2$ and
        $\mathbb{Z}$ cannot be isomorphic.

        \item Suppose there is an isomorphism $\varphi:\mathbb{Q}
        \times Z_2\to \mathbb{Q}$, then since $\varphi$ maps the identity
        in $\mathbb{Q}\times Z_2$ to the identity in $\mathbb{Q}$, we get
        $$\varphi(0,1) = 0$$
        Hence,
        \begin{align*}
            &\varphi(0,x) + \varphi(0,x) = \varphi(0+0,x^2) \\
            \implies \quad & 2\varphi(0,x) = \varphi(0,1) = 0 \\
            \implies \quad & \varphi(0,x) = 0 \\
        \end{align*}
        which contradicts the fact that $\varphi$ is injective.\\
        Therefore, $\mathbb{Q}\times Z_2$ and $\mathbb{Q}$ are not
        isomorphic.
    \end{enumerate}
\end{solution}

\begin{exercise}
    Let $\sigma = (1\ 2\ 3\ 4\ 5\ 6\ 7\ 8\ 9\ 10\ 11\ 12)$. For each of the
    following integers $a$, compute $\sigma^a$:
    $a=13,\ 65,\ 626,\ 1195,-6,-81,-570$ and $-1211$.\\
\end{exercise}

\begin{solution}
    \\ Since $\sigma$ is a $12$-cycle, then $\sigma^a = \sigma^r$ where
    is the residue of $a$ modulo $12$. Therefore,
    \begin{itemize}
        \item $\sigma^{13} = \sigma^1 = (1\ 2\ 3\ 4\ 5\ 6\ 7\ 8\ 9\ 10\ 11\
        12)$
        \item $\sigma^{65} = \sigma^5 = (1\ 6\ 11\ 4\ 9\ 2\ 7\ 12\ 5\ 10\ 3\
        8)$
        \item $\sigma^{626} = \sigma^2 = (1\ 3\ 5\ 7\ 9\ 11)(2\ 4\ 6\ 8\ 10\
        12)$
        \item $\sigma^{1195} = \sigma^7 = (1\ 8\ 3\ 10\ 5\ 12\ 7\ 2\ 9\ 4\
        11\ 6)$
        \item $\sigma^{-6} = \sigma^6 = (1\ 7)(2\ 8)(3\ 9)(4\ 10)(5\ 11)(6\
        12)$
        \item $\sigma^{-81} = \sigma^3 = (1\ 4\ 7\ 10)(2\ 5\ 8\ 11)(3\ 6\
        9\ 12)$
        \item $\sigma^{-570} = \sigma^6 = (1\ 7)(2\ 8)(3\ 9)(4\ 10)(5\ 11)
        (6\ 12)$
        \item $\sigma^{-1211} = \sigma^1 = (1\ 2\ 3\ 4\ 5\ 6\ 7\ 8\ 9\ 10\
        11\ 12)$
    \end{itemize}
\end{solution}

\begin{exercise}
    Prove that $\mathbb{Q}\times\mathbb{Q}$ is not cyclic.\\
\end{exercise}

\begin{solution}
    \\First, by contradiction, suppose that $\mathbb{Q}\times\mathbb{Q}$ is
    cyclic, then there exist a pair $(a,b)$ in $\mathbb{Q}\times\mathbb{Q}$
    such that $\langle (a,b)\rangle = \mathbb{Q}\times\mathbb{Q}$. obviously,
    $a$ must be nonzero because if $a=0$, it would be impossible to generate
    any element of $\mathbb{Q}\times\mathbb{Q}$ with a nonzero first
    coordinate. For the same reason, $b$ must be nonzero as well.\\
    Hence, there must be an integer $n$ such that $n\cdot(a,b)=(2a,3b)$
    which would imply
    \begin{align*}
        n\cdot(a,b)=(2a,3b) &\implies (na,nb)=(2a,3b) \\
        &\implies na=2a \text{ and } nb=3b \\
        &\implies n=2 \text{ and } n=3
    \end{align*}
    a contradiction. Therefore, $\mathbb{Q}\times\mathbb{Q}$ is not cyclic.
    \\
\end{solution}

\begin{exercise}
    Assume $|x|=n$ and $|y|=m$. Suppose that $x$ and $y$ commute:$xy=yx$.
    Prove that $|xy|$ divides the least common multiple of $m$ and $n$.
    Need this be true if $x$ and $y$ do \textit{not} commute? Give an
    example of commuting elements $x$,$y$ such that the order of $xy$ is
    not equal to the least common multiple of $|x|$ and $|y|$.\\
\end{exercise}

\begin{solution}
    \\First, let's prove the claim made at the beginning of the exercise.
    \begin{proof}
        Assume $|x|=n$ and $|y|=m$ and suppose that $x$ and $y$ commute.
        Then, by Exercise $24$ of Section $1.1$:
        $$(xy)^{\lcm(n,m)}=x^{\lcm(n,m)}y^{\lcm(n,m)}=1$$
        Thus, by Proposition $3$, $|xy| \dvd \lcm(n,m)$.\\
    \end{proof}
    In Exercise $6$ of Section $2.1$, we proved that for a non-abelian
    group, the set of elements of finite order may not be a subgroup because
    there exist two elements $f$ and $g$ of order $2$ in $S_{\mathbb{Z}}$
    such that $|f\circ g|=\infty$. Hence, this same example works as a
    counterexample in our situation.\\
    As an example of commuting $x$ and $y$ such that $|xy|\neq \lcm(|x|,
    |y|)$, we can take $x=y=r^2\in D_8$ since $\lcm(|x|,|y|)=\lcm(2,2)=2$
    but $|xy|=|r^4|=|1|=1$.\\
\end{solution}

\begin{exercise}
    Find a presentation for $Z_n$ with one generator.\\
\end{exercise}

\begin{solution}
    \\ Since $Z_n=\langle x \rangle$ and every property of $Z_n$ can be 
    deduced from the fact that $x^n=1$, then
    $$Z_n=\langle x \ | \ x^n=1\rangle$$
\end{solution}

\begin{exercise}
    Show that if $H$ is any group and $h$ is an element of $H$ with $h^n=1$,
    then there is a unique homomorphism from $Z_n=\langle x\rangle$ to $H$
    such that $x\mapsto h$.\\
\end{exercise}

\begin{solution}
    \\Since $h$ has order $n$, then $\langle h\rangle$ is a cyclic group of
    order $h$ which implies that there is a isomorphism $\varphi:Z_n \to
    \langle h\rangle$ such that $x^k\mapsto h^k$ for all integers $k$. In
    particular, for $k=1$, $\varphi(x)=h$. If we change the codomain to
    $H$, then $\varphi$ becomes a homomorphism.\\
    Let $\psi:Z_n \to H$ be a homomorphism satisfying $x\mapsto h$,
    then by Exercise $1$ of Section $1.6$, $x^k\mapsto h^k$ for all integers
    $k$. Hence, $\varphi=\psi$ so such homomorphism is unique.\\
\end{solution}

\begin{exercise}
    Show that if $H$ is any group and $h$ is an element of $H$, then there
    is a unique homomorphism from $\mathbb{Z}$ to $H$ such that
    $1\mapsto h$.\\
\end{exercise}

\begin{solution}
    \\First, define the map $\varphi:\mathbb{Z}\to H$ by $k\mapsto h^k$.
    Obviously, by properties of exponents and by definition, $\varphi$
    is a homomorphism from $\mathbb{Z}$ to $H$ such that $1\mapsto h$.\\
    To show the uniqueness of such homomorphism, let $\psi:\mathbb{Z}\to H$
    be a homomorphism mapping $1$ to $h$, then by Exercise $1$ of Section
    $1.6$, $\psi(k)=h^k=\varphi(k)$ for all integers $k$. Hence, $\varphi
    = \psi$. Therefore, such homomorphism is unique.\\
\end{solution}

\begin{exercise}
    Let $p$ be a prime and let $n$ be a positive integer. Show that if $x$ is an element of the group $G$ such that $x^{p^n}=1$ then $|x|=p^m$ for some $m\leq n$.\\
\end{exercise}

\begin{solution}
    \\ If $x^{p^n}=1$, then $|x|$ must divide $p^n$ by Proposition $3$. However, the only divisors of $p^n$ are $p^m$ where $m\leq n$ so $|x|$ must be equal to $p^m$ for a $m\leq n$.\\
\end{solution}

\begin{exercise}
    Let $p$ be an odd prime and let $n$ be a positive integer. Use the Binomial Theorem to show that $(1+p)^{p^{n-1}}\equiv 1 \pmod{p^n}$ but $(1+p)^{p^{n-2}}\not\equiv 1 \pmod{p^n}$. Deduce that $1+p$ is an element of order $p^{n-1}$ in the multiplicative group $(\mathbb{Z}/p^n\mathbb{Z})^{\times}$.\\
\end{exercise}

\begin{solution}
    \\The goal of this exercise is to show that $1+p$ is an element of order $p^{n-1}$ is the multiplicative group $(\mathbb{Z}/p^n\mathbb{Z})^{\times}$. To prove it, we will prove separatly the cases where $n=1$ and $n \geq 2$. When $n=1$, then $1+p \equiv 1 \pmod{p}$ is simply the identity which means it has order $1 = p^{n-1}$. \\
    Consider now the case where $n \geq 2$. Let $k$ be a positive integer and consider $m_k \in \mathbb{N}$ the number of times $p$ divdes $k!$ and notice that $k!=p^{m_k}t$ where $t$ is relatively prime with $p$. With a little bit of number theory and by counting the number of times the successive powers of $p$ divide $k!$, we can prove that 
    $$m_k = \sum_{i=1}^{\infty} \biggl \lfloor \frac{k}{p^i} \biggr \rfloor$$
    Let's prove that $k - m_k \geq 1$ when $k \geq 1$ and $k - m_k \geq 2$ when $k \geq 2$. To do so, notice that for $k \geq 4$, we get 
    \begin{align*}
        4 \leq k &\implies k \leq 2k - 4 \\
        &\implies \frac{k}{2} \leq k - 2
    \end{align*}
    which implies
    \begin{align*}
        m_k &= \sum_{i=1}^{\infty} \biggl \lfloor \frac{k}{p^i} \biggr \rfloor \\
        &\leq \sum_{i=1}^{\infty}\frac{k}{p^i} \\
        &= \frac{k}{p} \sum_{i=0}^{\infty}\frac{1}{p^i} \\
        &= \frac{k}{p}\frac{1}{1 - \frac{1}{p}} \\
        &= \frac{k}{p-1} \\
        &\leq \frac{k}{2} \\
        &\leq k - 2
    \end{align*}
    By rearranging the terms, we get
    $$k - m_k \geq 2$$
    For $k = 2$, notice that $m_k$ is always $0$ since $p$ is odd, thus : $k - m_k = 2 - 0 \geq 2$
    Similarly, when $k=3$, $m_k = 1$ only if $p=3$ and $0$ otherwise. Thus, $m_k \leq 1$ which implies $k - m_k \geq 3 - 1 = 2$. Therefore, the inequality holds for all $k \geq 2$. \\
    From this, $k - m_k \geq 1$ obviously holds for all $k \geq 2$. When $k = 1$, then $m_k$ is always $0$ so $k - m_k = 1 - 0 \geq 1$. Therefore, for all $k \geq 1$:
    $$k - m_k \geq 1$$
    Using these inequalities, we get that for $k \geq 2$
    \begin{align*}
        k - m_k \geq 2 &\implies n - 2 - m_k + k \geq n \\
        &\implies p^n \dvd p^{n - 2 - m_k + k} \\
        &\implies p^{n - 2 - m_k + k} \equiv 0 \pmod{p^n}
    \end{align*}
    and similarly for $k \geq 1$:
    $$p^{n - 1 - m_k + k} \equiv 0 \pmod{p^n}$$
    Consider the binomial coefficients $\binom{p^{n-1}}{k}$ and $\binom{p^{n-2}}{k}$ and rewrite them as
    \begin{align*}
        \binom{p^{n-1}}{k} &= \frac{(p^{n-1})!}{(p^{n-1}-k)! \cdot k!} \\
        &= \frac{p^{n-1}(p^{n-1}-1)\dots(p^{n-1}-k +1)}{p^{m_k}t}
    \end{align*}
    and
    \begin{align*}
        \binom{p^{n-2}}{k} &= \frac{(p^{n-2})!}{(p^{n-2}-k)! \cdot k!} \\
        &= \frac{p^{n-2}(p^{n-2}-1)\dots(p^{n-2}-k +1)}{p^{m_k}t} 
    \end{align*}
    Since both are integers, then using the fact that $t$ is relatively prime with $p$ and Gauss' Lemma, we get
    \begin{align*}
        t \dvd p^{m_k}t &\implies t \dvd p^{n-1}(p^{n-1}-1)\dots(p^{n-1}-k +1) \\
        &\implies t \dvd (p^{n-1}-1)\dots(p^{n-1}-k +1) \\
        &\implies \frac{(p^{n-2}-1)\dots(p^{n-2}-k +1)}{t} \in \mathbb{Z} \\
        &\implies \binom{p^{n-1}}{k} = p^{n-1-m_k}c_k, \qquad c_k \in \mathbb{Z}
    \end{align*}
    and similarly:
    $$\binom{p^{n-2}}{k} = p^{n-2-m_k}d_k, \qquad d_k \in \mathbb{Z}$$
    Therefore, using the congruences we developed before, for $k \geq 1$:
    $$\binom{p^{n-1}}{k}p^k \equiv p^{n-1-m_k +k}c_k \equiv 0 \pmod{p^n}$$
    and for $k \geq 2$:
    $$\binom{p^{n-2}}{k}p^k \equiv p^{n-2-m_k +k}d_k \equiv 0 \pmod{p^n}$$
    Therefore, using the Binomial Theorem:
    \begin{align*}
        (1+p)^{p^{n-1}} &\equiv \sum_{k=0}^{p^{n-1}}\binom{p^{n-1}}{k}p^k \\
        &\equiv \binom{p^{n-1}}{0}p^0 + \sum_{k=1}^{p^{n-1}}\binom{p^{n-1}}{k}p^k\\
        &\equiv 1 + \sum_{k=1}^{n-1}0 \\
        &\equiv 1 \pmod{p^n}
    \end{align*}
    and 
    \begin{align*}
        (1+p)^{p^{n-2}} &\equiv \sum_{k=0}^{p^{n-2}}\binom{p^{n-2}}{k}p^k \\
        &\equiv \binom{p^{n-2}}{0}p^0 + \binom{p^{n-2}}{1}p^1 
        + \sum_{k=2}^{p^{n-2}}\binom{p^{n-2}}{k}p^k \\
        &\equiv 1 + p^{n-1} + \sum_{k=2}^{p^{n-2}}0 \\
        &\equiv 1 + p^{n-1} \\
        &\not\equiv 1 \pmod{p^n}
    \end{align*}
    Using Exercise 20, this implies that $|1+p| = p^m$ where $m \leq n-1$. However, if $m \leq n-2$, then it easily follows that $(1+p)^{p^{n-2}} \equiv 1$, a contradiction. Therefore, $1+p$ has order $n-1$ in the multiplicative group $(\mathbb{Z}/p^n\mathbb{Z})^{\times}$.\\
\end{solution}

\begin{exercise}
    Let $n$ be an integer $\geq 3$. Use the Binomial Theorem to show that $(1+2^2)^{2^{n-2}} \equiv 1 \pmod{2^n}$ but $(1+2^2)^{2^{n-3}} \not\equiv 1 \pmod{2^n}$. Deduce that $5$ is an element of order $2^{n-2}$ in the multiplicative group $(\mathbb{Z}/2^n\mathbb{Z})^{\times}$.\\
\end{exercise}

\begin{solution}
    \\ In a similar way as for the previous exercise , the goal here is to prove that $5$ has order $2^{n-2}$. The solution to this exercise will be similar to the previous one. First, let $k$ be a positive integer and define $m_k$ as the power of $2$ in the prime decomposition of $k!$. From this, we get that
    $$k! = 2^{m_k}t, \qquad (p,t)=1$$
    Let's prove show that
    \begin{equation} \tag{1}
        2k - 2 - m_k \geq 0
    \end{equation}
    and also
    \begin{equation} \tag{2}
        2k - 3 - m_k \geq 0
    \end{equation}
    if $k \geq 2$. To do so, first notice that for $k \geq 3$, it's easy to see that $k \leq 2k - 3$. Moreover, since we can express $m_k$ as an inifinite series in the following way, then
    \begin{align*}
        m_k &= \sum_{i=1}^{\infty} \biggl \lfloor \frac{k}{2^i} \biggr \rfloor \\
        &\leq \sum_{i=1}^{\infty}\frac{k}{2^i} \\
        &= k \sum_{i=1}^{\infty}\frac{1}{2^i} \\
        &= k \\
        &\leq 2k - 3
    \end{align*}
    Thus, by rearring the terms, we get inequality $(2)$ for $k \geq 3$. If $k=2$, then $m_k = 1$ which implies 
    $$2k - 3 - m_k = 4 - 3 - 1 \geq 0$$
    Therefore, inequality $(2)$ holds for all $k \geq 2$ as we wanted. It directly follows that inequality $(1)$ holds for $k \geq 2$ as well. When $k=1$, $m_k = 0$ so
    $$2k - 2 - m_k = 2 - 1 - 0 \geq 0$$
    Therefore, inequality $(1)$ holds for all positive integers $k$. \\
    From this, it is easy to see that for each inequality, adding $n$ on both sides implies the following congruences:
    \begin{equation}\tag{3}
        2^{n - 2 - m_k + 2k} \equiv 0 \pmod{2^n}
    \end{equation}
    and when $k \geq 2$:
    \begin{equation}\tag{4}
        2^{n - 3 - m_k + 2k} \equiv 0 \pmod{2^n}
    \end{equation}
    Consider now the binomial coefficients $\binom{2^{n-2}}{k}$ and $\binom{2^{n-3}}{k}$. As in the previous exercise, using Gauss' Lemma and by the fact that these coefficients are integers, for $k \leq 2^{n-2}$ we have:
    \begin{align*}
        \binom{2^{n-2}}{k} = \frac{2^{n-2}(2^{n-2} - 1)\dots (2^{n-2} - k + 1)}{k!} &\implies k! \dvd 2^{n-2}(2^{n-2} - 1)\dots (2^{n-2} - k + 1) \\
        &\implies t \dvd 2^{n-2-m_k}(2^{n-2} - 1)\dots (2^{n-2} - k + 1) \\
        &\implies t \dvd (2^{n-2} - 1)\dots (2^{n-2} - k + 1) \\
        &\implies \frac{(2^{n-2} - 1)\dots (2^{n-2} - k + 1)}{t} \in \mathbb{Z}\\
        &\implies \binom{2^{n-2}}{k} = 2^{n-2-m_k}c_k, \quad c_k \in \mathbb{Z} \\
        &\implies \binom{2^{n-2}}{k}2^{2k} = 2^{n-2-m_k + 2k}c_k \\
        &\implies \binom{2^{n-2}}{k}2^{2k} \equiv 0 \pmod{2^n}
    \end{align*}
    Similarly, for $k \in \Iint{2}{2^{n-3}}$, with the exact same methods, we can show that
    $$\binom{2^{n-3}}{k}2^{2k} \equiv 0 \pmod{2^n}$$
    Therefore, using the Binomial Theorem:
    \begin{align*}
        (1+2^2)^{2^{n-2}} &\equiv \sum_{k=0}^{2^{n-2}}\binom{2^{n-2}}{k}2^{2k} \\
        &\equiv \binom{2^{n-2}}{0}2^0 + \sum_{k=1}^{2^{n-2}}\binom{2^{n-2}}{k}2^{2k} \\
        &\equiv 1 + \sum_{k=1}^{2^{n-2}}0 \\
        &\equiv 1 \pmod{2^n}
    \end{align*}
    and
    \begin{align*}
        (1+2^2)^{2^{n-3}} &\equiv \sum_{k=0}^{2^{n-3}}\binom{2^{n-3}}{k}2^{2k} \\
        &\equiv \binom{2^{n-3}}{0}2^0 + \binom{2^{n-3}}{1}2^2 + \sum_{k=2}^{2^{n-3}}\binom{2^{n-3}}{k}2^{2k} \\
        &\equiv 1 + 2^{n-1} + \sum_{k=1}^{2^{n-3}}0 \\
        &\equiv 1 + 2^{n-1} \\
        &\not\equiv 1 \pmod{2^n}
    \end{align*}
    As in the previous exercise, by Exercise 20, this implies that $5$ has order $2^m$ for some $m \leq n-2$ but we also know that $m$ cannot be less than $n-3$. Therefore, $5$ is an element of order $2^{n-2}$ in $(\mathbb{Z}/2^n\mathbb{Z})^{\times}$. \\
\end{solution} 

\begin{exercise}
    Show that $(\mathbb{Z}/2^n\mathbb{Z})^{\times}$ is not cyclic for any $n \geq 3$. [Find two distinct subgroups of order $2$.]\\ 
\end{exercise}

\begin{solution}
    \\Let $n \geq 3$, then by the previous exercise, $5$ has order $2^{n-2}$. Therefore, $5^{2^{n-3}}$ must have order $2$. Moreover, we also showed that $5^{2^{n-3}} \equiv 1 + 2^{n-1} \pmod{2^n}$ so $1+2^{n-1}$ has order $2$ in $(\mathbb{Z}/2^n\mathbb{Z})^{\times}$. Let's prove that $1+2^{n-1} \not\equiv -1$. By contradiction:
    \begin{align*}
        1+2^{n-1} \equiv -1 &\implies 2^{n-1} \equiv -2 \pmod{2^n}\\
        &\implies 2^n \equiv -4\\
        &\implies 4 \equiv 0\\
        &\implies 2^n \dvd 2^2 \\
        &\implies n \leq 2
    \end{align*}
    A contradiction since $n \geq 3$. Thus, $1+2^{n-1}$ and $-1$ are distinct elements of order $2$, it follows that $\langle 1 + 2^{n-1} \rangle$ and $\langle -1 \rangle$ are distinct subgroups of order $2$. Suppose by contradiction that $(\mathbb{Z}/2^n\mathbb{Z})^{\times}$ is cyclic, then it has a unique subgroup of order $2$ (by Theorem $7.(3)$). Therefore, it is not cylic. \\
\end{solution}

\begin{exercise}
    Let $G$ be a finite group and let $x \in G$.
    \begin{enumerate}[label = \textbf{(\alph*)}]
        \item Prove that if $g\in N_G(\langle x \rangle)$ then $gxg^{-1}=x^a$ for some $a \in \mathbb{Z}$.
        \item Prove conversely that if $gxg^{-1} = x^a$ for some $a\in \mathbb{Z}$ then $g\in N_G(\langle x \rangle)$. [Show first that $gx^kg^{-1} = (gxg^{-1})^k = x^{ak}$ for any integer $k$, so that $g\langle x \rangle g^{-1} \leq \langle x \rangle$. If $x$ has order $n$, show the elements $gx^ig^{-1}$, $i=0,1,\dots ,n-1$ are distinct, so that $|g\langle x \rangle g^{-1}|= |x|=n$ and conclude that $g\langle x \rangle g^{-1}= x$.]
    \end{enumerate}
    Note that this cuts down some of the work in computing normalizers of cyclic subgroups since one does not have to check $ghg^{-1} \in \langle x \rangle$ for every $h \in \langle x \rangle$.\\
\end{exercise}

\begin{solution}
    \begin{enumerate}[label = \textbf{(\alph*)}]
        \item If $g \in N_G(\langle x \rangle)$, then by definition, $g \langle x \rangle g^{-1} = \langle x \rangle$. Since $gxg^{-1} \in g \langle x \rangle g^{-1}$, the equality between the sets tells us that $gxg^{-1} \in \langle x \rangle$. Therefore, $gxg^{-1} = x^a$ for some $a \in \mathbb{Z}$.
        \item Suppose now that $gxg^{-1} = x^a$ for some $a \in \mathbb{Z}$. Let's prove by induction on $k \geq 0$ that 
        $$gx^kg^{-1} = x^{ak}$$
        holds. When $k=0$, we get
        $$gx^0g^{-1} = gg^{-1} = 1 = x^0 = x^{a0}$$
        Suppose now that $gx^jg^{-1} = x^{aj}$ for some $j \geq 0$, then
        \begin{align*}
            gx^jg^{-1} = x^{aj} &\implies (gx^jg^{-1})(gxg^{-1}) = (x^{aj})(x^a) \\
            &\implies gx^jxg^{-1} = x^{aj}x^a \\
            &\implies gx^{j+1}g^{-1} = x^{aj+a} \\
            &\implies gx^{j+1}g^{-1} = x^{a(j+1)}
        \end{align*}
        Thus, by induction, it holds for all $k \geq 0$. \\ Let $k < 0$, then $-k \geq 0$ which implies
        \begin{align*}
            gx^{-k}g^{-1} = x^{-ak} &\implies (gx^{-k}g^{-1})^{-1} = (x^{-ak})^{-1} \\
            &\implies (g^{-1})^{-1}(x^{-k})^{-1}(g)^{-1} = x^{-(-ak)} \\
            &\implies gx^kg^{-1} = x^{ak}
        \end{align*}
        Therefore, it holds for all integers $k$. It directly follows that $g\langle x \rangle g^{-1} \leq \langle x \rangle$ since the conjugate of any element in $\langle x \rangle$ is in $\langle x \rangle$ as well. \\
        By finiteness of $G$, $\langle x \rangle$ must be finite as well. Let $n$ be the order of $\langle x \rangle$, then for $i,j \in \Iint{0}{n-1}$, $x^i=x^j$ if and only if $i=j$. Let $i,j \in \Iint{0}{n-1}$, then 
        \begin{align*}
            gx^ig^{-1} = gx^jg^{-1} &\implies g^{-1}(gx^ig^{-1})g = g^{-1}(gx^jg^{-1})g\\
            &\implies x^i = x^j \\
            &\implies i = j
        \end{align*}
        Thus, $g\langle x \rangle g^{-1}$ has at least $n$ distinct elements. But $g\langle x \rangle g^{-1} \leq \langle x \rangle$ so $g\langle x \rangle g^{-1}$ has at most $n$ distinct elements. Hence, $g\langle x \rangle g^{-1}$ is equal to $\langle x \rangle$ since it is a subgroup of the same cardinality of $n$. Therefore, $g \in N_G(\langle x \rangle)$.\\
    \end{enumerate}
\end{solution}

\begin{exercise}
    Let $G$ be a cyclic group of order $n$ and let $k$ be an integer relatively prime to $n$. Prove that the map $x \mapsto x^k$ is surjective. Use Lagrange's Theorem (Exercise $19$, Section $1.7$) to prove the same is true for any finite group of order $n$. (For such $k$ each element has a $k$\textsuperscript{th} root in $G$. It follows from Cauchy's Theorem in Section $3.2$ that if $k$ is not relatively prime to the order of $G$ then the map $x \mapsto x^k$ is not surjective.) \\
\end{exercise}

\begin{solution}
    \\ First, suppose that $G$ is cyclic and let $g$ be a generator. By Porposition 6, $G = \langle g^k \rangle$ since $(k,n) = 1$. Hence, for all $y \in G$, there is an integer $a$ such that $y = (g^k)^a = (g^a)^k$. If we let $x = g^a$, then $y$ is the image of $x$ under the map $x\mapsto x^k$. Therefore, the map is surjective.\\
    Let's now prove it for the general case when $G$ may not be cyclic. Let $y \in G$ and let's show that there is a $x \in G$ such that $y = x^k$. Obviously, $y$ is in the cyclic subgroup $\langle y \rangle$ which has an order $m$ that divides $n$ (by Lagrange's Theorem). Hence, we must have $(k,m) = 1$ as well. Applying what we proved above for the cyclic group $\langle y \rangle$ tells us that the map $x \mapsto x^k$ with domain $\langle y \rangle$ is onto $\langle y \rangle$. Therefore, there is a $x \in \langle y \rangle \subseteq G$ such that $y = x^k$ which proves that the map is surjective. \\
\end{solution}

\begin{exercise}
    Let $Z_n$ be a cyclic group of order $n$ and for each integer $a$ let
    $$\sigma_a : Z_n  \to Z_n \quad \text{ by } \quad \sigma_a (x) = x^a \text{ for all } x\in Z_n $$
    \begin{enumerate}[label = \textbf{(\alph*)}]
        \item Prove that $\sigma_a$ is an automorphism of $Z_n$ if and only if $a$ and $n$ are relatively prime (automorphisms were introduced in Exercise 20, Section 1.6).
        \item Prove that $\sigma_a = \sigma_b$ if and only if $a \equiv b \pmod{n}$
        \item Prove that \textit{every} automorphism of $Z_n$ is equal to $\sigma_a$ for some integer $a$.
        \item Prove that $\sigma_a \circ \sigma_b = \sigma_{ab}$. Deduce that the map $\overline{a} \mapsto \sigma_a$ is an isomorphism of $(\mathbb{Z}/n\mathbb{Z})^{\times}$ onto the automorphism group of $Z_n$ (so Aut($Z_n$) is an abelian group of order $\varphi(n)$). \\
    \end{enumerate}
\end{exercise}

\begin{solution}
    \\ In this exercise, denote by $z$ a generator of $Z_n$.
    \begin{enumerate}[label = \textbf{(\alph*)}]
        \item Suppose that $a$ and $n$ are relatively prime, then by Proposition 6, $Z_n = \langle z^a \rangle$. From this, since $\sigma_a$ is simply the map $z^b \mapsto (z^a)^b$, then by Theorem 4, it is an isomorphism from $\langle z \rangle$ to $\langle z^a \rangle$. Therefore, $\sigma_a$ is an automorphism of $Z_n$. \\
        Suppose now that $a$ and $n$ have a common divisor $d  > 1$, then $n = kd$ and $a = k'd$ for some integers $1 < k < n$ and $1 < k' < a$. From this, define $x = z^k$ and notice that $x \neq 1$ by properties of $k$. However, we have
        $$\sigma_a (x) = (z^k)^a = z^{kk'd} = (z^{n})^{k'} = 1$$
        Hence, $\sigma_a(x) = \sigma_a(1)$ so it is not injective. Therefore, $\sigma_a$ cannot be an automorphism of $Z_n$.
        \item Suppose that $\sigma_a = \sigma_b$. Since $|z|=n$, then
        \begin{align*}
            \sigma_a(z) = \sigma_b(z) &\implies z^a = z^b \\
            &\implies z^{a-b} = 1 \\
            &\implies n \dvd a-b \\
            &\implies a \equiv b \pmod{n}
        \end{align*}
        Suppose now that $a \equiv b \pmod{n}$ and let $x\in G$. Since $G$ has order $n$, then $|x| \dvd n$. Moreover, $n \dvd a-b$ which implies that $|x| \dvd a-b$. Therefore,
        \begin{align*}
            x^{a-b} = 1 &\implies x^a = x^b \\
            &\implies \sigma_a(x) = \sigma_b(x)
        \end{align*}
        Since it holds for all $x$ in $G$, then the functions are equal.
        \item Let $\varphi$ be an automorphism of $Z_n$ and let $a$ be an integer that satisfies 
        $$\varphi(z) = z^a$$
        To prove that $\varphi = \sigma_a$, let $x\in G$ and $b\in \mathbb{Z}$ such that $x = z^b$, then
        \begin{align*}
            \varphi(x) &= \varphi(z^b) \\
            &= (\varphi(z))^b \\
            &= (z^a)^b \\
            &= (z^b)^a \\
            &= \sigma_a(x)
        \end{align*}
        Therefore, $\varphi = \sigma_a$ so any automorphism can be written in this form.
        \item Let $a$ and $b$ be integers and $x \in G$:
        \begin{align*}
            (\sigma_a \circ \sigma_b)(x) &= \sigma_a(\sigma_b(x)) \\
            &= \sigma_a(x^b) \\
            &= (x^b)^a \\
            &= x^{ab} \\
            &= \sigma_{ab}(x)
        \end{align*}
        Thus, $\sigma_a \circ \sigma_b = \sigma_{ab}$.\\
        Define now the function $\sigma : (\mathbb{Z}/n\mathbb{Z})^{\times} \to \text{Aut}(Z_n)$ defined by 
        $$\sigma(\overline{a}) = \sigma_a$$
        Notice that the fact that $\sigma$ is well-defined and injective is exactly what we proved in part (b) of this exercise. The fact that $\sigma$ is a group homomorphism is exactly what we proved at the beginning of part (d). The fact that $\sigma$ is surjective  and onto Aut($Z_n$) is exactly what we proved in part (a) and (c). Therefore, $\sigma$ is an isomorphism of $(\mathbb{Z}/n\mathbb{Z})^{\times}$ to Aut($Z_n$).
    \end{enumerate}
\end{solution}