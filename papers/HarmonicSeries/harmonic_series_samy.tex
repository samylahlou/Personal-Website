\documentclass[12pt, oneside]{article}

%% Language and font encodings
\usepackage[english]{babel}
\usepackage[utf8x]{inputenc}
\usepackage[T1]{fontenc}
\usepackage{amsfonts}

%% Sets page size and margins
\usepackage[a4paper,top=3cm,bottom=2cm,left=3cm,right=3cm,marginparwidth=1.75cm]{geometry}

%% Useful packages
\usepackage{amsmath}
\usepackage{amssymb}
\usepackage{amsthm}
\usepackage{stmaryrd}
\usepackage{graphicx}
\usepackage[colorinlistoftodos]{todonotes}
\usepackage[colorlinks=true, allcolors=blue]{hyperref}
\usepackage{enumitem}

%Commands definitions
\renewcommand{\thesection}{\thechapter\Alph{section}}
\newcommand{\setbackgroundcolour}{\pagecolor[rgb]{0.1,0.1,0.1}}  
\newcommand{\settextcolour}{\color[rgb]{1,1,1}}    
\newcommand{\invertbackgroundtext}{\setbackgroundcolour\settextcolour}
\newcommand{\R}{\mathbb{R}}
\newcommand{\Q}{\mathbb{Q}}
\newcommand{\Z}{\mathbb{Z}}
\newcommand{\N}{\mathbb{N}}
\newcommand{\Iint}[2]{\llbracket #1 , #2 \rrbracket}
\renewcommand{\Im}{\text{Im}}
\newcommand{\card}{\text{card}}

%Command execution. 
%If this line is commented, then the appearance remains as usual.
%\invertbackgroundtext

%Set QED symbol to blacksquare
\renewcommand\qedsymbol{$\blacksquare$}

%% Table packages
\usepackage{array}
\newcolumntype{P}[1]{>{\centering\arraybackslash}p{#1}}
\usepackage{multirow}

% Définir un nouveau compteur d'exercice
\newcounter{exercise}[section] % Compteur qui est remis à zéro à chaque nouvelle section
%\renewcommand{\theexercise}{\arabic{exercise}}

% Définir un environnement pour l'exercice
\newenvironment{exercise}{
    \refstepcounter{exercise} % Incrémenter le compteur
    \noindent\textbf{Exercise \arabic{exercise}} 
    \\ 
}{} % Fin de l'environnement

% Définir un environnement pour les solutions
\newenvironment{solution}{
    \noindent\textbf{Solution} 
}{}


\title{New Proof of the Divergence of the Harmonic Series}
\author{Samy Lahlou}

\begin{document}
\maketitle

\begin{itemize}
    \item talk about the doccument with the different proofs of the divergence (citation at the end)
    \item state the Alternating Series test (prove it ?)
    \item prove that the Alternating Harmonic Series converges
    \item lemma for grouping terms
    \item proof by contradiction of the divergence of the harmonic series
    \item ask to be contacted if a similar proof already exists
\end{itemize}

Unrigorous proof:\\
Suppose that he Harmonic Series Converges, define
$$H = 1 + \frac{1}{2} + \frac{1}{3} + \frac{1}{4} + \frac{1}{5} + \frac{1}{6} + ...$$
Since we know that the series
$$L = 1 - \frac{1}{2} + \frac{1}{3} - \frac{1}{4} + \frac{1}{5} - \frac{1}{6} + ...$$
converges to a nonzero real number, then 
\begin{align*}
    H - L &= 1 + \frac{1}{2} + \frac{1}{3} + \frac{1}{4} + \frac{1}{5} + \frac{1}{6} + ... \\
    & - \left(1 - \frac{1}{2} + \frac{1}{3} - \frac{1}{4} + \frac{1}{5} - \frac{1}{6} + ...\right) \\
    &= 0 + 2\cdot \frac{1}{2} + 0 + 2\cdot \frac{1}{4} + 0 + 2\cdot \frac{1}{6} + ... \\
    &= 1 + \frac{1}{2} + \frac{1}{3} + \frac{1}{4} + \frac{1}{5} + \frac{1}{6} + ... \\
    &= H
\end{align*}
which implies
$$L = 0$$
A contradiction, so the Harmonic Series Diverges.
\end{document}