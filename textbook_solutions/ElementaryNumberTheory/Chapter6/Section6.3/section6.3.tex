\section{The Greatest Integer Function}

\begin{exercise}
    Given integers $a$ and $b > 0$, show that there exists a unique integer $r$ with $0 \leq r < b$ satisfying $a = [a/b]b + r$. \\
\end{exercise}

\begin{solution}
    First, we have that $a = qb + r$ for some integers $q$ and $r$. It follows that $a/b = q + r/b$. Since $0 \leq r < b$, then $0 \leq r/b < 1$. It follows that $[a/b] = q$, and hence, that $a = [a/b]b + r$. Note that this $r$ is unique because otherwise, it would contradict the uniqueness of the pair $q,r$.\\
\end{solution}

\begin{exercise}
    Let $x$ and $y$ be real numbers. Prove that the greatest integer function satisfies the following properties:
    \begin{enumerate}
        \item $[x+n] = [x] + n$ for any integer $n$.
        \item $[x] + [-x] = 0$ or $-1$, according as $x$ is an integer or not. [\textit{Hint:} Write $x = [x] + \theta$, with $0 \leq \theta < 1$, so that $-x = -[x] - 1 + (1 - \theta)$.]
        \item $[x] + [y] \leq [x+y]$ and, when $x$ and $y$ are positive, $[x][y] \leq [xy]$.
        \item $[x/n] = [[x]/n]$ for any positive integer $n$. [\textit{Hint:} Let $x/n = [x/n] + \theta$, where $0 \leq \theta < 1$; then $[x] = n[x/n] + [n\theta]$.]
        \item $[nm/k] \geq n[m/k]$ for positive integers $n,m,k$.
        \item $[x] + [y] + [x + y] \leq [2x] + [2y]$. [\textit{Hint:} Let $x = [x] + \theta$, $0 \leq \theta < 1$, and $y = [y] + \theta'$, $0 \leq \theta' < 1$. Consider cases in which neither, one, or both of $\theta$ and $\theta'$ are greater than or equal to $\frac{1}{2}$.]
    \end{enumerate}
\end{exercise}

\begin{solution}
    \begin{enumerate}
        \item Since we can write $x = [x] + \theta$ for some $0 \leq \theta < 1$, then $x+ n = k + \theta$ where $k = [x] + n$ is an integer. It follows that $[x + n] = k = [x] + n$.
        \item If $x$ is an integer, then clearly $[x] + [-x] = x + (-x) = 0$. If $x$ is not an integer, then we can write $x = [x] + \theta$ where $0 < \theta < 1$. Hence, $-x = -[x]  - 1 + (1 - \theta)$ which implies that $[-x] = - [x] - 1$, and thus, that $[x] + [-x] = [x] - [x] - 1 = -1$.
        \item Since $[x] \leq x$ and $[y] \leq y$, then $[x] + [y] \leq x + y$. Since $[x] + [y]$ is an integer, then it must be less than or equal to the greatest integer smaller than $x + y$, in other words, $[x] + [y] \leq [x + y]$.
        
        Similarly, if $x$ and $y$ are positive, then $[x][y] \leq xy$ which implies that $[x][y] \leq [xy]$ using the same argument as for the sum.
        \item Since $x/n = [x/n] + \theta$ for some $0 \leq \theta < 1$, then $x = n[x/n] + n\theta$, and hence, $[x] = [n[x/n] + n\theta] = n[x/n] + [n\theta]$ by part (a). Dividing both sides by $n$ gives us $[x]/n = [x/n] + [n\theta]/n$. Since $0 \leq [n\theta] \leq n \theta < n$, then $0 \leq [n\theta]/n < 1$, and hence, $[[x]/n] = [x/n]$.
        \item Since $[m/k] \leq m/k$, then $n[m/k] \leq nm/k$. Since $n[m/k]$ is an integer less than $nm/k$, then it must be less than or equal to the greatest integer smaller than $nm/k$. Thus, $n[m/k] \leq [nm/k]$.
        \item Let $x = [x] + \theta_1$ and $y = [y] + \theta_2$ where $0 \leq \theta_i < 1$. When $\theta_1 + \theta_2 < 1$, we have that $x + y = [x] + [y] + (\theta_1 + \theta_2)$ where $0 \leq \theta_1 + \theta_2 < 1$, which implies that $[x + y] = [x] + [y]$. When $1 \leq \theta_1 + \theta_2 < 2$, we have that $x + y = [x] + [y] + 1 + \theta$ where $0 \leq \theta_1 + \theta_2 - 1 < 1$, and hence, $[x + y] = [x] + [y] + 1$. In both cases, we have that $[x + y] \leq [x] + [y]$. \\
    \end{enumerate}
\end{solution}

\begin{exercise}
    Find the highest power of $5$ dividing $1000!$ and the highest power of 7 dividing $2000!$. \\
\end{exercise}

\begin{solution}
    Using Theorem 6-9, we have that the highest power of 5 dividing $1000!$ is
    $$\left[\frac{1000}{5}\right] + \left[\frac{1000}{5^2}\right] + \left[\frac{1000}{5^3}\right] + \left[\frac{1000}{5^4}\right]= 200 + 40 + 8 + 1 = 249.$$
    Similarly, the highest power of 7 dividing $2000!$ is
    $$\left[\frac{2000}{7}\right] + \left[\frac{2000}{7^2}\right] + \left[\frac{2000}{7^3}\right] = 285 + 40 + 5 = 330.$$ \\
\end{solution}

\begin{exercise}
    For an integer $n \geq 0$, show that $[n/2] - [-n/2] = n.$ \\
\end{exercise}

\begin{solution}
    If $n$ is even, then $n = 2k$ which implies that
    $$[n/2] - [-n/2] = k - (-k) = 2k = n.$$
    If $n = 2k+1$, then $[n/2] = [k + 1/2] = k$ and $[-n/2] = [-k - 1/2] = -k - 1$. Hence, $[n/2] - [n/2] = k - (-k - 1) = 2k + 1 = n$. Therefore, the formula holds for all $n$.\\
\end{solution}

\begin{exercise}
    \begin{enumerate}
        \item Verify that $1000!$ terminates in $249$ zeros.
        \item For what values of $n$ does $n!$ terminate in 37 zeros? \\
    \end{enumerate}
\end{exercise}

\begin{solution}
    \begin{enumerate}
        \item We already showed that the highest power of 5 dividing $1000!$ is 249. Since $[x] \leq [y]$ when $x \leq y$, then it follows from Theorem 6-9 that the highest power of 2 dividing $1000!$ is higher than 249. It follows that the highest power of 10 dividing $1000!$ is 249, and equivalently, $1000!$ ends with 249 zeros.
        \item Using an argument similar to the one used in the previous part, it suffices to find the values of $n$ such that $\sum_{k=1}^{\infty}[n/5^k] = 37$. From this formula, we get that $[n/5] \leq 37$, and hence, that $n \leq 189$. Since $n = 5^3$ gives $\sum_{k=1}^{\infty}[5^3/5^k] = 31$ which implies that $126 \leq n \geq 189$. From this range, we get that
        $$\sum_{k=1}^{\infty}\left[\frac{n}{5^k}\right] = \left[\frac{n}{5}\right] + \left[\frac{n}{5^2}\right] + \left[\frac{n}{5^3}\right] = \left[\frac{n}{5}\right] + \left[\frac{n}{5^2}\right] + 1.$$
        Moreover, the fact that $126 \leq n \leq 189$ implies that $5 \leq [n/5^2] \leq 7$. It follows that $[n/5] = 29$, $30,$ or 31, which implies that $145 \leq n \leq 160$. Checking the values of $\sum_{k=1}^{\infty}[n/5^k]$ in this range gives us that the possible values of $n$ are $n = 150, 151, 152, 153, 154$. \\
    \end{enumerate}
\end{solution}

\begin{exercise}
    If $n \geq 1$ and $p$ is a prime, prove that
    \begin{enumerate}
        \item $(2n)/(n!)^2$ is an even integer. [\textit{Hint:} Use induction on $n$.]
        \item The exponent of the highest power of $p$ which divides $(2n)!/(n!)^2$ is
        $$\sum_{k=1}^{\infty}([2n/p^k] - 2[n/p^k]).$$
        \item In the prime factorization of $(2n)!/(n!)^2$ the exponent of any prime $p$ such that $n < p < 2n$ is equal to 1.
    \end{enumerate}
\end{exercise}

\begin{solution}
    \begin{enumerate}
        \item Simply notice that
        $$\frac{(2n)!}{(n!)^2} = \frac{2n}{n} \cdot \frac{(2n-1)!}{(n-1)!\cdot n!} = 2\binom{2n-1}{n}.$$
        Since $\binom{2n-1}{n}$ is an integer, then it follows that $(2n)/(n!)^2$ is an even integer.
        \item The highest exponent of $p$ dividing $(2n)/(n!)^2$ is the highest exponent of $p$ dividing $(2n)!$ minus the highest exponent of $p$ dividing $(n!)^2$. Moreover, the highest exponent of $p$ dividing $(n!)^2$ is simply twice the highest exponent of $p$ dividing $n!$. Putting everything together and using Theorem 6-9 gives us that the highest exponent of $p$ dividing $(2n)/(n!)^2$ is 
        $$\sum_{k=1}^{\infty}([2n/p^k] - 2[n/p^k]).$$
        \item When $n < p < 2n$, we get that $[2n/p] = 1$ and $[n/p] = 0$; and that $[2n/p^k] = [n/p^k] = 0$ for all $k \geq 2$. Hence, using part (b), we get that in the prime factorization of $(2n)!/(n!)^2$ the exponent of $p$ is
        $$\sum_{k=1}^{\infty}([2n/p^k] - 2[n/p^k]) = (1 - 2\cdot 0) + 0 + 0 + \dots = 1.$$  \\
    \end{enumerate}
\end{solution}

\begin{exercise}
    Let the positive integer $n$ be written in terms of powers of the prime $p$ so that $n = a_kp^k + \dots + a_2p^2 + a_1p + a_0$ where $0 \leq a_i < p$. Show that the exponent of the highest power of $p$ appearing in the prime factorization of $n!$ is 
    $$\frac{n - (a_k + \dots + a_2 + a_1 + a_0)}{p-1}.$$
\end{exercise}

\begin{solution}
    Using Theorem 6-9, we have that the highest power of $p$ dividing $n!$ is
    \begin{align*}
        \sum_{i=1}^{\infty}\left[\frac{a_kp^k + \dots + a_2p^2 + a_1p + a_0}{p^i}\right] &= \sum_{i=1}^{k}(a_kp^{k-i} + \dots + a_i) \\
        &= \sum_{i=1}^{k}\sum_{j=i}^{k}a_j p^{j-i} \\
        &= \sum_{j=1}^{k}\sum_{i=1}^{j}a_jp^{j-i} \\
        &= \sum_{j=1}^{k}a_jp^j\sum_{i=1}^{j}p^{-i} \\
        &= \sum_{j=1}^{k}a_jp^j \cdot \frac{1}{p}\cdot \frac{1 - p^{-j}}{1 - p^{-1}} \\
        &= \frac{1}{p-1} \sum_{j=1}^{k}a_jp^j (1 - p^{-j}) \\
        &= \frac{\sum_{j=1}^{k}a_jp^j - \sum_{j=1}^{k}a_j}{p-1} \\
        &= \frac{(n- a_0) - (a_k + \dots + a_2 + a_1)}{p-1} \\
        &= \frac{n - (a_k + \dots + a_2 + a_1 + a_0)}{p-1}.\\
    \end{align*}
\end{solution}

\begin{exercise}
    \begin{enumerate}
        \item Using Problem 7, show that the exponent of the highest power of $p$ dividing $(p^k - 1)!$ is $[p^k - (p-1)k - 1]/(p-1)$. [\textit{Hint:} Recall the identity
        $$p^k - 1 = (p-1)(p^{k-1} + \dots + p^2 + p + 1).]$$
        \item Determine the highest power of 3 dividing $80!$ and the highest power of 7 dividing $2400!$. [\textit{Hint:} $2400 = 7^4 - 1$.]
    \end{enumerate}
\end{exercise}

\begin{solution}
    \begin{enumerate}
        \item Since $p^k - 1 = (p-1)(p^{k-1} + \dots + p^2 + p + 1)$, then it means that we can write $p^k - 1$ as $a_{k-1}p^{k-1} + \dots + a_1 p + a_0$ where $a_i = p-1$ for all $i$. Hence, using Problem 7, we get that the highest power of $p$ dividing $(p^k - 1)!$ is
        $$\frac{(p^k - 1) - (a_{k-1} + \dots + a_1 + a_0)}{p-1} = \frac{p^k  - (p-1)k - 1}{p-1}.$$
        \item Since $80 = 3^4 - 1$, then part (a) tells us that the exponent of the highest power of $3$ dividing 80 is $[80 - 2\cdot 4]/2 = 40 - 4 = 36$. Similarly, the exponent of the highest power of $7$ dividing $2400 = 7^4 - 1$ is $[2400 - 6\cdot 4]/6 = 400 - 4 = 396$. \\
    \end{enumerate}
\end{solution}

\begin{exercise}
    Find an integer $n \geq 1$ such that the highest power of 5 contained in $n!$ is 100. [\textit{Hint:} Since the sum of the coefficients of the powers of 5 needed to express $n$ in the base 5 is at least 1, begin by considering the equation $(n-1)/4 = 100.$] \\
\end{exercise}

\begin{solution}
    Let $a_0, a_1, ..., a_k$ be the coefficients of the base 5 expression of $n$, then Problem 7 tells us that $n$ must satisfy the equation $n - (a_k + \dots + a_1 + a_0) = 400$. If the sum the $a_i$'s is 1, then $n = 401$ which is a contradiction since 401 is not a power of 5. If the sum of the $a_i$'s is 2, then $n = 402 = 3\cdot 5^2 + 1\cdot 5^2 + 0\cdot 5 + 2$ which doesn't satisfy the condition that the sum of the coefficients is 2. Similarly, the sum of the coefficients cannot be 3 or 4 because $403 = 3\cdot 5^2 + 1\cdot 5^2 + 0\cdot 5 + 3$ and $404 = 3\cdot 5^2 + 1\cdot 5^2 + 0\cdot 5 + 4$. Next, if the sum of the coefficients is $5$, then $n = 405 = 3\cdot 5^2 + 1\cdot 5^2 + 1\cdot 5 + 0$ which is consistent with the fact that the sum of the coefficients is 5. Indeed, by Problem 7, we have that the highest power of 5 dividing $405! = (3\cdot 5^2 + 1\cdot 5^2 + 1\cdot 5)!$ is
    $$\frac{405 - (3 + 1 + 1)}{4} = \frac{400}{4} = 100.$$
    Therefore, it holds for $n = 405$. \\
\end{solution}

\begin{exercise}
    Given a positive integer $N$, show that
    \begin{enumerate}
        \item $\sum_{n=1}^{N}\mu(n)[N/n] = 1$;
        \item $|\sum_{n=1}^{N}\mu(n) /n| \leq 1$.
    \end{enumerate}
\end{exercise}

\begin{solution}
    \begin{enumerate}
        \item Let $F(n) = \sum_{d \ \mid \ n}\mu(d)$, then we know that $F(1) = 1$ and $F(n) = 0$ for all $n \geq 2$ (Theorem 6-6). Thus, by Theorem 6-11:
        $$\sum_{n=1}^{N}\mu(n)\left[\frac{N}{n}\right] = \sum_{n=1}^{N}F(n) = 1.$$
        \item For all $n \leq N$, define $\theta_n = N/n - [N/n]$, then $0 \leq \theta_n < 1$, and $\theta_1 = 0$. Hence, using part (a):
        \begin{align*}
            \left|\sum_{n=1}^{N}\mu(n)\frac{N}{n}\right| &= \left|\sum_{n=1}^{N}\mu(n)\left(\left[\frac{N}{n}\right] + \theta_n\right)\right|  \\
            &=  \left|\sum_{n=1}^{N}\mu(n)\left[\frac{N}{n}\right] + \sum_{n=1}^{N}\mu(n)\theta_n\right|  \\
            &= \left|1 + \sum_{n=2}^{N}\mu(n)\theta_n\right|  \\
            &\leq 1 + \sum_{n=2}^{N}|\mu(n)\theta_n| \\
            &\leq 1 +  \sum_{n=2}^{N}1 \\
            &= N.
        \end{align*}
        Thus, we have shown that
        $$\left|\sum_{n=1}^{N}\mu(n)\frac{N}{n}\right| \leq N.$$
        Dividing both sides by $N$ gives us the desired inequality. \\
    \end{enumerate}
\end{solution}

\begin{exercise}
    Illustrate Problem 10 in the case $N = 6$.\\
\end{exercise}

\begin{solution}
    \begin{enumerate}
        \item \begin{align*}
            & \ \ \sum_{n=1}^{6}\mu(n)\left[\frac{6}{n}\right] \\
            &= \mu(1)\left[\frac{6}{1}\right] + \mu(2)\left[\frac{6}{2}\right] + \mu(3)\left[\frac{6}{3}\right] + \mu(4)\left[\frac{6}{4}\right] + \mu(5)\left[\frac{6}{5}\right] + \mu(6)\left[\frac{6}{6}\right] \\
            &= 1\cdot 6 + (-1)\cdot 3 + (-1)\cdot 2 + 0 \cdot 1 + (-1) \cdot 1 + 1\cdot 1\\
            &= 6 - 3 - 2 - 1 + 1 \\
            &= 1.
        \end{align*}
        \item \begin{align*}
            \sum_{n=1}^{6}\frac{\mu(n)}{n} &= \frac{\mu(1)}{1} + \frac{\mu(2)}{2} + \frac{\mu(3)}{3} + \frac{\mu(4)}{4} + \frac{\mu(5)}{5} + \frac{\mu(6)}{6} \\
            &= 1 - \frac{1}{2} - \frac{1}{3} - \frac{1}{5} + \frac{1}{6} \\
            &= \frac{2}{15}. \\
        \end{align*}
    \end{enumerate}
\end{solution}

\begin{exercise}
    Verify that the formula
    $$\sum_{n=1}^{N}\lambda(n)[N/n] = [\sqrt{N}]$$
    holds for any positive integer $N$. [\textit{Hint:} Apply Theorem 6-11 to the multiplicative function $F(n) = \sum_{d \ \mid \ n}\lambda(d)$, noting that there are $[\sqrt{N}]$ perfect squares not exceeding $n$.]\\
\end{exercise}

\begin{solution}
    Recall that if we define $F(n) = \sum_{d \ \mid \ n}\lambda(d)$, then $F(n)$ is equal to 1 if and only if $n$ is a square, otherwise $F(n) = 0$. Thus, by Theorem 6-11:
    $$\sum_{n=1}^{N}\lambda(n)[N/n] = \sum_{n=1}^{N}F(n) = [\sqrt{N}]$$
    using the fact that there are $[\sqrt{N}]$ squares between 1 and $N$. \\
\end{solution}

\begin{exercise}
    If $N$ is a positive integer, establish that
    \begin{enumerate}
        \item $N = \sum_{n=1}^{2N}\tau(n) - \sum_{n=1}^{N}[2N/n]$;
        \item $\tau(N) = \sum_{n=1}^{N}([N/n] - [(N-1)/n])$.
    \end{enumerate}
\end{exercise}

\begin{solution}
    \begin{enumerate}
        \item Using Corollary 1, we have that
        $$\sum_{n=1}^{2N}\tau(n) - \sum_{n=1}^{N}\left[\frac{2N}{n}\right] = \sum_{n=1}^{2N}\left[\frac{2N}{n}\right] - \sum_{n=1}^{N}\left[\frac{2N}{n}\right] = \sum_{n=N+1}^{2N}\left[\frac{2N}{n}\right].$$
        Next, we have that for all $N+1 \leq n \leq 2N$, $[2N/n] = 1$, and hence:
        $$\sum_{n=1}^{2N}\tau(n) - \sum_{n=1}^{N}\left[\frac{2N}{n}\right] = \sum_{n=N+1}^{2N}\left[\frac{2N}{n}\right] = \sum_{n=N+1}^{2N}1 = N.$$
        \item Using Corollary 1:
        \begin{align*}
            \sum_{n=1}^{N}\left(\left[\frac{N}{n}\right] - \left[\frac{N_1}{n}\right]\right) &= \sum_{n=1}^{N}\left[\frac{N}{n}\right] - \sum_{n=1}^{N}\left[\frac{N-1}{n}\right] \\
            &= \sum_{n=1}^{N}\tau(n) - \sum_{n=1}^{N-1}\left[\frac{N-1}{n}\right] - \left[\frac{N - 1}{N}\right] \\
            &= \sum_{n=1}^{N}\tau(n) - \sum_{n=1}^{N-1}\tau(n) - 0\\
            &= \tau(N).
        \end{align*}
    \end{enumerate}
\end{solution}