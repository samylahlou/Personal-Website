\section{Dimension}

\begin{exercise}
    Show that the subspaces of $\R^2$ are precisely $\{0\}$, all lines in $\R^2$ containing the origin, and $\R^2$. \\
\end{exercise}

\begin{solution}
    \\ Let $V$ be a subspace of $\R^2$. Since $\R^2$ has dimension 2, then $0 \leq \dim V \leq 2$ by Proposition 2.37. If $V$ has dimension 0, then its basis must be the empty set. In that case, $V = \text{span}(\varnothing) = \{0\}$. If $V$ has dimension 1, then its basis must contain a single non-zero vector. It follows that 
    $$V = \text{span}(u) = \{\lambda u : \lambda \in \R\}.$$
    But notice that $\lambda u$ with $\lambda \in \R$ is simply the equation of a line with direction vector $u$ passing through the origin (take $\lambda = 0$), hence, $V$ is a line passing through the origin. Finaly, if $V$ has dimension 2, then by Proposition 2.39, $V = \R^2$. Therefore, since $V$ was an arbitrary subspace, then subspaces of $\R^2$ are $\{0\}$, all lines in $\R^2$ containing the origin, and $\R^2$.\\
    To prove that the subspaces of $\R^2$ are precisely these subsets, let's show that any of these subsets are subspaces. Trivialy, $\{0\}$ and $\R^2$ are indeed subspaces of $\R^2$. Now, let $L$ be a line in $\R^2$ passing through the origin, then $L$ must have a direction vector $u$. Moreover, if $L$ passes through the point $P$, then we can write
    $$L = \{P + \lambda u : \lambda\}$$
    Since $L$ contains the origin, then we can take $P = 0$ which implies that
    $$L = \{\lambda u : \lambda\} = \text{span}(u)$$
    which is a subspace of $\R^2$. Therefore, the subspaces of $\R^2$ are precisely $\{0\}$, all lines in $\R^2$ containing the origin, and $\R^2$.\\
\end{solution}

\begin{exercise}
    Show that the subspaces of $\R^3$ are precisely $\{0\}$, all lines in $\R^3$ containing the origin, all planes in $\R^3$ containing the origin, and $\R^3$. \\
\end{exercise}

\begin{solution}
    \\ Let $V$ be a subspace of $\R^3$. Since $\R^3$ has dimension 3, then $0 \leq \dim V \leq 3$ by Proposition 2.37. If $V$ has dimension 0, then its basis must be the empty set. In that case, $V = \text{span}(\varnothing) = \{0\}$. If $V$ has dimension 1, then its basis must contain a single non-zero vector. It follows that 
    $$V = \text{span}(u) = \{\lambda u : \lambda \in \R\}.$$
    But notice that $\lambda u$ with $\lambda \in \R$ is simply the equation of a line with direction vector $u$ passing through the origin (take $\lambda = 0$), hence, $V$ is a line passing through the origin. If $V$ has dimension 2, then there are two linearly independent vectors $u_1, u_2$ such that
    $$V = \text{span}(u_1, u_2) = \{\alpha u_1 + \beta u_2 : \alpha, \beta \in \R\}$$ 
    But notice that $\alpha u_1 + \beta u_2$ is simply the equation of a plane containing the origin described by the two vectors $u_1$ and $u_2$. Hence, $V$ is a plane containing the origin. Finaly, if $V$ has dimension 3, then by Proposition 2.39, $V = \R^3$. Thus, subspaces of $\R^3$ are $\{0\}$, lines and planes containing the origin and $\R^3$.\\
    To prove that the subspaces of $\R^3$ are precisely these subsets, let's show that any of these subsets are subspaces. Trivialy, $\{0\}$ and $\R^3$ are indeed subspaces of $\R^3$. Now, let $L$ be a line in $\R^3$ passing through the origin, then $L$ must have a direction vector $u$. Moreover, if $L$ passes through the point $P$, then we can write
    $$L = \{P + \lambda u : \lambda\}$$
    Since $L$ contains the origin, then we can take $P = 0$ which implies that
    $$L = \{\lambda u : \lambda\} = \text{span}(u)$$
    which is a subspace of $\R^3$. Similarly, if $P$ is a plane in $\R^3$ containing the origin, then it must contain two linearly independent vectors $u_1, u_2$ that describe the orientation of the plane. Moreover, if $A$ is a point on the plan $P$, then the vectors in the $P$ are described by the equation
    $$A + \alpha u_1 + \beta u_2$$
    Since $P$ contains the origin, then take $A = 0$ to get:
    $$P = \{\alpha u_1 + \beta u_2 : \alpha, \beta \in \R\} = \text{span}(u_1, u_2)$$
    It follows that $P$ is a subspace of $\R^3$. Therefore, the subspaces of $\R^2$ are precisely $\{0\}$, all lines in $\R^2$ containing the origin, and $\R^2$.\\
\end{solution}

\begin{exercise}
    \vspace*{-0.6cm}
    \begin{enumerate}[label=(\alph*)]
        \item Let $U = \{p \in \mathcal{P}_4(\F):p(6) = 0\}$. Find a basis of $U$.
        \item Extend the basis in (a) to a basis of $\mathcal{P}_4(\F)$.
        \item Find a subspace $W$ of $\mathcal{P}_4(\F)$ such that $\mathcal{P}_4(\F) = U \oplus W$.\\
    \end{enumerate}
\end{exercise}

\begin{solution}
    \begin{enumerate}[label=(\alph*)]
        \item Consider the list $p_1, p_2, p_3, p_4$ defined by
        \begin{align*}
            p_1(x) = x-6 \qquad &\qquad p_2(x) = x^2-6x \\
            p_3(x) = x^3-6x^2 \qquad &\qquad p_4(x) = x^4-6x^3 
        \end{align*}
        This list spans $U$ because given any $p\in U$, then $p$ is a polynomial of degree four that has 6 as a root. Hence, we can factorize $x-6$ such that $p(x) = (x-6)(ax^3 + bx^2 + cx+d)$ for some $a,b,c,d \in \F$. Thus:
        \begin{align*}
            p(x) &= (x-6)(ax^3 + bx^2 + cx+d) \\
            &= a(x^4 - 6x^3) + b(x^3 - 6x^2) + c(x^2 - 6x) + d(x-6) \\
            &= ap_4(x) + bp_3(x) + cp_2(x) + dp_1(x)
        \end{align*}
        which shows that the list spans $U$. To show that it is linearly independent, take scalars $a,b,c,d \in \F$ and notice that
        \begin{align*}
            &ap_4(x) + bp_3(x) + cp_2(x) + dp_1(x) = 0 \\
            \implies \ & a(x^4 - 6x^3) + b(x^3 - 6x^2) + c(x^2 - 6x) + d(x-6) = 0\\
            \implies \ & ax^4 + (b - 6a)x^3 + (c - 6b)x^2 + (d - 6c)x + (-6d) = 0 \\
            \implies \ & \begin{cases}
                a=0 \\ b-6a = 0 \\ c - 6b = 0 \\ d - 6c = 0 \\ -6d = 0
            \end{cases} \\
            \implies \ & a=b=c=d=0
        \end{align*}
        Therefore, $p_1, p_2, p_3, p_4$ is a basis of $U$.
        \item Since the list $p_1, p_2, p_3, p_4$ is linearly independent in $U$, then it must be linearly independent in $\mathcal{P}_4(\F)$. Hence, we can extend it to a basis of $\mathcal{P}_4(\F)$. To do so, we only need to add one single polynomial to our list because we already know a basis of $\mathcal{P}_4(\F)$ of size 5: 1, $x$, $x^2$, $x^3$, $x^4$. It is easy to notice that the constant polynomial 1 cannot be written as a linear combination of $p_1, p_2, p_3, p_4$ because for any scalars $a,b,c,d \in \F$:
        \begin{align*}
            &ap_4(x) + bp_3(x) + cp_2(x) + dp_1(x) = 1 \\
            \implies \ & a(x^4 - 6x^3) + b(x^3 - 6x^2) + c(x^2 - 6x) + d(x-6) = 1\\
            \implies \ & ax^4 + (b - 6a)x^3 + (c - 6b)x^2 + (d - 6c)x + (-6d) = 1 \\
            \implies \ & \begin{cases}
                a=0 \\ b-6a = 0 \\ c - 6b = 0 \\ d - 6c = 0 \\ -6d = 1
            \end{cases} \\
            \implies \ & a=b=c=d=0 \text{ and } d -\frac{1}{6}\\
        \end{align*}
        A contradiction. Therefore, by Section 2A Exercise 13, the list $1, p_1, p_2, p_3, p_4$ is linearly independent. Since the list has length 5, then it must be a basis by Proposition 2.38.
        \item Since $1, p_1, p_2, p_3, p_4$ is a basis of $\mathcal{P}_4(\F)$, then we can easily get
        $$\mathcal{P}_4(\F) = U \oplus \F$$
        where $\F$ denotes the set of constant polynomials. \\
    \end{enumerate}
\end{solution}

\begin{exercise}
    \vspace*{-0.6cm}
    \begin{enumerate}[label=(\alph*)]
        \item Let $U = \{p \in \mathcal{P}_4(\F):p''(6) = 0\}$. Find a basis of $U$.
        \item Extend the basis in (a) to a basis of $\mathcal{P}_4(\F)$.
        \item Find a subspace $W$ of $\mathcal{P}_4(\F)$ such that $\mathcal{P}_4(\F) = U \oplus W$.\\
    \end{enumerate}
\end{exercise}

\begin{solution}
    \begin{enumerate}[label=(\alph*)]
        \item Consider the list $p_1, p_2, p_3, p_4$ defined by
        \begin{align*}
            p_1(x) = 1 \qquad &\qquad p_2(x) = x \\
            p_3(x) = \frac{1}{6}x^3 - 3x^2 \qquad &\qquad p_4(x) = \frac{1}{12}x^4-x^3 
        \end{align*}
        To show that it is linearly independent, take scalars $a,b,c,d \in \F$ and notice that
        \begin{align*}
            &ap_4(x) + bp_3(x) + cp_2(x) + dp_1(x) = 0 \\
            \implies \ & a\left(\frac{1}{12}x^4-x^3\right)  + b\left(\frac{1}{6}x^3-3x^2\right) + cx + d = 0\\
            \implies \ & \frac{a}{12}x^4 + \left(\frac{b}{6} - a\right)x^3 + (-3b)x^2 + cx + d = 0 \\
            \implies \ & \begin{cases}
                \frac{a}{12}=0 \\ \frac{b}{6}-a = 0 \\ - 3b = 0 \\ c = 0 \\ d = 0
            \end{cases} \\
            \implies \ & a=b=c=d=0
        \end{align*}
        Therefore, $p_1, p_2, p_3, p_4$ is linearly independent in $U$. To prove that it is a basis, consider its span. If $p_1, p_2, p_3, p_4$ don't span $U$, then we must be able to extend it to a basis of $U$. However, since $\dim U \leq \dim \mathcal{P}_4(\F) = 5$, then we can add only one polynomial. In this case, $U$ has a basis of length 5 which implies that $U = \mathcal{P}_4(\F)$ by Proposition 2.39, a contradiction since $x^2 \notin U$. It follows that the list $p_1, p_2, p_3, p_4$ must span $U$. Therefore, it is a basis of $U$.
        \item Since the list $p_1, p_2, p_3, p_4$ is linearly independent in $U$, then it must be linearly independent in $\mathcal{P}_4(\F)$. Hence, we can extend it to a basis of $\mathcal{P}_4(\F)$. To do so, we only need to add one single polynomial to our list because we already know a basis of $\mathcal{P}_4(\F)$ of size 5: 1, $x$, $x^2$, $x^3$, $x^4$. It is easy to notice that the polynomial $x^2$ cannot be written as a linear combination of $p_1, p_2, p_3, p_4$ since $x^2 \notin U$. Therefore, by Section 2A Exercise 13, the list $x^2, p_1, p_2, p_3, p_4$ is linearly independent. Since the list has length 5, then it must be a basis by Proposition 2.38.
        \item Since $x^2, p_1, p_2, p_3, p_4$ is a basis of $\mathcal{P}_4(\F)$, then we can easily get
        $$\mathcal{P}_4(\F) = U \oplus \F x^2$$
        where $\F x^2$ denotes the span of the polynomial $x^2$. \\
    \end{enumerate}
\end{solution}

\begin{exercise}
    \vspace*{-0.6cm}
    \begin{enumerate}[label=(\alph*)]
        \item Let $U = \{p \in \mathcal{P}_4(\F):p(2) = p(5)\}$. Find a basis of $U$.
        \item Extend the basis in (a) to a basis of $\mathcal{P}_4(\F)$.
        \item Find a subspace $W$ of $\mathcal{P}_4(\F)$ such that $\mathcal{P}_4(\F) = U \oplus W$.\\
    \end{enumerate}
\end{exercise}

\begin{solution}
    \begin{enumerate}[label=(\alph*)]
        \item Consider the list $p_1, p_2, p_3, p_4$ defined by
        \begin{align*}
            p_1(x) = 1 \qquad &\qquad p_2(x) = (x- 2)(x-5) \\
            p_3(x) = x(x- 2)(x-5) \qquad &\qquad p_4(x) = x^2 (x- 2)(x-5)
        \end{align*}
        To show that it is linearly independent, take scalars $a,b,c,d \in \F$ and notice that
        \begin{align*}
            &ap_4(x) + bp_3(x) + cp_2(x) + dp_1(x) = 0 \\
            \implies \ & a(x^4 - 7x^3 + 10x^2)  + b(x^3 - 7x^2 + 10x) + c(x^2 - 7x + 10) + d = 0\\ 
            \implies \ & ax^4 + (b - 7a)x^3 + (c-7b + 10a)x^2 + (10b - 7c)x + (10c+d) = 0 \\
            \implies \ & \begin{cases}
                a=0 \\ b-7a = 0 \\ c-7b+10a = 0 \\ 10b-7c = 0 \\ 10c+d = 0
            \end{cases} \\
            \implies \ & a=b=c=d=0
        \end{align*}
        Therefore, $p_1, p_2, p_3, p_4$ is linearly independent in $U$. To prove that it is a basis, consider its span. If $p_1, p_2, p_3, p_4$ don't span $U$, then we must be able to extend it to a basis of $U$. However, since $\dim U \leq \dim \mathcal{P}_4(\F) = 5$, then we can add only one polynomial. In this case, $U$ has a basis of length 5 which implies that $U = \mathcal{P}_4(\F)$ by Proposition 2.39, a contradiction since $x^2 \notin U$. It follows that the list $p_1, p_2, p_3, p_4$ must span $U$. Therefore, it is a basis of $U$.
        \item Since the list $p_1, p_2, p_3, p_4$ is linearly independent in $U$, then it must be linearly independent in $\mathcal{P}_4(\F)$. Hence, we can extend it to a basis of $\mathcal{P}_4(\F)$. To do so, we only need to add one single polynomial to our list because we already know a basis of $\mathcal{P}_4(\F)$ of size 5: 1, $x$, $x^2$, $x^3$, $x^4$. It is easy to notice that the polynomial $x^2$ cannot be written as a linear combination of $p_1, p_2, p_3, p_4$ since $x^2 \notin U$. Therefore, by Section 2A Exercise 13, the list $x^2, p_1, p_2, p_3, p_4$ is linearly independent. Since the list has length 5, then it must be a basis by Proposition 2.38.
        \item Since $x^2, p_1, p_2, p_3, p_4$ is a basis of $\mathcal{P}_4(\F)$, then we can easily get
        $$\mathcal{P}_4(\F) = U \oplus \F x^2$$
        where $\F x^2$ denotes the span of the polynomial $x^2$. \\
    \end{enumerate}
\end{solution}

\begin{exercise}
    \vspace*{-0.6cm}
    \begin{enumerate}[label=(\alph*)]
        \item Let $U = \{p \in \mathcal{P}_4(\F):p(2) = p(5) = p(6)\}$. Find a basis of $U$.
        \item Extend the basis in (a) to a basis of $\mathcal{P}_4(\F)$.
        \item Find a subspace $W$ of $\mathcal{P}_4(\F)$ such that $\mathcal{P}_4(\F) = U \oplus W$.\\
    \end{enumerate}
\end{exercise}

\begin{solution}
    \begin{enumerate}[label=(\alph*)]
        \item Consider the list $p_1, p_2, p_3$ defined by
        \begin{align*}
            p_1(x) &= 1 \\
            p_2(x) &= (x-2)(x-5)(x-6) \\
            p_3(x) &= x(x-2)(x-5)(x-6)
        \end{align*}
        To show that it is linearly independent, take scalars $a,b,c \in \F$ and notice that
        \begin{align*}
            &ap_3(x) + bp_2(x) + cp_1(x) = 0 \\
            \implies \ & a(x^4-13x^3+52x^2-60x)  + b(x^3-13x^2+52x-60) + c = 0\\ 
            \implies \ & ax^4 + (b - 13a)x^3 + (52a-13b)x^2 + (52b - 60a)x + (c-60b) = 0 \\
            \implies \ & \begin{cases}
                a=0 \\ b - 13a = 0 \\ 52a-13b = 0 \\ 52b - 60a = 0 \\ c-60b = 0
            \end{cases} \\
            \implies \ & a=b=c=0
        \end{align*}
        Therefore, $p_1, p_2, p_3$ is linearly independent in $U$. Let's now show that it spans $U$. To do so, let $p$ be an arbitrary polynomial in $U$, then $p - p(2)$ must have roots at $x=2,5,6$ which means that
        $$p(x) - p(2)= (x-2)(x-5)(x-6)q(x)$$
        where $q$ is a polynomial of degree 1. Thus, there exist scalars $a,b \in \F$ such that $q(x) = ax+b$. Thus,
        \begin{align*}
            p(x) &= (x-2)(x-5)(x-6)q(x) + p(2) \\
            &= (x-2)(x-5)(x-6)(ax+b) + p(2) \\
            &= ax(x-2)(x-5)(x-6) + b(x-2)(x-5)(x-6) + p(2) \\
            &= ap_3(x) + bp_2(x) + p(2)p_1(x) \\
            &\in \text{span}(p_1, p_2, p_3)
        \end{align*}
        Therefore, $p_1, p_2, p_3$ is a basis for $U$.
        \item Since the list $p_1, p_2, p_3$ is linearly independent in $U$, then it must be linearly independent in $\mathcal{P}_4(\F)$. Hence, we can extend it to a basis of $\mathcal{P}_4(\F)$. To do so, we need to add two polynomials to our list because $\dim \mathcal{P}_4(\F) = 5$. It is easy to see that the polynomial $x$ cannot be written as a linear combination of $p_1, p_2, p_3$ since $x \notin U$. Therefore, by Section 2A Exercise 13, the list $x, p_1, p_2, p_3$ is linearly independent. Let's add one last polynomial to our list to make it a basis of $U$. Suppose that $x^2 \in \text{span}(x, p_1, p_2, p_3)$, then there exist scalars $a,b,c,d \in \F$ such that
        $$x^2 = ax + bp_1(x) + cp_2(x) + dp_3(x)$$
        But notice that $bp_1 + cp_2 + dp_3 \in U$ so just define it as $p_U$, then we have
        $$x^2 = ax + p_U(x)$$
        where $p_U(2) = p_U(5) = p_U(6)$. This, if we plug-in $x=2, 5, 6$, we get the following system of equations:
        \begin{align*}
            & \begin{cases}
                4 = 2a + p_U(2)\\ 25 = 5a + p_U(2) \\ 36 = 6a + p_U(2)
            \end{cases} \\
            \implies \ & \begin{cases}
                21 = 3a \\ 11 = a
            \end{cases}
        \end{align*}
        A contradiction that shows that $x^2$ is not in the span of $x,p_1, p_2, p_3$. Therefore, by Section 2A Exercise 13, the list $x, x^2, p_1, p_2, p_3$ is linearly independent. Since the list has length 5, then it must be a basis by Proposition 2.38.
        \item Since $x, x^2, p_1, p_2, p_3$ is a basis of $\mathcal{P}_4(\F)$, then we can easily get
        $$\mathcal{P}_4(\F) = U \oplus \text{span}(x, x^2)$$\\
    \end{enumerate}
\end{solution} 

\begin{exercise}
    \vspace*{-0.6cm}
    \begin{enumerate}[label=(\alph*)]
        \item Let $U = \{p \in \mathcal{P}_4(\F):\int_{-1}^{1}p = 0\}$. Find a basis of $U$.
        \item Extend the basis in (a) to a basis of $\mathcal{P}_4(\F)$.
        \item Find a subspace $W$ of $\mathcal{P}_4(\F)$ such that $\mathcal{P}_4(\F) = U \oplus W$.\\
    \end{enumerate}
\end{exercise}

\begin{solution}
    \begin{enumerate}[label=(\alph*)]
        \item Consider the list $p_1, p_2, p_3, p_4$ defined by
        \begin{align*}
            p_1(x) = x \qquad &\qquad p_2(x) = x^2-\frac{1}{3} \\
            p_3(x) = x^3 \qquad &\qquad p_4(x) = x^4 - \frac{1}{5}
        \end{align*}
        To show that it is linearly independent, take scalars $a,b,c,d \in \F$ and notice that
        \begin{align*}
            &ap_4(x) + bp_3(x) + cp_2(x) + dp_1(x) = 0 \\
            \implies \ & a\left(x^4 - \frac{1}{5}\right)  + bx^3 + c\left(x^2 - \frac{1}{3}\right) + dx = 0\\ 
            \implies \ & ax^4 + bx^3 + cx^2 + dx - \left(\frac{a}{5} +\frac{c}{3}\right) = 0 \\
            \implies \ & \begin{cases}
                a=0 \\ b = 0 \\ c = 0 \\ d = 0 \\ \frac{a}{5} +\frac{c}{3}= 0
            \end{cases} \\
            \implies \ & a=b=c=d=0
        \end{align*}
        Therefore, $p_1, p_2, p_3, p_4$ is linearly independent in $U$. To prove that it is a basis, consider its span. If $p_1, p_2, p_3, p_4$ don't span $U$, then we must be able to extend it to a basis of $U$. However, since $\dim U \leq \dim \mathcal{P}_4(\F) = 5$, then we can add only one polynomial. In this case, $U$ has a basis of length 5 which implies that $U = \mathcal{P}_4(\F)$ by Proposition 2.39, a contradiction since $1 \notin U$. It follows that the list $p_1, p_2, p_3, p_4$ must span $U$. Therefore, it is a basis of $U$.
        \item Since the list $p_1, p_2, p_3, p_4$ is linearly independent in $U$, then it must be linearly independent in $\mathcal{P}_4(\F)$. Hence, we can extend it to a basis of $\mathcal{P}_4(\F)$. To do so, we only need to add one single polynomial to our list because $\dim \mathcal{P}_4(\F) = 5$. It is easy to notice that the polynomial 1 cannot be written as a linear combination of $p_1, p_2, p_3, p_4$ since $1 \notin U$. Therefore, by Section 2A Exercise 13, the list $1, p_1, p_2, p_3, p_4$ is linearly independent. Since the list has length 5, then it must be a basis by Proposition 2.38.
        \item Since $1, p_1, p_2, p_3, p_4$ is a basis of $\mathcal{P}_4(\F)$, then we can easily get
        $$\mathcal{P}_4(\F) = U \oplus \F$$
        where $\F$ denotes the subspace of constant polynomials. \\
    \end{enumerate}
\end{solution}

\begin{exercise}
    Suppose $v_1, ..., v_m$ is linearly independent in $V$ and $w \in V$. Prove that
    $$\dim \text{span}(v_1 + w, ..., v_m + w) \geq m-1.$$
\end{exercise}

\begin{solution}
    \\ Consider the list $v_1 + w, ..., v_m + w$ of length $m$ and suppose by contradiction that
    $$\dim \text{span}(v_1 + w, ..., v_m + w) = d \leq m-2,$$
    then there exists a linearly independent list of vectors $u_1, ..., u_d$ in $V$ such that
    $$\text{span}(v_1 + w, ..., v_m + w) = \text{span}(u_1, ..., u_d).$$
    For each $i \in \{1, ..., n\}$, the previous equation implies that
    $$v_i + w = \alpha_1 u_1 + ... + \alpha_d u_d,$$
    which itself implies that
    $$v_i = \alpha_1 u_1 + ... + \alpha_d u_d - w \in \text{span}(u_1, ..., u_d, w).$$
    Since it holds for all $i \in \{1, ..., n\}$, then
    $$\{v_1, ..., v_m\} \subset \text{span}(u_1, ..., u_d, w) \implies \text{span}(v_1, ..., v_m) \leq \text{span}(u_1, ..., u_d, w).$$
    Since subspaces have a dimensions less than the vector space they are contained in (Proposition 2.37), then
    \[ \dim \text{span}(v_1, ..., v_m) \leq \dim \text{span}(u_1, ..., u_d, w). \tag*{(1)} \]
    The $v_i$'s are linearly independent so they form a basis for their span. It follows that
    \[\dim \text{span}(v_1, ..., v_m) = m. \tag*{(2)} \]
    Moreover, the list $u_1, ..., u_d, w$ is spanning its span (obviously), so it must contain a basis. Since the list has length $d+1$, then the dimension of the span must be less than $d+1$. But since $d \leq m-2$, then
    \[\dim \text{span}(u_1, ..., u_d, w) \leq m-1. \tag*{(3)} \]
    Combining equations (1), (2) and (3) gives us
    $$m \leq m-1$$
    which is clearly a contradiction. Therefore,
    $$\dim \text{span}(v_1 + w, ..., v_m + w) \geq m-1.$$\\
\end{solution}

\begin{exercise}
    Suppose $m$ is a positive integer and $p_0, p_1, ..., p_m \in \mathcal{P}(\F)$ are such that each $p_k$ has degree $k$. Prove that $p_0, p_1, ..., p_m$ is a basis of $\mathcal{P}_m(\F)$.\\
\end{exercise}

\begin{solution}
    \\ Since the list $p_0,p_1, ..., p_m$ has length $m+1$ and we already know that $\dim \mathcal{P}_m(\F) = m+1$, then it suffices to show that the list is linearly independent by Proposition 2.38. Let's prove by induction on $m$ that any list $p_0, p_1, ..., p_m$ such that each $p_k$ has degree $k$ must be linearly independent. \\
    For the base case, take $m = 0$ and consider the list $p_0$ where $p_0$ is a polynomial of degree $0$. Hence $p_0$ must be a nonzero constant polynomial (since the zero polynomial has degree $-\infty$). But we know from Section 2A Exercise 4(a) that the list containing $p_0$ only must be linearly independent since it is nonzero. This proves that the statement holds for $m=0$. \\
    Suppose now that it holds for an integer $m \geq 0$ and consider the list $p_0, p_1, ..., p_{m+1}$ such that each $p_k$ has degree $k$. By our assumption, we know that the list $p_0, p_1, ..., p_m$ is linearly independent. If we take arbitrary scalars $\alpha_0, ..., \alpha_{m+1}$ satisfying
    $$\alpha_0 p_0 + \alpha_1 p_1 + ... \alpha_{m+1} p_{m+1} = 0,$$
    then notice that $\alpha_{m+1}$ must be equal to zero since $p_{m+1}$ is only polynomial in the linear combination containing a $x^{m+1}$ term. Thus, we get that
    $$\alpha_0 p_0 + \alpha_1 p_1 + ... \alpha_m p_m = 0.$$
    But by linear independence of the $p_i$'s, we know that $\alpha_0 = ... = \alpha_m = 0$. It follows that the new list is linearly independent as well. Therefore, by induction, all such lists must be linearly independent and hence, a basis for $\mathcal{P}_m(\F)$.\\ 
\end{solution}

\begin{exercise}
    Suppose $m$ is a positive integer. For $0 \leq k \leq m$, let 
    $$p_k(x) = x^k(1-x)^{m-k}.$$
    Show that $p_0, ..., p_m$ is a basis of $\mathcal{P}_m(\F)$.\\
\end{exercise}

\begin{solution}
    \\ Since the list $p_0,p_1, ..., p_m$ has length $m+1$ and we already know that $\dim \mathcal{P}_m(\F) = m+1$, then it suffices to show that the list is linearly independent by Proposition 2.38. \\
    Let $\alpha_0, \alpha_1, ..., \alpha_m \in \F$ be scalars such that
    $$\alpha_0 p_0 + \alpha_1 p_1 + ... + \alpha_m p_m = 0.$$
    The polynomial on the left hand side is equal to zero, this implies that the coefficients in front of each monomial of the form $x^k$ are zero. Given a $k$ between 0 and $m$, notice that the polynomial $p_k$ can be written as
    $$p_k(x) = x^k \sum_{i=0}^{m-k}\binom{m-k}{i}(-1)^ix^i = \sum_{i=k}^{m}\binom{m-k}{i}(-1)^{i-k}x^i$$
    using the Binomial Formula. It follows that the lowest degree term in $p_k$ is $x^k$. Therefore, in the list $p_0, ..., p_m$, $p_0$ is the only polynomial containing a constant term. Hence, the constant term in the polynomial $\alpha_0 p_0 + \alpha_1 p_1 + ... + \alpha_m p_m$ is $\alpha_0$. It follows that $\alpha_0 = 0$. Thus, we now have the equation
    $$\alpha_1 p_1 + ... + \alpha_m p_m = 0.$$
    In the list $p_1, ..., p_m$, the polynomial $p_1$ is the only polynomial containing the term $x^1$. Thus, the coefficient in front of the term $x^1$ in the polynomial $\alpha_1 p_1 + ... + \alpha_m p_m$ is $\alpha_1$. It follows that $\alpha_1 = 0$. If we continue in this manner, we can prove by induction that all the $\alpha_i$'s are zero. Therefore, the list is linearly independent and hence, a basis of $\mathcal{P}_m(\F)$.\\
\end{solution}

\begin{exercise}
    Suppose $U$ and $V$ are both four-dimensional subspaces of $\C^6$. Prove that there exist two vectors in $U \cap W$ such that neither of these vectors is a scalar multiple of the other. \\
\end{exercise}

\begin{solution}
    \\ By Proposition 2.43, we know that
    \[ \dim(U+W) = \dim U + \dim W - \dim(U \cap W) \tag*{(1)} \]
    Since $U + W \leq V$, then by Proposition 2.37, $\dim (U+W) \leq \dim V = 6$. Thus, if we substitute this inequality and the known values into equation (1), we get: 
    $$6 \geq 4 + 4 - \dim (U \cap W),$$
    which can be rearranged into
    $$\dim(U\cap W) \geq 2.$$
    Thus, if we denote by $d$ the dimension of $U \cap W$, then there exists a basis $v_1, ..., v_d$ of $U \cap W$. Since it is a basis and $d \geq 2$, then we can take the vectors $v_1, v_2 \in U\cap W$ and assert that they are linearly independent. Therefore, there exist two vectors in $U \cap W$ such that neither of these vectors is a scalar multiple of the other (by linear independence). \\
\end{solution}

\begin{exercise}
    Suppose that $U$ and $V$ are subspaces of $\R^8$ such that $\dim U = 3$, $\dim W = 5$, and $U+W = \R^8$. Prove that $\R^8 = U \oplus W$. \\
\end{exercise}

\begin{solution}
    \\ Since we already know that $U + W = \R^8$, it suffices to prove that $U \cap W = \{0\}$. To do so, notice that $U + W = \R^8$ implies $\dim (U + W) = 8$. Using the formula in Proposition 2.43, we get
    $$\dim U + \dim V - \dim(U\cap W) = 8.$$
    If we plug-in the known values, we get
    $$4 + 4 - \dim(U \cap W) = 8,$$
    which can be rearranged into
    $$\dim (U \cap W) = 0.$$
    But the only zero-dimensional vector space is the trivial vector space $\{0\}$. Hence, $U \cap W = \{0\}$. Therefore, $\R^8 = U \oplus W$. \\
\end{solution}

\begin{exercise}
    Suppose $U$ and $W$ are both five-dimensional subspaces of $\R^9$. Prove that $U \cap W \neq \{0\}$. \\
\end{exercise}

\begin{solution}
    \\ Since $U + W \leq \R^9$, then by Proposition 2.37, $\dim (U+W) \leq \dim \R^9 = 9$. Using the formula in Proposition 2.43, we get
    $$\dim U + \dim V - \dim(U\cap W) \leq 9.$$
    If we plug-in the known values, we get
    $$5 + 5 - \dim(U \cap W) \leq 9,$$
    which can be rearranged into
    $$\dim (U \cap W) \geq 1.$$
    Therefore, $U \cap W$ cannot be $\{0\}$ since otherwise, its dimension would be 0.\\ 
\end{solution}

\begin{exercise}
    Suppose $V$ is a ten-dimensional vector space and $V_1, V_2, V_3$ are subspaces of $V$ with $\dim V_1 = \dim V_2 = \dim V_3 = 7$. Prove that $V_1 \cap V_2 \cap V_3 \neq \{0\}$. \\
\end{exercise}

\begin{solution}
    \\ First, consider the subspace $V_1 \cap V_2$. Since $V_1 + V_2 \leq V$, then $\dim (V_1 + V_2) \leq 10$. It follows that
    $$\dim(V_1 \cap V_2) = \dim V_1 + \dim V_2 - \dim (V_1 + V_2) \geq 7 + 7 - 10 = 4.$$
    Now, consider the subspace $V_1 \cap V_2 \cap V_3$ as the intersection between $V_1 \cap V_2$ and $V_3$. Since $(V_1 \cap V_2) + V_3 \leq V$, then $\dim((V_1 \cap V_2) + V_3) \leq 10$. It follows that
    \begin{align*}
        \dim (V_1 \cap V_2 \cap V_3) &= \dim(V_1 \cap V_2) + \dim V_3 - \dim ((V_1 \cap V_2) + V_3) \\
        &\geq 4 + 7 - 10 \\
        &= 1
    \end{align*}
    Thus, $V_1 \cap V_2 \cap V_3$ cannot be $\{0\}$ since otherwise, its dimension would be 0. \\
\end{solution}

\begin{exercise}
    Suppose $V$ is finite-dimensional and $V_1, V_2, V_3$ are subspaces of $V$ with $\dim V_1 + \dim V_2 + \dim V_3 > 2\dim V$. Prove that $V_1 \cap V_2 \cap V_3 \neq \{0\}$. \\
\end{exercise}

\begin{solution}
    \\ First, consider the subspace $V_1 \cap V_2$. Since $V_1 + V_2 \leq V$, then $\dim (V_1 + V_2) \leq \dim V$. It follows that
    $$\dim(V_1 \cap V_2) = \dim V_1 + \dim V_2 - \dim (V_1 + V_2) \geq \dim V_1 + \dim V_2 - \dim V.$$
    Now, consider the subspace $V_1 \cap V_2 \cap V_3$ as the intersection between $V_1 \cap V_2$ and $V_3$. Since $(V_1 \cap V_2) + V_3 \leq V$, then $\dim((V_1 \cap V_2) + V_3) \leq \dim V$. It follows that
    \begin{align*}
        \dim (V_1 \cap V_2 \cap V_3) &= \dim(V_1 \cap V_2) + \dim V_3 - \dim ((V_1 \cap V_2) + V_3) \\
        &\geq \dim V_1 + \dim V_2 - \dim(V_1 + V_2) + \dim V_3 - \dim V \\
        &\geq \dim V_1 + \dim V_2 + \dim V_3 - \dim V - \dim V \\
        &> 2\dim V - 2 \dim V \\
        &= 0
    \end{align*}
    Thus, $V_1 \cap V_2 \cap V_3$ cannot be $\{0\}$ since otherwise, its dimension would be 0. \\
\end{solution}

\begin{exercise}
    Suppose $V$ is finite-dimensional and $U$ is a subspace of $V$ with $U \neq V$. Let $n = \dim V$ and $m = \dim U$. Prove that there exist $n-m$ subspaces of $V$, each of dimension $n-1$, whose intersection equals $U$. \\
\end{exercise}

\begin{solution}
    \\ Let $u_1, ..., u_m$ be a basis of $U$ and extend it to a basis
    $$u_1, ..., u_m, v_1, ..., v_{n-m}$$
    of $V$. For each $i \in \{1, ..., n-m\}$, define the subspace $V_i$ of $V$ as the span of the list $u_1, ..., u_m, v_1, ..., v_{n-m}$ except the vector $v_i$. Hence, $V_i$ is the span of $n - 1$ linearly independent vectors so $\dim V_i = n-1$. Consider now the intersection $V_1 \cap ... \cap V_{n-m}$. Since each $V_i$ is the span of a list containing a basis of $U$, then $U$ is a subspace of all the $V_i$'s. It follows that
    $$U \leq \bigcap_{i=1}^{n-m}V_i.$$
    Let $v$ be an arbitrary vector in $\bigcap_{i=1}^{n-m}V_i$, since $\bigcap_{i=1}^{n-m}V_i \leq V$, then
    $$v = \alpha_1 u_1 + ... + \alpha_m u_m + \beta v_1 + ... + \beta_{n-m} v_{n_m}$$
    for some scalars $\alpha_1, ..., \alpha_m, \beta_1, ..., \beta_{n-m} \in \F$. Let $j \in \{1, ..., n\}$, since $v \in \bigcap_{i=1}^{n-m}V_i$, then $v \in V_j$ in particular. Since $V_j$ is the span of the $u_i$'s and $v_i$'s except $v_j$, then $V_j$ contains
    $$v_0 = \alpha_1 u_1 + ... + \alpha_m u_m + \beta v_1 + ... + \beta_{n-m} v_{n_m}$$
    where $\beta_j = 0$. Hence, $V_j$ is a subspace that contains both $v$ and $v_0$, so it follows that
    $$\beta_j v_j = v - v_0 \in V_j.$$
    If $\beta_j$ is non-zero, then $v_j \in V_j$. But since $V_j$ already contains all the vectors in the basis of $V$ except $v_j$, then $V_j$ contains a basis of $V$. It follows that $V \leq V_j$ so $n \leq n - 1$, a contradiction. Therefore, $\beta_j = 0$. Since it holds for all $j \in \{1, ..., n-m\}$, then all the possibly non-zero coefficients in the linear combination of $v$ are the coefficients in front of the $u_i$'s. Hence:
    $$v = \alpha_1 u_1 + ... + \alpha_m u_m + 0  + ... + 0 \in \text{span}(u_1, ..., u_m) = U.$$
    Since it holds for all $v \in \cap_{i=1}^{n-m}V_i$, then $\cap_{i=1}^{n-m}V_i \leq U$. Therefore, $U = \cap_{i=1}^{n-m}V_i$, which proves that there exist $n-m$ subspaces of $V$, each of dimension $n-1$, whose intersection equals $U$. \\
\end{solution}

\begin{exercise}
    Suppose that $V_1, ..., V_m$ are finite-dimensional subspaces of $V$. Prove that $V_1 + \dots + V_m$ is finite-dimensional and
    $$\dim(V_1 + \dots + V_m) \leq \dim V_1 + \dots + \dim V_m.$$
\end{exercise}

\begin{solution}
    \\ Let $i \in \{1, ..., m\}$ and define $d_i$ as the dimension of $V_i$. Since $V_i$ is finite-dimensional, then it has a finite basis $v_1^{(i)}, ..., v_{d_i}^{(i)}$. Consider now the list which merges all of these bases:
    $$v_1^{(1)}, ..., v_{d_1}^{(1)}, v_1^{(2)}, ..., v_{d_2}^{(2)}, ..., v_1^{(m)}, ..., v_{d_m}^{(m)}$$
    This new list has length $d_1 + d_2 + ... + d_m < \infty$. Moreover, it is easy to see that it is spanning $V_1 + \dots V_m$ since for all $u \in V_1 + \dots V_m$, we can write $u$ as $u_1 + ... + u_m$ where $u_i \in V_i$ for all $i \in \{1, ..., m\}$. Hence, for all $i \in \{1, ..., n\}$, since $v_1^{(i)}, ..., v_{d_i}^{(i)}$ is a basis of $V_i$, then there exist scalars $\alpha_1^{(i)}, ..., \alpha_{d_i}^{(i)} \in \F$ such that
    $$u_i = \alpha_1^{(i)} v_1^{(i)} + ... + \alpha_{d_i}^{(i)} v_{d_i}^{(i)}.$$
    It follows that
    $$u = \alpha_1^{(1)} v_1^{(1)} + ... + \alpha_{d_1}^{(1)} v_{d_1}^{(1)} + ... + \alpha_1^{(m)} v_1^{(m)} + ... + \alpha_{d_m}^{(m)} v_{d_m}^{(m)}$$
    which proves that the list
    $$v_1^{(1)}, ..., v_{d_1}^{(1)}, v_1^{(2)}, ..., v_{d_2}^{(2)}, ..., v_1^{(m)}, ..., v_{d_m}^{(m)}$$
    spans $V_1 + \dots + V_m$. Hence, $V_1 + \dots + V_m$ is finite dimensional since it contains a finite spanning list. Moreover, any spanning list must contain a basis. It follows that there exists a sublist of the one presented that it a basis of $V_1 + \dots + V_m$. This list has a length less than or equal to the length of the presented list (since it is a sublist) which is equal to $d_1 + ... + d_m$. But since it is a basis, then it has length $\dim (V_1 + \dots + V_m)$. Therefore:
    $$\dim (V_1 + \dots + V_m) \leq d_1 + \dots + d_m = \dim V_1 + \dots + \dim V_m.$$ \\
\end{solution}

\begin{exercise}
    Suppose $V$ is finite-dimensional, with $\dim V = n \geq 1$. Prove that there exist one-dimensional subspaces $V_1, ..., V_n$ of $V$ such that
    $$V = V_1 \oplus \dots \oplus V_n.$$
\end{exercise}

\begin{solution}
    \\ Let $v_1, ..., v_n$ be a basis of $V$ and for each $i \in \{1, ..., n\}$, define the subspace $V_i$ as the span of the vector $v_i$. Let's prove that $V = V_1 \oplus \dots \oplus V_n.$ First, it is clear that $V_1 + \dots + V_n \leq V$. Moreover, for all $v \in V$, since $v_1, ..., v_n$ is a basis of $V$, there exist scalars $\alpha_1, ..., \alpha_n \in \F$ such that
    $$v = \alpha_1 v_1 + ... + \alpha_n v_n.$$
    For all $i \in \{1, ..., n\}$, the term $\alpha_i v_i \in V_i$. It follows that $v \in V_1 + \dots + V_n$. Therefore, $V = V_1 + \dots + V_n$. To prove that the sum is direct, it suffices to show that zero has a unique representation as a sum of elements in the $V_i$'s (Proposition 1.45). But this follows from the fact the $v_i$'s are linearly independent since it is a basis. Therefore,
    $$V = V_1 \oplus \dots \oplus V_n.$$ \\
\end{solution}

\begin{exercise}
    Explain why you might guess, motivated by analogy with the formula for the number of elements in the union of three finite sets, that if $V_1, V_2, V_3$ are subspaces of a finite-dimensional vector space, then 
    \begin{align*}
        \dim(V_1 + &V_2 + V_3) = \\
        &= \dim V_1 + \dim V_2 + \dim V_3 \\
        & \ \ \ - \dim(V_1 \cap V_2) - \dim(V_1 \cap V_3) - \dim(V_2 \cap V_3) \\
        & \ \ \ + \dim(V_1 \cap V_2 \cap V_3).
    \end{align*}
    Then either prove the formula above or give a counterexample. \\
\end{exercise}

\begin{solution}
    \\ Given three finite sets $S_1, S_2, S_3$, we can easily derive the following formula for the cardinality of the union of the three sets:
    \begin{align*}
        \#(S_1 \cup S_2 \cup S_3) &= \#(S_1 \cup S_2) + \# S_3 - \# ((S_1 \cup S_2) \cap S_3) \\
        &= \#S_1 + \# S_2 - \# (S_1\cap S_2) + \# S_3 - \# ((S_1 \cap S_3) \cup (S_1 \cap S_3)) \\
        &= \# S_1 + \# S_2 + \# S_3 - \#(S_1 \cap S_2) \\
        & \ \ \ - \# (S_1 \cap S_3) - \# (S_2 \cap S_3) + \# (S_1 \cap S_2 \cap S_3)
    \end{align*}
    Now, using the correspondence between finite sets and finite-dimensional vector spaces, cardinality and dimension, unions and sums, we could guess that the analoguous fomula for finite-dimensional vector spaces is
    \begin{align*}
        \dim(V_1 + V_2 + V_3) &=  \dim V_1 + \dim V_2 + \dim V_3 - \dim(V_1 \cap V_2) \\
        & \ \ \ - \dim (V_1 \cap V_3) - \dim (V_2 \cap V_3) + \dim (V_1 \cap V_2 \cap V_3)
    \end{align*}
    However, this formula is false. To see why, consider the following counterexample: Take $V_1$ to be the span of the vector $(1, 0) \in \R^2$, $V_2$ to be the span of $(0,1) \in \R^2$ and $V_3$ to be the span of $(1,1) \in \R^2$. Since the three vectors span $\R^2$, then
    $$V_1 + V_2 + V_3 = \text{span}((1,0), (0,1), (1,1)) = \R^2.$$
    This implies that $\dim(V_1 + V_2 + V_3) = 2$. Moreover, we also have
    $$\dim V_1 = \dim V_2 = \dim V_3 = 1$$
    and 
    $$V_1 \cap V_2 = V_1 \cap V_3 = V_2 \cap V_3 = V_1 \cap V_2 \cap V_3 = \{0\}.$$
    Thus, if the formula for the dimension of the sum of three subspaces was true, we would have:
    $$2 = 1 + 1 + 1 - 0 - 0 - 0 + 0$$
    which is clearly a contradiction. Therefore, the formula is false. \\
\end{solution}

\begin{exercise}
    Prove that if $V_1, V_2$ and $V_3$ are subspaces of a finite-dimensional vector space, then
    \begin{align*}
        \dim & (V_1 + V_2 + V_3) = \\
        &= \dim V_1 + \dim V_2 + \dim V_3 \\
        & \ \ \ - \frac{\dim(V_1 \cap V_2) + \dim(V_1 \cap V_3) + \dim(V_2 \cap V_3)}{3} \\
        & \ \ \ - \frac{\dim((V_1 + V_2) \cap V_3) + \dim((V_1 + V_3) \cap V_2) + \dim((V_2 + V_3) \cap V_1)}{3}.
    \end{align*}
\end{exercise}

\begin{solution}
    \\ First, recall that $V_1 + V_2 + V_3 = (V_1 + V_2) + V_3$. Thus, using the formula in Proposition 2.43, we get
    $$\dim(V_1 + V_2 + V_3) = \dim(V_1 + V_2) + \dim V_3 - \dim((V_1 + V_2)\cap V_3).$$
    Now, applying the same formula to $\dim (V_1 + V_2)$ gives us
    \begin{align*}
        \dim(V_1 + V_2 + V_3) &= \dim V_1 + \dim V_2 + \dim V_3 \\ & \qquad - \dim(V_1 \cap V_2) - \dim((V_1 + V_2)\cap V_3). \tag*{(1)}
    \end{align*}
    We can repeat this process by writing $V_1 + V_2 + V_3$ as $(V_1 + V_3) + V_2$ to get
    \begin{align*}
        \dim(V_1 + V_2 + V_3) &= \dim V_1 + \dim V_2 + \dim V_3 \\ & \qquad - \dim(V_1 \cap V_3) - \dim((V_1 + V_3)\cap V_2). \tag*{(2)}
    \end{align*}
    Again, by writing $V_1 + V_2 + V_3$ as $(V_2 + V_3) + V_1$, we get
    \begin{align*}
        \dim(V_1 + V_2 + V_3) &= \dim V_1 + \dim V_2 + \dim V_3 \\ & \qquad - \dim(V_2 \cap V_3) - \dim((V_2 + V_3)\cap V_1). \tag*{(3)}
    \end{align*}
    If we add equations (1), (2) and (3) together, we get
    \begin{align*}
        3\dim & (V_1 + V_2 + V_3) = \\
        &= 3\dim V_1 + 3\dim V_2 + 3\dim V_3 \\
        & \ \ \ - \dim(V_1 \cap V_2) - \dim(V_1 \cap V_3) - \dim(V_2 \cap V_3) \\
        & \ \ \ - \dim((V_1 + V_2) \cap V_3) - \dim((V_1 + V_3) \cap V_2) - \dim((V_2 + V_3) \cap V_1).
    \end{align*}
    By dividing by 3 on both sides, we obtain
    \begin{align*}
        \dim & (V_1 + V_2 + V_3) = \\
        &= \dim V_1 + \dim V_2 + \dim V_3 \\
        & \ \ \ - \frac{\dim(V_1 \cap V_2) + \dim(V_1 \cap V_3) + \dim(V_2 \cap V_3)}{3} \\
        & \ \ \ - \frac{\dim((V_1 + V_2) \cap V_3) + \dim((V_1 + V_3) \cap V_2) + \dim((V_2 + V_3) \cap V_1)}{3}.
    \end{align*}
    which is the desired formula.
\end{solution}