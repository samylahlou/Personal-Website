\documentclass{article}

%% Language and font encodings
\usepackage[english]{babel}
\usepackage[utf8x]{inputenc}
\usepackage[T1]{fontenc}
\usepackage{amsfonts}
\usepackage[margin=41mm]{geometry}

%% Sets page size and margins
%\usepackage[a4paper,top=3cm,bottom=2cm,left=3cm,right=3cm,marginparwidth=1.75cm]{geometry}

%% Useful packages
\usepackage{amsmath}
\usepackage{xcolor}
\usepackage{amssymb}
\usepackage{amsthm}
\usepackage{stmaryrd}
\usepackage{graphicx}
\usepackage{enumitem}

%Theorem
\newtheorem*{theorem}{Theorem}
\newtheorem*{proposition}{Proposition}
\newtheorem*{lemma}{Lemma}
\newtheorem*{definition}{Definition}

%Commands definitions
\newcommand{\C}{\mathbb{C}}
\newcommand{\R}{\mathbb{R}}
\newcommand{\Q}{\mathbb{Q}}
\newcommand{\Z}{\mathbb{Z}}
\newcommand{\N}{\mathbb{N}}
\newcommand{\F}{\mathbb{F}}
\newcommand{\E}{\mathbb{E}}
\newcommand{\K}{\mathbb{K}}
\renewcommand{\L}{\mathcal{L}}
\newcommand{\lnorm}[2]{\left\lVert#2 \right\rVert_{#1}}
\newcommand{\norm}[1]{\left\lVert#1 \right\rVert}
\newcommand{\abs}[1]{\left\lvert#1 \right\rvert}
\newcommand{\Iint}[2]{\llbracket #1 , #2 \rrbracket}
\renewcommand{\Im}{\text{Im}}
\newcommand{\card}{\text{card}}
\newcommand{\isomorphic}{\cong}
\newcommand{\Aut}{\text{Aut}}
\newcommand{\td}{\textcolor{red}{\textbf{TODO}}}

%Example environment
\newenvironment{example}{\noindent\textbf{Example:} \vspace{-0.2cm}\begin{itemize}}{\end{itemize}}

%Remark environment
\newenvironment{remark}{\noindent\textbf{Remark:}}{}

%Set QED symbol to blacksquare
\renewcommand\qedsymbol{$\blacksquare$}


\title{MATH 457 Notes : Galois Theory}
\author{Samy Lahlou}
\date{}

\begin{document}

\maketitle

These notes are based on lectures given by Professor Henri Darmon at McGill University in Winter 2025. The subject of these lectures is Representation Theory and Galois Theory but I chose to take notes only for the Galois Theory part. \\
As a disclaimer, it is more than possible that I made some mistakes. Feel free to correct me or ask me anything about the content of this document at the following address : samy.lahloukamal@mcgill.ca

\tableofcontents

\newpage

\section{Fields Extensions}

\begin{definition}[Field Extension]
    If and $\E$ and $\F$ are fields, we say that $E$ is an extension of $F$ if $F$ is a subfield of $E$.
\end{definition}

\begin{remark}
    If $\E$ is an extension of $\F$, then $\E$ is also a vector space over $\F$.
\end{remark}

\begin{definition}
    Given a fields $\E$ and $\F$ and $\alpha \in \E$ where $\E$ is an extension of $\F$, we denote by $\F[\alpha]$ the ring generated by $\F$ and $\alpha$, i.e., $\F[\alpha]$ is the intersection of all the fields containing both $\F$ and $\alpha$. Similarly, we denote by $\F(\alpha)$ the field generated by $\F$ and $\alpha$.\\
    Hence, there is a natural inclusion from $\F[\alpha]$ to $\F(\alpha)$.
\end{definition}

\begin{definition}
    The degree of $\E$ over $\F$ is the dimension of $\E$ as a $\F$ vector space. It is written as $[\E:\F]$. If the degree is finite, we say that $\E/\F$ is finite.
\end{definition}

\begin{example}
    \item $[\C : \R] = 2$ since $\R \subset \C$ and $\C$ is a 2-dimensional $\R$-vector space.
    \item $[\C : \Q] = \infty$ since $\Q \subset \C$ and $\C$ is an $\infty$-dimensional $\Q$-vector space. Using the Axiom of Choice, we can construct a basis for this vector space, it is called the Hamel basis.
    \item Let $\F$ be a field and $\E = \F[x]/(p)$ where $p$ is an irreducible polynomial of degree $n$, then 
    $$\E = \{a_0 + a_1x + ... + a_{n-1}x^{n-1}\}$$
    so $[\E : \F] = n$ since $\E$ contains $\F$ (the constant polynomials) and has basis $\{1, x, ..., x^{n-1}\}$.
    \item Let $\F$ be a field and $\E = \F(x)$ be the fraction field of $\F[x]$, then $[\E : \F] = \infty$.
    \item Given an irreducible polynomial $p$ over $\Q$ and a root $\alpha$ of $p$, then 
    $$\Q[\alpha] = \Q(\alpha) = \Q[x] /(p)$$
    is an extension of $\Q$ of degree $\deg p$. The isomorphism $\Q(\alpha) \isomorphic \Q[x] /(p)$ comes from the valuation map $ev_\alpha : \Q[x] /(p) \to \Q(\alpha)$.
\end{example}

\begin{theorem}[Multiplicativity of the degree]
    Given three fields $\K \subset \F \subset \E$, we have
    $$[\E : \K] = [\E : \F] [\F : \K].$$
\end{theorem}

\begin{proof}
    If one of the degree is infinite, the proof is trivial, hence, assume that the degrees are finite. Call $[\E : \F] = n$ and $[\F : \K] = m$. Let $\alpha_1, ..., \alpha_n \in \F$ be a basis for $\E$ as a $\F$-vector space and $\beta_1, ..., \beta_m \in \K$ be a basis for $\F$ as a $\K$-vector space. Notice that for all $a \in \E$, there exist elements $\lambda_1, ..., \lambda_n \in \F$ such that
    $$a = \lambda_1 \alpha_1  + ... + \lambda_n \alpha_n$$
    is the unique representation of $a$ as a linear combination of the basis $\alpha_1, ..., \alpha_n$. But for each $\lambda_i$, we know that there exist elements $\lambda_{i1}, ..., \lambda_{im} \in \K$ such that 
    $$\lambda_i = \lambda_{i1}\beta_1 + ... + \lambda_{im}\beta_m$$. Thus,
    $$a = \sum_{i=1}^{n}\sum_{j=1}^{m}\lambda_{ij}\alpha_i \beta_j.$$
    Therefore, $\{\alpha_i \beta_j\}_{i,j}$ is a $\K$ basis for $\E$. Hence, it follows that the dimension of $\E$ as $K$-vector space is $n\cdot m$.
\end{proof}

\section{Ruler and Compass Constructions}

\begin{definition}
    A complex number is constructible by ruler and compass if it can be obtained from rational numbers by successive applications of field operations (+, -, $\times$, division) and square roots. Using fields, we can say that a number is cinstructible if it is contained in a sequence of quadratic extensions of $\Q$.
\end{definition}

The set of elements constructible by ruler and compass is an extension of $\Q$ of infinite degree. The goal is to characterize the set of numbers which can be constructible by ruler and compass.

\begin{theorem}
    If $\alpha \in \R$ is a root of an irreducible cubic polynomial over $\Q$, then $\alpha$ is not constructible by ruler and compass.
\end{theorem}

\begin{proof}
    Suppose that $\alpha$ is constructible, then there are finite field extensions 
    $$\Q \subset \F_1 \subset ... \subset \F_n$$
    with $\F_{i+1} = \F_i(\sqrt{a_i})$ for some $a_i \in \F_i$. Hence, for all $i$, we have that $[F_{i+1} : F_i]$ since $\{1, \sqrt{a_i}\}$ is a basis for $F_{i+1}$ as a $\F_i$-vector space. Thus, by multiplicativity of the degree, $[\F_n : \Q] = 2^n$. Moreover, we know that $[\Q(\alpha) : \Q] = 3$ so we get the following diagram : \td. Contradiction.
\end{proof}

\begin{example}
    \item (Duplicating the cube) $p(x) = x^3 - 2$ and $\alpha = \sqrt[3]{2}$ cannot be constructible.
    \item (Trissection of angle) $p(x) = x^3 - 3x + \frac{1}{2}$ and $\alpha = \cos(2\pi/9)$:
    $$\cos(3\theta) = \cos^3 \theta - 3\cos(\theta)(1 - \cos^2\theta)$$
\end{example}

\begin{definition}[Algebraic Numbers]
    Let $\E / \F$ be a finite extension. An element $\alpha \in \E$ is algebraic over $\F$ if $\alpha$ is the root of a polynomial in $\F[x]$.
\end{definition}

\begin{example}
    \item $\sqrt{2} \in \R$ is algebraic over $\Q$ since it solves the polynomial $x^2 - 1 \in \Q[x]$.
    \item $i \in \C$ is algebraic over $\Q$ since it solves the polynomial $x^2 + 1 \in \Q[x]$.
    \item $\pi$ is not algebraic over $\Q$ but it is algebraic over $\Q(\pi^3)$.
    \item The set of $\alpha \in \R$ which are algebraic over $\Q$ is countable (Cantor).
\end{example}

\begin{lemma}
    If $\E / \F$ is a finite extension, then every $\alpha \in \E$ is algebraic over $\F$.
\end{lemma}

\begin{proof}
    Let $n$ be the degree of $\E / \F$, then the set $\{1, \alpha, \alpha^2, ..., \alpha^n\}$ cannot be linearly independent since it contains $n+1$ elements. Hence, there exist scalars \td 
\end{proof}

\begin{definition}[Automorphism Group]
    The automorphism group of $\E / \F$ is
    $$\Aut(\E / \F) = \{\sigma: \E \to \E : \sigma \text{ preserves the operations and } \sigma|_{\F} = \text{id} \}$$
\end{definition}

As a consequence, if $\sigma \in \Aut(\E / \F)$, then $\sigma(0) = 0$, $\sigma(1) = 1$ and $\sigma(a^{-1}) = \sigma(a)^{-1}$.

\begin{proposition}
    If $[\E : \F]$ is finite then $\Aut(\E / \F)$ acts on $\E$ with finite orbits.
\end{proposition}

\begin{proof}
    Let $\alpha \in \E$, let's show that $\alpha$ has only finitely many translates by the action of $\Aut(\E / \F)$. By the previous Lemma, we know that $\alpha$ is algebraic so there is a polynomial $a_n x^n + ... + a_0 \in \F[x]$ satisfied by $\alpha$.  By plugging-in $x = \alpha$, we have
    $$a_n \alpha^n + ... a_1 \alpha + a_0 = 0.$$
    Let $\sigma \in \Aut(\E / \F)$, then applying $\sigma$ on both sides of the previous equation gives us 
    $$\sigma(a_n \alpha^n + ... a_1 \alpha + a_0) = 0.$$
    Using the fact that $\sigma$ preserves addition and multiplication, we get
    $$\sigma(a_n)\sigma(a_n)^n + ... + \sigma(a_1)\sigma(\alpha) + \sigma(a_0) = 0.$$
    Finally, since $\sigma$ fixes the elements of $\F$, then 
    $$a_n\sigma(a_n)^n + ... + a_1\sigma(\alpha) + a_0 = 0.$$
    It follows that $\sigma(\alpha)$ must be a root of the same polynomial. Hence, the orbit of $\alpha$ is a subset of the roots of the polynomial that it satisfies (that we fixed at the beginning of the proof). Since polynomials over fields have finitely many roots, then $\alpha$ has a finite orbit.
\end{proof}

Notice that the same proof can be applied if $\E / \F$ is a finite extension such that all elements of $\E$ are algebraic over $\F$, i.e., if $\E / \F$ is an algebraic extension.

\begin{theorem}
    If $[\E : \F] < \infty$, then $\#\Aut(\E / \F) < \infty$.
\end{theorem}

\begin{proof}
    Let $\alpha_1, ..., \alpha_n$ be generators for $\E$ over $\F$, then for all $\sigma \in \Aut(\E / \F)$, if we know the behavior of $\sigma$ on the generators, then we know the behavior of $\sigma$ on $\E$. Since there are finitely many generators and each generator has a finite orbit, then there are finitely many possible $\sigma$.
\end{proof}

\begin{example}
    \item Suppose that $\E$ is generated over $\F$ be a single element $\alpha$. Let $p \in \F[x]$ be the minimal polynomial of $\alpha$. Consider the evaluation map 
    \begin{align*}
        ev_{\alpha} &: \F[x] \to \F[\alpha] \\
        & x \mapsto \alpha \\
        & f(x) \mapsto f(\alpha)
    \end{align*}
    We get that $\ker(ev_{\alpha}) = (p)$. Hence, by the isomorphism theorem, $\F[\alpha]/(p) \isomorphic \F[\alpha]$. Since $\F[\alpha]$ is an integral domain, then $\F[\alpha]/(p)$ \td  
\end{example}

\end{document}