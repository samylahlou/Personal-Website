\section{Subgroups Generated by a Subset of a Group}

\begin{exercise}
    Prove that if $H$ is a subgroup of $G$ then $H = \langle H \rangle$. \\
\end{exercise}

\begin{solution}
    \\ First, since $H$ is a subgroup of $G$ that contains $H$, then $\langle H \rangle \subseteq H$. Moreover, since $\langle H \rangle$ is the intersections of all the subgroups of $G$ that contains $H$, then $H \subseteq \langle H \rangle$. Therefore, $H = \langle H \rangle$.\\
\end{solution}

\begin{exercise}
    Prove that if $A$ is a subset of $B$ then $\langle A \rangle \subseteq \langle B \rangle$. Give an example where $A \subseteq B$ with $A \neq B$ but $\langle A \rangle \neq \langle B \rangle$. \\
\end{exercise}

\begin{solution}
    \\ Since $B \subseteq \langle B \rangle$, then $A \subseteq \langle B \rangle$. Hence, $\langle B \rangle$ is a subgroup of $G$ that contains $A$. Therefore, $\langle A \rangle \subseteq \langle B \rangle$. \\
    As an example, take $A = G\setminus \{1\}$ and $B = G$. Obviously, $A$ is a proper subset of $B$. However, the only subgroup of $G$ that contains $A$ is $G$ itself. Therefore,
    $$ \langle A \rangle = G =  \langle B \rangle$$
\end{solution}

\begin{exercise}
    Prove that if $H$ is an abelian subgroup of a group $G$ then $ \langle H, Z(G) \rangle$ is abelian. Give an explicit example of an abelian subgroup $H$ of a group $G$ such that $ \langle H, C_G(H) \rangle$ is not abelian. \\
\end{exercise}

\begin{solution}
    \\ First, to show that $\langle H, Z(G) \rangle$ is abelian, let's use the fact that $ \langle H, Z(G) \rangle = \overline{H\cup Z(G)}$. Let $a,b \in \langle H, Z(G) \rangle$, then $a = a_1^{e_1}\dots a_n^{e_n}$ and $b = b_1^{f_1}\dots b_m^{f_m}$ for some $a_i,b_j \in H\cup Z(G)$, $e_i,f_j \in \mathbb{Z}$ and $n,m \in \mathbb{N}$. But notice here that every element commute with every other element, hence we can rearrange the terms in any way we like without changing the value of the expression:
    \begin{align*}
        ab &= a_1^{e_1}\dots a_n^{e_n}b_1^{f_1}\dots b_m^{f_m} \\
        &= b_1^{f_1}\dots b_m^{f_m}a_1^{e_1}\dots a_n^{e_n} \\
        &= ba
    \end{align*}
    Therefore, $\langle H, Z(G) \rangle$ is abelian. \\
    As an example of a group $G$, and an abelian subgroup $H$ such that $\langle H, C_G(H) \rangle$ is not abelian, let $G = S_5$ and $H = \{1, (1 \ 2)\}$. Obviously, $H$ is abelian and $H \subseteq C_{S_5}(H)$, thus: 
    $$\langle H, C_{S_5}(H) \rangle = \langle H\cup C_{S_5}(H)\rangle = \langle C_{S_5}(H) \rangle = C_{S_5}(H)$$
    Notice that both $(3 \ 4)$ and $(4 \ 5)$ are in $C_{S_5}(H)$ but 
    $$(3 \ 4)(4 \ 5) = (3 \ 4 \ 5)$$
    $$(4 \ 5)(3 \ 4) = (3 \ 5 \ 4)$$
    Therefore, $C_{S_5}(H) = \langle H, C_{S_5}(H) \rangle$ is not abelian.
    \\
\end{solution}

\begin{exercise}
    Prove that if $H$ is a subgroup of $G$ then $H$ is generated by the set $H - \{1\}$. \\
\end{exercise}

\begin{solution}
    \\ Let's show that $H$ is the smallest subgroup of $G$ that contains $H - \{1\}$. Obviously, $H$ is a subgroup of $G$ that contains $H - \{1\}$. To prove that it is the smallest, let $S$ be a subgroup of $G$ that contains $H - \{1\}$, since it is a subgroup, then $1 \in S$. Thus, $H = (H - \{1\})\cup \{1\} \subseteq S$. Hence, $H$ is the smallest subgroup of $G$ that contains $H-\{1\}$. By uniqueness, it implies that $H = \langle H - \{1\} \rangle$. Therefore, $H$ is generated by $H - \{1\}$.\\
\end{solution}

\begin{exercise}
    Prove that the subgroup generated by any two distinct elements of order 2 in $S_3$ is all of $S_3$.\\
\end{exercise}

\begin{solution}
    \\ A set of two elements in $S_3$ of order $2$ must be of the form $\{(a \ b), (b \ c)\}$ where $a,b,c$ are distinct integers between $1$ and $3$ since the elements of order $2$ are exactly
    $$(1 \ 2) \quad (1 \ 3) \quad (2 \ 3)$$
    We know that $\langle (a \ b), (b \ c) \rangle \leq S_3$ so its order must divide 6 by Lagrange's Theorem. We also know that $\langle (a \ b), (b \ c) \rangle$ contains $1$, $(a \ b)$, $(b \ c)$ and $(a \ b \ c)$ since 
    $$(a \ b)(b \ c) = (a \ b \ c)$$
    and $\langle (a \ b), (b \ c) \rangle$ is closed under multiplication. Thus, its order is greater than 4. Since the order of $\langle (a \ b), (b \ c) \rangle$ also divides 6, then it must be 6. Therefore, $\langle (a \ b), (b \ c) \rangle$ is a subgroup of order 6 of $S_3$ so $\langle (a \ b), (b \ c) \rangle$ must be all of $S_3$. \\
\end{solution}

\begin{exercise}
    Prove that the subgroup of $S_4$ generated by $(1 \ 2)$ and $(1 \ 2)(3 \ 4)$ is a noncylic group of order $4$.\\
\end{exercise}

\begin{solution}
    \\ Let $S =  \{(1 \ 2), (1 \ 2)(3 \ 4)\}$ and $H = \{1, (1 \ 2), (3 \ 4), (1 \ 2)(3 \ 4)\}$ and let's show that $H = \langle S \rangle$. Obviously, $H$ is a subgroup of $S_4$ that contains $S$ so $\langle S \rangle \subseteq H$. Moreover, for any subgroup $A$ of $G$ that contains $S$, $A$ must contain the identity, $S$ and the product of $(1 \ 2)$ and $(1 \ 2)(3 \ 4)$ which is simply $(3 \ 4)$. Hence, $H\subseteq A$ which implies that $H \subseteq \langle S \rangle$. Therefore, $\langle S \rangle$ is a subgroup of order which is noncyclic since no element has order 4 (every element has order 1 or 2).\\
\end{solution}

\begin{exercise}
    Prove that the subgroup of $S_4$ generated by $(1 \ 2)$ and $(1 \ 3)(2 \ 4)$ is isomorphic to the dihedral group of order 8.\\
\end{exercise}

\begin{solution}
    \\ First, recall that $D_8$ has the following presentation:
    $$D_8 = \langle a,b \ | \ a^2 = b^2 = (ab)^n = 1 \rangle$$
    Notice that
    \begin{align*}
        &(1 \ 2)^2 = 1 \\
        &((1 \ 3)(2 \ 4))^2 = (1 \ 3)^2(2 \ 4)^2 = 1 \\
        &((1 \ 2)(1 \ 3)(2 \ 4))^4 = (1 \ 3 \ 2 \ 4)^4 = 1
    \end{align*}
    Hence, $(1 \ 2)$ and $(1 \ 3)(2 \ 4)$ satisfy the same relations as $a$ and $b$. Thus, there is a surjective homomorphism from $\langle (1 \ 2), (1 \ 3)(2 \ 4) \rangle$ to $D_8$ that maps $(1 \ 2)$ to $a$ and $(1 \ 3)(2 \ 4)$ to $b$. By surjectivity, we know that the order of $\langle (1 \ 2), (1 \ 3)(2 \ 4) \rangle$ is greater than 8. Moreover, using the homomorphism, we can easily find the subgroup $\{1, (1 \ 3 \ 2 \ 4), (1 \ 2)(3 \ 4), (1 \ 4 \ 2 \ 3), (1 \ 2), (1 \ 3)(2 \ 4), (3 \ 4), (1 \ 4)(2 \ 3)\}$ of $S_4$ of order 8 that contains our two generators. Thus, the order of $\langle (1 \ 2), (1 \ 3)(2 \ 4) \rangle$ is less than 8. Therefore, $\langle (1 \ 2), (1 \ 3)(2 \ 4) \rangle$ has order 8 which implies that the surjective homomorphism between $\langle (1 \ 2), (1 \ 3)(2 \ 4) \rangle$ onto $D_8$ is actually an isomorphism.
    \\
\end{solution}

\begin{exercise}
    Prove that $S_4 = \langle (1 \ 2 \ 3 \ 4), (1 \ 2 \ 4 \ 3) \rangle$. \\
\end{exercise}

\begin{solution}
    \\ I don't know yet if there is an easier method but it is sufficient to prove that at least 13 elements of $S_4$ can be written as a product of $(1 \ 2 \ 3 \ 4), (1 \ 2 \ 4 \ 3)$. To make the notation easier, let $a = (1 \ 2 \ 3 \ 4)$ and $b = (1 \ 2 \ 4 \ 3)$. By Lagrange's Theorem, this would mean that $\langle (1 \ 2 \ 3 \ 4), (1 \ 2 \ 4 \ 3) \rangle$ has order that divides $24$ while having an order biggr than $13$.
    \begin{multicols}{3}
    \begin{itemize}
        \item $1 = a a^{-1}$
        \item $(1 \ 2 \ 3 \ 4) = a$
        \item $(1 \ 3)(2 \ 4) = a^2$
        \item $(1 \ 4 \ 3 \ 2) = a^3$
        \item $(1 \ 2 \ 4 \ 3) = b$
        \item $(1 \ 4)(2 \ 3) = b^2$
        \item $(1 \ 3 \ 4 \ 2) = b^3$
        \item $(1 \ 3 \ 2) = ab$
        \item $(1 \ 2 \ 3) = abab$
        \item $(1 \ 4 \ 2) = ba$
        \item $(1 \ 2 \ 4) = baba$
        \item $(2 \ 4) = ab^2$
        \item $(1 \ 3) = a^3b^2$
    \end{itemize}
    \end{multicols}
    \noindent Therefore, $\langle (1 \ 2 \ 3 \ 4), (1 \ 2 \ 4 \ 3) \rangle = S_4$. \\
\end{solution}

\begin{exercise}
    Prove that $SL_2(\mathbb{F}_3)$ is the subgroup of $GL_2(\mathbb{F}_3)$ generated by $\begin{pmatrix}
        1 & 1\\ 
        0 & 1
      \end{pmatrix}$, and 
    $\begin{pmatrix}
        1 & 0\\ 
        1 & 1
    \end{pmatrix}$. [Recall from Exercise 9 of Section 1 that $SL_2(\mathbb{F}_3)$ is the subgroup of matrices of determinant 1. You may assume this subgroup has order 24 - this will be an exercise in Section 3.2.] \\
\end{exercise}

\begin{solution}
    \\ As for the previous exercise, let's show that at least 13 elements of $SL_2(\mathbb{F}_3)$ can be written as a product of $\begin{pmatrix} 1 & 1\\ 0 & 1 \end{pmatrix}$, and $\begin{pmatrix} 1 & 0\\ 1 & 1 \end{pmatrix}$. To make it easier, let 
    $$A = \begin{pmatrix} 1 & 1\\ 0 & 1 \end{pmatrix} 
    \qquad B = \begin{pmatrix} 1 & 0\\ 1 & 1 \end{pmatrix}$$
    Then we get 
    \begin{multicols}{3}
        \begin{itemize}
            \item $I_3 = A A^{-1}$
            \item $\begin{pmatrix} 1 & 1\\ 0 & 1 \end{pmatrix} = A$
            \item $\begin{pmatrix} 1 & 2\\ 0 & 1 \end{pmatrix} = A^2$
            \item $\begin{pmatrix} 1 & 0\\ 1 & 1 \end{pmatrix} = B$
            \item $\begin{pmatrix} 1 & 0\\ 2 & 1 \end{pmatrix} = B^2$
            \item $\begin{pmatrix} 1 & 1\\ 1 & 2 \end{pmatrix} = BA$
            \item $\begin{pmatrix} 1 & 2\\ 1 & 0 \end{pmatrix} = BA^2$
            \item $\begin{pmatrix} 1 & 1\\ 2 & 0 \end{pmatrix} = B^2 A$
            \item $\begin{pmatrix} 1 & 2\\ 2 & 2 \end{pmatrix} = B^2 A^2$
            \item $\begin{pmatrix} 2 & 1\\ 1 & 1 \end{pmatrix} = AB$
            \item $\begin{pmatrix} 0 & 2\\ 1 & 1 \end{pmatrix} = A^2 B$
            \item $\begin{pmatrix} 0 & 1\\ 2 & 1 \end{pmatrix} = A B^2$
            \item $\begin{pmatrix} 2 & 2\\ 2 & 1 \end{pmatrix} = A^2 B^2$
        \end{itemize}
        \end{multicols}
        \noindent Therefore, $\left<  \begin{pmatrix} 1 & 1\\ 0 & 1 \end{pmatrix}, \begin{pmatrix} 1 & 0\\ 1 & 1 \end{pmatrix} \right> = SL_2(\mathbb{F}_3)$. \\
\end{solution}

\begin{exercise}
    Prove that subgroup of $SL_2(\mathbb{F}_3)$ generated by $\begin{pmatrix} 0 & -1\\ 1 & 0 \end{pmatrix}$ and $\begin{pmatrix} 1 & 1\\ 1 & -1 \end{pmatrix}$ is isomorphic to the quaternion group of order 8. [Use a presentation of $Q_8$.]\\
\end{exercise}

\begin{solution}
    \\ Recall that $Q_8$ has the following presentation:
    $$Q_8 = \left< i,j \ | \ i^4 = 1, \ i^2 = j^2 \ ij = ji^{-1} \right>$$
    To make things easier, define 
    $$A = \begin{pmatrix} 0 & -1\\ 1 & 0 \end{pmatrix} \qquad B = \begin{pmatrix} 1 & 1\\ 1 & -1 \end{pmatrix}$$
    Notice that
    $$A^2 = \begin{pmatrix} -1 & 0\\ 0 & -1 \end{pmatrix}$$
    and 
    $$B^2 = \begin{pmatrix} 2 & 0\\ 0 & 2 \end{pmatrix} = \begin{pmatrix} -1 & 0\\ 0 & -1 \end{pmatrix}$$
    Thus, $A^2 = B^2$, and obviously: $A^4 = I_3$. Moreover, 
    $$ABA = \begin{pmatrix} 0 & -1\\ 1 & 0 \end{pmatrix} \begin{pmatrix} 1 & 1\\ 1 & -1 \end{pmatrix} \begin{pmatrix} 0 & -1\\ 1 & 0 \end{pmatrix} = \begin{pmatrix} 1 & 1\\ 1 & -1 \end{pmatrix} = B$$
    which implies 
    $$AB = BA^{-1}$$
    Therefore, $A$ and $B$ satisfy the same relations as $i$ and $j$ in the group presentation of $Q_8$ so the function mapping $A$ to $i$ and $B$ to $j$ has a unique extension into a surjective homomorphism. Moreover, we can easily from this correspondence find a subgroup of $SL_2(\mathbb{F}_3)$ or order 8 that contains $\langle A,B\rangle$ so $\langle A,B\rangle$ has order 8 which means that the surjective homomorphism must be an isomorphism. Therefore, $\langle A,B\rangle$ is isomorphic to $Q_8$. \\
\end{solution}

\begin{exercise}
    Show that $SL_2(\mathbb{F}_3)$ and $S_4$ are two nonisomorphic groups of order 24. \\
\end{exercise}

\begin{solution}
    \\ By contradiction, suppose that they are indeed isomorphic, then by the previous exercise, it must be that $S_4$ has a subgroup isomorphic to $Q_8$. Thus, $S_4$ has elements $\sigma$ and $\tau$ of order 2 that satisfiy
    $$\sigma^2 = \tau^2 \qquad \sigma\tau\sigma = \tau$$
    The only elements of order $4$ in $S_4$ are 4-cycles. Hence, the first condition on $\sigma$ and $\tau$ implies that 
    $$\sigma = (1 \ a \ b \ c) \qquad \tau = (1 \ c \ b \ a)$$
    for some distinct $a,b,c \in \{2,3,4\}$. The second condition tells us that 
    \begin{align*}
        \tau &= \sigma\tau\sigma \\
        &= (1 \ a \ b \ c) (1 \ c \ b \ a) (1 \ a \ b \ c) \\
        &= (1 \ a \ b \ c) \\
        &= \sigma
    \end{align*}
    A contradiction since that $\sigma$ and $\tau$ are distinct. Therefore, $SL_2(\mathbb{F}_3)$ and $S_4$ are two nonisomorphic groups of order 24. \\
\end{solution}

\begin{exercise}
    Prove that the subgroup of upper triangular matrices in $GL_3(\mathbb{F}_2)$ is isomorphic to the dihedral group of order 8 (cf. Exercise 16, Section 1) [First find the order of this subgroup.] \\
\end{exercise}

\begin{solution}
    \\ First, to determine the number of upper triangular matrices in $GL_3(\mathbb{F}_2)$, recall that each such matrice has the following form:
    $$\begin{pmatrix} * & * & * \\ 0 & * & * \\ 0 & 0 & * \end{pmatrix}$$
    However, if the element of coordinate 1,1 is 0, then the matrix cannot be invertible since one of its column vectors are not linearly independant which is impossible since we are considering elements of $GL_3(\mathbb{F}_2)$. Thus, the first entry of the first line is 1. Similarly, the last entry of the last line is 1 as well. Thus, every such matrix has the following form:
    $$\begin{pmatrix} 1 & * & * \\ 0 & * & * \\ 0 & 0 & 1 \end{pmatrix}$$
    If the entry 2,2 in the middle of the matrix is 0, then the entry above it cannot be 0 since it would give a zero vector as a column vector of the matrix. But it cannot be one since the first two column vectors of the matrix would be equal (which would make the matrix non invertible). Therefore, the entry in the middle must be 1:
    $$\begin{pmatrix} 1 & * & * \\ 0 & 1 & * \\ 0 & 0 & 1 \end{pmatrix}$$
    Now, for such matrices, it is easy to see that whatever way we choose to fill in the *, it will be an upper triangular matrix in $GL_3(\mathbb{F}_2)$. Therefore, there are 8 upper triangular matrices in $GL_3(\mathbb{F}_2)$. To make the notation easier, let's denote each matrix as a vector in $\mathbb{F}_2^3$ in the following way:
    $$(a, b, c) \sim \begin{pmatrix} 1 & a & b \\ 0 & 1 & c \\ 0 & 0 & 1 \end{pmatrix}$$
    Hence, the multiplication is defined as follows:
    $$(a,b,c) \cdot (e,f,g) = (a+e, f+ag+b, c+g)$$
    To show that our group is isomorphic to $D_8$, let's find two matrices that act as $r$ and $s$ in our group. Let $S = (1,0,0)$ and $R = (1,1,1)$ be two matrices. Notice that 
    $$S^2 = (1,0,0)\cdot(1,0,0) = (2,0,0) = (0,0,0) = I_3$$
    and 
    \begin{align*}
        R^2 &= (1,1,1)\cdot (1,1,1) = (2,3,2) = (0,1,0) \\
        R^3 &= (1,1,1)\cdot (0,1,0) = (1,2,1) = (1,0,1) \\
        R^4 &= (1,1,1)\cdot (1,0,1) = (2,2,2) = (0,0,0) = I_3
    \end{align*}
    and finaly:
    \begin{align*}
        RSR &= (1,1,1)\cdot (1,0,0) \cdot (1,1,1) \\
        &= (1,1,1)\cdot (0,0,1) \\
        &= (1,0,0) \\
        &= S
    \end{align*}
    which implies that $SR = R^{-1}S$. Therefore, the function that maps $R$ to $r \in D_8$ and $S$ to $s \in D_8$ can be extended to a homomorphism. Notice that we already showed that $R$ and $S$ generate 5 of the 8 matrices we are considering in this exercise. By Lagrange's Theorem, this implies that $R$ and $S$ are generators of the group of upper traingular matrices in $GL_3(\mathbb{F}_2)$. Hence, the homomorphism is surjective. Since both groups have order 8, then it must be an isomorphism. Therefore, the group of upper traingular matrices in $GL_3(\mathbb{F}_2)$ is isomorphic to $D_8$. \\
\end{solution}

\begin{exercise}
    Prove that the multiplicative group of positive rational numbers is generated by the set $\{\frac{1}{p} \ | \ p \text{ is a prime}\}$. \\
\end{exercise}

\begin{solution}
    \\ To make the notation easier, let $P = \{\frac{1}{p} \ | \ p \text{ is a prime}\}$. By proposition 9, we know that $\langle P  \rangle = \overline{P}$ so it is sufficient to prove that $\mathbb{Q}_{>0} = \overline{P}$. Obviously, $\overline{P} \subseteq \mathbb{Q}_{>0}$. Let $q \in \mathbb{Q}_{>0}$, then $q = a/b$ where $a$ and $b$ are positive integers. There are four possible cases, if both $a$ and $b$ are equal to 1, then $q \in \overline{P}$ follows from the fact that $\overline{P}$ is a subgroup. If $a=1$ and $b$ is an integer bigger than 2, then $b$ can be written as a product of prime numbers:
    $$b = p_1^{a_1} \dots p_n^{a_n}$$
    Thus,
    $$q = \frac{1}{b} = \frac{1}{p_1^{a_1} \dots p_n^{a_n}} = \left(\frac{1}{p_1}\right)^{a_1} \dots \left(\frac{1}{p_n}\right)^{a_n} \in \overline{P}$$
    If $b=1$ and $a$ is an integer bigger than 2, then we just proved that $1/q = 1/a$ must be in $\overline{P}$. Since $\overline{P}$ is a subgroup, then it follows that $q \in \overline{P}$. \\
    If both $a$ and $b$ are integers bigger than 2, then we just showed that both $a$ and $1/b$ are in $\overline{P}$. Again, since $\overline{P}$ is a subgroup, then it follows that $q = a\frac{1}{b} \in \overline{P}$.\\
    In every possible case, we showed that $q \in \overline{P}$ so it follows that $\mathbb{Q}_{>0} \subseteq \overline{P}$. Therefore, $\mathbb{Q}_{>0} = \overline{P} = \langle \{\frac{1}{p} \ | \ p \text{ is a prime}\}  \rangle$. \\
\end{solution}

\begin{exercise}
    A group $H$ is called \textit{finitely generated} if there is a finite set $A$ such that $H = \langle A \rangle$.
    \begin{enumerate}[label = \textbf{(\alph*)}]
        \item Prove that every finite group is finitely generated.
        \item Prove that $\mathbb{Z}$ is finitely generated.
        \item Prove that every finitely generated subgroup of the additive group $\mathbb{Q}$ is cyclic. [If $H$ is a finitely generated subgroup of $\mathbb{Q}$, show that $H \leq \langle \frac{1}{k} \rangle$, where $k$ is the product of all denominators which appear in a set of generators of $H$.]
        \item Prove that $\mathbb{Q}$ is not finitely generated. 
    \end{enumerate}
\end{exercise}

\begin{solution}
    \begin{enumerate}[label = \textbf{(\alph*)}]
        \item If $G$ is a group, then obviously, $G = \langle G \rangle$. So if $G$ is finite, it can be written as $\langle H \rangle$ where $H$ is a finite set ($H = G$).
        \item Since $\mathbb{Z} = \langle 1 \rangle$, then it directly follows that it is finitely generated.
        \item Let $H$ be a finitely generated subgroup of $\mathbb{Q}$, then there exist rationals $a_i/b_i$ such that
        $$H = \left< \frac{a_1}{b_1}, \dots , \frac{a_n}{b_n} \right>$$
        Define $k = \prod_{i=1}^{n}b_i$ and let's show that the generators of $H$ are all contained in $\langle \frac{1}{k} \rangle$. Let $j \in \Iint{1}{n}$ and notice that
        $$\frac{a_j}{b_j} = \frac{a_j \prod_{\substack{i=1 \\ i \neq j}}^{n}b_i}{b_j\prod_{\substack{i=1 \\ i \neq j}}^{n}b_i} = \frac{a_j \prod_{\substack{i=1 \\ i \neq j}}^{n}b_i}{k} = a_j \prod_{\substack{i=1 \\ i \neq j}}^{n}b_i \cdot \frac{1}{k} \in \left< \frac{1}{k} \right>$$
        Thus, we showed that $\{ \frac{a_1}{b_1}, \dots , \frac{a_n}{b_n} \} \subseteq \left< \frac{1}{k} \right>$. Hence, by Exercise 2:
        $$H = \left< \frac{a_1}{b_1}, \dots , \frac{a_n}{b_n} \right> \leq \left<\left< \frac{1}{k} \right>\right> = \left< \frac{1}{k} \right>$$
        Therefore, $H$ must be cyclic since it is a subgroup of the cyclic group $\left< \frac{1}{k} \right>$.

        \item By contradiction, if $\mathbb{Q}$ is finitely generated, then by our previous proof, it would mean that $\mathbb{Q}$ is cyclic, i.e., there is a rational $a/b$ such that $\mathbb{Q} = \langle \frac{a}{b} \rangle$. But this is impossible since it implies that any element of $\mathbb{Q}$ can be written as a multiple of $a/b$ and some rationals ($\frac{1}{b+1}$ for example) cannot be written as $an/b$. Therefore, $\mathbb{Q}$ is not finitely generated.\\
    \end{enumerate}
\end{solution}

\begin{exercise}
    Exhibit a proper subgroup of $\mathbb{Q}$ which is not cyclic. \\
\end{exercise}

\begin{solution}
    \\ Consider the set $H = \{\frac{1}{p} \ | \ p \text{ prime and } p\leq 3\}$ and the subgroup that it generates. Let's first prove that $\langle H \rangle$ is not cyclic. By contradiction, suppose that $\langle H \rangle$ is cyclic, then there are positive integers $a$ and $b$ such that $\langle H \rangle = \langle \frac{a}{b} \rangle$. Since there are infinitely many odd primes, then there is an odd prime $p$ such that $b \leq p$. By definition of $H$ and by our assumption, we have 
    \begin{align*}
        \frac{1}{p} \in H &\implies \frac{1}{p} \in \langle H \rangle \\
        &\implies \frac{1}{p} \in \left< \frac{a}{b} \right> \\
        &\implies \frac{1}{p} = n\frac{a}{b}, \qquad \text{ for some } n\in \mathbb{Z} \\
        &\implies b = nap \\
        &\implies p \dvd b
    \end{align*}
    which is a contradiction since both integers are positive and $b \leq p$. Therefore, $\langle H \rangle$ cannot be cyclic. \\
    Let's now prove that $\langle H \rangle$ is a proper subgroup of $\mathbb{Q}$. Suppose by contradiction that $1/2 \in \langle H \rangle$, then there exist some integers $a_i$'s and odd primes $p_i$'s such that 
    $$\frac{1}{2} = \sum_{i=1}^{n}\frac{a_i}{p_i}$$
    By rearranging the terms, we get 
    $$2 \cdot c = \prod_{i=1}^{n}p_i$$ 
    where  $c := \sum_{i=1}^{n}(\frac{a_i}{p_i}\prod_{j=1}^{n}p_j) \in \mathbb{Z}$. Thus, 2 divides the product of the $p_i$'s which implies that there is a $j \in \Iint{1}{n}$ such that $2 \dvd p_j$. A contradiction with the fact that the $p_i$'s are odd primes. Therefore, $1/2 \notin \langle H \rangle$ which proves that $\langle H \rangle$ is noncyclic proper subgroup of $\mathbb{Q}$.\\
\end{solution}

\begin{exercise}
    A subgroup $M$ of a group $G$ is called a \textit{maximal subgroup} if $M \neq G$ and the only subgroups of $G$ which contain $M$ are $M$ and $G$.
    \begin{enumerate}[label = \textbf{(\alph*)}]
        \item Prove that if $H$ is a proper subgroup of the finite group $G$ then there is a maximal subgroup of $G$ containing $H$.
        \item Show that the subgroup of all rotations in a dihedral group is a maximal subgroup.
        \item Show that if $G = \langle x \rangle$ is a cyclic group of order $n \geq 1$ then a subgroup $H$ is maximal if and only if $H = \langle x^p \rangle$ for some prime $p$ dividing $n$.
    \end{enumerate}
\end{exercise}

\begin{solution}
    \begin{enumerate}[label = \textbf{(\alph*)}]
        \item Let $H$ be a proper subgroup of the finite group $G$ of order $n \in \mathbb{Z}^+$. Suppose by contradiction that there is no maximal subgroup of $G$ containing $H$, then we can define the increasing sequence $\{H_k\}_k$ of subgroups of $G$ recursively where each $H_k$ is a proper subset of $H_{k+1}$. \\
        Define $H_1$ as $H$ and for $k \in \mathbb{Z}^+$, since $H_k$ is a subgroup of $G$ that contains $H$, then it cannot be contained in a maximal subgroup of $G$, otherwise, $H$ would be contained in a maximal subgroup of $G$. Hence, there must be a subgroup $H_{k+1}$ of $G$ that contains $H_k$ which is different from $H_k$ and $G$. Thus, $H_k < H_{k+1}$ for all $k \in \mathbb{Z}^+$. \\
        Notice that the sequence of sets is strictly increasing so the sequence of their respective order $\{|H_k|\}_k$ must blow up to infinity. Therefore, there exist a subgroup $H_k$ of $G$ of order strictly greater than $G$. A contradiction which implies that $H$ must be contained in a maximal subgroup of $G$.
        \item Let $M = \langle r \rangle$ and let's show that it is maximal. First, obviously $M \neq G$ since $s \notin M$. Now, let $H$ be a subgroup of $G$ that contains $M$, then there are two cases. If $H$ contains no elements of the form $sr^i$ where $i\in Z$, then it follows that $H = M$. If there is an integer $i$ such that $sr^i \in H \geq M$, then it must also contain $r^{n-i}$ which implies
        $$s = (sr^i)(r^{n-i}) \in H$$
        Thus, $\{r,s\} \subseteq H$ which implies (by Exercise 1 and 2):
        $$D_{2n} = \langle r,s \rangle \subseteq \langle H \rangle = H$$
        Hence, $H = D_{2n}$. Since any subgroup of $G$ that contains $M$ is either $M$ or $D_{2n}$, then $M$ is maximal by definition.
        \item Let $G=\langle x \rangle$ be a cyclic group of order $n \geq 1$ and $H$ a subgroup of $G$. Suppose first that $H$ is maximal, since $G$ is cyclic, then it follows that $H = \langle x^m \rangle$ for $m$ divides $n$. By contradiction, if $m$ is composite, then there exist integers $1 < a,b < m$ such that $m = ab$. Consider the subgroup $\langle x^a \rangle$ and notice that $\langle x^m \rangle \leq \langle x^a \rangle$ which implies by our assumption on the maximality of $H$ that $\langle x^a \rangle = H$ or $\langle x^a \rangle = G$. If $\langle x^a \rangle = H = \langle x^m \rangle$, then it means that $x^a \in \langle x^m \rangle$ so $x^a = x^{km}$ for some integer. Thus,
        \begin{align*}
            x^a = x^{km} &\implies x^{km-a} = 1 \\
            &\implies n \dvd km - a \\
            &\implies m \dvd km - a \\
            &\implies m \dvd a
        \end{align*}
        a contradiction so $\langle x^a \rangle \neq H$. Hence, we must have $\langle x^a \rangle = G = \langle x \rangle$. But this means that $x \in \langle x^a \rangle$ and for some integer $k$:
        \begin{align*}
            x = x^{ak} &\implies x^{ak - 1} = 1 \\
            &\implies n \dvd ak - 1 \\
            &\implies a \dvd ak - 1 \\
            &\implies a \dvd 1
        \end{align*}
        again, a contradiction since $a$ is a positive integer strictly greater than $1$. Hence, $\langle x^a \rangle$ is neither equal to $H$ or $G$. Thus, it contradicts the fact that $H$ is maximal so our assumption that $m$ is composite must be false. Therefore, $H = \langle x^m \rangle$ where $m$ is a prime dividing $n$. \\
        Suppose now that $H = \langle x^p \rangle$ where $p$ is a prime dividing $n$ and let's show that $H$ is maximal. Let $\langle x^d \rangle$ be a subgroup of $G$ that contains $H$, then $d$ divides $n$ which means that $d = rn$ for some integer $r$. Moreoever,  
        \begin{align*}
            \langle x^p \rangle \leq \langle x^d \rangle &\implies x^p \in  \langle x^d \rangle \\
            &\implies x^p = x^{kd}, \qquad k\in\mathbb{Z} \\
            &\implies x^{kd - p} = 1 \\
            &\implies n \dvd kd - p \\
            &\implies kd - p = tn, \qquad t\in\mathbb{Z} \\
            &\implies kd - p = trd \\
            &\implies p = (k - tr)d \\
            &\implies d \dvd p \\
            &\implies d = 1 \ \text{ or } \ d = p
        \end{align*}
        If $d = 1$, then $\langle x^d \rangle = G$, if $d = p$, $\langle x^d \rangle = H$. Therefore, by definition, $H$ is maximal.\\
    \end{enumerate}
\end{solution}

\begin{exercise}
    This is an exercise involving Zorn's Lemma (see Appendix \MakeUppercase{\romannumeral 1}) to prove that every nontrivial finitely generated group posseses maximal subgroups. Let $G$ be a finitely generated group, say $G = \langle g_1,g_2,\dots ,g_n \rangle$, and let $\mathcal{S}$ be the set of all proper subgroups of $G$. Then $\mathcal{S}$ is partially ordered by inclusion. Let $\mathcal{C}$ be a chain in $\mathcal{S}$.
    \begin{enumerate}[label = \textbf{(\alph*)}]
        \item Prove that the union, $H$, of all the subgroups in $\mathcal{C}$ is a subgroup of $G$.
        \item Prove that $H$ is a \textit{proper} subgroup. [if not, each $g_i$ must line in $H$ and so must lie in some element of the chain $\mathcal{C}$. Use the definition of a chain to arrive at a contradiction.]
        \item Use Zorn's Lemma to show that $\mathcal{S}$ has a maximal element (which is, by definition, a maximal subgroup).
    \end{enumerate}
\end{exercise}

\begin{solution}
    \begin{enumerate}[label = \textbf{(\alph*)}]
        \item Obviously, $H = $ is a subset of $G$. To show that it is actually a subgroup, let $x,y \in H$ and let's show that $xy^{-1} \in H$. By definition of $H$, there exist sets $C_1$ and $C_2$ in $\mathcal{C}$ such that $x \in C_1$ and $y \in C_2$. Since $\mathcal{C}$ is a chain, we must have $C_1 \subseteq C_2$ or $C_2 \subseteq C_1$, in both cases, $C_1 \cup C_2$ is a subgroup of $G$ contained in $\mathcal{C}$. Hence, $x,y \in C_1 \cup C_2$ which implies that $xy^{-1} \in C_1 \cup C_2 \subseteq H$. To finish the argument, notice that $H$ is nonempty since it is a union of nonempty sets. Therefore, $H$ is a subgroup of $G$.
        \item Suppose by contradiction that $H = G$, then there exist sets $C_1, \dots, C_n \in \mathcal{C}$ such that $g_i \in C_i$ for all $i \in \Iint{1}{n}$. Since the inclusion is a total ordering on $\mathcal{C}$, then $\cup_{i=1}^nC_i$ is equal to $C_j$ for some $j \in \Iint{1}{n}$. Thus, $\cup_{i=1}^nC_i$ is an element of $\mathcal{S}$ so it is a proper subgroup of $G$ that contains all of the generators of $G$. However, notice that this implies (by Exercise 1 and 2):
        \begin{align*}
            \{g_1,\dots,g_n\} \subseteq \bigcup_{i=1}^nC_i &\implies \langle g_1,\dots,g_n \rangle \subseteq \left< \bigcup_{i=1}^nC_i \right> \\
            &\implies G \subseteq \bigcup_{i=1}^nC_i \\
            &\implies \bigcup_{i=1}^nC_i = G
        \end{align*}
        which is a contradiction. Therefore, $H$ is a proper subgroup of $G$.
        \item First, notice that $H$ is an upperbound for $\mathcal{C}$ since each $C \in \mathcal{C}$ satisfies $C \subseteq H$. Therefore, we showed in the previous parts of this exercise that each chain in $\mathcal{S}$ has an upperbound. Thus, by Zorn's Lemma, it directly follows that there is an element $M \in \mathcal{S}$ such that if $M \subseteq M'$ where $M'$ is a proper subgroup of $G$, then $M' = G$. Equivalently, if $M'$ is any subgroup of $G$ that satisfies $M \subseteq M'$, then we either have $M' = G$ or $M' = M$. Therefore, by definition, $G$ has a maximal element. \\
    \end{enumerate}
\end{solution}

\begin{exercise}
    Let $p$ be a prime and let $Z = \{z\in\mathbb{C} | z^{p^n}=1 \text{ for some } n\in \mathbb{Z}^+\}$ (so $Z$ is the multiplicative group of all $p$-power roots of unity in $\mathbb{C}$). For each $k \in \mathbb{Z}^+$ let $H_k = z\in\mathbb{C} | z^{p^k}=1$ (the group of $p^k$-th roots of unity). Prove the following:
    \begin{enumerate}[label = \textbf{(\alph*)}]
        \item $H_k \leq H_m$ if and only if $k \leq m$
        \item $H_k$ is cyclic for all $k$ (assume that for any $n \in \mathbb{Z}^+$, $\{e^{2\pi it/n} | t = 0,1,\dots , n-1\}$ is the set of all $n$\textsuperscript{th} roots of 1 in $\mathbb{C}$)
        \item every proper subgroup of $Z$ equals $H_k$ for some $k \in \mathbb{Z}^+$ (in particular, every proper subgroup of $Z$ is finite and cyclic)
        \item $Z$ is not finitely generated. \\
    \end{enumerate}
\end{exercise}

\begin{solution}
    \begin{enumerate}[label = \textbf{(\alph*)}]
        \item Let $k$ and $m$ be positive integers. Suppose that $H_k \leq H_m$ and let $z$ be a complex number of order $p^k$ ($z = e^{\frac2\pi i/p^k}$ for example), then we obviously have $z \in H_k$. By our assumption, we also have $z \in H_m$ which implies that $z^{p^m} = 1$:
        \begin{align*}
            z^{p^m} = 1 &\implies |z| \dvd p^m \\
            &\implies p^k \dvd p^m \\
            &\implies k \leq m
        \end{align*}
        To prove the reverse inclusion, suppose now that $k \leq m$ and let $z$ be an arbitrary element of $H_k$, then notice that
        \begin{align*}
            z \in H_k &\implies z^{p^k} = 1 \\
            &\implies (z^{p^k})^{p^{m-k}} = 1^{p^{m-k}} \\
            &\implies z^{p^kp^{m-k}} = 1 \\
            &\implies z^{p^{m}} = 1 \\
            &\implies z \in H_m
        \end{align*}
        Therefore, $H_k \leq H_m$.

        \item Let $k \in \mathbb{Z}^+$, since $H_k = \{e^{2\pi it/p^k} | t = 0,1,\dots , p^k-1\}$, then it obviously implies by the $2\pi$-periodicity of $e^{i\theta}$ that
        \begin{align*}
            H_k &= \{e^{2\pi it/p^k} \ | \ t = 0,1,\dots , p^k-1\} \\
            &= \{e^{2\pi it/p^k} \ | \ t \in \mathbb{Z}\} \\
            &= \{(e^{2\pi i/p^k})^t \ | \ t \in \mathbb{Z}\} \\
            &= \langle e^{2\pi i/p^k} \rangle
        \end{align*}
        Therefore, $H_k$ is cyclic.

        \item Let $H$ be a proper subgroup of $Z$. For each $h \in H$, consider the set $S_h$ defined by
        $$S_h = \{n \in \mathbb{Z}^+ \dvd h \in H_n\}$$
        Since, $Z = \cup_{n=1}^{\infty}H_n$, then $S_h \neq \varnothing$. Thus, by the well ordering of $\mathbb{Z}^+$, $S_h$ has a minimum $m_h$. Hence, there is a $t \in \Iint{0}{p^{m_h} - 1}$ such that $h = e^{2\pi i t / p^{m_h}}$. Moreover, since $m_h$ is the minimum of $S_h$, we must have $(t,p) = 1 = (t,p^{m_h})$. Therefore, since $e^{2\pi i / p^{m_h}}$ has order $p^{m_h}$, then by the theorems from the previous section:
        \begin{align*}
            \langle h \rangle &= \left< (e^{2\pi i / p^{m_h}})^t \right> \\
            &= \langle e^{2\pi i / p^{m_h}} \rangle \\
            &= H_{m_h}
        \end{align*}
        From this, it is easy to see that
        $$H = \bigcup_{h \in H}\langle h \rangle = \bigcup_{h \in H} H_{m_h}$$
        Consider now the set $M = \{m_h \dvd h\in H\}$, if $M$ is unbounded above, then $H_n \leq H$ for all $n \in \mathbb{Z}^+$ which implies that $H = Z$. A contradiction since $H$ is a proper subgroup of $Z$. Hence, since $M$ is a set of integers, then it has a maximum $k$. Thus, there is a $h \in H$ such that
        $\langle h \rangle = H_k$ which implies $H_k \leq H$. Moreover, for any $h \in H$, 
        \begin{align*}
            H_{m_h} \leq H_k &\implies \bigcup_{n=1}^{\infty}H_{m_h} \leq H_k \\
            &\implies H \leq H_k
        \end{align*}
        Therefore, $H = H_k$ for some $k \in \mathbb{Z}^+$.

        \item Let $H$ be an arbitrary proper subgroup of $Z$, then by part (c), we know that $H = H_k$ for some $k$. Hence, there is a proper subgroup of $Z$, for example $H_{k+1}$, that contains $H_k$ and that is distinct from $H$ and $Z$. Thus, $H$ cannot be a maximal subgroup of $Z$. Therefore, $Z$ cannot be finitely generated since it has no maximal subgroup (by Exercise 17). \\
    \end{enumerate}
\end{solution}

\begin{exercise}
    A nontrivial abelian group $A$ (written multiplicatively) is called \textit{divisible} if for each element $a \in A$ and each nonzero integer $k$ there is a an element $x \in A$ such that $x^k = a$, i.e., each element has a $k$\textsuperscript{th} root in $A$ (in additive notation, each element is the $k$\textsuperscript{th} multiple of some element in $A$).
    \begin{enumerate}[label = \textbf{(\alph*)}]
        \item Prove that the additive group of rational numbers, $\mathbb{Q}$, is divisible.
        \item Prove that no finite abelian group is divisible. \\ 
    \end{enumerate}
\end{exercise}

\begin{solution}
    \begin{enumerate}[label = \textbf{(\alph*)}]
        \item To prove that the additive group $\mathbb{Q}$ is divisible, we need to prove that any rational is the $k$\textsuperscript{th} multiple of some rational for any integer $k$. To do so, let $a/b$ be an arbitrary rational and $k$ an arbitrary nonzero integer, define $x$ to be equal to $a/kb$. Then, obviously, $kx = a/b$. Therefore, $\mathbb{Q}$ is divisible.
        \item Let $A$ be a finite nontrivial abelian group of order $n \in \mathbb{Z}^+$, let $a \in A$ be an element different than 1. Notice that for all $x \in A$,
        $$x^n = 1 \neq a$$
        so $a$ has no $n$\textsuperscript{th} root. Therefore, $A$ is not divisible. \\
    \end{enumerate}
\end{solution}

\begin{exercise}
    Prove that if $A$ and $B$ are nontrivial abelian groups, then $A \times B$ is divisible if and only if both $A$ and $B$ are divisible groups. \\
\end{exercise}

\begin{solution}
    \\ Suppose first that $A \times B$ is divisible, to show that both $A$ and $B$ are divisible, let $a \in A$, $b \in B$ and $k \in \mathbb{Z} - \{0\}$, by divisibility of $A \times B$, there is a tuple $(x,y) \in A \times B$ such that $(x,y)^k = (a,b)$. Thus, there is a $x \in A$ such that $x^k = a$ and a $y \in B$ such that $y^k = b$. Therefore, both $A$ and $B$ are divisible. \\
    Suppose now that both $A$ and $B$ are divisible. Let $(a,b) \in A\times B$ and $k \in \mathbb{Z} - \{0\}$, then by divisibility of $A$ and $B$, there is a $x \in A$ such that $x^k = a$ and a $y \in B$ such that $y^k = b$. Thus, $(x,y)^k = (a,b)$. Therefore, $A \times B$ is divisible.
\end{solution}