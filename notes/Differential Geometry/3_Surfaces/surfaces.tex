\section{Surfaces}

\begin{definition}
    A subset of $\R^n$ is a regular $C^k$ surface ($n \geq 2$, $1 \leq k \leq \infty$) if for all points $p \in S$, there are open subsets $U \subseteq \R^2$, $p \in V \subseteq \R^n$, and a $C^k$ homemorphism $\varphi : U \to V \cap S \subseteq \R^n$ such that $D_q\varphi : \R^2 \to \R^n$ is one-to-one (has rank 2) for all $q \in U$.
\end{definition}

\begin{proposition}[Graphs are surfaces]
    If $U \subseteq \R^2$ is open and $f : U \to \R$ is a $C^k$ function, then the graph of $f$ in $\R^3$ is a regular $C^k$ surface.
\end{proposition}

\begin{proof}
    Use $V = \R^3$, $S = \text{graph}(f) \subseteq \R^3$ and $\varphi : U \to \R^3 : (x,y) \mapsto (x,y,f(x,y))$. Note that $\varphi$ is a $C^k$ bijection from $U$ to $S$ with a continuous inverse $\pi: (x,y,z) \mapsto (x,y)$. The jacobian of $\phi$ is
    $$J_{\varphi} = \begin{pmatrix}
        1 & 0 \\
        0 & 1 \\
        \frac{\partial f}{\partial x} & \frac{\partial f}{\partial y} 
    \end{pmatrix}$$
    which has rank 2 for any input it is given (the two columns are always linearly independant, for any the value for the partial derivatives).
\end{proof}

For example, the set defined by $z = x^2 + y^2$ is a regular $C^{\infty}$ surface.

\begin{proposition}[Level sets are surfaces]
    Suppose $\tilde{V} \subseteq \R^3$ is open, $F : \tilde{V} \to \R$ is a $C^k$ function, and $a \in \R$ is a regular value, then $F^{-1}(a)$ is a regular $C^k$ surface.
\end{proposition}

\begin{proof}
    Given $p \in F^{-1}(a)$, we have $\nabla F(p) \neq 0$. Without loss of generality, we can assume that $\frac{\partial F}{\partial z}(p) \neq 0$. Define $G : \tilde{V} \to \R^3$ by $G(x,y,z) = (x,y,F(x,y,z))$ and compute its Jacobian:
    $$J_G = \begin{pmatrix}
        1 & 0 & 0 \\
        0 & 1 & 0 \\
        \frac{\partial F}{\partial x}(p) & \frac{\partial F}{\partial y}(p) & \frac{\partial F}{\partial z}(p)
    \end{pmatrix}$$
    Since $\det J_g(p) = \frac{\partial F}{\partial z}(p) \neq 0$, then we can apply the Inverse Function Theorem. From this, we get that there is a $C^k$ homeomorphism $H : \hat{U} \to V \subseteq \tilde{V}$ which is inverse to $G$. Let $U$ be the projection on $\R^2$ of the set $\hat{S} = \{z = a\}$ and define the homemorphism $\psi : U \to \hat{S} \cap \hat{U} : (x,y) \mapsto (x,y,a)$. To conclude, define the $C^k$ homeomorphism $\varphi = H \circ \psi : U \to F^{-1}(a) \cap V$. Its Jacobian is $J_{\varphi}(q) = J_H(\psi(q))J_{\psi}(q)$
    $$J_{\varphi}(q) = J_H(\psi(q))J_{\psi}(q) =J_G(H(\psi(q)))^{-1}\begin{pmatrix}
        1 & 0 \\
        0 & 1 \\
        0 & 0 \\
    \end{pmatrix}.$$
    Since the first matrix is invertoble and the second matrix as rank 2, then their multiplication has rank 2.
\end{proof}

Notice that the theorem for level sets is a generalization of the theorem for graphs. For example, $x^2 + y^2 + z^2 = 1$ is a regular $C^{\infty}$ surface. Some other examples: hyperboloid with 2-sheets given by the equation $x^2 + y^2 - z^2 = -1$, 1-sheet $x^2 + y^2 - z^2 = 1$. These examples show that surfaces can be disconnected.

For the moment, everything done on surfaces in $\R^3$ applies to curves in $\R^2$ (curves with no boundary). In general ($1 \leq m \leq n$, $1 \leq k \leq \infty$), we define the general notion of manifold as follows:
\begin{definition}[Manifold]
    A subset $M$ of $\R^n$ is a regular $C^k$ $m$-manifold if for all points $p \in M$, there are open subsets $U \subseteq \R^m$, $p \in V \subseteq \R^n$, and a $C^k$ homemorphism $\varphi : U \to V \cap S \subseteq \R^n$ such that $D_q\varphi : \R^m \to \R^n$ is has rank $m$ for all $q \in U$.
\end{definition}
A curve is a 1-manifold, a surface is a 2-manifold.

\begin{theorem}[Surfaces = local graphs]
    The subset $S \subseteq \R^3$ is a regular $C^k$ surface if and only if for all points $p \in S$, there are open subsets $U \subseteq \R^2$, $V \subseteq \R^3$ with $p \in V$ such that $V \cap S$ is the graph of a $C^k$ function $U \to \R$, i.e., it can be described by an equation of the form $z = f(x,y)$, $y = g(x,z)$, or $x = h(y,z)$.
\end{theorem}

\begin{proof}
    The direct implication will be proved in the notes, the reverse implication follows from the fact that every graph is a surface.
\end{proof}

For example, the unit sphere is not the graph of a function, but it does locally. Exercise: Show that $z^2 = x^2 + y^2$ and $z = \sqrt{x^2 + y^2}$ are NOT surfaces. 

\begin{proposition}
    Suppose $S \subseteq \R^3$ is a regular $C^k$ surface, and $U_1, U_2 \subseteq \R^2$ are open subsets with parametrization $\varphi_i : U_i \to S$ which are $C^k$ and satisfy the regularity conditions, then $g = \varphi_2^{-1} \circ \varphi_1$ and $h = \varphi_1^{-1} \circ \varphi_2$ are $C^k$ homeomorphisms. Hence, we can say $\varphi_2 = \varphi_1 \circ h$ is a reparametrization of $\varphi_1$ using the change of parameters $h : U_2 \to U_1$.
\end{proposition}

\begin{definition}
    We call $\varphi : U \subseteq \R^2 \to \R^3$ a regular $C^k$ \textit{surface patch} if $\varphi$ is $C^k$ and $d\varphi_q : \R^2 \to \R^3$ is one-to-one for all $q \in U$. This is analogous to path.
\end{definition}

Remark that there is NO "arc length reparametrization" of surface patches ! We will look at Gauss' Theorema Egregium and Isothermal coordinates.

\begin{definition}
    Fix $q \in U$. The \textit{tangent space} $T_q\varphi$ to $\varphi$ at $q$ is the image of the derivative of $\varphi$ at $q$.
\end{definition}

\begin{proposition}
    If $\tilde{\varphi} : \tilde{U} \to \R^3$ is a regular $C^k$ reparametrization of $\varphi : U \to \R^3$ using the change of parameters $h : \tilde{U} \to U$, then $T_{\tilde{q}}\tilde{\varphi} = T_q \varphi$ where $q = h(\tilde{q})$.
\end{proposition}

\begin{proof}
    By definition, $T_{\tilde{q}}\tilde{\varphi} = d \tilde{\varphi}_{\tilde{q}}(\R^2)$. By the chain rule:
    $$d \tilde{\varphi}_{\tilde{q}}(\R^2) = (d \varphi_{h(\tilde{q})} \circ d h_{\tilde{q}})(\R^2).$$
    But since $h: \tilde{U} \subseteq \R^2 \to U \subseteq \R^2$ is invertible, then $dh_{\tilde{q}}(\R^2) = \R^2$. Therefore, 
    $$T_{\tilde{q}}\tilde{\varphi} = d \varphi_{h(\tilde{q})}(d h_{\tilde{q}}(\R^2)) = T_q \varphi .$$
\end{proof}

Hence, we can now define the tangent space $T_p S$ of a surface. As an exercise, we can show that the tangent space at a point $p$ in a curve in $\mathcal{C} \subseteq S$ is a one dimensional subspace of the tangent space at $p$ of the surface $S$.

Example: Let $\gamma : (0,1) \to \R^3$ be a regular $C^2$ path with $\kappa > 0$ ($\iff \{\dot{\gamma}, \ddot{\gamma}\}$ is linearly independent). Define $\varphi : \R \times (0,1) \to \R^3$ by $\varphi(s,t) = \gamma(t) + s\dot{\gamma}$, then
$$J_{\varphi}(s,t) = \begin{pmatrix}
    | & | \\
    \dot{\gamma}(t) & \dot{\gamma}(t) + s\ddot{\gamma}(t) \\
    | & |
\end{pmatrix}.$$
The columns are linearly independent if and only if $s \neq 0$. Therefore, $\varphi$ is a surface patch on the open subset $\{(s,t) \ : \ s\neq 0, \ 0 < t < 1\}$. \\

Example: Let $\mathbb{S}^2$ be the unit sphere in $\R^3$ centered at 0. Let $p = \frac{1}{\sqrt{3}}(1,1,1) \in \mathbb{S}^2$. We can parametrize the top hemisphere of this surface by the graph of the function $z = \sqrt{1 - x^2 - y^2}$, hence, $\varphi_N : U \to \mathbb{S}^2$ is given by $\varphi(x,y) = (x,y, \sqrt{1 - x^2 - y^2})$ where $U$ is the unit disk centered at the origin. Thus, we can write $p = \varphi(q)$ where $q = (\frac{1}{\sqrt{3}}, \frac{1}{\sqrt{3}})$. Using the definition of the tangent space, we get
$$J_q \varphi_N = \Im\left( \begin{bmatrix}
    1 & 0 \\ 0 & 1 \\ -1 & -1
\end{bmatrix}\right) = \text{span}\left(\begin{bmatrix} 1 \\ 0 \\ - 1\end{bmatrix}, \begin{bmatrix} 0 \\ 1 \\ - 1\end{bmatrix}\right).$$

\begin{definition}[Unit Normal]
    We say that $n \in \R^3$ is a \textit{unit normal }to $S$ at $p \in S$ if $\norm{n} = 1$ and $n \perp T_p S$. There are exactly two unit normal to $S$ at $p$.
\end{definition}

\begin{definition}[Orientation of a Surface]
    An orientation on $S$ is a continuous function $N : S \to \mathbb{S}^2$ such that $N(p)$ is a unit normal to $S$ at $p$ for all $p \in S$. We say that $S$ is orientable if there exists an orientation on $S$.
\end{definition}

Remark: If $S$ is connected and orientable, then there are exactly two orientations.

\begin{proposition}
    Level sets are orientable.
\end{proposition}

\begin{proof}
    \td Use the gradient.
\end{proof}

\td

Non-example: There are non orientable surfaces. More precisely, define $\gamma : \R \to \R^3$ by $\gamma(t) = (2\cos t, 2 \sin t, 0)$, and define $\varphi : (-1, 1) \times \R \to \R^3$ by $\varphi(s,t) = \gamma(t) + s\cos(t/2)N(t) + s\sin(t/2)B(t)$. Here, $N(t) = (-\cos t, -\sin t, 0)$ and $B(t) = (0,0,1)$. The Mobius band is defined as the image of $\varphi$. We still need to verify the regularity conditions by computing the Jacobian etc [\td]. Let's now verify that it is not orientable by focusing on the central circle at $s = 0$:
$$J_{\varphi}(0,t) = \begin{pmatrix}
    -\cos(t)\cos(t/2) & -2\sin(t) \\
    -\sin(t)\cos(t/2) & -2\cos(t) \\
    \sin(t/2) & 0
\end{pmatrix}.$$
Since the first column gives us the tangent vector to the strip at $t$, and the second column gives us the tangent to the circle at $t$, then these two vectors span the tangent plane at $t$. Hence, if we take their cross product, we get a normal vector:
$$N(t) = \frac{1}{2}d\varphi_{(0,t)}(e_1) \times d\varphi_{(0,t)}(e_2) = \begin{pmatrix}
    -\cos(t)\sin(t/2) \\
    -\sin(t)\sin(t/2) \\
    -\cos(t/2)
\end{pmatrix}.$$
But now, we see that at $t = 0$ and at $t = 2\pi$, we get two opposite vectors, hence the surface is not orientable using this potential orientation we just defined. But since this function of normal vector is continuous everywhere except at $t = 0$, then using the fact that there are only two possible orientability for a surface, we get that we actually proved that the surface is not orientable at all, not just for the function we just defined [\td].

\begin{definition}[Differentiability of a function defined on a surface]
    Let $S \subseteq \R^3$ be a regular $C^k$ surface. The function $f: S \to \R^n$ is $C^k$ if for all $p \in S$, there is a local parametrization $\varphi : U \to S$ near $p$ such that $f \circ \varphi : U \to \R^n$ is $\C^k$. The expression $f\circ \varphi$ is sometimes called $f$ in local coordinates.
\end{definition}

\begin{lemma}
    Let $S,\tilde{S} \subseteq \R^3$ be regular $C^k$ surfaces. A function $f : \tilde{S} \to S$ is $C^k$ if and only if for all $\tilde{p} \in \tilde{S}$, there are local parametrizations $\tilde{\varphi} : \tilde{U} \to \tilde{S}$ near $\tilde{p}$ and $\varphi : U \to S$ near $f(\tilde{p})$ such that $\varphi^{-1} \circ f \circ \tilde{\varphi} : \tilde{U} \to U$ is $C^k$.
\end{lemma}

\noindent Exercise: $f : \tilde{S} \to S$ a $C^1$ function, then there is a well-defined derivative $df_{\tilde{p}} : T_{\tilde{p}}\tilde{S} \to T_{f(\tilde{p})}S$. [\td]\\

\noindent Exercise: Take the unit sphere $\S^2 \subseteq \R^3$. Define the stereographic projection $\pi_N : \R^2 \to \S^2$ based at the north pole $N = (0,0,1)$ by etc. Fix a constant $c \in \R^2$ and let $M_c : \R^2 \to \R^2$ be the rigid motion that translates by $c$: $M_c(p) = p+c$. From this, define $f_c : \S^2 \to \S^2$ by
$$f_c(p) = \begin{cases}
    N & p=N,\\
    \pi_N\circ M_c \circ \pi_n^{-1}(p) & p \neq N.
\end{cases}$$
The exercise is to show that $f_c$ is smooth (especially at $N$) [\td].\\

\begin{definition}
    Let $S \subseteq \R^3$ be a regular surface. The \textit{First Fundamental Form} assigns to each $p \in S$ the quadratic form $\norm{\cdot}^2 : T_p S \to \R$.
\end{definition}

\begin{definition}
    Suppose $f : \tilde{S} \to S$ is $C^1$, we say that it is a \textit{local isometry} if for all vectors $\tilde{v} \in T_{\tilde{p}}\tilde{S}$, $\norm{\tilde{v}}^2 = \norm{df_{\tilde{p}}(\tilde{v})}^2$.
\end{definition}

\begin{definition}
    The function $f : \tilde{S} \to S$ is an \textit{isometry} if it is a bijective local isometry.
\end{definition}

\noindent Exercise: If $f : \tilde{S} \to S$ is a restriction of a rigid motion $M : \R^3 \to \R^3$, then $f$ is an isometry [\td].

In local coordinates, there is some $\varphi : U \to S$ which is a local parametrization near a point $p \in S$. For each $q \in U$, we get a a quadratic form $I_q : \R^2 \to \R$ that sends $u \in \R^2$ to $\norm{d\varphi_q(u)}^2$. We can define an inner product on $\R^2$ for each $q \in U$:
$$\langle u_1, u_2 \rangle_q = \frac{1}{4}(I_q(u_1 + u_2) - I_q(u_1 - u_2)).$$

\noindent Exercise: Check that this is indeed an inner product [\td].