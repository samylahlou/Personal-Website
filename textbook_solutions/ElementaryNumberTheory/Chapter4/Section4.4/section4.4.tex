\section{Linear Congruences}

\begin{exercise}
    Solve the following linear congruences:
    \begin{enumerate}
        \item $25x \equiv 15 \pmod{29}$.
        \item $5x \equiv 2 \pmod{26}$.
        \item $6x \equiv 15 \pmod{21}$.
        \item $36x \equiv 8 \pmod{102}$.
        \item $34x \equiv 60 \pmod{98}$.
        \item $140x \equiv 133 \pmod{301}$. [\textit{Hint:} $\gcd(140, 301) = 7$.]
    \end{enumerate}
\end{exercise}

\begin{solution}
    \begin{enumerate}
        \item Let $d = \gcd(25, 29) = 1$, then by Theorem 4-7, there is a unique solution. Since $7\cdot 25 - 6 \cdot 29 = 1$, then $7 \cdot 25 \equiv 1 \pmod{29}$. Hence, multiplying the original equation by 7 on both sides gives us $x \equiv 7 \cdot 15 \equiv 105 \equiv 18 \pmod{29}$. Therefore, $x = 18$ is the unique solution.
        \item Let $d = \gcd(5, 26) = 1$, then by Theorem 4-7, there is a unique solution. Since $1\cdot 26 - 5 \cdot 5 = 1$, then $(-5) \cdot 5 \equiv 1 \pmod{26}$. Hence, multiplying the original equation by $-5$ on both sides gives us $x \equiv (-5) \cdot 2 \equiv -10 \equiv 16 \pmod{26}$. Therefore, $x = 16$ is the unique solution.
        \item Let $d = \gcd(6, 21) = 3$, then by Theorem 4-7, there are three incongruent solutions since $d \mid 15$. To find a special solution, divide the original congruence by 3 to get the equation $2x \equiv 5 \pmod{7}$ and multiply both sides of the congruence by $4$ to get $x \equiv 20 \equiv 6 \pmod{7}$. It follows that the solutions of the original equation are $x = 6 + 7t$ for $t = 0,1,2$. Therefore, the solutions are precisely $x = 6, 13, 20$.
        \item Let $d = \gcd(36, 102) = 6$. Since 8 is not divisible by $d$, then the equation has no solutions.
        \item Let $d = \gcd(34, 98) = 2$, then by Theorem 4-7, there are two incongruent solutions since $d \mid 60$. To find a special solution, divide the original congruence by 2 to get the equation $17x \equiv 30 \pmod{49}$ and multiply both sides of the congruence by $23$ to get
        $$x \equiv 23\cdot 30 \equiv 690 \equiv 4 \pmod{49}.$$ It follows that the solutions of the original equation are $x = 4 + 49t$ for $t = 0,1$. Therefore, the solutions are precisely $x = 4$ and $x = 53$.
        \item Let $d = \gcd(140, 301) = 7$, then by Theorem 4-7, there are seven incongruent solutions since $d \mid 133$. To find a special solution, divide the original congruence by 7 to get the equation $20x \equiv 19 \pmod{43}$ and multiply both sides of the congruence by $15$ to get $x \equiv 285 \equiv 27 \pmod{43}$. It follows that the solutions of the original equation are $x = 27 + 43t$ for $t = 0,1,2,3,4,5,6$. Therefore, the solutions are precisely $x = 27, 70, 113, 156, 199, 242, 285$. \\
    \end{enumerate}
\end{solution}

\begin{exercise}
    Using congruences, solve the Diophantine equations below:
    \begin{enumerate}
        \item $4x + 51y = 9$. [\textit{Hint:} $4x \equiv 9 \pmod{51}$ gives $x = 15 + 51t$, while $51y \equiv 9 \pmod{4}$ gives $y = 3 + 4s$. Find the relation between $s$ and $t$.]
        \item $12x + 25y = 331$.
        \item $5x - 53y = 17$.
    \end{enumerate}
\end{exercise}

\begin{solution}
    \begin{enumerate}
        \item First, take the equation modulo 51 to get the congruence $4x \equiv 9 \pmod{51}$. Since $4 \cdot 13 - 51 = 1$, then mumtiplying the congruence by $13$ on both sides gives us $x \equiv 9 \cdot 13 \equiv 117 \equiv 15 \pmod{51}$. Hence, $x = 15 + 51t$. Similarly, we can take the original equation modulo 4 to get the congruence $3y \equiv 1 \pmod{4}$. Multiplying by 3 on both sides gives $x \equiv 3 \pmod{4}$. Thus, $y = 3 + 4s$. Plugging $x = 15 + 51t$ and $y = 3 + 4s$ in the original equation gives us 
        $$4(15 + 51t) + 51(3 + 4s) = 9$$
        which can be simplified into $t + s = -1$ and so $s = -1 - t$. Hence, we have $x = 15 + 51t$ and $s = -1 - 4t$ for an integer $t$. Let's now prove that any pair of integers of this form is a solution of the equation:
        $$4(15 + 51t) + 51(-1-4t) = 4 \cdot 15 -51 = 9$$
        for every integer $t$. Therefore, the solutions are precisely $x = 15 + 51t$ and $y = -1 - 4t$ for all integers $t$.
        \item First, take the equation modulo 25 to get the congruence $12x \equiv 6 \pmod{25}$. Since $25 - 2 \cdot 12 = 1$, then mumtiplying the congruence by $-2$ on both sides gives us $x \equiv (-2) \cdot 6 \equiv 13 \pmod{25}$. Hence, $x = 13 + 25t$. Similarly, we can take the original equation modulo 12 to get the congruence $y \equiv 7 \pmod{12}$. Thus, $y = 7 + 12s$. Plugging $x = 13 + 25t$ and $y = 7 + 12s$ in the original equation gives us 
        $$12(13 + 25t) + 25(7 + 12s) = 331$$
        which can be simplified into $t + s = 0$ and so $s =- t$. Hence, we have $x = 13 + 25t$ and $s = 7 - 12t$ for an integer $t$. Let's now prove that any pair of integers of this form is a solution of the equation:
        $$12(13 + 25t) + 25(7-12t) = 12 \cdot 13 + 25 \cdot 7 = 331$$
        for every integer $t$. Therefore, the solutions are precisely $x = 13 + 25t$ and $y = 7 - 12t$ for all integers $t$.
        \item First, take the equation modulo 53 to get the congruence $5x \equiv 17 \pmod{53}$. Since $2 \cdot 53 - 21\cdot 5 = 1$, then multiplying the congruence by $-21$ on both sides gives us $x \equiv (-21) \cdot 17 \equiv -39 \equiv 14 \pmod{53}$. Hence, $x = 14 + 53t$. Similarly, we can take the original equation modulo 5 to get the congruence $2y \equiv 2 \pmod{5}$. Multiplying by 3 on both sides gives $x \equiv 1 \pmod{5}$. Thus, $y = 1 + 5s$. Plugging $x = 14 + 53t$ and $y = 1 + 5s$ in the original equation gives us 
        $$5(14 + 53t) - 53(1 + 5s) = 17$$
        which can be simplified into $t -s = 0$ and so $s = t$. Hence, we have $x = 14 + 53t$ and $s = 1 + 5t$ for an integer $t$. Let's now prove that any pair of integers of this form is a solution of the equation:
        $$5(14 + 53t) - 53(1 + 5t) = 5 \cdot 14 - 53 = 17$$
        for every integer $t$. Therefore, the solutions are precisely $x = 14 + 53t$ and $y = 1 + 5t$ for all integers $t$. \\
    \end{enumerate}
\end{solution}

\begin{exercise}
    Find all solutions of the linear congruence $3x - 7y \equiv 11 \pmod{13}$. \\
\end{exercise}

\begin{solution}
    First, fix $y = t$ an rearrange the equation into $3x \equiv 11 + 7t \pmod{13}$. Since $\gcd(3, 13) = 1$, then this equation has precisely one solution. To find it, multiply both sides of the congruence by 9 to get $x \equiv 9(11 + 7t) \pmod{13}$. Hence, we have that the solutions are $x = 9(11 + 7t)$, $y = t$ for all $t$ between 0 and 12. \\
\end{solution}

\begin{exercise}
    Solve each of the following sets of simultaneous congruences:
    \begin{enumerate}
        \item $x \equiv 1 \pmod 3$, $x \equiv 2 \pmod 5$, $x \equiv 3 \pmod 7$
        \item $x \equiv 5 \pmod{11}$, $x \equiv 14 \pmod{29}$, $x \equiv 15 \pmod{31}$
        \item $x \equiv 5 \pmod 6$, $x \equiv 4 \pmod{11}$, $x \equiv 3 \pmod{17}$
        \item $2x \equiv 1 \pmod 5$, $3x \equiv 9 \pmod 6$, $4x \equiv 1 \pmod 7$, $5x \equiv 9 \pmod{11}$
    \end{enumerate}
\end{exercise}

\begin{solution}
    \begin{enumerate}
        \item Let $n = 3 \cdot 5 \cdot 7 = 105$, $N_1 = n / 3 = 35$, $N_2 = n / 5 = 21$ and $N_3 = n / 7 = 15$. Consider the linear congruences 
        $$35x \equiv 1 \pmod{3}, \qquad 21x \equiv 1 \pmod{5}, \qquad 15x \equiv 1 \pmod{7}$$
        which have the respective solutions $x_1 = 2$, $x_2 = 1$ and $x_3 = 1$. Define the integer $\overline{x} = 52 \equiv 1\cdot 35 \cdot 2 + 2 \cdot 21 \cdot 1 + 3 \cdot 15 \cdot 1 \pmod{n}$. By Theorem 4-8, we have that general solution is given by $x = 52 + 105t$ where $t$ is any integer.
        \item Let $n = 11 \cdot 29 \cdot 31 = 9889$, $N_1 = n / 11 = 899$, $N_2 = n / 29 = 341$ and $N_3 = n / 31 = 319$. Consider the linear congruences 
        $$899x \equiv 1 \pmod{11}, \qquad 341x \equiv 1 \pmod{29}, \qquad 319x \equiv 1 \pmod{31}$$
        which have the respective solutions $x_1 = 7$, $x_2 = 4$ and $x_3 = 7$. Define the integer $\overline{x} = 4944 \equiv 5\cdot 899 \cdot 7 + 14 \cdot 341 \cdot 4 + 15 \cdot 319 \cdot 7 \pmod{n}$. By Theorem 4-8, we have that general solution is given by $x = 4944 + 9889t$ where $t$ is any integer.
        \item Let $n = 6 \cdot 11 \cdot 17 = 1122$, $N_1 = n / 6 = 187$, $N_2 = n / 11 = 102$ and $N_3 = n / 17 = 66$. Consider the linear congruences 
        $$187x \equiv 1 \pmod{6}, \qquad 102x \equiv 1 \pmod{11}, \qquad 66x \equiv 1 \pmod{17}$$
        which have the respective solutions $x_1 = 1$, $x_2 = 4$ and $x_3 = 8$. Define the integer $\overline{x} = 785 \equiv 5\cdot 187 \cdot 1 + 4 \cdot 102 \cdot 4 + 3 \cdot 66 \cdot 8 \pmod{n}$. By Theorem 4-8, we have that general solution is given by $x = 785 + 1122t$ where $t$ is any integer.
        \item First, let's rewrite and simplify each individual congruences. The congruence $2x \equiv 1 \pmod 5$ is equivalent to $x \equiv 3 \pmod 5$ by multiplying by 3 on both sides. Let's solve the congruence $3x \equiv 9 \pmod{6}$ using the methods that we developed in this section. If we let $d = \gcd(3, 6) = 3$, then we have that $d \mid 9$ which implies that the congruence has 3 solutions modulo 6. To find these solutions, divide the congruence by $d$ to get the congruence $x \equiv 1 \pmod 2$. It follows that the solutions are $1, 3, 5$ which correspond precisely to the odd integers. Therefore, the congruence $3x \equiv 9 \pmod 6$ is precisely equivalent to the congruence $x \equiv 1 \pmod{2}$. Next, multiplying on both sides of the congruence $4x \equiv 1 \pmod 7$ by 2 shows that it is equivalent to the congruence $x \equiv 2 \pmod 7$. Finally, multiplying on both sides of the congruence $5x \equiv 9 \pmod{11}$ by 9 shows that it is equivalent to the congruence $x \equiv 4 \pmod{11}$. Therefore, the system becomes 
        $$x \equiv 3 \pmod 5, \quad x \equiv 1 \pmod 2, \quad x \equiv 2 \pmod 7, \quad x \equiv 4 \pmod{11}.$$
        Let $n = 5 \cdot 2 \cdot 7 \cdot 11 = 770$, $N_1 = n / 5 = 154$, $N_2 = n / 2 = 385$, $N_3 = n / 7 = 110$ and $N_4 = n / 11 = 70$. Consider the linear congruences
        \begin{align*}
            154x &\equiv 1 \pmod{5}, &385x &\equiv 1 \pmod{2}, \\
            110x &\equiv 1 \pmod{7}, &70x &\equiv 1 \pmod{11},
        \end{align*} 
        which have the respective solutions $x_1 = 4$, $x_2 = 1$, $x_3 = 3$ and $x_4 = 3$. Define the integer $\overline{x} = 653 \equiv 3\cdot 154 \cdot 4 + 1\cdot 385 \cdot 1 + 2\cdot 110 \cdot 3 + 4 \cdot 70 \cdot 3 \pmod{n}$. By Theorem 4-8, we have that general solution is given by $x = 653 + 770t$ where $t$ is any integer. \\
    \end{enumerate}
\end{solution}

\begin{exercise}
    Solve the linear congruence $17x \equiv 3 \pmod{2\cdot 3\cdot 5 \cdot 7}$ by solving the system
    $$17x \equiv 3 \pmod 2, \ 17x \equiv 3 \pmod 3, \ 17x \equiv 3 \pmod 5, \ 17x \equiv 3 \pmod 7.$$
\end{exercise}

\begin{solution}
    First, let's solve the system of equation. Let's rewrite it as follows:
    $$x \equiv 1 \pmod 2, \quad 2x \equiv 0 \pmod 3, \quad 2x \equiv 3 \pmod 5, \quad 3x \equiv 3 \pmod 7.$$
    Next, multiply both sides of the second congruence by 2, multiply both sides of the third congruence by 3 and cancel out the 3's in the last congruence to obtain the system
    $$x \equiv 1 \pmod 2, \quad x \equiv 0 \pmod 3, \quad x \equiv 4 \pmod 5, \quad x \equiv 1 \pmod 7.$$
    Let's solve this system using the method developed before. Let $n = 2 \cdot 3 \cdot 5 \cdot 7 = 210$, $N_1 = n / 2 = 105$, $N_2 = n / 3 = 70$, $N_3 = n / 5 = 42$ and $N_4 = n / 7 = 30$. Let's now solve the congruences $N_i x_i = 1 \pmod{n_i}$ where $n_1 = 2$, $n_2 = 3$, $n_3 = 5$ and $n_4 = 7$. We get that $x_1 = 1$, $x_2 = 1$, $x_3 = 3$ and $x_4 = 4$. Now, we can define
    $$\overline{x} = 99 \equiv 1 \cdot 105 \cdot 1 + 0 \cdot 70 \cdot 1 + 4 \cdot 42 \cdot 3 + 1 \cdot 30 \cdot 4.$$
    By Theorem 4-8, we have that 99 solves the original system of equation. In other words, we have that 2, 3, 5 and 7 divide the quantity $17\cdot 99 - 3$. Since they are relatively prime, we get that their product also divides this quantity. Therefore, we have that $17 \cdot 99 \equiv 3 \pmod{2 \cdot 3 \cdot 5 \cdot 7}$ and so $x = 99$ is the unique solution modulo $2 \cdot 3 \cdot 5 \cdot 7$. \\
\end{solution}

\begin{exercise}
    Find the smallest integer $a > 2$ such that 
    $$2 \mid a, \ 3 \mid a + 1, \ 4 \mid a + 2, \ 5 \mid a+ 3, \ 6 \mid a + 4.$$
\end{exercise}

\begin{solution}
    First, notice that we can discard the condition that $2 \mid a$ since $4 \mid a + 2$ already implies it. Moreover, we rewrite each of these conditions as the following congruences:
    $$a \equiv 2 \pmod{3}, \quad a \equiv 2 \pmod{4}, \quad a \equiv 2 \pmod{5}, \quad a \equiv 2 \pmod{6}.$$
    If we focus on the first three congruences, then by Theorem 4-8, we have that there is a unique solution modulo $3 \cdot 4 \cdot 5 = 60$. Since 2 is clearly a solution, then now have to solve the congruence $a \equiv 2 \pmod 6$ such that $a = 2 + 60t > 2$. When $t = 1$, we have that $a = 62$ satisfies $a \equiv 2 \pmod 6$. Thus, $a = 62$ is indeed the smallest integer $a > 2$ such that the divisibility conditions are satisfied. \\
\end{solution}

\begin{exercise}
    \begin{enumerate}
        \item Obtain three consecutive integers each having a square factor. [\textit{Hint:}  Find an integer $a$ such that $2^2 \mid a$, $3^2 \mid a + 1$, $5^2 \mid a + 2$.]
        \item Obtain three consecutive integers, the first of which is divisible by a square, the second by a cube, and the third by a fourth power.
    \end{enumerate}
\end{exercise}

\begin{solution}
    \begin{enumerate}
        \item Let's find an integer $a$ such that $2^2 \mid a$, $3^2 \mid a+ 1$ and $5^2 \mid a + 2$. Equivalently, let's solve the system
        $$a \equiv 0 \pmod{2^2}, \qquad a \equiv - 1 \pmod{3^2}, \qquad a \equiv - 2 \pmod{5^2}$$
        by applying the method developed in this section. Let $n = 2^2 \cdot 3^2 \cdot 5^2 = 900$, $N_1 = n / 2^2 = 225$, $N_2 = n / 3^2 = 100$ and $N_3 = n / 5^2 = 36$. Next, notice that $x_2 = 1$ and $x_3 = 16$ satisfy the equations $N_2 x_2 \equiv 1 \pmod{3^2}$ and $N_3 x_3 \equiv 1 \pmod{5^2}$ so we can define the integer 
        $$\overline{x} = 548 \equiv 0 + (-1) \cdot 100 \cdot 1 + (-2) \cdot 36 \cdot 16 \pmod{900}.$$
        It follows that the three consecutive integers 548, 549, 550 are respectively divisible by $2^2$, $3^2$ and $5^2$.
        \item Let's find an integer $a$ such that $5^2 \mid a$, $3^3 \mid a+ 1$ and $2^4 \mid a + 2$. Equivalently, let's solve the system
        $$a \equiv 0 \pmod{5^2}, \qquad a \equiv - 1 \pmod{3^3}, \qquad a \equiv - 2 \pmod{2^4}$$
        by applying the method developed in this section. Let $n = 5^2 \cdot 3^3 \cdot 2^4 = 10800$, $N_1 = n / 5^2 = 432$, $N_2 = n / 3^3 = 400$ and $N_3 = n / 2^4 = 675$. Next, notice that $x_2 = 16$ and $x_3 = 11$ satisfy the equations $N_2 x_2 \equiv 1 \pmod{3^3}$ and $N_3 x_3 \equiv 1 \pmod{2^4}$ so we can define the integer 
        $$\overline{x} = 350 \equiv 0 + (-1) \cdot 400 \cdot 16 + (-2) \cdot 675 \cdot 11 \pmod{10800}.$$
        It follows that the three consecutive integers 350, 351, 352 are respectively divisible by $5^2$, $3^3$ and $2^4$. \\
    \end{enumerate}
\end{solution}

\begin{exercise}
    (Brahmagupta, 7th century A.D.) When eggs in a basket are removed 2,3,4,5,6 at a time there remain respectively, 1, 2, 3, 4, 5 eggs. When they are taken out 7 at a time, none are left over. Find the smallest number of eggs that could have been contained in the basket. \\
\end{exercise}

\begin{solution}
    First, let's rewrite this problem as a system of congruence. Let $x$ be the number of eggs in the basket, then the first part of the problem states that $x \equiv -1 \pmod{i}$ for all $i = 2,3,4,5,6$. The second part states that $x \equiv 0 \pmod 7$. Hence, we have the system
    $$x \equiv -1 \pmod 2, \qquad x \equiv -1 \pmod 3, \qquad x \equiv -1 \pmod 4,$$
    $$x \equiv -1 \pmod 5, \qquad x \equiv -1 \pmod 6, \qquad x \equiv 0 \pmod 7.$$
    Notice that the congruences $x \equiv -1 \pmod 2$ and $x \equiv -1 \pmod 3$ are equivalent to the fact that 2 and 3 divide $x + 1$. Since they are relatively prime, then $6 \mid x + 1$ and so we get that the congruence $x \equiv -1 \pmod 6$ is already implied by the others. Hence, we can remove it from the system since it is redundant. Next, from the congruence $x \equiv -1 \pmod{4}$, we directly get that $x \equiv -1 \pmod 2$ and so this latter congruence can also be removed from the system. Hence, our system is now
    \begin{align*}
        x &\equiv -1 \pmod 3, & x &\equiv -1 \pmod 4, \\
        x &\equiv -1 \pmod 5, & x &\equiv 0 \pmod 7.
    \end{align*}
    Now, notice that 3, 4, and 5 are relatively prime and so by the method we developed in this section, there is a unique solution modulo $3 \cdot 4 \cdot 5 = 60$. Since $-1$ is clearly a solution, then the previous system is equivalent to
    $$x \equiv -1 \pmod{60}, \qquad x \equiv 0 \pmod 7.$$
    Using the first congruence, we get that $x$ must be one of the numbers 59, 119, 179, \dots which are of the form $60t - 1$. By looking at these numbers, we see that 59 is not divisible by 7 but 119 is. Therefore, the smallest number of eggs that could have been contained in the basket is 119. \\
\end{solution}

\begin{exercise}
    The basket-of-eggs problem problem is often phrased in the following form: One egg remains when the eggs are removed from the basket 2, 3, 4, 5, or 6 at a time; but, no eggs remain if they are removed 7 at a time. Find the smallest number of eggs that could have been in the basket. \\
\end{exercise}

\begin{solution}
    The solution to this problem will be very similar to the solution to the previous one. First, let's rewrite this problem as a system of congruence. Let $x$ be the number of eggs in the basket, then the first part of the problem states that $x \equiv 1 \pmod{i}$ for all $i = 2,3,4,5,6$. The second part states that $x \equiv 0 \pmod 7$. Hence, we have the system
    $$x \equiv 1 \pmod 2, \qquad x \equiv 1 \pmod 3, \qquad x \equiv 1 \pmod 4,$$
    $$x \equiv 1 \pmod 5, \qquad x \equiv 1 \pmod 6, \qquad x \equiv 0 \pmod 7.$$
    Notice that the congruences $x \equiv 1 \pmod 2$ and $x \equiv 1 \pmod 3$ are equivalent to the fact that 2 and 3 divide $x - 1$. Since they are relatively prime, then $6 \mid x - 1$ and so we get that the congruence $x \equiv 1 \pmod 6$ is already implied by the others. Hence, we can remove it from the system since it is redundant. Next, from the congruence $x \equiv 1 \pmod{4}$, we directly get that $x \equiv 1 \pmod 2$ and so this latter congruence can also be removed from the system. Hence, our system is now
    \begin{align*}
        x &\equiv 1 \pmod 3, & x &\equiv 1 \pmod 4, \\
        x &\equiv 1 \pmod 5, & x &\equiv 0 \pmod 7.
    \end{align*}
    Now, notice that 3, 4, and 5 are relatively prime and so by the method we developed in this section, there is a unique solution modulo $3 \cdot 4 \cdot 5 = 60$. Since 1 is clearly a solution, then the previous system is equivalent to
    $$x \equiv 1 \pmod{60}, \qquad x \equiv 0 \pmod 7.$$
    Using the first congruence, we get that $x$ must be one of the numbers 61, 121, 181, 241 \dots which are of the form $60t + 1$. By looking at these numbers, we see that 61, 121 and 181 are not divisible by 7 but $241 = 7 \cdot 33$ is. Therefore, the smallest number of eggs that could have been contained in the basket is 241. \\
\end{solution} 

\begin{exercise}
    (Ancient Chinese Problem.) A band of 17 pirates stole a sack of gold coins. When they tried to divide the fortune into equal portions, 3 coins remained. In the ensuing brawl over who should get the extra coins, one pirate was killed. The wealth was redistributed, but this time an equal division left 10 coins. Again an argument developed in which another pirate was killed. But now the total fortune was evenly distributed among the survivors. What was the least number of coins that could have been stolen. \\
\end{exercise}

\begin{solution}
    First, let's translate this problem into a system of congruences. If we denote by $x$ the number of stolen coins, we get the following system:
    $$x \equiv 3 \pmod{17}, \qquad x \equiv 10 \pmod{16}, \qquad x \equiv 0 \pmod{15}.$$
    Since 17, 16 and 15 are relatively prime, then we can directly apply the method developed in this section. Let $a_1 = 3$, $a_2 = 10$, $a_3 = 0$, $n_1 = 17$, $n_2 = 16$, $n_3 = 15$, $n = n_1n_2n_3 = 4080$, $N_1 = n/n_1 = 240$, $N_2 = n / n_2 = 255$ and $N_3 = n / n_3 = 272$. Let's now solve the congruences $N_i x_i \equiv 1 \pmod{n_i}$ to obtain $x_1 = 9$ and $x_2 = -1$. Now, define 
    \begin{align*}
        \overline{x} &= a_1N_1x_1 + a_2 N_2 x_2 + a_3 N_3 x_3 \\
        &= 3 \cdot 240 \cdot 9 + 10 \cdot 255 \cdot (-1) + 0 \\
        &= 6480 - 2550 \\
        &= 3930,
    \end{align*}
    then by Theorem 4-8, $\overline{x}$ is the unique solution to this system modulo $4080$. Since 3930 is the least positive integer of the form $\overline{x} + 4080t$, then 3930 is the least number of coins that could have been stolen. \\
\end{solution}

\begin{exercise}
    Prove that the congruences
    $$x \equiv a \pmod n  \ \text{ and } \ x \equiv b \pmod m$$
    admit a simultaneous solution if and only if $\gcd(n,m) \mid a - b$; if a solution exists, confirm that it is unique modulo $\lcm(n,m)$.\\
\end{exercise}

\begin{solution}
    First, suppose that a solution $x$ exists, then $x = a + kn$ and $x = b + k'm$ for some integers $k$ and $k'$. It follows that $a + kn = b + k'm$ and so $a - b = -kn + k'm$. Since $\gcd(n,m) \mid -kn + k'm$, then $\gcd(n,m) \mid a - b$. Conversely, if $xn + ym = \gcd(n,m) \mid a-b$, then $a - b = kxn + kym$ for some integer $k$. If we let $x = a - kxn = b + kym$, then it is clear that $x \equiv a \pmod n$ and $x \equiv b \pmod m$. This proves the equivalence.

    Suppose now that we have two solutions $x$ and $x'$, then $x \equiv x' \pmod a$ and $x \equiv x' \pmod m$. Equivalently, we have that $n$ and $m$ divide $x - x'$ and so $\lcm(n,m) \mid x - x'$. Therefore, $x \equiv x' \pmod{\lcm(n,m)}$ which implies that it is unique modulo $\lcm(n,m)$. \\
\end{solution}

\begin{exercise}
    Use Problem 11 to show that the system
    $$x \equiv 5 \pmod 6  \ \text{ and } \ x \equiv 7 \pmod{15}$$
    does not possess a solution. \\
\end{exercise}

\begin{solution}
    Since $\gcd(6, 15) = 3$ does not divide $5 - 7 = -2$, then by Problem 11, there is no simultaneous solution to this system of congruences. \\
\end{solution}

\begin{exercise}
    If $x \equiv a \pmod n$, prove that either $x \equiv a \pmod{2n}$ or $x \equiv a + n \pmod{2n}$. \\
\end{exercise}

\begin{solution}
    We have that $x = a + kn$ for some integer $k$ and by the Division Algorithm, we also have that $x = b + 2k'n$ for some integer $k'$. It follows that $a + kn = b + 2k'n$ and so $b = a + n(k - 2k') = a + nk_0$. If we take this equation modulo $2n$, we get that $nk_0$ is either congruent to 0 or $n$ modulo $2n$ and so $b \equiv a \pmod{2n}$ or $b \equiv a + n \pmod{2n}$. It follows that $x \equiv a \pmod{2n}$ or $x \equiv a + n \pmod{2n}$. \\
\end{solution}

\begin{exercise}
    A certain integer between 1 and 1200 leaves the remainders 1, 2, 6 when divided by 9, 11, 13 respectively. What is the integer? \\
\end{exercise}

\begin{solution}
    If we translate this problem into a system of congruences, we get that the number, which we will call $x$, must satisfy
    $$x \equiv 1 \pmod 9, \qquad x \equiv 2 \pmod{11}, \qquad x \equiv 6 \pmod{13}.$$
    Since 9, 11, and 13 are all relatively prime, then we can apply Theorem 4-8. Let $n = 9 \cdot 11 \cdot 13 = 1287$, $N_1 = 11 \cdot 13 = 143$, $N_2 = 9 \cdot 13 = 117$ and $N_3 = 9 \cdot 11 = 99$. Let $x_i$ denote the inverse of $N_i$ modulo 9, 11, and 13 successively, then $x_1 = 8$, $x_2 = -3$ and $x_3 = 5$. Hence, we can define $\overline{x} = 838 \equiv 1 \cdot N_1x_1 + 2 \cdot N_2x_2 + 6 \cdot N_3x_3 \pmod n$. By Theorem 4-8, the solutions to the system are precisely $838 + 1287t$ where $t$ is an integer. Since we want $1 \leq x \leq 1200$, then 838 is the integer we want. \\
\end{solution}

\begin{exercise}
    \begin{enumerate}
        \item Find an integer having the remainders 1, 2, 5, 5 when divided by 2, 3, 6, 12, respectively. (Yih-hing, died 717.)
        \item Find an integer having the remainders 2, 3, 4, 5 when divided by 3, 4, 5, 6, respectively. (Bhaskara, born 1114.)
        \item Find an integer having the remainders 3, 11, 15 when divided by 10, 13, 17, respectively. (Regiomontanus, 1436-1473)
    \end{enumerate}
\end{exercise}

\begin{solution}
    \begin{enumerate}
        \item It suffices to find a simultaneous solution to the congruences
        \begin{align*}
            x &\equiv 1 \pmod{2} & x &\equiv 2 \pmod{3} \\
            x &\equiv 5 \pmod{6} & x &\equiv 5 \pmod{12}
        \end{align*}
        First, notice that the first two congruences can be rewritten as
        $$x \equiv -1 \pmod 2 \qquad x \equiv - 1 \pmod{3}.$$
        Since 2 and 3 are relatively prime, then applying Theorem 4-8 tells us that there is a unique solution modulo 6. Since -1 is clearly a solution, then these two equations are equivalent to the equation $x \equiv -1 \equiv 5 \pmod{6}$ which was already in our system. Thus, we only need to find a simultaneous solution to the equations
        $$x \equiv 5 \pmod{6} \qquad x \equiv 5 \pmod{12}.$$
        But now, notice that the second equation implies the first one so the only thing we need to do is to find an integer $x$ such that its residue modulo 12 is 5. If we take $x = 5$, then it clearly satisfies all the conditions.
        \item It suffices to find a simultaneous solution to the congruences
        \begin{align*}
            x &\equiv -1 \pmod{3} & x &\equiv -1 \pmod{4} \\
            x &\equiv -1 \pmod{5} & x &\equiv -1 \pmod{6}.
        \end{align*}
        From the first and second congruences, we have that both 3 and 4 divide $x + 1$ and so 12 divide $x + 1$ as well ($\gcd(3,4) = 1$). This clearly implies that $6 \mid x + 1$ and hence, $x \equiv -1 \pmod 6$. It follows that we don't need to consider the last congruence, and so we need to solve the following system:
        $$x \equiv -1 \pmod{3}, \qquad x \equiv -1 \pmod{4}, \qquad x\equiv -1 \pmod 5.$$
        Since 3, 4 and 5 are relatively prime, then Theorem 4-8 tells us that there is a unique solution modulo $3 \cdot 4 \cdot 5 = 60$. Since $-1$ is clearly a solution, then we have that $60t - 1$ represents all possible solutions. Thus, we can take $x = 59$.
        \item We need to solve the following system of congruences:
        $$x \equiv 3 \pmod{10}, \qquad x \equiv 11 \pmod{13}, \qquad x\equiv 15 \pmod{17}.$$
        Since 10, 13 and 17 are relatively prime, then we can directly apply the method of Theorem 4-8. Let $n = 10 \cdot 13 \cdot 17 = 2210$, $N_1 = n / 10 = 221$, $N_2 = n / 13 = 170$ and $N_3 = n / 17 = 130$. After solving the equations $N_i x_i \equiv 1 \pmod{n_i}$, we get that $x_1 = 1$, $x_2 = 1$ and $x_3 = -3$. From this, we can define the integer $\overline{x} = 1103 \equiv 3 \cdot 221 \cdot 1 + 11 \cdot 170 \cdot 1 + 15 \cdot 130 \cdot (-3) \pmod n$. Indeed, we can check that 1103 checks all the conditions. \\
    \end{enumerate}
\end{solution} 

\begin{exercise}
    Let $t_n$ denote the $n$th triangular number. For which values of $n$ does $t_n$ divide $t_1^2 + t_2^2 + \dots + t_n^2$? [\textit{Hint:} Since $t_1^2 + t_2^2 + \dots + t_n^2 = t_n(3n^3 + 12n^2 + 13n + 2)/30$, it suffices to determine those $n$ satisfying $3n^3 + 12n^2 + 13n + 2 \equiv 0 \pmod{2 \cdot 3 \cdot 5}$.]\\
\end{exercise}

\begin{solution}
    As the hint tells us, lets find the $n$'s satisfying $3n^3 + 12n^2 + 13n + 2 \equiv 0 \pmod{2 \cdot 3 \cdot 5}$. By the Chinese Remainder Theorem, it suffices to solve the following system of congruences:
    \begin{align*}
        n^3 + n&\equiv 0 \pmod{2} \\
        n + 2 &\equiv 0 \pmod{3} \\
        (3n+2)(n^2+1) &\equiv 0 \pmod{5} 
    \end{align*}
    which we obtain after simplifications. First, notice that $n^3 + n$ is always even so the first equation can be discarded. The second equation can be replaced by $n \equiv 1 \pmod 3$ using the rules of congruences. The last equation tells us that 5 divides one of $3n + 2$ or $n^2 + 1$. For the first term, this happens when $n \equiv 1 \pmod 5$. For the second term, this happens when $n \equiv 2,3 \pmod 5$. Thus, the last congruence is equivalent to $n \equiv 1,2,3 \pmod 5$. Hence, our system now becomes
    \begin{align*}
        n &\equiv 1 \pmod{3} \\
        n &\equiv 1,2,3 \pmod{5} 
    \end{align*}
    To solve it, let's solve separatly the three systems where the last equation is $n \equiv 1 \pmod 5$, $n \equiv 2 \pmod 5$ and $n \equiv 3 \pmod 5$ successively. This is not harder than solving just one system because most of the method remains unchanged. Let's apply the method of Theorem 4-8. Let $n = 3 \cdot 5 = 15$, $N_1 = n / 3 = 5$, $N_2 = n / 5 = 3$. After solving the congruences $N_i x_i \equiv 1 \pmod{n_i}$, we get $x_1 = x_2 = 2$. Hence, the solutions of the system are
    $$1 \cdot 5 \cdot 2 + i \cdot 3 \cdot 2 = 10 + 6i$$
    up to a multiple of 15 and for $i = 1,2,3$. It follows that the solutions are precisely $n \equiv 1,7, 13 \pmod{15}$. \\
\end{solution}

\begin{exercise}
    Find the solutions of the system of congruences
    \begin{align*}
        3x + 4y &\equiv 5 \pmod{13} \\
        2x + 5y &\equiv 7 \pmod{13}.
    \end{align*}
\end{exercise}

\begin{solution}
    First, notice that $9 \cdot 3 \equiv 1 \pmod{13}$ and $7 \cdot 2 \equiv 1 \pmod{13}$ so if we multiply both sides of equations 1 and 2 by 9 and 7 respectively, we get the two equations
    \begin{align*}
        x + 10y &\equiv 6 \pmod{13} \\
        x + 9y &\equiv 10 \pmod{13}.
    \end{align*}
    If we subtract the first equation by the second one, we get that $y \equiv 9 \pmod{13}$. If we plug this last congruence into the second congruence above, we get that
    $$x \equiv 10 - 9 \cdot 9 \equiv 10 - 3 \equiv 7 \pmod{13}.$$ 
    Therefore, the solutions are given by $x \equiv 7 \pmod{13}$, $y \equiv 9 \pmod{13}$.\\
\end{solution}

\begin{exercise}
    Obtain the two incongruent solutions modulo 210 of the system
    \begin{align*}
        2x &\equiv 3 \pmod{5} \\
        4x &\equiv 2 \pmod{6} \\
        3x &\equiv 2 \pmod 7. 
    \end{align*}
\end{exercise}

\begin{solution}
    By multipliying the first equation by 3, we obtain the equivalent equation $x \equiv 4 \pmod{5}$. Let's solve the second equation. If we divide the congruence by 2, we get the equation $2x \equiv 1 \pmod 3$ which is equivalent to $x \equiv 2 \pmod{3}$. If we lift this solution modulo 6, we get that the two solutions of the second equation are $x \equiv 2, 5 \pmod 6$. In the third equation, multiply both sides by 5 to get $x \equiv 3 \pmod 7$. Therefore, our system is now
    \begin{align*}
        x &\equiv 4 \pmod{5} \\
        x &\equiv 2,5 \pmod{6} \\
        x &\equiv 3 \pmod 7. 
    \end{align*}
    To solve this system, we need to solve the two independent systems where the second equation changes and merge the solutions. To do this, let's apply Theorem 4-8. Let $n = 5 \cdot 6 \cdot 7 = 210$, $N_1 = n / 5 = 42$, $N_2 = n / 6 = 35$ and $N_3 = n / 7 = 30$. By solving the equations $N_i x_i \equiv 1 \pmod{n_i}$, we get $x_1 = 3$, $x_2 = -1$, $x_3 = 4$. Hence, if we compute
    $$4 \cdot 42 \cdot 3 + i \cdot 35 (-1) + 3 \cdot 30 \cdot 4 = 864 - 35i \equiv 234 - 35i \pmod{210}$$
    and replace $i$ by 2 and 5, we get that the two incongruent solutions modulo 210 are 59 and 164. \\
\end{solution} 