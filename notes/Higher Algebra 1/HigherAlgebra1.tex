\documentclass{article}

%% Language and font encodings
\usepackage[english]{babel}
\usepackage[utf8x]{inputenc}
\usepackage[T1]{fontenc}
\usepackage{amsfonts}
\usepackage[margin=41mm]{geometry}

%% Sets page size and margins
%\usepackage[a4paper,top=3cm,bottom=2cm,left=3cm,right=3cm,marginparwidth=1.75cm]{geometry}

%% Useful packages
\usepackage[colorlinks=true, allcolors=black]{hyperref}
\usepackage{amsmath}
\usepackage{xcolor}
\usepackage{amssymb}
\usepackage{amsthm}
\usepackage{stmaryrd}
\usepackage{graphicx}
\usepackage{enumitem}
\usepackage{tikz, pgfplots}
\usetikzlibrary{positioning}

%Theorem
\theoremstyle{plain}
\newtheorem{theorem}{Theorem}[subsection]
\newtheorem{proposition}[theorem]{Proposition}
\newtheorem{lemma}[theorem]{Lemma}
\newtheorem*{corollary}{Corollary}
\theoremstyle{definition}
\newtheorem*{definition}{Definition}

%Usual Sets
\newcommand{\C}{\mathbb{C}}
\newcommand{\R}{\mathbb{R}}
\newcommand{\Q}{\mathbb{Q}}
\newcommand{\Z}{\mathbb{Z}}
\newcommand{\N}{\mathbb{N}}
\newcommand{\F}{\mathbb{F}}
\newcommand{\E}{\mathbb{E}}
\newcommand{\K}{\mathbb{K}}
\newcommand{\Zn}[1]{\mathbb{Z}/ #1 \mathbb{Z}}

%Special Sets
\newcommand{\Iint}[2]{\llbracket #1 , #2 \rrbracket}

%Math Operators
\let\Re\relax
\let\Im\relax
\DeclareMathOperator{\Im}{Im}
\DeclareMathOperator{\Re}{Re}
\DeclareMathOperator{\Null}{null}
\DeclareMathOperator{\range}{range}
\DeclareMathOperator{\card}{card}
\DeclareMathOperator{\Aut}{Aut}
\DeclareMathOperator{\Hom}{Hom}
\DeclareMathOperator{\Gal}{Gal}
\DeclareMathOperator{\Char}{char}

%Others
\newcommand{\td}{\textcolor{red}{\textbf{TODO}}}
\newcommand{\isomorphic}{\cong}

%Example environment
\newenvironment{example}{\noindent\textbf{Example:} \vspace{-0.2cm}\begin{itemize}}{\end{itemize}}

%Remark environment
\newenvironment{remark}{\noindent\textbf{Remark:}}{}

%Notation environment
\newenvironment{notation}{\noindent\textit{Notation:}}{}

%Terminology environment
\newenvironment{terminology}{\noindent\textit{Terminology:}}{}

%Set QED symbol to blacksquare
\renewcommand\qedsymbol{$\blacksquare$}

\title{MATH 570 Notes : Higher Algebra 1}
\author{Samy Lahlou}
\date{}

\begin{document}

\maketitle

These notes are based on lectures given by Professor Eyal Goren at McGill University in Fall 2025. The subject of these lectures is \td. As a disclaimer, it is more than possible that I made some mistakes. Feel free to correct me or ask me anything about the content of this document at the following address : samy.lahloukamal@mcgill.ca

\tableofcontents

\newpage

\section{Categories and Functors}

\subsection{Definitions}

\begin{definition}
    A category $\underline{C}$ is
    \begin{enumerate}
        \item a collection of objects $\text{ob}(\underline{C})$,
        \item for any $A,B \in \text{ob}(\underline{C})$ a set $\text{Mor}_{\underline{C}}(A,B)$ with an associative composition law,
        \item For all $A \in \text{ob}(\underline{C})$, there is a morphism $1_A$ such that for all $f \in \text{Mor}_{\text{ob}(\underline{C})}(A,B)$, we have $f \circ 1_A = f$ and $1_B \circ f = f$.
        \item 
    \end{enumerate}
\end{definition}

\begin{definition}
    A morphism $f \in \text{Mor}(A,B)$ is an isomorphism if there exists a $g \in \textit{Mor}(A,B)$ such that $g \circ f = 1_A$ and $f \circ g = 1_B$.
\end{definition}

\subsection{Initial and Final Objects}

\begin{definition}
    An object $A \in \text{ob}(\underline{C})$ is initial (final) if for any object $B \in \text{ob}(\underline{C})$, $\text{Mor}(A,B)$ has only one element ($\text{Mor}(B,A)$ has only one element). $A$ is a zero object if it's both initial and final.
\end{definition}

\begin{proposition}
    An initial object (if it exists) is unique up to a unique isomorphism (similar, final).
\end{proposition}

\begin{proof}
    Suppose $A, A' \in \text{Ob}(\underline{C})$ are initial, let $f \in \text{Mor}(A,A')$ and $g \in \text{Mor}(A', A)$ be the unique such morphisms, then $g \circ f \in \text{Mor}(A,A) = \{1_A\}$ and so $g \circ f = 1_A$. Similarly, $f \circ g = 1_{A'}$. It follows that $A$ and $A'$ are isomorphic.
\end{proof}

\begin{example}
    \item Sets: $\text{Ob}(\underline{C})$ are sets, $\text{Mor}(A,B)$ are functions from $A$ to $B$. The empty set is the unique initial object. The singletons are precisely the final objects. It follows that zero objects don't exist.
    \item ${}_{R}\text{Mor}$ ($\text{Mor}_R$): $R$ is a ring (always with $1$ and is associative), $\text{Ob}({}_{R}\text{Mor})$ are left $R$-modules $M$, functions are $R$-modules homomorphisms $f : M \to N$. The zero-module $\{0\}$ is the unique initial object and also the unique final object. Hence, it is a zero object in that category.
    \item $\text{Gps}$ ($\text{AbGps}$): The objects are (abelian) groups, the morphisms are (abelian) group homomorphisms, as for the previous example, there is a unique zero-object: the group $\{1\}$.
\end{example}

\subsection{Functors}

\begin{definition}[Covariant and Contravariant Functors]
    A covariant (contravariant) functor $F : \underline{C} \to \underline{D}$ is the following:
    \begin{enumerate}
        \item For any object $A \in \text{Ob}(\underline{C})$, $FA \in \text{Ob}(\underline{D})$.
        \item For any morphism $f \text{Mor}_{\underline{C}}(A,B)$, we have a morphism $Ff \in \text{Mor}_{\underline{D}}(FA,FB)$ ($Ff \in \text{Mor}_{\underline{D}}(FB,FA)$) such that $F1_A = 1_{FA}$ and $F(g \circ f) = Fg \circ Ff$ ($F(g \circ f) = Ff \circ Fg$).
    \end{enumerate}
\end{definition}

\begin{definition}[Faithful]
    We say that $F$ is faithful if whenever $Ff = Fg$ for some $f,g \in \text{Mor}(A,B)$, then $f = g$.
\end{definition}

\begin{definition}[Full]
    We say that $F$ is full if given any $h \in \text{Mor}(FA, FB)$, there exists a $f \in \text{Mor}(A,B)$ such that $Ff = h$.
\end{definition}

\begin{definition}[Essentially Surjective]
    We say that $F$ is full if any $C \in \text{Ob}(\underline{D})$ is isomorphic to $FA$ for some $A \in \text{Ob}(\underline{C})$.
\end{definition}

\begin{example}
    \item Forgetful functors: for example, the functor $F : \text{Gps} \to \text{Sets}$ defined by $FA = A$, $Ff = f$ forgets the group structure of the objects. This functor is not full but it is faithful. With some logic, we can prove that it is also essentially surjective.
    \item Consider the functor $F : \text{Rings} \to \text{Gps}$ such that $FR = R^*$ and $Ff = f|_{R^*}$. Is it faithful, full, essentially surjective ?
    \item If $k$ is a field, then ${}_k \text{Mod}$ is the same as the category of $k$-vector spaces where the morphisms are the $k$-linear maps. From this categroy, we can consider the contravariant functor $F : _k\text{VSp} \to _k\text{VSp}$ that sends $V$ to its dual and homo-morphisms to their transpose.
    \item The category $\text{Rep}(G)$, where $G$ is a fixed finite group, is the category of finite linear complex representations of $G$.
    \item We can define the functor $F : \text{FinGps} \to \text{Rings}$ by $FG = k[G]$ where $k$ is a field.
\end{example}

Next time: we'll see that the category of representations of $G$ is equivalent to the category ${}_{\C[G]}\text{Mod}$.

\subsection{Morphisms of Functors}

Let $F,G : \underline{C} \to \underline{D}$ be two functors of the same variance. A morphism of functors $\varphi : F \to G$ is a collection of marphisms in $\underline{D}$ such that 
$$commutative diagram$$
(inverse the arrows if $F$ and $G$ are contravariant). \\

\begin{example}
    \item Let $\underline{C} = \underline{D} = {}_k\text{VSp}$ and $F : \underline{C} \to \underline{C}$ be the duality functor, then \td 
\end{example}

\subsection{Equivalence of categories}

Two categories $\underline{C}$ and $\underline{D}$ are called (\textcolor{red}{anti}) equivalent if there are co(\textcolor{red}{ntra})variant functors $F : \underline{C} \to \underline{D}$ and $G : \underline{D} \to \underline{C}$ such that $F \circ G \isomorphic 1_{\underline{D}}$ and $G \circ F \isomorphic 1_{\underline{C}}$ (isomorphic means all the $\varphi_A$ are isomorphisms).\\

\begin{example}
    \item Let $G$ be a finite group, then the categories $\text{Rep}(G)$ and $\text{f.g.}_{\C[G]}$ are equivalent. \td 
    \item Let $k$ be a field and let $\underline{C}$ be the category composed of the vector spaces $k^0$, $k^1$, $k^2$, ..., then $\text{Mor}(k^a, k^b) = \text{M}_{ab}(k)$. Now, if we let $\underline{D} = \text{f.d.}_k\text{VSp}$ be the category of finite dimensional vector spaces, then this category is uncountable whereas the previous one is countable. Let $F : \underline{C} \to \underline{D}$ be the functor defined by $Fk^a = k^a$ and $FM$ is the linear map that maps $x$ to $Mx$, then $F$ is full, faithful and essentially surjective. To define $G : \underline{D} \to \underline{C}$, choose for any vector space $V$ an isomorphism $i_v : V \to k^{\dim(V)}$, this induces an isomorphism between $\Hom(V,W)$ and $\Hom(k^{\dim(V)}, k^{\dim(W)})$ by the map $T \mapsto i_w T i_V^{-1}$. Thus, we can define $G$ by $GV = k^{\dim(V)}$ and $GT = i_w T i_V^{-1}$. If we choose $i_{k^n}$ to be the identity, then we get $GF = 1$ and $FG \isomorphic 1$.
\end{example}

\begin{theorem}
    A functor $F : \underline{C} \to \underline{D}$ is an (\textcolor{red}{anti}) equivalence of categories if and only if $F$ is full, faithfull and essentially surjective.
\end{theorem}

\begin{proof}
    \td
\end{proof}

\begin{theorem}[Morita's Theorem]
    Let $R$ be a ring and $n \geq 1$ be an integer. The categories ${}_R\text{Mod}$ and ${}_{M_n(R)}\text{Mod}$ are equivalent.
\end{theorem}

\begin{proof}
    \td
\end{proof}

\end{document}