\chapter{Affine Algebraic Sets}

\section{Algebraic Preliminaries}

\begin{exercise}
    Let $R$ be a domain. (a) If $F,G$ are forms of degree $r,s$ respectively in $R[X_1, ..., X_n]$, show that $FG$ is a form of degree $r+s$. (b) Show that any factor of a form in $R[X_1, ..., X_n]$ is also a form.\\
\end{exercise}

\begin{solution}
    \begin{enumerate}
        \item The form $F$ is a sum of monomials of degree $r$ and $G$ is a sum of monomials of degree $s$. Hence, using the definition of the product of two polynomials, the polynomial $FG$ is a sum of products $ab$ where $a$ is a monomial of degree $r$ and $b$ is a monomial of degree $s$. Hence, $ab$ is a monomial of degree $r + s$ so $FG$ is a sum of monomials of degree $r + s$. Therefore, equivalently, $FG$ is a form of degree $r + s$.
        \item Let $f$ be a divisor of a form $F$ of degree $N$, then $F = fg$ for some polynomial $g$. Suppose that $f$ is not a form, then we can write $f = f_a + f_{a+1} + \dots + f_b$ where $f_i$ is a form of degree $i$, $a < b$, and $f_a, f_b \neq 0$. Similarly, we can write $g = g_c + g_{c+1} + \dots + g_d$ where $g_i$ is a form of degree $i$, $c \leq d$, and $g_c, g_d \neq 0$. It follows that
        $$F = fg = f_ag_c + [f_ag_{c+1} + f_{a+1}g_c] + \dots + f_bg_d$$
        is the decomposition of $F$ into a sum of forms. However, $f_ag_c \neq 0$ is a form of degree $a+c$ and $f_bg_d \neq 0$ is a form of degree $b + d > a+ c$. But this impossible because $F$ is already a form so it cannot be decomposed into nonzero forms of different degree.\\
    \end{enumerate}
\end{solution}

\begin{exercise}
    Let $R$ be a UFD, $K$ the quotient field of $R$. Show that every element $z$ of $K$ may be written $z = a/b$, where $a,b \in R$ have no common factors; this representative is unique up to units of $R$.\\
\end{exercise}

\begin{solution}
    Let $z \in K$ and write it as $a/b$ with $a,b \in R$. Since $R$ is a UFD, then we can define $d = \gcd(a,b)$, $a = a'd$ and $b = b'd$. Since $d$ is the greatest common divisor of $a$ and $b$, then $a'$ and $b'$ have no common divisor. Hence, $z = a/b = a'/b'$ where the numerator and the denominator have no common divisor. Next, we need to show that this representation of $z$ is unique up to multiplication by units. If $c/d$ is another representation of $z$ as a quotient of two elements in $R$ with $\gcd(c,d) = 1$, then $a'/b' = c/d$. From this equation, we get that $a'd = b'c$. Hence, $a'$ divides $b'c$, but $\gcd(a,b) = 1$ so $a'$ divides $c$. Similarly, $c$ divides $a'd$ but $\gcd(c,d) = 1$ so $c$ divides $a'$. Thus, $c' = ua'$ where $u$ is a unit. If we plug this into the equation $a'd = b'c$, we get that $a'd = b'ua'$, and hence, $d = ub'$. Therefore, $c$ and $d$ are equal to $a$ and $b$, up to multiplication by a unit.\\
\end{solution}

\begin{exercise}
    Let $R$ be a PID, let $P$ be a nonzero, proper, prime ideal in $R$. (a) Show that $P$ is generated by an irreducible element. (b) Show that $P$ is maximal.\\
\end{exercise}

\begin{solution}
    \begin{enumerate}
        \item Since $R$ is a PID, then $P = (p)$ where $p \in R$. Since $P$ is nonzero, and proper, then $p$ is not zero, nor a unit. Suppose that $p$ is reducible, then there exist two non-units and nonzero elements $a,b \in R$ such that $p = ab$. Since $P$ is prime, then $ab = p \in (p)$ implies that $a \in (p)$ or $b \in (p)$. If $a \in (p)$, $p$ divides $a$. Since $a$ divides $p$, then $ab = p = au$ where $u$ is a unit, so $b=u$, a contradiction. Therefore, $p$ is irreducible.
        \item Let $I = (f)$ be an ideal containing $P$, then $p \in (f)$ which implies that $f$ divides $p$. But the only divisors of $p$ are units and unit multiples of $p$. In the first case, $I = R$, in the second case, $I = P$. Therefore, $P$ is maximal. \\
    \end{enumerate}
\end{solution}

\begin{exercise}
    Let $k$ be an infinite field, $F \in k[X_1, ..., X_n]$. Suppose $F(a_1, \dots, a_n) = 0$ for all $a_1, \dots, a_n \in k$. Show that $F=0$. (\textit{Hint:} Write $F = \sum F_i X_n^i$, $F_i \in k[X_1, ..., X_{n-1}]$. Use induction on $n$, and the fact that $F(a_1, ..., a_{n-1}, X_n)$ has only a finite number of roots if any $F_i(a_1, ..., a_{n-1}) \neq 0$.) \\
\end{exercise}

\begin{solution}
    Let's prove this by induction on $n$. When $F$ is a polynomial in $k[X]$ such that $F(a) = 0$ for all $a \in k$, then $F = 0$ because $F$ has at most finitely many roots when it is nonzero. Next, suppose that $G \in k[X_1, ..., X_{n-1}]$ with $G(a_1, ..., a_{n-1}) = 0$ for all $(a_1, ..., a_{n-1}) \in k^{n-1}$ implies that $G=0$, and consider a polynomial $F \in k[X_1, ..., X_n]$ with $F(a_1, ..., a_n) = 0$ for all $(a_1, ..., a_n) \in k^n$. If we view $F$ as an elements of $k[X_1, ..., X_{n-1}][X_n]$, then we can write $F = \sum_i F_i X_n^i$ where $F_i \in k[X_1, ..., X_{n-1}]$. Fix $(a_1, ..., a_{n-1}) \in k^{n-1}$ and consider the polynomial $f(X) = F(a_1, ..., a_{n-1}, X) = \sum_i F_i(a_1, ..., a_{n-1})X^i$, then $f(a_n) = 0$ for all $a$. By the case $n = 1$, we have that $f = 0$, and hence, that $\sum_iF_i(a_1, ..., a_{n-1})X^i = 0$. By properties of polynomials, it follows that $F_i(a_1, ..., a_{n-1}) = 0$ for all $i$. Since it holds for all $(a_1, ..., a_{n-1}) \in k^{n-1}$, then by induction, $F_i = 0$ for all $i$. Hence, $F = \sum_i F_i X_n^i = 0$. Therefore, by induction, it holds for all $n$. \\
\end{solution}

\begin{exercise}
    Let $k$ be any field. Show that there are an infinite number of irreducible monic polynomials in $k[X]$. (\textit{Hint:} Suppose $F_1, ..., F_n$ were all of them, and factor $F_1\cdot \cdot \cdot F_n + 1$ into irreducible factors.)\\
\end{exercise}

\begin{solution}
    Suppose that there are finitely many irreducible monic polynomials $F_1, ..., F_n$. Define the polynomial $F = F_1\cdot \cdot \cdot F_n + 1$, then this polynomial must be divisible by $F_i$ for some $i$. However, $F_i$ certainly divides $F_1\cdot \cdot \cdot F_n$ so $F_i$ must divide $F - F_1\cdot \cdot \cdot F_n = 1$. But this is a contradiction since the only monic polynomial dividing 1 is 1, which is not irreducible since it is a unit. Therefore, by contradiction, there are infinitely many irreducible polynomials. \\
\end{solution}

\begin{exercise}
    Show that any algebraically closed field is infinite. (\textit{Hint:} The irreducible monic polynomials are $X-a$, $a \in k$.)\\
\end{exercise}

\begin{solution}
    Clearly, every polynomial of the form $X-a$ with $a \in k$ is irreducible. Moreover, since $k$ is algebraically closed, then every polynomial can be factored into a product of linear factors so the irreducible polynomials are precisely the polyomials of the form $X - a$ with $a \in k$. It is easy to see then that there are as many irreducible polynomials as elements in $k$. In the previous Problem, it is shown that there are always infinitely many irreducible polynomials. Therefore, $k$ is infinite. \\
\end{solution}

\begin{exercise}
    Let $k$ be a field, $F \in k[X_1, ..., X_n]$, $a_1, ..., a_n \in k$. (a) Show that
    $$F = \sum\lambda_{(i)}(X_1-a_1)^{i_1}\cdot \cdot \cdot (X_n-a_n)^{i_n}, \quad \lambda_{(i)} \in k.$$
    (b) If $F(a_1, ..., a_n) = 0$, show that $F = \sum_{i=1}^n (X_i - a_i)G_i$ for some (not unique) $G_i$ in $k[X_1, ..., X_n]$.\\
\end{exercise}

\begin{solution}
    \begin{enumerate}
        \item Notice that $F = F(X_1+a_1-a_1, ..., X_n + a_n - a_n)$. Hence, if we let $G = F(X_1+a_1, ..., X_n + a_n) \in k[X_1, ..., X_n]$, then we can write
        $$G = \sum\lambda_{(i)}X_1^{i_1}\cdot \cdot \cdot X_n^{i_n},$$
        then
        $$F = G(X_1 - a_1, ..., X_n - a_n) = \sum\lambda_{(i)}(X_1-a_1)^{i_1}\cdot \cdot \cdot (X_n-a_n)^{i_n}$$
        for some $\lambda_{(i)} \in k$.
        \item If $F(a_1, ..., a_n) = 0$, then $G(0, ..., 0) = 0$, so $G$ has no constant term. Let's prove that $G$ can be written as $\sum X_i H_i$ where $H_i \in k[X_1, ..., X_n]$. We can write $G$ as $X_1 H_1 + H_1'$ where none of the monomials in $H_1'$ contain a factor of $X_1$. Next, write $H_1'$ as $X_2H_2 + H_2'$ where none of the monomials in $H_2'$ contain a factor of $X_1$ or $X_2$. If we repeat this process, we get that $G = \sum X_i H_i$. It follows that
        $$F = \sum (X_i - a_i)H_i(X_1 - a_1, ..., X_n - a_n).$$
    \end{enumerate}
\end{solution}

\section{Affine Space and Algebraic Sets}

\begin{exercise}
    Show that the algebraic subsets of $\A^1(k)$ are just the finite subsets, together with $\A^1(k)$ itself. \\
\end{exercise}

\begin{solution}
    Let $S$ be an arbitrary set polynomials in $k[x]$. If $S$ is empty or $S = \{0\}$, then $V(S) = \A^1(k)$. If $S$ contains a nonzero polynomial $F$, then $V(S) \subset V(F)$. Since $F$ is a one-variable polynomial, then it has finitely many roots, and so $V(F)$ is finite. It follows that $V(S)$ is a finite subset of $\A^1(k)$. Conversely, we know that any finite subsets are algebraic sets. Therefore, the algebraic subsets of $\A^1(k)$ are precisely the finite subsets and $\A^1(k)$ itself. \\
\end{solution}

\begin{exercise}
    If $k$ is a finite field, show that every subset of $\A^1(k)$ is algebraic. \\
\end{exercise}

\begin{solution}
    Let $U$ be a subset of $\A^n(k)$, then $U$ us finite and so it must be algebraic by property n°5 of the section. \\
\end{solution}

\begin{exercise}
    Give an example of a countable collection of algebraic sets whose union is not algebraic. \\
\end{exercise}

\begin{solution}
    If we let $k = \R$, and $V_i = \{i\}$ for all $i \in \Z$, then all the $V_i$'s are algebraic subsets of $A^1(k)$ but their union is not since it would contradict Problem 1.8. \\
\end{solution}

\begin{exercise}
    Show that the following are algebraic sets:
    \begin{enumerate}[label=(\alph*)]
        \item $\{(t, t^2, t^3) \in \A^3(k) | t \in k\}$;
        \item $\{(\cos(t), \sin(t)) \in \A^2(\R) | t \in \R\}$;
        \item the set of points in $\A^2(\R)$ whose polar coordinates $(r, \theta)$ satisfy the equation $r = \sin(\theta)$.
    \end{enumerate}
\end{exercise}

\begin{solution}
    \begin{enumerate}[label=(\alph*)]
        \item Let $U = \{(t, t^2, t^3) \in \A^3(k) | t \in k\}$ and $V = V(x^3 - z, x^2 - y)$. Let $(x,y,z) \in U$, then there is a $t \in k$ such that $x=t$, $y=t^2$, and $z = t^3$. It follows that $x^3 - z = 0$ and $x^2 - y = 0$. Hence, $(x,y,z) \in V$. Conversely, if $(x,y,z) \in V$, and if we let $t = x$, then $y = x^2 = t^2$ and $z = x^3 = t^3$. Hence, $(x,y,z) \in U$. Therefore, $U = V$ and so $U$ is algebraic.
        \item Since the set $U = \{(\cos(t), \sin(t)) \in \A^2(\R) | t \in \R\}$ is simply the unit circle in $\R^2$, then we know that we can rewrite it as $\{(x,y) \in \A^2(\R) | x^2 + y^2 = 1\}$, or simply as $V(x^2 + y^2 - 1)$. Thus, it is an algebraic set.
        \item Using the fact that $r = \sqrt{x^2 + y^2}$ and $\sin(\theta) = y/\sqrt{x^2 + y^2}$, we get that the set $\{(x,y) \in \A^2(\R) | r = \sin(\theta)\}$ is equal to
        $$\{(x,y) \in \A^2(\R) | \sqrt{x^2 + y^2} = y/\sqrt{x^2 + y^2}\}.$$
        Equivalently, we can rewrite it as $V(x^2 + y^2 - y)$. Hence, it is an algebraic set. \\
    \end{enumerate}
\end{solution}

\begin{exercise}
    Suppose $C$ is an affine plane curve, and $L$ is a line in $\A^2(k)$, $L \not\subset C$. Suppose $C = V(F)$, $F \in k[X,Y]$ a polynomial of degree $n$. Show that $L \cap C$ is a finite set of no more than $n$ points. (\textit{Hint:} Suppose $L = V(Y - (aX+b))$, and consider $F(X,aX+b) \in k[X]$.) \\
\end{exercise}

\begin{solution}
    Let $F \in k[X,Y]$ be a polynomial of degree $n$ and suppose that $C = V(F)$. Since $L$ is a line, then we can write $L = V(G)$ with $G = aX + bY + c$ for some $a,b,c\in k$ where $a$ and $b$ cannot be both zero. Suppose without loss of generality that $b$ is non-zero, then $(X,Y) \in L$ if and only if $Y = -\frac{a}{b}X - \frac{c}{b}$. Consider now the polynomial $F(X, -\frac{a}{b}X - \frac{c}{b}) \in k[X]$, then this is a polynomial of degree $n$. Let $X$ be one of its roots, then $(X, -\frac{a}{b}X - \frac{c}{b}) \in L \cap C$. Conversely, if $(X,Y) \in L \cap C$, then $Y = -\frac{a}{b}X - \frac{c}{b}$ which implies that $F(X,Y) = 0$ and so $X$ is a root of $F(X, -\frac{a}{b}X - \frac{c}{b})$. Therefore, the roots of $X$ are in bijection with the elements in $L \cap C$. Since $F(X, -\frac{a}{b}X - \frac{c}{b})$ is a polynomial of degree $n$, then it has at most $n$ roots, and hence, $L \cap C$ is a finite set of no more than $n$ points. \\
\end{solution}

\begin{exercise}
    Show that each of the following sets is not algebraic:
    \begin{enumerate}[label=(\alph*)]
        \item $\{(x,y) \in \A^2(\R) \mid y = \sin(x)\}$.
        \item $\{(z,w) \in \A^2(\C) \mid |z|^2 + |w|^2 = 1\}$, where $|x + iy|^2 = x^2 + y^2$ for $x,y \in \R$.
        \item $\{(\cos(t), \sin(t), t) \in \A^3(\R) | t \in \R\}$;
    \end{enumerate}
\end{exercise}

\begin{solution}
    \begin{enumerate}[label=(\alph*)]
        \item Suppose that $\{y = \sin(x)\} = V(S)$ for some $S \subset k[X,Y]$. Since $\{y = \sin(x)\} \neq \A^2(\R)$, then $S$ must contain a nonzero polynomial $F \in k[X,Y]$, of degree $n$. Since $F \in S$, then $F(x, \sin(x)) =0$ for all $x \in \R$. Consider now the polynomial $F(0,X) \in k[X]$, then this polynomial must have at most $n$ roots. However, there are infinitely many $x \in \R$ such that $\sin(x) = 0$ since $F$ has infinitely many roots. Therefore, by contradiction, the set $\{y = \sin(x)\}$ is not algebraic.
        \item Suppose that $\{|z|^2 + |w|^2 = 1\} = V(S)$, since $\{|z|^2 + |w|^2 = 1\} \neq \A^2(\C)$, then there must be a nonzero polynomial $F \in k[X,Y]$ such that $F(z,w) = 0$ for all $z,w \in \C$ such that $|z|^2 + |w|^2 = 1$. Let $C = V(F)$, and let $L = V(X)$ be a line. We know that $L \not\subset C$ since $(0,2) \in L$ but $(0,2) \not\in C$. Hence, by Problem 1.12, $L\cap C$ is finite. However, for every complex number $w$ with $|w| = 1$, we have that $(0,w) \in L \cap C$. Since there are infinitely many such complex numbers $w$, then $L \cap C$ is infinite, a contradiction. Therefore, $\{|z|^2 + |w|^2 = 1\}$ is not an algebraic set.
        \item Suppose that $\{(\cos(t), \sin(t), t)\}_{t\in\R} = V(S)$ for some $S \subset k[X,Y, Z]$. Since $\{(\cos(t), \sin(t), t)\}_{t\in\R} \neq \A^3(\R)$, then $S$ must contain a nonzero polynomial $F \in k[X,Y,Z]$, of degree $n$. Since $F \in S$, then $F(\cos(t), \sin(t), t) =0$ for all $t \in \R$. Consider now the polynomial $F(1, 0, X) \in k[X]$, then this polynomial must have at most $n$ roots. However, there are infinitely many $t \in \R$ such that $\cos(x) = 1$, $\sin(x) = 0$, and hence, $F(1,0,x) = F(\cos(x), \sin(x), x) = 0$, so $F$ has infinitely many roots. Therefore, by contradiction, the set is not algebraic. \\
    \end{enumerate}
\end{solution}

\begin{exercise}
    Let $F$ be a nonconstant polynomial in $k[X_1, \dots, X_n]$, $k$ is algebraically closed. Show that $\A^n(k) \setminus V(F)$ is infinite if $n \geq 1$, and $\A^n(k) \setminus V(F)$ is infinite if $n \geq 2$. Conclude that the complement of any proper algebraic set is infinite. (\textit{Hint:} See Problem 1.4.)  \\
\end{exercise}

\begin{solution}
    First, notice that $k$ must be infinite since it is algebraically closed. Hence, the case $n = 1$ follows from Problem 1.8. For the case $n = 2$, suppose that there are finitely many $(x,y)$ such that $F(x,y) \neq 0$. Let $(x_0, y_0)$ be such a pair, then we get that there finitely many $y$ such that $F(x_0, y) = 0$. But since $F(x_0, y) \in k[y]$, then we get a contradiction with the case $n = 1$. Therefore, $\A^n(k) \setminus V(F)$ is infinite for the case $n = 2$.
    
    Now, let $n \geq 1$, $F \in k[X_1, ..., X_n]$, and suppose that $\A^n(k) \setminus V(F)$ is finite. Let $(a_1, ..., a_n) \in \A^n(k)$ be such that $F(a_1, ..., a_n) \neq 0$, then the polynomial $F(a_1, ..., a_{n-1}, X) \in k[X]$ is such that there are finitely many $x \in k$ such that it is nonzero. But this is in contradiction with the case $n = 1$. Thus, by contradiction, the set $\A^n(k) \setminus V(F)$ is infinite. \\
\end{solution}

\begin{exercise}
    Let $V \subset \A^n(k)$, $W \subset \A^m(k)$ be algebraic sets. Show that
    $$V \times W = \{(a_1, ..., a_n, b_1, ..., b_m) \mid (a_1, ..., a_n) \in V, (b_1, ..., b_m) \in W\}$$
    is an algebraic set in $\A^{n+m}(k)$. It is called the \textit{product} of $V$ and $W$.\\
\end{exercise}

\begin{solution}
    For all $F \in V$ and $G \in W$, define $F^{\times}, G^{\times} \in k[X_1, ..., X_{n+m}]$ by
    $$F^{\times}(X_1, ..., X_{n+m}) = F(X_1, ..., X_n) \ \text{ and } \ G^{\times}(X_1, ..., X_{n+m}) = G(X_{n+1}, ..., X_{n+m}).$$
    If we consider the set $V(\{F^{\times}\}_{F \in V})$, we get that it is equal to the elements in $\A^{n+m}(k)$ where the first $n$ entries are in $V$. Similarly, $V(\{G^{\times}\}_{G \in W})$ is equal to the elements in $\A^{n+m}(k)$ where the last $m$ entries are in $V$. It follows that $V(\{F^{\times}\}_{F \in V} \cup \{G^{\times}\}_{G \in W}) = V \times W$. Therefore, $V \times W$ is an algebraic set. 
\end{solution}

\section{The Ideal of a Set of Points}

\begin{exercise}
    Let $V$, $W$ be algebraic sets in $\A^n(k)$. Show that $V = W$ if and only if $I(V) = I(W)$. \\
\end{exercise}

\begin{solution}
    If $V = W$, then $I(V) = I(W)$ trivially. Now, if $I(V) = I(W)$, then $V(I(V)) = V(I(W))$. Since $V$ and $W$ are algebraic sets, then $V(I(V)) = V$ and $V(I(W)) = W$. Hence, $V = W$. \\
\end{solution}

\begin{exercise}
    (a) Let $V$ be an algebraic set in $\A^n(k)$, $P \in \A^n(k)$ a point not in $V$. Show that there is a polynomial $F \in k[X_1, ..., X_n]$ such that $F(Q) = 0$ for all $Q \in V$, but $F(P) = 1$. (\textit{Hint:} $I(V) \neq I(V\cup \{P\})$.) (b) Let $P_1, ..., P_r$ be distinct points in $\A^n(k)$, not in an algebraic set $V$. Show that there are polynomials $F_1, ..., F_r \in I(V)$ such that $F_i(P_j) = 0$ if $i \neq j$, and $F_i(P_i) = 1$. (\textit{Hint:} Apply (a) to the union of $V$ and all but one point.) (c) With $P_1, ..., P_r$ and $V$ as in (b), and $a_{ij} \in k$ for $1 \leq i,j \leq r$, show that there are $G_i \in I(V)$ with $G_i(P_j) = a_{ij}$ for all $i$ and $j$. (\textit{Hint:} Consider $\sum_ja_{ij}F_j$.) \\
\end{exercise}

\begin{solution}
    \begin{enumerate}[label=(\alph*)]
        \item Since $P \notin V$, then $V \neq V\cup \{P\}$ and so $I(V) \neq I(V\cup \{P\})$ by Problem 1.16. Since $V \subset V \cup \{P\}$, then $I(V \cup \{P\}) \subset I(V)$ and so it must be that $I(V) \not\subset I(V \cup \{P\})$. This implies that there is a $F_0 \in I(V) \subset k[X_1, ..., X_n]$ such that $F_0 \notin I(V \cup \{P\})$. In other words, there is polynomial $F_0 \in k[X_1, ..., X_n]$ such that $F_0(Q) = 0$ for all $Q\in V$ and $F_0(P) \neq 0$. Define $F(x) = F_0(P)^{-1}F_0(x)$ for all $x \in \A^n(k)$, then $F(Q) = 0$ for all $Q \in V$ and $F(P) = 1$.
        \item Let $i\in \Iint{1}{r}$, since $V \cup \{P_j\}_{j \neq i}$ is an algebraic set that doesn't contain $P_i$, then by part (a) there is a function $F_i \in I(V)$ such that $F_i(P_j) = 0$ whenever $j \neq i$ and $F_i(P_i) = 1$. 
        \item Since the set up is the same as in (b), then we already showed that, for all $i \in \Iint{1}{r}$, there exists a polynomial $F_i \in I(V)$ such that $F_i(P_j) = 0$ for all $j \Iint{1}{r} \setminus\{i\}$ and $F_i(P_i) = 1$. For all $i \in \Iint{1}{r}$, define $G_i = \sum_{k=1}^{r}a_{ik}F_k$. Since the $F_i$'s are all in the ideal $I(V)$, then $G_i \in I(V)$. Moreover, for all $j \in \Iint{1}{r}$, we have that $G_i(P_j) = \sum_{k=1}^{r}a_{ik}F_k(P_j) = a_{ij}$. \\
    \end{enumerate}
\end{solution}

\begin{exercise}
    Let $I$ be an ideal in a ring $R$. If $a^n \in I$, $b^m \in I$, show that $(a+b)^{n+m} \in I$. Show that $\rad(I)$ is an ideal. Show that any prime ideal is radical. \\
\end{exercise}

\begin{solution}
    By the Binomial Formula, we have that
    $$(a+b)^{n+m} = \sum_{k=0}^{n+m}\binom{n+m}{k}a^kb^{n+m-k}.$$
    When $k \in \Iint{0}{n-1}$, we have that the term of index $k$ is in $I$ because it can be written as $\binom{n+m}{k}a^kb^{n-k}\cdot b^m$ and $b^m$ is in $I$. Similarly, if $k \in \Iint{n}{n+m}$, the term of index $k$ can be written as $\binom{n+m}{k}a^{k-n}b^{n+m-k}\cdot a^n$, and $a^n \in I$, so it is also in $I$. All the terms in the sum above are in $I$ so $(a+b)^{n+m}$ must be in $I$.
    
    Since $\rad(I) \supset I$ and $0 \in I$, then $0 \in \rad(I)$. If $a,b \in \rad(I)$, then there exist positive integers $n$ and $m$ such that $a^n, b^m \in I$. By the previous part, $(a+b)^{n+m} \in I$ which implies that $a+b \in \rad(I)$. Thus, $\rad(I)$ is closed under addition. Finally, if $a \in \rad(I)$ and $r \in R$, then there is a positive integer $n$ such that $a^n \in I$. Since $(ra)^n = r^na^n \in I$, then $ra \in \rad(I)$. Therefore, $\rad(I)$ is an ideal.
    
    Let $I$ be a prime ideal, to prove that $I$ is radical, it suffices to prove that $\rad(I) \subset I$. Let $i \in \rad(I)$, then $i^n \in I$ where $n \geq 1$ is the smallest positive integer for which $i^n \in I$. If $n = 1$, we are done. Suppose that $n > 1$, then we have that $i\cdot i^{n-1} \in I$. Since $I$ is prime, then either $i$ or $i^{n-1}$ is in $I$. Since $n$ is minimal, then both cases are impossible. Therefore, $\rad(I) = I$ and so $I$ is radical.\\
\end{solution}

\begin{exercise}
    Show that $I = (X^2 + 1) \subset \R[X]$ is a radical (even a prime) ideal, but $I$ is not the ideal of any set in $\A^1(\R)$. \\
\end{exercise}

\begin{solution}
    Suppose that $p^n \in (X^2 + 1)$, then $X^1 + 2 \mid p^n$. Since $X^2 + 1$ is irreducible in $\R[X]$, then $X^2 + 1 \mid p$ which implies that $p \in (X^2 + 1)$. Therefore, $(X^2 + 1)$ is radical. Suppose that $(X^2 + 1) = I(X)$ for some $X \subset \A^1(\R)$. If $X = \varnothing$, then $(X^2 + 1) = I(\varnothing) = \R[X]$ which is false since $X \notin (X^2 + 1)$. Hence, there is a $x \in \A^1(\R)$ such that $x \in X$. Since $X^2 + 1 \in I(X)$, then $x^2 + 1 = (X^2 + 1)(x) = 0$ which is impossible since $X^2 + 1$ has no roots in $\R$. Therefore, $(X^2 + 1)$ is not the ideal of any set in $\A^1(\R)$. \\
\end{solution}

\begin{exercise}
    Show that for any ideal $I$ in $k[X_1, ..., X_n]$, $V(I) = V(\rad(I))$, and $\rad(I) \subset I(V(I))$. \\
\end{exercise}

\begin{solution}
    Since $I \subset \rad(I)$, then $V(\rad(I)) \subset V(I)$. Conversely, let $P \in V(I)$ and $F \in \rad(I)$, then $F^n \in I$ for some $n \geq 1$. It follows that $F^n(P) = 0$ and so $F(P)$. Thus, $P \in V(\rad(I))$. Therefore, $V(I) = V(\rad(I))$.
    
    Now, from the fact that $S \subset I(V(S))$ for all subsets $S$ of $k[X_1, ..., X_n]$, if we take $S = \rad(I)$, then $\rad(I) \subset I(V(\rad(I))) = I(V(I))$ by the first part of the exercise. \\
\end{solution}

\begin{exercise}
    Show that $I = (X_1 - a_1, ..., X_n - a_n) \subset k[X_1, ..., X_n]$ is a maximal ideal, and that the natural homomorphism from $k$ to $k[X_1, ..., X_n]/I$ is an isomorphism. \\
\end{exercise}

\begin{solution}
    Let $J$ be an ideal properly containing $I$, and let $F \in J \setminus I$. Write $F$ as
    $$F(X_1, ..., X_n) = F_k(X_1, ..., X_{n-1})X_n^k + \dots + F_0(X_1, ..., X_{n-1}).$$
    Since $F \in J$ and $F_k(X_1, ..., X_{n-1})(X_n - a_n)^k \in I \subset J$, then $F - F_k(X_1, ..., X_{n-1})(X_n - a_n)^k \in J$ where the resulting polynomial is now a polynomial in $k[X_1, ..., X_{n-1}][X_n]$ of degree $k-1$. By repeating this process, we can obtain a polynomial $G \in J$ which is a polynomial in $k[X_1, ..., X_{n-1}]$. Again, if we repeat the same procedure, we get a constant polynomial $H \in J$. This constant polynomial is nonzero because otherwise, we would get that $F$ is the sum of all the polynomials in $I$ that we subtracted and so $F$ would be in $I$. Thus, $H \in J$ is a nonzero constant so $1 \in J$. It follows that $J = k[X_1, ..., X_n]$. Therefore, $I$ is maximal.
    
    Consider the natural homomorphism $\varphi : k \to k[X_1, ..., X_n]/I$ that maps each scalar to equivalence class of its respective constant polynomial. This is a well defined homomorphism because it is equal to the natural homomorphism from $k$ to $k[X_1, ..., X_n]$ composed with the projection homomorphism from $k[X_1, ..., X_n]$ to $k[X_1, ..., X_n]/I$. This homomorphism is injective because it is a field homomorphism. This homomorphism is injective because in the first part of this exercice, we showed that every polynomial in $k[X_1, ..., X_n]$ is equivalent to a constant polynomial. There-fore, $\varphi$ is an isomorphism. 
\end{solution}

\section{The Hilbert Basis Theorem}

\begin{exercise}
    Let $I$ be an ideal in a ring $R$, $\pi : R \to R/I$ the natural homomorphism. (a) Show that for every ideal $J'$ of $R/I$, $\pi^{-1}(J') = J$ is an ideal of $R$ containg $I$, and for every ideal $J$ of $R$ containg $I$, $\pi(J) = J'$ is an ideal of $R/I$. This sets up a natural one-to-one correspondence between $\{$ideals of $R/I \}$ and $\{$ideals of $R$ that contain $I \}$. (b) Show that $J'$ is a radical ideal if and only if $J$ is radical. Similarly for prime and maximal ideals. (c) Show that $J'$ is finitely generated if $J$ is. Conclude that $R/I$ is Noetherian if $R$ is Noetherian. Any ring of the form $k[X_1, ..., X_n]/I$ is Noetherian. \\
\end{exercise}

\begin{solution}
    \begin{enumerate}[label=(\alph*)]
        \item Let $J'$ be an ideal of $R/I$ and consider the set $J = \pi^{-1}(J')$. Since $\pi(0) = I = 0 \in J'$, then $0 \in \pi^{-1}(J') = J$. Let $a,b \in J$, then $\pi(a), \pi(b) \in J'$ which implies that $\pi(a+b) = \pi(a) + \pi(b) \in J'$. It follows that $a+b \in J$. Similarly, if $a \in J$ and $r \in R$, then $\pi(a) \in J'$ which implies that $\pi(ra) = \pi(r)\pi(a) \in J'$. Hence, $ra \in J$. Thus, $J$ is an ideal of $R$. Finally, for all $i \in I$, we have that $\pi(i) = 0 \in J'$ so $i \in J$. Therefore, $J$ is an ideal of $R$ containing $I$.
        
        Next, let $J$ be an ideal of $R$ that contains $I$, and define $J' = \pi(J)$. Since $0 \in J$ and $\pi(0) = 0$, then $0 \in J'$. Let $\pi(a), \pi(b) \in J'$, then $\pi(a) + \pi(b) = \pi(a+b) \in J'$ since $a + b \in J$. If $\pi(r) \in R/I$ and $\pi(a) \in J'$, then $\pi(r)\pi(a) = \pi(ra) \in J'$ since $ra \in J$. Therefore, $J'$ is an ideal of $R/I$.
        \item Suppose that $J'$ is radical and let $j^n \in J$, then $\pi(j)^n = \pi(j^n) \in J'$. Since $J'$ is radical, then $\pi(j) \in J' = \pi(J)$ which implies that $j \in \pi^{-1}(\pi(J))$. Therefore, $J$ is radical. Conversely, suppose that $J$ is radical and let $\pi(j)^n \in J'$, then $\pi(j^n) \in J' = \pi(J)$. It follows that $j^n \in J$ which implies that $j \in J$. Thus, $\pi(j) \in J'$. Therefore, $J'$ is radical.
        
        Suppose that $J'$ is prime and let $ab \in J$, then $\pi(a)\pi(b) = \pi(ab) \in J'$. By the primality of $J'$, either $\pi(a)$ or $\pi(b)$ is in $J'$. Without loss of generality, if $\pi(a) \in J'$ then $a \in J$. It follows that $J$ is prime. Conversely, if $J$ is prime and $\pi(ab) = \pi(a)\pi(b) \in J'$, then $ab \in J$ which implies that $a \in J$ (wlog). It follows that $\pi(a) \in J'$. Therefore, $J$ is prime.
        
        Suppose that $J'$ is maximal and let $M$ be an ideal properly containing $J$, then its image $M' = \pi(M)$ is an ideal that contains $J'$. Let $m \in M \setminus J$, then $\pi(m) \in M'\setminus J'$ because if $\pi(m) \in J'$, then $m \in J$. Hence, $M'$ properly contains $J'$. Since $J'$ is maximal, then $M' = R/I$. By the bijection between the ideal classes, we must have $M = R$. Therefore, $J$ is a maximal ideal. Conversely, suppose that $J$ is maximal and let $M'$ be an ideal that properly contains $J'$. Let $M = \pi^{-1}(M')$, then $M$ is an ideal that contains $J$. Since $M'$ contains $J'$ properly, there must be an $m \in M$ such that $\pi(m) \in M' \setminus J'$. From this, it must be that $m \in M\setminus J$ and so $M$ contains $J$ properly. Since $J$ is maximal, then $M = R$ and so $M' = R/I$. Therefore, $J'$ is maximal.
        \item Suppose that $J$ is finitely generated, then we can write $J = (j_1, ..., j_n)$. Let $\pi(j) \in J'$, then
        $$\pi(j) = \pi\left(\sum_kr_kj_k\right) = \sum_k \pi(r_k)\pi(j_k) \in (\pi(r_1), \dots, \pi(r_n)).$$
        It follows that $J' \subset (\pi(r_1), \dots, \pi(r_n))$. On the other hand, since the $\pi(j_k)$'s are in $J'$, then $(\pi(r_1), \dots, \pi(r_n)) \subset J'$. Therefore, $J'$ is finitely generated.

        Now, suppose that $R$ is Noetherian and let $J'$ be an ideal of $R/I$, then $J' = \pi(J)$ where $J$ is an ideal of $R$ containing $I$. Since $R$ is Noetherian, then $J$ is finitely generated and so $J'$ is also finitely generated. Therefore, $R/I$ is Noetherian.
    \end{enumerate}
\end{solution}

\section{Irreducible Components of an Algebraic Set}

\begin{exercise}
    Give an example of a collection $\mathscr{S}$ of ideals in a Noetherian ring such that no maximal member of $\mathscr{S}$ is a maximal ideal. \\
\end{exercise}

\begin{solution}
    Take the ring to be $\Z$ and take the collection $\mathscr{S} = \{(4)\}$ containing only one element. It is clear that $(4)$ is a maximal element of $\mathscr{S}$ even though it is not a maximal ideal of $\Z$. \\
\end{solution}

\begin{exercise}
    Show that every proper ideal in a Noetherian ring is contained in a maximal ideal. (\textit{Hint:} If $I$ is the ideal, apply the lemma to $\{$proper ideals that contain $I\}$.) \\
\end{exercise}

\begin{solution}
    Let $I$ be a proper ideal and let $\mathscr{S}$ be the collection of proper ideals that contain $I$. By the lemma, $\mathscr{S}$ must contain a maximal element $J$ that contain $I$. If $J$ is not a maximal ideal, then it must be itself contained in a proper ideal of the ring. However, this new proper ideal would be in $\mathscr{S}$ and would contradict the maximality of $J$ in $\mathscr{S}$. Therefore, $J$ is a maximal ideal that contain $I$. \\
\end{solution}

\begin{exercise}
    (a) Show that $V(Y - X^2) \subset \A^2(\C)$ is irreducible; in fact, $I(V(Y - X^2))= (Y - X^2)$. (b) Decompose $V(Y^4 - X^2, Y^4 - X^2Y^2 + XY^2 - X^3) \subset \A^2(\C)$ into irreducible components. \\
\end{exercise}

\begin{solution}
    \begin{enumerate}[label=(\alph*)]
        \item Let's show that $I(V(Y - X^2))$ is a prime ideal. First, notice that $I(V(Y - X^2)) = (Y - X^2)$. Since $Y - X^2$ is irreducible in $\C[X,Y]$, then $(Y - X^2)$ is prime. Therefore, $V(Y - X^2)$ is irreducible.
        \item First, notice that $Y^4 - X^2 = (Y^2 - X)(Y^2 + X)$ and $Y^4 - X^2Y^2 + XY^2 - X^3 = (Y^2 - X^2)(Y^2 + X) = (Y - X)(Y+X)(Y^2 + X)$. Let's prove that our algebraic set $V$ is equal to $V(Y^2 + X) \cup \{(1,1)\}\cup \{(1,-1)\}$. First, if $(x, y) \in V$, then $y^4 = x^2$ and $(y - x)(y + x)(y^2 + x) = 0$. From the second equation, we have the following cases: if $y = x$, then either $x = y = 0$ or $y^4 = x^2$ implies that $y^2 = 1$, and hence, that $x = y = \pm 1$; if $y = -x$, then either $x = y = $ or the same considerations as before gives us that $y = \pm 1$, and hence, $x = \mp 1$, $y = \pm 1$; finally, if $y^2 = -x$, then $y^2 + x = 0$. The outcome of all of these cases show that $(x,y) \in V(Y^2 + X) \cup \{(1,1)\}\cup \{(1,-1)\}$. The converse is even easier to prove, if $y^2 + x = 0$, then $(x,y)$ satisfy both polynomials describing $V$ and so it is in $V$; if $x = 1$ and $y = \pm 1$, then it clearly satisfies both polynomials again and so it is in $V$. Therefore, $V = V(Y^2 + X) \cup \{(1,1)\}\cup \{(1,-1)\}$. The three sets are clearly irreducible algebraic sets so we are done. \\
    \end{enumerate}
\end{solution}

\begin{exercise}
    Show that $F = Y^2 + X^2(X-1)^2 \in \R[X,Y]$ is an irreducible polynomial, but $V(F)$ is reducible. \\
\end{exercise}

\begin{solution}
    By contradiction, suppose that there exist two polynomials $p, q \in \R[X,Y]$ such that $y^2 + x^2(x-1)^2 = p(x,y)q(x,y)$, then in particular, if we fix any $x \in \R \setminus \{0,1\}$, we can view this equation as an equation where $y$ is the only variable. But this would mean that the polynomial $y^2 + x^2(x-1)^2$ is reducible in $\R[y]$ which is false. Therefore, the polynomial $F$ is irreducible.
    
    To show that $V(F)$ is reducible, notice that the only solutions to the equation $Y^2 + X^2(X-1)^2 = 0$ are $(0,0)$ and $(1,0)$. Therefore, $V(F) = \{(0,0)\} \cup \{(1,0)\}$. \\
\end{solution}

\begin{exercise}
    Let $V$, $W$ be algebraic sets in $\A^n(k)$, with $V \subset W$. Show that each irreducible component of $V$ is contained in some irreducible component of $W$. \\
\end{exercise}

\begin{solution}
    First, write $V = \bigcup_i V_i$ and $W = \bigcup_j W_j$ where the unions are finite. Since $V \subset W$, then for each $i$, we have $V_i = \bigcup_j(W_j \cap V_i)$. Since $V_i$ is irreducible, then we must have that $V_i = W_j \cap V_i \subset W_j$ for some $j$. Therefore, every irreducible component of $V$ is contained in an irreducible component of $W$. \\
\end{solution}

\begin{exercise}
    If $V = V_1 \cup \dots \cup V_r$ is the decomposition of an algebraic set into irreducible components, show that $V_i \not\subset \bigcup_{i \neq j}V_j$. \\
\end{exercise}

\begin{solution}
    First, recall that $V_i \not\subset V_j$ for all $i \neq j$. Next, suppose that $V_i \subset \bigcup_{i \neq j}V_j$, then $V_i = \bigcup_{i \neq j}(V_j \cap V_i)$. But since $V_i$ is irreducible, then $V_i = V_j \cap V_i \subset V_j$ for some $i \neq j$. This contradiction proves that the original claim holds. \\
\end{solution}

\begin{exercise}
    Show that $\A^n(k)$ is irreducible if $k$ is infinite. \\
\end{exercise}

\begin{solution}
    Simply notice that when $k$ is infinite, we have $I(\A^n(k)) = \{0\}$ and $\{0\}$ is a prime ideal. Therefore, $\A^n(k)$ is irreducible. \\
\end{solution}

\section{Algebraic Subsets of the Plane}

\begin{exercise}
    Let $k = \R$. (a) Show that $I(V(X^2 + Y^2 + 1)) = (1)$. (b) Show that every algebraic subset of $\A^2(\R)$ is equal to $V(F)$ for some $F \in \R[X,Y]$. \\
\end{exercise}

\begin{solution}
    \begin{enumerate}[label=(\alph*)]
        \item Since $X^2 + Y^2 + 1$ has no roots in $\A^2(\R)$, then
        $$I(V(X^2 + Y^2 + 1)) = I(\varnothing) = \A^2(\R) = (1).$$
        \item Let $V$ be an algebraic subset of $\A^2(\R)$. If $V = \A^2(\R)$, then $V = V(0)$; if $V = \varnothing$, then $V = V(1)$; if $V = \{(a,b)\}$, then we can write $V$ as $V((X-a)^2 + (Y-b)^2)$ because $(X-a)^2 + (Y-b)^2 = 0$ if and only if $X = a$ and $Y = b$; if $V$ is any other algebraic set, then we already know that $V = V(F)$ for some $F \in \R[X,Y]$ using Corollary 2.\\
    \end{enumerate}
\end{solution}

\begin{exercise}
    (a) Find the irreducible components of $V(Y^2 - XY - X^2Y + X^3)$ in $\A^2(\R)$, and also in $\A^2(\C)$. (b) Do the same for $V(Y^2 - X(X-1)^2)$, and for $V(X^3 + X - X^2Y - Y)$.\\
\end{exercise}

\begin{solution}
    \begin{enumerate}[label=(\alph*)]
        \item Notice that $Y^2 - XY - X^2Y + X^3 = (Y-X^2)(Y-X)$, then $V(Y^2 - XY - X^2Y + X^3) = V(Y-X^2) \cup V(Y-X)$. This decomposition is the irreducible decomposition in $\C[X,Y]$ because both $Y-X^2$ and $Y-X$ are irreducible, and using Corollary 3. This is also the irreducible decomposition in $\A^2(\C)$ because both $Y-X^2$ and $Y-X$ are irreducible in $\R[X,Y]$ and both have an infinite associated algebraic set.
        \item The first polynomial is irreducible and has infinitely many solutions in $\A^2(\R)$, hence, its associated algebraic set is irreducible. The second polynomial has the same factorization into irreducible polynomials in $\A^2(\R)$ and in $\A^2(\C)$ so again, the factorization is the same in the two cases. \\
    \end{enumerate}
\end{solution}

\section{Hilbert's Nullstellensatz}

\begin{exercise}
    Show that both theorems and all of the corollaries are false if $k$ is not algebraically closed. \\
\end{exercise}

\begin{solution}
    For the Weak-Nullstellensatz, take $k = \R$, $n = 1$, $I = (X^2 + 1)$, then $V(I) = \varnothing$ since $X^2 + 1$ has no roots in $\R$. The same set up disproves the Hilbert's Nullstellensatz, as well as Corollary 1 and 2. For the third corollary, it suffices to look at $V(Y^2 + X^2(X-1)^2)$ in $\R$ and see that even though the polynomial is irreducible, the algebraic set is reducible into $\{(0,0)\}$ and $\{(1,0)\}$. Similarly, if we take $I = (X^2 + Y^2) \subset \R[X,Y]$, then $V(I) = \{(0,0)\}$ is finite but $\R[X,Y]/(X^2 + Y^2)$ is infinite dimensional. This last example shows that corollary 4 doesn't hold if $k$ is not algebraically closed. \\
\end{solution}

\begin{exercise}
    (a) Decompose $V(X^2 + Y^2 - 1, X^2 - Z^2 - 1) \subset \A^3(\C)$ into irreducible components. (b) Let $V = \{(t,t^2, t^3) \in \A^3(\C) | t \in \C\}$. Find $I(V)$, and show that $V$ is irreducible.  \\
\end{exercise}

\begin{solution}
    \begin{enumerate}[label=(\alph*)]
        \item First, notice that $(x^2 + y^2 - 1, x^2 - z^2 -1) = (x^2 + y^2 - 1, (y+iz)(y-iz))$. It follows that
        $$V(x^2 + y^2 - 1, x^2 - z^2 -1) = V(x^2 + y^2 - 1, y+iz) \cup V(x^2 + y^2 - 1, y-iz).$$
        Since
        \begin{align*}
            \C[x,y,z]/(x^2 + y^2 - 1, y+iz) &\cong (\C[x,y,z]/(y+iz))/(x^2+y^2-1) \\
            &\cong \C[x,y]/(x^2 + y^2 - 1)
        \end{align*}
        is an integral domain, then the ideal $(x^2 + y^2 - 1, y+iz)$ is prime, and hence, $V(x^2 + y^2 - 1, y+iz)$ is irreducible. Similarly, the algebraic set $V(x^2 + y^2 - 1, y-iz)$ is irreducible as well. Therefore, we have found the decomposition of the algebraic set $V(x^2 + y^2 - 1, x^2 - z^2 - 1)$ into irreducible algebraic sets.
        \item First, notice that $V = V(Y - X^2, Z - X^3)$. Hence, if we let $I = (Y - X^2, Z - X^3)$ and notice that $\C[X,Y,Z]/I \cong \C[X]$ is an integral domain, then we get that $I$ is a prime ideal, and hence, $I(V) = I(V(I)) = I = (Y - X^2, Z - X^3)$ by the Nullstellensatz. \\
    \end{enumerate}
\end{solution}

\begin{exercise}
    Let $R$ be a UFD. (a) Show that a monic polynomial of degree two or three in $R[X]$ is irreducible if and only if it has no roots in $R$. (b) The polynomial $X^2 - a \in R[X]$ is irreducible if and only if $a$ is not a square in $R$. \\
\end{exercise}

\begin{solution}
    \begin{enumerate}[label=(\alph*)]
        \item First, notice that if a polynomial of degree 2 or 3 is reducible, then it must be divisible by a polynomial of degree 1. Moreover, in the decomposition of the polynomial into two smaller (in the sense of the degree) polynomials, the leading coefficients of each smaller polynomial must be inverses of each other since the original polynomial is monic. In that case, we can multiply the two polynomials by constants such that the degree one polynomial dividing the original polynomial is monic as well. It follows that a reducible, monic, degree 2 or 3 polynomial must have a root since it is divisible by a polynomial which has a root. The converse is easy.
        \item This follows from part (a) using the fact that $X^2 - a$ has a root if and only if $a$ has a root in $R$. \\
    \end{enumerate}
\end{solution}

\begin{exercise}
    Show that $V(Y^2 - X(X-1)(X-\lambda)) \subset \A^2(k)$ is an irreducible curve for any algebraically closed field $k$, and any $\lambda \in k$. \\
\end{exercise}

\begin{solution}
    Let's show that the polynomial $y^2 - x(x-1)(x-\lambda)$ is irreducible. By contradiction, let $p(x,y)$ and $q(x,y)$ be two nontrivial polynomials such that $p(x,y)q(x,y) = y^2 - x(x-1)(x-\lambda)$, then the highest degree of $y$ in $p_1$ and $p_2$ must be either 0 and 2, or 1 in both. However, the first case would imply that $y^2 - x(x-1)(x-\lambda)$ is factorizable by a polynomial in $x$, which is not the case. Thus, we must have $p(x,y) = p_1(x)y + p_0(x)$ and $q(x,y) = q_1(x)y + q_0(x)$ for some one variable polynomials $p_1,p_0,q_1,q_0$. In that case,
    \begin{align*}
        y^2 - &x(x-1)(x-\lambda) \\
        &= p_1(x)q_1(x)y^2 + (p_1(x)q_0(x) + p_0(x)q_1(x))y + p_0(x)q_0(x)
    \end{align*}
    which implies that $p_1(x)q_1(x) = 1$, $p_1(x)q_0(x) + p_0(x)q_1(x) = 0$ and $p_0(x)q_0(x) = x(x-1)(x-\lambda)$. Since $p_1(x)q_1(x) = 1$, then $p_1(x) = q_1(x) = 1$ without loss of generality. It follows that $q_0(x) = -p_0(x)$, and hence, $-p_0(x)^2 = x(x-1)(x-\lambda)$. But $x(x-1)(x-\lambda)$ is a polynomial of degree 3 while $-p_0(x)^2$ must have an even degree. Therefore, by contradiction, $y^2 - x(x-1)(x-\lambda)$ is irreducible, implying that $V(Y^2 - X(X-1)(X-\lambda))$ is irreducible. \\
\end{solution}

\begin{exercise}
    Let $I = (Y^2 - X^2, Y^2 + X^2) \subset \C[X,Y]$. Find $V(I)$ and $\dim_{\C}(\C[X,Y]/I)$. \\
\end{exercise}

\begin{solution}
    Notice that $I = (Y^2 - X^2, Y^2 + X^2) = (X^2, Y^2)$. Hence, $V(I) = V(X^2, Y^2) = \{(0,0)\}$. Since
    $$\C[X,Y]/(X^2,Y^2) \cong \{c_0 + c_1X + c_1Y + c_2XY : c_i \in \C\},$$
    then $\dim_{\C}(\C[X,Y]/I) = 4$.\\
\end{solution}

\begin{exercise}
    Let $k$ be any field, $F \in K[X]$ a polynomial of degree $n > 0$. Show that the residues of $\overline{1}, \overline{X}, ..., \overline{X}^{n-1}$ form a basis of $K[X]/(F)$ over $K$. \\
\end{exercise}

\begin{solution}
    First, it is clear that the elements $\overline{1}, \overline{X}, ..., \overline{X}^{n-1}$ span the elements in $K[X]/(F)$ since every polynomial in $K[X]$ is equivalent to a polynomial of degree at most $n-1$ modulo $F$. Next, let's show that $n$ elements are linearly independent: let $c_0, c_1, ..., c_{n-1} \in K$ such that
    $$c_0\overline{1} + c_1\overline{X} + \dots + c_{n-1}\overline{X}^{n-1} = \overline{0},$$
    in $K[X]/(F)$, then equivalently:
    $$c_0 + c_1X + \dots + c_{n-1}X^{n-1} = p(X)F(X)$$
    for some polynomial $p \in K[X]$. The degree of the left hand side is necessarily less than or equal to $n-1$; however, the degree of the right hand side is either 0 or strictly greater than $n-1$. For the equality to hold, it must be that the left hand side has degree 0, and hence, that $c_i = 0$ for all $i$.
    
    Therefore, the elements $\overline{1}, \overline{X}, \dots, \overline{X}^{n-1}$ are linearly independent, and hence, they form a basis of $K[X]/(F)$. \\
\end{solution}

\begin{exercise}
    Let $R = k[X_1,..., X_n]$, $k$ algebraically closed, $V = V(I)$. Show that there is a natural one-to-one correspondence between algebraic subsets of $V$ and radical ideals in $k[X_1, ..., X_n]/I$, and that irreducible algebraic sets (resp. points) correspond to prime ideals (resp. maximal ideals). (See Problem 1.22.) \\
\end{exercise}

\begin{solution}
    We already know that there is a one-to-one correspondence between the radical ideals of $k[X_1, ..., X_n] / I$ and the radical ideals of $k[X_1, ..., X_n]$ containing $I$. By Nullstellensatz, we know that there is a one-to-one correspondence between the radical ideals of $k[X_1, ..., X_n]$ and the algebraic sets. Since $V(J) \supset V(I)$ if and only if $J \subset I$, then in particular, there is a one-to-one correspondence between the radical ideals of $k[X_1, ..., X_n]$ containing $I$ and the algebraic sets contained in $V(I)$. Therefore, there is a one-to-one correspondence between the radical ideals of $k[X_1, ..., X_n] / I$ and the algebraic sets contained in $V(I)$. Since an ideal of $R/I$ is prime if and only if its associated ideal is prime in $R$, then an ideal is prime in $R/I$ if and only if the associated algebraic set conatined in $V(I)$ is irreducible (from the equivalence between prime and irreducible set). The logic is the same for the maximal ideal case. \\
\end{solution}

\begin{exercise}
    (a) Let $R$ be a UFD, and let $P = (t)$ be a principal, proper, prime ideal. Show that there is no prime ideal $Q$ such that $0 \subset Q \subset P$, $Q \neq 0$, $Q \neq P$. (b) Let $V = V(F)$ be an irreducible hypersurface in $\A^n$. Show that there is no irreducible algebraic set $W$ such that $V \subset W \subset \A^n$, $W \neq V$, $W \neq \A^n$. \\
\end{exercise}

\begin{solution}
    \begin{enumerate}[label=(\alph*)]
        \item Suppose that there is an ideal $Q$ such that $0 \subset Q \subset P$, $Q \neq 0$, $Q \neq P$. Since $Q$ is nonzero, then there is a nonzero element $q \in Q$. Since $R$ is a UFD, then there is a maximal number $n$ such that $t$ divides $q$ since $t$ is irreducible. Since $Q \subset P$, then $q = k_1t$ for some $k \in R$. But $Q$ is prime so $k_1t \in Q$ implies that $k_1 \in Q$ or $t \in Q$. But $t$ cannot be in $Q$ since otherwise, $P \subset Q$, a contradiction. Hence, $k_1 \in Q$ which implies that $k_1 = k_2t$. Applying this argument as much as we want implies that for all $i$, there is an integer $k_i$ such that $q = k_it^i$. In particular, when $i = n +1$, we get that $t^{n+1}$ divides $q$. But this contradicts the fact that $n$ is maximal. Therefore, such a prime ideal $Q$ cannot exist.
        \item If we consider the irreducible divisors of $F$, then the fact that $V(F)$ is irreducible implies that $F = F_0^n$ where $F_0$ is irreducible. It follows that $I(V) = (F_0)$ is principal, proper and prime. Hence, applying part (a) to $R = k[X_1, ..., X_n]$ and $t = F_0$ gives us that there is no prime ideal $J$ such that $0 \subsetneq J \subsetneq I(V)$. Since there is a one-to-one correspondence between prime ideals and irreducible algebraic sets, the last statement is equivalent to the fact that there is no irreducible algebraic set $W$ such that $V \subsetneq W \subsetneq \A^n$. \\
    \end{enumerate}
\end{solution}

\begin{exercise}
    Let $I = (X^2 - Y^3, Y^2 - Z^3) \subset k[X,Y,Z]$. Define $\alpha : k[X,Y,Z] \to k[T]$ by $\alpha(X) = T^9$, $\alpha(Y) = T^6$, $\alpha(Z) = T^4$. (a) Show that every element of $k[X,Y,Z]/I$ is the residue of an element $A + XB + YC + XYD$ for some $A,B,C,D \in k[Z]$. (b) If $F = A + XB + YC + XYD$, $A,B,C,D \in k[Z]$, and $\alpha(F) = 0$, compare like powers of $T$ to conclude that $F = 0$. (c) Show that $\ker(\alpha) = I$, so $I$ is prime, $V(I)$ is irreducible, and $I(V(I)) = I$. \\
\end{exercise}

\begin{solution}
    \begin{enumerate}[label=(\alph*)]
        \item Simply notice that for polynomial $F$ in $k[X,Y,Z]$, this polynomial is equivalent to a polynomial $PX + Q$, $P,Q \in k[Y,Z]$, modulo $X^2 - Y^3$, and every polynomial $H \in k[Y,Z]$ is equivalent to $RY + S$, $R,S \in k[Z]$, modulo $Y^2 - Z^3$. It follows that every element in $k[X,Y,Z]$ is equivalent to a polynomial of the form $A + XB + YC + XYD$ for some polynomials $A,B,C,D \in k[Z]$ modulo $I$. It follows that every element in $k[X,Y,Z]/I$ is the residue of such a polynomial.
        \item If $F = A + XB + YC + XYD$ with $\alpha(F) = 0$, then 
        $$0 = \alpha(A) + T^9 \alpha(B) + T^6\alpha(C) + T^{15} \alpha(D).$$
        Since $\alpha(Z) = T^4$, then the powers of $T$ in $\alpha(A)$ are all multiples of 4. Similarly, the powers of $T$ in $T^9\alpha(B)$ are all of the form $4k+1$, the powers of $T$ in $T^6\alpha(C)$ are all of the form $4k+2$ and the powers of $T$ in $T^{15}\alpha(D)$ are all of the form $4k+3$. Therefore, the only way the sum can be equal to zero is when all the coefficients of the polynomials $A,B,C,D$ are zero, and hence, that $A = B = C = D = 0$, or finally, when $F = 0$.
        \item Since $\alpha(X^2 - Y^3) = \alpha(Y^2 - Z^3) = 0$, then $I \subset \ker(\alpha)$. Next, by part(a), we know that for an arbitrary $F$, we have that $F = A + XB + YC + XYD + E$ where $A,B,C,D \in k[Z]$ and $E \in I$. Hence, $\alpha(F) = 0$ implies that
        $$\alpha(A + XB + YC + XYD) + \alpha(E) = 0.$$
        But $I \subset \ker(\alpha)$ so $\alpha(E) = 0$. Hence,
        $$\alpha(A + XB + YC + XYD) = 0$$
        which implies that $A + XB + YC + XYD = 0$ by part (b). Thus, $F = E \in I$ which proves that $\ker(\alpha) \subset I$. Therefore, $\ker(\alpha) = I$. 
        
        To show that $I$ is prime, let $AB \in I = \ker(\alpha)$ which implies that $\alpha(AB) = 0$. But $\alpha(AB) = \alpha(A)\alpha(B)$ so either $\alpha(A)$ or $\alpha(B)$ is zero. Equivalently, one of $A$ or $B$ must be in $\ker(\alpha) = I$. Therefore, $I$ is prime. By the Nullstellensatz, it follows that $V(I)$ is irreducible, and $I(V(I)) = I$. \\
    \end{enumerate}
\end{solution}

\section{Modules; Finiteness Conditions}

\begin{exercise}
    If $S$ is module-finite over $R$, then $S$ is ring-finite over $R$. \\
\end{exercise}

\begin{solution}
    If $S$ is module-finite over $R$, then there are elements $s_1, ..., s_n \in S$ such that every element in $S$ is of the form $r_1s_1 + \dots + r_n s_n$ with $r_i \in R$. Consider now the ring $R[s_1, ..., s_n]$, the smallest ring containing $R$ and the elements $s_1, ..., s_n$. Clearly, this ring is contained in $S$ since $S$ is a ring conatining $R$ and the elements $s_1, ..., s_n$. Conversely, if $r_1s_1 + \dots + r_n s_n \in S$, then $r_1s_1 + \dots + r_n s_n \in R[s_1, ..., s_n]$ because $R[s_1, ..., s_n]$ is a ring, so it contains all the elements $r_is_i$, and also their sum. Therefore, $S = R[s_1, ..., s_n]$, so $S$ is ring-finite over $R$. \\
\end{solution}

\begin{exercise}
    Show that $S = R[X]$ (the ring of polynomials in one variable) is ring-finite over $R$, but not module-finite. \\
\end{exercise}

\begin{solution}
    It is clear that the ring $S$ is ring-finte since it is equal to the smallest ring containing $R$ and $X$. Next, suppose that $S$ is module-finite, then there exist polynomials $p_1, ..., p_n$ such that every polynomial can be written as an $R$-linear combination of $p_1, ..., p_n$. However, if we let $N$ be the maximal degree of the $p_i$'s, then any linear combination of the $p_i$'s will have a degree at most $N$. It follows that not all polynomials in $S$ can be written as an $R$-linear combination of the $p_i$'s. Therefore, $S$ is not module-finite. \\
\end{solution}

\begin{exercise}
    If $L$ is ring-finite over $K$ ($K,L$ fields) then $L$ is a finitely generated field extension of $K$. \\
\end{exercise}

\begin{solution}
    If $L$ is ring-finite over $K$, then $L = K[v_1, ..., v_n]$ for some $v_1, ..., v_n \in L$. But since $L$ is a field, then the quotient field of $K[v_1, ..., v_n]$ is itself, i.e., $K[v_1, ..., v_n] = K(v_1, ..., v_n)$. Therefore $L = K(v_1, ..., v_n)$, which proves that $L$ is a finitely generated field extension of $K$. \\
\end{solution}

\begin{exercise}
    Show that $L = K(X)$ (the field of rational functions in one variable) is a finitely generated field extension of $K$, but $L$ is not ring-finite over $K$. (\textit{Hint:} If $L$ were ring-finite over $K$, a common denominator of ring generators would be an element $b \in K[X]$ such that for all $z \in L$, $b^nz \in K[X]$ for some $n$; but let $z = 1/c$, where $c$ doesn’t divide $b$ (Problem 1.5).) \\
\end{exercise}

\begin{solution}
    It is clear that $L$ is a finitely generated field extension of $K$ since it can be written as the $K(X)$. Next, suppose that $L = K[p_1/q_1, ..., p_n/q_n]$ for some $p_i/q_i \in L$, then every rational function in one variable can be written as a polynomial in $p_i/q_i$. But notice that for any polynomial in $p_i/q_i$, there exists an integer $N \geq 1$ such that multiplying the polynomial by $(q_1\cdot \cdot \cdot q_n)^N$ gives a polynomial in $X$. Now, consider the rational function $1/(q_1\cdot \cdot \cdot q_n + 1) \in L$. For all $N \geq 1$, $(q_1\cdot \cdot \cdot q_n)^N/(q_1\cdot \cdot \cdot q_n + 1)$ is a not a polynomial in $X$ because the numerator is not divisible by the denominator. Thus, by contradiction, $L$ is not ring-finite. \\
\end{solution}

\begin{exercise}
    Let $R$ be a subring of $S$, $S$ a subring of $T$.
    \begin{enumerate}
        \item If $S = \sum Rv_i$, $T = \sum S w_j$, show that $T = \sum R v_i w_j$.
        \item If $S = R[v_1, ..., v_n]$, $T = S[w_1, ..., w_m]$, show that $T = R[v_1, ..., v_n, w_1, ..., w_m]$.
        \item If $R,S,T$ are fields, and $S = R(v_1, ..., v_n)$, $T = S(w_1, ..., w_m)$, show that $T = R(v_1, ..., v_n, w_1, ..., w_m)$.\\
    \end{enumerate} 
\end{exercise}

\begin{solution}
    \begin{enumerate}
        \item Let $\sum s_j w_j$ be an arbitrary element of $T$, then each $s_j$ can be written as $\sum_i r_{ij}v_i$. Hence:
        $$\sum_j s_j w_j = \sum_j \left(\sum_i r_{ij}v_i\right)w_j = \sum_j\sum_i r_{ij}v_iw_j.$$
        Therefore, $T = \sum Rv_i w_j$ because every element in $T$ can be written as an $R$-linear combination of $v_iw_j$.
        \item Let $p(w_1, ..., w_m)$ be an arbitrary element in $T$ with $p \in S[X_1, ..., X_m]$, then $p(w_1, ..., w_m)$ is a succession of additions and multiplications of elements in $S$ and $w_1, ..., w_m$. But since every element $s$ in $S$ can be written as a polynomial $p_s(v_1, ..., v_n)$ with $p_s \in R[X_1, ..., X_n]$, then $p(w_1, ..., w_m)$ is an addition and multiplication of polynomials in $v_1, ..., v_n$ and $w_1, ..., w_m$. Since additions and multiplications of polynomials in $v_1, ..., v_n, w_1, ..., w_m$ is still a polynomial in $v_1, ..., v_n, w_1, ..., w_m$, then every element of $T$ can be written as a polynomial in $v_1, ..., v_n, w_1, ..., w_m$. Therefore, $T = R[v_1, ..., v_n, w_1, ..., w_m]$.
        \item Since $T$ contains $S$ and the elements $w_1, ..., w_m$, and $S$ contains $R$ and the elements $v_1, ..., v_n$, then $T$ is a field containing $R$ and the elements $v_1, ..., v_n,$ $ w_1, ..., w_m$. It follows that $R(v_1, ..., v_n, w_1, ..., w_m) \subseteq T$. Conversely, since $R(v_1, ..., v_n) \subseteq R(v_1, ..., v_n, w_1, ..., w_m)$, then $S \subseteq R(v_1, ..., v_n, w_1, ..., w_m)$. It follows that $T = S(w_1, ..., w_m) \subseteq R(v_1, ..., v_n, w_1, ..., w_m)$. Therefore, $T = R(v_1, ..., v_n, w_1, ..., w_m)$.\\
    \end{enumerate}
\end{solution}

\section{Integral Elements}

\begin{exercise}
    Let $R$ be a subring of $S$, $S$ a subring of (a domain) $T$. If $S$ is integral over $R$, and $T$ is integral over $S$, show that $T$ is integral over $R$. (\textit{Hint:} Let $z \in T$, so we have $z^n + a_1z^{n-1} + ... + a_n = 0$, $a_i \in S$. Then $R[a_1, ..., a_n, z]$ is module-finite over $R$.) \\
\end{exercise}

\begin{solution}
    Let $z \in T$, since $T$ is integral over $S$, then there exists elements $a_1, ..., a_n \in S$ such that $z^n + a_1z^{n-1} + ... + a_n = 0$. It follows that $z$ is integral over $R[a_1, ..., a_n]$, so the ring $R[a_1, ..., a_n, z]$ is module-finite. Since $R[v]$ is a subring of $R[a_1, ..., a_n, z]$, then $z$ is integral over $R$ by Proposition 3. Therefore, $T$ is integral over $R$. \\
\end{solution}

\begin{exercise}
    Suppose (a domain) $S$ is ring-finite over $R$. Show that $S$ is module-finite over $R$ if and only if $S$ is integral over $R$. \\
\end{exercise}

\begin{solution}
    If $S$ is module-finite over $R$, then for all $z \in S$, we have that $R[z]$ is a subring of $S$, which is module-finite over $R$, so $z$ is integral over $R$. Therefore, $S$ is integral over $R$.
    
    Conversely, suppose that $S$ is integral over $R$. Since $S$ is ring-finite over $R$, then $S = R[v_1, ..., v_n]$ for some $v_1, ..., v_n \in S$. For all $i$, there exists an integer $n_i$ such that every power $v_i^m$ with $m > n_i$ can be written in terms of $v_i$, $v_i^2$, ..., $v_i^{n_i}$ (because $v_i$ solves a polynomial equation). Since $S = R[v_1, ..., v_n]$, then every element in $S$ can be written as $p(v_1, ..., v_n)$ where $p(x_1, ..., x_n)$ is a polynomial in $R[x_1, ..., x_n]$. Since every polynomial is a sum of monomials of the form $\prod_i v_i^{m_i}$, then rewriting each higher powers of $v_i$ in terms of the lower powers gives us that $p(v_1, ..., v_n)$ is a sum of monomials of the form $\prod_i v_i^{m_i}$ with the restriction $m_i \leq n_i$. Thus, every element of $S$ can be written as a linear combination of these monomials, which there are finitely many. Thus, $S$ is generated as a module by these monomials. Therefore, $S$ is module-finite. \\
\end{solution}

\begin{exercise}
    Let $L$ be a field, $k$ an algebraically closed subfield of $L$. (a) Show that any element of $L$ that is algebraic over $k$ is already in $k$. (b) An algebraically closed field has no module-finite field extensions except itself. \\
\end{exercise}

\begin{solution}
    \begin{enumerate}
        \item Let $z \in L$ be algebraic over $k$, then there is a (monic) polynomial $p(x) \in k[x]$ such that $p(z) = 0$. Since $k$ is algebraically closed, then $p(x) = (x-a_1)\cdot \cdot \cdot (x - a_k)$ for some $a_1, ..., a_n \in k$. But since $(z-a_1)\cdot \cdot \cdot (z - a_k) = 0$, then $z = a_i$ for some $i$, and hence, $z \in k$.
        \item Let $K$ be a module-finite extension of $k$ and let $z \in K$, then $k[z]$ is contained in a module-finite ring over $k$, so $z$ must be integral over $k$. Hence, $z$ is algebraic over $k$, which implies that $z \in k$ by part (a). Therefore, $K = k$. \\
    \end{enumerate}
\end{solution}

\begin{exercise}
    Let $K$ be a field, $L = K(X)$ the field of rational functions in one variable over $K$. (a) Show that any element of $L$ that is integral over $K[X]$ is already in $K[X]$.  (\textit{Hint:} If $z^n + a_1 z^{n-1} + \dots = 0$, write $z = F/G$, $F,G$ relatively prime. Then $F^n + a_1F^{n-1}G + \dots = 0$, so $G$ divides $F$.) (b) Show that there is no nonzero element $F \in K[X]$ such that for every $z \in L$, $F^nz$ is integral over $K[X]$ for some $n > 0$. (\textit{Hint:} See Problem 1.44.) \\
\end{exercise}

\begin{solution}
    \begin{enumerate}
        \item Let $p/q \in K(X)$ be integral over $K[X]$, then there exist polynomials $a_1, ..., a_n \in K[X]$ such that
        $$\frac{p^n}{q^n} + a_1 \frac{p^{n-1}}{q^{n-1}} + \dots + a_{n-1}\frac{p}{q} + a_n = 0.$$
        Multiplying both sides by $q^n$ and isolating $p^n$ gives us
        $$p^n = -a_1p^{n-1}q - \dots - a_{n-1}pq^{n-1} - a_n q^n.$$
        Now, let $b$ be an irreducible divisor of $q$, then $b$ divides the right hand side, and hence, the left hand side which is equal to $p^n$. Since $b$ is irreducible, then $b$ divides $p$. But this shows that every irreducible factor of $q$ divides $p$, which directly implies that $q$ divides. Hence, $p/q \in K[X]$.
        \item Such an element cannot exist because it would mean that for every $p/q \in L$, there is a $n$ such that $F^n p/q \in K[X]$. If $p = 1$ and $q = F + 1$, then $F^n/(F+1) \notin K[X]$ for all $n$, contradicting the existence of $F$. \\
    \end{enumerate}
\end{solution}

\begin{exercise}
    Let $K$ be a subfield of a field $L$. (a) Show that the set of elements of $L$ that are algebraic over $K$ is a subfield of $L$ containing $K$. (\textit{Hint:} If $v^n + a_1v^{n-1} + \dots + a_n = 0$, and $a_n \neq 0$, then $v(v^{n-1} + \dots) = -a_n$.) (b) Suppose $L$ is module-finite over $K$, and $K \subset R \subset L$. Show that $R$ is a field. \\
\end{exercise}

\begin{solution}
    \begin{enumerate}
        \item An element of $L$ is algebraic over $K$ if and only if it is integral over $K$. Hence, the subset of elements of $L$ algebraic over $K$ form a subring of $L$. Hence, it suffices to show that if an element is algebraic over $K$, then its inverse is algebraic over $K$. Let $z \in L$ be algebraic over $K$, then $a_nz^n + \dots + a_0 = 0$ for some $a_i \in K$ such that $a_0 \neq 0$. It follows that $a_0(z^{-1})^n + \dots + a_n = 0$ which proves that $z^{-1}$ is algebraic pver $K$. Therefore, the elements that are algebraic over $K$ form a subfield of $L$ which contains $K$ (because every element of $K$ is algebraic over $K$).
        \item There must be a mistake in the statement of this question because it is clearly false. \\
    \end{enumerate}
\end{solution}

\section{Field Extensions}

\begin{exercise}
    Let $K$ be a field $F \in K[X]$ an irreducible monic polynomial of degree $n > 0$. (a) Show that $L = K[X]/(F)$ is a field, and if $x$ is the residue of $X$ in $L$, then $F(x) = 0$. (b) Suppose $L'$ is a field extension of $K$, $y \in L'$ such that $F(y) = 0$. Show that the homomorphism from $K[X]$ to $L'$ that takes $X$ to $y$ induces an isomorphism of $L$ with $K(y)$. (c) With $L'$, $y$ as in (b), suppose $G \in K[X]$ and $G(y) = 0$. Show that $F$ divides $G$. (d) Show that $F = (X - x)F_1$, $F_1 \in L[X]$.\\
\end{exercise}

\begin{solution}
    \begin{enumerate}
        \item By Problem 1.3, the ideal $(F)$ is maximal so $K[X]/(F)$ is a field. In this field, the residue of $F(X)$ is 0. Since $F(x)$ is equal to the residue of $F(X)$, then $F(x) = 0$.
        \item The image of the homomorphism $\phi : K[X] \to L'$ mapping $p(X)$ to $p(y)$ is $K[y]$ by definition. Hence, we have the isomorphism $K[y] \cong K[X]/\ker \phi$. Since $K[X]$ is a PID, then $\ker \phi = (f)$ for some $f \in K[X]$. Since $\phi(F) = F(y) = 0$, then $F \in \ker \phi = (f)$. It follows that $f$ divides $F$, and so by irreducibility of $F$, $f$ is either 1 or $F$. If $f = 1$, then $\ker \phi = k[X]$ which is a contradiction, so $f = F$. Hence, $K[y] \cong K[X]/(F) \cong L$. Since $L$ is a field, then $K[y] = K(y)$. Therefore, $L \cong K(y)$.
        \item If $G(y) = 0$, then $G \in \ker \phi = (F)$. It follows that $F$ divides $G$.
        \item Since $x$ is a root $F$, then viewing $F$ as a polynomial in $L[X]$ gives us that $F = (X - x)F_1$ where $F_1 \in L[X]$. \\
    \end{enumerate}
\end{solution}

\begin{exercise}
    Let $K$ be a field, $F \in K[X]$. Show that there is a field $L$ containing $K$ such that $F = \prod_{i=1}^{n}(X - x_i) \in L[X]$. (\textit{Hint:} Use Problem 1.51(d) and induction on the degree.) $L$ is called a splitting field of $F$. \\
\end{exercise}

\begin{solution}
    When $n = 1$, it is trivial because it suffices to take $L = K$. When $n > 1$, suppose that the statement holds for all polynomials of degree strictly smaller than $n$. By Problem 1.51(d), we can construct a field $L_0$ that contains $K$ and where $F = (X-a_n)F_1$ where $F_1 \in L[X]$. By induction, let $L$ be a field containing $L_0$ such that $F_1 = (X - a_1) \cdot \cdot \cdot (X - a_{n-1})$, then $L$ is a field containing $K$ such that $F = (X - a_1) \cdot \cdot \cdot (X - a_n)$. \\
\end{solution}

\begin{exercise}
    Suppose $K$ is a field of characteristic zero, $F$ an irreducible monic polynomial in $K[X]$ of degree $n>0$. Let $L$ be a splitting field of $F$, so $F = \prod_{i=1}^{n}(X - x_i)$, $x_i \in L$. Show that the $x_i$ are distinct. (\textit{Hint:} Apply Problem 1.51(c) to $G = F_X$; if $(X-x)^2$ divides $F$, then $G(x) = 0$.)\\
\end{exercise}

\begin{solution}
    Let $x$ be a root of $F$ in $L$ and suppose that $(X - x)^2$ divides $F$, then $(X - x)$ divides $F_X \in K[X]$. Hence, $G(x) = 0$, so $F$ divides $F_X$. But this is impossible since $F_X$ has a degree smaller than $F$. Thus, $(X - x)^2$ doesn't divide $F$, and so the $x_i$ are distinct.  \\
\end{solution}

\begin{exercise}
    Let $R$ be a domain with quotient field $K$, and let $L$ be a finite algebraic extension of $K$. (a) For any $v \in L$, show that there is a nonzero $a \in R$ such that $av$ is integral over $R$. (b) Show that there is a basis $v_1, ..., v_n$ for $L$ over $K$ (as a vector space) such that each $v_i$ is integral over $R$. \\
\end{exercise}

\begin{solution}
    \begin{enumerate}
        \item Since $L$ is algebraic over $K$, there are elements $a_i,b_i \in R$ such that
        $$\frac{a_n}{b_n}v^n + ... + \frac{a_1}{b_1}v + \frac{a_0}{b_0} = 0.$$
        Rewriting all of the fractions such that the have the same denominator, we get 
        $$\frac{a_n'}{b}v^n + ... + \frac{a_1'}{b}v + \frac{a_0'}{b} = 0$$
        which implies that 
        $$a_n'v^n + ... + a_1'v + a_0' = 0.$$
        Finally, multiplying both sides by $a_n'^{n-1}$ gives us
        $$(a_n' v)^n + \dots + (a_1'a_n'^{n-2})(a_n'v) + a_0'a_n'^{n-1} = 0$$
        which shows that $a_n'v$ is integral over $R$. Finally, $a_n'$ is nonzero because $a_n$ is nonzero and none of the $b_i$ are zero.
        \item Since $L$ is a finite extension of $K$, there is a basis $u_1, ..., u_n$ for $L$ over $K$. By part (a), there exist nonzero elements $a_i \in R$ such that $v_i := a_i u_i$ is integral over $R$. Since scaling the elements in a basis don't change the fact that the elements form a basis, then $v_1, ..., v_n$ is a basis for $L$ over $K$ composed only of elements which are integral over $K$. \\
    \end{enumerate}
\end{solution}