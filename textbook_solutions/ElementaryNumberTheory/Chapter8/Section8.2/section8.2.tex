\section{Primitive Roots for Primes}

\begin{exercise}
    If $p$ is an odd prime, prove that
    \begin{enumerate}
        \item the only incongruent solutions of $x^2 \equiv 1 \pmod p$ are 1 and $p-1$;
        \item the congruence $x^{p-2} + \dots + x^2 + x + 1 \equiv 0 \pmod p$ has exactly $p-2$ incongruent solutions and they are the integers 2, 3, ..., $p-1$.\\
    \end{enumerate}
\end{exercise}

\begin{solution}
    \begin{enumerate}
        \item Let $x$ be a solution of the congruence $x^2 \equiv 1 \pmod p$, then equivalently, $(x + 1)(x-1) \equiv 0 \pmod p$. Since $p$ is prime, then $x \equiv 1 \pmod p$ or $x \equiv -1 \pmod p$. Taking one representative for each gives us that $x = 1$ and $x = p-1$ are the only two incongruent solutions of the congruence.
        \item Since the congruence $x^{p-1} - 1 \equiv 0 \pmod p$ has exactly $p-1$ incongruent solutions, namely 1, 2, ..., $p-1$, then the following equivalent congruence has the same solutions:
        $$(x - 1)(x^{p-2} + \dots + x^2 + x + 1) \equiv 0 \pmod p.$$
        Let $x$ be an integer not congruent to 0 or 1 modulo $p$, then $x$ satisfies the congruence above. Since it is a product, then either $x \equiv 1 \pmod p$ or $x$ is a solution of the congruence
        $$x^{p-2} + \dots + x^2 + x + 1 \equiv 0 \pmod p.$$
        But since $x$ is assumed to be not congruent to 1, then $x$ is necessarily a solution to the latter congruence. Moreover, we clearly have that when $x$ is congruent to 0 or 1, $x$ is not a solution to the latter congruence. Therefore, the incongruent solutions of the congruence
        $$x^{p-2} + \dots + x^2 + x + 1 \equiv 0 \pmod p$$
        are 2, 3, ..., $p-1$. \\
    \end{enumerate}
\end{solution}

\begin{exercise}
    Verify that each of the congruences $x^2 \equiv 1 \pmod{15}$, $x^2 \equiv -1 \pmod{65}$ and $x^2 \equiv -2 \pmod{33}$ has four incongruent solutions; hence, Lagrange's Theorem need not hold if the modulus is a composite number. \\
\end{exercise}

\begin{solution}
    The congruence $x^2 \equiv 1 \pmod{15}$ is equivalent to the problem of finding the integers $x$ such that $(x - 1)(x+ 1)$ is a multiple of 15. A product $ab$ is a multiple of 15 if one of the following condition holds: $a$ is a multiple of 15, $b$ is a multiple of 15, $a$ is a multiple of 3 and $b$ is a multiple of 5, or $a$ is a multiple of 5 and $b$ is a multiple of 15. To find a solution in the first case, it suffices to find an $x$ such that $x + 1$ is a multiple of 15, we find that $x = 14$ is the only solution between 0 and 14. To find a solution in the second case, it suffices to find an $x$ such that $x - 1$ is a multiple of 15, we find that $x = 1$ is the only solution between 0 and 14. For the third case, we need to find an $x$ such that $x - 1$ is a multiple of 3 and $x + 1$ is a multiple of 5, we find that $x = 4$ is the only solution between 0 and 14. Finally, for the fourth case, we need to find an $x$ such that $x - 1$ is a multiple of 5 and $x + 1$ is a multiple of 3, we find that $x = 11$ is the only solution between 0 and 14. Therefore, there are four incongruent solutions: 1, 4, 11, 14.
    
    If we rewrite the second equation as $(x - 8)(x + 8) \equiv 0 \pmod{65}$ and use the same method as before, we get that there are four incongruent solutions: 8, 18, 47, 57.
    
    Similarly, if we rewrite the third equation as $(x - 8)(x + 8) \equiv 0 \pmod{33}$ and use the same method as before, we get that there are four incongruent solutions: 8, 14, 19, 25.\\
\end{solution}

\begin{exercise}
    Determine all the primitive roots of the primes $p = 11$, 19, and 23, expressing each as a power of some one of the roots.\\
\end{exercise}

\begin{solution}
    To find a primitive root of 11, we start by checking if 2 is a primitive root. Since $2^5 \equiv -1 \pmod{11}$, then 2 has order 10 modulo 11 which means that 2 is a primitive root. It follows that the other primitive roots are $2^3 \equiv 8 \pmod 11$, $2^7 \equiv 7 \pmod{11}$ and $2^9 \equiv 6 \pmod{11}$.
    
    To find a primitive root of 19, we start by checking if 2 is a primitive root. Since $2^9 \equiv -1 \pmod{11}$, then 2 has order 18 modulo 19 which means that 2 is a primitive root. It follows that the other primitive roots are $2^5 \equiv 13 \pmod{19}$, $2^7 \equiv 14 \pmod{19}$, $2^{11} \equiv 15 \pmod{19}$, $2^{13} \equiv 3 \pmod{19}$, and $2^{17} \equiv 10 \pmod{19}$.
    
    To find a primitive root of 23, we start by checking if 2 is a primitive root. Since $2^{11} \equiv 1 \pmod{23}$, then 2 is not a primitive root of 23. Similarly, $3^{11} \equiv 1 \pmod{23}$, so 3 is not a primitive root of 23. However, since $5^{11} \equiv -1 \pmod{23}$, then 5 is a primitive root of 23. It follows that the otehr primitive roots of 23 are $5^3 \equiv 10 \pmod{23}$, $5^5 \equiv 20 \pmod{23}$, $5^7 \equiv 17 \pmod{23}$, $5^9 \equiv 11 \pmod{23}$, $5^{13} \equiv 21 \pmod{23}$, $5^{15} \equiv 19 \pmod{23}$, $5^{17} \equiv 15 \pmod{23}$, $5^{19} \equiv 7 \pmod{23}$, and $5^{21} \equiv 14 \pmod{23}$. \\
\end{solution}

\begin{exercise}
    Given that 3 is a primitive root of 43, find 
    \begin{enumerate}
        \item all positive integers less than 43 having order 6 modulo 43;
        \item all positive integers less than 43 having order 21 modulo 43. \\
    \end{enumerate}
\end{exercise}

\begin{solution}
    \begin{enumerate}
        \item Since 3 is a primitive root of 43, then the integers having order 6 modulo 43 are precisely the integers of the form $3^k$ where $k$ satisfies $\phi(43)/\gcd(k, \phi(43)) = 6$, or equivalently, where $k$ satisfies $\gcd(k, 42) = 7$. These integers are precisely 7 and 35. Hence, the positive integers less than 43 having order 6 modulo 43 are $3^7 \equiv 37 \pmod{43}$ and $3^{35} \equiv 7 \pmod{43}$. 
        \item Since 3 is a primitive root of 43, then the integers having order 21 modulo 43 are precisely the integers of the form $3^k$ where $k$ satisfies $\phi(43)/\gcd(k, \phi(43)) = 21$, or equivalently, where $k$ satisfies $\gcd(k, 42) = 2$. These integers are precisely 2, 4, 8, 10, 16, 20, 22, 26, 32, 34, 38, and 40. Hence, the positive integers less than 43 having order 21 modulo 43 are
        \begin{align*}
            3^2 &\equiv 9 \pmod{43}; & 3^{22} &\equiv 40 \pmod{43}; \\
            3^4 &\equiv 38 \pmod{43}; & 3^{26} &\equiv 15 \pmod{43}; \\
            3^8 &\equiv 25 \pmod{43}; & 3^{32} &\equiv 13 \pmod{43}; \\
            3^{10} &\equiv 10 \pmod{43}; & 3^{34} &\equiv 31 \pmod{43}; \\
            3^{16} &\equiv 23 \pmod{43}; & 3^{38} &\equiv 17 \pmod{43}; \\
            3^{20} &\equiv 14 \pmod{43}; & 3^{40} &\equiv 24 \pmod{43}. \\
        \end{align*}
    \end{enumerate}
\end{solution}

\begin{exercise}
    Find all positive integers less than 61 having order 4 modulo 61. \\
\end{exercise}

\begin{solution}
    First, let's find a primitive root of 61. Since $2^{30} \equiv -1 \pmod{61}$, then 2 is a primitive root of 61. It follows that the integers less than 61 having order 4 modulo 61 are precisely the integers of the form $2^k$ where $k$ satisfies $\phi(61)/\gcd(k, \phi(61)) = 4$, equivalently, where $k$ satisfies $\gcd(k, 60) = 15$. These integers $k$ are precisely the integers 15 and 45. Therefore, positive integers less than 61 having order 4 modulo 61 are $2^{15} \equiv 11 \pmod{61}$ and $2^{45} \equiv 50 \pmod{61}$. \\
\end{solution}

\begin{exercise}
    Assuming that $r$ is a primitive root of the odd prime $p$, establish the following facts:
    \begin{enumerate}
        \item The congruence $r^{(p-1)/2} \equiv -1 \pmod p$ holds.
        \item If $r'$ is any other primitive root of $p$, then $rr'$ is not a primitive root of $p$. [\textit{Hint:} By part (a), $(rr')^{(p-1)/2} \equiv 1 \pmod p$.]
        \item If the integer $r'$ is such that $rr' \equiv 1 \pmod p$, then $r'$ is a primitive root of $p$. \\
    \end{enumerate}
\end{exercise}

\begin{solution}
    \begin{enumerate}
        \item Since $r^{p-1} \equiv 1 \pmod{p}$, then $(r^{(p-1)/2} - 1)(r^{(p-1)/2} + 1) \equiv 0 \pmod p$. Since $p$ is prime, then $r^{(p-1)/2} \equiv 1 \pmod p$ or $r^{(p-1)/2} \equiv -1 \pmod p$. The first case cannot hold because it would contradict the fact that $r$ has order $p-1$. Therefore, $r^{(p-1)/2} \equiv -1 \pmod p$.
        \item By part (a), $r^{(p-1)/2} \equiv -1 \pmod p$ and $(r')^{(p-1)/2} \equiv -1 \pmod p$. Multiplying the two equations gives us $(rr')^{(p-1)/2} \equiv 1 \pmod p$ which shows that $rr'$ is not a primitive root.
        \item Let $r'$ be an integer such that $rr' \equiv 1 \pmod p$, then for all $1 \leq k < p-1$, we have $r^k (r')^k \equiv 1 \pmod p$. If $(r')^k \equiv 1 \pmod p$, then $r^k \equiv 1 \pmod p$, a contradiction. Therefore, $r^k \not\equiv 1 \pmod p$ for all $1 \leq k < p-1$, so $r'$ is a primitive root. \\
    \end{enumerate}
\end{solution}

\begin{exercise}
    For a prime $p > 3$, prove that the primitive roots of $p$ occur in pairs $r,r'$ where $rr' \equiv 1 \pmod p$. [\textit{Hint:} If $r$ is a primitive root of $p$, consider the integer $r' = r^{p-2}$.] \\
\end{exercise}

\begin{solution}
    Let $r$ be a primitive root of $p$, then by Problem 6 part (c), we have that the integer $r'$ satisfying $rr' \equiv 1 \pmod p$ is also a primitive root of $p$. Next, this integer $r'$ is necessarily not congruent to $r$ because otherwise, we could rewrite the previous equation as $r^2 \equiv r^{p-1} \pmod p$ and conclude that $r^{p-3} \equiv 1 \pmod p$, which is impossible since $p > 3$. Therefore, the primitive roots occur in pairs $(r,r')$. \\
\end{solution}

\begin{exercise}
    Let $r$ be a primitive root of the odd prime $p$. Prove that
    \begin{enumerate}
        \item if $p \equiv 1 \pmod 4$, then $-r$ is also a primitive root of $p$;
        \item if $p \equiv 3 \pmod 4$, then $-r$ has order $(p-1)/2$ modulo $p$. \\
    \end{enumerate}
\end{exercise}

\begin{solution}
    \begin{enumerate}
        \item Since $p \equiv 1 \pmod 4$, then $(p-1)/2$ is an even number. It follows that $(-r)^{(p-1)/2} = r^{(p-1)/2} \not\equiv 1 \pmod p$. Therefore, $-r$ is a primitive root of $p$.
        \item Since $p \equiv 3 \pmod 4$, then $(p-1)/2$ is an odd number. It follows that $(-r)^{(p-1)/2} = -r^{(p-1)/2} \equiv 1 \pmod p$. Hence, $-r$ has an order equal to a divisor $(p-1)/2$. Since $(p-1)/2$ is odd, then its divisors must be odd as well. Thus, if we let $d$ be the order of $-r$, then $(-r)^d \equiv d \pmod p$, which is equivalent to $r^d \equiv -1 \pmod p$. It follows that $d = (p-1)/2$, and therefore, that $-r$ has order $(p-1)/2$ modulo $p$. \\
    \end{enumerate}
\end{solution}

\begin{exercise}
    Give a different proof of Theorem 5-5 by showing that if $r$ is a primitive root of the prime $p \equiv 1 \pmod 4$, then $r^{(p-1)/4}$ satisfies the quadratic congruence $x^2 + 1 \equiv 0 \pmod p$. \\
\end{exercise}

\begin{solution}
    Define the integer $x = r^{(p-1)/4}$, then $x^4 \equiv 1 \pmod p$, and hence, $x^2 -1 \equiv 0$ or $x^2 + 1 \pmod 0 \pmod p$. The first case implies that $r^{(p-1)/2} = x^2 \equiv 1 \pmod p$, which contradicts the fact that $r$ is a primitive root. Thus, the second case must hold, and hence, $x^2 + 1 \equiv 0 \pmod p$. Conversely, if there is an integer $x$ such that $x^2 + 1 \equiv 0 \pmod p$, then $x$ has order 4. But the order of $x$ must divide $p-1$ so $p \equiv 1 \pmod p$, proving Theorem 5-5. \\
\end{solution}

\begin{exercise}
    Use the fact that each prime $p$ has a primitive root to give a different proof of Wilson's Theorem. [\textit{Hint:} If $p$ has a primitive root $r$, then by Theorem 8-4, $(p-1)! \equiv r^{1 + 2 + \dots + (p-1)} \pmod p$.] \\
\end{exercise}

\begin{solution}
    Let $r$ be a primitive root of $p$, then the integers $r^1, ..., r^{p-1}$ are congruent to the integers $1, ..., (p-1)$ in a different order. It follows that $(p-1)! \equiv r^{1 + 2 + \dots + (p-1)} \pmod p$. Since $1 + 2 + \dots + (p-1) = (p-1)p/2$, then when $p$ is odd, we have that
    $$(p-1)! \equiv (r^p)^{(p-1)/2} \equiv r^{(p-1)/2} \equiv 1 \pmod p.$$
    Since the case when $p$ is even is trivial, this concludes the proof of Wilson's Theorem. \\
\end{solution}

\begin{exercise}
    If $p$ is a prime, show that the product of the $\phi(p-1)$ primitive roots of $p$ is congruent modulo $p$ to $(-1)^{\phi(p-1)}$. [\textit{Hint:} If $f$ is a primitive root of $p$, then then the integer $r^k$ is a primitive root of $p$ provided that $\gcd(k, p-1) = 1$; now use Theorem 7-7.] \\
\end{exercise}

\begin{solution}
    When $p > 3$, we know that the primitive roots occur in pairs $(r,r')$ where $rr' \equiv 1 \pmod p$ (Problem 7). Thus, if we group the primitive roots in such pairs in the product of all primitive roots of $p$, then the result will be equal to 1 modulo $p$. Since $p > 3$ in that case, then $\phi(p-1)$ is even, and hence, $(-1)^{\phi(p-1)} = 1 = $ the product of all primitive roots of $p$ modulo $p$. When $p = 2$, the only primitive root is 1 and $(-1)^{\phi(2-1)} \equiv 1 \pmod p$ so the statement holds. Finally, when $p = 3$, the only primitive root is -1 and $(-1)^{\phi(3 - 1)} = (-1)^1 = -1$. Therefore, the statement is true for all primes. \\
\end{solution}

\begin{exercise}
    For an odd prime $p$ verify that the sum
    $$1^n + 2^n + 3^n + \dots + (p-1)^n \equiv \begin{cases}
        0 \pmod p \text{ if } (p-1) \notdivides n \\
        -1 \pmod p \text{ if } (p-1) \divides n
    \end{cases}$$
    [\textit{Hint:} If $(p-1) \notdivides n$, and $r$ is a primitive root of $p$, then the sum is congruent modulo $p$ to
    $$1 + r^n + r^{2n} + \dots + r^{(p-2)n} \equiv \frac{r^{(p-1)n} - 1}{r^n - 1}.]$$ \\
\end{exercise}

\begin{solution}
    Let $r$ be a primitive root of $p$, then
    $$1^n + 2^n + \dots + (p-1)^n \equiv (r^n)^0 + (r^n)^1 + \dots + (r^n)^{p-2} \pmod p.$$
    Using the formula for geometric series, we get that
    $$(r^n - 1)(1^n + 2^n + \dots + (p-1)^n) \equiv (r^n)^{p-1} - 1 \pmod p.$$
    By Fermat's Little Theorem:
    $$(r^n - 1)(1^n + 2^n + \dots + (p-1)^n) \equiv 0 \pmod p.$$
    When $n$ is not a multiple of $p-1$, we have that $r^n$ is not congruent to 1 modulo $p$, and hecne, the above equation becomes
    $$1^n + 2^n + \dots + (p-1)^n \equiv 0 \pmod p.$$ 
    Next, if $n$ is a multiple of $p-1$, then $r^n \equiv 1 \pmod p$, giving us
    $$1^n + 2^n + \dots + (p-1)^n \equiv 1^0 + 1^1 + \dots + 1^{p-2} = p-1 \equiv -1 \pmod p.$$
    Therefore:
    $$1^n + 2^n + 3^n + \dots + (p-1)^n \equiv \begin{cases}
        0 \pmod p \text{ if } (p-1) \notdivides n \\
        -1 \pmod p \text{ if } (p-1) \divides n
    \end{cases}$$ 
\end{solution}