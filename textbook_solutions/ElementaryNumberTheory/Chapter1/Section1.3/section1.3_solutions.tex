\section{Early Number Theory}

\begin{exercise}
    \begin{enumerate}
        \item A number is triangular if and only if it is of the form $n(n+1)/2$ for some $n \geq 1$.
        \item The integer $n$ is a triangular number if and only if $8n+1$ is a perfect square.
        \item The sum of any two consecutive triangular number is a perfect square.
        \item If $n$ is a triangular number, then so are $9n+1$, $25n+3$ and $49n + 6$.
    \end{enumerate}
\end{exercise}

\begin{solution}
    \begin{enumerate}
        \item We already proved in the previous sections that
        $$1 + 2 + 3 \dots + n = \frac{n(n+1)}{2}$$
        so it directly follows that a number of the form of one of the side of the equation can be equivalently written in the form of the other side of the equation.
        \item  First, let $n$ be a triangular number, then there is an integer $k$ for which $n = k(k+1)/2$. It follows that
        \begin{align*}
            8n+1 = 4k(k+1) + 1 = 4k^2 + 4k + 1 = (2k+1)^2
        \end{align*}
        which shows that $8n+1$ is a perfect square. Suppose now that $8n+1$ is a perfect square for a given integer $n$. Since $8n+1$ is odd, then it must be the square of an odd number: $8n+1 = (2k+1)^2$. Thus:
        \begin{align*}
            8n+1 = (2k + 1)^2 &\implies 8n+1 = 4k^2 + 4k + 1 \\
            &\implies n = \frac{1}{2}k^2 + \frac{1}{2}k \\
            &\implies n = \frac{k(k+1)}{2}.
        \end{align*}
        Since $n$ can be written as $k(k+1)/2$, then it is a triangular number.
        \item Let $a$ and $b$ be triangular numbers, then $a$ can be written as $n(n+1)/2$. Since $b$ must have the same form while being the direct successor of $a$, then $b$ must be equal to $(n+1)(n+2)/2$. Hence:
        \begin{align*}
            a + b &= \frac{n(n+1)}{2} + \frac{(n+1)(n+2)}{2} \\
            &= \frac{n^2 + n + n^2 + 3n + 2}{2} \\
            &= \frac{2n^2 + 4n + 2}{2} \\
            &= n^2 + 2n + 1 \\
            &= (n+1)^2
        \end{align*}
        and so the sum of two consecutive triangular numbers is a perfect square.
        \item Let $n$ be a triangular number, then $n$ can be written as $k(k+1)/2$. It follows that
        \begin{align*}
            9n+1 &= 9\cdot \frac{k(k+1)}{2} + 1 \\
            &= \frac{1}{2}(9k(k+1) + 2) \\
            &= \frac{1}{2}(9k^2 + 9k + 2) \\
            &= \frac{1}{2}((3k+1)^2 + (3k+1)) \\
            &= \frac{(3k+1)((3k+1) + 1)}{2},
        \end{align*}
        \begin{align*}
            25n + 3 &= 25\cdot \frac{k(k+1)}{2} + 3 \\
            &= \frac{1}{2}(25k(k+1) + 6) \\
            &= \frac{1}{2}(25k^2 + 25k + 6) \\
            &= \frac{1}{2}((5k+2)^2 + (5k+2)) \\
            &= \frac{(5k+2)((5k+2) + 1)}{2}
        \end{align*}
        \begin{align*}
            49n + 6 &= 49\cdot \frac{k(k+1)}{2} + 6 \\
            &= \frac{1}{2}(49k(k+1) + 12) \\
            &= \frac{1}{2}(49k^2 + 49k + 12) \\
            &= \frac{1}{2}((7k+3)^2 + (7k+3)) \\
            &= \frac{(7k+3)((7k+3) + 1)}{2}
        \end{align*}
        and so $9n+1$, $25n+3$ and $49n + 6$ are all triangular numbers.
    \end{enumerate}
\end{solution}

\begin{exercise}
    If $t_n$ denotes the $n$th triangular number, prove that in terms of the binomial coefficients
    \begin{equation} \tag*{$n \geq 1.$}
        t_n = \binom{n+1}{2},
    \end{equation}
\end{exercise}

\begin{solution}
    \\ Let $n \geq 1$. We already proved that we can write $t_n$ as $n(n+1)/2$, so using the definition of the binomial coefficients, we have that 
    $$\binom{n+1}{2} = \frac{(n+1)!}{(n-1)! 2!} = \frac{(n+1)n}{2} = t_n$$
    which proves the desired formula.
\end{solution}

\begin{exercise}
    Derive the following formula for the sum of triangular numbers, attributed to the Hindu mathematician Aryabhatta (circa 500 A.D.):
    \begin{equation} \tag*{$n \geq 1.$}
        t_1 + t_2 + t_3 + \dots + t_n = \frac{n(n+1)(n+2)}{6},
    \end{equation}
    [\textit{Hint:} Group the terms on the left-hand side in pairs, noting the identity $t_{k-1} + t_k = k^2$.] \\
\end{exercise}

\begin{solution}
    \\ Let's prove it by cases. If $n = 2k$, then
    \begin{align*}
        t_1 + t_2 + \dots + t_{2k-1} + t_{2k} &= (t_1 + t_2) + \dots + (t_{2k-1} + t_{2k}) \\
        &= 2^2 + 4^2 + \dots + (2k)^2 \\
        &= 4(1^2 + 2^2 + \dots + k^2) \\
        &= 4\cdot \frac{k(k+1)(2k+1)}{6} \\
        &= \frac{2k(2k+2)(2k+1)}{6} \\
        &= \frac{n(n+2)(n+1)}{6}.
    \end{align*}
    Suppose now that $n = 2k+1$, then using the previous result:
    \begin{align*}
        t_1 + t_2 + \dots + t_{n-1} + t_{n} &= \frac{(n-1)n(n+1)}{6} + \frac{n(n+1)}{2} \\
        &= \frac{n(n+1)(n-1 + 3)}{6} \\
        &= \frac{n(n+1)(n+2)}{6}.
    \end{align*}
    Therefore, the formula is true for all $n \geq 1$. \\
\end{solution}

\begin{exercise}
    Prove that the square of any odd multiple of 3 is the difference of two triangular numbers; specifically that
    $$9(2n+1)^2 = t_{9n+4} - t_{3n+1}.$$
\end{exercise}

\begin{solution}
    \\ By direct calculation:
    \begin{align*}
        t_{9n+4} - t_{3n+1} &= \frac{(9n+4)(9n+5)}{2} - \frac{(3n+1)(3n+2)}{2} \\
        &= \frac{81n^2 + 81n + 20 - 9n^2 - 9n - 2}{2} \\
        &= \frac{72n^2 + 72n + 18}{2} \\
        &= 36n^2 + 36n + 9 \\
        &= 9(4n^2 + 4n + 1) \\
        &= 9(2n+1)^2.
    \end{align*}
\end{solution}

\begin{exercise}
    In the sequence of triangular numbers, find
    \begin{enumerate}
        \item two triangular numbers whose sum and difference are also triangular numbers;
        \item three successive triangular numbers whose product is a perfect square;
        \item three successive triangular numbers whose sum is a perfect square.
    \end{enumerate}
\end{exercise}

\begin{solution}
    \begin{enumerate}
        \item Take $15 = 1 + 2 + 3 + 4 + 5$ and $21 = 1 + 2 + 3 + 4 + 5 + 6$ since their sum is $36 = 1 + 2 + 3 + 4 + 5 + 6 + 7 + 8$ and their difference is $6 = 1 + 2 + 3$. 
        \item Take 
        $$300 = 1 + 2 + 3 + \dots + 24,$$
        $$325 = 1 + 2 + 3 + \dots + 25,$$
        $$351 = 1 + 2 + 3 + \dots + 26$$
        since their product is
        $$300 \cdot 325 \cdot 351 = (5850)^2.$$
        \item Take $15 = 1 + 2 + 3 + 4 + 5$, $21 = 1 + 2 + 3 + 4 + 5 + 6$ and $28 = 1 + 2 + 3 + 4 + 5 + 6 + 7$ since their sum is
        $$15 + 21 + 28 = 64 = 8^2.$$
    \end{enumerate}
\end{solution}

\begin{exercise}
    \begin{enumerate}
        \item If the triangular number $t_n$ is a perfect square, prove that $t_{4n(n+1)}$ is also a square.
        \item Use part (a) to find three examples of squares which are also triangular numbers. 
    \end{enumerate}
\end{exercise}

\begin{solution}
    \begin{enumerate}
        \item Suppose that $t_n$ is a perfect square, then there exists a $k$ such that $k^2 = n(n+1)/2$. It follows that
        \begin{align*}
            t_{4n(n+1)} &= \frac{4n(n+1)[4n(n+1) + 1]}{2} \\
            &= 2^2 \cdot \frac{n(n+1)}{2} \cdot (4n^2 + 4n + 1) \\
            &= (2k(2n+1))^2
        \end{align*}
        which shows that $t_{4n(n+1)}$ is a square.
        \item Using part (a), it suffices to find one such number to deduce infinitely many others. Since $6^2 = 1+ 2 + 3 +4 + 5 + 6 + 7 + 8 = t_8$, then $t_{288}$ and $t_{332928}$ must also be squares. 
    \end{enumerate}
\end{solution}

\begin{exercise}
    Show that the difference between squares of two consecutive triangular numbers is always a cube. \\
\end{exercise}

\begin{solution}
    \\Let $t_n$ and $t_{n+1}$ be two consecutive triangular numbers, then
    \begin{align*}
        t_{n+1}^2 - t_{n}^2 &= \frac{(n+1)^2(n+2)^2}{4} - \frac{n^2(n+1)^2}{4} \\
        &= \frac{(n+1)^2}{4}\left[(n+2)^2 - n^2\right] \\
        &= \frac{(n+1)^2}{4}(4n+4) \\
        &= (n+1)^3.
    \end{align*}
\end{solution}

\begin{exercise}
    Prove that the sum of the reciprocals of the first $n$ triangular numbers is less than 2; that is,
    $$1/1 + 1/3 + 1/6 + 1/10 + \dots + 1/t_n < 2.$$
    [\textit{Hint:} Observe that $\displaystyle \frac{2}{n(n+1)} = 2\left( \frac{1}{n} - \frac{1}{n+1} \right).$] \\
\end{exercise}

\begin{solution}
    \\ By direct calculation:
    \begin{align*}
        \frac{1}{1} + \frac{1}{3}+ \frac{1}{10} +\dots + \frac{1}{t_n} &= \frac{2}{1 \cdot 2} + \frac{2}{2 \cdot 3} + \frac{2}{3 \cdot 4} + \dots + \frac{2}{n(n+1)} \\
        &= 2\left( \frac{1}{1} - \frac{1}{2} \right) + 2\left( \frac{1}{2} - \frac{1}{3} \right) + \dots + 2\left( \frac{1}{n} - \frac{1}{n+1} \right) \\
        &= 2\left(\frac{1}{1} - \frac{1}{2} + \frac{1}{2} - \frac{1}{3} + \dots + \frac{1}{n} - \frac{1}{n+1}\right) \\
        &= 2\left(1 - \frac{1}{n+1}\right) \\
        &< 2.
    \end{align*}
\end{solution}

\begin{exercise}
    \begin{enumerate}
        \item Establish the identity $t_x = t_y + t_z$, where 
        $$x = 1/2 \ n(n+3) + 1, \quad y = n + 1, \quad z = 1/2 \ n(n+3),$$
        and $n \geq 1$, thereby proving that there are infinitely many triangular numbers which are the sum of two other such numbers.
        \item Find three examples of triangular numbers which are sums of two other triangular numbers.
    \end{enumerate}
\end{exercise}

\begin{solution}
    \begin{enumerate}
        \item By direct calculation:
        \begin{align*}
            t_y + t_z &= \frac{y(y+1)}{2} + \frac{z(z+1)}{2} \\
            &= \frac{(n+1)(n+2)}{2} + \frac{ \displaystyle \frac{n(n+3)}{2} \left(\frac{n(n+3)}{2} + 1\right)}{2} \\
            &= \frac{(n+1)(n+2)}{2} + \frac{n(n+3)(n(n+3) + 2)}{8}\\
            &= \frac{[n(n+3)]^2 + 2n(n+3) + 4(n+1)(n+2)}{8} \\
            &= \frac{[n(n+3)]^2 + 2n(n+3) + 4n^2 + 4\cdot 3n + 8}{8} \\
            &= \frac{[n(n+3)]^2 + 2n(n+3) + 4n(n+3) + 8}{8} \\
            &= \frac{[n(n+3)]^2 + 6n(n+3) + 8}{8} \\
            &= \frac{[n(n+3) + 2][n(n+3) + 4]}{8} \\
            &= \frac{ \displaystyle \left(\frac{n(n+3)}{2} + 1\right)\left(\frac{n(n+3)}{2} + 2\right)}{2} \\
            &= \frac{x(x+1)}{2} \\
            &= t_x.
        \end{align*}
        \item By taking plugging $n = 1$, $n = 2$ and $n = 3$ in the previous equation, we obtain that $t_3 = t_2 + t_2$, $t_6 = t_5 + t_3$ and $t_{10} = t_9 + t_4$.
    \end{enumerate}
\end{solution}