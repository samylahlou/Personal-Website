\section{Series and Products of Prime Numbers}

In this section, we explore the link that Euler established between two seemingly very unrelated fields of mathematics: the series he studied in his 1735 paper (which is presented in the previous section) and number theory. Such a correspondence is unexpected since on one hand, the study of series is purely analytic, and hence, deals with continuous objects, while on the other hand, the theory of numbers is purely discrete.

This link was established in the paper \textit{Variae observationes circa series infinitas} \cite{euler1737variae} written in 1737 and published in 1744. This paper contains a very large number of theorems about series and infinite products. More specifically, the goal of this paper is to study series where the terms are not generated by a formula but with a more intricate rule, as we will see later. Even though there is a lot of very interesting results and theorems, we will only be interested in a few.

However, before diving into the paper, we first need to understand the distinction Euler made between "$\infty$" and "$\ln(\infty)$". Using our knowledge of limits, it would be tempting to understand $\ln(\infty)$ as the limit of $\ln(n)$ as $n$ goes to infinity, and hence obtain $\ln(\infty) = \infty$. However, Euler treated these two \textit{values} differently. For Euler, these symbols contain an additional information about the way a series (or any sequence in general) approaches its limit. If a series is equal to $\ln(\infty)$, then it diverges to infinity very slowly, as slowly as the logarithm function diverges to infinity. Hence, this notation indicates the rate at which a function diverges. He calls the symbol $\infty$ the \textit{absolute infinite}, and it is to be viewed as the rate of divergence of a sequence which diverges to infinity at the same rate as its input. For example, Euler showed in his paper \textit{De Progressionibus Harmonicis Observationes} \cite{euler1740progressionibus}, written in 1734, that the Harmonic Series is equal to $\ln(\infty)$. To do so, he recalled the polynomial expansion of the logarithm function to obtain the equation

$$\ln\left(1 + \frac{1}{n}\right) = \frac{1}{n} - \frac{1}{2n^2} + \frac{1}{3n^3} - \frac{1}{4n^4} + \dots$$
which implies that applying this formula for $n = 1,..., k$ and isolating for the $\frac{1}{n}$ term gives the following equations:
\begin{align*}
    \frac{1}{1} &= \ln\left(\frac{2}{1}\right) \quad + \ \  \frac{1}{2\cdot 1^2} - \frac{1}{3\cdot 1^2} + \frac{1}{4\cdot 1^2} - \frac{1}{5\cdot 1^2} + \dots \\
    \frac{1}{2} &= \ln\left(\frac{3}{2}\right)  \quad + \ \  \frac{1}{2\cdot 2^2} - \frac{1}{3\cdot 2^2} + \frac{1}{4\cdot 2^2} - \frac{1}{5\cdot 2^2} + \dots \\
    \vdots & \qquad \  \vdots \qquad \qquad \qquad \vdots \qquad \ \quad \vdots \qquad \ \quad \vdots \qquad \ \quad \vdots  \\
    \frac{1}{k} &= \ln\left(\frac{k+1}{k}\right) + \frac{1}{2\cdot k^2} - \frac{1}{3\cdot k^2} + \frac{1}{4\cdot k^2} - \frac{1}{5\cdot k^2} + \dots \\
\end{align*}
Next, by taking the sum on both sides column by column, and using the multiplicative property of the logarithm, Euler obtained 
\begin{align*}
    1 + \frac{1}{2} + \frac{1}{3} + \dots + \frac{1}{k} &= \ln(k+1) + \frac{1}{2}\left(1 + \frac{1}{2^2} + \dots + \frac{1}{k^2}\right) \\
    & \qquad \qquad \quad \  - \frac{1}{3}\left(1 + \frac{1}{2^3} + \dots + \frac{1}{k^3}\right)\\
    & \qquad \qquad \quad \  + \frac{1}{4}\left(1 + \frac{1}{2^4} + \dots + \frac{1}{k^4}\right) \\
    & \qquad \qquad \qquad \ etc \dots
\end{align*}
Then, by letting $k$ go to infinity, he obtained
\begin{equation} \label{euler asymptotic of harmonic series}
    1 + \frac{1}{2} + \frac{1}{3} + \frac{1}{4} + \dots = \ln(\infty) + C
\end{equation}
where $C$ is defined by
$$C = \frac{1}{2}\left(1 + \frac{1}{2^2} + \frac{1}{3^2} + \dots\right) - \frac{1}{3}\left(1 + \frac{1}{2^3} + \frac{1}{3^3} + \dots \right) + \frac{1}{4}\left(1 + \frac{1}{2^4} + \frac{1}{3^4} + \dots \right) - \dots$$
Euler observed that the series defining $C$ converges (we can convince ourselves that this is true using the convergence of the series of the reciprocals and the Alternating Series Test) and even approximated it to be $C \approx 0.577218$. Since $C$ is a constant, Euler deduced from equation (\ref{euler asymptotic of harmonic series}) the following one:
\begin{equation} \label{harmonic = ln(infinity)}
    1 + \frac{1}{2} + \frac{1}{3} + \frac{1}{4} + \dots = \ln(\infty)
\end{equation}
since adding a constant does not change the rate at which the series diverges. Concerning the constant $C$ defined above, Euler considered it to be of great interest since it appears in a lot of other results. Today, this constant is denoted by the greek letter $\gamma$, and it is called the Euler-Mascheroni constant. This constant is very mysterious and we know very little about it even though it comes up in a lot of different places. We don't know yet for sure if it is irrational.

Today, these distinct \textit{infinites} are replaced by the notion of asymptotic behaviors. For example, with the standard modern notation, we would write equation (\ref{euler asymptotic of harmonic series}) as
\begin{equation} \label{modern asymptotic of harmonic series}
    \sum_{k=1}^{n}\frac{1}{k} = \ln(n) + \gamma + o(1).
\end{equation}
For more informations and a more rigorous treatment of asymptotic behaviors of sequences and functions, I recommend reading \autoref{chap:bigOnotation} which is entirely dedicated to this subject of great importance for the next chapters.

We are now ready to understand the theorems that will be interesting for us in the paper. The first six theorems of the paper are dedicated to finding the limits of various series where the terms follow intricate rules. With a modern notation, Euler studied subsets $A$ of the natural numbers and the corresponding series
$$\sum_{\substack{n=1 \\ n \in A}}^{\infty}\frac{f(n)}{n}$$
where $f: A \to \{\pm 1\}$ is a function that decides the sign of each term. For example, the first theorem of the paper states that if $A$ is the set of all numbers of the form $m^n - 1$, and $f$ puts a positive sign to each element of the set, then the corresponding series is equal to
$$\frac{1}{3} + \frac{1}{7} + \frac{1}{8} + \frac{1}{15} + \frac{1}{24} + \dots = 1.$$
Euler attributes this theorem to Christian Goldbach (1690 - 1764), a Prussian mathematician. In a similar way, the third theorem of the paper states that if $A$ is the set of multiples of 4 that are one less or one more than a power of an odd number, and $f$ puts a positive sign to elements that exceed a power by a unit and a minus sign to the other elements, then the corresponding series is equal to
$$\frac{\pi}{4} = 1 - \frac{1}{8} - \frac{1}{24} + \frac{1}{28} - \frac{1}{48} - \frac{1}{80} - \frac{1}{120} - \frac{1}{124} - \dots$$
Finally, the sixth theorem states that if $A$ is the set of numbers that are one less than squares that can also be written as another power, and $f$ gives a positive sign to each element of $A$, then the corresponding series is equal to
$$\frac{7}{4} - \frac{\pi^2}{6} = \frac{1}{15} + \frac{1}{63} + \frac{1}{80} + \frac{1}{255} + \frac{1}{624} + \dots$$
As mentionned above, even though these results are very surprising, it is the next theorem that will be interesting for our story and that we will discuss in more details. His seventh theorem is the following:
\begin{equation} \label{produit = harmonic}
    \frac{2\cdot 3 \cdot 5 \cdot 7 \cdot 11 \cdot \dots}{1\cdot 2 \cdot 4 \cdot 6 \cdot 10 \cdot \dots} = 1 + \frac{1}{2} + \frac{1}{3} + \frac{1}{4} + \frac{1}{5} + \dots
\end{equation}
where, on the left hand side, the numerator is the product of all prime numbers and the denominator is the product of the numbers that are one less than the numbers in the numerator, and the right hand side is the Harmonic Series. Right before stating his seventh theorem, Euler points out that infinite products are "not less admirable" than infinite sums. Hence, equation (\ref{produit = harmonic}) relates the Harmonic Series to its analoguous, and not less interesting, infinite product. Let's take a look at the proof of equation (\ref{produit = harmonic}) that Euler proposed.

The first step of his proof is to let $x$ be equal to the Harmonic Series and to notice that 
$$\frac{1}{2}x = \frac{1}{2} + \frac{1}{4} + \frac{1}{6} + \frac{1}{8} + \dots$$
is the series of the reciprocals of the even numbers. From that, he deduced that
$$\frac{1}{2}x = x - \frac{1}{2}x = \left(1 + \frac{1}{2} + \frac{1}{3} + \frac{1}{4} \dots\right) - \left(\frac{1}{2} + \frac{1}{4} + \frac{1}{6} + \frac{1}{8} \dots\right) = 1 + \frac{1}{3} + \frac{1}{5} + \dots$$
is the series of the reciprocals of the odd numbers. Next, he divides both sides of the previous equation by 3 to get
$$\frac{1}{2}\cdot\frac{1}{3}x = \frac{1}{3} + \frac{1}{9} + \frac{1}{15} + \frac{1}{21} + \dots$$
which implies that
\begin{align*}
    \frac{1}{2}\cdot \frac{2}{3}x &= \frac{1}{2}x - \frac{1}{2}\cdot\frac{1}{3}x \\
    &= \left(1 + \frac{1}{3} + \frac{1}{5} + \frac{1}{7} + \dots \right) - \left(\frac{1}{3} + \frac{1}{9} + \frac{1}{15} + \frac{1}{21} + \dots \right) \\
    &= 1 + \frac{1}{5} + \frac{1}{7} + \frac{1}{11} + \frac{1}{13} + \frac{1}{17} + \dots
\end{align*}
is the series of the reciprocals of the numbers that are not divisible by 2 or 3. In the same way, he concluded that
$$\frac{1}{2}\cdot \frac{2}{3}\cdot \frac{4}{5}x = 1 + \frac{1}{7} + \frac{1}{11} + \frac{1}{13} + \dots$$
is the series of the reciprocals of the numbers that are not divisible by 2, 3 or 5. Thus, by extending this pattern to infinity, he concluded that
$$\frac{1\cdot 2 \cdot 4 \cdot 6 \cdot 10 \cdot \dots}{2\cdot 3 \cdot 5 \cdot 7 \cdot 11 \cdot \dots}x = 1$$
and so
$$\frac{2\cdot 3 \cdot 5 \cdot 7 \cdot 11 \cdot \dots}{1\cdot 2 \cdot 4 \cdot 6 \cdot 10 \cdot \dots} = 1 + \frac{1}{2} + \frac{1}{3} + \frac{1}{4} + \frac{1}{5} + \dots$$
The lack of rigor in this proof makes it hard to learn anything from these manipulations since the only trick used in the proof is, by today's standards, \textit{illegal}. However, I still chose to present this proof because right after this theorem, Euler generalized his result to obtain a new theorem which is of great interest for our broader study of L-functions. His eighth theorem is the following:
\begin{equation} \label{Euler Product}
    \boxed{\frac{2^n}{2^n - 1}\cdot\frac{3^n}{3^n - 1} \cdot \frac{5^n}{5^n - 1} \cdot \frac{7^n}{7^n - 1} \dots = \frac{1}{1^n} + \frac{1}{2^n} + \frac{1}{3^n} + \frac{1}{4^n} + \dots}
\end{equation}
where the infinite product on the left hand side is taken over all the prime numbers. It is this formula that creates a first link between the prime numbers and the series on the right hand side of the equation. Notice that if we plug-in $n = 1$, we get Euler's previous theorem.

To prove equation (\ref{Euler Product}), the technique is precisely the same as before. Euler let $x$ be equal to the series on the left hand side of equation (\ref{Euler Product}) and divided it by $2^n$ to obtain
$$\frac{1}{2^n}x = \frac{1}{2^n} + \frac{1}{4^n} + \frac{1}{6^n} + \frac{1}{8^n} + \dots$$
Thus, by looking at the difference between this new series with $x$, he obtained
$$\frac{2^n - 1}{2^n}x = \left(\frac{1}{2^n} + \frac{1}{4^n} + \frac{1}{6^n}  + \dots\right) - \left(\frac{1}{1^n} + \frac{1}{2^n} + \frac{1}{3^n} + \dots\right) = \frac{1}{1^n} + \frac{1}{3^n} + \frac{1}{5^n} + \dots$$
where the sum on the right hand side is taken over all the odd numbers. Again, by extending this process to infinity, he concluded in the same way as before that
$$\left(\frac{2^n - 1}{2^n}\cdot\frac{3^n - 1}{3^n} \cdot \frac{5^n - 1}{5^n} \cdot \frac{7^n - 1}{7^n} \cdot \dots\right) x = 1$$
from which he easily deduced equation (\ref{Euler Product}).

Notice that even though this proof is nearly exactly the same as the previous one, it is way more rigorous by today's standard since in the case $n > 1$, all the series involved in the proof are now convergent (by the $p$-series test). From this theorem, Euler deduced that with $n = 2$ and using equation (\ref{pi^2/6}), he obtained
\begin{equation} \label{product for pi^2/6}
    \frac{4 \cdot 9 \cdot 25 \cdot 49 \cdot \dots }{3 \cdot 8 \cdot 24 \cdot 48 \cdot \dots} = \frac{\pi^2}{6}.
\end{equation}
which seems far from obvious at first sight since it relates an infinite product of squares of prime numbers with the square of the constant $\pi$. We can obtain similar formulas if we let $n$ be any other positive even number using the fact that Euler found a way to find a closed formula for the series on the right hand side of equation (\ref{Euler Product}). Moreover, by writing equation (\ref{Euler Product}) as
$$ \left(\frac{1}{1^n} + \frac{1}{2^n} + \frac{1}{3^n} + \frac{1}{4^n} + \dots\right)^{-1} = \left(1 - \frac{1}{2^n}\right)\left(1 - \frac{1}{3^n}\right)\left(1 - \frac{1}{5^n}\right)\left(1 - \frac{1}{7^n}\right)\dots $$
and by expanding the right hand side, we obtain
\begin{equation}
    \left(\frac{1}{1^n} + \frac{1}{2^n} + \frac{1}{3^n} + \frac{1}{4^n} + \dots\right)^{-1} = 1 - \frac{1}{2^n} - \frac{1}{3^n} - \frac{1}{5^n} + \frac{1}{6^n} - \frac{1}{7^n} + \frac{1}{10^n} - \dots
\end{equation}
where the sum on the right hand side is taken over all the square-free integers, and the sign of each term is determined by the number of prime numbers dividing the term. For example, if we plug-in $n = 2$, we obtain
\begin{equation} \label{6/pi^2}
    \frac{6}{\pi^2} = 1 - \frac{1}{2^2} - \frac{1}{3^2} - \frac{1}{5^2} + \frac{1}{6^2} - \frac{1}{7^2} + \frac{1}{10^2} - \dots
\end{equation}

As we just showed with equations (\ref{product for pi^2/6}) and (\ref{6/pi^2}), we can deduce a lot of surprising formulas from equation (\ref{Euler Product}). Euler spent the next ten theorems exploring the consequences of his product formula. Finally, after all of these results, Euler states his final theorem of the paper, the nineteenth theorem: 
\begin{equation} \label{sum of 1/p = infinity}
    \boxed{\frac{1}{2} + \frac{1}{3} + \frac{1}{5} + \frac{1}{7} + \frac{1}{11} + \dots = \ln(\ln(\infty))}
\end{equation}
where the left hand side is the sum of the reciprocals of the prime numbers. This theorem is of great importance because, as Euler points it out himself, not only it proves that there are infinitely many prime numbers as Euclid proved it nearly two millenials before, but it also shows that the prime numbers are, in a sense, infinitely more numerous than the squares. The argument is that there are obviously infinitely many squares in the natural numbers, but the squares are so sparse that the series
$$\frac{1}{1^2} + \frac{1}{2^2} + \frac{1}{3^2} + \frac{1}{4^2} + \frac{1}{5^2} + \frac{1}{6^2} + \dots$$
converges. However, there are infinitely many prime numbers, and equation (\ref*{sum of 1/p = infinity}) tells us that the sum of their reciprocals is infinite. Thus, the prime numbers are more dense since the series of their reciprocals diverges. Let's take a look at how Euler proved it.

First, he let 
\begin{align*}
    \frac{1}{2} + \frac{1}{3} + \frac{1}{5} + \frac{1}{7} + \frac{1}{11} + \dots &= A \\
    \frac{1}{2^2} + \frac{1}{3^2} + \frac{1}{5^2} + \frac{1}{7^2} + \frac{1}{11^2} + \dots &= B \\
    \frac{1}{2^3} + \frac{1}{3^3} + \frac{1}{5^3} + \frac{1}{7^3} + \frac{1}{11^3} + \dots &= C \\
    \frac{1}{2^4} + \frac{1}{3^4} + \frac{1}{5^4} + \frac{1}{7^4} + \frac{1}{11^4} + \dots &= D \\
    etc\dots \qquad \qquad &
\end{align*}
where the sums on the left hand side are taken over the prime numbers. Then, he observed that by dividing both sides of the second equation by 2, the third equation by 3, the fourth equation by 4, etc..., he would obtain 
\begin{align*}
    A &= \quad \ \ \: \frac{1}{2} + \quad \ \ \: \frac{1}{3} + \quad \ \ \: \frac{1}{5} + \quad \ \ \: \frac{1}{7} + \quad \ \ \: \frac{1}{11} + \dots \\
    \frac{1}{2}B &= \frac{1}{2}\cdot \frac{1}{2^2} + \frac{1}{2}\cdot\frac{1}{3^2} + \frac{1}{2}\cdot\frac{1}{5^2} + \frac{1}{2}\cdot\frac{1}{7^2} + \frac{1}{2}\cdot\frac{1}{11^2} + \dots \\
    \frac{1}{3}C &= \frac{1}{3}\cdot\frac{1}{2^3} + \frac{1}{3}\cdot\frac{1}{3^3} + \frac{1}{3}\cdot\frac{1}{5^3} + \frac{1}{3}\cdot\frac{1}{7^3} + \frac{1}{3}\cdot\frac{1}{11^3} + \dots \\
    \frac{1}{4}D &= \frac{1}{4}\cdot\frac{1}{2^4} + \frac{1}{4}\cdot\frac{1}{3^4} + \frac{1}{4}\cdot\frac{1}{5^4} + \frac{1}{4}\cdot\frac{1}{7^4} + \frac{1}{4}\cdot\frac{1}{11^4} + \dots \\
    & \qquad \qquad etc\dots
\end{align*}
By taking the sum on both sides column by column, we get
\begin{align*}
    A + \frac{1}{2}B + \frac{1}{3}C + \frac{1}{4}D + \dots &= \left(\frac{1}{2} + \frac{1}{2}\cdot \frac{1}{2^2} + \frac{1}{3}\cdot\frac{1}{2^3} + \frac{1}{4}\cdot\frac{1}{2^4} + \dots \right) \\
    & \! \; + \left(\frac{1}{3} + \frac{1}{2}\cdot \frac{1}{3^2} + \frac{1}{3}\cdot\frac{1}{3^3} + \frac{1}{4}\cdot\frac{1}{3^4} + \dots \right) \\
    & \! \; + \left(\frac{1}{5} + \frac{1}{2}\cdot \frac{1}{5^2} + \frac{1}{3}\cdot\frac{1}{5^3} + \frac{1}{4}\cdot\frac{1}{5^4} + \dots \right) \\
    & \! \; + \left(\frac{1}{7} + \frac{1}{2}\cdot \frac{1}{7^2} + \frac{1}{3}\cdot\frac{1}{7^3} + \frac{1}{4}\cdot\frac{1}{7^4} + \dots \right) \\
    & \! \; + \quad  etc\dots \\
    &= \ln\left(\frac{1}{1 - \frac{1}{2}}\right) + \ln\left(\frac{1}{1 - \frac{1}{3}}\right) + \ln\left(\frac{1}{1 - \frac{1}{5}}\right) + \ln\left(\frac{1}{1 - \frac{1}{7}}\right) + \dots \\
    &= \ln\left(\frac{2}{1}\right) + \ln\left(\frac{3}{2}\right) + \ln\left(\frac{5}{4}\right) + \ln\left(\frac{7}{6}\right) + \dots \\
    &= \ln\left(\frac{2\cdot 3 \cdot 5 \cdot 7 \cdot \dots}{1\cdot 2 \cdot 4 \cdot 6 \cdot \dots}\right)
\end{align*}
But now, in the last expression of the equation, Euler recalled his seventh theorem (equation (\ref{produit = harmonic})) to deduce the following equation:
\begin{equation} \label{A + B/2 + ...  = ln(infty)}
    A + \frac{1}{2}B + \frac{1}{3}C + \frac{1}{4}D + \dots = \ln\left(1 + \frac{1}{2} + \frac{1}{3} + \frac{1}{4} + \frac{1}{5} + \dots\right)
\end{equation}
Since the Harmonic Series diverges, then the right hand side of the equation diverges, and certainly the left hand side diverges as well. Next, Euler claims that the expression
$$\frac{1}{2}B + \frac{1}{3}C + \frac{1}{4}D + \dots$$
is finite. To see why, we can notice that
$$B = \frac{1}{2^2} + \frac{1}{3^2} + \frac{1}{5^2} + \dots \leq \frac{1}{2^2} + \frac{1}{3^2} + \frac{1}{4^2} + \frac{1}{5^2} + \dots,$$
where the series on the right hand side is the sum of the reciprocals of the squares without the first term. By interpreting the series on the right hand side as the area of the rectangles having width 1 and height equal to the term of the series as in \autoref{fig:visual series} and \autoref{fig:visual integral}, we obtain the inequality
$$B \leq \frac{1}{2^2} + \frac{1}{3^2} + \frac{1}{4^2} + \frac{1}{5^2} + \dots \leq \int_{1}^{\infty}\frac{1}{x^2}dx = 1.$$
Similarly, with the same argument, we obtain
$$C \leq \frac{1}{2} \qquad \qquad D \leq \frac{1}{3} \qquad \qquad E \leq \frac{1}{4} \qquad \qquad etc ...$$
and so it follows that
$$\frac{1}{2}B + \frac{1}{3}C + \frac{1}{4}D + \dots \leq \frac{1}{2 \cdot 1} + \frac{1}{3\cdot 2} + \frac{1}{4 \cdot 3} + \dots = 1 < \infty$$
using equation (\ref{telescoping series}). Therefore, the expression $\frac{1}{2}B + \frac{1}{3}C + \frac{1}{4}D + \dots$ is a constant so in equation (\ref{A + B/2 + ...  = ln(infty)}), since the left hand side diverges, then $A$ must be the only infinite term. \\

\begin{figure}[h]
\centering
\begin{minipage}{.5\textwidth}
      \centering
      \begin{tikzpicture}[scale=0.7]
        \begin{axis}[
            axis lines = left,
            xmin=0.5,
            ymin=0,
            ymax=1.3,
            ymajorticks=false
        ]
    
        \addplot [domain=1:5, samples=100, color=black,] {(1/x^2) + 0.001} node[above=1cm, pos=0.5] {$\displaystyle y = \frac{1}{x^2}$};
    
        \addplot [domain=1:2,samples=2, color=gray, opacity=0.4, name path=f1,]{0.251};
        \addplot [domain=1:2,samples=2, color=gray, name path=g1,]{0};
        \addplot[gray, opacity=0.4] fill between[of=f1 and g1, soft clip={domain=1:2}];
    
        \addplot [domain=2:3,samples=2, color=gray, opacity=0.4, name path=f2,]{0.1121};
        \addplot [domain=2:3,samples=2, color=gray, name path=g2,]{0};
        \addplot[gray, opacity=0.4] fill between[of=f2 and g2, soft clip={domain=2:3}];
    
        \addplot [domain=3:4,samples=2, color=gray, opacity=0.4, name path=f3,]{0.0635};
        \addplot [domain=3:4,samples=2, color=gray, name path=g3,]{0};
        \addplot[gray, opacity=0.4] fill between[of=f3 and g3, soft clip={domain=3:4}];
        
        \end{axis}
    \end{tikzpicture}
    \caption{Visual interpretation of}
    \text{$\frac{1}{2^2} + \frac{1}{3^2} + \frac{1}{4^2} + \frac{1}{5^2} + \dots$}
    \label{fig:visual series}
\end{minipage}%
\begin{minipage}{.5\textwidth}
      \centering
      \begin{tikzpicture}[scale=0.7]
        \begin{axis}[
            axis lines = left,
            xmin=0.5,
            ymin=0,
            ymax=1.3,
            ymajorticks=false
        ]
    
        \addplot [domain=1:5, samples=100, color=black, name path=f] {(1/x^2) + 0.001} node[above=1cm, pos=0.5] {$\displaystyle y = \frac{1}{x^2}$};
    
        \addplot [domain=1:5,samples=2, color=gray, name path=g,]{0};

        \addplot[gray, opacity=0.4] fill between[of=f and g, soft clip={domain=1:5}];
        
        \end{axis}
    \end{tikzpicture}
    \caption{Visual interpretation of}
    \text{$\int_{1}^{\infty}\frac{1}{x^2}dx$}
    \label{fig:visual integral}
\end{minipage}
\end{figure}
If we look at equation (\ref{A + B/2 + ...  = ln(infty)}) again and use the fact that everything except $A$ is a constant on the left hand side, we obtain the simpler equation
\begin{equation}
    A = \ln\left(1 + \frac{1}{2} + \frac{1}{3} + \frac{1}{4} + \frac{1}{5} + \dots \right).
\end{equation}
Finally, by using the definition of $A$ and the fact that the Harmonic Series is equal to $\ln(\infty)$ (see equation (\ref{harmonic = ln(infinity)})), Euler obtained
\begin{equation} \label{prime harmonic = ln(ln(infty))}
    \frac{1}{2} + \frac{1}{3} + \frac{1}{5} + \frac{1}{7} + \frac{1}{11} + \dots = \ln(\ln(\infty)).
\end{equation}
For Euler, this was a striking result. Here is what he had to say about this discovery at the beginning of an article he wrote in 1775 called \textit{De summa seriei ex numeris primis formatae [...]} \cite{euler1785summa}:

\begin{quotation}
    Even as Euclid had demonstrated that the multitude of prime numbers is infinite, many years ago I also showed that the sum of the series of the reciprocals of the primes [...] is infinitely large; more precisely, I showed that it has the magnitude of the logarithm of the harmonic series [...] which seems not just a little remarkable, since commonly the harmonic series is counted as the smallest kind of infinite. \footnote{Translated from the Latin by Jordan Bell, Department of Mathematics, University of Toronto, Toronto, Ontario, Canada.}
\end{quotation}

The divergence of the reciprocals of the prime number can be more rigorously stated, with a modern notation, as follows:
\begin{equation}
    \sum_{\substack{p \leq n \\ p \text{ prime}}}\frac{1}{p} = \ln(\ln(n)) + M + o(1)
\end{equation}
where $M$ is a constant called that Meissel-Martens constant. This constant is the analoguous of the Euler constant $\gamma$ for the series of the reciprocals of the prime numbers.

Euler's paper finishes with the proof of the nineteenth theorem. Again, if we forget about rigor, the proof is full of creativity, and it really shows how easy it was for Euler to use all of these seemingly complicated formulas. We often say that Euler's identity:
$$e^{i\pi} + 1 = 0,$$
is beautiful because it links a lot of different concepts in one single equation. However, by looking at Euler's use of logarithms, series, integrals, infinite products, prime numbers, telescoping series, etc \dots, in the previous proof and in his whole paper, it is clear that Euler's identity is, by far, not the only of Euler's work to display such deep links between these different mathematical objects.

At the beginning of this section, I mentionned that the paper we studied would create a link between series of the reciprocals of the powers, and prime numbers. Now that we went over the theorems presented in the paper, we see that this link was established with equation (\ref{Euler Product}). However, for the moment, this link may not seem very surprising. After all, this equation can be interpreted as another way of saying that any natural number can be uniquely written as a product of prime numbers, and that's it. The only original thing about this equation is that it involves series and infinite products. However, this precise equation will turn out to have a great importance in the future of Number Theory. Precisely one century after Euler's Paper, the mathematician Peter Lejeune Dirichlet would use this same equation to prove his famous theorem on arithmetic progressions, and hence, be one of the founder of Analytic Number Theory. 

The second important result of the paper, is equation (\ref{sum of 1/p = infinity}). Again, this result can be interpreted as another way of saying that there are infinitely many primes numbers. What is original about this equation is, again, that it is stated using series, and that it also carries an additional information about the density of the prime number as a subset of the natural numbers. Euler's quote from earlier in this section is from a paper he wrote in 1775 in which he extended his work on the series of the reciprocals of the prime numbers. In this paper, he proved that the series of the reciprocals of the prime numbers of the form $4n+1$ diverges and that the series of the reciprocals of the prime numbers of the form $4n-1$ diverges as well. He even conjectured that the series of the reciprocals of the prime numbers of the form $100n + 1$ diverges. As a direct corollary, we have that there are infinitely many prime numbers of the form $4n+1$ and of the form $4n-1$. These theorems will be extended into a more general theorem proved by Dirichlet: there are infinitely many prime numbers of the form $an + b$ where $a$ and $b$ have no common divisors. Let's end this section with a quote by William Dunham that summerizes well our previous discussion on Euler's work.

\begin{quotation}
    Those familiar with the prime number theorem may forget how wondrous a thing it is, linking primes to the natural logarithm function. Yet this is precisely the sort of connection - between discrete and continuous - that Euler first perceived [...]. If Euler does not quite deserve to be called the "parent" of analytic number theory, let us at least credit him with being its obvious grandparent. \footnote{Quote from the end of Chapter 4, Euler : The Master of Us All \cite{dunham1999euler}. }
\end{quotation}