\section{Measurable Spaces and Functions}

\begin{exercise}
    Show that $\mathcal{S} = \{\cup_{n \in K}(n, n+1]:K\subset\N\}$ is a $\sigma$-algebra on $\R$.\\
\end{exercise}

\begin{solution}
    \\ As most of the proofs showing that a collection is a $\sigma$-algebra, let's split this one into three parts:
    \begin{itemize}
        \item ($\varnothing \in \mathcal{S}$) Since $\varnothing \subset \Z$, then $\cup_{n \in \varnothing}(n, n+1] \in \mathcal{S}$. However, notice that $\cup_{n \in \varnothing}(n, n+1] = \varnothing$. It follows that $\varnothing \in \mathcal{S}$.
        \item (closed under complements) Let $A \in \mathcal{S}$, then there exists a $K_0 \subset \Z$ such that $A = \cup_{n \in K_0}(n, n+1]$
        Consider $K_1 = \Z \setminus K_0$ and its associated element $B = \cup_{n \in K_1}(n, n+1]$ in $\mathcal{S}$. Since $A \cap B = \varnothing$ and $A \cup B = \R$, then $B = \R \setminus A$. Hence, $A^c \in \mathcal{S}$ which proves that $\mathcal{S}$ is closed under complements.
        \item (closed under countable union) Let $\{A_i\}_i$ be a countable collection of elements in $\mathcal{S}$, then for all $i \in \N$, there is a subset $K_i$ of $\Z$ such that $A_i = \cup_{n \in K_i}(n, n+1]$. Consider $K = \cup_{i=1}^{\infty}K_i \subset \Z$ and $A = \cup_{n \in K}(n, n+1] \in \mathcal{S}$. By consruction, $A = \cup_{i=1}^{\infty}A_i \in \mathcal{S}$. Therefore, $\mathcal{S}$ is closed under countable union.
    \end{itemize}
    Therefore, $\mathcal{S}$ is a $\sigma$-algebra on $\R$. \\
\end{solution}

\begin{exercise}
    Verify both bullet points in Example 2.28. \\
\end{exercise}

\begin{solution}
    \begin{itemize}
        \item Suppose $X$ is a set and $\mathcal{A}$ is the set of subsets of $X$ that consist of exactly one element:
        $$\mathcal{A} = \{\{x\}:x\in \mathcal{A}\}$$
        Define $\mathcal{S}$ to be the smallest $\sigma$-algebra on $X$ generated by $\mathcal{A}$. Let's prove that $\mathcal{S}$ is precisely the collection of subsets of $X$ that are countable or co-countable. To make it easier, denote by $\mathcal{M}$ the collection of subsets of $X$ that are countable or co-countable. \\
        Hence, we need to prove that $\mathcal{S} = \mathcal{M}$. We already know from example 2.24 that $\mathcal{M}$ is a $\sigma$-algebra on $X$. Moreover, it is easy to see that $\mathcal{A} \subset \mathcal{M}$. It follows that $\mathcal{S} \subset \mathcal{M}$.\\
        To prove the reverse inclusion, let $E \in \mathcal{M}$, then one of $E$ or $E^c$ is countable. If $E$ is countable, then we can simply write $E$ as the countable union of the singletons of its elements, hence, a countable union of elements in $\mathcal{A} \subset \mathcal{S}$. This would imply that $E \in \mathcal{S}$. Similarly, if $E^c$ is countable, then with the same argument, $E^c \in \mathcal{S}$ which also implies that $E \in \mathcal{S}$. Thus, $\mathcal{M} \subset \mathcal{S}$. It follows that $\mathcal{S} = \mathcal{M}$.
        \item Let $\mathcal{A} = \{(0,1), (0, \infty)\}$ and denote by $\mathcal{S}$ the smallest $\sigma$-algebra containing $\mathcal{A}$. Define the collection 
        $$E = \{\varnothing, \ (0,1), \ (0, \infty), \ (-\infty, 0]\cup[1, \infty), \ (-\infty, 0], \ [1, \infty), \ (-\infty, 1), \ \R\}$$
        Let's show that $\mathcal{S} = E$. First, let's prove that $E$ is a $\sigma$-algebra:
        \begin{itemize}
            \item ($\varnothing \in E$) By definition of $E$.
            \item (closed under complement) 
            \begin{itemize}
                \item $\R \setminus \varnothing = \R \in E$
                \item $\R \setminus (0,1) = (-\infty, 0]\cup[1, \infty) \in E$
                \item $\R \setminus (0,\infty) = (-\infty, 0] \in E$
                \item $\R \setminus ((-\infty, 0]\cup[1, \infty)) = (0,1) \in E$
                \item $\R \setminus [1, \infty) = (-\infty,1) \in E$
                \item $\R \setminus (-\infty, 0] = (0,\infty) \in E$
                \item $\R \setminus (-\infty, 1) = [1,\infty) \in E$
                \item $\R \setminus \R = \varnothing \in E$
            \end{itemize}
            \item (closed under countable union) Since $E$ is finite, then it suffices to check that $E$ is closed under the regular union between two sets. To be faster, I skipped the trivial unions that involve $\R$ or $\varnothing$.
            \begin{itemize}
                \item $(0,1)\cup(0, \infty) = (0, \infty) \in E$
                \item $(0, 1) \cup ((-\infty, 0]\cup[1, \infty)) = \R \in E$
                \item $(0, 1) \cup [1, \infty) = (0, \infty) \in E$
                \item $(0, 1) \cup (-\infty, 0] = (-\infty, 1) \in E$
                \item $(0, 1) \cup (-\infty, 1) = (-\infty, 1) \in E$
                \item $(0, \infty) \cup ((-\infty, 0]\cup[1, \infty)) = \R \in E$
                \item $(0, \infty) \cup [1, \infty) = (0, \infty) \in E$
                \item $(0, \infty) \cup (-\infty, 0] = \R \in E$
                \item $(0, \infty) \cup (-\infty, 1) = \R \in E$
                \item $((-\infty, 0]\cup[1, \infty)) \cup [1, \infty) = ((-\infty, 0]\cup[1, \infty)) \in E$
                \item $((-\infty, 0]\cup[1, \infty))\cup (-\infty, 0] = ((-\infty, 0]\cup[1, \infty)) \in E$
                \item $((-\infty, 0]\cup[1, \infty)) \cup (-\infty, 1) = \R \in E$
                \item $[1, \infty)\cup (-\infty, 0] = ((-\infty, 0]\cup[1, \infty)) \in E$
                \item $[1, \infty) \cup (-\infty, 1) = \R \in E$
                \item $(-\infty, 0] \cup (-\infty, 1) = (-\infty, 1) \in E$
            \end{itemize}
        \end{itemize}
        Therefore, $E$ is a $\sigma$-algebra on $\R$ that contains $\mathcal{A}$. It follows that $\mathcal{S} \subset E$. For the reverse inclusion, let's prove that any element in $E$ can be constructed from elements in $\mathcal{A}$ using the operations of $\sigma$-algebras, i.e., complements and unions:
        \begin{itemize}
            \item $\varnothing$ is in $\mathcal{S}$ because $\mathcal{S}$ is a $\sigma$-algebra.
            \item (0,1) is in $\mathcal{S}$ because it is in $\mathcal{A}$.
            \item $(0, \infty)$ is in $\mathcal{S}$ because it is in $\mathcal{A}$.
            \item $(-\infty, 0]\cup[1, \infty)$ is in $\mathcal{S}$ because it is the complement of (0,1) which is in $\mathcal{S}$.
            \item $[1, \infty)$ is in $\mathcal{S}$ because it can be written as $(0, \infty) \setminus (0,1)$ and we already know that $(0, \infty)$ and $(0,1)$ are in $\mathcal{S}$.
            \item $(-\infty, 0]$ is in $\mathcal{S}$ because it can be written as $((-\infty, 0]\cup[1, \infty)) \setminus [1, \infty)$ and we already know that both sets are in $\mathcal{S}$.
            \item $(-\infty, 1)$ is in $\mathcal{S}$ because it is the complement of $[1, \infty)$ which is in $\mathcal{S}$.
            \item $\R$ is in $\mathcal{S}$ because $\mathcal{S}$ is a $\sigma$-algebra.
        \end{itemize}
        Therefore, $\mathcal{S} = E$.\\
    \end{itemize}
\end{solution}

\begin{exercise}
    Suppose $\mathcal{S}$ is the smallest $\sigma$-algebra on $\R$ containing $\{(r, s]:r,s\in\Q\}$. Prove that $\mathcal{S}$ is the collection of Borel subsets of $\R$. \\
\end{exercise}

\begin{solution}
    \\ To make things easier, let's denote by $\mathcal{B}$ the collection of Borel subsets of $\R$. We need to prove that $\mathcal{S} = \mathcal{B}$. First, notice that for all rationals $r,s\in\Q$ with $r < s$, the set $(r, s] \in \mathcal{B}$ since it can be written as $(r, \infty) \setminus (s, \infty)$ and both are open (so Borel) sets. It follows that $\mathcal{S} \subset \mathcal{B}$.\\
    For the reverse inclusion, let's show that $\mathcal{S}$ contains every open sets. Let's prove first that $\mathcal{S}$ contains open interval Let $(a, b)$ be an open interval with $a < b \in \R$. By density of $\Q$ in $\R$, there exist two sequence $\{q_n\}_n$ and $\{s_n\}$ of rationals that satisfy the following properties : $\{q_n\}_n$ is decreasing and converges to $a$, $\{s_n\}_n$ is increasing and converges to $b$. Since all of the terms are rationals, then $(r_n, s_n] \in \mathcal{S}$ for all $n\in \N$. But $\mathcal{S}$ is a $\sigma$-algebra so 
    $$(a,b) = \bigcup_{n=1}^{\infty}(r_n, s_n] \in \mathcal{S}$$
    Hence, $\mathcal{S}$ contains every open interval. Now, using the fact that any open set can be written as a countable union of open intervals, it easily follows that $\mathcal{S}$ actually contains every open set. Therefore, $\mathcal{B} \subset \mathcal{S}$ since $\mathcal{B}$ is generated by the open sets so $\mathcal{S} = \mathcal{B}$.\\
\end{solution}

\begin{exercise}
    Suppose $\mathcal{S}$ is the smallest $\sigma$-algebra on $\R$ containing $\{(r, n]:r\in\Q, n\in\N\}$. Prove that $\mathcal{S}$ is the collection of Borel subsets of $\R$. \\
\end{exercise}

\begin{solution}
    \\ For this proof, I will use the result of the previous exercise. Hence, define $E = \{(r, n]:r\in\Q, n\in\N\}$, $E_0 = \{(r, s]:r,s\in\Q\}$ and consider $\mathcal{S}_0$ to be the smallest $\sigma$-algebra generated by $E_0$. Let's denote by $\mathcal{B}$ the collection of Borel subsets of $\R$. Obviously, since $E \subset E_0 \subset \mathcal{S}_0$, then $\mathcal{S} \subset \mathcal{S}_0$. \\
    For the reverse inclusion, let's show that $E_0 \subset \mathcal{S}$. Let $(r, s] \in E_0$ with $r,s \in \Q$. Notice that for all integers $n \geq r$, $(r, n] \in E \subset \mathcal{S}$, hence, taking their union gives us
    $$(r, \infty) = \bigcup_{n \geq r}(r, n] \in \mathcal{S}$$
    Similarly, $(s, \infty) \in \mathcal{S}$ for the same reasons. Hence, $(r, s] = (r, \infty) \setminus (s, \infty) \in \mathcal{S}$. It follows that $E_0 \subset \mathcal{S}$ which implies that $\mathcal{S}_0 \subset \mathcal{S}$. Therefore
    $$\mathcal{S} = \mathcal{S}_0 = \mathcal{B}$$\\
\end{solution}

\begin{exercise}
    Suppose $\mathcal{S}$ is the smallest $\sigma$-algebra on $\R$ containing $\{(r, r+1):r\in\Q\}$. Prove that $\mathcal{S}$ is the collection of Borel subsets of $\R$. \\
\end{exercise}

\begin{solution}
    \\ Let $E = \{(r, r+1):r\in\Q\}$ and denote by $\mathcal{B}$ the collection of Borel subsets of $\R$. Moreover, let $E_0 = \{(r, n]:r\in\Q, n\in\N\}$ and define $\mathcal{S}_0$ as the smallest $\sigma$-algebra containing $E_0$. By exercise 4, we know that
    $$\mathcal{S}_0 = \mathcal{B}$$
    Let's show that $\mathcal{S} = \mathcal{B}$. Obviously, since every element in $E$ is a Borel set, then $E \subset \mathcal{B}$ which implies that $\mathcal{S} \subset \mathcal{B}$. \\
    For the reverse inclusion, Let's show that $E_0 \subset \mathcal{S}$. Let $(r, n]$ be an arbitrary element of $E_0$ with $r \in \Q$ and $n \in \N$. By definition of $E$, we know that for all $k \geq 0$, the set $(r + \frac{1}{2}k, r + \frac{1}{2}k + 1) \in E \subset \mathcal{S}$. Thus, since $\mathcal{S}$ is closed under countable unions,
    $$(r, \infty) = \bigcup_{k=1}^{\infty}\left(r + \frac{1}{2}k, r + \frac{1}{2}k + 1\right) \in \mathcal{S}$$
    Similarly, $(n, \infty) \in \mathcal{S}$ for the same reasons. Hence, $(r,s] = (r, \infty) \setminus (n, \infty) \in \mathcal{S}$. It follows that $E_0 \subset \mathcal{S}$ which implies $\mathcal{B} = \mathcal{S}_0 \subset \mathcal{S}$. Therefore, $\mathcal{S} = \mathcal{B}$.\\
\end{solution}

\begin{exercise}
    Suppose $\mathcal{S}$ is the smallest $\sigma$-algebra on $\R$ containing $\{[r, \infty):r\in\Q\}$. Prove that $\mathcal{S}$ is the collection of Borel subsets of $\R$. \\
\end{exercise}

\begin{solution}
    \\ Let $E = \{[r, \infty):r\in\Q\}$ and denote by $\mathcal{B}$ the collection of Borel subsets of $\R$. Moreover, let $E_0 = \{(r, s]:r,s \in \Q\}$ and define $\mathcal{S}_0$ as the smallest $\sigma$-algebra containing $E_0$. By exercise 3, we know that
    $$\mathcal{S}_0 = \mathcal{B}$$
    Let's show that $\mathcal{S} = \mathcal{B}$. Since every element of $E$ is closed, then $E \subset \mathcal{B}$ (closed sets are Borel sets). It follows that $\mathcal{S} \subset \mathcal{B}$.\\
    For the reverse inclusion, let's prove that $E_0 \subset \mathcal{S}$. Let $(r, s]$ be an arbitrary set in $E_0$ with $r < s \in \Q$, then by definition of $E$, both $[r + \frac{1}{n}, \infty)$ and $[s + \frac{1}{n}, \infty)$ are contained in $E$ and hence in $\mathcal{S}$. Since $\mathcal{S}$ is a $\sigma$-algebra, then
    $$(r, \infty) = \bigcup_{n=1}^{\infty} \biggl[ r + \frac{1}{n}, \infty \biggr) \in \mathcal{S} $$
    and 
    $$(s, \infty) = \bigcup_{n=1}^{\infty} \biggl[ s + \frac{1}{n}, \infty \biggr) \in \mathcal{S} $$
    It follows that $(r, s] = (r, \infty) \setminus (s, \infty) \in \mathcal{S}$. Hence, $E_0 \subset \mathcal{S}$ which implies $\mathcal{B} = \mathcal{S}_0 \subset \mathcal{S}$. Therefore, $\mathcal{S} = \mathcal{B}$. \\
\end{solution}

\begin{exercise}
    Prove that the collection of Borel subsets of $\R$ is translation invariant. More precisely, prove that if $B \subset \R$ is a Borel set and $t \in \R$, then $t+B$ is a Borel set.\\
\end{exercise}

\begin{solution}
    \\ Let $B \subset \R$ be a Borel set and $t$ be an arbitrary real number. Let's show that $t+B$ is a Borel set. Consider the function $f : \R \to \R$ defined by $x \mapsto x - t$. Since $f$ is continuous, then $f$ is Borel measurable. It follows that $f^{-1}(B)$ is a Borel set. However, notice that for all $x \in \R$:
    \begin{align*}
        x \in f^{-1}(B) &\iff f(x) \in B \\
        &\iff x-t \in B \\
        &\iff x \in t + B
    \end{align*}
    Hence,  $t + B = f^{-1}(B)$. Therefore, $t + B$ is a Borel set which proves that the collection of Borel sets is translation invariant.\\
\end{solution}

\begin{exercise}
    Prove that the collection of Borel subsets of $\R$ is dilation invariant. More precisely, prove that if $B \subset \R$ is a Borel set and $t \in \R$, then $tB$ (which is defined to be $\{tb:b\in B\}$) is a Borel set.\\
\end{exercise}

\begin{solution}
    \\ Let $B \subset \R$ be a Borel set and $t$ be an arbitrary real number. Let's show that $tB$ is a Borel set. Notice that the case $t = 0$ is trivial since $tB = \{0\}$ in that case and $\{0\}$ is a Borel set. Consider the function $f : \R \to \R$ defined by $x \mapsto \frac{1}{t}x$. Since $f$ is continuous, then $f$ is Borel measurable. It follows that $f^{-1}(B)$ is a Borel set. However, notice that for all $x \in \R$:
    \begin{align*}
        x \in f^{-1}(B) &\iff f(x) \in B \\
        &\iff \frac{1}{t}x \in B \\
        &\iff x \in tB
    \end{align*}
    Hence,  $tB = f^{-1}(B)$. Therefore, $tB$ is a Borel set which proves that the collection of Borel sets is dilation invariant.\\
\end{solution}

\begin{exercise}
    Give an example of a measurable space $(X, \mathcal{S})$ and a function $f: X \to \R$ such that $|f|$ is $\mathcal{S}$-measurable but $f$ is not $\mathcal{S}$-measurable. \\
\end{exercise}

\begin{solution}
    \\ Consider $(X, \mathcal{S}) = (\R, \{ \varnothing, \R\})$ and the function $f:X \to \R$ defined by 
    $$f(x) = \begin{cases}
        1 & x \geq 0 \\
        -1 & x < 0
    \end{cases}$$
    Notice that $|f| = \chi_{\R}$ and hence $\mathcal{S}$-measurable since $\R \in \mathcal{S}$. However, 
    $$f^{-1}(\{1\}) = [0, \infty) \notin \mathcal{S}$$
    even if $\{1\}$ is a Borel set. Therefore, $|f|$ is $\mathcal{S}$-measurable but not $f$. \\ 
\end{solution}

\begin{exercise}
    Show that the set of real numbers that have a decimal expansion with the digit 5 appearing infinitely often is a Borel set. \\
\end{exercise}

\begin{solution}
    \\ This proof will have three steps but the idea is the following : 
    \begin{enumerate}
        \item Construct the set of reals that contains no digit 5 in their decimal part in a process similar to the construction of the Cantor set. By construction, show that this set is a Borel set.
        \item Construct, using the previous set, the set of reals with finitely many 5's in their decimal expansion. By construction, show that this set is a Borel set.
        \item Simply take the complement of the previous set. It follows that the desired set is Borel.
    \end{enumerate}
    (Step 1) Let's construct the set of reals that contains no 5 in their decimal part. Let's construct recursively a sequence of sets that converges to the desired set. To do so, define the Borel set $M_0 = \R$ which simply represents the reals. Define $M_1$ which represents the set of reals that contains no 5 in their first decimal and $M_2$ as the set representing the reals with no digit 5 in their first two decimals. To generalize this process, suppose that $M_n$ is a Borel which represents the reals which contains no 5 in their first $n$ decimals, to construct $M_{n+1}$, simply remove from $M_n$ the reals with a 5 in their $(n+1)$st decimal. Notice that the set of of reals with a 5 in their $(n+1)$st decimal can be written as follows : $\frac{1}{10^{n+1}}(10\Z + 5)$. Hence,
    $$M_{n+1} = M_n \setminus \frac{1}{10^{n+1}}(10\Z + 5)$$
    By properties of $\sigma$-algebras and exercise 7 and 8, $M_{n+1}$ is also a Borel set. Thus, if we define 
    $$M = \bigcap_{n=0}^{\infty}M_n$$
    by construction of the $M_n$'s, we have that $M$ is precisely the (Borel) set of reals that contains no 5's in their decimal part (such reals can contain a 5 in their decimal representation but only in the integer part). \bigbreak
    \noindent (Step 2) To construct the set of reals with finitely many 5's in their decimal expansion, notice that if $x \in \R$ has finitely many 5's in their decimal expansion, then there is a natural number $n$ such that $10^n x$ has no 5's in its decimal part. From this observation, we get that the set
    $$N = \bigcup_{n=1}^{\infty}\frac{1}{10^n}M$$ 
    is precisely the set of reals with finitely 5's in their decimal expansion. Moreover, by exercise 8 and by properties of $\sigma$-algebras, $N$ is a Borel set. \bigbreak
    \noindent (Step 3) By construction, $N^c$ must be the set of reals with infinitely many 5's in their decimal expansion. Since the collection of Borel sets is closed under complements, then $N^c$ is a Borel set. We could also have shown that it has outer measure 0 by the construction on the interval $[0,1]$ and then extending to the reals but the proof would have been longer. \\
\end{solution}

\begin{exercise}
    Suppose $\mathcal{T}$ is a $\sigma$-algebra on a set $\mathcal{Y}$ and $X \in \mathcal{T}$. Let $\mathcal{S} = \{E \in \mathcal{T} : E\subset X\}$.
    \begin{enumerate}[label = (\alph*)]
        \item Show that $\mathcal{S} = \{F \cap X : F \in \mathcal{T}\}$.
        \item Show that $\mathcal{S}$ is a $\sigma$-algebra on $X$.\\
    \end{enumerate}
\end{exercise}

\begin{solution}
    \begin{enumerate}[label = (\alph*)]
        \item Let $E \in \mathcal{S}$, then $E \in \mathcal{T}$ and $E \subset X$. It follows that $E = E \cap X \in \{F \cap X : F \in \mathcal{T}\}$. Hence, $\mathcal{S} \subset \{F \cap X : F \in \mathcal{T}\}$. For the reverse inclusion, let $F \cap X$ be an arbitrary element of $\{F \cap X : F \in \mathcal{T}\}$, then $F \in \mathcal{T}$ which implies that $F \cap X \in \mathcal{T}$. Moreover, $F \cap X \subset X$ so $F \cap X \in \mathcal{S}$. Therefore, $\mathcal{S} = \{F \cap X : F \in \mathcal{T}\}$.
        \item First, since $\varnothing \in \mathcal{T}$ and $\varnothing \subset X$, then $\varnothing \in \mathcal{S}$. Now, if $E$ is an arbitrary element of $\mathcal{S}$, then its complement, $X \setminus E$ is still in $\mathcal{T}$ and obviously is a subset of $X$. Hence, $E^c \in \mathcal{S}$. Thus, $\mathcal{S}$ is closed under complements. Suppose that $\{E_n\}_n$ is a countable collection of elements in $\mathcal{S}$, then they all are in $\mathcal{T}$ and all are subsets of $X$. It follows that their union is still in $\mathcal{T}$ and still a subset of $X$. Hence, their union is in $\mathcal{S}$. Therefore, $\mathcal{S}$ is a $\sigma$-algebra. \\
    \end{enumerate}
\end{solution}

\begin{exercise}
    Suppose $f : \R \to \R$ is a function.
    \begin{enumerate}[label = (\alph*)]
        \item For $k \in \N$, let
        \begin{align*}
            G_k = \{a \in \R : &\text{ there exists } \delta > 0 \text{ such that } |f(b) - f(c)| < \tfrac{1}{k} \\ & \text{ for all } b,c \in (a - \delta, a + \delta) \}.
        \end{align*}
        Prove that $G_k$ is an open subset of $\R$ for each $k \in \N$.
        \item Prove that the set of points at which $f$ is continuous equals $\cap_{k=1}^{\infty}G_k$.
        \item Conclude that the set of points at which $f$ is continuous is a Borel set.\\
    \end{enumerate}
\end{exercise}

\begin{solution}
    \begin{enumerate}[label = (\alph*)]
        \item Let $k \in \N$ and let's prove that $G_k$ is open by proving that every point is an interior point of the set. Let $x \in G_k$, then by definition, there is a $\delta > 0$ such that
        $$|f(y) - f(z)| < \frac{1}{k}$$
        for all $y, z \in  (x - \delta, x + \delta)$. Let's show that $(x - \delta, x + \delta) \subset G_k$. Let $x_0 \in (x - \delta, x + \delta)$ and define $\delta_0 = \min(x_0 - x + \delta, x+\delta - x_0)$. It follows that 
        $$(x_0 - \delta_0, x_0 + \delta_0) \subset (x - \delta, x + \delta)$$
        Hence, for all $y, z \in (x_0 - \delta_0, x_0 + \delta_0)$, we have $y, z \in  (x - \delta, x + \delta)$ which implies
        $$|f(y) - f(z)| < \frac{1}{k}$$
        Thus, $x_0 \in G_k$. Since it holds for all $x_0 \in (x - \delta, x + \delta)$, then $(x - \delta, x + \delta) \subset G_k$. Since it holds for all $x \in G_k$, then $G_k$ is open.
        \item  Let's show that the elements in $\cap_{k=1}^{\infty}G_k$ are precisely the points on which $f$ is continuous. Let $x$ be a real number such that $f$ is continuous at $x$. Let $k \in \N$, then by continuity of $f$ at $x$, there is a $\delta > 0$ such that
        $$|f(y) - f(x)| < \frac{1}{2k}$$
        whenever $y \in (x - \delta, x + \delta)$. Hence, for all $b,c \in (x - \delta, x + \delta)$, by the triangle inequality:
        $$|f(b) - f(c)| < \frac{1}{k}$$
        Thus, $x \in G_k$. Since it   holds for all $k \in \N$, then $x \in \cap_{k=1}^{\infty}G_k$. It follows that $C_f \subset \cap_{k=1}^{\infty}G_k$.\\
        For the reverse inclusion, let $x$ be an arbitrary element of $\cap_{k=1}^{\infty}G_k$, let's show that $f$ is continuous at $x$ using the $\epsilon$-$\delta$ definition. Let $\epsilon > 0$, then by the Archimedean Property of $\R$, there is a $n \in \N$ such that $\frac{1}{n} < \epsilon$. But recall that $x \in \cap_{k=1}^{\infty}G_k \subset G_n$, hence, there is a $\delta > 0$ such that 
        $$|f(b) - f(c)| < \frac{1}{n}$$
        whenever $b,c \in (x - \delta, x + \delta)$. Let $y \in (x - \delta, x + \delta)$, since $x$ is also in $(x - \delta, x + \delta)$, then
        $$|f(x) - f(y)| < \frac{1}{n} < \epsilon$$
        Thus, by definition, $f$ is continuous at $x$. Therefore, $\cap_{k=1}^{\infty}G_k$ is precisely the set of points on which $f$ is continuous.
        \item The set of points at which $f$ is continuous can be written as a countable intersection of open sets. Since open sets are Borel sets and Borel sets are closed under countable intersections, then the set of points at which $f$ is continuous is a Borel set. \\
    \end{enumerate}
\end{solution}

\begin{exercise}
    Suppose $(X, \mathcal{S})$ is a measurable space, $E_1, ..., E_n$ are disjoint subsets of $X$, and $c_1, ..., c_n$ are distinct nonzero real numbers. Prove that $c_1\chi_{E_1} + ... + c_n\chi_{E_n}$ is an $\mathcal{S}$-measurable function if and only if $E_1, ..., E_n \in \mathcal{S}$.\\
\end{exercise}

\begin{solution}
    \\ ($\implies$) Suppose that $c_1\chi_{E_1} + ... + c_n\chi_{E_n}$ is $\mathcal{S}$-measurable, then for all borel sets $B$,
    $$(c_1\chi_{E_1} + ... + c_n\chi_{E_n})^{-1}(B)\in\mathcal{S}$$
    Hence, for all $k \in \Iint{1}{n}$, since $\{c_k\}$ is a Borel set, then 
    $$(c_1\chi_{E_1} + ... + c_n\chi_{E_n})^{-1}(\{c_k\})\in\mathcal{S}$$
    But notice that
    $$(c_1\chi_{E_1} + ... + c_n\chi_{E_n})^{-1}(\{c_k\}) = E_k$$
    since the $E_i$'s are disjoint and the $c_i$'s are distinct. It follows that $E_1, ..., E_n \in \mathcal{S}$.\\
    $( \ \Longleftarrow \ )$ Suppose that $E_1, ..., E_n \in \mathcal{S}$, then for all $k \in \Iint{1}{n}$, the function $\chi_{E_k}$ is $\mathcal{S}$-measurable. Moreover, for all $k \in \Iint{1}{n}$, since $g_k : x \mapsto c_k x$ is continuous, then it is Borel measurable. It follows that $c_k \chi_{E_k} = g_k \circ \chi_{E_k}$ is $\mathcal{S}$-measurable. Since measurable functions are closed under addition, then $c_1\chi_{E_1} + ... + c_n\chi_{E_n}$ is $\mathcal{S}$-measurable. \\
\end{solution}

\begin{exercise}
    \begin{enumerate}[label = (\alph*)]
        \item Suppose $f_1, f_2, ...$ is a sequence of functions from a set $X$ to $\R$. Explain why 
        \begin{align*}
            &\{x \in X : \text{ the sequence } f_1(x), f_2(x), ... \text{ has a limit in } \R \} \\
            &= \bigcap_{n=1}^{\infty}\bigcup_{j=1}^{\infty}\bigcap_{k=j}^{\infty}(f_j - f_k)^{-1}((- \tfrac{1}{n}, \tfrac{1}{n})).
        \end{align*}
        \item Suppose $(X, \mathcal{S})$ is a measurable space and $f_1, f_2, ...$ is a sequence of $\mathcal{S}$-measurable functions from $X$ to $\R$. Prove that
        $$\{x \in X : \text{ the sequence } f_1(x), f_2(x), ... \text{ has a limit in } \R \}$$
        is an $\mathcal{S}$-measurable subset of $X$.\\
    \end{enumerate}
\end{exercise}
 
\begin{solution}
    \begin{enumerate}[label = (\alph*)]
        \item First, to make it easier to read, denote by $E$ the set $$\{x \in X : \text{ the sequence } f_1(x), f_2(x), ... \text{ has a limit in } \R \}$$
        Let $x \in E$ be arbitrary, then by definition, the sequence $\{f_n(x)\}_n$ is a convergent sequence in $\R$. It follows that $\{f_n(x)\}_n$ is a Cauchy sequence. Let $n \in \N$, since $\frac{1}{n} > 0$, then there is a $j \in \N$ such that 
        $$|f_a(x) - f_b(x)| < \frac{1}{n}$$
        for all $a,b \geq j$. In particular, for all $k \geq j$, we have
        $$|f_j(x) - f_k(x)| < \frac{1}{n}$$
        Notice that this can be written as
        $$\frac{1}{n} < (f_j - f_k)(x) < \frac{1}{n}$$
        which again can be written as
        $$x \in (f_j - f_k)^{-1}((-\tfrac{1}{n}, \tfrac{1}{n}))$$
        Since it holds for all $j \geq k$, then
        $$x \in \bigcap_{j = k}^{\infty}(f_j - f_k)^{-1}((-\tfrac{1}{n}, \tfrac{1}{n}))$$
        Since there is a $k \in \N$ such that it holds, then
        $$x \in \bigcup_{k=1}^{\infty} \bigcap_{j = k}^{\infty}(f_j - f_k)^{-1}((-\tfrac{1}{n}, \tfrac{1}{n}))$$
        Since it holds for all $n \in \N$, then
        $$x \in \bigcap_{n=1}^{\infty}\bigcup_{j=1}^{\infty}\bigcap_{k=j}^{\infty}(f_j - f_k)^{-1}((- \tfrac{1}{n}, \tfrac{1}{n}))$$
        It follows that $E \subset \bigcap_{n=1}^{\infty}\bigcup_{j=1}^{\infty}\bigcap_{k=j}^{\infty}(f_j - f_k)^{-1}((- \tfrac{1}{n}, \tfrac{1}{n}))$. \\
        Now, for the reverse inclusion, suppose that
        $$x \in \bigcap_{n=1}^{\infty}\bigcup_{j=1}^{\infty}\bigcap_{k=j}^{\infty}(f_j - f_k)^{-1}((- \tfrac{1}{n}, \tfrac{1}{n}))$$
        Let's prove that $\{f_n(x)\}_n$ is a Cauchy sequence. Let $\epsilon > 0$, then by the Archimedean Property, there is an integer $N \in \N$ such that $\frac{1}{N} < \epsilon$. By our assumption on $x$, it follows that
        $$x \in \bigcup_{j=1}^{\infty}\bigcap_{k=j}^{\infty}(f_j - f_k)^{-1}((- \tfrac{1}{2N}, \tfrac{1}{2N}))$$
        But it means that there is a $j \in \N$ such that
        $$x \in \bigcap_{k=j}^{\infty}(f_j - f_k)^{-1}((- \tfrac{1}{2N}, \tfrac{1}{2N}))$$
        Let $r, s \geq j$, then the previous statement about $x$, it implies that
        $$x \in (f_j - f_r)^{-1}((- \tfrac{1}{2N}, \tfrac{1}{2N}))$$
        and
        $$x \in (f_j - f_s)^{-1}((- \tfrac{1}{2N}, \tfrac{1}{2N}))$$
        which are both equivalent to 
        $$|f_j(x) - f_r(x)| < \frac{1}{2N}$$
        $$|f_j(x) - f_s(x)| < \frac{1}{2N}$$
        By the triangle inequality, this gives us 
        $$|f_r(x) - f_s(x)| < \frac{1}{N} < \epsilon$$
        Thus, $\{f_n(x)\}_n$ is a Cauchy sequence and by completeness of $\R$, we get that the sequence $\{f_n(x)\}_n$ converges in $\R$.
        Therefore, 
        $$E = \bigcap_{n=1}^{\infty}\bigcup_{j=1}^{\infty}\bigcap_{k=j}^{\infty}(f_j - f_k)^{-1}((- \tfrac{1}{n}, \tfrac{1}{n}))$$

        \item Since for all $n,j \in \N$ and $k \geq j$, the function $f_j - f_k$ is $\mathcal{S}$-measurable, then 
        $$(f_j - f_k)^{-1}((-\tfrac{1}{n}, \tfrac{1}{n})) \in \mathcal{S}$$
        since $(-\frac{1}{n}, \frac{1}{n})$ is a Borel set. Since it holds for all $k \geq j$, then 
        $$\bigcap_{k=j}^{\infty}(f_j - f_k)^{-1}((-\tfrac{1}{n}, \tfrac{1}{n})) \in \mathcal{S}$$
        Similarly, since it holds for all $j \in \N$, then
        $$\bigcup_{j=1}^{\infty}\bigcap_{k=j}^{\infty}(f_j - f_k)^{-1}((-\tfrac{1}{n}, \tfrac{1}{n})) \in \mathcal{S}$$
        Again, since it holds for all $n \in \N$, then
        $$\bigcap_{n=1}^{\infty}\bigcup_{j=1}^{\infty}\bigcap_{k=j}^{\infty}(f_j - f_k)^{-1}((-\tfrac{1}{n}, \tfrac{1}{n})) \in \mathcal{S}$$
        which proves our claim. \\
    \end{enumerate}
\end{solution}

\begin{exercise}
    Suppose $X$ is a set and $E_1, E_2, ...$ is a disjoint sequence of subsets of $X$ such that $\cup_{i=1}^{\infty}E_i = X$. Let $\mathcal{S} = \{\cup_{k \in K}E_k : K \subset \N \}$.
    \begin{enumerate}[label = (\alph*)]
        \item Show that $\mathcal{S}$ is a $\sigma$-algebra on $X$.
        \item Prove that a function from $X$ to $\R$ is $\mathcal{S}$-measurable if and only if the function is constant on $E_k$ for every $k \in \N$.\\
    \end{enumerate}
\end{exercise}

\begin{solution}
    \begin{enumerate}[label = (\alph*)]
        \item Since this statement is a generalization of Exercise 1, then the proof will be very similar. As most of the proofs showing that a collection is a $\sigma$-algebra, let's split this one into three parts:
        \begin{itemize}
            \item ($\varnothing \in \mathcal{S}$) Since $\varnothing \subset \Z$, then $\cup_{k \in \varnothing}E_k \in \mathcal{S}$. However, notice that $\cup_{k \in \varnothing}E_k = \varnothing$. It follows that $\varnothing \in \mathcal{S}$.
            \item (closed under complements) Let $A \in \mathcal{S}$, then there exists a $K_0 \subset \Z$ such that $A = \cup_{k \in K_0}E_k$
            Consider $K_1 = \Z \setminus K_0$ and its associated element $B = \cup_{k \in K_1}E_k$ in $\mathcal{S}$. Since $A \cap B = \varnothing$ and $A \cup B = X$, then $B = X \setminus A$. Hence, $A^c \in \mathcal{S}$ which proves that $\mathcal{S}$ is closed under complements.
            \item (closed under countable union) Let $\{A_i\}_i$ be a countable collection of elements in $\mathcal{S}$, then for all $i \in \N$, there is a subset $K_i$ of $\Z$ such that $A_i = \cup_{k \in K_i}E_k$. Consider $K = \cup_{i=1}^{\infty}K_i \subset \Z$ and $A = \cup_{k \in K}E_k \in \mathcal{S}$. By consruction, $A = \cup_{i=1}^{\infty}A_i \in \mathcal{S}$. Therefore, $\mathcal{S}$ is closed under countable union.
        \end{itemize}
        Therefore, $\mathcal{S}$ is a $\sigma$-algebra on $X$.
        \item Let $f : X \to \R$, suppose first that $f$ is constant on $E_k$ for every $k \in \N$. Call $c_k$ the constant value of $f$ on $E_k$. To show that $f$ is $\mathcal{S}$-measurable, let $B \subset \R$ be a Borel set. Notice that
        $$f^{-1}(B) = f^{-1}(B \cap \{c_k\}_k) = \bigcup_{k; c_k \in B}E_k  \in \mathcal{S}$$
        It follows that $f$ is $\mathcal{S}$-measurable. \\
        Suppose now that $f$ is not constant on all $E_k$'s, then, there is a $k_0 \in \N$ and distinct real noumbers $a$ and $b$ such that both $a,b \in f(E_{k_0})$. What we get is that $f^{-1}(\{a\}) \subsetneq E_{k_0}$. Hence, since that $E_k$'s are disjoint, then we cannot write $f^{-1}(\{a\})$ as a union of $E_k$'s. Thus, $f^{-1}(\{a\}) \notin \mathcal{S}$ even if $\{a\}$ is a Borel set. It follows that $f$ is not $\mathcal{S}$-measurable. \\
    \end{enumerate}
\end{solution}

\begin{exercise}
    Suppose $\mathcal{S}$ is a $\sigma$-algebra on a set $X$ and $A \subset X$. Let
    $$\mathcal{S}_A = \{E \in \mathcal{S} : A \subset E \text{ or } A\cap E = \varnothing\}$$
    \begin{enumerate}[label = (\alph*)]
        \item Prove that $\mathcal{S}_A$ is a $\sigma$-algebra on $X$.
        \item Suppose $f : X \to \R$ is a function. Prove that $f$ is $\mathcal{S}$-measurable if and only if $f$ is measurable with respect to $\mathcal{S}$ and $f$ is constant on $A$.\\
    \end{enumerate}
\end{exercise}

\begin{solution}
     \begin{enumerate}[label = (\alph*)]
        \item First, since $\varnothing \in \mathcal{S}$ and $\varnothing \subset A$, then $\varnothing \in \mathcal{S}_A$. Now, take an arbitrary set $E$ in $\mathcal{S}_A$, then we either have $A \subset E$ or $A \cap E = \varnothing$:
        \begin{itemize}
            \item If $A \subset E$, then $X \setminus E \subset X \setminus A$. It follows that $(X \setminus E) \cap A = \varnothing$. But since $X \setminus E \in \mathcal{S}$, then $X \setminus E \in \mathcal{S}_A$.
            \item If $A \cap E = \varnothing$, then for all $x \in A$, having $x \in E$ would lead to $x \in A \cap E \neq \varnothing$ which is a contradiction. Hence, $x \in X \setminus E$. Hence, $A \subset X \setminus E$. But since $X \setminus E \in \mathcal{S}$, then $X \setminus E \in \mathcal{S}_A$.
        \end{itemize}
        In all cases, we get that $E^c \in \mathcal{S}_A$. Now, let $\{E_i\}_i$ be a countable collection of elements in $\mathcal{S}_A$, let's show that $\cup_{i=1}^{\infty}E_i \in \mathcal{S}_A$ by cases. If each $E_i$ satisfy $A \cap E_i = \varnothing$, then we must have $A\cap\cup_{i=1}^{\infty}E_i = \varnothing$. Since $\cup_{i=1}^{\infty}E_i \in \mathcal{S}$, then we get that $\cup_{i=1}^{\infty}E_i \in \mathcal{S}_A$. However, if one of the $E_i$'s satisfies $A \subset E_i$, then we get
        $$A \subset E_i \subset \bigcup_{i=1}^{\infty}E_i$$
        Thus, in both cases, it follows that $\cup_{i=1}^{\infty}E_i \in \mathcal{S}_A$. Therefore, $\mathcal{S}_A$ is a $\sigma$-algebra.
        \item ($\implies$) Suppose that $f$ is $\mathcal{S}_A$-measurable and let $B$ be a Borel subset of $\R$, then
        $$f^{-1}(B) \in \mathcal{S}_A \subset \mathcal{S}$$
        which proves that $f$ is $\mathcal{S}$-measurable. Suppose that $f$ is nonconstant on $A$, then there exist $x_0, x_1 \in A$ such that $f(x_0) \neq f(x_1)$. Consider $f^{-1}(\{f(x_0)\})$, then by our assumption, it is contained in $\mathcal{S}_A$ (since $\{f(x_0)\} is a Borel set$). Hence, it means that one of $A \subset f^{-1}(\{f(x_0)\})$ or $A\cap f^{-1}(\{f(x_0)\}) = \varnothing$ holds. But $A \subset f^{-1}(\{f(x_0)\})$ cannot hold since $x_1 \in A$ and $x_1 \notin f^{-1}(\{f(x_0)\})$. Similarly, for the same reason, $A\cap f^{-1}(\{f(x_0)\}) = \varnothing$ cannot hold as well. From this contradiction, we get that $f$ must be constant on $A$. \\
        $ ( \ \Longleftarrow \ ) $  Suppose that $f$ is $\mathcal{S}$-measurable and $f \equiv c$ on $A$ for some $c \in \R$. Let $B \subset \R$ be a Borel set, then, by our assumption, $f^{-1}(B) \in \mathcal{S}$. Now, notice that we either have $c \in B$ or $c \notin B$. If $c \in B$, then it follows that $A \subset f^{-1}(B)$. In that case, $f^{-1}(B) \in \mathcal{S}_A$. If $c \notin B$, then no elements of $A$ are in $f^{-1}(B)$. Hence, $A \cap f^{-1}(B) = \varnothing$. Again, in that case, $f^{-1}(B) \in \mathcal{S}_A$. Therefore, $f$ is $\mathcal{S}_A$-measurable. \\
     \end{enumerate}
\end{solution}

\begin{exercise}
    Suppose $X$ is a Borel subset of $\R$ and $f : X \to \R$ is a function such that $\{x \in X : f \text{ not continuous at } x\}$ is a countable set. Prove that $f$ is a Borel measurable function. \\
\end{exercise}

\begin{solution}
    \\ Fix $a \in \R$ and let's show that $f^{-1}((a, \infty))$ is a Borel set. Let $x \in X$, then $x \in f^{-1}((a, \infty))$ if and only if $f(x) > a$. If $f$ is continuous at $x$, then there is a $\delta_x > 0$ such that 
    $$f((x - \delta_x, x + \delta_x)\cap X) \subset (a, \infty)$$
    which can also be written as 
    $$(x - \delta_x, x + \delta_x)\cap X \subset f^{-1}((a, \infty))$$
    It follows that 
    $$f^{-1}((a, \infty)) = \left[ X \cap \bigcup_{\substack{x \ : \ f \text{ is conti-} \\ \text{nuous at } x }}(x - \delta_x, x + \delta_x)\right] \cup \{x \in X : f \text{ not continuous at } x\}$$
    But notice that $\cup_{x}(x - \delta_x, x + \delta_x)$ is an open set so it is also a Borel set. Since $X$ is Borel as well, then $X \cap \cup_{x}(x - \delta_x, x + \delta_x)$ is a Borel set. It follows that $f^{-1}((a, \infty))$ is a Borel since any countable set is a Borel set. Therefore, $f$ is Borel measurable.\\
\end{solution}

\begin{exercise}
    Suppose $f : \R \to \R$ is differentiable at every element of $\R$. Prove that $f'$ is a Borel measurable function from $\R$ to $\R$. \\
\end{exercise}

\begin{solution}
    \\ Consider the sequence $f_1, f_2, ...$ of functions defined as follows:
    $$f_n(x) = \frac{f\left(x + \frac{1}{n}\right) - f(x)}{\frac{1}{n}} $$
    for all $x \in \R$ and $n \in \N$. Since $f$ is differentiable, then $f$ is continuous. It follows that each $f_n$ is continuous as well and consequently, Borel measurable. Now, notice that for all $x \in \R$,
    $$\lim_{n \rightarrow \infty}f_n(x) = \lim_{n \rightarrow \infty} \frac{f\left(x + \frac{1}{n}\right) - f(x)}{\frac{1}{n}} = f'(x)$$
    Thus, $f'$ must be Borel measurable as well. \\
\end{solution}

\begin{exercise}
    Suppose $X$ is a nonempty set and $\mathcal{S}$ is the $\sigma$-algebra on $X$ consisting of all subsets of $X$ that are either countable or have a countable complement in $X$. Give a characterization of the $\mathcal{S}$-measurable real-valued functions on $X$.\\
\end{exercise}

\begin{solution}
    \\ In this proof, I will show that that the $\mathcal{S}$-measurable functions are precisely the functions that are constant except on a countable set. Notice that the case where $X$ is finite or countable is easy to prove since in that case, $\mathcal{S} = 2^X$ and hence, every function is $\mathcal{S}$-measurable. Moreover, every function is constant except on a countable set. Hence, the characterization is proved in that case. \\
    Suppose that $X$ is uncountable. Let $f : X \to \R$ be a function such that $f$ is constant except on a countable set. It follows that there is a $x_0 \in X$ such that $f(x) = f(x_0)$ for all $x \in X$ except for countably many $x$. Let $a \in \R$ and consider the set $f^{-1}((a, \infty))$. If $a \geq f(x_0)$, then for all $x \in X$,
    \begin{align*}
        x \in f^{-1}((a, \infty)) &\implies f(x) > a \\
        &\implies f(x) > f(x_0) \\
        &\implies f(x) \neq f(x_0) \\
        &\implies x \in \{x \in X : f(x) \neq f(x_0)\}
    \end{align*}
    Hence, 
    $$f^{-1}((a, \infty)) \subset \{x \in X : f(x) \neq f(x_0)\}$$
    which is countable. It follows that $f^{-1}((a, \infty)) \in \mathcal{S}$. If $a < f(x_0)$, then similarly, for all $x \in X$,
    \begin{align*}
        x \in f^{-1}((a, \infty))^c &\implies f(x) \leq a \\
        &\implies f(x) < f(x_0) \\
        &\implies x \in \{x \in X : f(x) \neq f(x_0)\}
    \end{align*}
    which implies
    $$f^{-1}((a, \infty))^c \subset \{x \in X : f(x) \neq f(x_0)\}$$
    Hence, $f^{-1}((a, \infty))^c$ is countable so $f^{-1}((a, \infty)) \in \mathcal{S}$. Therefore, since it holds for all cases and for all $a \in \R$, it follows that $f$ is $\mathcal{S}$-measurable. \\
    For the converse, consider an $\mathcal{S}$-measurable function $f$ and let's show that it is constant on a countable set. Define the sets
    $$A = \{a \in \R : f^{-1}((a, \infty)) \text{ is countable}\}$$
    $$B = \{b \in \R : f^{-1}((-\infty, b]) \text{ is countable}\}$$
    By the assumption that $f$ is measurable, we must have $A \cup B = \R$. Moreover, if $x \in A \cap B$, then both $f^{-1}((x, \infty))$ and $f^{-1}((-\infty, x])$ are countable. However,
    $$X = f^{-1}((x, \infty)) \cup f^{-1}((-\infty, x])$$
    which would imply that $X$ is countable. A contradiction since we assumed that $X$ is uncountable. Hence, $A \cap B = \varnothing$. Now, notice that both $A$ and $B$ are nonempty. By contradiction, if $B = \varnothing$, then $A = \R$. Hence, for all $n \in \N$, we get that $f^{-1}((-n, \infty))$ is countable. However, since
    $$X = \bigcup_{n=1}^{\infty}f^{-1}((-n, \infty))$$
    then it would imply that $X$ is countable. A contradiction that shows that $B$ is nonempty. The proof for $A$ is the same. The last two important properties of $A$ and $B$ are the following, if $a \in A$ and $a'$ is a really number greater than $a$, then $a' \in A$. Similarly, if $b \in B$ and $b'$ is a real number smaller than $b$, then $b' \in B$. Let's prove it for $A$ only (the proof is the same for $B$). Let $a \in A$ and $a' \geq a$, then 
    $$(a', \infty) \subset (a, \infty)$$
    which implies 
    $$f^{-1}((a', \infty)) \subset f^{-1}((a, \infty))$$
    But $f^{-1}((a, \infty))$ is countable so $f^{-1}((a', \infty))$ is countable as well. It follows that $a' \in A$. \\
    All of these properties of $A$ and $B$ show that $A$ is nonempty and bounded below by any element of $B$ and $B$ is nonempty and bounded above by any element of $A$. Moreover, since $A \cup B = \R$, then $\sup B = \inf A$. Define $c = \sup B$, then we either have
    $$B = (-\infty, c) \qquad \qquad A = [c, \infty)$$
    or
    $$B = (-\infty, c] \qquad \qquad A = (c, \infty)$$
    In both cases, we have
    $$(- \infty, c) \subset B \quad \text{ and } \quad (c, \infty) \subset A$$
    Thus, for all $n \in \N$, we have $c - \frac{1}{n} \in B$ and $c + \frac{1}{n} \in A$. It follows that
    \begin{align*}
        f^{-1}(\R \setminus c) &= f^{-1}((-\infty, c) \cup (c, \infty)) \\
        &= f^{-1}((-\infty, c)) \cup f^{-1}((c, \infty)) \\
        &= f^{-1}\left( \bigcup_{n=1}^{\infty}\biggl( (-\infty, c-\frac{1}{n} \biggr] \right) \cup f^{-1}\left( \bigcup_{n=1}^{\infty}\left(c+\frac{1}{n} , \infty\right) \right) \\
        &= \bigcup_{n=1}^{\infty} f^{-1}\left( \biggl(-\infty, c-\frac{1}{n} \biggr] \right) \cup \bigcup_{n=1}^{\infty} f^{-1}\left(\left(c+\frac{1}{n} , \infty\right) \right) 
    \end{align*}
    which shows that $f^{-1}(\R \setminus c)$ is countable since it the union of two countable unions of countable sets. Thus, it means that $f(x) \neq c$ only for countably many $x \in X$. Therefore, $f$ is constant except on a countable set. \\ 
\end{solution}

\begin{exercise}
    Suppose $(X, \mathcal{S})$ is a measurable space and $f,g : X \to \R$ are $\mathcal{S}$-measurable functions. Prove that if $f(x) > 0$ for all $x \in X$, then $f^g$ (which is the function whose value at $x\in X$ equals $f(x)^{g(x)}$) is an $\mathcal{S}$-measurable function. \\
\end{exercise}

\begin{solution}
    \\ First, recall that both functions
    $$\ln : (0, \infty) \to \R$$
    $$\exp : \R \to [0, \infty)$$
    are continuous on their respective domains (which are Borel sets). Hence, both function of Borel measurable. Since $\Im(f) \subset (0, \infty)$, then $\ln \circ f$ is $\mathcal{S}$-measurable. Since $g$ and $\ln \circ f$ are both $\mathcal{S}$-measurable functions from $X$ to $\R$, then $g \cdot (\ln \circ f)$ is $\mathcal{S}$-measurable. Again, since $\Im(g \cdot (\ln \circ f)) \subset \R$ and $\exp$ is $\mathcal{S}$-measurable, then $f^g = \exp \circ (g \cdot (\ln \circ f))$ is $\mathcal{S}$-measurable. \\
\end{solution}

\begin{exercise}
    Prove 2.52. \\
\end{exercise}

\begin{solution}
    \\ Suppose $(X, \mathcal{S})$ is a measurable space and $f : X \to [-\infty, \infty]$ is a function such that 
    $$f^{-1}((a, \infty]) \in \mathcal{S}$$
    for all $a \in \R$. Let's show that $f$ is $\mathcal{S}$-measurable. To do so, define the collection
    $$T = \{A \subset [-\infty, \infty] : f^{-1}(A) \in \mathcal{S} \}$$
    By properties of the inverse image of $f$, the collection $T$ is a $\sigma$-algebra. Moreover, by our assumption on $f$, $\{(a, \infty] : a \in \R\} \subset T$. Let $(a, b)$ be an arbitrary open interval, since $(a,b) = (a, \infty] \setminus (b, \infty]$ and $T$ is closed under set differences, then $(a,b) \in T$. Since it holds for all open intervals, then $T$ contains every open interval. Since every open set can be written as a countable union of open intervals, then $T$ contains every open set. Since $T$ is a $\sigma$-algebra that contains every open set, then it must contain every Borel subsets of $\R$. Now, since 
    $$\{ \infty \} = \bigcap_{n=1}^{\infty}(n, \infty] \in T$$
    and
    $$\{ -\infty \} = \bigcap_{n=1}^{\infty}[-\infty, n] \in T$$
    Then every Borel subset of $[-\infty, \infty]$ is contained in $T$. Therefore, for all Borel subset $B$ of $[-\infty, \infty]$,
    $$f^{-1}(B) \in \mathcal{S}$$
    It follows that $f$ is $\mathcal{S}$-measurable.\\
\end{solution}

\begin{exercise}
    Suppose $B \subset \R$ and $f : B \to \R$ is an increasing function. Prove that $f$ is continuous at every element of $B$ except for a countable subset of $B$. \\
\end{exercise}

\begin{solution}
    \\ First, let's prove that every discontinuity is a jump discontinuity by showing that the left and right limits at a point always exist. Let $x_0 \in B$ and notice that if there is no $x \in B$ such that $x < x_0$, then it follows that $\lim_{x \rightarrow x_0^-}f(x) = f(x_0)$. Hence, suppose that there some points $x \in B$ such that $x < x_0$ and consider the set $E_x^- = \{f(x) : x < x_0\}$. By our assumptions, the set $E_x^-$ is nonempty and bounded above by $f(x_0)$. Hence, by completeness of $\R$, we can define $s = \sup E_x^-$. Moreover, by properties of the supremum, it is easy to see that 
    $s = \lim_{x \rightarrow x_0^-}f(x)$. Thus, for all $x_0 \in B$, its left limit exists. Similarly, if we define the set $E_x^+ = \{f(x) : x > x_0\}$ and take its infimimum, we get that $ \lim_{x \rightarrow x_0^+}f(x)$ exists as well. Hence, for all $x \in B$, to make the notation lighter, 
    $$f(x-) = \lim_{x \rightarrow x_0^-}f(x)$$
    $$f(x+) = \lim_{x \rightarrow x_0^+}f(x)$$
    Since $f$ is increasing, then for all $x \in B$, we have that $f(x)$ is an upper bound for $E_x^-$ and a lower bound for $E_x^+$. It follows that $f(x-) \leq f(x+)$ for all $x \in B$.
    \\ Consider now the set $D_f$ of discontinuity points of $f$. Notice that for all $d \in D_f$, we can define the sets 
    $$F_d^- = \{x \in B : x < d\} = (- \infty, d) \cap B$$
    $$F_d^+ = \{x \in B : x > d\} = (d,\infty) \cap B$$
    Notice that there are at most two points such that one of $F_d^-$ and $F_d^+$ is empty. Let $A$ denote the set of such points, then our observation can be translated by $\card(A) \leq 2$. Now define $B = \{d \in D_f : \sup F_d^- = \inf F_d^+ = d\}$ and $C = D_f \setminus (A \cup B)$. By construction, we have $A \cup B \cup C = D_f$. Since $A$ is finite, our goal now is to prove that both $B$ and $C$ are countable.\\

    For all $d \in B$, we must have $\sup F_d^- < d$ or $\inf F_d^+ > d$. If $\sup F_d^- < d$, then define $q_d$ to be any rational inside the interval $(m_d, d)$ where $m_d$ is the midpoint between $\sup F_d^-$ and $d$ (i.e. $m_d = \frac{\sup F_d^- + d}{2}$). Otherwise, if $\inf F_d^+ > d$, define $q_d$ to be any rational inside the interval $(d, m_d')$ where $m_d'$ is the midpoint between $\inf F_d^+$ and $d$ (i.e. $m_d' = \frac{\inf F_d^+ + d}{2}$). Now, let's prove that the function $g : B \to \Q$ defined by $d \mapsto q_d$ is injective. Let $d_1, d_2 \in B$ such that $d_1 < d_2$, let's show that $q_{d_1} \neq q_{d_2}$ by cases.
    \begin{itemize}
        \item If $\sup F_{d_1}^- < d_1$ and $\sup F_{d_2}^- < d_2$, then $q_{d_1}$ and $q_{d_2}$ are rationals between $(m_{d_1}, d_1)$ and $(m_{d_2}, d_2)$ respectively. It follows that $q_{d_1} < d_1$. Moreover, $\sup F_{d_2}^- < m_{d_2} < q_{d_2}$ and $d_1 \in F_{d_2}^-$ so 
        $$q_{d_1} < d_1 \leq \sup F_{d_2}^- < d_2$$
        which shows that $q_{d_1} \neq q_{d_2}$.
        \item If $\sup F_{d_1}^- < d_1$ and $\inf F_{d_2}^+ > d_2$, then $q_{d_1}$ is a rational inside the interval $(m_{d_1}, d_1)$ and $q_{d_2}$ is a rational inside the interval $(d_2,m_{d_1}')$. Hence, $q_{d_1} < d_1$ and $d_2  < q_{d_2}$. It follows that
        $$q_{d_1} < d_1 < d_2  < q_{d_2}$$ 
        which shows that $q_{d_1} \neq q_{d_2}$.
        \item If $\inf F_{d_1}^+ > d_1$ and $\sup F_{d_2}^- < d_2$, then $q_{d_1}$ is a rational inside the interval $(d_1, m_{d_1}')$ and $q_{d_2}$ is a rational inside the interval $(m_{d_2}, d_2)$. Hence, $q_{d_1} < m_{d_1}'$ and $m_{d_2} < d_2$. Since $d_1 \in F_{d_2}^-$, then $d_1 \leq \sup F_{d_2}^-$, similarly, $\inf F_{d_1}^+ \leq d_2$. Hence:
        $$q_{d_1} < m_{d_1}' = \frac{\inf F_{d_1}^+ + d_1}{2} \leq \frac{d_2 + d_1}{2} \leq \frac{\sup F_{d_2}^- + d_2}{2} = m_{d_2} < q_{d_2}$$
        which shows that $q_{d_1} \neq q_{d_2}$.
        \item If $\inf F_{d_1}^+ > d_1$ and $\inf F_{d_2}^+ > d_2$, then $q_{d_1}$ is a rational inside the interval $(d_1, m_{d_1}')$ and $q_{d_2}$ is a rational inside the interval $(d_2, m_{d_2}')$. It follows that $q_{d_1} < m_{d_1}'$ and $d_2 < q_{d_2}$. Since $d_2 \in F_{d_1}^+$, then
        $$q_{d_2} < m_{d_1}' < \inf F_{d_1}^+ \leq d_2 < q_{d_2}$$
        which shows that $q_{d_1} \neq q_{d_2}$.
    \end{itemize}
    Therefore, $g$ is injective so $\card B \leq \card \Q$ which implies that $B$ is countable.\\
    Let's now prove that $C$ is countable. Let $d \in C$. Since $f$ is continuous at $d$ if and only if $f(d-) = f(d+)$, then $f$ is discontinuous at $d$ if and only if $f(d-) < f(d+)$. It follows that $f(d-) < f(d+)$. Hence, define $q_d$ to be any rational in the nonempty interval $(f(d-), f(d+))$. Let's prove that the function $h : C \to \Q$ defined by $d \mapsto q_d$ is injective. Let $d_1, d_2 \in C$ such that $d_1 < d_2$. Hence, by definition of $C$, we must have $d_1 = \inf F_{d_1}^+$. It follows that there must be a $x \in B$ such that $d_1 < x < d_2$. Hence, by monotonicity of $f$, we have $f(d_1) \leq f(x) \leq f(d_2)$. By definition of $E_{d_1}^+$ and $E_{d_2}^-$, we have
    $$f(d_1+) = \inf E_{d_1}^+ \leq f(x) \leq \sup E_{d_2}^- = f(d_2-)$$
    Therefore, 
    $$q_{d_1} < f(d_1+) \leq f(d_2-) < q_{d_2}$$
    which shows that $q_{d_1} \neq q_{d_2}$. Hence, $h$ is injective so $\card C \leq \card \Q$. Thus, $C$ is countable. Since $D_f = A \cup B \cup C$ and $A$, $B$, and $C$ are countable, then $D_f$ must be countable as well. Therefore, $f$ is continuous except on the countable set $D_f$.\\
\end{solution}

\begin{exercise}
    Suppose $f : \R \to \R$ is a strictly increasing function. Prove that the inverse function $f^{-1} : f(\R) \to \R$ is a continuous function. \\
    $[$\textit{Note that this exercise does not have as a hypothesis that $f$ is continuous.}$]$\\
\end{exercise}

\begin{solution}
    \\ Let $x_0 \in f(\R)$ and let's prove that
    $$\lim_{x \rightarrow x_0}f^{-1}(x) = f^{-1}(x_0)$$
    Let $\epsilon > 0$ and define $x_1 = f^{-1}(x_0) - \epsilon$ and $x_2 = f^{-1}(x_0) + \epsilon$ which are both in the domain of $f^{-1}$. Notice that $x_1 < x_0 < x_2$. Let $\delta = \min(x_0 - x_1, x_2 - x_0)$ and let $x \in f(\R)$ such that $0 < |x - x_0| < \delta$, hence:
    \begin{align*}
        x_1 < x < x_2 &\implies f(f^{-1}(x_0) - \epsilon) < x < f(f^{-1}(x_0) + \epsilon) \\
        &\implies f^{-1}(x_0) - \epsilon < f^{-1}(x) < f^{-1}(x_0) + \epsilon \\
        &\implies |f^{-1}(x) - f^{-1}(x_0)| \epsilon
    \end{align*}
    So $\lim_{x \rightarrow x_0}f^{-1}(x) = f^{-1}(x_0)$ which implies that $f^{-1}$ is continuous at $x_0$. Therefore, $f^{-1}$ is continuous on $f(\R)$.\\
\end{solution}

\begin{exercise}
    Suppose $B \subset \R$ is a Borel set and $f : \R \to \R$ is a strictly increasing function. Prove that $f(B)$ is a Borel set. \\
\end{exercise}

\begin{solution}
    \\ First, we show that $f(\R)$ is a Borel set. Since $f$ has countably many discontinuities, then we can enumerate them as $d_1, d_2, ...$. Let $i \in \N$, define 
    $$D_i = [f(d_i -), f(d_i + )] \setminus \{f(d_i)\}$$
    which represents the set of points not attained by $f$ because of $d_i$. In this proof, I denote by $f(x-)$ the limit of $f$ as $z \rightarrow x$ from the left and $f(x+)$ the limit of $f$ as $z \rightarrow x$ from the right. Notice that each $D_i$ is a Borel set so $\cup_{i=1}^{\infty}D_i$ is a Borel set as well. Define now
    $$I_m = (-\infty, \inf_{\R}f]$$
    $$I_M = [\sup_{\R}f, \infty)$$
    and consider that $I_m = \varnothing$ if $f$ is unbounded below and $I_M = \varnothing$ if $f$ is unbounded above. Again, in all cases, both $I_m$ and $I_M$ are Borel sets. It follows that the set
    $$I = \R \setminus \left[I_m \cup I_M \cup \bigcup_{i=1}^{\infty}D_i\right]$$
    is a Borel set. The main goal of this proof is to show that $f(\R) = I$.
    \begin{itemize}
        \item ($\implies$) Let $y \in I_m \cup I_M \cup \bigcup_{i=1}^{\infty}D_i$. Let's prove by cases that $y \in \R \setminus f(\R)$.
        \begin{itemize}
            \item If $y \in I_m$, then $y \leq \inf_{\R}f$. If $y = f(x_0)$ for some $x_0 \in \R$, then strict monotonicity of $f$, we have:
            $$\inf_{\R}f \leq f(x_0 - 1) < y \leq \inf_{\R}f$$
            A contradiction that shows that $y \in \R \setminus f(\R)$.
            \item If $y \in I_M$, the proof is the same as in the previous case.
            \item If $\cup_{i=1}^{\infty}D_i$, then there is a $i \in \N$ such that $y \in D_i = [f(d_i -), f(d_i + )] \setminus \{f(d_i)\}$. Suppose by contradiction that $y = f(x_0)$ for some $x_0 \in \R$. Since $y \neq f(d_i)$, then we either have $x_0 < d_i$ or $d_i < x_0$. Suppose without loss of generality that $x_0 < d_i$, then $y \in \{f(x) : x < d_i\}$ which implies that
            $$y \leq \sup \{f(x) : x < d_i\}$$
            But $y \in [f(d_i -), f(d_i + )] \setminus \{f(d_i)\}$ so $y \geq f(d_i-) = \sup \{f(x) : x < d_i\}$. It follows that $y = \sup \{f(x) : x < d_i\}$. However, since $x_0 < d_i$, then there exists a $x_1 \in (x_0, d_i)$. Since $x_0 <x_1$ and $x_1 < d_i$, then 
            $$\sup \{f(x) : x < d_i\} = y = f(x_0) < f(x_1) \leq \sup \{f(x) : x < d_i\}$$
            A contradiction that shows that $y \in \R \setminus f(\R)$.
        \end{itemize}
        Therefore, $I_m \cup I_M \cup \bigcup_{i=1}^{\infty}D_i \subset \R \setminus f(\R)$ which is equivalent to $f(\R) \subset I$.
        \item $( \ \Longleftarrow \ )$ Let's now prove the reverse inclusion. Let $y \in \R \setminus f(\R)$, if $y < f(x)$ for all $x \in \R$, then $y \in I_m \subset I_m \cup I_M \cup \bigcup_{i=1}^{\infty}D_i$. If $y > f(x)$ for all $x \in \R$, then $y \in I_M \subset I_m \cup I_M \cup \bigcup_{i=1}^{\infty}D_i$. Hence, we can suppose that there exist real numbers $x_1$ and $x_2$ such that $f(x_1) < y < f(x_2)$. Consider now the sets
        $$A_1 = \{x \in \R : f(x) < y\}$$
        $$A_2 = \{x \in \R : f(x) > y\}$$
        Since $x_1 \in A_1$ and $x_2 \in A_2$, then the sets are nonempty. Moreover, since $y \neq f(x)$ for all $x \in \R$, then $A_1 \cup A_2 = \R$. Since $f$ is strictly increasing, then any element of $A_2$ is an upper bound for $A_2$, it follows that there is a $c \in \R$ such that either
        $$A_1 = (-\infty, c] \qquad A_2 = (c, \infty)$$
        or 
        $$A_1 = (-\infty, c)\qquad A_2 = [c, \infty)$$
        Suppose without loss of generality that $f(c) < y$, then 
        $$A_1 = (-\infty, c] \qquad A_2 = (c, \infty)$$
        Let's show that $f$ is discontinuous at $c$. Since $y$ is a lower bound for the set $f(A_2)$, then $y \leq \inf f(A_2)$. However, since $f$ is increasing, notice that
        $$f(c+) = \inf \{f(x) : x > c\} = \inf f((c, \infty)) = \inf f(A_2)$$
        Thus,
        $$f(c) < y \leq \inf f(A_2) = f(c+)$$
        which shows that $f$ is discontinuous at $c$ (otherwise, we would have $f(c) = f(c+)$). Thus, there is a $i \in \N$ such that $c = d_i$. Hence, 
        $$y \in (f(d_i), f(d_i+)] \subset D_i \subset I_M \subset I_m \cup I_M \cup \bigcup_{i=1}^{\infty}D_i$$
        Therefore, $\R \setminus f(\R) \subset I_M \subset I_m \cup I_M \cup \bigcup_{i=1}^{\infty}D_i$ which is equivalent to $I \subset f(\R)$.
    \end{itemize}
    Now that we showed that $f(\R) = I$, it follows that $f(\R)$ is a Borel set. Using the previous exercice, we get that $f^{-1} : f(\R) \to \R$ is a continuous function, hence, Borel measurable. It follows that $(f^{-1})^{-1}(B)$ is a Borel set. However, since
    $$(f^{-1})^{-1}(B) = \{y \in f(\R) : f^{-1}(y) \in B\} = \{y \in f(\R) : y \in f(B)\} = f(B)$$
    then $f(B)$ is a Borel set. \\
\end{solution}

\begin{exercise}
    Suppose $B \subset \R$ and $f : B \to \R$ is an increasing function. Prove that there exists a sequence $f_1, f_2, ...$ of strictly increasing functions from $B$ to $\R$ such that
    $$f(x) = \lim_{k \rightarrow \infty}f_k(x)$$
    for every $x \in B$.\\
\end{exercise}

\begin{solution}
    \\ Consider the sequence of functions $f_1, f_2, ...$ defined by
    $$f_n(x) = f(x) + \frac{1}{n}x$$
    for all $x \in B$ and $n \in \N$. Since $f$ is increasing and $\frac{1}{n}x$ is strictly increasing for all $n \in \N$, then every function in the sequence is strictly increasing. Moreover, for all $x \in B$:
    $$\lim_{k \rightarrow \infty}f_k(x) = f(x) + x\lim_{k \rightarrow \infty}\frac{1}{k} = f(x)$$
    which proves our claim.\\
\end{solution}

\begin{exercise}
    Suppose $B \subset \R$ and $f : B \to \R$ is a bounded increasing function. Prove that there exists an increasing function $g : \R \to \R$ such that $g(x) = f(x)$ for all $x \in B$. \\
\end{exercise}

\begin{solution}
    \\ First, assume that $B$ is non empty. For all $x \in \R$, define the set 
    $$E_x = \{f(y) : y \leq x, \ y \in B \}$$
    Consider the function $g : \R \to \R$ defined by
    $$g(x) = \begin{cases}
        \sup E_x & \text{if } E_x \neq \varnothing \\
        \inf_{\R}f & \text{if  } E_x = \varnothing
    \end{cases}$$
    Notice that $g$ is well defined since $f$ is bounded. Let's show that $g|_B = f$. Take $x \in B$ and notice that $E_x$ is non empty since it contains $f(x)$. Moreover, since $f$ is increasing, then $f(x)$ is an upperbound for $E_x$. It follows that $g(x) = \sup E_x = f(x)$. \\
    Let's now show that $g$ is increasing. If we take $a,b \in \R$ such that $a < b$, then we can proceed by cases. 
    \begin{itemize}
        \item If both $a$ and $b$ are in $B$, then $g(a) \leq g(b)$ follows from the fact that $f$ is increasing :
        $$g(a) = f(a) \leq f(b) = g(b)$$

        \item If $a \in B$ but $b \notin B$, then $f(a) \in E_b \neq \varnothing$ which implies
        $$g(a) = f(a) \leq \sup E_b = g(b)$$
        
        \item If $a \notin B$ and $b \in B$, then either $E_a$ is empty and we get
         $$g(a) = \inf_{\R}f \leq f(b) = g(b)$$
        either $E_a$ is nonempty and we get that $f(b)$ is an upperbound for $E_a$. Hence:
        $$g(a) = \sup E_a \leq f(b) = g(b)$$
        
        \item If both $a$ and $b$ are not in $B$, then we, again, have different possible cases.
        \begin{itemize}
            \item If $E_b = \varnothing$, then we must have $E_a = \varnothing$ as well since $E_a \subset E_b$. Hence,
            $$g(a) = g(b) = \inf_{\R}f$$
            \item If $E_b \neq \varnothing$ and $E_a = \varnothing$, then there exists a $x \in B$ such that $f(x) \in E_b$. It follows that
            $$g(a) = \inf_{\R}f \leq f(x) \leq \sup E_b = g(b)$$
            \item If $E_b \neq \varnothing$ and $E_a \neq \varnothing$, then we get $E_a \subset E_b$ which implies that $\sup E_a \leq \sup E_b$. Thus:
            $$g(a) = \sup E_a \leq \sup E_b = g(b)$$
        \end{itemize}
    \end{itemize}
    After this tedious proof by cases, we now have shown that $g$ is an increasing function that extends $f$ on $\R$. \\
\end{solution}

\begin{exercise}
    Prove or give a counterexample: If $(X, \mathcal{S})$ is a measurable space and 
    $$f : X \to [-\infty, \infty]$$
    is a function such that $f^{-1}((a, \infty)) \in \mathcal{S}$ for every $a \in \R$, then $f$ is an $\mathcal{S}$-measurable function. \\
\end{exercise}

\begin{solution}
    \\ Let $X = \R$ and $\mathcal{S}$ be the $\sigma$-algebra of subsets of $\R$ that are either countable or have a countable complement. Define $f : X \to [-\infty, \infty]$ as
    $$f(x) = \begin{cases}
        +\infty & \text{if } x\geq 0 \\
        -\infty & \text{if } x < 0
    \end{cases}$$
    Notice that for all $a \in \R$, we have $f^{-1}((a, \infty)) = \varnothing \in \mathcal{S}$. However, we cannot conclude that $f$ is $\mathcal{S}$-measurable because $\{\infty \}$ is a Borel subset of $[-\infty, \infty]$ but $f^{-1}(\{\infty\}) = [0, \infty) \notin \mathcal{S}$. Therefore, $f$ is not $\mathcal{S}$-measurable. \\
\end{solution}

\begin{exercise}
    Suppose $f : B \to \R$ is a Borel measurable function. Define $g : \R \to \R$ by
    $$g(x) = \begin{cases}
        f(x) & \text{if } x \in B \\
        0 & \text{if } x \in \R \setminus B
    \end{cases}$$
    Prove that $g$ is a Borel measurable function. \\
\end{exercise}

\begin{solution}
    \\ Fix $a \in \R$ and let's show that $g^{-1}((a, \infty))$ is a Borel set. First, notice that $B$ is a Borel set since $B = f^{-1}(\R)$ and $f$ is Borel measurable. If $a \geq 0$, then we can show that $g^{-1}((a, \infty)) = f^{-1}((a, \infty))$ : if $g(x) > a \geq 0$, then it must be that $x \in B$ which implies that $f(x) = g(x) > a$; and if $f(x) > a$, then it directly follows that $g(x) = f(x) > a$. Thus, $g^{-1}((a, \infty))$ is a Borel set. Now, if $a < 0$, then for the same reasons as above, $g^{-1}((a, \infty)) = f^{-1}((a, \infty)) \cup (\R \setminus B)$ which is again a Borel set. Therefore, $g$ is Borel measurable.\\
\end{solution}

\begin{exercise}
    Give an example of a measurable space $(X, \mathcal{S})$ and a family $\{f_t\}_{t \in \R}$ such that each $f_t$ is an $\mathcal{S}$-measurable function from $X$ to $[0, 1]$, but the function $f : X \to [0,1]$ defined by 
    $$f(x) = \sup \{f_t(x) : t \in \R\}$$
    is not $\mathcal{S}$-measurable. \\
    $[$\textit{Compare this exercise to 2.53, where the index set is $\N$ rather than $\R$}.$]$\\
\end{exercise}

\begin{solution}
    \\ Let $X = \R$ and $\mathcal{S}$ be the $\sigma$-algebra of subsets of $\R$ that are countable or have a countable complement. For all $t < 0$, define $f_t$ to be the constant function zero. If $t \geq 0$, define $f_t = \chi_{\{t\}}$. For each $t \in \R$, the function $f_t$ is $\mathcal{S}$-measurable by construction. However, notice that the function defined by
    $$f(x) = \sup \{f_t(x) : t \in \R\}$$
    for all $x \in \R$ is simply the characteristic function of the interval $[0, \infty)$, i.e. $f = \chi_{[0, \infty)}$. But we know that $\chi_E$ is $\mathcal{S}$-measurable if and only if $E \in \mathcal{S}$. However, $[0, \infty) \notin \mathcal{S}$ so $f$ is not $\mathcal{S}$-measurable. \\
\end{solution}

\begin{exercise}
    Show that
    $$\lim_{j \rightarrow \infty} \left( \lim_{k \rightarrow \infty} (\cos(j! \pi x))^{2k} \right) = \begin{cases}
        1 & \text{if } x \text{ is rational}, \\
        0 & \text{if } x \text{ is irrational}
    \end{cases}$$
    for every $x \in \R$.\\
    $[$\textit{This example is due to Henri Lebesgue.}$]$ \\
\end{exercise}

\begin{solution}
    \\ Let $x \in \R$ and consider the case where $x$ is a rational number, then, there exist $a \in \Z$ and $b \in \N$ such that $x = a/b$. For all $j \geq b$, we have $j!x \in \Z$ since $j!$ is a multiple of $b$ which cancels out with the denominator of $x$. But we know that the cosine of an integer multiple of $\pi$ is either 1 or -1, hence, for all $k \in \N$, we have $\cos(j!\pi x)^{2k} = 1$. Since it holds for all $k \in \N$, then
    $$\lim_{k \rightarrow \infty} (\cos(j! \pi x))^{2k} = 1$$
    Since it holds for all $j$ large enough, i.e. $j \geq b$, then
    $$\lim_{j \rightarrow \infty} \left( \lim_{k \rightarrow \infty} (\cos(j! \pi x))^{2k} \right) = 1$$
    Now, consider the case where $x$ is irrational and let $j \in \N$. Since the cosine of $y\pi$ is 1 or -1 if and only if $y$ is an integer, then $\cos(j!\pi x) \in (-1, 1)$ since $j! x$ is an irrational number. It follows that $(\cos(j!\pi x))^2$ is also in the interval $(-1, 1)$. Hence, if we think of $\{(\cos(j!\pi x))^{2k}\}_k$ as a geometric series with $q = (\cos(j!\pi x))^2$, then
    $$\lim_{j \rightarrow \infty} \left( \lim_{k \rightarrow \infty} (\cos(j! \pi x))^{2k} \right) = 0$$
    Since it holds for all $j \in \N$, then
    $$\lim_{j \rightarrow \infty} \left( \lim_{k \rightarrow \infty} (\cos(j! \pi x))^{2k} \right) = 0$$
    which proves our claim. 
\end{solution}