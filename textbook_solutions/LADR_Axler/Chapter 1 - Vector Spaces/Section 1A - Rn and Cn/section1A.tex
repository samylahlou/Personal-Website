\chapter{Vector Spaces}
\section{$\R^n$ and $\C^n$}

\begin{exercise}
    Show that $\alpha + \beta = \beta + \alpha$ for all $\alpha, \beta \in \C$. \\
\end{exercise}

\begin{solution}
    \\ First, suppose that
    $$\alpha = a+ ib \quad \text{ and } \quad \beta = c + id$$
    where $a,b,c,d \in \R$, then
    \begin{align*}
        \alpha + \beta &= (a + ib) + (c + id) \\
        &= (a + c) + i(b + d) \\
        &= (c + a) + i(d + b) \\
        &= (c + id) + (a + ib) \\
        &= \beta + \alpha
    \end{align*}
    which proves that addition is commutative in $\C$ using the fact that it is commutative in $\R$. \\
\end{solution}

\begin{exercise}
    Show that $(\alpha + \beta) + \lambda = \alpha + (\beta + \lambda)$ for all $\alpha, \beta, \lambda \in \C$. \\
\end{exercise}

\begin{solution}
    \\ First, suppose that
    $$\alpha = a+ ib, \quad \beta = c + id \quad \text{ and } \quad \lambda = e + if$$
    where $a,b,c,d,e,f \in \R$, then
    \begin{align*}
        (\alpha + \beta) + \lambda &= [(a + ib) + (c + id)] + (e + if) \\
        &= [(a + c) + i(b + d)] + (e + if) \\
        &= ([a + c] + e) + i([b + d] + f) \\
        &= (a + [c + e]) + i(b + [d + f]) \\
        &= (a + ib) + [(c + e) + i(d + f)] \\
        &= (a + ib) + [(c + id) + (e + if)] \\
        &= \alpha + (\beta + \lambda)
    \end{align*}
    which proves that addition is associative in $\C$ using the fact that it is associative in $\R$. \\
\end{solution}

\begin{exercise}
    Show that $(\alpha\beta)\lambda = \alpha (\beta \lambda)$ for all $\alpha, \beta, \lambda \in \C$. \\
\end{exercise}

\begin{solution}
    \\ First, suppose that
    $$\alpha = a+ ib, \quad \beta = c + id \quad \text{ and } \quad \lambda = e + if$$
    where $a,b,c,d,e,f \in \R$, then
    \begin{align*}
        (\alpha \beta)\lambda &= [(a + ib)(c + id)](e + if) \\
        &= [(ac - bd) +  i(ad + bc)] (e + if) \\
        &= ([ac - bd]e - [ad + bc]f) + i ([ac - bd]f + [ad + bc]e) \\
        &= (ace - bde - adf - bcf) + i(acf - bdf + ade + bce) \\
        &= (a[ce - fd] - b [cf + de)) + i (a[cf + de] + b[ce - fd]) \\
        &= (a + ib) [(ce - fd) + i(cf + de)] \\
        &= (a + ib) [(c + id) (e + if)] \\
        &= \alpha (\beta \lambda)
    \end{align*}
    which proves that multiplication is associative in $\C$ using the fact that multiplication is associative and addition is commutative in $\R$. \\
\end{solution}

\begin{exercise}
    Show that $\lambda(\alpha + \beta) = \lambda \alpha + \lambda \beta$ for all $\lambda, \alpha, \beta \in \C$. \\
\end{exercise}

\begin{solution}
    \\ First, suppose that
    $$\alpha = a+ ib, \quad \beta = c + id \quad \text{ and } \quad \lambda = e + if$$
    where $a,b,c,d,e,f \in \R$, then
    \begin{align*}
        \lambda(\alpha + \beta) &= (e + if)[(a+ib) + (c + id)] \\
        &= (e + if)[(a+c) + i(b+d)] \\
        &= [e(a+c) - f(b+d)] + i[e[b+d] + f[a+c]] \\
        &= (ea + ec - fb - fd) + i(eb + ed + fa + fc) \\
        &= [(ea - fb) + i (eb + fa)] + [(ec - fd) + i(ed + fc)] \\
        &= [(e + if)(a + ib)] + [(e + if)(c + id)] \\
        &= \lambda\alpha + \lambda\beta
    \end{align*}
    which proves the distributivity in $\C$ using the distributivity in $\R$.\\
\end{solution}

\begin{exercise}
    Show that for every $\alpha \in \C$, there exists a unique $\beta \in \C$ such that $\alpha + \beta = 0$. \\
\end{exercise}

\begin{solution}
    \\ Let $\alpha = a + ib$ and consider $\beta = (-a) + i(-b)$, then we get
    \begin{align*}
        \alpha + \beta &= (a + ib) + ([-a] + i[-b]) \\
        &= (a + [-a]) + i(b + [-b]) \\
        &= 0 + i0 \\
        &= 0
    \end{align*}
    which proves the existence of such a complex number $\beta$. To prove the uniqueness of such a complex number, let $\beta_1$ and $\beta_2$ be two complex numbers satisfying $\alpha + \beta_1 = 0$ and $\alpha + \beta_2 = 0$, this implies that $\alpha + \beta_1 = \alpha + \beta_2$. If we add $\beta_1$ on both sides, we get
    \begin{align*}
        \beta_1 + (\alpha + \beta_1) = \beta_1 + (\alpha + \beta_2) &\implies (\beta_1 + \alpha) + \beta_1 = (\beta_1 + \alpha) + \beta_2 \\
        &\implies (\alpha + \beta_1) + \beta_1 = (\alpha + \beta_1) + \beta_2 \\
        &\implies 0 + \beta_1 = 0 + \beta_2 \\
        &\implies \beta_1 = \beta_2
    \end{align*}
    which proves that such a complex number is unique.\\
\end{solution}

\begin{exercise}
    Show that for every $\alpha \in \C$ with $\alpha \neq 0$, there exists a unique $\beta \in \C$ such that $\alpha \beta = 1$. \\
\end{exercise}

\begin{solution}
    \\ Let $\alpha = a + ib \neq 0$, then notice that we must have $a^2 + b^2 \neq 0$. Hence, consider 
    $$\beta = \left(\frac{a}{a^2 + b^2}\right) + i\left(-\frac{b}{a^2 + b^2}\right)$$
    Thus, we get
    \begin{align*}
        \alpha \beta &= (a + ib) \left[\left(\frac{a}{a^2 + b^2}\right) + i\left(-\frac{b}{a^2 + b^2}\right)\right]\\
        &= \left(a\left(\frac{a}{a^2 + b^2}\right) - b\left(-\frac{b}{a^2 + b^2}\right)\right) + i\left(a\left(-\frac{b}{a^2 + b^2}\right) + b\left(\frac{a}{a^2 + b^2}\right)\right) \\
        &= \frac{a^2 + b^2}{a^2 + b^2} + i\frac{-ab + ba}{a^2 + b^2} \\
        &= 1 + i0 \\
        &= 1
    \end{align*}
    which proves the existence of such a complex number $\beta$. To prove the uniqueness of such a complex number, let $\beta_1$ and $\beta_2$ be two complex numbers satisfying $\alpha\beta_1 = 1$ and $\alpha \beta_2 = 1$, this implies that $\alpha \beta_1 = \alpha \beta_2$. If we multiply by $\beta_1$ on both sides, we get
    \begin{align*}
        \beta_1 (\alpha \beta_1) = \beta_1 (\alpha \beta_2) &\implies (\beta_1 \alpha) \beta_1 = (\beta_1 \alpha) \beta_2 \\
        &\implies (\alpha  \beta_1)  \beta_1 = (\alpha  \beta_1)  \beta_2 \\
        &\implies 1 \cdot \beta_1 = 1 \cdot \beta_2 \\
        &\implies \beta_1 = \beta_2
    \end{align*}
    which proves that such a complex number is unique.\\
\end{solution}

\begin{exercise}
    Show that
    $$\frac{-1 + \sqrt{3}i}{2}$$
    is a cube root of 1 (meaning that its cube equals 1). \\
\end{exercise}

\begin{solution}
    \\This is pretty straightforward:
    \begin{align*}
        \left(\frac{-1 + \sqrt{3}i}{2}\right)^3 &= \frac{(-1 + \sqrt{3}i)^3}{2^3} \\
        &= \frac{(-1)^3 + 3(-1)^2(\sqrt{3}i) + 3(-1)^1(\sqrt{3}i)^2 + (\sqrt{3}i)^3}{8} \\
        &= \frac{-1 + 3\sqrt{3}i + 3\cdot 3 - 3(\sqrt{3}i)}{8} \\
        &= \frac{8}{8} \\
        &= 1\\
    \end{align*}
\end{solution}

\begin{exercise}
    Find two distinct square roots of $i$.\\
\end{exercise}

\begin{solution}
    \\ Consider $\alpha = \frac{\sqrt{2}}{2} + i\frac{\sqrt{2}}{2}$ and $\beta = -\frac{\sqrt{2}}{2} - i\frac{\sqrt{2}}{2}$. Hence,
    \begin{align*}
        \alpha^2 &= \left(\frac{\sqrt{2}}{2} + i\frac{\sqrt{2}}{2}\right)^2 \\
        &= \left(\frac{\sqrt{2}}{2}\right)^2 + 2\cdot \frac{\sqrt{2}}{2} \cdot i\frac{\sqrt{2}}{2} + \left(i\frac{\sqrt{2}}{2}\right)^2 \\
        &= \frac{2}{4} + i - \frac{2}{4} \\
        &= i
    \end{align*}
    and
    \begin{align*}
        \beta^2 &= \left(-\frac{\sqrt{2}}{2} - i\frac{\sqrt{2}}{2}\right)^2 \\
        &= \left(-\frac{\sqrt{2}}{2}\right)^2 + 2\cdot \left( -\frac{\sqrt{2}}{2}\right)\cdot \left( - i\frac{\sqrt{2}}{2}\right) + \left(-i\frac{\sqrt{2}}{2}\right)^2 \\
        &= \frac{2}{4} + i - \frac{2}{4} \\
        &= i
    \end{align*}
    Therefore, $\alpha$ and $\beta$ are two distinct square roots of $i$. \\
\end{solution}

\begin{exercise}
    Find $x \in \R^4$ such that
    $$(4, -3, 1, 7) + 2x = (5, 9, -6, 8).$$
\end{exercise}

\begin{solution}
    \\ First, suppose that such an element $x$ exists, then there exist $a,b,c,d \in \R$ such that $x = (a,b,c,d)$ and 
    $$(4 + 2a, -3 + 2b, 1 + 2c, 7 + 2d) = (5, 9, -6, 8)$$
    But notice that this is equivalent to the following system of equations:
    $$\begin{cases}
        4 + 2a = 5 \\ -3 +2b = 9 \\ 1 + 2c = -6 \\ 7 + 2d = 8
    \end{cases}$$
    which implies that
    $$\begin{cases}
        a = \frac{1}{2} \\ b = 6 \\ c = \frac{7}{2} \\ d = \frac{1}{2}
    \end{cases}$$
    Therefore, we get that $x = (\frac{1}{2}, 6, \frac{7}{2}, \frac{1}{2}) \in \R^4$ is indeed a solution to our original equation. \\
\end{solution}

\begin{exercise}
    Explain why there is does not exist $\lambda \in \C$ such that
    $$\lambda (2 - 3i, 5 + 4i, -6 + 7i) = (12 - 5i, 7 + 22i, -32 -9i).$$
\end{exercise}

\begin{solution}
    \\ By contradiction, suppose there exists a complex number $\lambda = a + ib$ such that
    $$\lambda (2 - 3i, 5 + 4i, -6 + 7i) = (12 - 5i, 7 + 22i, -32 -9i)$$
    Then, we would get the following system of equation:
    $$\begin{cases}
        \lambda(2 - 3i) = 12 - 5i \\ \lambda(5 + 4i) = 7 + 22i \\ \lambda(-6 + 7i) = -32 - 9i
    \end{cases}$$
    which is equivalent to
    $$\begin{cases}
        \lambda = 3 + 2i \\ \lambda = 3 + 2i \\ \lambda = \frac{129}{85} + i\frac{278}{85}
    \end{cases}$$
    We clearly have a contradiction since $3 + 2i \neq \frac{129}{85} + i\frac{278}{85}$. Therefore, there doesn't exist such a complex number $\lambda$.\\
\end{solution}

\begin{exercise}
    Show that $(x + y) + z = x + (y + z)$ for all $x,y,z \in \F^n$. \\
\end{exercise}

\begin{solution}
    \\ First, write
    $$x = (x_1, ..., x_n), \quad y = (y_1, ..., y_n) \quad \text{ and } \quad z = (z_1, ..., z_n)$$
    Since addition is commutative in $\F$, we get
    \begin{align*}
        (x + y) + z &= [(x_1, ..., x_n) + (y_1, ..., y_n)] + (z_1, ..., z_n) \\
        &= (x_1 + y_1, ..., x_n + y_n) + (z_1, ..., z_n) \\
        &= ([x_1 + y_1] + z_1, ..., [x_n + y_n] + z_n) \\
        &= (x_1 + [y_1 + z_1], ..., x_n + [y_n + z_n]) \\
        &= (x_1, ..., x_n) + (y_1 + z_1, ..., y_n + z_n) \\
        &= (x_1, ..., x_n) + [(y_1, ..., y_n) + (z_1, ..., z_n)] \\
        &= x + (y+z)
    \end{align*}
    which proves that addition is associative in $\F^n$.\\
\end{solution}

\begin{exercise}
    Show that $(ab)x = a(bx)$ for all $x \in \F^n$ and all $a,b \in \F$.\\
\end{exercise}

\begin{solution}
    \\ First, write $x = (x_1, ..., x_n)$. Using associativity of multiplication in $\F$, we get
    \begin{align*}
        (ab)x &= (ab)(x_1, ..., x_n) \\
        &= ((ab)x_1, ..., (ab)x_n) \\
        &= (a(bx_1), ..., a(bx_n)) \\
        &= a(bx_1, ..., bx_n) \\
        &= a[b(x_1, ..., x_n)] \\
        &= a(bx)
    \end{align*}
    which proves the desired formula for all $x \in \F^n$ and all $a,b \in \F$.\\
\end{solution}

\begin{exercise}
    Show that $1x = x$ for all $x \in \F^n$.\\
\end{exercise}

\begin{solution}
    \\ Let $x = (x_1, ..., x_n) \in \F^n$. Hence,
    \begin{align*}
        1x &= 1(x_1, ..., x_n) \\
        &= (1\cdot x_1, ..., 1 \cdot x_n) \\
        &= (x_1, ..., x_n) \\
        &= x
    \end{align*}
    which proves the desired formula for all $x \in \F^n$.\\
\end{solution}

\begin{exercise}
    Show that $\lambda(x+y) = \lambda x + \lambda y$ for all $\lambda \in \F$ and $x,y \in \F^n$.\\
\end{exercise}

\begin{solution}
    \\ Let $\lambda \in \F$ and $x,y \in \F^n$ with $x = (x_1, ..., x_n)$ and $y = (y_1, ..., y_n)$. Using distributivity in $\F$, we get
    \begin{align*}
        \lambda(x + y) &= \lambda [(x_1, ..., x_n) + (y_1, ..., y_n)] \\
        &= \lambda(x_1 + y_1, ..., x_n + y_n) \\
        &= (\lambda(x_1 + y_1), ..., \lambda (x_n + y_n))\\
        &= (\lambda x_1 + \lambda y_1, ..., \lambda x_n + \lambda y_n) \\
        &= (\lambda x_1, ..., \lambda x_n) + (\lambda y_1, ..., \lambda y_n) \\
        &= \lambda (x_1, ..., x_n) + \lambda (y_1, ...,y_n) \\
        &= \lambda x + \lambda y
    \end{align*}
    which proves the desired formula.\\
\end{solution}

\begin{exercise}
    Show that $(a + b)x = ax + bx$ for all $a,b \in \F$ and all $x \in \F^n$. \\
\end{exercise}

\begin{solution}
    \\ Let $a,b \in \F$ and $x = (x_1, ..., x_n) \in \F^n$. Using distributivity in $\F$, we get
    \begin{align*}
        (a+b)x &= (a+b)(x_1, ..., x_n) &= ((a+b)x_1, ..., (a+b)x_n) \\
        &= (ax_1 + bx_1, ..., ax_n + bx_n) \\
        &= (ax_1, ..., ax_n) + (bx_1, ..., bx_n) \\
        &= a(x_1, ..., x_n) + b(x_1, ..., x_n) \\
        &= ax + bx
    \end{align*}
    which proves the desired formula.
\end{solution}