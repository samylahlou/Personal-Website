\chapter{Euler's Generalization of Fermat's Theorem}

\section{Leonhard Euler}

There are no exercices in this section.

\section{Euler's Phi-Function}

\begin{exercise}
    Calculate $\phi(1001)$, $\phi(5040)$, and $\phi(36,000)$. \\
\end{exercise}

\begin{solution}
    Since $1001 = 7 \cdot 11 \cdot 13$, then 
    $$\phi(1001) = \phi(7)\phi(11)\phi(13) = 6\cdot 10 \cdot 12 = 720.$$
    Since $5040 = 2^4\cdot 3^2 \cdot 5 \cdot 7$, then
    $$\phi(5040) = 2^3(2 - 1) \cdot 3(3-1) \cdot (5-1) \cdot (7-1) = 1152.$$
    Since $36,000 = 2^5 \cdot 3^2 \cdot 5^3$, then
    $$\phi(36,000) = 2^4(2-1)\cdot 3(3-1) \cdot 5^2(5-1) = 9600.$$ \\
\end{solution}

\begin{exercise}
    Verify that the equality $\phi(n) = \phi(n+1) = \phi(n+2)$ holds when $n = 5186$.\\
\end{exercise}

\begin{solution}
    When $n = 5186$, we have $n = 2 \cdot 2593$, $n + 1 = 3 \cdot 7 \cdot 13 \cdot 19$, and $n + 2 = 2^2 \cdot 1297$. Hence:
    \begin{align*}
        \phi(n) &= (2-1) (2593 - 1) = 2592\\
        \phi(n+1) &= (3-1)(7-1)(13-1)(19-1) = 2592 \\
        \phi(n+2) &= 2(2-1)(1297-1) = 2592. \\
    \end{align*}
\end{solution}

\begin{exercise}
    Show that the integers $m = 3^k \cdot 568$ and $n = 3^k \cdot 638$, where $k \geq 0$, satisfy simultaneously
    $$\tau(m) = \tau(n), \ \sigma(m) = \sigma(n), \ \phi(m) = \phi(n).$$\\
\end{exercise}

\begin{solution}
    Since $m = 2^3 \cdot 3^k \cdot 71$ and $n = 2 \cdot 3^k \cdot 11 \cdot 29$, then
    \begin{align*}
        \tau(m) &= (3+1)(k+1)(1+1) = (1 + 1)(k+1)(1+1)(1+1) = \tau(n)\\
        \sigma(m) &= (1 + 2 + 2^2 + 2^3)\sigma(3^k)(1 + 71) = 1080\sigma(3^k) = (1 + 2)\sigma(3^k)(1 + 11)(1 + 29) = \sigma(n)\\
        \phi(m) &= 2^2(2-1)\phi(3^k)(71-1) = 280\phi(3^k) = (2-1)\phi(3^k)(11-1)(29-1) = \sigma(n)
    \end{align*}
    which proves the claim. \\
\end{solution}

\begin{exercise}
    Establish each of the assertions below:
    \begin{enumerate}
        \item If $n$ is an odd integer, then $\phi(2n) = \phi(n)$.
        \item If $n$ is an even integer, then $\phi(2n) = 2\phi(n)$.
        \item $\phi(3n) = 3\phi(n)$ if and only if $3 \mid n$.
        \item $\phi(3n) = 2\phi(n)$ if and only if $3 \not| \ n$.
        \item $\phi(n) = n/2$ if and only if $n = 2^k$ for some $k \geq 1$. [\textit{Hint:} Write $n = 2^kN$, where $N$ is odd, and use the condition $\phi(n) = n/2$ to show that $N=1$.]\\
    \end{enumerate}
\end{exercise}

\begin{solution}
    \begin{enumerate}
        \item If $n$ is odd, then $\gcd(2,n) = 1$, and so it follows that $\phi(2n) = \phi(2)\phi(n) = \phi(n)$.
        \item If $n$ is even, then we can write $n = 2^km$ where $k \geq 1$ and $m$ such that $\gcd(2,m) = 1$. Hence, by multiplicativity:
        $$\phi(2n) = \phi(2^{k+1}m) = \phi(2^{k+1})\phi(m) = 2^k\phi(m) = 2\phi(2^k)\phi(m) = 2\phi(n).$$
        \item Write $n = 3^km$ where $\gcd(3,m) = 1$. If $3\mid n$, then $k \geq 1$ which implies that $3^k(3-1) = 3\cdot 3^{k-1}(3-1)$, and hence, that $\phi(3^{k+1})\phi(m) = 3\phi(3^k)\phi(m)$. Equivalently, this means that $\phi(3n) = 3\phi(n)$. Conversely, if $3 \not | \ n$, then $k = 0$, and hence, $\phi(3n) = \phi(3)\phi(n) = 2\phi(n) \neq 3\phi(n)$.
        \item We showed in the previous part that $3 \not | \ n$ implies that $\phi(3n) = 2\phi(n)$. Conversely, when $3 \mid n$, we know that $\phi(3n) = 3\phi(n)$, and hence, $\phi(3n) \neq 2\phi(n)$.
        \item If $n = 2^k$ for some $k \geq 1$, then
        $$\phi(n) = \phi(2^k) = 2^{k-1} = n/2.$$
        Conversely, suppose that $\phi(n) = n/2$ and write $n = p_1^{k_1}\cdot \cdot \cdot p_r^{k_r}$, then we can rewrite the equation as
        $$n \left(1 - \frac{1}{p_1}\right)\cdot \cdot \cdot \left(1 - \frac{1}{p_r}\right) = \frac{n}{2}.$$
        Rearranging the equation gives us
        $$2(p_1 - 1)\cdot \cdot \cdot (p_r - 1) = p_1\cdot \cdot \cdot p_r.$$
        Since the left hand side is even, then one of the primes in the right hand side must be 2, and hence, $p_1 = 2$. The equation becomes
        $$(p_2 - 1)\cdot \cdot \cdot (p_r - 1) = p_2\cdot \cdot \cdot p_r.$$
        But this equation is impossible (the left hand side is clearly strictly less than the right hand side) when $r \geq 2$, if $r = 1$, this equation is simply $1 = 1$. Hence, $r = 1$, and hence, $n = p_1^{k_1} = 2^{k_1}$ for some $k_1 \geq 1$.  \\
    \end{enumerate}
\end{solution}

\begin{exercise}
    Prove that the equation $\phi(n) = \phi(n+2)$ is satisfied by $n = 2(2p-1)$ whenever $p$ and $2p-1$ are both odd primes. \\
\end{exercise}

\begin{solution}
    Simply notice that
    $$\phi(n) = \phi(2)\phi(2p-1) = 2p-2$$
    and
    $$\phi(n+2) = \phi(4p) = \phi(4)\phi(p) = 2(p-1).$$
    Hence, it follows that $\phi(n) = \phi(n+2)$.\\
\end{solution}

\begin{exercise}
    Show that there are infinitely many integers $n$ for which $\phi(n)$ is a perfect square. [\textit{Hint:} Consider the integers $n = 2^{k+1}$ for $k = 1,2,\dots$.] \\
\end{exercise}

\begin{solution}
    Let $k$ be an arbitrary integer and $n = 2^{2k+1}$, then $\phi(n) = 2^{2k} = (2^k)^2$. Since $(2^k)^2 \neq (2^q)^2$ for $k \neq q$, then there are infinitely many $n$ such that $\phi(n)$ is a square since there are infinitely many $k$'s. \\
\end{solution}

\begin{exercise}
    Verify the following:
    \begin{enumerate}
        \item For any positive integer $n$, $\frac{1}{2}\sqrt{n} \leq \phi(n) \leq n$. [\textit{Hint:} Write $n = 2^{k_0}p_1^{k_1}\cdot \cdot \cdot p_r^{k_r}$, so $\phi(n) = 2^{k_0 - 1}p_1^{k_1-1}\cdot \cdot \cdot p_r^{k_r - 1}(p_1 - 1)\cdot \cdot \cdot (p_r - 1)$. Now, use the inequalities $p-1 > \sqrt{p}$ and $k - \frac{1}{2} \geq k/2$ to obtain $\phi(n) \geq 2^{k_0-1}p_1^{k_1/2}\cdot \cdot \cdot p_r^{k_r/2}$.]
        \item If the integer $n > 1$ has $r$ distinct prime factors, then $\phi(n) \geq n/2^r$.
        \item If $n > 1$ is a composite number, then $\phi(n) \leq n - \sqrt{n}$. [\textit{Hint:} Let $p$ be the smallest prime divisor of $n$, so that $p \leq \sqrt{n}$. Then $\phi(n) \leq n(1 - 1/p)$.] \\
    \end{enumerate}
\end{exercise}

\begin{solution}
    \begin{enumerate}
        \item Let $n = 2^{k_0}p_1^{k_1}\cdot \cdot \cdot p_r^{k_r}$, then $\phi(n) = 2^{k_0 - 1}p_1^{k_1-1}\cdot \cdot \cdot p_r^{k_r - 1}(p_1 - 1)\cdot \cdot \cdot (p_r - 1)$. Since $(p_i - 1) \geq \sqrt{p_i}$, then we get that $\phi(n) \geq \frac{1}{2}\cdot 2^{k_0}p_1^{k_1-\frac{1}{2}}\cdot \cdot \cdot p_r^{k_r -\frac{1}{2}}$. Next, since $k_0 \geq k_0/2$ and $k_i - \frac{1}{2} \geq k_i/2$, then $\phi(n) \geq \frac{1}{2}2^{k_0/2}p_1^{k_1/2}\cdot \cdot \cdot p_r^{k_r} = \frac{1}{2}\sqrt{n}$. Finally, the inequality $\phi(n) \leq n$ follows from the definition of $\phi$.
        \item Let $n = p_1^{k_1}\cdot \cdot \cdot p_r^{k_r}$, then 
        $$\frac{n}{\phi(n)} = \frac{n}{n\left(1 - \frac{1}{p_1}\right)\cdot \cdot \cdot \left(1 - \frac{1}{p_r}\right)} = \frac{p_1}{p_1 - 1}\cdot \cdot \cdot \frac{p_r}{p_r - 1}\leq 2^r$$
        which we can rearrange into $\phi(n) \geq n/2^r$.
        \item Let $p$ be a prime dividing $n$, since $n$ is composite, then $p \leq \sqrt{n}$, hence, $p\sqrt{n} \leq n$. It follows that the numbers $p$, $2p$, ..., $\lfloor \sqrt{n}\rfloor p$ are all less than $n$, and not relatively prime to $n$. There are $\lfloor \sqrt{n}\rfloor$ such numbers. If we let $q$ be a second prime divisor of $n$ (if $n$ is a power of $p$, then the statement is clear from the formula for $\phi(p^k)$ and the inequality $k-1 \geq k/2$ for $k \geq 2$), then we know that $q$ is a number distinct from $p$, $2p$, ..., $\lfloor \sqrt{n} \rfloor$ that is not relatively prime to $n$. Hence, we have found $\lfloor \sqrt{n} \rfloor + 1$ numbers less than $n$ which are not relatively prime to it. Therefore, $\phi(n) \leq n - (\lfloor \sqrt{n} \rfloor + 1) \leq n - \sqrt{n}$. \\
    \end{enumerate}
\end{solution}

\begin{exercise}
    Prove that if the integer $n$ has $r$ distinct odd prime factors, then $2^r \mid \phi(n)$.\\
\end{exercise}

\begin{solution}
    Write $n = 2^{k_0}p_1^{k_1}\cdot \cdot \cdot p_r^{k_r}$ where $k_0 \geq 0$ and $k_i \geq 1$ for $1 \leq i \leq r$, then
    $$\phi(n) = \phi(2^{k_0})p_1^{k_1 -1}\cdot \cdot \cdot p_r^{k_r - 1}(p_1 - 1)\cdot \cdot \cdot (p_r - 1).$$
    Since all the factors $p_i - 1$ are even, and there are $r$ such factors, then $2^r$ divides $\phi(n)$. \\
\end{solution}

\begin{exercise}
    Prove that:
    \begin{enumerate}
        \item If $n$ and $n+2$ are a pair of twin primes, then $\phi(n + 2) = \phi(n) + 2$; this also holds for $n=12$, 14, and 20.
        \item If $p$ and $2p+1$ are both odd primes, then $n = 4p$ satisfies $\phi(n+2) = \phi(n) + 2$.\\
    \end{enumerate}
\end{exercise}

\begin{solution}
    \begin{enumerate}
        \item For any prime $p$, we have that $\phi(p) = p-1$. It follows that
        $$\phi(n+2) = n+1 = (n-1) + 2 = \phi(n) + 2$$
        when both $n$ and $n+2$ are prime. When $n = 12$:
        $$\phi(n+2) = \phi(2)\phi(7) = 6 = \phi(2^2)\phi(3) + 1 = \phi(n) + 2.$$
        When $n = 14$:
        $$\phi(n+2) = \phi(2^4) = 8 = \phi(2)\phi(7) + 2 = \phi(n) + 2.$$
        When $n = 20$:
        $$\phi(n+2) = \phi(2)\phi(11) = 10 = \phi(2^2)\phi(5) + 2 = \phi(n) + 2.$$
        \item Suppose that $p$ and $2p+1$ are odd primes, then for $n = 4p$, we have
        $$\phi(n+2) = \phi(4p+2) = \phi(2)\phi(2p+1) = 2p$$
        and
        $$\phi(n) + 2 = \phi(4p) + 2 = \phi(4)\phi(p) + 2 = 2(p-1) + 2 = 2p$$
        which proves that $\phi(n + 2) = \phi(n) + 1$. \\
    \end{enumerate}
\end{solution}

\begin{exercise}
    If every prime that divides $n$ also divides $m$, establish that $\phi(nm) = n\phi(m)$; in particular, $\phi(n^2) = n\phi(n)$ for every positive integer $n$. \\
\end{exercise}

\begin{solution}
    Write $n = p_1^{k_1}\cdot \cdot \cdot p_r^{k_r}$ and $m = p_1^{t_1}\cdot \cdot \cdot p_r^{t_r}q_1^{s_1}\cdot \cdot \cdot q_u^{s_u}$, then
    \begin{align*}
        \phi(nm) &= \phi(p_1^{k_1 + t_1}\cdot \cdot \cdot p_r^{k_r + t_r}q_1^{s_1}\cdot \cdot \cdot q_u^{s_u}) \\
        &=  p_1^{k_1 + t_1 - 1}(p_1 - 1)\cdot \cdot \cdot p_r^{k_r + t_r - 1}(p_r - 1) \phi(q_1^{s_1}\cdot \cdot \cdot q_u^{s_u}) \\
        &= [p_1^{k_1}\cdot \cdot \cdot p_r^{k_r}]\cdot p_1^{t_1 - 1}(p_1 - 1)\cdot \cdot \cdot p_r^{t_r - 1}(p_r - 1)\phi(q_1^{s_1}\cdot \cdot \cdot q_u^{s_u}) \\
        &= n \phi(p_1^{t_1}\cdot \cdot \cdot p_r^{t_r})\phi(q_1^{s_1}\cdot \cdot \cdot q_u^{s_u}) \\
        &= n \phi(m).
    \end{align*}
     \\
\end{solution}

\begin{exercise}
    \begin{enumerate}
        \item If $\phi(n) \mid n - 1$, prove that $n$ is a square-free integer. [\textit{Hint:} Assume that $n$ has the prime factorization $n = p_1^{k_1}p_2^{k_2}\cdot \cdot \cdot p_r^{k_r}$, where $k_1 \geq 2$. Then $p_1 \mid \phi(n)$, whence $p_1 \mid n-1$, which leads to a contradiction.]
        \item Show that if $n = 2^k$ or $n = 2^k3^j$, with $k$ and $j$ positive integers, then $\phi(n) \mid n$. \\
    \end{enumerate}
\end{exercise}

\begin{solution}
    \begin{enumerate}
        \item Let $p$ be a prime divisor of $n$ and let $k$ be the exponent of $p$ in the prime factorization of $n$, then $\phi(n) = \phi(p^k)\phi(n/p^k) = p^{k-1}(p-1)\phi(n/p^k)$. When $k \geq 2$, we have that $p \mid \phi(n)$ using the expression above, but since $\phi(n) \mid n-1$, then $p \mid n-1$. This is impossible because $p \mid n$. Thus, $k = 1$, and hence, $n$ is square-free.
        \item Let $n = 2^k3^k$ with $k \geq 1$. If $j = 0$, then $\phi(n) = 2^{k-1} \mid 2^k = n$. When $j \geq 1$, we get $\phi(n) = 2^{k-1}3^{j-1}(3-1) = 2^k3^{j-1} \mid 2^k3^j = n$. \\
    \end{enumerate}
\end{solution}

\begin{exercise}
    If $n = p_1^{k_1}p_2^{k_2}\cdot \cdot \cdot p_r^{k_r}$, derive the inequalities
    \begin{enumerate}
        \item $\sigma(n)\phi(n) \geq n^2(1 - 1/p_1^2)(1 - 1/p_2^2)\cdot \cdot \cdot (1 - 1/p_r^2)$ and 
        \item $\tau(n)\phi(n) \geq n$. [\textit{Hint:} Show that $\tau(n)\phi(n) \geq 2^r \cdot n(1/2)^r$.] \\
    \end{enumerate}
\end{exercise}

\begin{solution}
    \begin{enumerate}
        \item First, notice that
        \begin{align*}
            \sigma(n) &= (1 + \dots + p_1^{k_1 - 1} + p_1^{k_1})\cdot \cdot \cdot (1 + \dots + p_r^{k_r - 1} + p_r^{k_r}) \\
            &\geq (p_1^{k_1} + p_1^{k_1 - 1})\cdot \cdot \cdot (p_r^{k_r} + p_r^{k_r - 1}) \\
            &= p_1^{k_1}\cdot \cdot \cdot p_r^{k_r}(1 + 1/p_1)\cdot \cdot \cdot (1 + 1/p_r) \\
            &= n(1 + 1/p_1)\cdot \cdot \cdot (1 + 1/p_r).
        \end{align*}
        Multiplying the resulting inequality by
        $$\phi(n) = n(1 - 1/p_1)\cdot \cdot \cdot (1 - 1/p_r)$$
        gives us
        $$\sigma(n)\tau(n) \geq n^2(1 - 1/p_1^2)\cdot \cdot \cdot (1 - 1/p_r^2).$$
        \item Since
        \begin{align*}
            \tau(n) &= (k_1 + 1)(k_2 + 1)\cdot \cdot \cdot (k_r + 1) \\
            &\geq \left(1 - \frac{1}{p_1}\right)^{-1}\left(1 - \frac{1}{p_2}\right)^{-1} \cdot \cdot \cdot \left(1 - \frac{1}{p_r}\right)^{-1} \\
            &= \frac{n}{\phi(n)},
        \end{align*}
        then $\tau(n)\phi(n) \geq n$. \\
    \end{enumerate}
\end{solution}

\begin{exercise}
    Assuming that $d \mid n$, prove that $\phi(d) \mid \phi(n)$. [\textit{Hint:} Work with the prime factorizations of $d$ and $n$.] \\
\end{exercise}

\begin{solution}
    Let $d = p_1^{k_1}\cdot \cdot \cdot p_r^{k_r}$ and $n = p_1^{t_1}\cdot \cdot \cdot p_r^{t_r}q_1^{s_1}\cdot \cdot \cdot q_u^{s_u}$ with $k_i \leq t_i$, then $p_i^{k_i-1} \mid p_i^{t_i-1}$. It follows that $\prod_{i=1}^{r}p_i^{k_i-1} \mid \prod_{i=1}^{r}p_i^{t_i-1}$. Using the rules of divisibility, we get that $\phi(d)$, which is equal to $\prod_{i=1}^{r}p_i^{k_i-1}(p_i - 1)$ divides $\prod_{i=1}^{r}p_i^{t_i-1}(p_i - 1)$, which in turns divides $\prod_{i=1}^{r}p_i^{t_i-1}(p_i - 1)\prod_{i=1}^{u}q_i^{s_i-1}(q_i - 1) = \phi(n)$. Therefore, $\phi(d) \mid \phi(n)$.\\
\end{solution}

\begin{exercise}
    Obtain the following two generalizations of Theorem 7-2:
    \begin{enumerate}
        \item For positive integers $m$ and $n$,
        $$\phi(m)\phi(n) = \phi(mn)\phi(d)/d,$$
        where $d = \gcd(m,n)$.
        \item For positive integers $m$ and $n$,
        $$\phi(m)\phi(n) = \phi(\gcd(m,n))\phi(\lcm(m,n)).$$
    \end{enumerate}
\end{exercise}

\begin{solution}
    \begin{enumerate}
        \item Let $m = p_1^{k_1}\cdot \cdot \cdot p_r^{k_r}a$ and $n = p_1^{t_1}\cdot \cdot \cdot p_r^{t_r}b$ where the $p_i$'s are the prime common to $m$ and $n$, and where $a$ and $b$ are prime relative to the $p_i$'s, then $d = \gcd(m,n)$ is also formed of the primes $p_1$, ..., $p_r$. It follows that
        \begin{align*}
            \phi(mn)\frac{\phi(d)}{d} &= \phi\left(ab\prod_i p_i^{k_i+t_i}\right) \prod_i\left(1 - \frac{1}{p_i}\right) \\
            &= \phi(a)\phi(b) \prod_i p_i^{k_i + t_i - 1}(p_i - 1) \prod_i \frac{p_i - 1}{p_i}\\
            &= \left[\phi(a) \prod_i p_i^{k_i - 1}(p_i - 1) \right] \left[\phi(b)\prod_i p_i^{t_i - 1}(p_i - 1) \right]\\  
            &= \phi(m)\phi(n).
        \end{align*}
        \item First, recall that $mn = \gcd(m,n)\lcm(m,n)$. Moreover, since $\gcd(m,n) \mid mn$, then
        $$\phi(mn) = \gcd(m,n)\phi\left(\frac{mn}{\gcd(m,n)}\right) = \gcd(m,n)\phi(\lcm(m,n))$$
        by Problem 10. Therefore, using part (a):
        $$\phi(m)\phi(n) = \phi(mn)\frac{\phi(\gcd(m,n))}{\gcd(m,n)} = \phi(\gcd(m,n))\phi(\lcm(m,n)).$$ \\
    \end{enumerate}
\end{solution}

\begin{exercise}
    Prove that:
    \begin{enumerate}
        \item There are infinitely many integers $n$ for which $\phi(n) = n/3$. [\textit{Hint:} Consider $n = 2^k3^j$, where $k$ and $j$ are positive integers.]
        \item There are no integers $n$ for which $\phi(n) = n/4$.\\
    \end{enumerate}
\end{exercise}

\begin{solution}
    \begin{enumerate}
        \item For all positive integers $k$ and $j$, if we let $n = 2^k3^j$, then
        $$\phi(n) = \phi(2^k)\phi(3^j) = 2^{k-1}3^{j-1}(3-1) = \frac{2^k3^j}{3} = \frac{n}{3}.$$
        Since there infinitely many integers of the form $2^k3^j$, then there are infinitely many integers $n$ such that $\phi(n) = n/3$.
        \item Suppose that there exists an integer $n = p_1^{k_1}\cdot \cdot \cdot p_r^{k_r}$ such that $\phi(n) = n/4$, then equivalently:
        $$4p_1^{k_1-1}(p_1 - 1)\cdot \cdot \cdot p_r^{k_r-1}(p_r - 1) = p_1^{k_1}\cdot \cdot \cdot p_r^{k_r}$$
        which we can rewrite as
        $$4(p_1 - 1)\cdot \cdot \cdot (p_r - 1) = p_1\cdot \cdot \cdot p_r.$$
        Since 2 divides the left hand side, then 2 divides $p_1\cdot \cdot \cdot p_r$ which implies that $p_i = 2$ for some $1 \leq i \leq r$. Without loss of generality, we can assume that $p_1 = 2$. If we cancel out $p_1$ and 2 on both sides, we get the equivalent equation
        $$2(p_1 - 1)\cdot \cdot \cdot (p_r - 1) = p_2\cdot \cdot \cdot p_r.$$ 
        Applying the same argument as before ensures that $p_i = 2$ for some $2 \leq i \leq r$, but it would imply that $p_i = p_1$ with $i \neq 1$. This is impossible because the $p_i$'s are distinct from each other. Therefore, by contradiction, there is no integer $n$ such that $\phi(n) = n/4$. \\
    \end{enumerate}
\end{solution}

\begin{exercise}
    Show that Goldbach's Conjecture implies that for each even integer $2n$ there exist integers $n_1$ and $n_2$ with $\phi(n_1) + \phi(n_2) = 2n$. \\
\end{exercise}

\begin{solution}
    By Goldbach's Conjecture, we know that there exist prime numbers $p$ and $q$ such that $p + q = 2n + 2$. Hence:
    $$\phi(p) + \phi(q) = (p-1) + (q-1) = 2n + 2 - 2 = 2n.$$\\
\end{solution}

\begin{exercise}
    Given a positive integer $k$, show that 
    \begin{enumerate}
        \item there are at most a finite number of integers $n$ for which $\phi(n) = k$;
        \item if the equation $\phi(n) = k$ has a unique solution, say $n = n_0$, then $4\mid n_0$. [\textit{Hint:} See Problem 4(a) and 4(b).]
    \end{enumerate}
    A famous conjecture of Carmichael is that the number of solutions of $\phi(n) = k$ cannot be equal to one.\\
\end{exercise}

\begin{solution}
    \begin{enumerate}
        \item Since $\phi(n) \geq \frac{1}{2}\sqrt{n}$ for all $n$, then $\phi(n) > k$ for all $n > 4k^2$. Hence, all the solutions of the equation $\phi(n) = k$ must be less than $4k^2$ which implies that there are only finitely many possible solutions.
        \item Write $n_0 = 2^km$ where $m$ is odd. If $k = 0$, then $\phi(2n_0) = \phi(2)\phi(n_0) = \phi(n_0) = k$, and hence, $2n_0$ is also a solution. This is impossible since $2n_0 \neq n_0$. Similarly, if $k = 1$, then $\phi(m) = \phi(2)\phi(m) = \phi(2m) = \phi(n_0) = k$. Again, this is impossible since $m \neq n_0$. Therefore, $k \geq 2$, and hence, $4 \mid n_0$. \\
    \end{enumerate}
\end{solution}

\begin{exercise}
    Find all solutions of $\phi(n) = 16$ and $\phi(n) = 24$. [\textit{Hint:} If $n = p_1^{k_1}p_2^{k_2}\cdot \cdot \cdot p_r^{k_r}$ satisfies $\phi(n) = k$, then $n = [k/\prod (p_i - 1)]\prod p_i$. Thus, then integers $d_i = p_i - 1$ can be determined by the conditions (1) $d_i \mid k$, (2) $d_i + 1$ is prime and (3) $k/\prod d_i$ contains no prime factor not in $\prod p_i$.] \\
\end{exercise}

\begin{solution}
    Write $n = p_1^{k_1}p_2^{k_2}\cdot \cdot \cdot p_r^{k_r}$, then $\phi(n) = 16$ implies that $\prod p_i^{k_i - 1}(p_i - 1) = 16$. Equivalently, dividing both sides by $\prod (p_i - 1)$ and then multiplying both sides by $\prod p_i$ gives us $n = [16/\prod(p_i - 1)]\prod p_i$. It follows that $p_i - 1$ divide 16 for all $i$, and hence, $p_i - 1 = 1,2,4,8,16$. Thus, $p_i = 2,3, 5, 9, 17$ but $p_i = 9$ is impossible since $p_i$ is prime. Thus, the possible prime factors of $n$ are 2,3, 5, and 17. If $n$ is divisible by 17, then $n = 17^km$ where $m$ is not divisible by 17, and hence, $\phi(n) = 16$ becomes $17^{k-1}\cdot 16 \cdot \phi(m) = 16$ which implies that $k=1$ and $m=1,2$. Thus, either $n = 17, 34$, or $n$ is only composed of 2, 3, and 5. When $n$ is composed of 2, 3, and 5, then $\prod (p_i - 1) < 16$, and hence, $2 \mid 16/\prod(p_i - 1)$. Since $n = [16/\prod(p_i - 1)]\prod p_i$, then $2 \mid n$ so 2 must be a prime factor of $n$. If $n = 2^k$, then $\phi(n) = 16$ implies that $n = 32$; if $n = 2^k3^j$, then $\phi(n) = 16$ implies that $n = 48$; if $n = 2^k5^t$, then $\phi(n) = 16$ implies that $n = 40$; if $n = 2^k3^j5^t$, then $\phi(n) = 16$ implies that $n = 60$. Therefore, the solutions to the equation $\phi(n) = 16$ are $n = 17, 32, 34, 40, 48, 60$.
    
    Next, let's solve the equation $\phi(n) = 24$. As we did in the first part, we have that $n = [24/\prod(p_i - 1)]\prod p_i$. Hence, the prime divisors of $n$ satisfy $p_i - 1 \mid 24$, and hence, $p_i$ must be equal to one of 2, 3, 5, 7, and 13. If $13 \mid n$, then $n = 13^k m$ with $\gcd(13, m) = 1$, and hence, $\phi(n) = 24$ implies that $\phi(m)\cdot 13^{k-1}\cdot 12 = 24$. Thus, $k = 1$ and $\phi(m) = 2$. Hence, $m$ is either 3, 4, or 6. Thus, $n$ can be 39, 52, or 78. Now, if $n$ is not divisible by 13, then $n$ can only be composed of 2, 3, 5, or 7. If $n$ is divisible by 7, then $n = 7^km$ with $\gcd(7,m) = 1$. In that case, $\phi(n) = 24$ becomes $7^{k-1}\cdot 6\phi(m) = 24$ which implies that $k = 1$ and $\phi(m) = 4$. Thus, $m = 5,8,10,12$ which implies that $n = 35, 56, 70, 84$. Now, if $n$ is not divisible by 7 either, then $n = 3^2m$ with $\gcd(3,m) = 1$, otherwise, $\phi(n)$ would not be divisible by 3 or would contain too much factors of 3. Hence, we have the following cases: if $n = 2^k3^2$, then $\phi(n) = 24$ implies that $n = 72$; if $n = 3^25^t$, then $\phi(n) = 24$ implies that $n = 45$, if $n = 2^k3^25^t$, then $\phi(n) = 24$ shows that $n = 90$. Therefore, $n = 35, 39, 45, 52, 56, 70, 72, 78, 84, 90$.\\
\end{solution}

\begin{exercise}
    \begin{enumerate}
        \item Prove that the equation $\phi(n) = 2p$, where $p$ is a prime number and $2p+ 1$ is composite, is not solvable.
        \item Prove that there is no solutions to the equation $\phi(n) = 14$, and that 14 is the smallest (positive) even integer with this property.
    \end{enumerate}
\end{exercise}

\begin{solution}
    \begin{enumerate}
        \item Write $n = p_1^{k_1}\cdot \cdot \cdot p_r^{k_r}$ and suppose that $\phi(n) = 2p$, then 
        $$p_1^{k_1-1}(p_1 - 1)\cdot \cdot \cdot p_r^{k_r - 1}(p_r - 1) = 2p$$
        which implies that $n$ must have at most one odd prime divisor (since $p$ is prime with $2p + 1$ composite, then $p \neq 2$ and so there is only one factor of 2 on both sides of these equations). It follows that $n$ is either a power of 2, a power of an odd prime $q$, or a number of the form $2^kq^j$ with $k,j \geq 1$. Since $p$ is odd, then $2p$ is not a power of 2, and hence, $\phi(n) \neq 2p$. If $n = q^j$, then $q^{j-1}(q-1) = 2p$. If $q - 1 = 2$, then $q = 3$ and so $3^{j-1} = p$ which implies that $p = q = 3$. But this is impossible since $2p + 1$ is composite while $2\cdot 3 + 1 = 7$ is prime; hence, $q > 3$. Since $q - 1 \neq 2$, then it must contain at most two prime factors distinct from $q$ (it cannot be a power of two higher than 2 since there is only one factor of 2 on the right hand side of the equation). In that case, since there are only two prime factors on the right hand side, then $q^{j - 1} = 1$ and so $q - 1 = 2p$ which implies that $q = 2p + 1$, a contradiction since $2p + 1$ is composite while $q$ is prime; hence, $n \neq q^j$ for some odd prime $q$. Finally, if $n = 2^kq^j$, then $2^{k-1}q^{j-1}(q-1) = 2p$. By the same considerations as before, $k = 1$ which implies that $q^{j-1}(q-1) = 2p$. But this is exactly the same equation as above which lead to a contradiction. Thus, this last case is also impossible. Therefore, the equation $\phi(n) = 2p$ has no solutions.
        \item Since $14 = 2\cdot 7$ and $2\cdot 7 + 1 = 15$ is composite, then $\phi(n) = 14$ has no solutions. This is the smallest such positive even number because $\phi(n) = 12$ has the solution $n = 36$, $\phi(n) = 10$ has the solution $n = 5$, $\phi(n) = 8$ has the solution $n = 16$, $\phi(n) = 6$ has the solution $n = 9$, $\phi(n) = 4$ has the solution $n = 8$, and $\phi(n) = 2$ has the solution $n = 4$. Therefore, 14 is the smallest positive even number with the property that $\phi(n) = 14$ has no solutions.\\
    \end{enumerate}
\end{solution}

\begin{exercise}
    If $p$ is a prime and $k \geq 2$, show that $\phi(\phi(p^k)) = p^{k-2}\phi((p-1)^2)$. \\
\end{exercise}

\begin{solution}
    By the usual rules and by Problem 10, we have
    \begin{align*}
        \phi(\phi(p^k)) &= \phi(p^{k-1}(p - 1)) \\
        &= \phi(p^{k-1})\phi(p-1) \\
        &= p^{k-2}(p-1)\phi(p-1) \\
        &= p^{k-2}\phi((p-1)^2).
    \end{align*}
\end{solution}