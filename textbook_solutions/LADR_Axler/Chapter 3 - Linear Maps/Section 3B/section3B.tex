\section{Null Spaces and Ranges}

\begin{exercise}
    Give an example of a linear map $T$ with $\dim \Null T = 3$ and $\dim \range T = 2$. \\
\end{exercise}

\begin{solution}
    \\ Consider the map $T \in \L(\R^5)$ defined by
    $$T(x_1, x_2, x_3, x_4, x_5) = (x_1, x_2, 0, 0, 0).$$ 
    The range of $T$ is equal to
    $$\{(x_1, x_2, 0, 0, 0) : x_1, x_2, x_3, x_4, x_5 \in \R\} = \{(x_1, x_2, 0, 0, 0) : x_1, x_2 \in \R\}.$$ 
    It follows that $\dim \range T = 2$. Concerning the null space, notice that
    \begin{align*}
        \Null T &= \{(x_1, x_2, x_3, x_4, x_5)\in \R^5 : (x_1, x_2, 0, 0, 0) = 0\} \\
        &= \{(x_1, x_2, x_3, x_4, x_5)\in \R^5 : x_1 = x_2 = 0\} \\
        &= \{(0,0, x_3, x_4, x_5)\in \R^5 : x_3, x_4, x_5 \in \R\}.
    \end{align*}
    It follows that $\dim \Null T = 2$. \\
\end{solution}

\begin{exercise}
    Suppoe $S,T \in \L(V)$ are such that $\range S \subset \Null T$. Prove that $(ST)^2 = 0$. \\
\end{exercise}

\begin{solution}
    \\ To prove this, let's show that $(ST)^2$ maps every vector in $V$ to 0. Let $v \in V$, then $Tv \in V$ so $STv \in \range S \subset \Null T$. It follows that $T(STv) = 0$. Thus, we obtain that 
    $$(ST)^2v = STSTv = ST(Stv) = S0 = 0.$$
    Therefore, $(ST)^2 = 0$. \\
\end{solution}

\begin{exercise}
    Suppose $v_1, ..., v_m$ is a list of vectors in $V$. Define $T \in \L(\F^m, V)$ by
    $$T(z_1, ..., z_m) = z_1v_1 + ... + z_mv_m.$$
    \begin{enumerate}[label=(\alph*)]
        \item What property of $T$ corresponds to $v_1, ..., v_m$ spanning $V$?
        \item What property of $T$ corresponds to the list $v_1, ..., v_m$ being linearly independent?\\
    \end{enumerate}
\end{exercise}

\begin{solution}
    \begin{enumerate}[label=(\alph*)]
        \item Let's prove that $T$ is surjective if and only if $v_1, ..., v_m$ spans $V$. If $T$ is surjective, then for all $v \in V$, there exists a vector $(z_1, ..., z_m) \in \F^m$ such that $T(z_1, ..., z_m) = v$. But this just means that for all $v \in V$, there exist scalars $z_1, ..., z_m \in \F$ such that $z_1 v_1 + ... + z_m v_m = v$. Hence, by definition, the list is spanning. \\
        To prove the converse, suppose that $v_1, ..., v_m$ spans $V$, then for all $v \in V$, there exist scalars, $z_1, ..., z_m$ such that $z_1 v_1 + ... + z_m v_m = v$. It follows that $T(z_1, ..., z_m) = v$. Therefore, by definition, $T$ is surjective. 
        \item Let's prove that $T$ is injective if and only if $v_1, ..., v_m$ is linearly independent. If $T$ is injective, then for all scalars $\alpha_1, ..., \alpha_m \in \F$ satisfying $\alpha_1 v_1 + ... + \alpha_m v_m = 0$, we can rewrite the equation as 
        $$T(\alpha_1, ..., \alpha_m) = 0.$$
        But notice that $T(0,...,0) = 0$ as well so have
        $$T(\alpha_1, ..., \alpha_m) = T(0,...,0).$$
        Thus, by injectivity, we get that $\alpha_1 = ... = \alpha_m = 0$. Therefore, the list is linearly independent. \\
        Suppose now that the list is linearly independent and let $(z_1, ..., z_m) \in \F^m$ be an element in the null space of $T$, then 
        $$T(z_1, ..., z_m) = 0,$$
        which can be rewritten as 
        $$z_1 v_1 + ... + z_m v_m = 0.$$
        By linear independence of $v_1, ..., v_m$, we obtain that $z_1 = ... = z_m = 0$. Thus, the null space of $T$ is precisely $\{0\}$. Therefore, by Proposition 3.15, we get that $T$ is injective. \\
    \end{enumerate}
\end{solution}

\begin{exercise}
    Show that $\{T \in \L(\R^5, \R^4) : \dim \Null T > 2\}$ is not a subspace of $\L(\R^5, \R^4)$. \\
\end{exercise}

\begin{solution}
    \\ Consider the maps $T_1, T_2 \in \L(\R^5, \R^4)$ defined by
    $$T_1(x_1, x_2, x_3, x_4, x_5) = (x_1, 0, 0, 0)$$
    $$T_2(x_1, x_2, x_3, x_4, x_5) = (0, x_2, 0, 0)$$
    for all $(x_1, x_2, x_3, x_4, x_5) \in \R^5$. Both $T_1, T_2$ have a 1-dimensional range so by the Fundamental Theorem of Linear Maps, both $T_1$ and $T_2$ have 3-dimensional null space. It follows that both $T_1$ and $T_2$ are in $\{T \in \L(\R^5, \R^4) : \dim \Null T > 2\}$. However, notice that
    $$(T_1 + T_2)(x_1, x_2, x_3, x_4, x_5) = (x_1, x_2, 0, 0)$$
    so it has a 2-dimensional range. Thus, by the Fundamental Theorem of Linear Maps, $\dim \Null (T_1 + T_2) = 2$ so $T_1 + T_2 \notin \{T \in \L(\R^5, \R^4) : \dim \Null T > 2\}$. Therefore, the set is not closed under addition which proves that it cannot be a subspace of $\L(\R^5, \R^4)$. \\
\end{solution}

\begin{exercise}
    Give an example of $T \in \L(\R^4)$ such that $\range T = \Null T$. \\
\end{exercise}

\begin{solution}
    \\ Consider the map $T \in \L(\R^4)$ defined by
    $$T(a,b,c,d) = (c, d, 0, 0)$$
    for all $a,b,c,d \in \R$. To prove that $\range T = \Null T$, let $(a,b,c,d) \in \range T$, then there exist real number $x_1, x_2, x_3, x_4$ such that
    $$(a,b,c,d) = T(x_1, x_2, x_3, x_4).$$
    By definition of $T$, we get that
    $$(a,b,c,d) = (x_3, x_4, 0, 0)$$
    which implies that $c = d = 0$. Hence, 
    $$T(a,b,c,d) = (c,d,0,0) = 0$$
    so $(a,b,c,d) \in \Null T$. To prove the reverse inclusion, let $(a,b,c,d) \in \Null T$, then
    $$T(a,b,c,d) = 0,$$
    so by definition of $T$:
    $$(c,d,0,0) = 0.$$
    It follows that $(a,b,c,d) = (a,b,0,0)$ which shows that we can write $(a,b,c,d)$ as the image of $(0,0,a,b)$, i.e., $(a,b,c,d)$ is in the range of $T$. Therefore, both sets are equal. \\
\end{solution}

\begin{exercise}
    Prove that there does not exist $T \in \L(\R^5)$ such that $\range T = \Null T$. \\
\end{exercise}

\begin{solution}
    \\ By contradiction, suppose that such a linear map $T$ exists, then by the Fundamental Theorem of Linear maps, we get that 
    $$\dim \R^5 = \dim \Null T + \dim \range T$$
    which we can rewrite as 
    $$5 = 2 \dim \range T.$$
    But this is a contradiction since 5 is odd. Therefore, no such transformation exists. \\
\end{solution}

\begin{exercise}
    Suppose $V$ and $W$ are finite-dimensional with $2 \leq \dim V \leq \dim W$. Show that $\{T \in \L(V,W) : T \text{ is not injective}\}$ is not a subspace of $\L(V,W)$. \\
\end{exercise}

\begin{solution}
    \\ Let $n = \dim V$, $m = \dim W$, let $v_1, ..., v_n$ be a basis of $V$ and $w_1, ..., w_m$ be a basis of $W$. Define the linear maps $T_1, T_2 \in \L(V,W)$ by
    $$T_1 v_i = \begin{cases} w_1 & \text{if } i= 1, \\ 0 & \text{if } i \geq 2. \end{cases}$$
    and
    $$T_2 v_i = \begin{cases} 0 & \text{if } i= 1, \\ w_i & \text{if } i \geq 2. \end{cases}$$
    Notice that both maps are not injective since for each of them, you can find a non-zero vector such that the map evaluated at this vector is 0 (for example, $T_1v_2 = 0$ and $T_2 v_1 = 0$). It follows that both maps are in $\{T \in \L(V,W) : T \text{ is not injective}\}$. Consider now the map $T = T_1 + T_2 \in \L(V,W)$, then $T$ satisfies
    $$T v_i = w_i$$
    for all $i$ between 1 and $n$. Since the list $w_1, ..., w_n$ is linearly independent, then for all $v = \alpha_1 v_1 + ... + \alpha_n v_n \in V$:
    \begin{align*}
        Tv = 0 &\iff T(\alpha_1 v_1 + ... + \alpha_n v_n) = 0 \\
        &\iff \alpha_1 w_1 + ... + \alpha_n w_n = 0 \\
        &\iff \alpha_1 = ... =  \alpha_n = 0 \\
        &\iff v = 0.
    \end{align*}
    It follows from Proposition 3.15 that $T$ is injective so $T \notin \{T \in \L(V,W) : T \text{ is not injective}\}$. Therefore, it is not a subspace of $\L(V,W)$ since it is not closed under addition.\\
\end{solution}

\begin{exercise}
    Suppose $V$ and $W$ are finite-dimensional with $\dim V \geq \dim W \geq 2$. Show that $\{T \in \L(V,W) : T \text{ is not surjective}\}$ is not a subspace of $\L(V,W)$. \\
\end{exercise}

\begin{solution}
    \\ Let $n = \dim V$, $m = \dim W$, let $v_1, ..., v_n$ be a basis of $V$ and $w_1, ..., w_m$ be a basis of $W$. Define the linear maps $T_1, ..., T_m \in \L(V,W)$ by $$T_i v_k = \begin{cases} w_i & \text{if } k= i, \\ 0 & \text{if } k \neq i. \end{cases}$$
    For each $i$, the range of $T_i$ is equal to the span of $w_i$ which is 1-dimensional. Since $\dim W \geq 2$, then the range of $T_i$ must be a proper subset of $W$ which implies that all $T_i$'s are not surjective. It follows that $T_i \in \{T \in \L(V,W) : T \text{ is not surjective}\}$ for all $i$ between 1 and $m$. Consider now the map $\tilde{T} = T_1 + ... + T_m$. For all $w \in W$, there exist scalars $\alpha_1, ..., \alpha_m$ such that $w = \alpha_1 w_1 + ... + \alpha_m w_m$. Thus, by linearity and definition of $\tilde{T}$, we get that
    \begin{align*}
        \tilde{T}(\alpha_1 v_1 + ... + \alpha_m v_m) &= T_1(\alpha_1 v_1 + ... + \alpha_m v_m) + ... + T_m(\alpha_1 v_1 + ... + \alpha_m v_m). \\
        &= [\alpha_1 T_1v_1 + ... + \alpha_m T_1v_m] + ... + [\alpha_1 T_mv_1 + ... + \alpha_m T_mv_m] \\
        &= [\alpha_1 w_1 + 0 + ... + 0] + ... + [0 + ... + 0 + \alpha_m w_m] \\
        &= \alpha_1 w_1 + ... + \alpha_m w_m \\
        &= w
    \end{align*}
    Hence, $w \in \range \tilde{T}$. Since it holds for all $w \in W$, then $\tilde{T}$ is surjective so $\tilde{T} \notin \{T \in \L(V,W) : T \text{ is not surjective}\}$. Therefore, it is not a subspace of $\L(V,W)$ since it is not closed under addition. \\
\end{solution}

\begin{exercise}
    Suppose $T \in \L(V,W)$ is injective and $v_1, ..., v_n$ is linearly independent in $V$. Prove that $Tv_1, ..., Tv_n$ is linearly independent in $W$. \\
\end{exercise}

\begin{solution}
    \\ Let $\alpha_1, ..., \alpha_n$ be scalars such that
    $$\alpha_1 Tv_1 + ... + \alpha_n Tv_n = 0,$$
    then by linearity,
    $$T(\alpha_1 v_1 + ... + \alpha_n v_n) = 0,$$
    by injectivity,
    $$\alpha_1 v_1 + ... + \alpha_n v_n = 0,$$
    and by linear independence of the list $v_1, ..., v_n$, we obtain that $\alpha_1 = ... = \alpha_n = 0$. Therefore, the list $Tv_1, ..., Tv_n$ is linearly independent. \\
\end{solution}

\begin{exercise}
    Suppose $v_1, ..., v_n$ spans $V$ and $T \in \L(V,W)$. Show that $Tv_1, ..., Tv_n$ spans $\range T$. \\
\end{exercise}

\begin{solution}
    \\ Let $w \in \range T$, then there exists a vector $v \in V$ such that $w = Tv$. Since $v_1, ..., v_n$ spans $V$, there exist scalars $\alpha_1, ..., \alpha_n \in \F$ such that
    $$v = \alpha_1 v_1 + ... + \alpha_n v_n.$$
    If we apply $T$ on both sides, we get that
    $$Tv = T(\alpha_1 v_1 + ... + \alpha_n v_n).$$
    By linearity of $T$ and definition of $v$, we get that
    $$w = \alpha_1 Tv_1 + ... + \alpha_n Tv_n.$$
    Since it holds for all $w \in \range T$, then the list $Tv_1, ..., Tv_n$ spans $\range T$. \\
\end{solution}

\begin{exercise}
    Suppose that $V$ is finite-dimensional and thet $T \in \L(V,W)$. Prove that there exists a subspace $U$ of $V$ such that
    $$U \cap \Null T = \{0\} \quad \text{ and } \quad \range T = \{Tu : u \in U\}.$$
\end{exercise}

\begin{solution}
    \\ Since $\Null T$ is a subspace of $V$, then by Proposition 2.33, there exists a subspace $U$ of $V$ such that $V = U \oplus \Null T$. It follows that $U \cap \Null T = \{0\}$. Moreover, since $V = U + \Null T$, then 
    \begin{align*}
        \range T &= \{Tv : v\in V\} \\
        &= \{T(u + n) : u \in U \text{ and } n \in \Null T\} \\
        &= \{Tu + Tn : u \in U \text{ and } n \in \Null T\} \\
        &= \{Tu : u \in U \text{ and } n \in \Null T\} \\
        &= \{Tu : u \in U\}
    \end{align*}
    Therefore, $U$ satisfies all the required properties. \\
\end{solution}

\begin{exercise}
    Suppose $T$ is a linear map from $\F^4$ to $\F^2$ such that
    $$\Null T = \{(x_1, x_2, x_3, x_4) \in \F^4 : x_1 = 5x_2 \text{ and } x_3 = 7x_4 \}.$$
    Prove that $T$ is surjective. \\
\end{exercise}

\begin{solution}
    \\ First, let's prove that $\Null T$ has dimension 2. Consider the vectors $u_1 = (5, 1, 0, 0)$ and $u_2 = (0, 0, 7, 1)$. Obviously, the list $u_1, u_2$ is linearly independent and is contained in $\Null T$. Moreover, for all $u \in U$, we know that there exist $x_2, x_4 \in \F$ such that $u = (5x_2, x_2, 7x_4, x_4)$. It follows that $u = x_2 u_1 + x_4 u_2$. Therefore, $u_1, u_2$ spans $\Null T$ which proves that it is a basis and that $\Null T$ has dimension 2. Thus, by the Fundamental Theorem of Linear Maps, we have
    $$\dim \F^4 = \dim \Null T + \dim \range T$$
    which we can now write as
    $$4 = 2 + \dim \range T.$$
    Therefore, $\dim \range T = 2 = \dim \F^2$. Since $\range T$ is a subspace of $\F^2$ of dimension 2, then $\range T = \F^2$. It follows that $T$ is surjective. \\
\end{solution}

\begin{exercise}
    Suppose $U$ is a three-dimensional subspace of $\R^8$ and that $T$ is a linear map from $\R^8$ to $\R^5$ such that $\Null T = U$. Prove that $T$ is surjective. \\
\end{exercise}

\begin{solution}
    \\ By the Fundamental Theorem of Linear Maps, we know that
    $$\dim \R^8 = \dim U + \dim \range T.$$
    By plugging-in the known values, we get 
    $$8 = 3 + \dim \range T.$$
    Thus, $\dim \range T = 5 = \dim \R^5$. Since $\range T$ is a subspace of $\R^5$ of dimension 5, then $\range T = \R^5$. Therefore, $T$ is surjective. \\
\end{solution}

\begin{exercise}
    Prove that there does not exist a linear map from $\F^5$ to $\F^2$ whose null space equals $\{(x_1, x_2, x_3, x_4, x_5) \in \F^5 : x_1 = 3x_2 \text{ and } x_3 = x_4 = x_5\}.$ \\
\end{exercise}

\begin{solution}
    \\ Define $U = \{(x_1, x_2, x_3, x_4, x_5) \in \F^5 : x_1 = 3x_2 \text{ and } x_3 = x_4 = x_5\}$ and let's determine its dimension. Consider the vectors $u_1 = (3,1,0,0,0)$ and $u_2 = (0,0,1,1,1)$ and notice that the list $u_1, u_2$ is linearly independent and contained in $U$. Moreover, for all $u = (x_1, x_2, x_3, x_4, x_5) \in U$, we have $x_1 = 3x_2$ and $x_3 = x_4 = x_5$. It follows that
    $$u = (3x_2, x_2, x_3, x_3, x_3) = x_2 u_1 + x_3 u_2.$$
    Therefore, the list $u_1, u_2$ spans $U$ so it is a basis of $U$. Hence, $\dim U = 2$. By contradiction, suppose that there is a map $T \in \L(\F^5, \F^2)$ such that $\Null T = U$, then by the Fundamental Theorem of Linear Maps, we have
    $$\dim \F^5 = \dim U + \dim \range T.$$
    If we plug-in the known values, we get that $\dim \range T = 3$. However, since $\range T$ is a subspace of $\F^2$, then
    $$3 = \dim \range T \leq \dim \F^2 = 2,$$
    a contradiction. Therefore, there does not exist a linear map from $\F^5$ to $\F^2$ whose null space equals $U$.\\
\end{solution}

\begin{exercise}
    Suppose there exists a linear map on $V$ whose null space and range are both finite-dimensional. Prove that $V$ is finite-dimensional. \\
\end{exercise}

\begin{solution}
    \\ Notice that we cannot use the Fundamental Theorem of Linear Maps since one of the assumptions of the Theorem is that $V$ is finite-dimensional. Let $u_1, ..., u_n$ be a basis of $\Null T$ and $v_1, ..., v_m$ be a basis of $\range T$. For each $i$ between 1 and $m$, define $w_i$ as a vector in $V$ satisfying $Tw_i = v_i$. Now, let $v \in V$ and let $\alpha_1, ..., \alpha_m \in \F$ be scalars such that
    $$Tv = \alpha_1 v_1 + ... + \alpha_m v_m,$$
    then by definition of the $w_i$'s and by linearity of $T$, we get that
    $$Tv = T(\alpha_1 w_1 + ... + \alpha_m w_m).$$
    By rearranging the terms, we get that
    $$T(v - [\alpha_1 w_1 + ... + \alpha_m w_m]) = 0$$
    so $v - (\alpha_1 w_1 + ... + \alpha_m w_m) \in \Null T$. It follows that there exist scalars $\beta_1, ..., \beta_n$ such that
    $$v - (\alpha_1 w_1 + ... + \alpha_m w_m) = \beta_1 u_1 + ... + \beta_n u_n.$$
    Thus,
    $$v = \beta_1 u_1 + ... + \beta_n u_n + \alpha_1 w_1 + ... + \alpha_m w_m$$
    which shows that $v$ is in the span of the list $u_1, ..., u_n, w_1, ..., w_m$. Since it holds for all $v \in V$, then the list $u_1, ..., u_n, w_1, ..., w_m$ spans $V$. Therefore, by definition, $V$ is finite-dimensional.\\
\end{solution}

\begin{exercise}
    Suppose $V$ and $W$ are both finite-dimensional. Prove that there exists an injective linear map from $V$ to $W$ if and only if $\dim V \leq \dim W$. \\
\end{exercise}

\begin{solution}
    \\ $(\implies)$ This direction is simply the converse of Proposition 3.22.\\
    $( \ \Longleftarrow \ )$ Suppose that $\dim V \leq \dim W$, let $v_1, ..., v_n$ be a basis of $V$ and $w_1, ..., w_m$ a basis of $W$. Define the map $T \in \L(V,W)$ by 
    $$Tv_i = w_i$$
    for all $i$ between 1 and $n$ (which can be done since $n \leq m$). Notice that by linear independence of $w_1, ..., w_n$, we get that for all $v = \alpha_1v_1 + ... + \alpha_n v_n \in V$,
    \begin{align*}
        Tv = 0 &\iff T(\alpha_1v_1 + ... + \alpha_n v_n) = 0 \\
        &\iff \alpha_1w_1 + ... + \alpha_n w_n = 0 \\
        &\iff \alpha_1 = ... = \alpha_n = 0 \\
        &\iff v = 0.
    \end{align*}
    Therefore, $\Null T = \{0\}$ which proves that $T$ is injective.\\
\end{solution}

\begin{exercise}
    Suppose $V$ and $W$ are both finite-dimensional. Prove that there exists a surjective linear map from $V$ to $W$ if and only if $\dim V \geq \dim W$. \\
\end{exercise}

\begin{solution}
    \\ $(\implies)$ This direction is simply the converse of Proposition 3.24.\\
    $( \ \Longleftarrow \ )$ Suppose that $\dim V \geq \dim W$, let $v_1, ..., v_n$ be a basis of $V$ and $w_1, ..., w_m$. Consider the linear transformation $T \in \L(V,W)$ defined by
    $$Tv_i = \begin{cases} w_i & \text{if } 1 \leq i \leq m, \\ 0 & \text{if } m+1 \leq i \leq n. \end{cases}$$
    Let $w = \alpha_1w_1 + ... + \alpha_m w_m \in V$, then
    $$T(\alpha_1v_1 + ... + \alpha_m v_m) = \alpha_1w_1 + ... + \alpha_m w_m = w$$
    which proves that $w \in \range T$. Therefore, $T$ is surjective. \\
\end{solution}

\begin{exercise}
    Suppose $V$ and $W$ are finite-dimensional and that $U$ is a subspace of $V$. Prove that there exists $T \in \L(V,W)$ such that $\Null T = U$ if and only if $\dim U \geq \dim V - \dim W$. \\
\end{exercise}

\begin{solution}
    \\ $(\implies)$ Suppose that there exists $T \in \L(V,W)$ such that $\Null T = U$. Since $\range T$ is a subspace of $W$, then $\dim \range T \leq \dim W$. It follows by the Fundamental Theorem of Linear maps that
    $$\dim V = \dim \Null T + \dim \range T \leq \dim U + \dim W$$
    which we can rearrange into 
    $$\dim U \geq \dim V - \dim W.$$
    $( \ \Longleftarrow \ )$ Suppose now that $\dim U \geq \dim V - \dim W$, let $u_1, ..., u_n$ be a basis of $U$, extend it to a basis $u_1, ..., u_n, v_1, ..., v_m$ of $V$ and let $w_1, ..., w_k$ be a basis of $W$. Consider the map $T \in \L(V,W)$ defined by
    $$Tu_i = 0 \quad \text{ and } \quad Tv_j = w_j$$
    for all $i \in \{1, ..., n\}$ and $j \in \{1, ..., m\}$. This can be done because by our assumptions, $k \geq m$. Let's find the null space of $T$. First, by construction, $U \subset \Null T$. Moreover, for all $v \in V$, there exist scalars $\alpha_1, ..., \alpha_n, \beta_1, ..., \beta_m \in \F$ such that
    $$v = \alpha_1u_1 + ... + \alpha_n u_n + \beta_1 v_1 + ... + \beta_m v_m.$$
    Thus, 
    \begin{align*}
        Tv = 0 &\implies T(\alpha_1u_1 + ... + \alpha_n u_n + \beta_1 v_1 + ... + \beta_m v_m) = 0\\
        &\implies \alpha_1Tu_1 + ... + \alpha_n Tu_n + \beta_1 Tv_1 + ... + \beta_m Tv_m = 0 \\
        &\implies 0 + ... + 0 + \beta_1 w_1 + ... + \beta_m w_m = 0 \\
        &\implies \beta_1 = ... = \beta_m = 0 \\
        &\implies v = \alpha_1u_1 + ... + \alpha_n u_n. \\
        &\implies v \in U.
    \end{align*}
    Therefore, $\Null T = U$.\\
\end{solution}

\begin{exercise}
    Suppose $W$ is finite-dimensional and $T \in \L(V,W)$. Prove that $T$ is injective if and only if there exists $S \in \L(W,V)$ such that $ST$ is the identity operator on $V$. \\
\end{exercise}

\begin{solution}
    \\ $(\implies)$ Suppose that $T$ is injective, then $\dim \Null T = \dim \{0\} = 0$ and $\dim \range T \leq \dim W < \infty$. Thus, by Exercise 15, $V$ is finite-dimensional. Let $v_1, ..., v_n$ be a basis of $V$, then by Exercise 8 and 9, $Tv_1, ..., Tv_n$ is a basis of $\range T$. It can be extended to a basis $Tv_1,..., Tv_n, w_1, ..., w_m$ of $W$. Define $S \in \L(W,V)$ by $S(Tv_i) = v_i$ for all $i \in \{1, ..., n\}$ and $Sw_i = 0$ for all $i \in \{1, ..., n\}$, then it follows that the operator $ST \in \L(V)$ satisfies $(ST)v_i = v_i$ for all $i \in \{1, ..., n\}$. By uniqueness in Lemma 3.4, it follows that $ST$ is the identity operator on $V$. \\
    $( \ \Longleftarrow \ )$ Suppose there exists $S \in \L(W,V)$ such that $ST$ is the identity operator on $V$, then for all $x, y \in V$, we have
    $$Tx = Ty \implies S(Tx) = S(Ty) \implies (ST)x = (ST)y \implies x = y.$$
    Therefore, $T$ is injective. \\
\end{solution}

\begin{exercise}
    Suppose $W$ is finite-dimensional and $T \in \L(V,W)$. Prove that $T$ is surjective if and only if there exists $S \in \L(W,V)$ such that $TS$ is the identity operator on $W$. \\
\end{exercise}

\begin{solution}
    \\ $(\implies)$ Suppose that $T$ is surjective and let $w_1, ..., w_n$ be a basis of $W$. Define $S \in \L(W,V)$ on the basis as follows: by surjectivity, let $Sw_i$ be equal to a vector $v_i$ that satisfies $Tv_i = w_i$, then for all $i$ between 1 and $n$, we have that
    $$(TS)w_i = T(Sw_i) = Tv_i = w_i.$$
    By uniqueness in Lemma 3.4, we get that $TS$ must be the identity operator on $W$. \\
    $( \ \Longleftarrow \ )$ Suppose there exists $S \in \L(W,V)$ such that $TS$ is the identity operator on $W$, then for all $w \in W$, we have that $T(Sw) = (TS)w = w$ which proves that $w$ is in the range of $T$. Since it holds for all $w \in W$, then $T$ is surjective. \\
\end{solution}

\begin{exercise}
    Suppose $V$ is finite-dimensional, $T \in \L(V,W)$, and $U$ is a subspace of $W$. Prove that $\{v \in V : Tv \in U\}$ is a subspace of $V$ and
    $$\dim \{v \in V : Tv \in U\} = \dim \Null T + \dim (U \cap \range T).$$
\end{exercise}

\begin{solution}
    \\ Let $T^{-1}U$ be the subset of $V$ containing the $v \in V$ satisfying $Tv \in U$. Let's prove that it is a subspace of $V$. It is non-empty because $T0 = 0 \in U$. Moreover, given $u,v \in T^{-1}U$, we have that $T(u+v) = Tu + Tv \in U$ since $U$ is closed under addition. Similarly, given $v \in T^{-1}U$ and $\alpha \in \F$, we have $T(\alpha v) = \alpha Tv \in U$. Therefore, it is a subspace of $V$. If we consider the restriction of $T$ on $T^{-1}U$, we can apply the Fundamental Theorem of Linear Maps to get
    \[ \dim T^{-1}U = \dim \Null T|_{T^{-1}U} + \dim \range T|_{T^{-1}U}. \tag*{(1)} \]
    But notice that the set of $v \in V$ such that $T|_{T^{-1}U}v = 0$ is the same as the set of $v \in V$ such that $Tv = 0$ since $0 \in U$. Thus, $\dim \Null T|_{T^{-1}U} = \dim \Null T$. Moreover, if $w \in \range T|_{T^{-1}U}$, then there is a $v \in T^{-1}U \subset V$ such that $T|_{T^{-1}U}v = Tv = w$ so $w \in \range T$. Moreover, since $v \in T^{-1}U$, then $w = Tv \in U$ so $w \in U \cap \range T$. It follows that $\range T|_{T^{-1}U} \subset U \cap \range T$. Conversely, if $w \in U \cap \range T$, then there is a $v \in V$ such that $Tv = w \in U$. But since $Tv \in U$, then by definition, $v \in T^{-1}U$, so it follows that $w = Tv = T|_{T^{-1}U}v$ which implies that $w \in \range T|_{T^{-1}U}$. Thus, $\range T|_{T^{-1}U} = U \cap \range T$. Plugging this into equation (1) gives us
    $$\dim \{v \in V : Tv \in U\} = \dim \Null T + \dim(U \cap \range T).$$
\end{solution}

\begin{exercise}
    Suppose $U$ and $V$ are finite-dimensional vector spaces and $S \in \L(V,W)$ and $T \in \L(U,V)$. Prove that
    $$\dim \Null ST \leq \dim \Null S + \dim \Null T.$$
\end{exercise}

\begin{solution}
    \\ First, notice that 
    $$u \in \Null ST \iff STu = 0 \iff Tu \in \Null S \iff u \in \{v \in U : Tv \in \Null S\}$$
    which implies that $\Null ST = \{v \in U : Tv \in \Null S\}$. Therefore, by Exercise 21, we get that
    \begin{align*}
        \dim \Null ST &= \dim \{v \in U : Tv \in \Null S\} \\
        &= \dim \dim \Null T + \dim(\Null S \cap \range T).
    \end{align*}
    But since $\Null S \cap \range T$ is a subspace of $\Null S$, then its dimension is less than or equal to $\dim \Null S$. Therefore,
    $$\dim \Null ST \leq \dim \Null T + \dim \Null S$$
    which is the desired inequality.\\
\end{solution}

\begin{exercise}
    Suppose $U$ and $V$ are finite-dimensional vector spaces and $S \in \L(V,W)$ and $T \in \L(U,V)$. Prove that
    $$\dim \range ST \leq \min\{\dim \range S, \dim \range T\}.$$
\end{exercise}

\begin{solution}
    \\ It suffices to show that $\dim \range ST$ is less than both $\dim \range S$ and $\dim \range T$. First, notice that for all $w \in \range ST$, there exists a $u \in U$ such that $w = STu = S(Tu)$. It follows that $w \in \range S$ so $\range ST \subset \range S$. Since both are vector spaces, then $\range ST$ is a subspace of $\range S$ which proves that $\dim \range ST \leq \dim \range S$. Consider now the restriction $S_0$ of $S$ on the subspace $\range T$ of $V$, then by the Fundamental Theorem of Linear Maps, we get that
    $$\dim \range T = \dim \Null S_0 + \dim \range S_0 \geq \dim \range S_0.$$
    But notice that if $w \in \range ST$, then there exists a $u \in U$ such that $w = STu = S(Tu)$. Since $Tu \in \range T$, then $w = S(Tu) \in \range S_0$ so we get that $\range ST \subset \range S_0$. Since both are vector spaces, we have that
    $$\dim \range ST \leq \dim \range S_0 \leq \dim \range T.$$
    Therefore, $\dim \range ST$ is less than both $\dim \range S$ and $\dim \range T$. \\
\end{solution}

\begin{exercise}
    \vspace{-5.5mm}
    \begin{enumerate}[label=(\alph*)]
        \item Suppose $\dim V = 5$ and $S,T \in \L(V)$ are such that $ST = 0$. Prove that $\dim \range TS \leq 2$.
        \item Given an example of $S,T \in \L(\F^5)$ with $ST = 0$ and $\dim \range TS = 2$. \\
    \end{enumerate}
\end{exercise}

\begin{solution}
    \begin{enumerate}[label=(\alph*)]
        \item First, notice that if $v \in \range T$, then there exists a $w \in V$ such that $v = Tw$. Thus, since $ST = 0$, then $Sv = STw = 0$ so $v \in \Null S$. It follows that $\range T \subset \Null S$. Since both are vector spaces, then $\range T$ is a subspace of $\Null S$ which implies that $\dim \range T \leq \dim \Null T$. Now, using the previous exercise, we get that
        $$\dim \range TS \leq \dim \range T \leq \dim \Null S$$
        and
        $$\dim \range TS \leq \dim \range S.$$
        If we suppose by contradiction that $\dim \range TS \geq 3$, then we get that
        $$3 \leq \dim \Null S \quad \text{ and } \quad 3\leq  \dim \range S.$$
        By adding both inequalities and using the Fundamental Theorem of Linear Maps, we conclude that
        $$6 \leq \dim \Null S + \dim \range S = \dim V = 5$$
        which is a contradiction. Therefore, $\dim \range TS \leq 2$.
        \item Consider the maps $S,T \in \L(\F^5)$ defined by
        $$S(x_1, x_2, x_3, x_4, x_5) = (x_1, x_2, x_3, 0, 0)$$
        $$T(x_1, x_2, x_3, x_4, x_5) = (0, 0, 0, x_1, x_2).$$
        for all $(x_1, x_2, x_3, x_4, x_5) \in \F^5$. Notice that for all $(x_1, x_2, x_3, x_4, x_5) \in \F^5$, we have
        $$ST(x_1, x_2, x_3, x_4, x_5) = S(0, 0, 0, x_1, x_2) = 0$$
        which shows that $ST = 0$. Moreover, for all $(x_1, x_2, x_3, x_4, x_5) \in \F^5$, we have 
        $$TS(x_1, x_2, x_3, x_4, x_5) = T(x_1, x_2, x_3, 0, 0) = (0, 0, 0, x_1, x_2)$$
        which implies that
        $$\range TS = \{(0, 0, 0, x_1, x_2) : x_1, x_2\in \F^5\}.$$
        Obviously, $(0, 0, 0, 1, 0), (0, 0, 0, 0, 1)$ is a basis of $\range TS$ so $\dim \range TS = 2$. \\
    \end{enumerate}
\end{solution}

\begin{exercise}
    Suppose that $W$ is finite-dimensional and $S,T \in \L(V,W)$. Prove that $\Null S \subset \Null T$ if and only if there exists $E \in \L(W)$ such that $T = ES$. \\
\end{exercise}

\begin{solution}
    \\ \td \\
\end{solution}

\begin{exercise}
    Suppose that $V$ is finite-dimensional and $S,T \in \L(V,W)$. Prove that $\range S \subset \range T$ if and only if there exists $E \in \L(V)$ such that $S = TE$. \\
\end{exercise}

\begin{solution}
    \\ \td \\
\end{solution}

\begin{exercise}
    Suppose $P \in \L(V)$ and $P^2 = P$. Prove that $V = \Null P \oplus \range P$. \\
\end{exercise}

\begin{solution}
    \\ \td \\
\end{solution}

\begin{exercise}
    Suppose $D \in \L(\P(\R))$ is such that $\deg Dp = (\deg p) - 1$ for every non-constant polynomial $p \in \P(\R)$. Prove that $D$ is surjective. \\
\end{exercise}

\begin{solution}
    \\ \td \\
\end{solution}

\begin{exercise}
    \td \\
\end{exercise}

\begin{solution}
    \\ \td \\
\end{solution}

\begin{exercise}
    \td \\
\end{exercise}

\begin{solution}
    \\ \td \\
\end{solution}

\begin{exercise}
    \td \\
\end{exercise}

\begin{solution}
    \\ \td \\
\end{solution}

\begin{exercise}
    \td \\
\end{exercise}

\begin{solution}
    \\ \td \\
\end{solution}

\begin{exercise}
    \td \\
\end{exercise}

\begin{solution}
    \\ \td \\
\end{solution}
