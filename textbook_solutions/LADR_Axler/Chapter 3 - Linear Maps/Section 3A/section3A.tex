\chapter{Linear Maps}

\section{Vector Space of Linear Maps}

\begin{exercise}
    Suppose $b,c \in \R$. Define $T: \R^3 \to \R^2$ by
    $$T(x,y,z) = (2x-4y + 3z + b, 6x + cxyz).$$
    Show that $T$ is linear if and only if $b = c = 0$. \\
\end{exercise}

\begin{solution}
    \\ ($\implies$) Suppose that $T$ is linear, then we know from Proposition 3.10 that $T0 = 0$. Thus, it follows that
    $$T(0,0,0) = (b, 0) = (0,0)$$
    which implies that $b = 0$. To prove that $c = 0$, notice that by linearity of $T$, we have
    $$T(2,2,2) = 2T(1,1,1).$$
    If we plug-in the values into the definition of $T$, we get
    $$(4 - 8 + 6 + 0, 12 + 8c) = 2(2 - 4 + 3 + 0, 6 + c)$$
    which is equivalent to
    $$(2, 12 + 8c) = (2, 12 + 2c).$$
    It follows that $12 + 8c = 12 + 2c$ which can only be true when $c = 0$. Thus, $b = c = 0$. \\
    $( \ \Longleftarrow \ )$ Suppose now that $b = c = 0$, then $T(x,y,z)$ becomes
    $$T(x,y,z) = (2x-4y + 3z, 6x)$$
    for all $x,y,z \in \R$. Let's show that $T$ is linear. First, take $(x,y,z), (x',y',z') \in \R^3$ and notice that
    \begin{align*}
        T((x,y,z) + (x',y',z')) &= T(x+x', y+y', z+z') \\
        &= (2(x + x')-4(y+y') + 3(z+z'), 6(x+x')) \\
        &= (2x - 4y + 3z + 2x' - 4y' + 3z', 6x + 6x') \\
        &= (2x-4y + 3z, 6x) + (2x'-4y' + 3z', 6x') \\
        &= T(x,y,z) + T(x',y',z')
    \end{align*}
    Moreover, given any $\lambda \in \R$ and $(x,y,z) \in \R^3$, we have
    \begin{align*}
        T(\lambda(x,y,z)) &= T(\lambda x, \lambda y, \lambda z) \\
        &= (2(\lambda x)-4(\lambda y) + 3(\lambda z), 6(\lambda x)) \\
        &= (\lambda(2x - 4y + 3z), \lambda(6x)) \\
        &= \lambda(2x - 4y + 3z, 6x) \\
        &= \lambda T(x,y,z)
    \end{align*}
    Therefore, $T$ is linear.\\
\end{solution}

\begin{exercise}
    Suppose $b,c \in \R$. Define $T : \mathcal{P}(\R) \to \R^2$ by
    $$Tp = \left(3p(4) + 5p'(6) + bp(1)p(2), \int_{-1}^{2}x^3p(x)dx + c \sin p(0) \right).$$
    Show that $T$ is linear if and only if $b = c = 0$. \\
\end{exercise}

\begin{solution}
    \\ ($\implies$) Suppose that $T$ is linear, then if we let $p$ be the constant polynomial equal to $\pi/2$, we get that $T$ must satisfy
    $$T(2p) = 2Tp.$$
    If we rewrite this using the definition of $T$ and $p$, we obtain
    $$\left(3\pi + b\pi^2, \pi\int_{-1}^{2}x^3dx + c\sin(\pi)\right) = 2\left(3\frac{\pi}{2} + b\frac{\pi^2}{4}, \frac{\pi}{2}\int_{-1}^{2}x^3dx + c\sin\left(\frac{\pi}{2}\right)\right)$$
    which can be simplified to
    $$\left(3\pi + b\pi^2, \pi\int_{-1}^{2}x^3dx\right) = \left(3\pi + b\frac{\pi^2}{2}, \pi\int_{-1}^{2}x^3dx + c\right).$$
    This gives us the following system of equations:
    $$\begin{cases}
        3\pi + b\pi^2 = 3\pi + b\frac{\pi^2}{2} \\
        \pi\int_{-1}^{2}x^3dx = \pi\int_{-1}^{2}x^3dx + c
    \end{cases} 
    \implies
    \begin{cases}
        b = \frac{1}{2}b \\
        c = 0
    \end{cases}
    \implies b = c = 0.$$
    $( \ \Longleftarrow \ )$ Suppose that $b = c = 0$, then for all $p \in \mathcal{P}(\R)$, we have
    $$Tp = \left(3p(4) + 5p'(6), \int_{-1}^{2}x^3p(x)dx \right).$$
    Thus, for any $p_1, p_2 \in \mathcal{P}(\R)$, we get 
    \begin{align*}
        T(p_1 + p_2) &= \left(3(p_1 + p_2)(4) + 5(p_1 + p_2)'(6), \int_{-1}^{2}x^3(p_1 + p_2)(x)dx \right) \\
        &= \left(3(p_1(4) + p_2(4)) + 5(p_1'(6) + p_2'(6)), \int_{-1}^{2}x^3(p_1(x) + p_2(x))dx \right) \\
        &= \left(3p_1(4) + 3p_2(4) + 5p_1'(6) + 5p_2'(6), \int_{-1}^{2}x^3p_1(x)dx + \int_{-1}^{2}x^3p_2(x)dx \right) \\
        &= \left(3p_1(4) + 5p_1'(6), \int_{-1}^{2}x^3p_1(x)dx \right) + \left(3p_2(4) + 5p_2'(6), \int_{-1}^{2}x^3p_2(x)dx \right) \\
        &= Tp_1 + Tp_2.
    \end{align*}
    Similarly, for all $\lambda \in \R$ and $p \in \mathcal{P}(\R)$, we have
    \begin{align*}
        T(\lambda p) &= \left(3(\lambda p)(4) + 5(\lambda p)'(6), \int_{-1}^{2}x^3(\lambda p)(x)dx \right) \\
        &= \left(3\lambda p(4) + 5\lambda p'(6), \int_{-1}^{2}x^3 \lambda p(x)dx \right) \\
        &= \left(\lambda (3p(4) + 5p'(6)), \lambda\int_{-1}^{2}x^3 p(x)dx \right) \\
        &= \lambda \left(3p(4) + 5p'(6), \int_{-1}^{2}x^3 p(x)dx \right) \\
        &= \lambda Tp.
    \end{align*}
    Therefore, $T$ is linear. \\
\end{solution}

\begin{exercise}
    Suppose that $T \in \L(\F^n, \F^m)$. Show that there exist scalars $A_{j,k} \in \F$ for $j = 1, ..., m$ and $k= 1, ..., n$ such that 
    $$T(x_1, ..., x_n) = (A_{1,1}x_1 + \dots + A_{1,n}x_n, ..., A_{m,1}x_1 + \dots + A_{m,n}x_n)$$
    for every $(x_1, ..., x_n) \in \F^n$. \\
\end{exercise}

\begin{solution}
    \\ Denote by $e_1, ..., e_n$ the standard basis of $\F^n$ and by $f_1, ..., f_m$ the standard basis for $\F^m$, then for all $k \in \{1, ..., n\}$, there exist scalars $A_{1,k}, ..., A_{m,k} \in \F$ such that
    $$Te_k = A_{1,k}f_1 + \dots + A_{m,k}f_m.$$
    Therefore, by linearity, for all $(x_1, ..., x_n) \in \F^n$:
    \begin{align*}
        T(x_1, ..., x_n) &= x_1Te_1 + \dots + x_n Te_n \\
        &= x_1(A_{1,1}f_1 + \dots + A_{m,1}f_m) + \dots + x_n(A_{1,n}f_1 + \dots + A_{m,n}f_m) \\
        &= (A_{1,1}x_1 + \dots + A_{1,n}x_n)f_1 + \dots + (A_{m,1}x_1 + \dots + A_{m,n}x_n)f_m \\
        &= (A_{1,1}x_1 + \dots + A_{1,n}x_n, ..., A_{m,1}x_1 + \dots + A_{m,n}x_n).
    \end{align*}
    Therefore, any linear transformation has this form. \\
\end{solution}

\begin{exercise}
    Suppose $T \in \L(V, W)$ and $v_1, ..., v_m$ is a list of vectors in $V$ such that $Tv_1, ..., Tv_m$ is a linearly independent list in $W$. Prove that $v_1, ..., v_m$ is linearly independent. \\
\end{exercise}

\begin{solution}
    \\ To prove that $v_1, ..., v_m$ is linearly independent, take arbitrary scalars $\alpha_1, ..., \alpha_m \in \F$ such that
    $$\alpha_1 v_1 + ... + \alpha_m v_m = 0.$$
    By evaluating on both sides by $T$, we get by linearity of $T$ the following equation:
    $$\alpha_1 Tv_1 + ... + \alpha_m Tv_m = 0.$$
    But since the list $Tv_1, ..., Tv_m$ is a linearly independent in $W$, then 
    $$\alpha_1 = ... = \alpha_m = 0$$
    which proves that $v_1, ..., v_m$ is linearly independent. \\
\end{solution}

\begin{exercise}
    Prove that $\L(V, W)$ is a vector space, as was asserted in 3.6. \\
\end{exercise}

\begin{solution}
    \\ We already proved in Section 1B Exercise 7 that for any nonempty set $S$ and vector space $U$, the set $U^S$ equipped with the usual addition and scalar multiplication is a vector space. Hence, if we let $S = V$ and $U = W$, we already know that the set of functions from $V$ tp $W$ is a vector space. Since $\L(V, W) \subset W^V$, then it suffices to show that $\L(V, W)$ is a subspace. \\
    First, notice that $\L(V, W)$ is non-empty since it contains the additive identity map: the constant zero map is linear. Given two linear maps $T_1, T_2 \in \L(V, W)$, we can show that $T_1 + T_2 \in \L(V, W)$ by proving that it is a linear map from $V$ to $W$. Hence, take arbitrary $x,y \in V$ and $\lambda \in \F$ to get:
    \begin{align*}
        (T_1 + T_2)(x + y) &= T_1(x + y) + T_2(x + y) \\
        &= T_1(x) + T_1(y) + T_2(x) + T_2(y) \\
        &= (T_1 + T_2)(x) + (T_1 + T_2)(y),
    \end{align*}
    and
    \begin{align*}
        (T_1 + T_2)(\lambda x) &= T_1(\lambda x) + T_2(\lambda x) \\
        &= \lambda T_1(x) + \lambda T_2(x) \\
        &= \lambda (T_1(x) + T_2(x)) \\
        &= \lambda (T_1 + T_2)(x).
    \end{align*}
    Thus, $\L(V, W)$ is closed under addition. Similarly, given a linear map $T \in \L(V, W)$ and $\alpha \in \F$, we get that $\alpha T \in \L(V, W)$ because for all $x,y \in V$ and $\lambda \in \F$, we have the following:
    \begin{align*}
        (\alpha T)(x + y) &= \alpha T(x + y) \\
        &= \alpha (T(x) + T(y)) \\
        &= \alpha T(x) + \alpha T(y) \\
        &= (\alpha T)(x) + (\alpha T)(y)
    \end{align*}
    and
    \begin{align*}
        (\alpha T)(\lambda x) &= \alpha T(\lambda x) \\
        &= \alpha \lambda T(x) \\
        &= \lambda \alpha T(x) \\
        &= \lambda (\alpha T)(x).
    \end{align*}
    Therefore, $\L(V, W)$ is a vector space since it is a subspace of $W^V$. \\
\end{solution}

\begin{exercise}
    Prove that the multiplication of linear maps has the associative, identity and distributive properties asserted in 3.8. \\
\end{exercise}

\begin{solution}
    \vspace{-0.25cm}
    \begin{itemize}
        \item (Associativity) Let $V_1, V_2, V_3, V_4$ be vector spaces and $T_1 : V_1 \to V_2$, $T_2 : V_2 \to V_3$ and $T_3 : V_3 \to V_4$ be linear maps. Associativity follows from the fact that for all $x \in V_1$:
        \begin{align*}
            ((T_1 T_2)T_3)(x) &= (T_1 T_2)(T_3(x)) \\
            &= T_1(T_2(T_3(x))) \\
            &= T_1(T_2T_3(x)) \\
            &= (T_1(T_2T_3))(x).
        \end{align*}
        Since it holds for all $x \in X$, then $(T_1 T_2)T_3 = T_1(T_2T_3)$.
        \item (Identity) Let $V$ and $W$ be vector space. Consider the identity map $I_V : V \to W$ and let's show that it is indeed linear. For all $x,y \in V$:
        $$I_V(x + y) = x + y = I_V(x) + I_V(y)$$
        and for any $\lambda \in \F$ and $x \in V$:
        $$I_V(\lambda x) = \lambda x = \lambda I_V(x).$$
        Therefore, $I_V$ is linear. To prove that it is the multiplicative identity in $\L(V,W)$, let $T: V \to W$ be a linear map and $x \in V$, then
        $$(I_V T)(x) = I_V(Tx) = Tx$$
        and
        $$(TI_V)(x) = T(I_Vx) = Tx$$
        so $I_V T = T I_V = T$ for all linear maps $T \in \L(V,W)$.
        \item (Distributivity 1) Let $U,V,W$ be vector spaces, $S_1, S_2 \in \L(V,W)$ and $T \in \L(U,V)$, then for all $x \in V$, we have
        \begin{align*}
            [(S_1 + S_2)T](x) &= (S_1 + S_2)(Tx) \\
            &= S_1(Tx) + S_2(Tx) \\
            &= (S_1 T)(x) + (S_2 T)(x) \\
            &= [S_1 T + S_2 T](x).
        \end{align*}
        Since it holds for all $x \in U$, then $(S_1 + S_2) T = S_1 T  + S_2 T$.
        \item (Distributivity 2) Let $U,V,W$ be vector spaces, $S \in \L(V,W)$ and $T_1, T_2 \in \L(U,V)$, then for all $x \in V$ and by linearity of $S$, we have
        \begin{align*}
            [S(T_1 + T_2)](x) &= S((T_1 + T_2)(x)) \\
            &= S(T_1(x) + T_2(x)) \\
            &= S(T_1(x)) + S(T_2(x)) \\
            &= (S T_1)(x) + (S T_2)(x) \\
            &= [ST_1 + ST_2](x).
        \end{align*}
        Since it holds for all $x \in U$, then $S(T_1 + T_2) = ST_1 + ST_2$.\\
    \end{itemize}
\end{solution}

\begin{exercise}
    Show that every linear map from a one-dimensional vector space to itself is multiplicative by some scalar. More precisely, prove that if $\dim V = 1$ and $T \in \L(V)$, then there exists $\lambda \in \F$ such that $Tv = \lambda v$ for all $v \in V$. \\
\end{exercise}

\begin{solution}
    \\ Since $\dim V = 1$, then there is a $v_0 \in V$ such that $V = \text{span}(v_0)$. We have $Tv_0 \in \text{span}(v_0)$ so there is a $\lambda \in \F$ satisfying $Tv_0 = \lambda v_0$. Take $v \in V$, since $v \in \text{span}(v_0)$, then there is an $\alpha \in \F$ such that $v = \alpha v_0$. Thus:
    $$Tv = T\alpha v_0 = \alpha Tv_0 = \alpha \lambda v_0 = \lambda v.$$
\end{solution}

\begin{exercise}
    Give an example of a function $\varphi : \R^2 \to \R$ such that
    $$\varphi(av) = a\varphi(v)$$
    for all $a \in \R$ and all $v \in \R^2$ but $\varphi$ is not linear.\\
\end{exercise}

\begin{solution}
    \\ Consider the function $\varphi : \R^2 \to \R$ defined by $\varphi(x,y) = \sqrt[3]{(x + y)^3}$, then for all $x,y \in \R$ and $a \in \R$, we have
    \begin{align*}
        \varphi(ax, ay) &= \sqrt[3]{(ax + ay)^3} \\
        &= \sqrt[3]{a^3(x + y)^3} \\
        &= a\sqrt[3]{(x + y)^3} \\
        &= a \varphi(x,y).
    \end{align*}
    However, notice that $\varphi(1, 0) = \varphi(0,1) = 1$ but $\varphi(1, 1) = \sqrt[3]{2}$ so $\varphi(1, 1) \neq \varphi(1, 0) + \varphi(0,1)$ so $\varphi$ is not linear. \\
\end{solution}

\begin{exercise}
    Give an example of a function $\varphi : \C \to \C$ such that
    $$\varphi(w + z) = \varphi(w) + \varphi(z)$$
    for all $w,z \in \C$ but $\varphi$ is not linear. (Here, $\C$ is thought of as a complex vector space.)\\
\end{exercise}

\begin{solution}
    \\ Consider the function $\varphi : \C \to \C$ defined by $\varphi(z) = \Re(z)$, then for all $w,z \in \C$, we know that
    $$\Re(w + z) = \Re(w) + \Re(z).$$
    However, $\Re(i) = 0$ and $i \Re(1) = i$ so $\Re(i \cdot 1) \neq i\Re(1)$. Therefore, $\varphi$ is not linear. \\
\end{solution}

\begin{exercise}
    Prove or give a counterexample: If $q \in \P(\R)$ and $T: \P(\R) \to \P(\R)$ is defined by $Tp = q \circ p$, then $T$ is a linear map. \\
\end{exercise}

\begin{solution}
    \\ \td \\
\end{solution}

\begin{exercise}
    Suppose $V$ is a finite-dimensional vector space and $T \in \L(V)$. Prove that $T$ is a scalar multiple of the identity if and only if $ST = TS$ for all $S \in \L(V)$. \\
\end{exercise}

\begin{solution}
    \\ \td \\
\end{solution}

\begin{exercise}
    Suppose $U$ is a subspace of $V$ with $U \neq V$. Suppose $S \in \L(U,W)$ and $S \neq 0$ (which means that $Su \neq 0$ for some $u \in U$). Define $T : V \to W$ by
    $$Tv = \begin{cases} Sv & \text{if } v \in U, \\ 0 & \text{if } v \in V \text{ and } v \notin U. \end{cases}$$
    Prove that $T$ is not a linear map on $V$. \\
\end{exercise}

\begin{solution}
    \\ \td \\
\end{solution}

\begin{exercise}
    Suppose $V$ is finite-dimensional. Prove that every linear map on a subspace of $V$ can be extended to a linear map on $V$. In other words, show that if $U$ is a subspace of $V$ and $S \in \L(U,W)$, then there exists $T \in \L(V,W)$ such that $Tu = Su$ for all $u \in U$. \\
\end{exercise}

\begin{solution}
    \\ \td \\
\end{solution}

\begin{exercise}
    Suppose $V$ is finite-dimensional with $\dim V > 0$, and suppose $W$ is infinite-dimensional. Prove that $\L(V,W)$ is infinite-dimensional. \\
\end{exercise}

\begin{solution}
    \\ \td \\
\end{solution}

\begin{exercise}
    Suppose $v_1, ..., v_m$ is a linearly dependent list of vectors in $V$. Suppose also that $W \neq \{0\}$. Prove that there exist $w_1, ..., w_m \in W$ such that no $T \in \L(V,W)$ satisfies $Tv_k = w_k$ for each $k = 1, ..., m$. \\
\end{exercise}

\begin{solution}
    \\ \td \\
\end{solution}

\begin{exercise}
    Suppose $V$ is finite-dimensional with $\dim V > 1$. Prove that there exist $S,T \in \L(V)$ such that $ST \neq TS$. \\
\end{exercise}

\begin{solution}
    \\ \td \\
\end{solution}

\begin{exercise}
    Suppose $V$ is finite-dimensional. Show that the only two-sided ideals of $\L(V)$ are $\{0\}$ and $\L(V)$. \\
\end{exercise}

\begin{solution}
    \\ \td \\
\end{solution}
