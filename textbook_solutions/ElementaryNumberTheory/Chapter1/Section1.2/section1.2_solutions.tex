\section{The Binomial Theorem}

\begin{exercise}
    Prove that for $n \geq 1$:
    \begin{enumerate}
        \item $\displaystyle \binom{2n}{n} = \frac{1\cdot 3 \cdot 5 \cdot \cdots \cdot (2n-1)}{n!}2^n .$
        \item $\displaystyle \binom{4n}{2n} = \frac{1\cdot 3 \cdot 5 \cdot \cdots \cdot (4n-1)}{[1\cdot 3 \cdot 5 \cdot \cdots \cdot (2n-1)]^2}\binom{2n}{n}.$
    \end{enumerate}
\end{exercise}

\begin{solution}
    \begin{enumerate}
        \item Let's prove it by induction. When $n = 1$, we have 
        $$\binom{2n}{n} = 2$$
        and 
        $$\frac{1\cdot 3 \cdot 5 \cdot \cdots \cdot (2n-1)}{n!}2^n = 2$$
        and so it holds in that case. If we now suppose that it holds when $n = k$ for some integer $k \geq 1$, then it follows that
        $$\frac{(2k)!}{(k!)^2} = \frac{1\cdot 3 \cdot 5 \cdot \cdots \cdot (2k-1)}{k!}2^k.$$
        Multiplying both sides by $\frac{(2k+1)(2k+2)}{(k+1)^2}$ gives us
        \begin{align*}
            \binom{2(k+1)}{k+1} &= \frac{(2k+1)(2k+2)}{(k+1)^2} \cdot \frac{1\cdot 3 \cdot 5 \cdot \cdots \cdot (2k-1)}{k!}2^k \\
            &= 2\frac{(2k+1)}{k+1} \cdot \frac{1\cdot 3 \cdot 5 \cdot \cdots \cdot (2k-1)}{k!}2^k \\
            &= \frac{1\cdot 3 \cdot 5 \cdot \cdots \cdot (2k+1)}{(k+1)!}2^{k+1} 
        \end{align*}
        which shows that the equation also holds for $n = k+1$. Therefore, by induction, it holds for all integers $n \geq 1$.
        \item First, notice that by part (a), it suffices to prove that
        $$\binom{4n}{2n} = \frac{1\cdot 3 \cdot 5 \cdot \cdots \cdot (4n-1)}{n! \cdot 1\cdot 3 \cdot 5 \cdot \cdots \cdot (2n-1)}2^n $$
        holds for all $n \geq 1$. Let's prove it by induction on $n$. When $n = 1$, then both sides are equal to 6 and so the statement holds in that case. Suppose now that it holds for some integer $n = k \geq 1$, then 
        $$\frac{(4k)!}{(2k!)^2} = \frac{1\cdot 3 \cdot 5 \cdot \cdots \cdot (4k-1)}{k! \cdot 1\cdot 3 \cdot 5 \cdot \cdots \cdot (2k-1)}2^k.$$
        Multiplying both sides by $\frac{(4k+1)(4k+2)(4k+3)(4k+4)}{(2k+1)^2(2k+2)^2}$ gives us
        \begin{align*}
            \binom{4(k+1)}{2(k+1)} &= \frac{(4k+1)(4k+2)(4k+3)(4k+4)}{(2k+1)^2(2k+2)^2} \cdot \frac{1\cdot 3 \cdot 5 \cdot \cdots \cdot (4k-1)}{k! \cdot 1\cdot 3 \cdot 5 \cdot \cdots \cdot (2k-1)}2^k \\
            &= \frac{(4k+2)(4k+4)}{(2k+1)(2k+2)^2} \cdot \frac{1\cdot 3 \cdot 5 \cdot \cdots \cdot (4k+1)}{k! \cdot 1\cdot 3 \cdot 5 \cdot \cdots \cdot (2k+1)}2^k \\
            &= \frac{4k+4}{(2k+2)(2k+2)} \cdot \frac{1\cdot 3 \cdot 5 \cdot \cdots \cdot (4k+1)}{k! \cdot 1\cdot 3 \cdot 5 \cdot \cdots \cdot (2k+1)}2^{k+1} \\
            &= \frac{1}{k+1} \cdot \frac{1\cdot 3 \cdot 5 \cdot \cdots \cdot (4k+1)}{k! \cdot 1\cdot 3 \cdot 5 \cdot \cdots \cdot (2k+1)}2^{k+1} \\
            &= \frac{1\cdot 3 \cdot 5 \cdot \cdots \cdot (4k+1)}{(k+1)! \cdot 1\cdot 3 \cdot 5 \cdot \cdots \cdot (2k+1)}2^{k+1}
        \end{align*}
        which shows that the equation also holds for $n = k+1$. Therefore, by induction, it holds for all integers $n \geq 1$.
    \end{enumerate}
\end{solution}

\begin{exercise}
    If $2\leq k \leq n - 2$, show that
    $$\binom{n}{k} = \binom{n-2}{k-2} + 2\binom{n - 2}{k-1} + \binom{n-2}{k}, \qquad n \geq 4.$$
\end{exercise}

\begin{solution}
    \\ This simply follows from Pascal's Rule:
    \begin{align*}
        \binom{n-2}{k-2} + 2\binom{n - 2}{k-1} + \binom{n-2}{k} &= \left[\binom{n-2}{k-2} + \binom{n - 2}{k-1}\right] + \left[\binom{n - 2}{k-1} + \binom{n-2}{k}\right] \\
        &= \binom{n-1}{k-1} + \binom{n-1}{k} \\
        &= \binom{n}{k}.
    \end{align*}
\end{solution}

\begin{exercise}
    For $n \geq 1$, derive each of the identities below:
    \begin{enumerate}
        \item $\displaystyle \binom{n}{0} + \binom{n }{1} + \binom{n }{2} + \dots + \binom{n }{n } = 2^n$;
        [\textit{Hint:} Let $a = b = 1$ in the binomial theorem.]
        \item $\displaystyle \binom{n}{0} - \binom{n }{1} + \binom{n }{2} - \dots + (-1)^n\binom{n }{n } = 0$;
        \item $\displaystyle \binom{n}{1} + 2\binom{n }{2} + 3\binom{n }{3} + \dots + n\binom{n }{n } = n2^{n-1}$;
        [\textit{Hint:} After expanding $n(1 + b)^{n-1}$ by the binomial theorem, let $b = 1$: note also that
        $$n \binom{n-1}{k } = (k+1)\binom{n }{k+1}.]$$
        \item $\displaystyle \binom{n}{0} + 2\binom{n }{1} + 2^2\binom{n }{2} + \dots + 2^n\binom{n }{n } = 3^n$;
        \item $\displaystyle \binom{n}{0} + \binom{n }{2} + \binom{n }{4} + \binom{n }{6 } + \dots$
        
        $\displaystyle \qquad \quad \ \binom{n}{1} + \binom{n }{3} + \binom{n }{5} + \dots = 2^{n-1};$
        [\textit{Hint:} Use parts (a) and (b).]
        \item $\displaystyle \binom{n}{0} - \frac{1}{2}\binom{n }{1} + \frac{1}{3}\binom{n }{2} - \dots + \frac{(-1)^n}{n+1}\binom{n }{n } = \frac{1}{n+1}$;
        [\textit{Hint:} the left-hand side equals
        
        $ \displaystyle \frac{1}{n+1} \left[\binom{n+1}{1} - \binom{n+1}{2} + \binom{n+1}{3} - \dots + (-1)^n\binom{n+1}{n+1}\right]$.]
    \end{enumerate}
\end{exercise}

\begin{solution}
    \begin{enumerate}
        \item Taking $a = b = 1$ in the Binomial Theorem gives us
        $$2^n = (1 + 1)^n = \sum_{k=0}^{n}\binom{n}{k}a^{n-k}b^k = \binom{n}{0} + \binom{n }{1} + \binom{n }{2} + \dots + \binom{n }{n }.$$
        \item Taking $a = 1$ and $b = -1$ in the Binomial Theorem gives us
        $$0 = (1 - 1)^n = \sum_{k=0}^{n}\binom{n}{k}a^{n-k}b^k = \binom{n}{0} - \binom{n }{1} + \binom{n }{2} - \dots (-1)^n \binom{n }{n }.$$
        \item From the hint, it follows that
        $$ \binom{n}{1} + 2\binom{n }{2} + 3\binom{n }{3} + \dots + n\binom{n }{n } = n\binom{n-1}{0} + n\binom{n-1}{1} + \dots + n\binom{n-1}{n-1} = n2^{n-1}$$
        where the last equality follows from part (a).
        \item Taking $a = 1$ and $b = 2$ in the Binomial Theorem gives us
        $$3^n = (1 + 2)^n = \sum_{k=0}^{n}\binom{n}{k}a^{n-k}b^k = \binom{n}{0} + 2\binom{n }{1} + 2^2\binom{n }{2} + \dots 2^n \binom{n }{n }.$$
        \item From part (b), we have that 
        $$\binom{n }{0} + \binom{n }{2} + \binom{n }{4} + \dots = \binom{n }{1} + \binom{n }{3} + \binom{n }{5} + \dots$$
        Thus, using part (a), we get
        \begin{align*}
            \binom{n }{0} + \binom{n }{2} + \binom{n }{4} + \dots &= \binom{n }{1} + \binom{n }{3} + \binom{n }{5} + \dots \\
            &= \frac{1}{2}\left[\binom{n }{0} + \binom{n }{1} + \binom{n }{2} + \dots + \binom{n }{n }\right] \\
            &= 2^{n-1}
        \end{align*}
        \item Using the hint, we easily get 
        \begin{align*}
            & \binom{n}{0} - \frac{1}{2}\binom{n }{1} + \frac{1}{3}\binom{n }{2} - \dots + \frac{(-1)^n}{n+1}\binom{n }{n } \\
            =& \frac{1}{n+1} \left[\binom{n+1}{1} - \binom{n+1}{2} + \binom{n+1}{3} - \dots + (-1)^n\binom{n+1}{n+1}\right] \\
            =& \frac{1}{n+1} \left(1 - \left[\binom{n }{0} - \binom{n+1}{1} + \binom{n+1}{2} - \binom{n+1}{3} + \dots + (-1)^{n+1}\binom{n+1}{n+1}\right]\right) \\
            =& \frac{1}{n+1}(1 - 0) \\
            =& \frac{1}{n+1}
        \end{align*}
    \end{enumerate}
\end{solution}

\begin{exercise}
    Prove that for $n \geq 1$:
    \begin{enumerate}
        \item $\displaystyle \binom{n }{r } < \binom{n }{r+1}$ if and only if $0 \leq r < \displaystyle \frac{1}{2}(n-1)$.
        \item $\displaystyle \binom{n }{ r} > \binom{n }{r+1}$ if and only if $n-1 \geq r > \displaystyle \frac{1}{2}(n-1)$.
        \item $\displaystyle \binom{n }{r } = \binom{n }{r+1 }$ if and only if $n$ is an odd integer, and $\displaystyle r = \frac{1}{2}(n-1)$.
    \end{enumerate}
\end{exercise}

\begin{solution}
    \begin{enumerate}
        \item Let $0 \leq r \leq n-1$ be an integer, then
        \begin{align*}
            \binom{n }{r } < \binom{n }{r+1} &\iff \frac{n!}{(n-r)! r!} < \frac{n!}{(n-r - 1)! (r+1)!} \\
            &\iff (n-r - 1)! (r+1)! < (n-r)! r! \\
            &\iff r+1 < n-r \\
            &\iff r < \frac{1}{2}(n-1).
        \end{align*}
        \item Let $0 \leq r \leq n-1$ be an integer, then
        \begin{align*}
            \binom{n }{r } > \binom{n }{r+1} &\iff \frac{n!}{(n-r)! r!} > \frac{n!}{(n-r - 1)! (r+1)!} \\
            &\iff (n-r - 1)! (r+1)! > (n-r)! r! \\
            &\iff r+1 > n-r \\
            &\iff r > \frac{1}{2}(n-1).
        \end{align*}
        \item Let $0 \leq r \leq n-1$ be an integer, then
        \begin{align*}
            \binom{n }{r } = \binom{n }{r+1} &\iff \frac{n!}{(n-r)! r!} = \frac{n!}{(n-r - 1)! (r+1)!} \\
            &\iff (n-r - 1)! (r+1)! = (n-r)! r! \\
            &\iff r+1 = n-r \\
            &\iff r = \frac{1}{2}(n-1) \\
            &\iff n = 2r+1.
        \end{align*}
    \end{enumerate}
\end{solution}

\begin{exercise}
    For $n \geq 1$, show that the expressions $\displaystyle \frac{(2n)!}{n!(n+1)!}$ and $\displaystyle \frac{(3n)!}{6^n n!}$ are both integers. \\
\end{exercise}

\begin{solution}
    \\ For the first expression, it suffices to notice that
    \begin{align*}
        \binom{2n }{n } - \binom{2n }{n+1} &= \frac{(2n)!}{n!n!} - \frac{(2n)!}{(n-1)! (n+1)!} \\
        &= \frac{(2n)!(n+1) - (2n)!n}{n!(n+1)!} \\
        &= \frac{(2n)!}{n!(n+1)!}.
    \end{align*}
    Since the binomial coefficients are integers, then it follows that the expression $\frac{(2n)!}{n!(n+1)!}$ is also an integer. For the second expression, let's prove it by induction. When $n = 1$, we have
    $$\frac{(3n)!}{6^n n!} = \frac{3!}{6 \cdot 1} = 1$$
    which proves that it holds for $n = 1$. Suppose now that the expression is an integer for some $n = k \geq 1$, then 
    \begin{align*}
        \frac{(3(k+1))!}{6^{k+1}(k+1)!} &= \frac{(3k+1)(3k+2)(3k+3)}{6(k+1)} \cdot \frac{(3k)!}{6^k k!} \\
        &= \frac{(3k+1)(3k+2)}{2} \cdot \frac{(3k)!}{6^k k!} 
    \end{align*}
    where $\frac{(3k)!}{6^k k!} $ is an integer by the inductive hypothesis. Moreover, notice that $3k+1$ and $3k+2$ are two consecutive numbers and so one of them must be divisible by two. Thus, $\frac{(3k+1)(3k+2)}{2}$ is also an integer. Therefore, the case $n = k+1$ also holds since $\frac{(3(k+1))!}{6^{k+1}(k+1)!}$ can be written as the product of two integers. \\
\end{solution}

\begin{exercise}
    \begin{enumerate}
        \item For $n \geq 2$, prove that
        $$\binom{2 }{2 } + \binom{3}{2} + \binom{4}{2} + \dots + \binom{n }{2} = \binom{n+1}{3}.$$
        [\textit{Hint} Use induction and Pascal's rule.]
        \item From part (a) and the fact that $\displaystyle \binom{ m }{2} + \binom{m+1 }{2} = m^2$ for $m \geq 2$, deduce the formula
        $$1^2 + 2^2 + 3^2 + \dots + n^2 = \frac{n(n+1)(2n+1)}{6}.$$
    \end{enumerate}
\end{exercise}

\begin{solution}
    \begin{enumerate}
        \item Let's prove it by induction on $n$. When $n =2$, we have
        $$\binom{2}{2} + \dots + \binom{n }{2} = \binom{2}{2} = 1 = \binom{3 }{3} = \binom{n + 1}{3}$$
        and so the proposition holds in that case. Suppose now that the proposition holds for $n = k \geq 2$, then 
        $$\binom{2 }{2 } + \binom{3}{2} + \binom{4}{2} + \dots + \binom{k }{2} = \binom{k+1}{3}.$$
        Adding $\displaystyle \binom{k+1}{2}$ on both sides gives
        $$\binom{2 }{2 } + \binom{3}{2} + \binom{4}{2} + \dots + \binom{k +1 }{2} = \binom{k+1 }{2} + \binom{k+1}{3} = \binom{k+2}{3}$$
        and so the proposition holds for $n = k+1$. Therefore, by induction, it holds for all $n \geq 2$.
        \item Using the fact that $\displaystyle \binom{ m }{2} + \binom{m+1 }{2} = m^2$, we can write
        \begin{align*}
            1^2 + 2^2 + 3^2 + \dots + n^2 &= 1 + \left[\binom{2}{2} + \binom{3}{2}\right] + \left[\binom{3}{2} + \binom{4}{2}\right] + \dots + \left[\binom{n }{2} + \binom{n+1}{2}\right] \\
            &= 2\left[\binom{2}{2} + \binom{3}{2} + \binom{4}{2} + \dots + \binom{n }{2}\right] + \binom{n+1}{2} \\
            &= 2\binom{n+1}{3} + \binom{n+1}{2} \\
            &= \frac{2(n+1)n(n-1)}{6} + \frac{(n+1)n}{2} \\
            &= \frac{2(n+1)n(n-1) + 3(n+1)n}{6} \\
            &= \frac{n(n+1)(2n+1)}{6}.
        \end{align*}
        which proves the desired formula.
    \end{enumerate}
\end{solution}

\begin{exercise}
    For $n \geq 1$, verify that 
    $$1^2 + 3^2 + 5^2 + \dots + (2n-1)^2 = \binom{2n+1}{3}.$$
\end{exercise}

\begin{solution}
    \\ Let's prove it by induction on $n$. When $n = 1$, we have 
    $$1^2 + \dots + (2n-1)^2 = 1 = \binom{3}{3} = \binom{2n+1}{3}.$$
    Thus, the proposition holds for $n = 1$. Suppose now that it holds for $n = k \geq 1$, then 
    \begin{align*}
        1^2 + 3^2 + 5^2 + \dots + (2k+1)^2 &= (1^2 + 3^2 + 5^2 + \dots + (2k-1)^2) + (2k+1)^2 \\
        &= \binom{2k+1}{3} + (2k+1)^2 \\
        &= \frac{(2k+1)(2k)(2k-1)}{6} + (2k+1)(2k+1) \\
        &= \frac{(2k+1)[2k(2k-1) + 6(2k+1)]}{6} \\
        &= \frac{(2k+1)(4k^2 + 10k + 6)}{6} \\
        &= \frac{(2k+3)(2k+2)(2k+1)}{6} \\
        &= \binom{2(k+1) + 1}{3}
    \end{align*}
    which shows that it holds for $n = k+1$. Therefore, by induction, the proposition holds for all $n \geq 1$.\\
\end{solution}

\begin{exercise}
    Establish the inequality $2^n < \displaystyle \binom{2n}{n } < 2^{2n}$ for $n > 1$. 
\end{exercise}

\begin{solution}
    \\ Let's prove it by induction on $n$. When $n = 2$, we have 
    $$2^n = 4 < 6 = \binom{2n}{2} < 16 = 2^{2n}$$
    and so it holds for this case. Suppose now that holds for an integer $n = k \geq 2$, then
    $$2^k < \binom{2k}{k } < 2^{2k}.$$
    Multiplying both sides by $\frac{(2k+2)(2k+1)}{(n+1)^2} = 2\frac{2k+1}{k+1}$ gives us
    $$2^{k+1} \leq 2^k \cdot 2\frac{2k+1}{k+1} < \binom{2(k+1)}{k+1} < 2^{2k} \cdot 2\frac{2k+1}{k+1} \leq 2^{2(k+1)} $$
    which shows that it holds for $n = k+1$. Therefore, by induction, the proposition holds for all integers $n > 1$.
\end{solution}