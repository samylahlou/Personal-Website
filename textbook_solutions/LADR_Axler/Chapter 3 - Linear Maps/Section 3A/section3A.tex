\chapter{Linear Maps}

\section{Vector Space of Linear Maps}

\begin{exercise}
    Suppose $b,c \in \R$. Define $T: \R^3 \to \R^2$ by
    $$T(x,y,z) = (2x-4y + 3z + b, 6x + cxyz).$$
    Show that $T$ is linear if and only if $b = c = 0$. \\
\end{exercise}

\begin{solution}
    \\ ($\implies$) Suppose that $T$ is linear, then we know from Proposition 3.10 that $T0 = 0$. Thus, it follows that
    $$T(0,0,0) = (b, 0) = (0,0)$$
    which implies that $b = 0$. To prove that $c = 0$, notice that by linearity of $T$, we have
    $$T(2,2,2) = 2T(1,1,1).$$
    If we plug-in the values into the definition of $T$, we get
    $$(4 - 8 + 6 + 0, 12 + 8c) = 2(2 - 4 + 3 + 0, 6 + c)$$
    which is equivalent to
    $$(2, 12 + 8c) = (2, 12 + 2c).$$
    It follows that $12 + 8c = 12 + 2c$ which can only be true when $c = 0$. Thus, $b = c = 0$. \\
    $( \ \Longleftarrow \ )$ Suppose now that $b = c = 0$, then $T(x,y,z)$ becomes
    $$T(x,y,z) = (2x-4y + 3z, 6x)$$
    for all $x,y,z \in \R$. Let's show that $T$ is linear. First, take $(x,y,z), (x',y',z') \in \R^3$ and notice that
    \begin{align*}
        T((x,y,z) + (x',y',z')) &= T(x+x', y+y', z+z') \\
        &= (2(x + x')-4(y+y') + 3(z+z'), 6(x+x')) \\
        &= (2x - 4y + 3z + 2x' - 4y' + 3z', 6x + 6x') \\
        &= (2x-4y + 3z, 6x) + (2x'-4y' + 3z', 6x') \\
        &= T(x,y,z) + T(x',y',z')
    \end{align*}
    Moreover, given any $\lambda \in \R$ and $(x,y,z) \in \R^3$, we have
    \begin{align*}
        T(\lambda(x,y,z)) &= T(\lambda x, \lambda y, \lambda z) \\
        &= (2(\lambda x)-4(\lambda y) + 3(\lambda z), 6(\lambda x)) \\
        &= (\lambda(2x - 4y + 3z), \lambda(6x)) \\
        &= \lambda(2x - 4y + 3z, 6x) \\
        &= \lambda T(x,y,z)
    \end{align*}
    Therefore, $T$ is linear.\\
\end{solution}

\begin{exercise}
    Suppose $b,c \in \R$. Define $T : \mathcal{P}(\R) \to \R^2$ by
    $$Tp = \left(3p(4) + 5p'(6) + bp(1)p(2), \int_{-1}^{2}x^3p(x)dx + c \sin p(0) \right).$$
    Show that $T$ is linear if and only if $b = c = 0$. \\
\end{exercise}

\begin{solution}
    \\ ($\implies$) Suppose that $T$ is linear, then if we let $p$ be the constant polynomial equal to $\pi/2$, we get that $T$ must satisfy
    $$T(2p) = 2Tp.$$
    If we rewrite this using the definition of $T$ and $p$, we obtain
    $$\left(3\pi + b\pi^2, \pi\int_{-1}^{2}x^3dx + c\sin(\pi)\right) = 2\left(3\frac{\pi}{2} + b\frac{\pi^2}{4}, \frac{\pi}{2}\int_{-1}^{2}x^3dx + c\sin\left(\frac{\pi}{2}\right)\right)$$
    which can be simplified to
    $$\left(3\pi + b\pi^2, \pi\int_{-1}^{2}x^3dx\right) = \left(3\pi + b\frac{\pi^2}{2}, \pi\int_{-1}^{2}x^3dx + c\right).$$
    This gives us the following system of equations:
    $$\begin{cases}
        3\pi + b\pi^2 = 3\pi + b\frac{\pi^2}{2} \\
        \pi\int_{-1}^{2}x^3dx = \pi\int_{-1}^{2}x^3dx + c
    \end{cases} 
    \implies
    \begin{cases}
        b = \frac{1}{2}b \\
        c = 0
    \end{cases}
    \implies b = c = 0.$$
    $( \ \Longleftarrow \ )$ Suppose that $b = c = 0$, then for all $p \in \mathcal{P}(\R)$, we have
    $$Tp = \left(3p(4) + 5p'(6), \int_{-1}^{2}x^3p(x)dx \right).$$
    Thus, for any $p_1, p_2 \in \mathcal{P}(\R)$, we get 
    \begin{align*}
        T(p_1 + p_2) &= \left(3(p_1 + p_2)(4) + 5(p_1 + p_2)'(6), \int_{-1}^{2}x^3(p_1 + p_2)(x)dx \right) \\
        &= \left(3(p_1(4) + p_2(4)) + 5(p_1'(6) + p_2'(6)), \int_{-1}^{2}x^3(p_1(x) + p_2(x))dx \right) \\
        &= \left(3p_1(4) + 3p_2(4) + 5p_1'(6) + 5p_2'(6), \int_{-1}^{2}x^3p_1(x)dx + \int_{-1}^{2}x^3p_2(x)dx \right) \\
        &= \left(3p_1(4) + 5p_1'(6), \int_{-1}^{2}x^3p_1(x)dx \right) + \left(3p_2(4) + 5p_2'(6), \int_{-1}^{2}x^3p_2(x)dx \right) \\
        &= Tp_1 + Tp_2.
    \end{align*}
    Similarly, for all $\lambda \in \R$ and $p \in \mathcal{P}(\R)$, we have
    \begin{align*}
        T(\lambda p) &= \left(3(\lambda p)(4) + 5(\lambda p)'(6), \int_{-1}^{2}x^3(\lambda p)(x)dx \right) \\
        &= \left(3\lambda p(4) + 5\lambda p'(6), \int_{-1}^{2}x^3 \lambda p(x)dx \right) \\
        &= \left(\lambda (3p(4) + 5p'(6)), \lambda\int_{-1}^{2}x^3 p(x)dx \right) \\
        &= \lambda \left(3p(4) + 5p'(6), \int_{-1}^{2}x^3 p(x)dx \right) \\
        &= \lambda Tp.
    \end{align*}
    Therefore, $T$ is linear. \\
\end{solution}

\begin{exercise}
    Suppose that $T \in \L(\F^n, \F^m)$. Show that there exist scalars $A_{j,k} \in \F$ for $j = 1, ..., m$ and $k= 1, ..., n$ such that 
    $$T(x_1, ..., x_n) = (A_{1,1}x_1 + \dots + A_{1,n}x_n, ..., A_{m,1}x_1 + \dots + A_{m,n}x_n)$$
    for every $(x_1, ..., x_n) \in \F^n$. \\
\end{exercise}

\begin{solution}
    \\ Denote by $e_1, ..., e_n$ the standard basis of $\F^n$ and by $f_1, ..., f_m$ the standard basis for $\F^m$, then for all $k \in \{1, ..., n\}$, there exist scalars $A_{1,k}, ..., A_{m,k} \in \F$ such that
    $$Te_k = A_{1,k}f_1 + \dots + A_{m,k}f_m.$$
    Therefore, by linearity, for all $(x_1, ..., x_n) \in \F^n$:
    \begin{align*}
        T(x_1, ..., x_n) &= x_1Te_1 + \dots + x_n Te_n \\
        &= x_1(A_{1,1}f_1 + \dots + A_{m,1}f_m) + \dots + x_n(A_{1,n}f_1 + \dots + A_{m,n}f_m) \\
        &= (A_{1,1}x_1 + \dots + A_{1,n}x_n)f_1 + \dots + (A_{m,1}x_1 + \dots + A_{m,n}x_n)f_m \\
        &= (A_{1,1}x_1 + \dots + A_{1,n}x_n, ..., A_{m,1}x_1 + \dots + A_{m,n}x_n).
    \end{align*}
    Therefore, any linear transformation has this form. \\
\end{solution}

\begin{exercise}
    Suppose $T \in \L(V, W)$ and $v_1, ..., v_m$ is a list of vectors in $V$ such that $Tv_1, ..., Tv_m$ is a linearly independent list in $W$. Prove that $v_1, ..., v_m$ is linearly independent. \\
\end{exercise}

\begin{solution}
    \\ To prove that $v_1, ..., v_m$ is linearly independent, take arbitrary scalars $\alpha_1, ..., \alpha_m \in \F$ such that
    $$\alpha_1 v_1 + ... + \alpha_m v_m = 0.$$
    By evaluating on both sides by $T$, we get by linearity of $T$ the following equation:
    $$\alpha_1 Tv_1 + ... + \alpha_m Tv_m = 0.$$
    But since the list $Tv_1, ..., Tv_m$ is a linearly independent in $W$, then 
    $$\alpha_1 = ... = \alpha_m = 0$$
    which proves that $v_1, ..., v_m$ is linearly independent. \\
\end{solution}

\begin{exercise}
    Prove that $\L(V, W)$ is a vector space, as was asserted in 3.6. \\
\end{exercise}

\begin{solution}
    \\ \td \\
\end{solution}

\begin{exercise}
    Prove that the multiplication of linear maps has the associative, identity and distributive properties asserted in 3.8. \\
\end{exercise}

\begin{solution}
    \\ \td \\
\end{solution}

\begin{exercise}
    Show that every linear map from a one-dimensional vector space to itself is multiplicative by some scalar. More precisely, prove that if $\dim V = 1$ and $T \in \L(V)$, then there exists $\lambda \in \F$ such that $Tv = \lambda v$ for all $v \in V$. \\
\end{exercise}

\begin{solution}
    \\ \td \\
\end{solution}